\documentclass[a4paper]{article}

\def\ntitle{Dynamical Systems}
\def\ndate{Oct 19, 2017}

\ifx \nauthor\undefined
  \def\nauthor{Qiangru Kuang}
\else
\fi

\ifx \ntitle\undefined
  \def\ntitle{Template}
\else
\fi

\ifx \nauthoremail\undefined
  \def\nauthoremail{qk206@cam.ac.uk}
\else
\fi

\ifx \ndate\undefined
  \def\ndate{\today}
\else
\fi

\title{\ntitle}
\author{\nauthor}
\date{\ndate}

%\usepackage{microtype}
\usepackage{mathtools}
\usepackage{amsthm}
\usepackage{stmaryrd}%symbols used so far: \mapsfrom
\usepackage{empheq}
\usepackage{amssymb}
\let\mathbbalt\mathbb
\let\pitchforkold\pitchfork
\usepackage{unicode-math}
\let\mathbb\mathbbalt%reset to original \mathbb
\let\pitchfork\pitchforkold

\usepackage{imakeidx}
\makeindex[intoc]

%to address the problem that Latin modern doesn't have unicode support for setminus
%https://tex.stackexchange.com/a/55205/26707
\AtBeginDocument{\renewcommand*{\setminus}{\mathbin{\backslash}}}
\AtBeginDocument{\renewcommand*{\models}{\vDash}}%for \vDash is same size as \vdash but orginal \models is larger
\AtBeginDocument{\let\Re\relax}
\AtBeginDocument{\let\Im\relax}
\AtBeginDocument{\DeclareMathOperator{\Re}{Re}}
\AtBeginDocument{\DeclareMathOperator{\Im}{Im}}
\AtBeginDocument{\let\div\relax}
\AtBeginDocument{\DeclareMathOperator{\div}{div}}

\usepackage{tikz}
\usetikzlibrary{automata,positioning}
\usepackage{pgfplots}
%some preset styles
\pgfplotsset{compat=1.15}
\pgfplotsset{centre/.append style={axis x line=middle, axis y line=middle, xlabel={$x$}, ylabel={$y$}, axis equal}}
\usepackage{tikz-cd}
\usepackage{graphicx}
\usepackage{newunicodechar}

\usepackage{fancyhdr}

\fancypagestyle{mypagestyle}{
    \fancyhf{}
    \lhead{\emph{\nouppercase{\leftmark}}}
    \rhead{}
    \cfoot{\thepage}
}
\pagestyle{mypagestyle}

\usepackage{titlesec}
\newcommand{\sectionbreak}{\clearpage} % clear page after each section
\usepackage[perpage]{footmisc}
\usepackage{blindtext}

%\reallywidehat
%https://tex.stackexchange.com/a/101136/26707
\usepackage{scalerel,stackengine}
\stackMath
\newcommand\reallywidehat[1]{%
\savestack{\tmpbox}{\stretchto{%
  \scaleto{%
    \scalerel*[\widthof{\ensuremath{#1}}]{\kern-.6pt\bigwedge\kern-.6pt}%
    {\rule[-\textheight/2]{1ex}{\textheight}}%WIDTH-LIMITED BIG WEDGE
  }{\textheight}% 
}{0.5ex}}%
\stackon[1pt]{#1}{\tmpbox}%
}

%\usepackage{braket}
\usepackage{thmtools}%restate theorem
\usepackage{hyperref}

% https://en.wikibooks.org/wiki/LaTeX/Hyperlinks
\hypersetup{
    %bookmarks=true,
    unicode=true,
    pdftitle={\ntitle},
    pdfauthor={\nauthor},
    pdfsubject={Mathematics},
    pdfcreator={\nauthor},
    pdfproducer={\nauthor},
    pdfkeywords={math maths \ntitle},
    colorlinks=true,
    linkcolor={red!50!black},
    citecolor={blue!50!black},
    urlcolor={blue!80!black}
}

\usepackage{cleveref}



% TODO: mdframed often gives bad breaks that cause empty lines. Would like to switch to tcolorbox.
% The current workaround is to set innerbottommargin=0pt.

%\usepackage[theorems]{tcolorbox}





\usepackage[framemethod=tikz]{mdframed}
\mdfdefinestyle{leftbar}{
  %nobreak=true, %dirty hack
  linewidth=1.5pt,
  linecolor=gray,
  hidealllines=true,
  leftline=true,
  leftmargin=0pt,
  innerleftmargin=5pt,
  innerrightmargin=10pt,
  innertopmargin=-5pt,
  % innerbottommargin=5pt, % original
  innerbottommargin=0pt, % temporary hack 
}
%\newmdtheoremenv[style=leftbar]{theorem}{Theorem}[section]
%\newmdtheoremenv[style=leftbar]{proposition}[theorem]{proposition}
%\newmdtheoremenv[style=leftbar]{lemma}[theorem]{Lemma}
%\newmdtheoremenv[style=leftbar]{corollary}[theorem]{corollary}

\newtheorem{theorem}{Theorem}[section]
\newtheorem{proposition}[theorem]{Proposition}
\newtheorem{lemma}[theorem]{Lemma}
\newtheorem{corollary}[theorem]{Corollary}
\newtheorem{axiom}[theorem]{Axiom}
\newtheorem*{axiom*}{Axiom}

\surroundwithmdframed[style=leftbar]{theorem}
\surroundwithmdframed[style=leftbar]{proposition}
\surroundwithmdframed[style=leftbar]{lemma}
\surroundwithmdframed[style=leftbar]{corollary}
\surroundwithmdframed[style=leftbar]{axiom}
\surroundwithmdframed[style=leftbar]{axiom*}

\theoremstyle{definition}

\newtheorem*{definition}{Definition}
\surroundwithmdframed[style=leftbar]{definition}

\newtheorem*{slogan}{Slogan}
\newtheorem*{eg}{Example}
\newtheorem*{ex}{Exercise}
\newtheorem*{remark}{Remark}
\newtheorem*{notation}{Notation}
\newtheorem*{convention}{Convention}
\newtheorem*{assumption}{Assumption}
\newtheorem*{question}{Question}
\newtheorem*{answer}{Answer}
\newtheorem*{note}{Note}
\newtheorem*{application}{Application}

%operator macros

%basic
\DeclareMathOperator{\lcm}{lcm}

%matrix
\DeclareMathOperator{\tr}{tr}
\DeclareMathOperator{\Tr}{Tr}
\DeclareMathOperator{\adj}{adj}

%algebra
\DeclareMathOperator{\Hom}{Hom}
\DeclareMathOperator{\End}{End}
\DeclareMathOperator{\id}{id}
\DeclareMathOperator{\im}{im}
\DeclareMathOperator{\coker}{coker}
\DeclarePairedDelimiter{\generation}{\langle}{\rangle}

%groups
\DeclareMathOperator{\sym}{Sym}
\DeclareMathOperator{\sgn}{sgn}
\DeclareMathOperator{\inn}{Inn}
\DeclareMathOperator{\aut}{Aut}
\DeclareMathOperator{\GL}{GL}
\DeclareMathOperator{\SL}{SL}
\DeclareMathOperator{\PGL}{PGL}
\DeclareMathOperator{\PSL}{PSL}
\DeclareMathOperator{\SU}{SU}
\DeclareMathOperator{\UU}{U}
\DeclareMathOperator{\SO}{SO}
\DeclareMathOperator{\OO}{O}
\DeclareMathOperator{\PSU}{PSU}
\DeclareMathOperator{\Sp}{Sp}


%hyperbolic
\DeclareMathOperator{\sech}{sech}

%field, galois heory
\DeclareMathOperator{\ch}{ch}
\DeclareMathOperator{\gal}{Gal}
\DeclareMathOperator{\emb}{Emb}



%ceiling and floor
%https://tex.stackexchange.com/a/118217/26707
\DeclarePairedDelimiter\ceil{\lceil}{\rceil}
\DeclarePairedDelimiter\floor{\lfloor}{\rfloor}


\DeclarePairedDelimiter{\innerproduct}{\langle}{\rangle}

%\DeclarePairedDelimiterX{\norm}[1]{\lVert}{\rVert}{#1}
\DeclarePairedDelimiter{\norm}{\lVert}{\rVert}



%Dirac notation
%TODO: rewrite for variable number of arguments
\DeclarePairedDelimiterX{\braket}[2]{\langle}{\rangle}{#1 \delimsize\vert #2}
\DeclarePairedDelimiterX{\braketthree}[3]{\langle}{\rangle}{#1 \delimsize\vert #2 \delimsize\vert #3}

\DeclarePairedDelimiter{\bra}{\langle}{\rvert}
\DeclarePairedDelimiter{\ket}{\lvert}{\rangle}




%macros

%general

%divide, not divide
\newcommand*{\divides}{\mid}
\newcommand*{\ndivides}{\nmid}
%vector, i.e. mathbf
%https://tex.stackexchange.com/a/45746/26707
\newcommand*{\V}[1]{{\ensuremath{\symbf{#1}}}}
%closure
\newcommand*{\cl}[1]{\overline{#1}}
%conjugate
\newcommand*{\conj}[1]{\overline{#1}}
%set complement
\newcommand*{\stcomp}[1]{\overline{#1}}
\newcommand*{\compose}{\circ}
\newcommand*{\nto}{\nrightarrow}
\newcommand*{\p}{\partial}
%embed
\newcommand*{\embed}{\hookrightarrow}
%surjection
\newcommand*{\surj}{\twoheadrightarrow}
%power set
\newcommand*{\powerset}{\mathcal{P}}

%matrix
\newcommand*{\matrixring}{\mathcal{M}}

%groups
\newcommand*{\normal}{\trianglelefteq}
%rings
\newcommand*{\ideal}{\trianglelefteq}

%fields
\renewcommand*{\C}{{\mathbb{C}}}
\newcommand*{\R}{{\mathbb{R}}}
\newcommand*{\Q}{{\mathbb{Q}}}
\newcommand*{\Z}{{\mathbb{Z}}}
\newcommand*{\N}{{\mathbb{N}}}
\newcommand*{\F}{{\mathbb{F}}}
%not really but I think this belongs here
\newcommand*{\A}{{\mathbb{A}}}

%asymptotic
\newcommand*{\bigO}{O}
\newcommand*{\smallo}{o}

%probability
\newcommand*{\prob}{\mathbb{P}}
\newcommand*{\E}{\mathbb{E}}

%vector calculus
\newcommand*{\gradient}{\V \nabla}
\newcommand*{\divergence}{\gradient \cdot}
\newcommand*{\curl}{\gradient \cdot}

%logic
\newcommand*{\yields}{\vdash}
\newcommand*{\nyields}{\nvdash}

%differential geometry
\renewcommand*{\H}{\mathbb{H}}
\newcommand*{\transversal}{\pitchfork}
\renewcommand{\d}{\mathrm{d}} % exterior derivative

%number theory
\newcommand*{\legendre}[2]{\genfrac{(}{)}{}{}{#1}{#2}}%Legendre symbol

%algebraic geometry
\DeclareMathOperator{\Spec}{Spec}
\DeclareMathOperator{\Proj}{Proj}

\begin{document}
\maketitle

\section{Stability}

It is clear what we mean by \emph{hyperbolic nodel focii} begin stable/unstable. We need to be more careful with other kinds of fixed points or other invariant sets because there are at least two distinct types of stabilities, i.e. saddle and centre.

\subsection{Definitions}

Consider a flow \(\phi_t(x)\),
\begin{defi}[Lyapunov Stability]
 A fixed point \(x_0\) is \emph{Lyapunov stable} if \(\forall\varepsilon>0, \exists\delta>0\) such that
\[
  |\phi_t(x)-x_0| < \varepsilon \, \forall t>0.
\]
\end{defi}

\begin{slogan}
  If it starts near, it stays near.
\end{slogan}

\begin{defi}[Quasi-asymptotic stability]
  A fixed point \(x_0\) is \emph{quasi-symptotically stable} if for all \(\varepsilon>0\) such that \(|x-x_0|<\delta\) imples that \(\phi(x)\to x_0\) as \(t\to \infty\).
\end{defi}

\begin{slogan}
  It tends to the fixed point eventually.
\end{slogan}

\begin{eg}\leavevmode
  \begin{itemize}
  \item \(\dot r = 0, dot \theta = 1\): \(\V 0\) is Lyapunov stable (take \(\varepsilon = \delta\)) but not quasi-asymptotic stable.
  \item \(\dot r = r(a-r^2), \dot \theta = \sin^2(\theta/2)\): \(\theta=0,r=1\) is quasi-aymptotically stable but not Lyapunov stable.
  \end{itemize}
\end{eg}
    
\begin{defi}[Asymptotic stability]
  A fixed piont \(x_0\) is \emph{asymptotically stable} if it is both Lyapunov stable and quasi-asymptotically stable.
\end{defi}

\begin{eg}
  A sink (all \(\lambda_i\) have \(\Re \lambda_i<0\)) is asymptotically stable (just take \(\delta\) small that the linear terms dominate), and clearly sources/saddles are not Lyapunov stable.
\end{eg}

To describe the stability of other invariant sets \(\Lambda\) we define
\[
  N_\delta(\Lambda) := \{x: \exists y\in\Lambda, |x-y|<\delta \}
\]
and say \(\phi_t(x)\to \Lambda\) if
\[
  \inf_{y\in \Lambda}\{|\phi_t(x)-y|\}\to 0 \text{ as } t\to \infty.
\]

\begin{defi}[Stability for an invariant set]
  \(\Lambda\) is Lyapunov stable if for all \(\varepsilon>0\) exists \(\delta>0\) such that \(x\in N_\varepsilon(\Lambda)\), \(\phi_t(x) \in N_\varepsilon(\Lambda)\) for all \(t>0\).

  \(\Lambda\) is quasi-asymptotically stable if exists \(\delta>0\) such that \(x\in N_\delta(\Lambda)\) such that \(\phi_t(x)\to \Lambda\) as \(t\to \infty\).

  \(\Lambda\) is aymptotically stable if it is both Lyapunov and quasi-asymptotically stable.
\end{defi}

\subsection{Lyapunov Functions}

Lyapounov functions allow us to say more about the stability of a fixed point which, without loss of generality, we may take to be at \(x=0\).

\begin{defi}[Lyapunov function]
  A continuously differentiable function \(V:\R^n\to \R\) is a \emph{Lyapunov function} for \(\dot{x} = f(x)\) on a domain \(D\) containing \(\V 0\) if it is
  \begin{itemize}
  \item positive-definite: \(V(\V 0) = 0\) and \(V(x) >0\) for all \(x\neq 0\) in \(D\). Informally, \(\V 0\) is the lowest point.
  \item non-increasing: \(\dot{V} = f\cdot\nabla V \leq 0\) for all \(x\in D\). Informally, it means all trajectories in \(D\) head ``downhill'' or level, but never ``uphill''.
  \end{itemize}
\end{defi}

These properties allow us to prove

\begin{thm}[Lyapunov's First Theorem]
  If a Lyapunov function exists then \(x=0\) is Lyaponov stable.
\end{thm}

\begin{proof}
  Wlog assume \(\varepsilon\) is sufficiently small that \(\{|x|\leq\varepsilon\} \subset D\). Let
  \[
    m = \inf \{V(x): |x| = \varepsilon\}.
  \]
  Since \(|x|=\varepsilon\) is compact, the infimum is attained and by the first property above, \(m>0\).

  Let \(C_{m,\varepsilon} = \{x: V(x)<m, |x| < \varepsilon \}\). Then for all \(x\in C_{m,\varepsilon}\), \(\phi_t(x) \in C_{m,\varepsilon}\) for all \(t>0\) as \(V\geq m\) on the boundary.

  Choose \(\delta\) such that \(\{|x|< \delta\} \subseteq C_{m,\varepsilon}\).
\end{proof}

Note that a trajectory can head ``downhill'' and not end up at \(x=0\). But there is a very important constraint:

\begin{thm}[La Salle's Invariance Principle]
  If \(V\) is a Lyapunov function on a bounded domain \(D\) and \(\mathcal O^+(x) \subseteq D\), then \(\phi_t(x)\) tends to an \emph{invariant} subset of \(\{x: \dot{V} = 0\} \cap D\).
\end{thm}

\begin{proof}
  \(V(\phi_t(x))\) is monotonically decreasing and bounded below by \(0\). Therefore \(V\to \alpha\) for some \(\alpha\geq0\). \(D\) is compact so the limit set \(\omega(x)\) is non-empty. \(\forall y\in \omega(x)\), \(\exists\{t_n\}\) such that \(\phi_{t_n}(x)\to y\) so \(V(y) = \alpha\) by continuity of \(V\). So \(V(\phi_t(y))=\alpha\) as \(\phi_t(y)\in \omega(x)\). Thus \(\dot{V}(y) = 0\) for all \(y\in\omega(x)\).

  Hence \(\omega(x) \subseteq \{ \dot{V} = 0\}\) and \(\omega(x)\) is invariant (c.f. section 1.4).
\end{proof}

\begin{cor}
  If \(V\) is a Lyapunov function on a domain \(D\) and the only invariant subset of \(\{\dot{V}=0\}\) is \(\{\V 0\}\) then \(x=0\) is aymptotically stable.
\end{cor}

\end{document}