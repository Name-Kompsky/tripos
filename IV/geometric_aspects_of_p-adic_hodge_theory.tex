\documentclass[a4paper]{article}

\def\npart{IV}

\def\ntitle{Geometric Aspects of p-adic Hodge Theory}
\def\nlecturer{T.\ Csige}

\def\nterm{Michaelmas}
\def\nyear{2020}

\ifx \nauthor\undefined
  \def\nauthor{Qiangru Kuang}
\else
\fi

\ifx \ntitle\undefined
  \def\ntitle{Template}
\else
\fi

\ifx \nauthoremail\undefined
  \def\nauthoremail{qk206@cam.ac.uk}
\else
\fi

\ifx \ndate\undefined
  \def\ndate{\today}
\else
\fi

\title{\ntitle}
\author{\nauthor}
\date{\ndate}

%\usepackage{microtype}
\usepackage{mathtools}
\usepackage{amsthm}
\usepackage{stmaryrd}%symbols used so far: \mapsfrom
\usepackage{empheq}
\usepackage{amssymb}
\let\mathbbalt\mathbb
\let\pitchforkold\pitchfork
\usepackage{unicode-math}
\let\mathbb\mathbbalt%reset to original \mathbb
\let\pitchfork\pitchforkold

\usepackage{imakeidx}
\makeindex[intoc]

%to address the problem that Latin modern doesn't have unicode support for setminus
%https://tex.stackexchange.com/a/55205/26707
\AtBeginDocument{\renewcommand*{\setminus}{\mathbin{\backslash}}}
\AtBeginDocument{\renewcommand*{\models}{\vDash}}%for \vDash is same size as \vdash but orginal \models is larger
\AtBeginDocument{\let\Re\relax}
\AtBeginDocument{\let\Im\relax}
\AtBeginDocument{\DeclareMathOperator{\Re}{Re}}
\AtBeginDocument{\DeclareMathOperator{\Im}{Im}}
\AtBeginDocument{\let\div\relax}
\AtBeginDocument{\DeclareMathOperator{\div}{div}}

\usepackage{tikz}
\usetikzlibrary{automata,positioning}
\usepackage{pgfplots}
%some preset styles
\pgfplotsset{compat=1.15}
\pgfplotsset{centre/.append style={axis x line=middle, axis y line=middle, xlabel={$x$}, ylabel={$y$}, axis equal}}
\usepackage{tikz-cd}
\usepackage{graphicx}
\usepackage{newunicodechar}

\usepackage{fancyhdr}

\fancypagestyle{mypagestyle}{
    \fancyhf{}
    \lhead{\emph{\nouppercase{\leftmark}}}
    \rhead{}
    \cfoot{\thepage}
}
\pagestyle{mypagestyle}

\usepackage{titlesec}
\newcommand{\sectionbreak}{\clearpage} % clear page after each section
\usepackage[perpage]{footmisc}
\usepackage{blindtext}

%\reallywidehat
%https://tex.stackexchange.com/a/101136/26707
\usepackage{scalerel,stackengine}
\stackMath
\newcommand\reallywidehat[1]{%
\savestack{\tmpbox}{\stretchto{%
  \scaleto{%
    \scalerel*[\widthof{\ensuremath{#1}}]{\kern-.6pt\bigwedge\kern-.6pt}%
    {\rule[-\textheight/2]{1ex}{\textheight}}%WIDTH-LIMITED BIG WEDGE
  }{\textheight}% 
}{0.5ex}}%
\stackon[1pt]{#1}{\tmpbox}%
}

%\usepackage{braket}
\usepackage{thmtools}%restate theorem
\usepackage{hyperref}

% https://en.wikibooks.org/wiki/LaTeX/Hyperlinks
\hypersetup{
    %bookmarks=true,
    unicode=true,
    pdftitle={\ntitle},
    pdfauthor={\nauthor},
    pdfsubject={Mathematics},
    pdfcreator={\nauthor},
    pdfproducer={\nauthor},
    pdfkeywords={math maths \ntitle},
    colorlinks=true,
    linkcolor={red!50!black},
    citecolor={blue!50!black},
    urlcolor={blue!80!black}
}

\usepackage{cleveref}



% TODO: mdframed often gives bad breaks that cause empty lines. Would like to switch to tcolorbox.
% The current workaround is to set innerbottommargin=0pt.

%\usepackage[theorems]{tcolorbox}





\usepackage[framemethod=tikz]{mdframed}
\mdfdefinestyle{leftbar}{
  %nobreak=true, %dirty hack
  linewidth=1.5pt,
  linecolor=gray,
  hidealllines=true,
  leftline=true,
  leftmargin=0pt,
  innerleftmargin=5pt,
  innerrightmargin=10pt,
  innertopmargin=-5pt,
  % innerbottommargin=5pt, % original
  innerbottommargin=0pt, % temporary hack 
}
%\newmdtheoremenv[style=leftbar]{theorem}{Theorem}[section]
%\newmdtheoremenv[style=leftbar]{proposition}[theorem]{proposition}
%\newmdtheoremenv[style=leftbar]{lemma}[theorem]{Lemma}
%\newmdtheoremenv[style=leftbar]{corollary}[theorem]{corollary}

\newtheorem{theorem}{Theorem}[section]
\newtheorem{proposition}[theorem]{Proposition}
\newtheorem{lemma}[theorem]{Lemma}
\newtheorem{corollary}[theorem]{Corollary}
\newtheorem{axiom}[theorem]{Axiom}
\newtheorem*{axiom*}{Axiom}

\surroundwithmdframed[style=leftbar]{theorem}
\surroundwithmdframed[style=leftbar]{proposition}
\surroundwithmdframed[style=leftbar]{lemma}
\surroundwithmdframed[style=leftbar]{corollary}
\surroundwithmdframed[style=leftbar]{axiom}
\surroundwithmdframed[style=leftbar]{axiom*}

\theoremstyle{definition}

\newtheorem*{definition}{Definition}
\surroundwithmdframed[style=leftbar]{definition}

\newtheorem*{slogan}{Slogan}
\newtheorem*{eg}{Example}
\newtheorem*{ex}{Exercise}
\newtheorem*{remark}{Remark}
\newtheorem*{notation}{Notation}
\newtheorem*{convention}{Convention}
\newtheorem*{assumption}{Assumption}
\newtheorem*{question}{Question}
\newtheorem*{answer}{Answer}
\newtheorem*{note}{Note}
\newtheorem*{application}{Application}

%operator macros

%basic
\DeclareMathOperator{\lcm}{lcm}

%matrix
\DeclareMathOperator{\tr}{tr}
\DeclareMathOperator{\Tr}{Tr}
\DeclareMathOperator{\adj}{adj}

%algebra
\DeclareMathOperator{\Hom}{Hom}
\DeclareMathOperator{\End}{End}
\DeclareMathOperator{\id}{id}
\DeclareMathOperator{\im}{im}
\DeclareMathOperator{\coker}{coker}
\DeclarePairedDelimiter{\generation}{\langle}{\rangle}

%groups
\DeclareMathOperator{\sym}{Sym}
\DeclareMathOperator{\sgn}{sgn}
\DeclareMathOperator{\inn}{Inn}
\DeclareMathOperator{\aut}{Aut}
\DeclareMathOperator{\GL}{GL}
\DeclareMathOperator{\SL}{SL}
\DeclareMathOperator{\PGL}{PGL}
\DeclareMathOperator{\PSL}{PSL}
\DeclareMathOperator{\SU}{SU}
\DeclareMathOperator{\UU}{U}
\DeclareMathOperator{\SO}{SO}
\DeclareMathOperator{\OO}{O}
\DeclareMathOperator{\PSU}{PSU}
\DeclareMathOperator{\Sp}{Sp}


%hyperbolic
\DeclareMathOperator{\sech}{sech}

%field, galois heory
\DeclareMathOperator{\ch}{ch}
\DeclareMathOperator{\gal}{Gal}
\DeclareMathOperator{\emb}{Emb}



%ceiling and floor
%https://tex.stackexchange.com/a/118217/26707
\DeclarePairedDelimiter\ceil{\lceil}{\rceil}
\DeclarePairedDelimiter\floor{\lfloor}{\rfloor}


\DeclarePairedDelimiter{\innerproduct}{\langle}{\rangle}

%\DeclarePairedDelimiterX{\norm}[1]{\lVert}{\rVert}{#1}
\DeclarePairedDelimiter{\norm}{\lVert}{\rVert}



%Dirac notation
%TODO: rewrite for variable number of arguments
\DeclarePairedDelimiterX{\braket}[2]{\langle}{\rangle}{#1 \delimsize\vert #2}
\DeclarePairedDelimiterX{\braketthree}[3]{\langle}{\rangle}{#1 \delimsize\vert #2 \delimsize\vert #3}

\DeclarePairedDelimiter{\bra}{\langle}{\rvert}
\DeclarePairedDelimiter{\ket}{\lvert}{\rangle}




%macros

%general

%divide, not divide
\newcommand*{\divides}{\mid}
\newcommand*{\ndivides}{\nmid}
%vector, i.e. mathbf
%https://tex.stackexchange.com/a/45746/26707
\newcommand*{\V}[1]{{\ensuremath{\symbf{#1}}}}
%closure
\newcommand*{\cl}[1]{\overline{#1}}
%conjugate
\newcommand*{\conj}[1]{\overline{#1}}
%set complement
\newcommand*{\stcomp}[1]{\overline{#1}}
\newcommand*{\compose}{\circ}
\newcommand*{\nto}{\nrightarrow}
\newcommand*{\p}{\partial}
%embed
\newcommand*{\embed}{\hookrightarrow}
%surjection
\newcommand*{\surj}{\twoheadrightarrow}
%power set
\newcommand*{\powerset}{\mathcal{P}}

%matrix
\newcommand*{\matrixring}{\mathcal{M}}

%groups
\newcommand*{\normal}{\trianglelefteq}
%rings
\newcommand*{\ideal}{\trianglelefteq}

%fields
\renewcommand*{\C}{{\mathbb{C}}}
\newcommand*{\R}{{\mathbb{R}}}
\newcommand*{\Q}{{\mathbb{Q}}}
\newcommand*{\Z}{{\mathbb{Z}}}
\newcommand*{\N}{{\mathbb{N}}}
\newcommand*{\F}{{\mathbb{F}}}
%not really but I think this belongs here
\newcommand*{\A}{{\mathbb{A}}}

%asymptotic
\newcommand*{\bigO}{O}
\newcommand*{\smallo}{o}

%probability
\newcommand*{\prob}{\mathbb{P}}
\newcommand*{\E}{\mathbb{E}}

%vector calculus
\newcommand*{\gradient}{\V \nabla}
\newcommand*{\divergence}{\gradient \cdot}
\newcommand*{\curl}{\gradient \cdot}

%logic
\newcommand*{\yields}{\vdash}
\newcommand*{\nyields}{\nvdash}

%differential geometry
\renewcommand*{\H}{\mathbb{H}}
\newcommand*{\transversal}{\pitchfork}
\renewcommand{\d}{\mathrm{d}} % exterior derivative

%number theory
\newcommand*{\legendre}[2]{\genfrac{(}{)}{}{}{#1}{#2}}%Legendre symbol

%algebraic geometry
\DeclareMathOperator{\Spec}{Spec}
\DeclareMathOperator{\Proj}{Proj}

\newcommand{\tilt}{\flat} % tilting
\newcommand{\perf}{\mathrm{perf}}
%\DeclareMathOperator{\perf}{perf} % perfection
\renewcommand{\c}[1]{\mathbf{#1}}
\newcommand{\Mod}{{\c{Mod}}}
\DeclareMathOperator{\Tor}{Tor} % torsion
\DeclareMathOperator{\Ext}{Ext} % extension
\newcommand{\sh}[1]{\mathcal{#1}} % sheaf
\DeclareMathOperator{\Spa}{Spa}
\renewcommand*{\O}{\mathcal{O}}


\newtheorem*{construction}{Construction}

\iffalse
\renewcommand*{\P}{\mathbb{P}}
\newcommand{\sh}[1]{\mathcal{#1}} % sheaf
\renewcommand*{\O}{\mathcal{O}}
\let\Sp\Relax
\DeclareMathOperator{\Sp}{Sp} % maximum spectrum
\DeclareMathOperator{\Max}{Max}
\DeclareMathOperator{\Spf}{Spf}
\DeclareMathOperator{\Spa}{Spa}
\DeclareMathOperator{\supp}{supp} % support of a valuation
\fi

\begin{document}

\begin{titlepage}
  \begin{center}
    \includegraphics[width=0.6\textwidth]{logo.jpg}\par
    \vspace{1cm}
    {\scshape\huge Mathamatics Tripos \par}
    \vspace{2cm}
    {\huge Part \npart \par}
    \vspace{0.6cm}
    {\Huge \bfseries \ntitle \par}
    \vspace{1.2cm}
    {\Large\nterm, \nyear \par}
    \vspace{2cm}
    
    {\large \emph{Lectures by } \par}
    \vspace{0.2cm}
    {\Large \scshape \nlecturer}
    
    \vspace{0.5cm}
    {\large \emph{Notes by }\par}
    \vspace{0.2cm}
    {\Large \scshape \href{mailto:\nauthoremail}{\nauthor}}
 \end{center}
\end{titlepage}

\tableofcontents

\section{Introduction}

Course structure:
\begin{enumerate}
\item introduction
\item Hodge-Tate decomposition for abelian varieties with good reduction
\item Hodge-Tate decomposition in generale (pro-étale cohomology)
\item integral aspects
\item some additional topics (Hodge-Tate decomposition theorem for rigid analytic varieties)
\end{enumerate}

\subsection{Hodge decomposition over \(\C\)}

Let \(X\) be a smooth projective variety over \(\C\). The \emph{Hodge decomposition} is a direct sum decomposition for all \(n \geq 0\)
\[
  H_{\mathrm{sing}}^n(X^{\mathrm{an}}, \C) = \bigoplus_{p + q = n} H^{p, q}
\]
where LHS is the singular cohomology of the \(\C\)-analytic manifold \(X^{\mathrm{an}}\) (the complex analytification) and on RHS
\[
  H^{p, q} = H^q(X^{\mathrm{an}}, \Omega_{X^{\mathrm{an}}}^p)
\]
with \(\Omega_{X^{\mathrm{an}}}^p\) denoting the sheaf of holomorphic \(p\)-forms. Moreover, complex conjugation acts on
\[
  H^n_{\mathrm{sing}}(X^{\mathrm{an}}, \C) \cong H_{\mathrm{sing}}^n(X^{\mathrm{an}}, \Q) \otimes \C
\]
via its action on \(\C\) and \(H^{p, q} = \overline{H^{q, p}}\). This is called a \emph{pure structure of weight \(n\)}.

These are proven via identifying \(H^{p, q}\) with Dolbeault cohomology and using the (very deep) theory of harmonic forms. However, part of the theory can be understood purely algebraically. It is known that \(H^n_{\mathrm{sing}}(X^{\mathrm{an}}, \C)\) gives the cohomology of the constant sheaf \(\C\) on \(X^{\mathrm{an}}\). On the other hand, consider the de Rham complex
\[
  \Omega_{X^{\mathrm{an}}}^\bullet = \O_{X^{\mathrm{an}}} \xrightarrow{\d} \Omega_{X^{\mathrm{an}}}^1 \xrightarrow{\d} \Omega_{X^{\mathrm{an}}}^2 \to \cdots
\]
Here \(\d\) is the usual derivation and the higher \(\d\)'s are given by
\[
  \d (\omega_1 \wedge \omega_2) = \d \omega_1 \wedge\omega_2 + (-1)^p \omega_1 \wedge\d \omega_2
\]
for \(\omega_1 \in \Omega_{X^{\mathrm{an}}}^p, \omega_2 \in \Omega_{X^{\mathrm{an}}}^q\). Taking hypercohomology
\[
  H^n_{\mathrm{dR}}(X^{\mathrm{an}}) := \H(X^{\mathrm{an}}, \Omega_{X^{\mathrm{an}}}^\bullet)
\]
we get the so-called de Rham cohomology group.

Embedding the constant sheaf \(\C\) into \(\O_{X^{\mathrm{an}}}\) induces a map \(\C \to \Omega_{X^{\mathrm{an}}}^\bullet\) of complexes of sheaves. The (holomorphic) Poincaré lemma states that this map is a quasi-isomorphism of sheaves. More precisely, one can cover \(X^{\mathrm{an}}\) by open balls and for any open ball \(U \subseteq X^{\mathrm{an}}\), the complex
\[
    0 \to \C \to \O_{X^{\mathrm{an}}}(U) \xrightarrow{\d} \Omega^1_{X^{\mathrm{an}}} \to \cdots
\]
is exact: any closed differential form can be integrated on an open ball. Thus
\[
  H^n_{\mathrm{sing}}(X^{\mathrm{an}}, \C) \cong H^n_{\mathrm{dR}}(X^{\mathrm{an}}).
\]
This is the comparison theorem between singular and de Rham cohomology.

Now the complex \(\Omega_{X^{\mathrm{an}}}^\bullet\) has a decreasing filtration of subcomplexes
\[
  \Omega_{X^{\mathrm{an}}}^{\geq p} := 0 \to \cdots \to 0 \to \Omega^p_{X^{\mathrm{an}}} \xrightarrow{\d} \Omega_{X^{\mathrm{an}}}^{p + 1} \xrightarrow{\d} \cdots
\]
We have that \(\operatorname{gr}^p \Omega^\bullet_{X^{\mathrm{an}}} \cong \Omega^p_{X^{\mathrm{an}}}\). It is well-known that there is a convergent spectral sequence associated to \(\Omega_{X^{\mathrm{an}}}^\bullet\) with the filtration above, called the \emph{Hodge to de Rham spectral sequence}
\[
  E_1^{pq} = H^q(X^{\mathrm{an}}, \Omega^p_{X^{\mathrm{an}}}) \Rightarrow H^{p + q}_{\mathrm{dR}}(X^{\mathrm{an}}).
\]
The filtration on \(H^n_{\mathrm{dR}}(X^{\mathrm{an}})\) given by the spectral sequence is called the \emph{Hodge filtration}.

Fact: the Hodge to de Rham spectral sequence degenerates at \(E_1\). This together with the comparison theorem gives the Hodge decomposition
\[
  H^n_{\mathrm{sing}}(X^{\mathrm{an}}, \C) = \bigoplus_{p + q = n} H^q(X^{\mathrm{an}}, \Omega_{X^{\mathrm{an}}}^p)
\]
for all \(n\).

\subsection{Algebraisation}

On a complex variety \(X\) we may consider the algebraic de Rham complex
\[
  \Omega_X^\bullet := \O_X \xrightarrow{\d} \Omega_X^1 \xrightarrow{\d} \cdots
\]
For \(X\) smooth these are locally free sheaves. The same way as above, we get the algebraic Hodge to de Rham spectral sequence
\[
  E_1^{pq} = H^q(X, \Omega_X^p) \Rightarrow H^{p + q}_{\mathrm{dR}}(X).
\]
Here we use the Zariski topology.

There are two natural maps
\begin{align*}
  H^q(X, \Omega_X^p) &\to H^q(X^{\mathrm{an}}, \Omega_{X^{\mathrm{an}}}^p) \\
  H^{p +q}_{\mathrm{dR}}(X) &\to H^{p +q}_{\mathrm{dR}}(X^{\mathrm{an}})
\end{align*}
all compatible with the maps in the above spectral sequences. By GAGA the first one is an isomorphism, and by a theorem of Grothendieck the second is also an isomorphism. Hence degeneration of the analytic Hodge to de Rham is equivalent to the degeneration of the algebraic counterpart.

However, there is no algebraic Poincaré lemma, the algebraic de Rham complex is not a resolution of \(\C\) and anyway the sheaf cohomology of \(\C\) is trivial in the Zariski topology.

\subsection{The case of a \(p\)-adic base field}

Let \(p\) be a prime. Recall \(\C_p\) is the completion of the algebraic closure \(\overline Q_p\) of \(\Q_p\). The Galois group \(\gal(\overline \Q_p/\Q_p)\) acts on \(\C_p\) by continuity. Let \(K\) be a finite extension of \(\Q_p\).



\printindex
\end{document}