\documentclass[a4paper]{article}

\def\npart{IB}

\def\ntitle{Complex Analysis}
\def\nlecturer{G.\ P.\ Paternain}

\def\nterm{Lent}
\def\nyear{2018}

\ifx \nauthor\undefined
  \def\nauthor{Qiangru Kuang}
\else
\fi

\ifx \ntitle\undefined
  \def\ntitle{Template}
\else
\fi

\ifx \nauthoremail\undefined
  \def\nauthoremail{qk206@cam.ac.uk}
\else
\fi

\ifx \ndate\undefined
  \def\ndate{\today}
\else
\fi

\title{\ntitle}
\author{\nauthor}
\date{\ndate}

%\usepackage{microtype}
\usepackage{mathtools}
\usepackage{amsthm}
\usepackage{stmaryrd}%symbols used so far: \mapsfrom
\usepackage{empheq}
\usepackage{amssymb}
\let\mathbbalt\mathbb
\let\pitchforkold\pitchfork
\usepackage{unicode-math}
\let\mathbb\mathbbalt%reset to original \mathbb
\let\pitchfork\pitchforkold

\usepackage{imakeidx}
\makeindex[intoc]

%to address the problem that Latin modern doesn't have unicode support for setminus
%https://tex.stackexchange.com/a/55205/26707
\AtBeginDocument{\renewcommand*{\setminus}{\mathbin{\backslash}}}
\AtBeginDocument{\renewcommand*{\models}{\vDash}}%for \vDash is same size as \vdash but orginal \models is larger
\AtBeginDocument{\let\Re\relax}
\AtBeginDocument{\let\Im\relax}
\AtBeginDocument{\DeclareMathOperator{\Re}{Re}}
\AtBeginDocument{\DeclareMathOperator{\Im}{Im}}
\AtBeginDocument{\let\div\relax}
\AtBeginDocument{\DeclareMathOperator{\div}{div}}

\usepackage{tikz}
\usetikzlibrary{automata,positioning}
\usepackage{pgfplots}
%some preset styles
\pgfplotsset{compat=1.15}
\pgfplotsset{centre/.append style={axis x line=middle, axis y line=middle, xlabel={$x$}, ylabel={$y$}, axis equal}}
\usepackage{tikz-cd}
\usepackage{graphicx}
\usepackage{newunicodechar}

\usepackage{fancyhdr}

\fancypagestyle{mypagestyle}{
    \fancyhf{}
    \lhead{\emph{\nouppercase{\leftmark}}}
    \rhead{}
    \cfoot{\thepage}
}
\pagestyle{mypagestyle}

\usepackage{titlesec}
\newcommand{\sectionbreak}{\clearpage} % clear page after each section
\usepackage[perpage]{footmisc}
\usepackage{blindtext}

%\reallywidehat
%https://tex.stackexchange.com/a/101136/26707
\usepackage{scalerel,stackengine}
\stackMath
\newcommand\reallywidehat[1]{%
\savestack{\tmpbox}{\stretchto{%
  \scaleto{%
    \scalerel*[\widthof{\ensuremath{#1}}]{\kern-.6pt\bigwedge\kern-.6pt}%
    {\rule[-\textheight/2]{1ex}{\textheight}}%WIDTH-LIMITED BIG WEDGE
  }{\textheight}% 
}{0.5ex}}%
\stackon[1pt]{#1}{\tmpbox}%
}

%\usepackage{braket}
\usepackage{thmtools}%restate theorem
\usepackage{hyperref}

% https://en.wikibooks.org/wiki/LaTeX/Hyperlinks
\hypersetup{
    %bookmarks=true,
    unicode=true,
    pdftitle={\ntitle},
    pdfauthor={\nauthor},
    pdfsubject={Mathematics},
    pdfcreator={\nauthor},
    pdfproducer={\nauthor},
    pdfkeywords={math maths \ntitle},
    colorlinks=true,
    linkcolor={red!50!black},
    citecolor={blue!50!black},
    urlcolor={blue!80!black}
}

\usepackage{cleveref}



% TODO: mdframed often gives bad breaks that cause empty lines. Would like to switch to tcolorbox.
% The current workaround is to set innerbottommargin=0pt.

%\usepackage[theorems]{tcolorbox}





\usepackage[framemethod=tikz]{mdframed}
\mdfdefinestyle{leftbar}{
  %nobreak=true, %dirty hack
  linewidth=1.5pt,
  linecolor=gray,
  hidealllines=true,
  leftline=true,
  leftmargin=0pt,
  innerleftmargin=5pt,
  innerrightmargin=10pt,
  innertopmargin=-5pt,
  % innerbottommargin=5pt, % original
  innerbottommargin=0pt, % temporary hack 
}
%\newmdtheoremenv[style=leftbar]{theorem}{Theorem}[section]
%\newmdtheoremenv[style=leftbar]{proposition}[theorem]{proposition}
%\newmdtheoremenv[style=leftbar]{lemma}[theorem]{Lemma}
%\newmdtheoremenv[style=leftbar]{corollary}[theorem]{corollary}

\newtheorem{theorem}{Theorem}[section]
\newtheorem{proposition}[theorem]{Proposition}
\newtheorem{lemma}[theorem]{Lemma}
\newtheorem{corollary}[theorem]{Corollary}
\newtheorem{axiom}[theorem]{Axiom}
\newtheorem*{axiom*}{Axiom}

\surroundwithmdframed[style=leftbar]{theorem}
\surroundwithmdframed[style=leftbar]{proposition}
\surroundwithmdframed[style=leftbar]{lemma}
\surroundwithmdframed[style=leftbar]{corollary}
\surroundwithmdframed[style=leftbar]{axiom}
\surroundwithmdframed[style=leftbar]{axiom*}

\theoremstyle{definition}

\newtheorem*{definition}{Definition}
\surroundwithmdframed[style=leftbar]{definition}

\newtheorem*{slogan}{Slogan}
\newtheorem*{eg}{Example}
\newtheorem*{ex}{Exercise}
\newtheorem*{remark}{Remark}
\newtheorem*{notation}{Notation}
\newtheorem*{convention}{Convention}
\newtheorem*{assumption}{Assumption}
\newtheorem*{question}{Question}
\newtheorem*{answer}{Answer}
\newtheorem*{note}{Note}
\newtheorem*{application}{Application}

%operator macros

%basic
\DeclareMathOperator{\lcm}{lcm}

%matrix
\DeclareMathOperator{\tr}{tr}
\DeclareMathOperator{\Tr}{Tr}
\DeclareMathOperator{\adj}{adj}

%algebra
\DeclareMathOperator{\Hom}{Hom}
\DeclareMathOperator{\End}{End}
\DeclareMathOperator{\id}{id}
\DeclareMathOperator{\im}{im}
\DeclareMathOperator{\coker}{coker}
\DeclarePairedDelimiter{\generation}{\langle}{\rangle}

%groups
\DeclareMathOperator{\sym}{Sym}
\DeclareMathOperator{\sgn}{sgn}
\DeclareMathOperator{\inn}{Inn}
\DeclareMathOperator{\aut}{Aut}
\DeclareMathOperator{\GL}{GL}
\DeclareMathOperator{\SL}{SL}
\DeclareMathOperator{\PGL}{PGL}
\DeclareMathOperator{\PSL}{PSL}
\DeclareMathOperator{\SU}{SU}
\DeclareMathOperator{\UU}{U}
\DeclareMathOperator{\SO}{SO}
\DeclareMathOperator{\OO}{O}
\DeclareMathOperator{\PSU}{PSU}
\DeclareMathOperator{\Sp}{Sp}


%hyperbolic
\DeclareMathOperator{\sech}{sech}

%field, galois heory
\DeclareMathOperator{\ch}{ch}
\DeclareMathOperator{\gal}{Gal}
\DeclareMathOperator{\emb}{Emb}



%ceiling and floor
%https://tex.stackexchange.com/a/118217/26707
\DeclarePairedDelimiter\ceil{\lceil}{\rceil}
\DeclarePairedDelimiter\floor{\lfloor}{\rfloor}


\DeclarePairedDelimiter{\innerproduct}{\langle}{\rangle}

%\DeclarePairedDelimiterX{\norm}[1]{\lVert}{\rVert}{#1}
\DeclarePairedDelimiter{\norm}{\lVert}{\rVert}



%Dirac notation
%TODO: rewrite for variable number of arguments
\DeclarePairedDelimiterX{\braket}[2]{\langle}{\rangle}{#1 \delimsize\vert #2}
\DeclarePairedDelimiterX{\braketthree}[3]{\langle}{\rangle}{#1 \delimsize\vert #2 \delimsize\vert #3}

\DeclarePairedDelimiter{\bra}{\langle}{\rvert}
\DeclarePairedDelimiter{\ket}{\lvert}{\rangle}




%macros

%general

%divide, not divide
\newcommand*{\divides}{\mid}
\newcommand*{\ndivides}{\nmid}
%vector, i.e. mathbf
%https://tex.stackexchange.com/a/45746/26707
\newcommand*{\V}[1]{{\ensuremath{\symbf{#1}}}}
%closure
\newcommand*{\cl}[1]{\overline{#1}}
%conjugate
\newcommand*{\conj}[1]{\overline{#1}}
%set complement
\newcommand*{\stcomp}[1]{\overline{#1}}
\newcommand*{\compose}{\circ}
\newcommand*{\nto}{\nrightarrow}
\newcommand*{\p}{\partial}
%embed
\newcommand*{\embed}{\hookrightarrow}
%surjection
\newcommand*{\surj}{\twoheadrightarrow}
%power set
\newcommand*{\powerset}{\mathcal{P}}

%matrix
\newcommand*{\matrixring}{\mathcal{M}}

%groups
\newcommand*{\normal}{\trianglelefteq}
%rings
\newcommand*{\ideal}{\trianglelefteq}

%fields
\renewcommand*{\C}{{\mathbb{C}}}
\newcommand*{\R}{{\mathbb{R}}}
\newcommand*{\Q}{{\mathbb{Q}}}
\newcommand*{\Z}{{\mathbb{Z}}}
\newcommand*{\N}{{\mathbb{N}}}
\newcommand*{\F}{{\mathbb{F}}}
%not really but I think this belongs here
\newcommand*{\A}{{\mathbb{A}}}

%asymptotic
\newcommand*{\bigO}{O}
\newcommand*{\smallo}{o}

%probability
\newcommand*{\prob}{\mathbb{P}}
\newcommand*{\E}{\mathbb{E}}

%vector calculus
\newcommand*{\gradient}{\V \nabla}
\newcommand*{\divergence}{\gradient \cdot}
\newcommand*{\curl}{\gradient \cdot}

%logic
\newcommand*{\yields}{\vdash}
\newcommand*{\nyields}{\nvdash}

%differential geometry
\renewcommand*{\H}{\mathbb{H}}
\newcommand*{\transversal}{\pitchfork}
\renewcommand{\d}{\mathrm{d}} % exterior derivative

%number theory
\newcommand*{\legendre}[2]{\genfrac{(}{)}{}{}{#1}{#2}}%Legendre symbol

%algebraic geometry
\DeclareMathOperator{\Spec}{Spec}
\DeclareMathOperator{\Proj}{Proj}

\newcommand*{\D}{\mathcal{D}}

\makeindex

\begin{document}

\begin{titlepage}
  \begin{center}
    \includegraphics[width=0.6\textwidth]{logo.jpg}\par
    \vspace{1cm}
    {\scshape\huge Mathamatics Tripos \par}
    \vspace{2cm}
    {\huge Part \npart \par}
    \vspace{0.6cm}
    {\Huge \bfseries \ntitle \par}
    \vspace{1.2cm}
    {\Large\nterm, \nyear \par}
    \vspace{2cm}
    
    {\large \emph{Lectures by } \par}
    \vspace{0.2cm}
    {\Large \scshape \nlecturer}
    
    \vspace{0.5cm}
    {\large \emph{Notes by }\par}
    \vspace{0.2cm}
    {\Large \scshape \href{mailto:\nauthoremail}{\nauthor}}
 \end{center}
\end{titlepage}

\tableofcontents

\section{Basic notions}

Some preliminary notations/definitions:

\begin{notation}\leavevmode
  \begin{itemize}
  \item \(D(a, r)\) is the open disc of radius \(r > 0\) and centred at \(a \in \C\).
  \item \(U \subseteq \C\) is open if for any \(a \in U\), there exists \(\varepsilon > 0\) such that \(D(a, \varepsilon) \subseteq U\).
  \item A \emph{curve} is a continuous map from a closed interval \(\varphi: [a, b] \to \C\). It is continuously differentiably, i.e.\ \(C^1\), if \(\varphi'\) exists and is continuous on \([a, b]\).
  \item An open set \(U \subseteq \C\) is \emph{path-connected} if for every \(z, w \in U\) there exists a curve \(\varphi: [0, 1] \to U\) with endpoints \(z, w\).
  \end{itemize}
\end{notation}

\begin{definition}[Domain]\index{domain}
  A \emph{domain} is an non-empty path-connected open subset of \(\C\).
\end{definition}

The goal of this course is to study functions \(f: U \to \C\) where \(U \subseteq \C\) is open or is a domain. Given such an \(f\), we may write
\[
  f(x + iy) = u(x, y) + iv(x, y)
\]
where \(u, v: U \to \R\) are the real and imaginary part of \(f\). Here we use \((x, y) \in \R^2\) to denote the coordinates of \(a + bi \in \C\).

\begin{definition}[Differentiable, holomorphic]\index{differentiable}\index{holomorphic}\leavevmode
  \begin{enumerate}
  \item \(f: U \to \C\) is differentiable at \(w \in U\) if the limit
    \[
      f'(w) = \lim_{z \to w} \frac{f(z) - f(w)}{z - w}
    \]
    exists. \(f'(w)\) is called the \emph{derivative} of \(f\) at \(w\).
  \item \(f\) is \emph{holomorphic} at \(w\) if there exists \(\varepsilon > 0\) such that \(f\) is differentiable at all points of \(D(w, \varepsilon)\). \(f\) is holomorphic on \(U\) if it is differentiable at all \(w \in U\). Equivalent, \(f\) is holomorphic at at all \(w \in U\).
  \end{enumerate}
\end{definition}

\begin{remark}\leavevmode
  \begin{enumerate}
  \item There is an alternative term \emph{analytic}. In actuality, it is the same as holomorphic for complex functions. However, it comes with a flavour associated with the Taylor expansion and sometimes defined in terms of such. Later in the course we will prove that the definitions are equivalent.
  \item Complex differentiation follows the same rules as real differentiation. For example, sums of differentiable functions are differentiable, product, quotient, chain rules etc also hold.
  \end{enumerate}
\end{remark}

\begin{definition}[Entire]\index{entire}
  An \emph{entire} function is a holomorphic function \(f: \C \to \C\).
\end{definition}

\begin{eg}\leavevmode
  \begin{enumerate}
  \item Polynomials are entire functions.
  \item If \(p(z)\) and \(q(z)\) are polynomials with \(q(z)\) not identically zero, then \(\frac{p}{q}\) is holomorphic on \(\C \setminus \{\text{zeros of } q\}\).
  \end{enumerate}
\end{eg}

In IB Analysis II we studied function from \(\R^n\) to \(\R^m\) and their differentiability. Indeed a complex function \(\C \to \C\) can be view as a function \(\R^2 \to \R^2\) so how does complex differentiability relates to differentiablitiy in \(\R^2\)? It turns out that in addition to satisfy the differentiability on \(\R^2\), the function has to satisfy a particular partial differential equation.

Recall that \(u\) is differentiable at \((c, d) \in U\) if there exists \((\lambda, \mu) \in \R^2\) such that
\[
  \frac{u(x, y) - u(c, d) - (\lambda(x - c) + \mu(y -d))}{\sqrt{(x - c)^2 + (y - d)^2}} \to 0
\]
as \((x, y) \to (c, d)\). In this case \(\D u(c, d) = (\lambda, \mu)\) is the derivative of \(u\) at \((c, d)\). If this holds then \(\lambda = u_x(c, d)\) and \(\mu = u_y(c, d)\), the partical derivatives of \(u\) at \((c, d)\).

\begin{theorem}[Cauchy-Riemann equations]\index{Cauchy-Riemann}
  \(f: U \to \C\) is differentiable at \(w = c + id \in U\) if and only if the functions \(u\) and \(v\) are differentiable at \((c, d)\) and
  \begin{align*}
    u_x(c, d) &= v_y(c, d) \\
    u_y(c, d) &= -v_x(c, d)
  \end{align*}
  in which case
  \[
    f'(w) = u_x(c, d) + iv_x(c, d).
  \]
\end{theorem}

\begin{proof}
  From the definition, \(f\) will be differentiable at \(w\) with derivative \(f'(w) = p + iq\) if and only if
  \[
    \lim_{z \to w} \frac{f(z) - f(w) - f'(w)(z - w)}{|z - w|} = 0
  \]
  or equivalently, splitting into real and imaginary parts,
  \begin{align*}
    \lim_{(x, y) \to (c, d)} \frac{u(x, y) - u(c, d) - (p(x - c) - q(y - d))}{\sqrt{(x - c)^2 + (y - d)^2}} &= 0 \\
    \lim_{(x, y) \to (c, d)} \frac{v(x, y) - v(c, d) - (q(x - c) + p(y - d))}{\sqrt{(x - c)^2 + (y - d)^2}} &= 0
  \end{align*}
  since \(f'(w)(z - w) = (p(x - c) - q(y - d)) + i(q(x - c) + p(y - d))\). So \(f\) is differentiable at \(w\) with derivative \(f'(w) = p + iq\) if and only if \(u\) and \(v\) are differentiable at \((c, d)\) with
  \begin{align*}
    \D u(c, d) &= (p, -q) \\
    \D v(c, d) &= (q, p)
  \end{align*}
  hence the result.
\end{proof}

\begin{eg}
  Let \(f(z) = \conj z\). Then \(f(x + iy) = x - iy\), \(u(x, y) = x, v(x, y) = -y\). We have
  \[
    u_x = 1 \neq -1 = v_y
  \]
  so \(f\) is not differentiable anywhere in \(\C\).
\end{eg}

\begin{remark}\leavevmode
  \begin{enumerate}
  \item We could have discovered Cauchy-Riemann as follows: let \(z = w + h\) where \(h \in \R\). Then
    \[
      f'(w) = \lim_{h \to 0} \frac{f(w + h) - f(w)}{h} = u_x(c, d) + iv_x(c, d).
    \]
    Let \(z = w + ih\), we get
    \[
      f'(w) = \lim_{h \to 0} \frac{f(w + ih) - f(w)}{h} = v_y(c, d) - iu_y(c, d).
    \]
    As \(f\) is assumed to be differentiable, these two must agree.
  \item Later on we'll see that if \(f\) is holomorphic, so is \(f'\). This will imply right away that all partial derivatives of \(u\) and \(v\) exist and are continuous, i.e.\ \(C^\infty\). Thus
    \begin{align*}
      u_{xx} &= v_{yx} \\
      u_{yy} &= -v_{xy}
    \end{align*}
    By symmetry of second derivatives, we get that \(u\) satisfies the Laplace equation
    \[
      u_{xx} + u_{yy} = 0.
    \]
    Hence \(f\) is holomorphic implies that the real and imaginary parts are harmonic functions.
  \end{enumerate}
\end{remark}

\begin{corollary}
  Let \(f = u + iv: U \to \C\). Suppose the functions \(u\) and \(v\) have continuous partial derivatives everywhere in \(U\) and satisfy Cauchy-Riemann equations. Then \(f\) is holomorphic in \(U\).
\end{corollary}

\begin{proof}
  From IB Analysis II, \(u\) and \(v\) are differentiable. The result follows from the previous theorem.
\end{proof}

\begin{corollary}
  Let \(f: D \to \C\) be holomorphic on a domain \(D\) and suppose \(f'(z) = 0\) for all \(z \in D\). Then \(f\) is constant on \(D\).
\end{corollary}

\begin{proof}
  Follows from the analogous result for differentiable functions on a path-connected subset of \(\R^2\) (Mean Value Inequality from IB Analysis II).
\end{proof}

\section{Power Series}

Recall that

\begin{theorem}[Radius of convergence]\index{radius of convergence}
  Let \(c_n\) be a sequence of complex numbers. Then there exists a unique \(R \in [0, \infty]\), the \emph{radius of convergence} of the series, such that
  \[
    \sum_{n = 0}^\infty c_n(z - a)^n,\, z, a \in \C
  \]
  converges absolutely if \(|z - a| < R\) and diverges if \(|z - a| > R\). If \(0 < r < R\) then the series converges uniformly on \(\{|z - a| \leq r\}\). The radius of convergence is given by
  \[
    R = \sup\{r \geq 0: |c_n|r^n \to 0\}.
  \]
\end{theorem}

\begin{theorem}
  Let \(f(z) = \sum_{n = 0}^\infty c_n(z - a)^n\) be a complex power series with radius of convergence \(R > 0\). Then
  \begin{enumerate}
  \item \(f\) is holomorphic on \(D(a, R)\),
  \item its derivative is given by the series \(\sum_{n = 1}^\infty nc_n(z - a)^{n - 1}\), which also has radius of convergence \(R\),
  \item \(f\) has derivatives of all orders on \(D(a, R)\) and \(f^{(n)}(a) = n!c_n\),
  \item if \(f\) vanishes identically on some disc \(D(a, \varepsilon)\) then \(c_n = 0\) for all \(n\).
  \end{enumerate}
\end{theorem}

\begin{proof}
  wlog assume \(a = 0\). Claim that \(\sum_{n = 1}^\infty nc_nz^{n - 1}\) has radius of convergence \(R\).

  \begin{proof}
    \(|nc_n| \geq |c_n|\) so its radius of convergence \(R'\) is at most \(R\).

    Suppose \(0 \leq r < R\) and pick \(\rho \in (r, R)\). Then \(\sum |c_n|\rho^n\) converges and
    \[
      \frac{n|c_n|r^{n - 1}}{|c_n|\rho^n} = \frac{n}{r}\underbrace{\left(\frac{r}{\rho}\right)^n}_{< 1} \to 0
    \]
    so by comparison test \(\sum nc_nr^{n - 1}\) converges and hence \(R' = R\).
  \end{proof}

  To show the derivative is of the desired form we shall do something clever. Consider the continous function
  \[
    h_n(z, w) = \sum_{j = 0}^{n - 1} z^j w^{n - 1 - j} =
    \begin{cases}
      \frac{z^n - w^n}{z - w} & \text{if } z \neq w \\
      nw^{n - 1} & \text{if } z = w
    \end{cases}
  \]
  for \(n \geq 1\). Consider the series
  \begin{equation}
    \label{eqn:hzw}
    \sum_{n = 1}^\infty c_nh_n(z, w).
    \tag{\(\ast\)}
  \end{equation}
  Claim that for every \(r < R\), \eqref{eqn:hzw} converges uniformly on the set \(\{(z, w): |z|, |w| \leq r\}\).

  \begin{proof}
    \(|c_nh_n(z, w)| \leq |c_n|nr^{n - 1} = M_n\). Since we know \(\sum M_n < \infty\), by Weierstrass M-test \eqref{eqn:hzw} converges uniformly.
  \end{proof}

  Thus it converges to a continuous function \(g(z, w)\). By definition
  \[
    g(z, w) =
    \begin{cases}
      \sum_{n = 1}^\infty c_n \frac{z^n - w^n}{z - w} = \frac{f(z) - f(w)}{z - w} & \text{if } z \neq w \\
      \sum_{n = 1}^\infty c_nnw^{n - 1} & \text{if } z = w
    \end{cases}
  \]
  As \(g\) is continuous, fixing \(w\) and letting \(z \to w\) we get
  \[
    \lim_{z \to w} \frac{f(z) - f(w)}{z - w} = \sum_{n = 1}^\infty nc_nw^{n - 1}.
  \]
  So \(f'(w)\) exists and equals to \(g(w, w)\) as desired. This proves (1) and (2). (3) follows by induction on \(n\). Finally if \(f\) vanishes identically on a disc about \(a\) then \(f^{(n)} = 0\) for all \(n\) and by (3) \(c_n = 0\) for all \(n\).
\end{proof}

\begin{proposition}[Weierstrass M-test]
  Let \(f_n\) be a sequence of functions such that \(|f_n(x)| \leq M_n\) for all \(x \in A\). If \(\sum M_n < \infty\) then \(\sum f_n(x)\) converges uniformly on \(A\).
\end{proposition}

\begin{proof}
  Exercise.
\end{proof}



\printindex

\iffalse
past lecture notes:
T. Scholl, K. Carne

Book
Ahlfors, Complex Analysis
\fi

\end{document}
