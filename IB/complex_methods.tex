\documentclass[a4paper]{article}

\def\npart{IB}

\def\ntitle{Complex Methods}
\def\nlecturer{R.\ E.\ Hunt}

\def\nterm{Lent}
\def\nyear{2018}

\ifx \nauthor\undefined
  \def\nauthor{Qiangru Kuang}
\else
\fi

\ifx \ntitle\undefined
  \def\ntitle{Template}
\else
\fi

\ifx \nauthoremail\undefined
  \def\nauthoremail{qk206@cam.ac.uk}
\else
\fi

\ifx \ndate\undefined
  \def\ndate{\today}
\else
\fi

\title{\ntitle}
\author{\nauthor}
\date{\ndate}

%\usepackage{microtype}
\usepackage{mathtools}
\usepackage{amsthm}
\usepackage{stmaryrd}%symbols used so far: \mapsfrom
\usepackage{empheq}
\usepackage{amssymb}
\let\mathbbalt\mathbb
\let\pitchforkold\pitchfork
\usepackage{unicode-math}
\let\mathbb\mathbbalt%reset to original \mathbb
\let\pitchfork\pitchforkold

\usepackage{imakeidx}
\makeindex[intoc]

%to address the problem that Latin modern doesn't have unicode support for setminus
%https://tex.stackexchange.com/a/55205/26707
\AtBeginDocument{\renewcommand*{\setminus}{\mathbin{\backslash}}}
\AtBeginDocument{\renewcommand*{\models}{\vDash}}%for \vDash is same size as \vdash but orginal \models is larger
\AtBeginDocument{\let\Re\relax}
\AtBeginDocument{\let\Im\relax}
\AtBeginDocument{\DeclareMathOperator{\Re}{Re}}
\AtBeginDocument{\DeclareMathOperator{\Im}{Im}}
\AtBeginDocument{\let\div\relax}
\AtBeginDocument{\DeclareMathOperator{\div}{div}}

\usepackage{tikz}
\usetikzlibrary{automata,positioning}
\usepackage{pgfplots}
%some preset styles
\pgfplotsset{compat=1.15}
\pgfplotsset{centre/.append style={axis x line=middle, axis y line=middle, xlabel={$x$}, ylabel={$y$}, axis equal}}
\usepackage{tikz-cd}
\usepackage{graphicx}
\usepackage{newunicodechar}

\usepackage{fancyhdr}

\fancypagestyle{mypagestyle}{
    \fancyhf{}
    \lhead{\emph{\nouppercase{\leftmark}}}
    \rhead{}
    \cfoot{\thepage}
}
\pagestyle{mypagestyle}

\usepackage{titlesec}
\newcommand{\sectionbreak}{\clearpage} % clear page after each section
\usepackage[perpage]{footmisc}
\usepackage{blindtext}

%\reallywidehat
%https://tex.stackexchange.com/a/101136/26707
\usepackage{scalerel,stackengine}
\stackMath
\newcommand\reallywidehat[1]{%
\savestack{\tmpbox}{\stretchto{%
  \scaleto{%
    \scalerel*[\widthof{\ensuremath{#1}}]{\kern-.6pt\bigwedge\kern-.6pt}%
    {\rule[-\textheight/2]{1ex}{\textheight}}%WIDTH-LIMITED BIG WEDGE
  }{\textheight}% 
}{0.5ex}}%
\stackon[1pt]{#1}{\tmpbox}%
}

%\usepackage{braket}
\usepackage{thmtools}%restate theorem
\usepackage{hyperref}

% https://en.wikibooks.org/wiki/LaTeX/Hyperlinks
\hypersetup{
    %bookmarks=true,
    unicode=true,
    pdftitle={\ntitle},
    pdfauthor={\nauthor},
    pdfsubject={Mathematics},
    pdfcreator={\nauthor},
    pdfproducer={\nauthor},
    pdfkeywords={math maths \ntitle},
    colorlinks=true,
    linkcolor={red!50!black},
    citecolor={blue!50!black},
    urlcolor={blue!80!black}
}

\usepackage{cleveref}



% TODO: mdframed often gives bad breaks that cause empty lines. Would like to switch to tcolorbox.
% The current workaround is to set innerbottommargin=0pt.

%\usepackage[theorems]{tcolorbox}





\usepackage[framemethod=tikz]{mdframed}
\mdfdefinestyle{leftbar}{
  %nobreak=true, %dirty hack
  linewidth=1.5pt,
  linecolor=gray,
  hidealllines=true,
  leftline=true,
  leftmargin=0pt,
  innerleftmargin=5pt,
  innerrightmargin=10pt,
  innertopmargin=-5pt,
  % innerbottommargin=5pt, % original
  innerbottommargin=0pt, % temporary hack 
}
%\newmdtheoremenv[style=leftbar]{theorem}{Theorem}[section]
%\newmdtheoremenv[style=leftbar]{proposition}[theorem]{proposition}
%\newmdtheoremenv[style=leftbar]{lemma}[theorem]{Lemma}
%\newmdtheoremenv[style=leftbar]{corollary}[theorem]{corollary}

\newtheorem{theorem}{Theorem}[section]
\newtheorem{proposition}[theorem]{Proposition}
\newtheorem{lemma}[theorem]{Lemma}
\newtheorem{corollary}[theorem]{Corollary}
\newtheorem{axiom}[theorem]{Axiom}
\newtheorem*{axiom*}{Axiom}

\surroundwithmdframed[style=leftbar]{theorem}
\surroundwithmdframed[style=leftbar]{proposition}
\surroundwithmdframed[style=leftbar]{lemma}
\surroundwithmdframed[style=leftbar]{corollary}
\surroundwithmdframed[style=leftbar]{axiom}
\surroundwithmdframed[style=leftbar]{axiom*}

\theoremstyle{definition}

\newtheorem*{definition}{Definition}
\surroundwithmdframed[style=leftbar]{definition}

\newtheorem*{slogan}{Slogan}
\newtheorem*{eg}{Example}
\newtheorem*{ex}{Exercise}
\newtheorem*{remark}{Remark}
\newtheorem*{notation}{Notation}
\newtheorem*{convention}{Convention}
\newtheorem*{assumption}{Assumption}
\newtheorem*{question}{Question}
\newtheorem*{answer}{Answer}
\newtheorem*{note}{Note}
\newtheorem*{application}{Application}

%operator macros

%basic
\DeclareMathOperator{\lcm}{lcm}

%matrix
\DeclareMathOperator{\tr}{tr}
\DeclareMathOperator{\Tr}{Tr}
\DeclareMathOperator{\adj}{adj}

%algebra
\DeclareMathOperator{\Hom}{Hom}
\DeclareMathOperator{\End}{End}
\DeclareMathOperator{\id}{id}
\DeclareMathOperator{\im}{im}
\DeclareMathOperator{\coker}{coker}
\DeclarePairedDelimiter{\generation}{\langle}{\rangle}

%groups
\DeclareMathOperator{\sym}{Sym}
\DeclareMathOperator{\sgn}{sgn}
\DeclareMathOperator{\inn}{Inn}
\DeclareMathOperator{\aut}{Aut}
\DeclareMathOperator{\GL}{GL}
\DeclareMathOperator{\SL}{SL}
\DeclareMathOperator{\PGL}{PGL}
\DeclareMathOperator{\PSL}{PSL}
\DeclareMathOperator{\SU}{SU}
\DeclareMathOperator{\UU}{U}
\DeclareMathOperator{\SO}{SO}
\DeclareMathOperator{\OO}{O}
\DeclareMathOperator{\PSU}{PSU}
\DeclareMathOperator{\Sp}{Sp}


%hyperbolic
\DeclareMathOperator{\sech}{sech}

%field, galois heory
\DeclareMathOperator{\ch}{ch}
\DeclareMathOperator{\gal}{Gal}
\DeclareMathOperator{\emb}{Emb}



%ceiling and floor
%https://tex.stackexchange.com/a/118217/26707
\DeclarePairedDelimiter\ceil{\lceil}{\rceil}
\DeclarePairedDelimiter\floor{\lfloor}{\rfloor}


\DeclarePairedDelimiter{\innerproduct}{\langle}{\rangle}

%\DeclarePairedDelimiterX{\norm}[1]{\lVert}{\rVert}{#1}
\DeclarePairedDelimiter{\norm}{\lVert}{\rVert}



%Dirac notation
%TODO: rewrite for variable number of arguments
\DeclarePairedDelimiterX{\braket}[2]{\langle}{\rangle}{#1 \delimsize\vert #2}
\DeclarePairedDelimiterX{\braketthree}[3]{\langle}{\rangle}{#1 \delimsize\vert #2 \delimsize\vert #3}

\DeclarePairedDelimiter{\bra}{\langle}{\rvert}
\DeclarePairedDelimiter{\ket}{\lvert}{\rangle}




%macros

%general

%divide, not divide
\newcommand*{\divides}{\mid}
\newcommand*{\ndivides}{\nmid}
%vector, i.e. mathbf
%https://tex.stackexchange.com/a/45746/26707
\newcommand*{\V}[1]{{\ensuremath{\symbf{#1}}}}
%closure
\newcommand*{\cl}[1]{\overline{#1}}
%conjugate
\newcommand*{\conj}[1]{\overline{#1}}
%set complement
\newcommand*{\stcomp}[1]{\overline{#1}}
\newcommand*{\compose}{\circ}
\newcommand*{\nto}{\nrightarrow}
\newcommand*{\p}{\partial}
%embed
\newcommand*{\embed}{\hookrightarrow}
%surjection
\newcommand*{\surj}{\twoheadrightarrow}
%power set
\newcommand*{\powerset}{\mathcal{P}}

%matrix
\newcommand*{\matrixring}{\mathcal{M}}

%groups
\newcommand*{\normal}{\trianglelefteq}
%rings
\newcommand*{\ideal}{\trianglelefteq}

%fields
\renewcommand*{\C}{{\mathbb{C}}}
\newcommand*{\R}{{\mathbb{R}}}
\newcommand*{\Q}{{\mathbb{Q}}}
\newcommand*{\Z}{{\mathbb{Z}}}
\newcommand*{\N}{{\mathbb{N}}}
\newcommand*{\F}{{\mathbb{F}}}
%not really but I think this belongs here
\newcommand*{\A}{{\mathbb{A}}}

%asymptotic
\newcommand*{\bigO}{O}
\newcommand*{\smallo}{o}

%probability
\newcommand*{\prob}{\mathbb{P}}
\newcommand*{\E}{\mathbb{E}}

%vector calculus
\newcommand*{\gradient}{\V \nabla}
\newcommand*{\divergence}{\gradient \cdot}
\newcommand*{\curl}{\gradient \cdot}

%logic
\newcommand*{\yields}{\vdash}
\newcommand*{\nyields}{\nvdash}

%differential geometry
\renewcommand*{\H}{\mathbb{H}}
\newcommand*{\transversal}{\pitchfork}
\renewcommand{\d}{\mathrm{d}} % exterior derivative

%number theory
\newcommand*{\legendre}[2]{\genfrac{(}{)}{}{}{#1}{#2}}%Legendre symbol

%algebraic geometry
\DeclareMathOperator{\Spec}{Spec}
\DeclareMathOperator{\Proj}{Proj}

\usepackage[normalem]{ulem}%strikethrough

\begin{document}

\begin{titlepage}
  \begin{center}
    \includegraphics[width=0.6\textwidth]{logo.jpg}\par
    \vspace{1cm}
    {\scshape\huge Mathamatics Tripos \par}
    \vspace{2cm}
    {\huge Part \npart \par}
    \vspace{0.6cm}
    {\Huge \bfseries \ntitle \par}
    \vspace{1.2cm}
    {\Large\nterm, \nyear \par}
    \vspace{2cm}
    
    {\large \emph{Lectures by } \par}
    \vspace{0.2cm}
    {\Large \scshape \nlecturer}
    
    \vspace{0.5cm}
    {\large \emph{Notes by }\par}
    \vspace{0.2cm}
    {\Large \scshape \href{mailto:\nauthoremail}{\nauthor}}
 \end{center}
\end{titlepage}

\tableofcontents

\section{Analytic Functions}

\subsection{The Complex Plane and the Riemann Sphere}

Any \(z \in \C\) can be written in the form \(x + iy\) where \(x = \Re z\) and \(y = \Im z\), \(x, y \in \R\) or \(re^{i\theta}\) where the \emph{modulus} \(|z| = r = \sqrt{x^2 + y^2}\) and the argument \(\theta = \arg z\) satisfies \(y = x\tan \theta\). The argument is defined only up to multiples of \(2\pi\). The \emph{principal value of the argument} is the value of \(\theta\) in the range \((-\pi, \pi]\). Note that the formula \(\tan^{-1} \frac{y}{x}\) gives the correct value for the principal value of \(\theta\) only if \(x > 0\). If \(x \leq 0\) then it might be out by \(\pm \pi\) (consider \(1 + i\) and \(1 - i\)).

An \emph{open set} \(D\) is a subset of \(\C\) which does not include its boundary\footnote{Hint: this is an applied course.}.

A \emph{neighbourhood} of a point \(z\) is an open set containing \(z\).

A \emph{domain} is an open set that is connected (i.e.\ cannot be split into wto disjoint open subsets). A \emph{simply-connected domain} is one with no holes (i.e.\ any curve lying in the domain can be shrunk continuouly to a point without leaving the domain).

Note that a hole could be caused merely be a particular function under consideration being undefined at a single point, e.g.\ \(\frac{1}{z}\).

The \emph{extended complex plane} is \(\C^* = \C \cup \{\infty\}\). We can reach the ``point at infinity'' be going off in any direction in the plane, and all are equivalent. Conceptually we may use the \emph{Riemann sphere}, which is a sphere resting on the complex plane with its ``south pole'' \(S\) at \(z = 0\). For any point \(z \in \C\), drawing a line through the ``north pole'' \(N\) of the sphere to \(z\) and noting where this line intersects the sphere specifies an equivalent point \(P\) on the sphere. Then \(\infty\) is equivalent to the ``north pole'' itself.

To investigate properties of \(\infty\) we use the substitution \(\zeta = \frac{1}{z}\). A function \(f(z)\) is said to have a particular property at infinity if \(f(\frac{1}{z})\) has the same property at \(0\).

\subsection{Complex Differentiation}

Recall the definition of differentiation for a real function \(f(x)\):
\[
  f'(x) = \lim_{\delta x \to 0} \frac{f(x + \delta x) - f(x)}{\delta x}.
\]
It is implicit that the limit must be the same whichever direction we approach from. Consider \(|x|\) at \(x = 0\) for example: if we approach from the right (\(\delta x \to 0^+\)) then the limit is \(+1\), whereas from the left (\(\delta x \to 0^-\)) it is \(-1\). Because these limits are different we say that \(|x|\) is not differentiable at \(x = 0\).

Now extend the defintion to complex function \(f(z)\). \(f\) is \emph{differentiable} at \(z\) if
\[
  f'(z) = \lim_{\delta z \to 0} \frac{f(z + \delta z) - f(z)}{\delta z}
\]
exists (and is therefore independent of direction of approach --- but now there is an infinity of possible directions).

We say that \(f\) is \emph{analytic} at a point \(z\) if there exists a neighbourhood of \(z\) throughtout which \(f'\) exists. The term \emph{regular} and \emph{holomorphic} are also used. A function which is analytic throughtout \(\C\) is called \emph{entire}.

A \emph{singularity} of \(f\) is a point at which it is not analytic, or not even defined.

The property of analyticity is in fact a surprisingly strong one. For example, two consequenses include
\begin{enumerate}
\item if a function is analytic then it is differentiable infinitely many times (c.f.\ the existence of real functions which can be differentiated \(N\) times but no more, for any given \(N\));
\item a bounded entire function is constant (c.f.\ \(\tanh x\) for \(x \in \R\), which is bounded but not constant).
\end{enumerate}

\subsection{Cauchy-Riemann Equations}

Separate \(f\) and \(z\) into real and imaginary parts:
\[
  f(z) = u(x, y) + iv(x, y)
\]
where \(z = x + iy\) and \(u, v\) are real functions. Suppose that \(f\) is differentiable at \(z\). We may take \(\delta z\) in any direction. First take it to be real, \(\delta x = \delta x\). Then
\begin{align*}
  f'(z) &= \lim_{\delta x \to 0} \frac{f(z + \delta x) - f(z)}{\delta x} \\
        &= \lim_{\delta x \to 0} \frac{u(x + \delta x, y) + iv(x + \delta x, y) - u(x, y) - iv(x, y)}{\delta x} \\
        &= \frac{\partial u}{\partial x} + i \frac{\partial v}{\partial x}
\end{align*}
Now take \(\delta z\) to be pure imaginary, \(\delta z = i\delta y\). Then
\begin{align*}
  f'(z) &= \lim_{\delta y \to 0} \frac{f(z + i\delta y) - f(z)}{i\delta y} \\
        &= \lim_{\delta y \to 0} \frac{u(x, y + \delta y) + iv(x, y + \delta y) - u(x, y) - iv(x, y)}{i\delta y} \\
        &= -i\frac{\p u}{\p y} + \frac{\p v}{\p y}.
\end{align*}
The two values for \(f'(z)\) are the same since \(f\) is differentiable. Thus
\begin{align*}
  \frac{\p u}{\p x} &= \frac{\p v}{\p u} \\
  \frac{\p u}{\p y} &= -\frac{\p v}{\p x}
\end{align*}
This is know as the \emph{Cauchy-Riemann equations}. The converse (that a function satisfying the Cauchy-Riemann equaitons is differentiable) is also true as long as we impose additional requirements, for example that the partial derivative \(u_x, u_y, v_x, v_y\) are continuous functions of \(x\) and \(y\), in the sense described in IB Analysis II.

\begin{eg}\leavevmode
  \begin{enumerate}
  \item \(f(z) = z\) is entire. Here \(u = x, v = y\) and the Cauchy-Riemann equations are satisfied.
  \item \(f(z) = e^z = e^x(\cos y + i \sin y)\) is entire since
    \begin{align*}
      \frac{\p u}{\p x} &= e^x \cos y = \frac{\p v}{\p y} \\
      \frac{\p u}{\p y} &= -e^x \sin y = -\frac{\p v}{\p x}
    \end{align*}
    The derivative is
    \[
      f'(z) = e^x \cos y + ie^x \sin y = e^x
    \]
    as expected.
  \item \(f(z) = z^n\), where \(n\) is a positive integer, is entire. Write \(z = r(\cos \theta + i \sin \theta)\) we obtain \(u = r^n \cos n\theta, v = r^n \sin n\theta\). We can check the Cauchy-Riemann equaitions using \(r = \sqrt{x^2 + y^2}\) and \(\tan \theta = \frac{y}{x}\). The derivative is \(nz^{n - 1}\) as we would expect!
  \item Any rational function, i.e.\ \(f(z) = \frac{P(z)}{Q(z)}\) where \(P\) and \(Q\) are polynomials, is analytic except at points where \(Q(z) = 0\). For instance \(f(z) = \frac{z}{z^2 + 1}\) is analytic except at \(\pm i\).
  \item Many standard real functions can be extended naturally to complex functions and obey the usual rule for their derivatives. For example \(f(z) = \sin z = \frac{e^{iz} - e^{-iz}}{2i}\) has derivative \(f'(z) = \cos z\). We can also write
    \begin{align*}
      \sin z &= \sin (x + iy) \\
             &= \sin x \cos iy + \cos x \sin iy \\
             &= \sin x \cosh y + i \cos x \sinh y
    \end{align*}
    This applies to other trigonometric functions.

    \(\log z = \log |z| + i \arg z\) has derivative \(\frac{1}{z}\).

    The product, quotient and chain rules hold in exactly the same way as for real function.
  \end{enumerate}
\end{eg}

\begin{eg}\leavevmode
  \begin{enumerate}
  \item \(f(z) = \Re z\) has \(u = x, v = 0\) but \(\frac{\p u}{\p x} = 1 \neq 0 = \frac{\p u}{\p y}\) so \(\Re z\) is nowhere analytic.
  \item \(f(z) = |z|\) has \(u = \sqrt{x^2 + y^2}, v = 0\) and is nowhere analytic.
  \item \(f(z) = \conj z = x - iy\) is nowhere analytic.
  \item \(f(z) = |z|^2 = x^2 + y^2\). The Cauchy-Riemann equaitions are satisfied only at the origin, so \(f\) is only differentiable at \(z = 0\). However it is not analytic there because there is no neighbourhood of \(0\) throughout which \(f\) is differentiable.
  \end{enumerate}
\end{eg}

\subsection{Analytic Continuation*}

If we are given the values of an analytic function in some restricted region --- which could be rather small, such as a short curve somewhere in the complex plane --- then there is a unique extension of the function to the rest of \(\C\) that is still analytic. (No proof given here.) The extension might have some singularities, and might be multivalued.

This fact can be useful in extending the domain of definition of a function. We shall see an example in Section 5.2.

\subsection{Harmonic Functions}

If \(f(z) = u + iv\) is analytic, then
\[
  \frac{\p^2 u}{\p x^2} = \frac{\p}{\p x} \frac{\p u}{\p x} = \frac{\p}{\p x} \frac{\p v}{\p y} = -\frac{\p^2 u}{\p y^2}
\]
so \(u\) satisfies Laplace's equaiotn in two dimension, \(\gradient^2 u = 0\). Similarly so does \(v\). A function satisfying Laplace's equation in an open set is said to be \emph{harmonic} there.

Functions \(u\) and \(v\) satisfying the Cauchy-Riemann equations are called \emph{harmonic conjugates}. If we know one then we can find the other, up to a constant. For example, consider \(u(x, y) = x^2 - y^2\), which is easily verified to be harmonic. Its harmonic conjugate \(v\) satisfies
\begin{align*}
  \frac{\p v}{\p y} &= 2x \implies v = 2xy + g(x) \\
  \frac{\p v}{\p x} &= 2y \implies 2y + g'(x) = 2y
\end{align*}
so \(g = \alpha\) for some constant \(\alpha\). The corresponding analytic function whose real part is \(u\) is therefore
\[
  f(z) = x^2 - y^2 + 2ixy + i\alpha = (x + iy)^2 + i\alpha = z^2 + i\alpha.
\]

If the domain is not simply-connected then this method might give a solution that is multi-valued. For example, if \(u = \frac{1}{2} \log(x^2 + y^2)\), which is harmonic in the domain \(|z| > 0\), the corresponding \(f(z)\) is \(\log z\), which is multi-valued (see \Cref{sec:multi-valued functions}).

We end this section by a geometric obseravation:

\begin{proposition}
  Contours of harmonic conjugate functions are perpendicular to each other:
\end{proposition}

\begin{proof}
  \(\gradient u\) is perpendicular to contours of \(u\) (i.e.\ curves \(u = \) constant), using a result from IA Vector Calculus. Similarly \(\gradient v\) is perpendicular to contour of \(v\). But
  \begin{align*}
    \gradient u \cdot \gradient v &= \frac{\p u}{\p x}\frac{\p v}{\p x} + \frac{\p u}{\p y}\frac{\p v}{\p y} \\
                                  &= \frac{\p u}{\p x}\left(-\frac{\p u}{\p y}\right) + \frac{\p u}{\p y}\frac{\p u}{\p x} \\
                                  &= 0
  \end{align*}
  and the result follows.
\end{proof}

\subsection{Multi-valued functions}\label{sec:multi-valued functions}

For \(z = re^{i\theta}\) we define \(\log z = \log r + i\theta\). There are thus infinitely many values of \(\log z\), for \(\theta\) may take an infinity of values. For example,
\[
  \log i = \frac{\pi i}{2} \text{ or } \frac{5\pi i}{2} \text{ or } -\frac{3\pi i}{2} \text{ or } \dots
\]
depending on which choice of \(\theta\) we make.

Missed a lecture

Note that a branch cut alone does not specify a branch (compare (b) above, with the principal branch which is a different branch even though it has the same branch cut) nor is a single value of the fucntion sufficient by itself (compare (a) and (c) above).

\subsubsection{Riemann Surfaces*}

Riemann imagined different branches as separate copies of \(\C\), stacked on top of each other but each one joined to the next at the branch cut. This structure is a \emph{Riemann surface}.

\subsubsection{Multiple Branch Cuts}

When there is more than one branch point we may need more than one branch cut. FOr \(f(z) = (z(z - 1))^{1/3}\) there are branch points at \(0\) and \(1\), so we need two branch cuts. A possibility is shown below. Then no curve can wrap round either \(0\) or \(1\).

For any \(z\) write \(z = re^{i\theta}\) and \(z - 1 = r_1e^{i\theta_1}\) with \(\theta \in (-\pi, \pi], \theta_1 \in [0, 2\pi)\) and define
\[
  f(z) = \sqrt[3]{rr_1} e^{i(\theta + \theta_1)/3}
\]
This is continuous so long as we don't cross either cut. Sometimes we need fewer branch cuts than we might think. See the worked example.

\subsection{Möbius Maps}

The Möbius map \(z \mapsto w = \frac{az + b}{cz + d}\) where \(ad - bc \neq 0\), is analytic except at \(z = -\frac{d}{c}\). It is useful to consider it as a map \(\C^* \to \C^*\) with
\begin{align*}
  -\frac{d}{c} &\mapsto \infty \\
  \infty &\mapsto \frac{a}{c}
\end{align*}
It is then bijective, the inverse being \(w \mapsto z = \frac{-dw + b}{cw - a}\), another Möbius map.

\begin{definition}[Circline]
  A \emph{circline} is either a circle or a line.
\end{definition}

Möbius maps take circlines to circlines.

\begin{proof}
  Any circline can be expressed as a circle of Apollonius
  \[
    |z - z_1| = \lambda|z - z_2|
  \]
  where \(z_1, z_2  \in \C, \lambda \in \R_{> 0}\). This is a result from IA Vectors and Matrices. The case \(\lambda = 1\) corresponds to a line and \(\lambda \neq 1\) to a circle. Apply a Möbius map,
  \begin{align*}
    \left| \frac{-dw + b}{cw - a} - z_1 \right| &= \lambda \left| \frac{-dw + b}{cw - a} - z_2 \right| \\
    |(cz_1 + d) - (az_1 + b)| &= \lambda|(cz_1 + d)w - (az_2 + b)| \\
    |w - w_1| &= \lambda \left| \frac{cz_2 + d}{cz_1 + d} \right| |w - w_2|
  \end{align*}
  where \(w_1 = \frac{az_1 + b}{cz_1 + d}, w_2 = \frac{az_2 + b}{cz_2 + d}\), which is another circle of Apollonius.

  The proof failsif either \(cz_1 + d = 0\) or \(cz_2 + d = 0\), but in either of these cases the equation trivially defines a circle.
\end{proof}

Geometrically it is clear that choosing three distinct points in \(\C^*\) uniquely specifies a circline (if one of the points is \(\infty\) then we have specified the straight line through the other two points).

Given \(\alpha, \beta, \eta, \alpha', \beta', \gamma' \in \C^*\) we can find a Möbius map which sends \(\alpha \mapsto \alpha'\) etc.

\begin{proof}
  The map
  \[
    f_1(z) = \frac{\beta - \gamma}{\beta - \alpha}\frac{z - \alpha}{z - \gamma}
  \]
  sends \(\alpha \mapsto 0, \beta \mapsto 1, \gamma \mapsto \infty\). Let \(f_2(z)\) be defined analogously for the primed version. Then \(f_2^{-1} \compose f_1\) is the required mapping. It is also a Möbius map as they are closed under composition.
\end{proof}

Putting all these results together, we conclude that we can find a Möbius map taking any given circline to any other.

\subsection{Conformal Maps}

\begin{definition}[Conformal map]
  A \emph{conformal map} \(f: U \to V\), where \(U, V\) are \emph{open} subsets of \(\C\), is one which is analytic with non-zero derivative in \(U\).
\end{definition}

Though not part of the definition, it is usual (and helpful) to require that \(f\) be one-to-one from \(U\) to \(V\).

An alternative definition is that a conformal map is one that preserves the angle (in both magnitude and orientation) between intersecting curves. We shall show that our definition implies this. The converse is also true (proof omittted) so the two definitions are equivalent.

Suppose that \(z_1(t)\) is a curve in \(\C\) parameterised by \(t \in \R\), which passes through a point \(z_0\) when \(t = t_1\). Suppose further that its tangent there, \(z_1'(t_1)\), has a wel-defined direction. Then \(z_1'(t) \neq 0\) and the curve makes an angle \(\phi = \arg z_1'(t_1)\) to the \(x\)-axis at \(z_0\). Consider the image of the curve, \(Z_1(t) = f(z_1(t))\). Its tangent direction at \(t = t_1\) is
\[
  Z_1'(t_1) = z_1'(t_1)f'(z_1(t_1)) = z_1'(t_1)f'(z_0)
\]
and therefore makes an angle with the \(x\)-axis of
\[
  \arg Z_1'(t_1) = \phi + \arg f'(z_0).
\]
Note that \(\arg f'(z_0)\) exists since \(f\) is conformal so \(f'(z_0) = 0\). In other words, the trangent direction is rotated by \(\arg f'(z_0)\).

Now if \(z_2(t)\) is another curve passing throught \(z_0\) then its tangent direction will also be rotated by \(\arg f'(z_0)\). The result follows.

Sometimes we do not know what \(V\), the image set of \(f\) acting on  \(U\), is in advance. Often the easiest way to find it is first to find the iamge of the boundary \(\p U\), which will form the boundary \(\p V\) of \(V\); but, since this does not reveal upon which side of \(\p V\) \(V\) lies, to then find the image of a single point of our choice within \(U\), which will lie within \(V\).

\begin{eg}\leavevmode
  \begin{enumerate}
  \item The map \(z \mapsto az + b\) where \(a, b \in \C, a \neq 0\), rotates by \(\arg a\), enlarges by \(|a|\) and translates by \(n\) and is conformal everywhere.
  \item \(f(z) = z^2\) is a conformal map from
    \[
      U = \{z: 0 < |z| < 1, 0 < \arg z < \frac{\pi}{2}\}
    \]
    to
    \[
      V = \{w: 0 < |w| < 1, 0 < \arg w < \pi\}.
    \]
    Note that the right angle between the two boundary curves at \(z = 1\) is preserved because \(f\) is conformal there. Similarly at \(z = 1\). But the right angle at \(z = 0\) is not preserved because \(f\) is not conformal there (\(f'(0) = 0\)). Fortunately this does not matter since \(U\) is an open set so does not include \(0\).
  \item How would we map the lef-hand half-plane
    \[
      U = \{z: \Re z < 0\}
    \]
    to a wedge
    \[
      V = \{w: -\frac{\pi}{4} < \arg w < \frac{\pi}{4}\}?
    \]
    We need to halve the angle, so by using \(z^{1/2}\), for which we need to choose a branch. The branch cut must not lie in \(U\) (since \(z^{1/2}\) is not analytic on the branch cut) so choose a cut along the negative imaginary axis: \(re^{i\theta} \mapsto \sqrt{r} e^{i\theta/2}\) where \(\theta \in (-\pi/2, 3\pi/2]\). Having defined this branch, we now apply \(z^{1/2}\) to \(U\) to produce the wedge \(\{z': \pi/4 < \arg z' < 3\pi/4\}\); so we just need to rotate through \(-\pi/2\). The final map is \(f(z) = -iz^{1/2}\).
  \item \(e^z\) takes rectangles conformally to sectors of annuli. With an appropriate choice of branch, \(\log z\) does the reverse.
  \item Möbius maps (which are conformal everywhere except at the point that it is sent to \(\infty\) are very useful in taking circles, or parts of them, to straight lines, or vice versa. Consider \(f(z) = \frac{z - 1}{z + 1}\) acting on the unit disc \(U = \{z: |z| < 1\}\). The boundary of \(U\) is a cricle; the three points \(-1, i\) and \(1\) lie on the circle and are mapped to \(\infty, i\) and \(0\) respectively. Therefore the image of \(\p U\) is the imaginary axis; since \(f(0) = -1\) we see that the image of \(U\) is the left-hand half-plane.

    The inverse map, which is \(z \mapsto \frac{1 + z}{1 - z}\), maps \(V\) to \(U\) conformally.

    Alternatively, \(w = \frac{z - 1}{z + 1}\) if and only if \(z = \frac{1 + w}{1 - w}\). So \(|z| < 1\) if and only if \(|w + 1| < |w - 1|\), i.e.\ \(w\) is close to \(-1\) than ti is to \(1\), which describges precisely the left-hand half-plane.

    In fact this particular map can usefully be depoloyed more generally on \emph{quadrants} of the unit disc or of the complex plane.
  \item \(f(z) = \frac{1}{z}\) is a simple Möbius map useful for acting on vertical or horizontal lines, which maps to circles passing through the origin with centres one of the axes, or for mapping sectors within the unit circle to sectors outside the circle, or vice versa.
  \end{enumerate}
\end{eg}

In practice, complicated conformal maps are usually built up from individual building blocks, each a simple conformal map; the required map is the composition of these (note that the composition of conformal maps is conformal, by the chain rule). Seee the worked examples.

\sout{Hungry. Skipped a lecture.}

Hungry. Skipped two lectures.

\section{Laurent series and Cauchy's theorem}

\subsection{Taylor and Laurent series}

\subsection{Zeroes}

\begin{eg}\leavevmode
  \begin{enumerate}
  \item \(z^3 + iz^2 + z + i = (z - i)(z + i)^2\) has a simple zero at \(z = i\) and a zero of order two at \(z = -i\).
  \item \(\sinh x\) has zero where
    \[
      \frac{e^z - e^{-z}}{2} = 0
    \]
    so \(z \in \pi i\Z\). The zeros are all simple since \(\cosh n\pi i = \cos n\pi \neq = 0\).
  \item Since \(\sinh z\) has a simple zero at \(z = \pi i\), \(\sinh^3 z\) has a zero of order \(3\) there. If needed, we can find its Taylor series about \(\pi i\) by write \(\zeta = z - \pi i\).
    \begin{align*}
      \sinh^3 z
      &= \sinh^3 (\zeta + \pi i) \\
      &= (-\sinh \zeta)^3 \\
      &= -(\zeta + \frac{1}{3!}\zeta^3 + \dots) ^3 \\
      &= -\zeta^3 - \frac{1}{2}\zeta^5 - \dots \\
      &= -(z - \pi i)^3 - \frac{1}{2}(a - \pi i)^5 + \dots
    \end{align*}
  \end{enumerate}
\end{eg}

\subsection{Laurent Series}

If \(f\) has a singularity at \(z_0\) then we cannot expect it to have a Taylor series there. Instead, if \(f\) is analytic in an annulus \(R_1 < |z - z_0| < R_2\) then it has a \emph{Laurent series} about \(z_0\)
\[
  f(z) = \sum_{n = \infty}^\infty a_n(z - z_0)^n
\]
convergent within the annulus. See the proof in the separate sheet.

It can be shown that the Laurent series for \(f\) about a particular \(z_0\) is unique within any given radius. Note that Taylor series are just a special case of Laurent series, with \(a_n = 0\) for all \(n < 0\).

\begin{eg}\leavevmode
  \begin{enumerate}
  \item \(\frac{e^z}{z^3}\) has a Laurent series about \(z_0 = 0\) given by
    \[
      \frac{e^z}{z^3} = \sum_{m = 0}^\infty \frac{z^{m - 3}}{m!} = \sum_{n = -3}^\infty \frac{z^n}{(n + 3)!}
    \]
    so \(a_n = \frac{1}{(n + 3)!}\) for \(n \geq -3\).
  \item \(e^{1/z}\) about \(z_0 = 0\) has
    \[
      e^{1/z} = 1 + \frac{1}{z} + \frac{1}{2!z^2} + \frac{1}{3!z^3} + \dots
    \]
    so \(a_n = \frac{1}{(-n)!}\) for \(n \leq 0\).
  \item If \(f(z) = \frac{1}{z - a}\) where \(a \in \C\) then \(f\) is analytic in \(|z| < |a|\) so it has a Taylor series about \(z = 0\) given by
      \[
        \frac{1}{z - a} = -\frac{1}{a} \left( 1 - \frac{z}{a} \right)^{-1} = -\sum_{n = 0}^\infty a^{-n - 1}z^n.
      \]
      The binomial expansion is valid since \(\left| \frac{z}{a} \right| < 1\). In \(|z| > |a|\) it has a Laurent series (in the annulus \(|a| < |z| < \infty\)) given by
      \[
        \frac{1}{z - a} = \frac{1}{z} \left( 1 - \frac{a}{z} \right)^{-1} = \sum_{m = 0}^\infty \frac{a^m}{z^{m + 1}}.
      \]
  \end{enumerate}
\end{eg}


\end{document}
