\documentclass[a4paper]{article}

\def\ntitle{Linear Algebra}
\def\ndate{Michaelmas, 2017 -- 2018}

\ifx \nauthor\undefined
  \def\nauthor{Qiangru Kuang}
\else
\fi

\ifx \ntitle\undefined
  \def\ntitle{Template}
\else
\fi

\ifx \nauthoremail\undefined
  \def\nauthoremail{qk206@cam.ac.uk}
\else
\fi

\ifx \ndate\undefined
  \def\ndate{\today}
\else
\fi

\title{\ntitle}
\author{\nauthor}
\date{\ndate}

%\usepackage{microtype}
\usepackage{mathtools}
\usepackage{amsthm}
\usepackage{stmaryrd}%symbols used so far: \mapsfrom
\usepackage{empheq}
\usepackage{amssymb}
\let\mathbbalt\mathbb
\let\pitchforkold\pitchfork
\usepackage{unicode-math}
\let\mathbb\mathbbalt%reset to original \mathbb
\let\pitchfork\pitchforkold

\usepackage{imakeidx}
\makeindex[intoc]

%to address the problem that Latin modern doesn't have unicode support for setminus
%https://tex.stackexchange.com/a/55205/26707
\AtBeginDocument{\renewcommand*{\setminus}{\mathbin{\backslash}}}
\AtBeginDocument{\renewcommand*{\models}{\vDash}}%for \vDash is same size as \vdash but orginal \models is larger
\AtBeginDocument{\let\Re\relax}
\AtBeginDocument{\let\Im\relax}
\AtBeginDocument{\DeclareMathOperator{\Re}{Re}}
\AtBeginDocument{\DeclareMathOperator{\Im}{Im}}
\AtBeginDocument{\let\div\relax}
\AtBeginDocument{\DeclareMathOperator{\div}{div}}

\usepackage{tikz}
\usetikzlibrary{automata,positioning}
\usepackage{pgfplots}
%some preset styles
\pgfplotsset{compat=1.15}
\pgfplotsset{centre/.append style={axis x line=middle, axis y line=middle, xlabel={$x$}, ylabel={$y$}, axis equal}}
\usepackage{tikz-cd}
\usepackage{graphicx}
\usepackage{newunicodechar}

\usepackage{fancyhdr}

\fancypagestyle{mypagestyle}{
    \fancyhf{}
    \lhead{\emph{\nouppercase{\leftmark}}}
    \rhead{}
    \cfoot{\thepage}
}
\pagestyle{mypagestyle}

\usepackage{titlesec}
\newcommand{\sectionbreak}{\clearpage} % clear page after each section
\usepackage[perpage]{footmisc}
\usepackage{blindtext}

%\reallywidehat
%https://tex.stackexchange.com/a/101136/26707
\usepackage{scalerel,stackengine}
\stackMath
\newcommand\reallywidehat[1]{%
\savestack{\tmpbox}{\stretchto{%
  \scaleto{%
    \scalerel*[\widthof{\ensuremath{#1}}]{\kern-.6pt\bigwedge\kern-.6pt}%
    {\rule[-\textheight/2]{1ex}{\textheight}}%WIDTH-LIMITED BIG WEDGE
  }{\textheight}% 
}{0.5ex}}%
\stackon[1pt]{#1}{\tmpbox}%
}

%\usepackage{braket}
\usepackage{thmtools}%restate theorem
\usepackage{hyperref}

% https://en.wikibooks.org/wiki/LaTeX/Hyperlinks
\hypersetup{
    %bookmarks=true,
    unicode=true,
    pdftitle={\ntitle},
    pdfauthor={\nauthor},
    pdfsubject={Mathematics},
    pdfcreator={\nauthor},
    pdfproducer={\nauthor},
    pdfkeywords={math maths \ntitle},
    colorlinks=true,
    linkcolor={red!50!black},
    citecolor={blue!50!black},
    urlcolor={blue!80!black}
}

\usepackage{cleveref}



% TODO: mdframed often gives bad breaks that cause empty lines. Would like to switch to tcolorbox.
% The current workaround is to set innerbottommargin=0pt.

%\usepackage[theorems]{tcolorbox}





\usepackage[framemethod=tikz]{mdframed}
\mdfdefinestyle{leftbar}{
  %nobreak=true, %dirty hack
  linewidth=1.5pt,
  linecolor=gray,
  hidealllines=true,
  leftline=true,
  leftmargin=0pt,
  innerleftmargin=5pt,
  innerrightmargin=10pt,
  innertopmargin=-5pt,
  % innerbottommargin=5pt, % original
  innerbottommargin=0pt, % temporary hack 
}
%\newmdtheoremenv[style=leftbar]{theorem}{Theorem}[section]
%\newmdtheoremenv[style=leftbar]{proposition}[theorem]{proposition}
%\newmdtheoremenv[style=leftbar]{lemma}[theorem]{Lemma}
%\newmdtheoremenv[style=leftbar]{corollary}[theorem]{corollary}

\newtheorem{theorem}{Theorem}[section]
\newtheorem{proposition}[theorem]{Proposition}
\newtheorem{lemma}[theorem]{Lemma}
\newtheorem{corollary}[theorem]{Corollary}
\newtheorem{axiom}[theorem]{Axiom}
\newtheorem*{axiom*}{Axiom}

\surroundwithmdframed[style=leftbar]{theorem}
\surroundwithmdframed[style=leftbar]{proposition}
\surroundwithmdframed[style=leftbar]{lemma}
\surroundwithmdframed[style=leftbar]{corollary}
\surroundwithmdframed[style=leftbar]{axiom}
\surroundwithmdframed[style=leftbar]{axiom*}

\theoremstyle{definition}

\newtheorem*{definition}{Definition}
\surroundwithmdframed[style=leftbar]{definition}

\newtheorem*{slogan}{Slogan}
\newtheorem*{eg}{Example}
\newtheorem*{ex}{Exercise}
\newtheorem*{remark}{Remark}
\newtheorem*{notation}{Notation}
\newtheorem*{convention}{Convention}
\newtheorem*{assumption}{Assumption}
\newtheorem*{question}{Question}
\newtheorem*{answer}{Answer}
\newtheorem*{note}{Note}
\newtheorem*{application}{Application}

%operator macros

%basic
\DeclareMathOperator{\lcm}{lcm}

%matrix
\DeclareMathOperator{\tr}{tr}
\DeclareMathOperator{\Tr}{Tr}
\DeclareMathOperator{\adj}{adj}

%algebra
\DeclareMathOperator{\Hom}{Hom}
\DeclareMathOperator{\End}{End}
\DeclareMathOperator{\id}{id}
\DeclareMathOperator{\im}{im}
\DeclareMathOperator{\coker}{coker}
\DeclarePairedDelimiter{\generation}{\langle}{\rangle}

%groups
\DeclareMathOperator{\sym}{Sym}
\DeclareMathOperator{\sgn}{sgn}
\DeclareMathOperator{\inn}{Inn}
\DeclareMathOperator{\aut}{Aut}
\DeclareMathOperator{\GL}{GL}
\DeclareMathOperator{\SL}{SL}
\DeclareMathOperator{\PGL}{PGL}
\DeclareMathOperator{\PSL}{PSL}
\DeclareMathOperator{\SU}{SU}
\DeclareMathOperator{\UU}{U}
\DeclareMathOperator{\SO}{SO}
\DeclareMathOperator{\OO}{O}
\DeclareMathOperator{\PSU}{PSU}
\DeclareMathOperator{\Sp}{Sp}


%hyperbolic
\DeclareMathOperator{\sech}{sech}

%field, galois heory
\DeclareMathOperator{\ch}{ch}
\DeclareMathOperator{\gal}{Gal}
\DeclareMathOperator{\emb}{Emb}



%ceiling and floor
%https://tex.stackexchange.com/a/118217/26707
\DeclarePairedDelimiter\ceil{\lceil}{\rceil}
\DeclarePairedDelimiter\floor{\lfloor}{\rfloor}


\DeclarePairedDelimiter{\innerproduct}{\langle}{\rangle}

%\DeclarePairedDelimiterX{\norm}[1]{\lVert}{\rVert}{#1}
\DeclarePairedDelimiter{\norm}{\lVert}{\rVert}



%Dirac notation
%TODO: rewrite for variable number of arguments
\DeclarePairedDelimiterX{\braket}[2]{\langle}{\rangle}{#1 \delimsize\vert #2}
\DeclarePairedDelimiterX{\braketthree}[3]{\langle}{\rangle}{#1 \delimsize\vert #2 \delimsize\vert #3}

\DeclarePairedDelimiter{\bra}{\langle}{\rvert}
\DeclarePairedDelimiter{\ket}{\lvert}{\rangle}




%macros

%general

%divide, not divide
\newcommand*{\divides}{\mid}
\newcommand*{\ndivides}{\nmid}
%vector, i.e. mathbf
%https://tex.stackexchange.com/a/45746/26707
\newcommand*{\V}[1]{{\ensuremath{\symbf{#1}}}}
%closure
\newcommand*{\cl}[1]{\overline{#1}}
%conjugate
\newcommand*{\conj}[1]{\overline{#1}}
%set complement
\newcommand*{\stcomp}[1]{\overline{#1}}
\newcommand*{\compose}{\circ}
\newcommand*{\nto}{\nrightarrow}
\newcommand*{\p}{\partial}
%embed
\newcommand*{\embed}{\hookrightarrow}
%surjection
\newcommand*{\surj}{\twoheadrightarrow}
%power set
\newcommand*{\powerset}{\mathcal{P}}

%matrix
\newcommand*{\matrixring}{\mathcal{M}}

%groups
\newcommand*{\normal}{\trianglelefteq}
%rings
\newcommand*{\ideal}{\trianglelefteq}

%fields
\renewcommand*{\C}{{\mathbb{C}}}
\newcommand*{\R}{{\mathbb{R}}}
\newcommand*{\Q}{{\mathbb{Q}}}
\newcommand*{\Z}{{\mathbb{Z}}}
\newcommand*{\N}{{\mathbb{N}}}
\newcommand*{\F}{{\mathbb{F}}}
%not really but I think this belongs here
\newcommand*{\A}{{\mathbb{A}}}

%asymptotic
\newcommand*{\bigO}{O}
\newcommand*{\smallo}{o}

%probability
\newcommand*{\prob}{\mathbb{P}}
\newcommand*{\E}{\mathbb{E}}

%vector calculus
\newcommand*{\gradient}{\V \nabla}
\newcommand*{\divergence}{\gradient \cdot}
\newcommand*{\curl}{\gradient \cdot}

%logic
\newcommand*{\yields}{\vdash}
\newcommand*{\nyields}{\nvdash}

%differential geometry
\renewcommand*{\H}{\mathbb{H}}
\newcommand*{\transversal}{\pitchfork}
\renewcommand{\d}{\mathrm{d}} % exterior derivative

%number theory
\newcommand*{\legendre}[2]{\genfrac{(}{)}{}{}{#1}{#2}}%Legendre symbol

%algebraic geometry
\DeclareMathOperator{\Spec}{Spec}
\DeclareMathOperator{\Proj}{Proj}

\theoremstyle{definition}
\newtheorem*{caution}{Caution}

\begin{document}
\maketitle

\tableofcontents

\section{Basics}

\begin{convention}
  Throughout this course, $\mathbb{F}$ denotes a general field. If you wish, think of it as $\mathbb{R}$ or $\mathbb{C}$.
\end{convention}

\subsection{Definition}

\begin{defi}
  An $\mathbb{F}$-vector space (or a vector space over $\mathbb{F}$) is an abelian group $(V, +)$ equipped with a function, called \emph{scalar multiplication}: $\mathbb{F}\times V \to V, (\lambda, v) \mapsto \lambda\cdot v$, satisfying the axioms
  \begin{itemize}
  \item distributive over vectors: $\lambda(v_1+v_2) = \lambda(v_1+v_2)$,
  \item distributive over scalars: $(\lambda_1+\lambda_2)v= \lambda_1 v+\lambda_2 v$,
  \item $\lambda(\mu v) = \lambda \mu v$,
  \item $1\cdot v = v$.
  \end{itemize}
\end{defi}

The additive unit of $V$ is denoted by $\V 0$.

\begin{eg}\leavevmode
  \label{eg:matrix as V}
  \begin{enumerate}
  \item $\forall n \in \mathbb{N}, \mathbb{F}^n$ is the space of column vectors of length $n$ with entries in $\mathbb{F}$. It is an vector space by entry-wise addition and entry-wise scalar multiplication.
  \item $M_{m,n}(\mathbb{F})$, the set of $m\times n$ matrices with entries in $\mathbb{F}$, with the operation defined as entry-wise addition.
    \item For any set $X$, $\mathbb{R}^X = \{f: X \to \mathbb{R}\}$, the set of $\mathbb{R}$-valued functions on $X$, with addition and scalar multiplication defined pointwise. For instance, $(f_1+f_2)(x) = f_1(x)+f_2(x)$.
  \end{enumerate}
\end{eg}

\begin{ex}\leavevmode
  \begin{enumerate}
  \item Check the above examples satisfy the axioms.
    \item $0\cdot v = \V 0$ and $(-1)\cdot v = -v$ for all $v \in V$.
  \end{enumerate}
\end{ex}

\subsection{Vector Subspace}

\begin{defi}
  Let $V$ be an $\mathbb{F}$-vector space. A subset $U \subseteq V$ is a \emph{subspace}, denoted $U \leq V$, if
  \begin{itemize}
  \item $\V 0 \in U$,
  \item $U$ is closed under addition: $\forall u_1, u_2 \in U, u_1+u_2 \in U$,
    \item $U$ is closed under scalar multiplication: $\forall u \in U, \forall \lambda \in \mathbb{F}, \lambda u \in U$.
  \end{itemize}
\end{defi}

\begin{ex}
  If $U$ is a subspace of $V$, then $U$ is also an $\mathbb{F}$-vector space.
\end{ex}

\begin{eg}\leavevmode
  \begin{enumerate}
  \item $V = \mathbb{R}^{\mathbb{R}}$, the set all functions from $\mathbb{R}$ to itself, has a (proper) subspace $C(\mathbb{R})$, the space of continuous functions on $\mathbb{R}$ as continuous functions are closed under addition and scalar multiplication. $C(\mathbb{R})$ in turn has a proper subspace $P(\mathbb{R})$, the set of all polynomials in $\mathbb{R}$.
    \item $\{(x_1,x_2,x_3) \in \mathbb{R}^3: x_1+x_2+x_3 = t\}$ where $t$ is some fixed constant is a subspace of $\mathbb{R}^3$ if and only if $t = 0$.
  \end{enumerate}
\end{eg}

\begin{prop}
  Let $V$ be an $\mathbb{F}$-vector space, $U, W \leq V$. Then $U \cap W \leq V$.
\end{prop}

\begin{proof}\leavevmode
  \begin{itemize}
  \item $\V 0 \in U, \V 0 \in V$ so $\V 0 \in U \cap W$.
    \item Suppose $u, w \in U \cap W$. Fix $\lambda, \mu \in \mathbb{F}$. As $U \leq V$, $\lambda u + \mu w \in U$. As $W \leq V$, $\lambda u +\mu w \in W$ so $\lambda u + \mu w \in U \cap W$. Take $\lambda = \mu = 1$ for vector addition and $\mu = 0$ for scalar multiplication.
  \end{itemize}
\end{proof}

\begin{eg}
  $V = \mathbb{R}^3, U = \{(x,y,z): x=0\}, W=\{(x,y,z):y=0\}$, then $U\cap W=\{(x,y,z):x=y=0\}$.
\end{eg}

\begin{note}
The union of a family of subspaces is \emph{almost never} a subspace. For example, $V = \mathbb{R}^2$, $U, V$ be $x$- and $y$-axis.
\end{note}

\begin{defi}
  Let $V$ be an $\mathbb{F}$-vector space, $U, W \leq V$, the \emph{sum} of $U$ and $W$ is the set
  \[
    U + W = \{u+w: u\in U, w\in W\}
  \]
\end{defi}

\begin{eg}
  Use the definition from the previous example, $U+W=V$.
\end{eg}

\begin{prop}
  $U+W \leq V$.
\end{prop}

\begin{proof}\leavevmode
  \begin{itemize}
  \item $\V 0 = \V 0 + \V 0 \in U+W$,
  \item $u_1,u_1\in U, w_1,w_2\in W$, $(u_1+w_2) + (u_2+w_2) = (u_1+u_2)+(w_1+w_2) \in U+W$,
    \item similar for scalar multiplication. Left as an exercise.
  \end{itemize}
\end{proof}

\begin{note}
  $U+W$ is the smallest subspace containing both $U$ and $W$. This is because all elements of the form $u+w$ are in such a space by closure under addition.
\end{note}

\begin{defi}
  Let $V$ be an $\mathbb{F}$-vector space, $U \leq V$. The \emph{quotient space} $V/U$ is the abelian gropup $V/U$ equipped with scalar multiplication
  \begin{align*}
    \mathbb{F} \times V/U &\to V/U \\
    (\lambda, v+U) &\mapsto \lambda v+U
  \end{align*}
\end{defi}

\begin{prop}
  This is well-defined and $V/U$ is an $\mathbb{F}$-vector space.
\end{prop}

\begin{proof}
  First check it is well-defined. Suppose $v_1+U= v_2+U \in V/U$. Then $v_1-v_2\in U$. Now use closure under scalar multiplication and distributivity, $\lambda v_1 - \lambda v_2 = \lambda(v_1-v_2)\in U$ so $\lambda v_1 + U = \lambda v_2 +U\in V/U$.
  Now check vector space axioms of $V/U$, which will follow from the axioms for $V$:
  \begin{itemize}
  \item $\lambda(\mu(v+U)) = \lambda(\mu v+U) = \lambda(\mu v)+U = (\lambda\mu) v+U = \lambda\mu(v+U)$,
  \item other axioms are left as an exercise.
  \end{itemize}
\end{proof}

\subsection{Span, Linear Independence \& Basis}

\begin{defi}
  Let $V$ be a $\F$-vector space, $S \subseteq V$ be a subset. The \emph{span} of $S$
  \[
    \langle S \rangle = \Big\{\sum_{s\in S} \lambda_s s : \lambda_s \in F \Big\}
  \]
  is the set of all the finite linear combinations of elements (i.e. all but finitely many of the $\lambda$ are zero) of $S$.
\end{defi}

\begin{rmk}
  $\langle S \rangle$ is the smallest subspace of $V$ containing all elements of $S$.
\end{rmk}

\begin{convention}
  $\langle \emptyset \rangle = \{\V 0\}$
\end{convention}

\begin{eg}\leavevmode
  \begin{enumerate}
 \item $V=\R^3$, $S = \{(1,0,0),(0,1,2),(3,-2,-4)\}$, $\langle S \rangle = \{(a,b,2b): a,b\in \R \}$
 \item For any set $X$, $\R^X$ is a vector space. For $x \in X$, define $\delta_x: X \to \R, \delta_x(x) = 1, \delta_x(y) = 0 \: \forall y \neq x$, then
   \[
     \langle \delta_x: x\in X \rangle = \{f\in \R^X: f \text{ has finite support} \}
   \]
  \end{enumerate} 
\end{eg}

\begin{defi}
  $S$ spans $V$ if $\langle S \rangle = V$.
\end{defi}

\begin{defi}
  $V$ is \emph{finite-dimensional} over $\F$ if it is spanned by a finite set.
\end{defi}

\begin{defi}
  The vectors $v_1,\ldots, v_n$ are \emph{linearly independent} over $\F$ if
  \[
    \sum_{i=1}^n \lambda_i = 0 \Rightarrow \lambda_i = 0 \: \forall i
  \]
  A subset $S \subset V$ is \emph{linearly independent} if every finite subset of $S$ is linearly independent.

  A subset if \emph{linearly dependent} if it is not linearly independent.
\end{defi}

\begin{eg}
  In the first example above, the three vectors are not linearly independent.
\end{eg}

\begin{ex}
  The set $\{\delta_x: x \in X\}$ is linearly independent.
\end{ex}

\begin{defi}
  $S$ is a \emph{basis} of $V$ if it is linearly independent and spans $V$.
\end{defi}

\begin{eg}\leavevmode
  \begin{enumerate}
  \item $\F^n$ has standard basis $\{e_1,e_2,\ldots,e_n\}$ where $e_i$ is the column vector with $1$ in the $i$th entry and $0$ elsewhere.
  \item $V=\C$ over $\C$ has natural basis $\{1\}$, but over $\R$ it has natural basis $\{1, i\}$.
  \item $V=P(\R)$, the space of real polynomials, has natural basis
    \[
      \{1, x, x^2, \dots \}.
    \]
    It is an exercise to check this carefully.
    \end{enumerate}
\end{eg}

\begin{lem}
  Let $V$ be a $\F$-vector space. The vectors $v_1,\ldots,v_n$ form a basis of $V$ if and only if each vector $v\in V$ has a unique expression
  \[
    v = \sum_{i=1}^n \lambda_i v_i, \lambda_i \in \F.
  \]
  
\end{lem}

\begin{proof}\leavevmode
  $\Rightarrow:$ Fix $v\in V$. The $v_i$ span $V$, so exists $\lambda_i \in \F$ such that $v = \sum \lambda_i v_i$. Suppose also $v = \sum \mu_i v_i$ for some $\mu_i \in \F$. Then the difference
  \[
    \sum (\mu_i - \lambda_i) v_i = \V 0.
  \]
  Since the $v_i$ are linearly independent, $\mu_i-\lambda_i = 0$ for all $i$.

  $\Leftarrow:$ The $v_i$ span $V$ by assumption. Suppose $\sum_{i=1}^n \lambda_i v_i = \V 0$. Note that $\V 0 = \sum_{i=0}^n 0 \cdot v_i$. By appying uniqueness to $\V 0$, $\lambda_i = 0$ for all $i$. 
\end{proof}

\begin{lem}
  If $v_1,\ldots, v_n$ spans $V$ over $\F$, then some subset of $v_1,\ldots,v_n$ is a basis of $V$ over $\F$.
\end{lem}

\begin{proof}
  If $v_1,\ldots, v_n$ is linearly independent then done. Otherwise for some $\ell$, there exist $\alpha_1, \ldots, \alpha_{\ell-1} \in \F$ such that
  \[
    v_\ell = \sum_{i=1}^{\ell-1} \alpha_i v_i.
  \]
  (If $\sum \lambda_i v_i = 0$, not all $\lambda_i$ is zero. Take $\ell$ maximal with $\lambda_\ell \neq 0$, then $\alpha_i = -\frac{\lambda_i}{\lambda_\ell}$.)

  Now $v_1,\ldots,v_{\ell-1},v_{\ell+1},\ldots,v_n$ still span $V$. Continue iteratively until we have linear independence.
\end{proof}

\begin{thm}[Steinitz Exchange Lemma]
  Let $V$ be a finite-dimensional vector space over $\F$. Take $v_1,\ldots,v_m$ to be linearly independent, $w_1,\ldots,w_n$ to span $V$. Then
  \begin{itemize}
  \item $m \leq n$, and
    \item reordering the $w_i$ if needed, $v_1,\ldots, v_m, w_{m+1},\ldots,w_n$ spans $V$.
  \end{itemize}
\end{thm}

\begin{proof}
  Proceed by induction. Suppose that we have replaced $\ell \geq 0$ of the $w_i$. Reordering $w_i$ if needed, $v_1,\ldots,v_\ell,w_{\ell+1},\ldots,w_n$ spans $V$.
  \begin{itemize}
  \item If $\ell = m$, done.
  \item If $\ell < m$, then $v_{\ell+1} = \sum_{i=1}^\ell \alpha_i v_i + \sum_{i> \ell} \beta_i w_i$. As the $v_i$ are linearly independent, $\beta_i \neq 0$ for some $i$. After reordering, $\beta_{\ell+1} \neq 0$,
    \[
      w_{\ell+1} = \frac{1}{\beta_{\ell+1}} (v_{\ell+1}-\sum_{i\leq \ell} \alpha_i v_i - \sum_{i>\ell+1} \beta_i w_i).
    \]
    Thus $v_1,\ldots, v_\ell, v_{\ell+1},w_{\ell+2},\ldots, w_n$ also spans $V$. After $m$ steps, we will replace $m$ of the $w_i$ by $v_i$. Thus $m \leq n$.
  \end{itemize}
\end{proof}

\subsection{Dimension}

\begin{thm}
  If $V$ is a finite-dimensional vector space over $\F$, then any two bases for $V$ have the same cardinality, which is called the \emph{dimension} of $V$, donoted $\dim_\F V$.
\end{thm}

\begin{proof}
  If $v_1,\ldots, v_n$ and $w_1,\ldots,w_m$ are both bases, then $\{v_i\}$ is linearly independent and $\{w_i\}$ spans $V$ so $n \leq m$. Similarly $m \leq n$.
\end{proof}

\begin{eg}
  $\dim_\C \C = 1$, but $\dim_\R \C = 2$.
\end{eg}

\begin{lem}
  Let \(V\) be a finite-dimensional vector space over \(\F\). If \(w_1,\ldots,w_\ell\) is a linearly independent set of vectors, we can extend it to a basis \(w_1,\ldots,w_\ell,w_{\ell+1},\ldots,v_n\).
\end{lem}

\begin{proof}
  Apply Steinitz exchange lemma to \(w_1,\ldots, w_\ell\) and any basis \(v_1,\ldots, v_n\).

  Or more direcly, if \(V=\langle w_1,\ldots, w_\ell \rangle\), done. Otherwise take \(v_{\ell+1} \in V\setminus\langle w_1,\ldots, w_\ell\rangle\). Now \(w_1,\ldots, w_\ell,w_{\ell+1}\) is linearly independent. Iterate.
\end{proof}

\begin{cor}
  Let \(V\) be a finite-dimensional vector space of dimension \(n\). Then
  \begin{enumerate}
  \item Any linearly independent set of vectors has at most \(n\) elements, with equality if and only if the set is a basis.
  \item Any spanning set of vectors has at least \(n\) elements, with equaility if and only if the set is a basis.
  \end{enumerate}
\end{cor}

\begin{slogan}
  Choose the best basis for the job.
\end{slogan}

\begin{thm}
  Let \(U, W\) be subspaces of \(V\). If \(V\) and \(W\) are finite-dimensional, so is \(U+W\) and
  \[
\dim(U+W) = \dim U + \dim W - \dim(U\cap W).
  \]
\end{thm}

\begin{proof}
  Pick basis \(v_1,\ldots, v_\ell\) of \(U\cap W\). Extend it to basis \(v_1,\ldots,v_\ell,u_1,\ldots,u_m\) of \(U\) and \(v_1,\ldots,v_\ell,w_1,\ldots,w_n\) of \(W\). Claim \(v_1,\ldots, v_\ell,u_1,\ldots,u_m,w_1,\ldots,w_n\) is a basis for \(U+W\):
  \begin{itemize}
  \item spanning: if \(u\in U\), then \(u= \sum \alpha_iv_i + \sum \beta_iu_i\) and if \(w\in W\), \(w = \sum_{}^{}\gamma_iv_i + \sum_{}^{}\delta_iw_i\), so \(u+w = \sum_{}^{}(\alpha_i + \gamma_i)v_i + \sum_{}^{}\beta_iu_i \sum_{}^{}\delta_iu_i\).
  \item linear independence: assume \(\sum \alpha_iv_i + \sum \beta_iu_i+ \sum \gamma_iw_i=0\). Rearrange, \(\sum\alpha_iv_i + \sum\beta_iu_i = -\sum\gamma_iw_i \in U\cap W\) so it equals to \(\sum\delta_iv_i\) for some \(\delta_i\in \F\) because \(v_i\) is a basis for \(U\cap W\). As \(v_i\) and \(w_i\) are linearly independent, \(\gamma_i=\delta_i=0\) for all \(i\). Thus \(\sum\alpha_iv_i + \sum\beta_iv_i=0\), so \(\alpha_i=\beta_i=0\) since \(v_i\) and \(u_i\) form a basis for \(U\).
  \end{itemize}
\end{proof}

\begin{thm}
  Let \(V\) be a finite-dimensional vector space over \(\F\) and \(U \leq V\), then \(U\) and \(V/U\) are also finite dimensional and
  \[
\dim V = \dim U + \dim V/U.
  \]
\end{thm}

\begin{proof}
  Left as an exercise. Outline: first show \(U\) is finite-dimensional, then let \(u_1,\ldots,u_\ell\) be a basis for \(U\). Extend it to a basis for \(V\), say \(u_1,\ldots,u_\ell,w_{\ell+1},\ldots,w_n\) of \(V\). Check \(w_{\ell+1}+U,\ldots,w_n+U\) form a basis for \(V/U\).
\end{proof}

\begin{cor}
  If \(U\) is a proper subspace of \(V\), which is finite-dimensional, then \(\dim U < \dim V\).
\end{cor}

\begin{proof}
  \(V/U \neq 0\) so \(\dim V/U > 0\).
\end{proof}

\subsection{Direct Sum}

\begin{defi}
  Let \(V\) be a vector space over \(\F\), \(U, W\leq V\). Then
  \[
V = U \oplus W
  \]
  if every element of \(V\) can be written as \(v=u+w, u\in U, w\in W\) uniquely. This is called the \emph{internal direct sum}. \(W\) is a \emph{direct complement} of \(U\) in \(V\).
\end{defi}

\begin{lem}
  Suppose \(U,W\leq V\), TFAE:
  \begin{enumerate}
  \item \(V = U \oplus W\),
  \item \(V=U+W\) and \(U\cap W = 0\),
  \item Given \(B_1\) any basis of \(U\), \(B_w\) any basis of \(V\), \(B=B_1\cup B_2\) is a basis of \(V\).
  \end{enumerate}
\end{lem}

\begin{proof}
  \(2 \Rightarrow 1\): any \(v\in V\) is \(u+w\) for some \(u\in U, w\in W\). Suppose \(u_1+w_1=u_2+w_2\), then \(u_1-u_2 = w_2-w_1 \in U\cap W = 0\). Thus \(u_1=u_2,w_1=w_2\).

  \(2 \Rightarrow 1\): \(B\) spans as any \(v\in V\) is \(u+w\). Write \(u\) in terms of \(B_1\) and \(w\) in terms of \(B_2\). Then \(u+w\) is a linear combination of elements of \(B\). To show \(B\) is linearly independent, suppose \(\sum_{v\in B} \lambda_v v = \V 0 = \V 0_V + \V 0_W\). Write LHS as \(\sum_{v\in B_1} \lambda_vv + \sum_{v\in B_2}\lambda_vv\). By uniqueness of expression, \(\sum_{v\in B_1}\lambda_vv=\V 0_V\) and \(\sum_{w\in B_2}\lambda_ww=\V 0_w\). As \(B_1, B_2\) are bases, all of the \(\lambda_v, \lambda_w\) are zero.

  \(2 \Rightarrow 1\): if \(v\in V, v=\sum_{x\in V}\lambda_xx = \sum_{u\in B_1}\lambda_uu + \sum_{w\in B_1}\lambda_ww\) so \(v\in U+W\). Conversely, if \(v\in U\cap W, v = \sum_{u\in B_1}\lambda_uu=\sum_{w\in B_2}\lambda_ww\) so all \(\lambda_u, \lambda_v\) are zero since \(B_1\cup B_2\) is linearly independent.
\end{proof}

\begin{lem}
  Let \(V\) be a finite-dimensional vector space over \(\F\) and \(U\leq V\). Then there exists a direct complement to \(U\) in \(V\).
\end{lem}

\begin{proof}
  Let \(u_1,\ldots, u_\ell\) be a basis for \(U\). Extend this to a basis \(u_1,\ldots, u_\ell,w_{\ell+1},\ldots,w_n\) for \(V\). Then \(\langle w_{\ell+1},\ldots,w_n\rangle\) is a direct complement of \(U\).
\end{proof}

\begin{caution}
  Direct complements are \emph{not} unique.
\end{caution}

\begin{defi}
  Suppose \(V_1,\ldots, V_\ell \leq V\), then
  \[
\sum_i V_i = V_1+\cdots+V_\ell = \{v_1+\cdots+v_\ell: v_i\in V_i\}.
  \]
  The direct sum is direct if
  \[
    v_1+\cdots+v_\ell = v_1'+\cdots+ v_\ell' \Rightarrow v_i = v_i' \text{ for all } i.
  \]
  It is denoted
  \[
V = \bigoplus_{i=1}^\ell V_i.
  \]
\end{defi}

\begin{ex}
  \(V_1,\ldots, V_\ell \leq V\), TFAE:
  \begin{enumerate}
  \item The sum \(\sum_i V_i\) is direct,
  \item \(V_i \cap \sum_{j\neq i}V_j = 0\) for all \(i\),
  \item For any basis \(B_i\) of \(V_i\), the union \(B=\bigcup_{i=1}^\ell B_i\) is a basis for \(\sum_i V_i\).
  \end{enumerate}
\end{ex}

\begin{defi}
  Let \(U, W\) be vector spaces over \(\F\). The \emph{external direct sum} is
  \[
U\oplus W = \{(u,w): u\in U, w\in W\}
  \]
  with pointwise addition and scalar multiplication.
\end{defi}

\section{Linear Maps}

\subsection{Definition}

\begin{defi}
  \(V, W\) two \(\F\)-vector space, a map \(\alpha: V\to W\) is \emph{linear} if
  \begin{itemize}
  \item \(\alpha(v_1 + v_2) = \alpha(v_1) + \alpha(v_2)\),
  \item \(\alpha(\lambda v) = \lambda \alpha(v)\).
  \end{itemize}
  This is equivalent to
  \[
\alpha(\lambda_1v_1+ \lambda_2v_2) = \lambda_1\alpha(v_1) + \lambda_2\alpha(v_2).
  \]
\end{defi}

\begin{eg}\leavevmode
  \begin{enumerate}
  \item Given an \(n\times m\) matrix \(A\) with coefficients in \(\F\), the map \(\alpha: \F^m\to \F^n, v\to Av\).
  \item Differentiation \(D: P(\R) \to P(\R), f\mapsto \frac{df}{dx}\).
  \item Integration \(I: C[0,1] \to C[0,1], f\to I(f)\) where \(I(f)(x) = \int_0^x f(t)dt\).
  \item Fix \(x\in [0,1]\), the map \(C[0,1]\to \R, f\mapsto f(x)\).
  \end{enumerate}
\end{eg}

\begin{note}
  Suppose \(U, V, W\) are \(\F\)-vector spaces, then
  \begin{enumerate}
  \item \(\id: V\to V\) is linear.
  \item \(U \stackrel{\alpha}{\to} V \stackrel{\beta}{\to} W\), if \(\alpha, \beta\) are linear then so is \(\beta \compose \alpha\).
  \end{enumerate}
\end{note}

\begin{lem}[Free functor \(\mathbf{Set} \to \mathbf{Vect}_K\)]
  Suppose \(V, W\) are \(\F\)-vector spaces and \(B\) is a basis for \(V\). If \(\alpha_0: B\to W\) is \emph{any} map, then there is a \emph{unique} linear map \(\alpha: V\to W\) extending \(\alpha_o\).
\end{lem}

\begin{proof}
  Let \(v\in V\). Write \(v = \sum \lambda_iv_i\) in a unique way. By linearity \(\alpha(v) = \alpha(\sum \lambda_iv_i) = \sum \lambda_i \alpha(v_i) = \sum \lambda_i \alpha_0(v_i)\). Uniqueness follows.
\end{proof}

\begin{note}\leavevmode
  \begin{itemize}
  \item This is true for infinite-dimensional vector space as well.
  \item Very often, to define a linear map, define it on a basis and extend it linearly to the vector space.
  \item Two linear maps \(\alpha_1,\alpha_2: V\to W\) are equal if they agree on a basis.
  \end{itemize}
\end{note}

\subsection{Isomorphism of Vector Spaces}

\begin{defi}
  Given \(V, W\) two \(\F\)-vector spaces, the map \(\alpha:V\to W\) is an \emph{isomorphism} if it is linear and bijective, denoted \(V \cong W\).
\end{defi}

\begin{lem}
  \(\cong\) is an equivalence relation on the class of all \(\F\)-vector spaces.
\end{lem}

\begin{proof}\leavevmode
  \begin{itemize}
  \item symmetric: obvious.
  \item reflexive: blah blah in lecture. Left as an exercise to reader.
  \item transitive: obvious.
  \end{itemize}
\end{proof}

\begin{thm}
  If \(V\) is an \(\F\)-vector space, then \(V \cong \F^n\) for some \(n\).
\end{thm}

\begin{proof}
  Choose a basis \(B\) for \(V\), say \(v_1,\ldots, v_n\). Define a map
  \begin{align*}
    V &\to \F^n \\
    \sum_{}^{ }\lambda_iv_i &\mapsto (\lambda_1,\ldots,\lambda_n)
  \end{align*}
which is an isomorphism.
\end{proof}

\begin{rmk}
  Choosing an isomorphism \(V \cong \F^n\) is equivalent to choosing a basis for \(V\). i.e. there is a bijection \(\{\alpha\in\Hom(V,W), \alpha\text{ bijective}\} \leftrightarrow \{\text{bases of } V\} \).
\end{rmk}

\begin{thm}
  Given two finite-dimensional \(\F\)-vector spaces \(V, W\), they are isomorphic if and only if they have the same dimension.
\end{thm}

\begin{proof}\leavevmode
  \begin{itemize}
  \item \(\Leftarrow\): \(V \cong \F^{\dim V} = \F^{\dim W} \cong W\).
  \item \(\Rightarrow\): let \(a:V\to W\) be an isomorphism and \(B\) be a basis for \(V\). Claim \(\alpha(B)\) is a basis for \(W\): \(\alpha(B)\) spans \(W\) due to surjectivity and \(\alpha(B)\) is linearly independent due to injectivity.
  \end{itemize}
\end{proof}

\begin{defi}
  Given \(\alpha: V\to W\),
  \begin{itemize}
  \item \(N(\alpha) = \ker \alpha = \{v\in V: \alpha(v) = 0\} \leq V\),
  \item \(\im \alpha = \{w\in W: \exists v\in V, \alpha(v) = w \} \leq W\).
  \end{itemize}
\end{defi}

\begin{prop}\leavevmode
  \begin{itemize}
  \item \(\alpha\) is injective if and only if \(N(\alpha) = 0\),
  \item \(\alpha\) is surjective if and only if \(\im \alpha = W\).
  \end{itemize}
\end{prop}

\begin{proof}
  Easy.
\end{proof}

\begin{eg}
  Let \(\alpha: C^\infty(\R) \to C^\infty(\R), \alpha(f)(t) = f''(t)+2f'(t)+5f(t)\). \(\ker \alpha = \{f:f''+2f'+5f=0\}\) and \(g\in \im \alpha\) if and only if there exists an \(f\) such that \(f''+2f'+5f=g\).
\end{eg}

\begin{thm}[First Isomorphism Theorem]
  Let \(\alpha: V\to W\) be a linear map. It induces an isomprhism
  \begin{align*}
    \bar \alpha: V/\ker \alpha &\to \im \alpha \\
    v + \ker \alpha &\mapsto \alpha(v)
  \end{align*}
\end{thm}

\begin{proof}
  Check the following:
  \begin{itemize}
  \item \(\bar \alpha\) is well-defined,
  \item \(\bar \alpha\) is linear: immediate from linearity of \(\alpha\),
  \item \(\bar \alpha\) is surjective.
  \end{itemize}
\end{proof}

\begin{defi}\leavevmode
  \begin{itemize}
  \item \(r(\alpha) = rk(\alpha) = \dim( \im \alpha)\) is the \emph{rank} of \(\alpha\),
  \item \(n(\alpha) = \dim N(\alpha)\) is the \emph{nullity} of \(\alpha\).
  \end{itemize}
\end{defi}

\begin{thm}[Rank-nullity Theorem]
  Let \(U, V\) be \(\F\)-vector spaces, \(\dim U < \infty\). Let \(\alpha:U\to V\) be a linear map. Then
  \[
\dim U = r(\alpha) + n(\alpha).
  \]
\end{thm}

\begin{proof}
  \(U/\ker \alpha \cong \im \alpha\) so \(\dim U - \dim (\ker \alpha) = \dim \im \alpha\). Rearrange.
\end{proof}

\begin{lem}
  Let \(V, W\) be \(\F\)-vector spaces with equal, finite dimension. Let \(\alpha:V\to W\) be linear, then TFAE:
  \begin{enumerate}
  \item \(\alpha\) is injective,
  \item \(\alpha\) is surjective,
  \item \(\alpha\) is an isomorphism.
  \end{enumerate}
\end{lem}

\begin{proof}
  Rank-nullity theorem.
\end{proof}

\subsection{\(\Hom_\F(V,W)\) as \(\F\)-vector space}

Suppose \(V\) and \(W\) are \(\F\)-vector spaces. Let \(L(V,W) = \{\alpha:V\to W, \alpha \text{ linear}\}\).

\begin{prop}
  \(L(V,W)\) is an \(\F\)-vector space, under operations
  \begin{align*}
    (\alpha_1+\alpha_2)(v) &= \alpha_1(v) + \alpha_2(v) \\
    (\lambda\alpha)(v) &= \lambda(\alpha(v))
  \end{align*}
\end{prop}

\begin{proof}
  \(\alpha_1+\alpha_2, \lambda\alpha\) as above are well-defined linear maps. The vector space axioms are satisfied.
\end{proof}

\begin{prop}
  \label{prop:dimension of linear map space}
  If both \(V\) and \(W\) are finite-dimensional over \(\F\) then so is \(L(V,W)\) and \(L(V,W) = \dim V \cdot \dim W\).
\end{prop}

\begin{proof}
  See Lemma~\ref{cor:dim of hom}.
\end{proof}

\subsubsection{Matrices, an Interlude}

\begin{defi}
  An \emph{\(m\times n\) matrix} over \(\F\) is an array with \(m\) rows and \(n\) columns with entries in \(\F\). We write
  \[
A = (a_{ij}), a_{ij}\in\F, 1\leq i \leq m, 1\leq j \leq n.
  \]
\end{defi}

\begin{defi}
  \(M_{m,n}(\F)\) is the set of all such \(m\times n\) matrices.
\end{defi}

\begin{prop}
  \(M_{m,n}(\F)\) is an \(\F\)-vector space and \(\dim M_{m,n}(\F) = m\cdot n\).
\end{prop}

\begin{proof}
  See Example~\ref{eg:matrix as V} for the proof of vector space axioms. For the dimension claim, a standard basis for \(M_{m,n}(F)\) is
  \[
    E_{ij}=
    \begin{pmatrix}
      0 & \dots & 0 \\
      \vdots & \ddots & \vdots \\
      0 & 1 & 0 \\
      \vdots & \ddots & \vdots \\
      0 & \dots & 0 
    \end{pmatrix}
  \]
  with \(1\) in the \((i,j)\)th entry so \(a_{ij} = \sum_{i,j}^{} a_{ij}E_{ij}\), from which span and linear independence follow. The basis has cardinality \(m\cdot n\).
\end{proof}

\subsubsection{Representation of Linear Maps by Matrices}

Let \(V\) and \(W\) be finite-dimensional \(\F\)-vector space, \(\alpha: V\to W\) linear. Let \(B = \{v_1,\ldots,v_n\}\) be a basis for \(V\), \(C = \{w_1,\ldots,w_m\}\) be a basis for \(W\). If \(v=\sum_{i}\lambda_iv_i \in V\), write
\[
[v]_B =
\begin{pmatrix}
  \lambda_1 \\
  \vdots \\
  \lambda_n
\end{pmatrix}
\in \F^n
\]
which is called the \emph{coordinate vector of \(v\) with respect to \(B\)}. Similarly \([w]_C\in \F^m\).

\begin{defi}
  \([\alpha]_{B,C}\) is the matrix representation of \(\alpha\) with respect to \(B\) and \(C\) with
  \begin{align*}
    [\alpha]_{B,C} &= \Big( [\alpha(v_1)]_C \: \Big| \: [\alpha(v_2)]_C \: \Big | \: \cdots \: \Big| \: [\alpha(v_n)]_C \big) \\
    &= (a_{ij})
  \end{align*}
  The matrix says
  \[
\alpha(v_j) = \sum_{i}^{ }a_{ij}w_i.
  \]
\end{defi}

\begin{lem}
  For any \(v\in V\),
  \[
[\alpha(v)]_C = [\alpha]_{B,C}\cdot [v]_B
  \]
  where \(\cdot\) is matrix multiplication applied to vectors.
\end{lem}

\begin{proof}
  Fix \(v =\sum_{j=1}^{n}\lambda_jv_j \in V\), so
  \[
[v]_B =
\begin{pmatrix}
  \lambda_1 \\
  \vdots \\
  \lambda_n
\end{pmatrix}
\]
\begin{align*}
  \alpha(v) &= \alpha\Big( \sum_{j}^{ }\lambda_jv_j \Big) \\
            &= \sum_{j}^{ }\lambda_j\alpha(v_j) \\
            &= \sum_{j}^{ }\lambda_j\Big( \sum_{i}^{ }\alpha_{i,j}w_i \Big) \\
            &= \sum_{i}^{ }\Big( \sum_{j}^{} \alpha_{i,j}\lambda_j \Big) w_i
\end{align*}
so the \(i\)th entry of \([\alpha]_{B,C}\) is \([v]_B\).
\end{proof}

\begin{lem}
  Suppose \(U \stackrel{\beta}{\to} V \stackrel{\alpha}{\to} W\) with \(\alpha, \beta\) linear, with \(\alpha\compose\beta: U\to W\). Let \(A, B,C\) be bases for \(U,V,W\) respectively. Then
  \[
[\alpha\compose\beta]_{A,C} = [\alpha]_{B,C}\cdot[\beta]_{A,B}.
  \]
\end{lem}

\begin{proof}
  \begin{align*}
    (\alpha\compose\beta)(u_\ell) &= \alpha(\beta(u_\ell)), \: u_\ell\in A\\
                                  &= \alpha\Big( \sum_{j}^{ }b_{jl}v_j \Big), \: v_j\in B \\
                                  &= \sum_{j}^{ }b_{jl}\alpha(v_j) \\
                                  &= \sum_{j}^{ }b_{jl}\sum_{i}^{ }a_{ij}w_i, \: w_i\in W \\
                                  &= \sum_{i}^{ }\Big( \sum_{j}^{ }a_{ij}b_{jl})w_i
  \end{align*}
\end{proof}

\begin{prop}
  Let \(V\) and \(W\) be \(\F\)-vector spaces with \(\dim V = n, \dim W = m\), then
  \[
L(V,W) \cong M_{m,n}(\F).
  \]
\end{prop}

\begin{proof}
  Fix bases \(B=\{v_1\ldots,v_n\}, C=\{w_1,\ldots,w_m\}\) for \(V\) and \(W\) respectively. Claim
  \begin{align*}
    \theta: L(V,W) &\to M_{m,n}(\F) \\
    \alpha &\mapsto [\alpha]_{B,C}
  \end{align*}
  is an isomorphism:
  \begin{itemize}
  \item linearity: \([\lambda_1\alpha_1+\alpha_2\alpha_2]_{B,C} = \alpha_1[\alpha_1]_{B,C} + \lambda_2[\alpha_2]_{B,C}\).
  \item surjectivity: given \(A = (a_{ij})\), let \(\alpha:v_j\mapsto \sum_{i=1}^{m}a_{ij}w_i \) and extend linearly. It follows that \(\alpha\in L(V,W)\) and \(\theta(\alpha) = A\).
  \item injectivity: \([\alpha]_{B,C} = \V 0\) implies that \(\alpha\) is the zero map.
  \end{itemize}
\end{proof}

\begin{cor}
  \label{cor:dim of hom}
  \[
\dim L(V,W) = \dim V \cdot \dim W.
  \]
\end{cor}

\begin{eg}
  Suppose \(\alpha:V\to W\), \(Y\leq V, Z\leq W\) with \(\alpha(Y)\leq Z\). Let \(B'=\{v_1,\ldots,v_k\}\) be a basis of \(Y\) and extend to \(B=\{v_1,\ldots,v_k,v_{k+1},v_n\}\) a basis for \(V\). Similarly \(C'\) and \(C\) for \(Z\) and \(W\).
  \begin{itemize}
  \item
    \[
[\alpha]_{B,C} =
\begin{pmatrix}
  A & B \\
  0 & C
\end{pmatrix}
    \]
    because for \(1\leq j \leq k\), \(\alpha(v_j)\) is a linear combination of \(w_i\) where \(1\leq i \leq l\).
  \item \([\alpha|_Y]_{B',C'} = A. \)
  \item \(\alpha\) induces a map
    \begin{align*}
      \bar\alpha: V/Y &\to W/Z \\
      v+ Y &\mapsto \alpha(v) + Z
    \end{align*}
    This is well-defined. Linearity follows from that of \(\alpha\). A basis for \(V/Y\) \(B''=\{v_{k+1}+Y,\ldots,v_n+Y\}\) and similary for \(W/Z\). It is an exercise to show \([\bar\alpha]_{B'',C''} = C \).
  \end{itemize}
\end{eg}
\end{document}