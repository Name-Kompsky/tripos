\documentclass[a4paper]{article}

\def\npart{IB}

\def\ntitle{Groups, Rings and Modules}
\def\nlecturer{O.\ Randal-Williams}

\def\nterm{Lent}
\def\nyear{2017}

\ifx \nauthor\undefined
  \def\nauthor{Qiangru Kuang}
\else
\fi

\ifx \ntitle\undefined
  \def\ntitle{Template}
\else
\fi

\ifx \nauthoremail\undefined
  \def\nauthoremail{qk206@cam.ac.uk}
\else
\fi

\ifx \ndate\undefined
  \def\ndate{\today}
\else
\fi

\title{\ntitle}
\author{\nauthor}
\date{\ndate}

%\usepackage{microtype}
\usepackage{mathtools}
\usepackage{amsthm}
\usepackage{stmaryrd}%symbols used so far: \mapsfrom
\usepackage{empheq}
\usepackage{amssymb}
\let\mathbbalt\mathbb
\let\pitchforkold\pitchfork
\usepackage{unicode-math}
\let\mathbb\mathbbalt%reset to original \mathbb
\let\pitchfork\pitchforkold

\usepackage{imakeidx}
\makeindex[intoc]

%to address the problem that Latin modern doesn't have unicode support for setminus
%https://tex.stackexchange.com/a/55205/26707
\AtBeginDocument{\renewcommand*{\setminus}{\mathbin{\backslash}}}
\AtBeginDocument{\renewcommand*{\models}{\vDash}}%for \vDash is same size as \vdash but orginal \models is larger
\AtBeginDocument{\let\Re\relax}
\AtBeginDocument{\let\Im\relax}
\AtBeginDocument{\DeclareMathOperator{\Re}{Re}}
\AtBeginDocument{\DeclareMathOperator{\Im}{Im}}
\AtBeginDocument{\let\div\relax}
\AtBeginDocument{\DeclareMathOperator{\div}{div}}

\usepackage{tikz}
\usetikzlibrary{automata,positioning}
\usepackage{pgfplots}
%some preset styles
\pgfplotsset{compat=1.15}
\pgfplotsset{centre/.append style={axis x line=middle, axis y line=middle, xlabel={$x$}, ylabel={$y$}, axis equal}}
\usepackage{tikz-cd}
\usepackage{graphicx}
\usepackage{newunicodechar}

\usepackage{fancyhdr}

\fancypagestyle{mypagestyle}{
    \fancyhf{}
    \lhead{\emph{\nouppercase{\leftmark}}}
    \rhead{}
    \cfoot{\thepage}
}
\pagestyle{mypagestyle}

\usepackage{titlesec}
\newcommand{\sectionbreak}{\clearpage} % clear page after each section
\usepackage[perpage]{footmisc}
\usepackage{blindtext}

%\reallywidehat
%https://tex.stackexchange.com/a/101136/26707
\usepackage{scalerel,stackengine}
\stackMath
\newcommand\reallywidehat[1]{%
\savestack{\tmpbox}{\stretchto{%
  \scaleto{%
    \scalerel*[\widthof{\ensuremath{#1}}]{\kern-.6pt\bigwedge\kern-.6pt}%
    {\rule[-\textheight/2]{1ex}{\textheight}}%WIDTH-LIMITED BIG WEDGE
  }{\textheight}% 
}{0.5ex}}%
\stackon[1pt]{#1}{\tmpbox}%
}

%\usepackage{braket}
\usepackage{thmtools}%restate theorem
\usepackage{hyperref}

% https://en.wikibooks.org/wiki/LaTeX/Hyperlinks
\hypersetup{
    %bookmarks=true,
    unicode=true,
    pdftitle={\ntitle},
    pdfauthor={\nauthor},
    pdfsubject={Mathematics},
    pdfcreator={\nauthor},
    pdfproducer={\nauthor},
    pdfkeywords={math maths \ntitle},
    colorlinks=true,
    linkcolor={red!50!black},
    citecolor={blue!50!black},
    urlcolor={blue!80!black}
}

\usepackage{cleveref}



% TODO: mdframed often gives bad breaks that cause empty lines. Would like to switch to tcolorbox.
% The current workaround is to set innerbottommargin=0pt.

%\usepackage[theorems]{tcolorbox}





\usepackage[framemethod=tikz]{mdframed}
\mdfdefinestyle{leftbar}{
  %nobreak=true, %dirty hack
  linewidth=1.5pt,
  linecolor=gray,
  hidealllines=true,
  leftline=true,
  leftmargin=0pt,
  innerleftmargin=5pt,
  innerrightmargin=10pt,
  innertopmargin=-5pt,
  % innerbottommargin=5pt, % original
  innerbottommargin=0pt, % temporary hack 
}
%\newmdtheoremenv[style=leftbar]{theorem}{Theorem}[section]
%\newmdtheoremenv[style=leftbar]{proposition}[theorem]{proposition}
%\newmdtheoremenv[style=leftbar]{lemma}[theorem]{Lemma}
%\newmdtheoremenv[style=leftbar]{corollary}[theorem]{corollary}

\newtheorem{theorem}{Theorem}[section]
\newtheorem{proposition}[theorem]{Proposition}
\newtheorem{lemma}[theorem]{Lemma}
\newtheorem{corollary}[theorem]{Corollary}
\newtheorem{axiom}[theorem]{Axiom}
\newtheorem*{axiom*}{Axiom}

\surroundwithmdframed[style=leftbar]{theorem}
\surroundwithmdframed[style=leftbar]{proposition}
\surroundwithmdframed[style=leftbar]{lemma}
\surroundwithmdframed[style=leftbar]{corollary}
\surroundwithmdframed[style=leftbar]{axiom}
\surroundwithmdframed[style=leftbar]{axiom*}

\theoremstyle{definition}

\newtheorem*{definition}{Definition}
\surroundwithmdframed[style=leftbar]{definition}

\newtheorem*{slogan}{Slogan}
\newtheorem*{eg}{Example}
\newtheorem*{ex}{Exercise}
\newtheorem*{remark}{Remark}
\newtheorem*{notation}{Notation}
\newtheorem*{convention}{Convention}
\newtheorem*{assumption}{Assumption}
\newtheorem*{question}{Question}
\newtheorem*{answer}{Answer}
\newtheorem*{note}{Note}
\newtheorem*{application}{Application}

%operator macros

%basic
\DeclareMathOperator{\lcm}{lcm}

%matrix
\DeclareMathOperator{\tr}{tr}
\DeclareMathOperator{\Tr}{Tr}
\DeclareMathOperator{\adj}{adj}

%algebra
\DeclareMathOperator{\Hom}{Hom}
\DeclareMathOperator{\End}{End}
\DeclareMathOperator{\id}{id}
\DeclareMathOperator{\im}{im}
\DeclareMathOperator{\coker}{coker}
\DeclarePairedDelimiter{\generation}{\langle}{\rangle}

%groups
\DeclareMathOperator{\sym}{Sym}
\DeclareMathOperator{\sgn}{sgn}
\DeclareMathOperator{\inn}{Inn}
\DeclareMathOperator{\aut}{Aut}
\DeclareMathOperator{\GL}{GL}
\DeclareMathOperator{\SL}{SL}
\DeclareMathOperator{\PGL}{PGL}
\DeclareMathOperator{\PSL}{PSL}
\DeclareMathOperator{\SU}{SU}
\DeclareMathOperator{\UU}{U}
\DeclareMathOperator{\SO}{SO}
\DeclareMathOperator{\OO}{O}
\DeclareMathOperator{\PSU}{PSU}
\DeclareMathOperator{\Sp}{Sp}


%hyperbolic
\DeclareMathOperator{\sech}{sech}

%field, galois heory
\DeclareMathOperator{\ch}{ch}
\DeclareMathOperator{\gal}{Gal}
\DeclareMathOperator{\emb}{Emb}



%ceiling and floor
%https://tex.stackexchange.com/a/118217/26707
\DeclarePairedDelimiter\ceil{\lceil}{\rceil}
\DeclarePairedDelimiter\floor{\lfloor}{\rfloor}


\DeclarePairedDelimiter{\innerproduct}{\langle}{\rangle}

%\DeclarePairedDelimiterX{\norm}[1]{\lVert}{\rVert}{#1}
\DeclarePairedDelimiter{\norm}{\lVert}{\rVert}



%Dirac notation
%TODO: rewrite for variable number of arguments
\DeclarePairedDelimiterX{\braket}[2]{\langle}{\rangle}{#1 \delimsize\vert #2}
\DeclarePairedDelimiterX{\braketthree}[3]{\langle}{\rangle}{#1 \delimsize\vert #2 \delimsize\vert #3}

\DeclarePairedDelimiter{\bra}{\langle}{\rvert}
\DeclarePairedDelimiter{\ket}{\lvert}{\rangle}




%macros

%general

%divide, not divide
\newcommand*{\divides}{\mid}
\newcommand*{\ndivides}{\nmid}
%vector, i.e. mathbf
%https://tex.stackexchange.com/a/45746/26707
\newcommand*{\V}[1]{{\ensuremath{\symbf{#1}}}}
%closure
\newcommand*{\cl}[1]{\overline{#1}}
%conjugate
\newcommand*{\conj}[1]{\overline{#1}}
%set complement
\newcommand*{\stcomp}[1]{\overline{#1}}
\newcommand*{\compose}{\circ}
\newcommand*{\nto}{\nrightarrow}
\newcommand*{\p}{\partial}
%embed
\newcommand*{\embed}{\hookrightarrow}
%surjection
\newcommand*{\surj}{\twoheadrightarrow}
%power set
\newcommand*{\powerset}{\mathcal{P}}

%matrix
\newcommand*{\matrixring}{\mathcal{M}}

%groups
\newcommand*{\normal}{\trianglelefteq}
%rings
\newcommand*{\ideal}{\trianglelefteq}

%fields
\renewcommand*{\C}{{\mathbb{C}}}
\newcommand*{\R}{{\mathbb{R}}}
\newcommand*{\Q}{{\mathbb{Q}}}
\newcommand*{\Z}{{\mathbb{Z}}}
\newcommand*{\N}{{\mathbb{N}}}
\newcommand*{\F}{{\mathbb{F}}}
%not really but I think this belongs here
\newcommand*{\A}{{\mathbb{A}}}

%asymptotic
\newcommand*{\bigO}{O}
\newcommand*{\smallo}{o}

%probability
\newcommand*{\prob}{\mathbb{P}}
\newcommand*{\E}{\mathbb{E}}

%vector calculus
\newcommand*{\gradient}{\V \nabla}
\newcommand*{\divergence}{\gradient \cdot}
\newcommand*{\curl}{\gradient \cdot}

%logic
\newcommand*{\yields}{\vdash}
\newcommand*{\nyields}{\nvdash}

%differential geometry
\renewcommand*{\H}{\mathbb{H}}
\newcommand*{\transversal}{\pitchfork}
\renewcommand{\d}{\mathrm{d}} % exterior derivative

%number theory
\newcommand*{\legendre}[2]{\genfrac{(}{)}{}{}{#1}{#2}}%Legendre symbol

%algebraic geometry
\DeclareMathOperator{\Spec}{Spec}
\DeclareMathOperator{\Proj}{Proj}

\DeclareMathOperator{\Ann}{Ann}

\makeindex

\begin{document}

\begin{titlepage}
  \begin{center}
    \includegraphics[width=0.6\textwidth]{logo.jpg}\par
    \vspace{1cm}
    {\scshape\huge Mathamatics Tripos \par}
    \vspace{2cm}
    {\huge Part \npart \par}
    \vspace{0.6cm}
    {\Huge \bfseries \ntitle \par}
    \vspace{1.2cm}
    {\Large\nterm, \nyear \par}
    \vspace{2cm}
    
    {\large \emph{Lectures by } \par}
    \vspace{0.2cm}
    {\Large \scshape \nlecturer}
    
    \vspace{0.5cm}
    {\large \emph{Notes by }\par}
    \vspace{0.2cm}
    {\Large \scshape \href{mailto:\nauthoremail}{\nauthor}}
 \end{center}
\end{titlepage}

\tableofcontents

\section{Groups}

\blindtext

\section{Rings}

\blindtext

\section{Modules}

\subsection{Definitions}

\begin{definition}[Module]\index{module}
  Let \(R\) be a commutative ring. A quadruple \((M, +, , 0_M, \cdot)\) is a \emph{\(R\)-module} if \((M, +, 0_M)\) is an abelian group and the operation \(- \cdot -: R \times M \to M\) satisfies
  \begin{align*}
    (r_1 + r_2) \cdot m &= r_1 \cdot m + r_2 \cdot m \\
    r \cdot (m_1 + m_2) &= r \cdot m_1 + r \cdot m_2 \\
    r_2 \cdot (r_1 \cdot m) &= (r_2r_2) \cdot m \\
    1_R \cdot m &= m
  \end{align*}
\end{definition}

\begin{eg}\leavevmode
  \begin{enumerate}
  \item If \(R = \F\) is a field then an \(\R\)-module is precisely an \(\F\)-vector space.
  \item For any ring \(R\), \(R^n = \underbrace{R \times \dots \times R}_{n \text{ times}}\) is an \(\R\)-module via
    \[
      r \cdot (r_1, \dots, r_n) = (rr_1, \dots, rr_n).
    \]
    In particular for \(n = 1\), \(R\) is an \(R\)-module.
  \item If \(I \ideal R\) then \(I\) is an \(R\)-module via
    \[
      r \cdot a = ra \in I.
    \]
    Also \(R/I\) is an \(\R\)-module via
    \[
      r \cdot (r_1 + I) = rr_1 + I \in R/I.
    \]
  \item For \(R = \Z\), an \(\R\)-module is precisely an abelian group. This is because the axiom for \(\cdot\) says that
    \begin{align}
      - \cdot -: \Z \times M &\to M \\
      (n, m) &\mapsto
               \begin{cases}
                 \underbrace{m + \dots + m}_{n \text{ times}} & n \geq 0 \\
                 -(\underbrace{m + \dots + m}_{n \text{ times}}) & n < 0 \\
               \end{cases}
    \end{align}
    so \(\cdot\) is uniquely determined by \(M\).
  \item Let \(\F\) be a field and \(V\) be a vector space over \(\F\). Let \(\alpha: V \to V\) be a linear map. Then we can make \(V\) into an \(\F[X]\)-module via
    \begin{align*}
      F[X] \times V &\to V \\
      (f, v) &\mapsto f(\alpha)(v)
    \end{align*}
    Different \(\alpha\)'s make \(V\) into different \(\F[x]\)-modules.
  \item Restriction of scalars: if \(\varphi: R \to S\) is a ring homomorphism and \(M\) is an \(S\)-module, then \(M\) becomes an \(R\)-modules via
    \[
      r \cdot_R m = \varphi(r) \cdot_s m.
    \]
  \end{enumerate}
\end{eg}

\begin{definition}[Submodule]\index{module!submodule}
  If \(M\) is an \(\R\)-module, \(N \subseteq M\) is a \emph{submodule} if \(N\) is a subgroup of \((M, +, 0_M)\) and for any \(n \in N, r \in R\), \(r \cdot n \in N\). Write \(N \leq M\).
\end{definition}

\begin{eg}
  A subset of \(R\) is a submodule if and only if it is an ideal.
\end{eg}

\begin{definition}[Quotient module]\index{module!quotient}
  If \(N \leq M\) is a submodule, the \emph{quotient module} \(M/N\) is the set of \(N\)-cosets in \((M, +, 0_M)\), i.e. the quotient abelian group with
  \[
    r \cdot (m + N) = r \cdot m + N.
  \]
\end{definition}

\begin{definition}[Homomorphism]\index{module!homomorphism}
  A function \(f: M \to N\) is an \emph{\(\R\)-module homomorphism} if it is a homomorphism of abelian groups and \(f(r \cdot m) = r \cdot f(m)\).
\end{definition}

\begin{eg}
  If \(R = \F\) is a field and \(V\) and \(W\) are \(\F\)-modules (i.e. \(\F\)-vector spaces), then a map is an \(\F\)-module homomorphism if and only if it is an \(\F\)-linear map.
\end{eg}

\begin{theorem}[First Isomorphism Theorem]\index{tbd}
  If \(f: M \to N\) is an \(R\)-module homomorphism then
  \begin{align*}
    \ker f &= \{m \in M: f(m) = 0\} \leq M \\
    \im f &= \{n \in N: n = f(m)\} \leq N
  \end{align*}
  and
  \[
    M/\ker f \cong \im f.
  \]
\end{theorem}

\begin{theorem}[Second Isomorphism Theorem]\index{tbd}
  Let \(A, B \leq M\) be submodules. Then
  \begin{align*}
    A + B &= \{m \in M: m = a + b, a \in A, b \in B\} \leq M \\
    A \cap B &\leq M
  \end{align*}
  and
  \[
    (A + B)/A \cong B/(A \cap B).
  \]
\end{theorem}

\begin{theorem}[Third Isomorphism Theorem]\index{tbd}
  Let \(N \leq L \leq M\) be a chain of submodules. Then
  \[
    \frac{M/N}{L/N} \cong M/L.
  \]
\end{theorem}

\begin{definition}[Annihilator]\index{annihilator}
  If \(M\) is an \(R\)-module and \(m \in M\), the \emph{annihilator} of \(m\) is
  \[
    \Ann(m) = \{r \in R: r \cdot m = 0_M\} \ideal R.
  \]

  The \emph{annihilator} of \(M\) is
  \[
    \Ann(M) = \bigcap_{m \in M} \Ann(m) \ideal R.
  \]
\end{definition}

\begin{definition}[Generated submodule]\index{module!submodule!generated}
  If \(M\) is an \(R\)-module and \(m \in M\), the \emph{submodule generated by \(m\)} is
  \[
    Rm = \{r \cdot m \in M: r \in R\}.
  \]
\end{definition}

\begin{note}
  Intuitively, the annihilator of an element is the stabiliser of a ring action and that of a moduel is the kernel. We also have
  \[
    Rm \cong R/\Ann(m).
  \]
\end{note}

\begin{definition}[Finitely generated]\index{finitely generated}
  \(M\) is \emph{finitely generated} if there are \(m_1, \dots, m_n \in M\) such that
  \[
    M = Rm_1 + \dots Rm_n = \{r_1m_1 + \dots + r_nm_r: r_i \in R\}.
  \]
\end{definition}

\begin{lemma}
  An \(R\)-module \(M\) is finitely generated if and only if there is a surjetion \(\varphi: R^n \surj M\) for some \(n\).
\end{lemma}

\begin{proof}\leavevmode
  \begin{itemize}
  \item \(\Rightarrow\): Suppose \(M = Rm_1 + \dots + Rm_n\). Define
    \begin{align*}
      \varphi: R^n &\to M \\
      (r_1, \dots, r_n) &\mapsto r_1m_1 + \dots + r_nm_m
    \end{align*}
    This is an \(R\)-module homomorphism and is surjective.
  \item \(\Leftarrow\): Let \(m_i = \varphi((0, \dots, 0, 1, 0, \dots, 0))\) with \(1\) in the \(i\)th position. Then
    \begin{align*}
      \varphi((r_1, \dots, r_n)) &= \varphi((r_1, 0, \dots, 0) + \dots + (0, \dots, 0, r_n)) \\
                                 &= \varphi((r_1, 0, \dots, 0)) + \dots + \varphi((0, \dots, 0, r_n)) \\
                                 &= r_1 \varphi((1, 0, \dots, 0)) + \dots + r_n \varphi((0, \dots, 0, 1)) \\
      &= r_1m_1 + \dots + r_nm_n
    \end{align*}
    As \(\varphi\) is surjective, \(M = Rm_1 + Rm_n\).
  \end{itemize}
\end{proof}

\begin{corollary}
  Let \(M\) be an \(R\)-module and \(N \leq M\). If \(M\) is finitely generated the so is \(M/M\).
\end{corollary}

\begin{proof}
  \[
    R^n \overset{f}{\surj} M \overset{\pi}{\surj} M/N.
  \]
\end{proof}

\begin{note}
  A module of a finitely generated \(R\)-module need \emph{not} to be finitely generated. For example,
  \[
    (X_1, X_1, \dots) \ideal \Z[X_1, X_1, \dots] = R
  \]
  is an \(R\)-module but note finitely generated, as otherwise it would be a finitely generated ideal.
\end{note}

\begin{eg}
  For \(\alpha \in \C\), \(\alpha\) is an algebraic integer if and only if \(\Z[\alpha]\) is a finitely generated \(\Z\)-module.
\end{eg}

\subsection{Direct Sums and Free Modules}

\begin{definition}[Direct sum]\index{direct sum}
  If \(M_1, \dots, M_k\) are \(R\)-modules, the \emph{direct sum} \(M_1 \oplus \dots \oplus M_k\) is the set \(M_1 \times \dots \times M_k\) with addition
  \[
    (m_1, \dots, m_k) + (m_1', \dots, m_k') = (m_1 + m_1', \dots, m_k + m_k')
  \]
  and \(R\)-module structure
  \[
    r \cdot (m_1, \dots, m_k) = (rm_1, \dots, rm_k).
  \]
\end{definition}

\begin{eg}
  \[
    R^n = \underbrace{R \oplus \dots \oplus R}_{n \text{ times}}.
  \]
\end{eg}

\begin{definition}[Independence]\index{independenc}
  Let \(m_1, \dots m_k \in M\). They are \emph{independent} if
  \[
    \sum_i r_i \cdot m_i 0
  \]
  implies that \(r_i = 0\) for all \(1 \leq i \leq k\).
\end{definition}

\begin{definition}[Free generation]\index{tbd}
  A subset \(S \subseteq M\) \emph{generates \(M\) freely} if
    \begin{enumerate}
    \item \(S\) generates \(M\).
    \item Any function \(\psi: S \to N\) to an \(R\)-module \(N\) extends to an \(R\)-module homomorphism \(\theta: M \to N\).
    \end{enumerate}
    \[
      \begin{tikzcd}
        S \ar[r, hook] \ar[dr, "\psi"] & \generation S \ar[d, dashed, "\theta"] \\
        & N
      \end{tikzcd}
    \]
\end{definition}

\begin{note}
  We can show this extension is unique: given \(\theta_1, \theta_2: M \to N\) two extensions of \(\psi\), \(\theta_1 - \theta_2:M \to N\) is an \(R\)-module homomorphism so \(\ker(\theta_1 - \theta_2) \leq M\). But \(\theta_1, \theta_2\) both extend \(\psi\) so \(S \subseteq \ker(\theta_1 - \theta_2)\). As \(S\) generates \(M\), \(M \leq \ker(\theta_1 - \theta_2)\) so \(\theta_1 = \theta_2\).
\end{note}

An \(R\)-module which is freely generated by \(S \subseteq M\) is said to be \emph{free} and \(S\) is called a \emph{basis}

\begin{proposition}
  For a finite subset \(S = \{m_1, \dots, m_k\} \subseteq M\), TFAE:
  \begin{enumerate}
  \item \(M\) is freely generated by \(S\).
  \item \(M\) is generated by \(S\) and \(S\) is independent.
  \item Every \(m \in M\) can be written as \(r_1m_1 + \dots + r_km_k\) for some unique \(r_i \in R\).
  \end{enumerate}
\end{proposition}

\begin{proof}\leavevmode
  \begin{itemize}
  \item \(1 \Rightarrow 2\): Let \(S\) generate \(M\) freely. If \(S\) is not independent, then there is a non-trivial relation
    \[
      \sum_{i = 1}^k r_im_i = 0
    \]
    with \(r_j \neq 0\). Let
    \begin{align*}
      \psi: S &\to R \\
      m_i &\mapsto
            \begin{cases}
              0_R & i \neq j \\
              1_R & i = j
            \end{cases}
    \end{align*}
    This extends to an \(R\)-module homomorphism \(\theta: M \to R\). Then
    \[
      0 = \theta(0) = \theta \left(\sum r_im_i \right) = \sum r_i\theta(m_i) = r_j.
    \]
    Absurd. Thus \(S\) is independent.
  \item The other steps follow similarly from those in IB Linear Algebra.
  \end{itemize}
\end{proof}

\begin{eg}
  Unlike vector spaces, a minimal generating set need not to be independent. For example \(\{2, 3\} \subseteq \Z\) generates \(\Z\) but is not linear independent as \((-3) \cdot 2 + (2) \cdot 3 = 0\).
\end{eg}

However, like vector spaces, in case a module is freely generated, it is isomorphic to direct sums of copies of the ring:

\begin{lemma}
  If \(S = \{m_1, \dots, m_k\} \subset M\) freely generates \(M\) then
  \[
    M \cong R^k
  \]
  as an \(R\)-module.
\end{lemma}

\begin{proof}
  This is entirely analogous to vector spaces. Let
  \begin{align*}
    f: R^k &\to M \\
    (r_1, \dots, r_k) &\mapsto \sum_i r_im_i
  \end{align*}
  It is surjective as \(S\) generates \(M\) and injective as \(m_i\)'s are independent.
\end{proof}

If an \(R\)-module is generated by \(m_1, \dots, m_k\), we have seen before that there is a surjection \(f: R^k \surj M\). We define

\begin{definition}[Relation module]\index{relation module}
  The \emph{relation module} for the generators is
  \[
    \ker f \leq \R^k.
  \]
\end{definition}

As \(M \cong R^k/\ker f\), knowing \(M\) is equivalent to knowing the relation module.

\begin{definition}[Finitely presented]\index{finitely presented}
  \(M\) is \emph{finitely presented} if there is a finitely generating set \(m_1, \dots, m_k\) for which the associated relation module is finitely generated.
\end{definition}

Let \(\{n_1, \dots, n_r\} \subseteq \ker f \leq R^k\) be a set of generators. Then
\[
  n_i =
  \begin{pmatrix}
    r_{1i} \\
    r_{ri} \\
    \vdots \\
    r_{ki}
  \end{pmatrix}
\]
and \(M\) is generated by \(m_1, \dots, m_k\) subject to relations
\[
  \sum_{j = 1}^k r_{ij} m_j = 0
\]
for \(1 \leq i\leq r\).

\begin{proposition}[Invariance of Dimension]
  If \(R^n \cong R^m\) then \(n = m\).
\end{proposition}

\begin{note}
  This does not hold in general for modules over non-commutative rings.
\end{note}

\begin{proof}
  As a general strategy, let \(I \ideal R\). Then
  \[
    IM = \left\{\sum a_im_i: a_i \in I, m_i \in M \right\} \leq M
  \]
  is a submodule as
  \[
    r \cdot \sum a_im_i = \sum (ra_i)m_i \in IM.
  \]
  Thus we have a quotient \(R\)-module \(M/IM\). We can make this into an \((R/I)\)-module via
  \[
    (r + I) \cdot (m + IM) = rm + IM.
  \]

  Let \(I \ideal R\) be a maximal proper ideal (this requires Zorn's Lemma). Then \(R/I\) is a field and therefore \(R^n \cong R^m\) implies
  \begin{align*}
    R^n/IR^n &\cong R^m/IR^m \\
    (R/I)^n &\cong (R/I)^m
  \end{align*}
  This is a vector space isomorphism so \(n = m\).
\end{proof}

We have classified all finite abelian groups (well, at least we declared so), i.e.\ \(\Z\)-modules. What if we want to classify all \(R\)-modules? That is going to be the final goal we will build towards.

Recall that \(M\) is finitely generated by \(m_1, \dots, m_k\) if and only if there is a surjection \(f: R^k \surj M\). \(M\) is finitely presentely if and only \(\ker f\) is finitely generated, say \(n_1, \dots, n_\ell\). Let
\[
  n_i =
  \begin{pmatrix}
    r_{1i} \\
    r_{ri} \\
    \vdots \\
    r_{ki}
  \end{pmatrix}
\]
then such an \(R\)-module \(M\) is determined by the matrix
\[
  \begin{pmatrix}
    r_{11} & r_{12} & \cdots & r_{1\ell} \\
    r_{r1} & & & \\
    \vdots & & \ddots & \\
    r_{k1} & & & r_{k\ell}
  \end{pmatrix}
  \in \matrixring_{k, \ell}(R).
\]

\subsection{Matrices over Euclidean Domains}

For this section assume \(R\) to be a Euclidean domain and let \(\varphi: R \setminus \{0\} \to \Z_{\geq 0}\) be the Euclidean function. For \(a, b \in R\), we have shown that \(\gcd(a, b)\) exists and is unique up to associates. In addition, the Euclidean algorithm shows that \(\gcd(a, b) = ax + by\) for some \(x, y \in R\).

What follows would be very similar to what we have learned in IB Linear Algebra --- in fact identical except at one place:

\begin{definition}[Elementary row operation]\index{tbd}
  \emph{Elementary row operation} on an \(m \times n\) matrix \(m\) with entries in \(R\) are
\end{definition}
\printindex
\end{document}