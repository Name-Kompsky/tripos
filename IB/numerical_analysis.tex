\documentclass[a4paper]{article}

\def\npart{IB}

\def\ntitle{Numerical Analysis}
\def\nlecturer{H.\ Fawzi}

\def\nterm{Lent}
\def\nyear{2018}

\ifx \nauthor\undefined
  \def\nauthor{Qiangru Kuang}
\else
\fi

\ifx \ntitle\undefined
  \def\ntitle{Template}
\else
\fi

\ifx \nauthoremail\undefined
  \def\nauthoremail{qk206@cam.ac.uk}
\else
\fi

\ifx \ndate\undefined
  \def\ndate{\today}
\else
\fi

\title{\ntitle}
\author{\nauthor}
\date{\ndate}

%\usepackage{microtype}
\usepackage{mathtools}
\usepackage{amsthm}
\usepackage{stmaryrd}%symbols used so far: \mapsfrom
\usepackage{empheq}
\usepackage{amssymb}
\let\mathbbalt\mathbb
\let\pitchforkold\pitchfork
\usepackage{unicode-math}
\let\mathbb\mathbbalt%reset to original \mathbb
\let\pitchfork\pitchforkold

\usepackage{imakeidx}
\makeindex[intoc]

%to address the problem that Latin modern doesn't have unicode support for setminus
%https://tex.stackexchange.com/a/55205/26707
\AtBeginDocument{\renewcommand*{\setminus}{\mathbin{\backslash}}}
\AtBeginDocument{\renewcommand*{\models}{\vDash}}%for \vDash is same size as \vdash but orginal \models is larger
\AtBeginDocument{\let\Re\relax}
\AtBeginDocument{\let\Im\relax}
\AtBeginDocument{\DeclareMathOperator{\Re}{Re}}
\AtBeginDocument{\DeclareMathOperator{\Im}{Im}}
\AtBeginDocument{\let\div\relax}
\AtBeginDocument{\DeclareMathOperator{\div}{div}}

\usepackage{tikz}
\usetikzlibrary{automata,positioning}
\usepackage{pgfplots}
%some preset styles
\pgfplotsset{compat=1.15}
\pgfplotsset{centre/.append style={axis x line=middle, axis y line=middle, xlabel={$x$}, ylabel={$y$}, axis equal}}
\usepackage{tikz-cd}
\usepackage{graphicx}
\usepackage{newunicodechar}

\usepackage{fancyhdr}

\fancypagestyle{mypagestyle}{
    \fancyhf{}
    \lhead{\emph{\nouppercase{\leftmark}}}
    \rhead{}
    \cfoot{\thepage}
}
\pagestyle{mypagestyle}

\usepackage{titlesec}
\newcommand{\sectionbreak}{\clearpage} % clear page after each section
\usepackage[perpage]{footmisc}
\usepackage{blindtext}

%\reallywidehat
%https://tex.stackexchange.com/a/101136/26707
\usepackage{scalerel,stackengine}
\stackMath
\newcommand\reallywidehat[1]{%
\savestack{\tmpbox}{\stretchto{%
  \scaleto{%
    \scalerel*[\widthof{\ensuremath{#1}}]{\kern-.6pt\bigwedge\kern-.6pt}%
    {\rule[-\textheight/2]{1ex}{\textheight}}%WIDTH-LIMITED BIG WEDGE
  }{\textheight}% 
}{0.5ex}}%
\stackon[1pt]{#1}{\tmpbox}%
}

%\usepackage{braket}
\usepackage{thmtools}%restate theorem
\usepackage{hyperref}

% https://en.wikibooks.org/wiki/LaTeX/Hyperlinks
\hypersetup{
    %bookmarks=true,
    unicode=true,
    pdftitle={\ntitle},
    pdfauthor={\nauthor},
    pdfsubject={Mathematics},
    pdfcreator={\nauthor},
    pdfproducer={\nauthor},
    pdfkeywords={math maths \ntitle},
    colorlinks=true,
    linkcolor={red!50!black},
    citecolor={blue!50!black},
    urlcolor={blue!80!black}
}

\usepackage{cleveref}



% TODO: mdframed often gives bad breaks that cause empty lines. Would like to switch to tcolorbox.
% The current workaround is to set innerbottommargin=0pt.

%\usepackage[theorems]{tcolorbox}





\usepackage[framemethod=tikz]{mdframed}
\mdfdefinestyle{leftbar}{
  %nobreak=true, %dirty hack
  linewidth=1.5pt,
  linecolor=gray,
  hidealllines=true,
  leftline=true,
  leftmargin=0pt,
  innerleftmargin=5pt,
  innerrightmargin=10pt,
  innertopmargin=-5pt,
  % innerbottommargin=5pt, % original
  innerbottommargin=0pt, % temporary hack 
}
%\newmdtheoremenv[style=leftbar]{theorem}{Theorem}[section]
%\newmdtheoremenv[style=leftbar]{proposition}[theorem]{proposition}
%\newmdtheoremenv[style=leftbar]{lemma}[theorem]{Lemma}
%\newmdtheoremenv[style=leftbar]{corollary}[theorem]{corollary}

\newtheorem{theorem}{Theorem}[section]
\newtheorem{proposition}[theorem]{Proposition}
\newtheorem{lemma}[theorem]{Lemma}
\newtheorem{corollary}[theorem]{Corollary}
\newtheorem{axiom}[theorem]{Axiom}
\newtheorem*{axiom*}{Axiom}

\surroundwithmdframed[style=leftbar]{theorem}
\surroundwithmdframed[style=leftbar]{proposition}
\surroundwithmdframed[style=leftbar]{lemma}
\surroundwithmdframed[style=leftbar]{corollary}
\surroundwithmdframed[style=leftbar]{axiom}
\surroundwithmdframed[style=leftbar]{axiom*}

\theoremstyle{definition}

\newtheorem*{definition}{Definition}
\surroundwithmdframed[style=leftbar]{definition}

\newtheorem*{slogan}{Slogan}
\newtheorem*{eg}{Example}
\newtheorem*{ex}{Exercise}
\newtheorem*{remark}{Remark}
\newtheorem*{notation}{Notation}
\newtheorem*{convention}{Convention}
\newtheorem*{assumption}{Assumption}
\newtheorem*{question}{Question}
\newtheorem*{answer}{Answer}
\newtheorem*{note}{Note}
\newtheorem*{application}{Application}

%operator macros

%basic
\DeclareMathOperator{\lcm}{lcm}

%matrix
\DeclareMathOperator{\tr}{tr}
\DeclareMathOperator{\Tr}{Tr}
\DeclareMathOperator{\adj}{adj}

%algebra
\DeclareMathOperator{\Hom}{Hom}
\DeclareMathOperator{\End}{End}
\DeclareMathOperator{\id}{id}
\DeclareMathOperator{\im}{im}
\DeclareMathOperator{\coker}{coker}
\DeclarePairedDelimiter{\generation}{\langle}{\rangle}

%groups
\DeclareMathOperator{\sym}{Sym}
\DeclareMathOperator{\sgn}{sgn}
\DeclareMathOperator{\inn}{Inn}
\DeclareMathOperator{\aut}{Aut}
\DeclareMathOperator{\GL}{GL}
\DeclareMathOperator{\SL}{SL}
\DeclareMathOperator{\PGL}{PGL}
\DeclareMathOperator{\PSL}{PSL}
\DeclareMathOperator{\SU}{SU}
\DeclareMathOperator{\UU}{U}
\DeclareMathOperator{\SO}{SO}
\DeclareMathOperator{\OO}{O}
\DeclareMathOperator{\PSU}{PSU}
\DeclareMathOperator{\Sp}{Sp}


%hyperbolic
\DeclareMathOperator{\sech}{sech}

%field, galois heory
\DeclareMathOperator{\ch}{ch}
\DeclareMathOperator{\gal}{Gal}
\DeclareMathOperator{\emb}{Emb}



%ceiling and floor
%https://tex.stackexchange.com/a/118217/26707
\DeclarePairedDelimiter\ceil{\lceil}{\rceil}
\DeclarePairedDelimiter\floor{\lfloor}{\rfloor}


\DeclarePairedDelimiter{\innerproduct}{\langle}{\rangle}

%\DeclarePairedDelimiterX{\norm}[1]{\lVert}{\rVert}{#1}
\DeclarePairedDelimiter{\norm}{\lVert}{\rVert}



%Dirac notation
%TODO: rewrite for variable number of arguments
\DeclarePairedDelimiterX{\braket}[2]{\langle}{\rangle}{#1 \delimsize\vert #2}
\DeclarePairedDelimiterX{\braketthree}[3]{\langle}{\rangle}{#1 \delimsize\vert #2 \delimsize\vert #3}

\DeclarePairedDelimiter{\bra}{\langle}{\rvert}
\DeclarePairedDelimiter{\ket}{\lvert}{\rangle}




%macros

%general

%divide, not divide
\newcommand*{\divides}{\mid}
\newcommand*{\ndivides}{\nmid}
%vector, i.e. mathbf
%https://tex.stackexchange.com/a/45746/26707
\newcommand*{\V}[1]{{\ensuremath{\symbf{#1}}}}
%closure
\newcommand*{\cl}[1]{\overline{#1}}
%conjugate
\newcommand*{\conj}[1]{\overline{#1}}
%set complement
\newcommand*{\stcomp}[1]{\overline{#1}}
\newcommand*{\compose}{\circ}
\newcommand*{\nto}{\nrightarrow}
\newcommand*{\p}{\partial}
%embed
\newcommand*{\embed}{\hookrightarrow}
%surjection
\newcommand*{\surj}{\twoheadrightarrow}
%power set
\newcommand*{\powerset}{\mathcal{P}}

%matrix
\newcommand*{\matrixring}{\mathcal{M}}

%groups
\newcommand*{\normal}{\trianglelefteq}
%rings
\newcommand*{\ideal}{\trianglelefteq}

%fields
\renewcommand*{\C}{{\mathbb{C}}}
\newcommand*{\R}{{\mathbb{R}}}
\newcommand*{\Q}{{\mathbb{Q}}}
\newcommand*{\Z}{{\mathbb{Z}}}
\newcommand*{\N}{{\mathbb{N}}}
\newcommand*{\F}{{\mathbb{F}}}
%not really but I think this belongs here
\newcommand*{\A}{{\mathbb{A}}}

%asymptotic
\newcommand*{\bigO}{O}
\newcommand*{\smallo}{o}

%probability
\newcommand*{\prob}{\mathbb{P}}
\newcommand*{\E}{\mathbb{E}}

%vector calculus
\newcommand*{\gradient}{\V \nabla}
\newcommand*{\divergence}{\gradient \cdot}
\newcommand*{\curl}{\gradient \cdot}

%logic
\newcommand*{\yields}{\vdash}
\newcommand*{\nyields}{\nvdash}

%differential geometry
\renewcommand*{\H}{\mathbb{H}}
\newcommand*{\transversal}{\pitchfork}
\renewcommand{\d}{\mathrm{d}} % exterior derivative

%number theory
\newcommand*{\legendre}[2]{\genfrac{(}{)}{}{}{#1}{#2}}%Legendre symbol

%algebraic geometry
\DeclareMathOperator{\Spec}{Spec}
\DeclareMathOperator{\Proj}{Proj}

\makeindex

\begin{document}

\begin{titlepage}
  \begin{center}
    \includegraphics[width=0.6\textwidth]{logo.jpg}\par
    \vspace{1cm}
    {\scshape\huge Mathamatics Tripos \par}
    \vspace{2cm}
    {\huge Part \npart \par}
    \vspace{0.6cm}
    {\Huge \bfseries \ntitle \par}
    \vspace{1.2cm}
    {\Large\nterm, \nyear \par}
    \vspace{2cm}
    
    {\large \emph{Lectures by } \par}
    \vspace{0.2cm}
    {\Large \scshape \nlecturer}
    
    \vspace{0.5cm}
    {\large \emph{Notes by }\par}
    \vspace{0.2cm}
    {\Large \scshape \href{mailto:\nauthoremail}{\nauthor}}
 \end{center}
\end{titlepage}

\tableofcontents

\setcounter{section}{-1}

\section{Introduction}

Numerical analysis is the study of \emph{algorithms} for continuous mathematics. Examples of problems in continuous mathematics are:
\begin{itemize}
\item solve \(f(x) = 0\) where \(f: \R^n \to \R\),
\item solve \(\frac{dx}{dt} = f(x)\) where \(f: \R^n \to \R^n\)
\item optimisation: find \(\min f(x)\) where \(x \in \R^n, f: \R^n \to \R\).
\end{itemize}

A note on complexity: we measure the complexity of an algorithm by the number of \emph{elementary operations} (\(+, \times, -, /\)) it needs.

Big \(O\) notation: for example \(O(n), O(n^2)\), where \(n\) is input size. We also have complexity \(O(f(n))\) if the number of operations is at most \(cf(n)\) where \(c > 0\).

\section{Polynomial Interpolations}

Denote a degree \(n\) polynomial by
\[
  p(x) = p_0 + p_1x + \dots + p_nx^n
\]
and let \(P_n[x]\) be the vector space of polynomials of degree at most \(n\). The interpolation problem is, given \(x_0, x_1, \dots, x_n \in \R\) and \(f_0, f_1, \dots, f_n \in \R\), find \(p \in P_n[x]\) such that \(p(x_i) = f_i\) for \(i = 0, \dots, n\).

\subsection{Lagrange formula}

Claim that
\[
  p(x) = \sum_{k = 0}^{n} f_k \underbrace{\prod_{\ell \neq k}^{ } \frac{x - x_\ell}{x_k - x_\ell}}_{L_k(x)}
\]
solves the problem.

Note that \(L_k(x_j) = \begin{cases} 1 & j = k \\ 0 &j \neq k \end{cases}\) and the result easily follows.

Now we prove the uniqueness of the solution. Assume \(q \in P_n[x]\) is another polynomial that interpolates the data. Then \(p - q\) has \(n + 1\) zeros. But a non-zero polynomial in \(P_n[x]\) has at most \(n\) zeros. Thus \(p - q\) must be zero.

This is an easy solution but what is its complexity? For each \(k\), the complexity of evaluating \(L_k(x)\) is \(O(n)\) so the total complexity of evaluating \(p(x)\) is \(O(n^2)\).

\subsubsection{Error of polynomial interpolation}

Let \(C^s[a, b]\) be the space of functions \([a, b] \to \R\) that are \(s\) times continuously differnetiable.

\begin{theorem}
  Let \(f \in C^{n + 1}[a, b]\) and let \(p \in P_n[x]\) interpolate \(f\) at distinct \(x_0, \dots x_n\) , i.e.\ \(p(x_i) = f(x_i)\) for \(i = 0, \dots, n\). Then for all \(x \in [a, b]\), there exists \(\xi \in [a, b]\) such that
  \[
    f(x) - p(x) = \frac{1}{(n + 1)!} f^{(n + 1)}(\xi) \prod_{i = 0}^n (x - x_i).
  \]
\end{theorem}

The last term \(\prod_{i = 0}^n (x - x_i)\) is called the \emph{nodal polynomial}.

\begin{proof}
  If \(x = x_i\) for some \(i\) then the result is trivially true. Assume \(x\) is distinct from \(x_i\)'s for all \(i\). Define
  \[
    \varphi(t) = f(t) - \left(p(t) + (f(x) - p(x)) \frac{\prod_{i = 0}^n(t - x_i)}{\prod_{i = 0}^n(x - x_i)} \right).
  \]
  Note that the second term is by construction the interpolating polynomial of \(f\) at \(x_0, \dots, x_n\) and \(x\). Thus
  \[
    \varphi(x_0) = \varphi(x_1) = \dots = \varphi(x_n) = \varphi(x) = 0
  \]
  so \(\varphi\) has \(n + 2\) zeros. Recall from IA Analysis I Rolle's Theorem: if \(g \in C^1[a, b]\) such that \(g(a) = g(b)\) then there exists \(\alpha \in (a, b)\) such that \(g'(\alpha) = 0\). Apply it to \(\varphi\), we deduce that \(\varphi'\) has \(n + 1\) zeros. Inductively we find that \(\varphi^{(n + 1)}\) has \(1\) zero, i.e.\ there exists \(\xi \in [a, b]\) such that \(\varphi^{(n + 1)}(\xi) = 0\). Thus
  \[
    0 = \varphi^{(n + 1)}(\xi) = f^{(n + 1)}(\xi) - \left( \underbrace{p^{(n + 1)}(\xi)}_{= 0} + (f(x) - p(x)) \frac{(n + 1)!}{\prod_{i = 0}^n(x - x_i)} \right).
  \]
  Rearrange,
  \[
    f(x) - p(x) = \frac{1}{(n + 1)!}f^{(n + 1)}(\xi) \prod_{i = 0}^n(x - x_i).
  \]
\end{proof}

\begin{eg}
  \([a, b] = [-5, 5]\), \(x_j = -5 + 10 \frac{j}{n}\) for \(j = 0, \dots, n\). Plot \(\prod_{i = 0}^n(x - x_i)\), we note that it vanishes at \(x_i\)'s but blows up near the endpoints.
\end{eg}

This is called \emph{Runge's phenomenon}. If one attempts to interpolate \(f(x) = \frac{1}{1 + x^2}\) using equispaced points on \([-5, 5]\) and plots the error \(f(x) - p(x)|\).

Thus equispaced points may not be the most suitable to minimise the error. The remedy is to look for points \(x_0, \dots, x_n\) such that \(|\prod_{i = 0}^n(x - x_i)|\) is small. This leads us to \emph{Chebyshev points}. For example in the previous case we should choose
\[
  x_j = 5 \cos \frac{(n - j)\pi}{n}.
\]
This choice of points is the one that minimises
\[
  \max_{x \in [a, b]} \left| \prod_{i =0}^n(x - x_i) \right|.
\]





\printindex

\iffalse
http://damtp.cam.ac.uk/user/hf323/L18-IB-NA/
\fi

\end{document}
