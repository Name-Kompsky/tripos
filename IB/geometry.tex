\documentclass[a4paper]{article}

\def\npart{IB}

\def\ntitle{Geometry}
\def\nlecturer{A.\ G.\ Kovalev}

\def\nterm{Lent}
\def\nyear{2018}

\ifx \nauthor\undefined
  \def\nauthor{Qiangru Kuang}
\else
\fi

\ifx \ntitle\undefined
  \def\ntitle{Template}
\else
\fi

\ifx \nauthoremail\undefined
  \def\nauthoremail{qk206@cam.ac.uk}
\else
\fi

\ifx \ndate\undefined
  \def\ndate{\today}
\else
\fi

\title{\ntitle}
\author{\nauthor}
\date{\ndate}

%\usepackage{microtype}
\usepackage{mathtools}
\usepackage{amsthm}
\usepackage{stmaryrd}%symbols used so far: \mapsfrom
\usepackage{empheq}
\usepackage{amssymb}
\let\mathbbalt\mathbb
\let\pitchforkold\pitchfork
\usepackage{unicode-math}
\let\mathbb\mathbbalt%reset to original \mathbb
\let\pitchfork\pitchforkold

\usepackage{imakeidx}
\makeindex[intoc]

%to address the problem that Latin modern doesn't have unicode support for setminus
%https://tex.stackexchange.com/a/55205/26707
\AtBeginDocument{\renewcommand*{\setminus}{\mathbin{\backslash}}}
\AtBeginDocument{\renewcommand*{\models}{\vDash}}%for \vDash is same size as \vdash but orginal \models is larger
\AtBeginDocument{\let\Re\relax}
\AtBeginDocument{\let\Im\relax}
\AtBeginDocument{\DeclareMathOperator{\Re}{Re}}
\AtBeginDocument{\DeclareMathOperator{\Im}{Im}}
\AtBeginDocument{\let\div\relax}
\AtBeginDocument{\DeclareMathOperator{\div}{div}}

\usepackage{tikz}
\usetikzlibrary{automata,positioning}
\usepackage{pgfplots}
%some preset styles
\pgfplotsset{compat=1.15}
\pgfplotsset{centre/.append style={axis x line=middle, axis y line=middle, xlabel={$x$}, ylabel={$y$}, axis equal}}
\usepackage{tikz-cd}
\usepackage{graphicx}
\usepackage{newunicodechar}

\usepackage{fancyhdr}

\fancypagestyle{mypagestyle}{
    \fancyhf{}
    \lhead{\emph{\nouppercase{\leftmark}}}
    \rhead{}
    \cfoot{\thepage}
}
\pagestyle{mypagestyle}

\usepackage{titlesec}
\newcommand{\sectionbreak}{\clearpage} % clear page after each section
\usepackage[perpage]{footmisc}
\usepackage{blindtext}

%\reallywidehat
%https://tex.stackexchange.com/a/101136/26707
\usepackage{scalerel,stackengine}
\stackMath
\newcommand\reallywidehat[1]{%
\savestack{\tmpbox}{\stretchto{%
  \scaleto{%
    \scalerel*[\widthof{\ensuremath{#1}}]{\kern-.6pt\bigwedge\kern-.6pt}%
    {\rule[-\textheight/2]{1ex}{\textheight}}%WIDTH-LIMITED BIG WEDGE
  }{\textheight}% 
}{0.5ex}}%
\stackon[1pt]{#1}{\tmpbox}%
}

%\usepackage{braket}
\usepackage{thmtools}%restate theorem
\usepackage{hyperref}

% https://en.wikibooks.org/wiki/LaTeX/Hyperlinks
\hypersetup{
    %bookmarks=true,
    unicode=true,
    pdftitle={\ntitle},
    pdfauthor={\nauthor},
    pdfsubject={Mathematics},
    pdfcreator={\nauthor},
    pdfproducer={\nauthor},
    pdfkeywords={math maths \ntitle},
    colorlinks=true,
    linkcolor={red!50!black},
    citecolor={blue!50!black},
    urlcolor={blue!80!black}
}

\usepackage{cleveref}



% TODO: mdframed often gives bad breaks that cause empty lines. Would like to switch to tcolorbox.
% The current workaround is to set innerbottommargin=0pt.

%\usepackage[theorems]{tcolorbox}





\usepackage[framemethod=tikz]{mdframed}
\mdfdefinestyle{leftbar}{
  %nobreak=true, %dirty hack
  linewidth=1.5pt,
  linecolor=gray,
  hidealllines=true,
  leftline=true,
  leftmargin=0pt,
  innerleftmargin=5pt,
  innerrightmargin=10pt,
  innertopmargin=-5pt,
  % innerbottommargin=5pt, % original
  innerbottommargin=0pt, % temporary hack 
}
%\newmdtheoremenv[style=leftbar]{theorem}{Theorem}[section]
%\newmdtheoremenv[style=leftbar]{proposition}[theorem]{proposition}
%\newmdtheoremenv[style=leftbar]{lemma}[theorem]{Lemma}
%\newmdtheoremenv[style=leftbar]{corollary}[theorem]{corollary}

\newtheorem{theorem}{Theorem}[section]
\newtheorem{proposition}[theorem]{Proposition}
\newtheorem{lemma}[theorem]{Lemma}
\newtheorem{corollary}[theorem]{Corollary}
\newtheorem{axiom}[theorem]{Axiom}
\newtheorem*{axiom*}{Axiom}

\surroundwithmdframed[style=leftbar]{theorem}
\surroundwithmdframed[style=leftbar]{proposition}
\surroundwithmdframed[style=leftbar]{lemma}
\surroundwithmdframed[style=leftbar]{corollary}
\surroundwithmdframed[style=leftbar]{axiom}
\surroundwithmdframed[style=leftbar]{axiom*}

\theoremstyle{definition}

\newtheorem*{definition}{Definition}
\surroundwithmdframed[style=leftbar]{definition}

\newtheorem*{slogan}{Slogan}
\newtheorem*{eg}{Example}
\newtheorem*{ex}{Exercise}
\newtheorem*{remark}{Remark}
\newtheorem*{notation}{Notation}
\newtheorem*{convention}{Convention}
\newtheorem*{assumption}{Assumption}
\newtheorem*{question}{Question}
\newtheorem*{answer}{Answer}
\newtheorem*{note}{Note}
\newtheorem*{application}{Application}

%operator macros

%basic
\DeclareMathOperator{\lcm}{lcm}

%matrix
\DeclareMathOperator{\tr}{tr}
\DeclareMathOperator{\Tr}{Tr}
\DeclareMathOperator{\adj}{adj}

%algebra
\DeclareMathOperator{\Hom}{Hom}
\DeclareMathOperator{\End}{End}
\DeclareMathOperator{\id}{id}
\DeclareMathOperator{\im}{im}
\DeclareMathOperator{\coker}{coker}
\DeclarePairedDelimiter{\generation}{\langle}{\rangle}

%groups
\DeclareMathOperator{\sym}{Sym}
\DeclareMathOperator{\sgn}{sgn}
\DeclareMathOperator{\inn}{Inn}
\DeclareMathOperator{\aut}{Aut}
\DeclareMathOperator{\GL}{GL}
\DeclareMathOperator{\SL}{SL}
\DeclareMathOperator{\PGL}{PGL}
\DeclareMathOperator{\PSL}{PSL}
\DeclareMathOperator{\SU}{SU}
\DeclareMathOperator{\UU}{U}
\DeclareMathOperator{\SO}{SO}
\DeclareMathOperator{\OO}{O}
\DeclareMathOperator{\PSU}{PSU}
\DeclareMathOperator{\Sp}{Sp}


%hyperbolic
\DeclareMathOperator{\sech}{sech}

%field, galois heory
\DeclareMathOperator{\ch}{ch}
\DeclareMathOperator{\gal}{Gal}
\DeclareMathOperator{\emb}{Emb}



%ceiling and floor
%https://tex.stackexchange.com/a/118217/26707
\DeclarePairedDelimiter\ceil{\lceil}{\rceil}
\DeclarePairedDelimiter\floor{\lfloor}{\rfloor}


\DeclarePairedDelimiter{\innerproduct}{\langle}{\rangle}

%\DeclarePairedDelimiterX{\norm}[1]{\lVert}{\rVert}{#1}
\DeclarePairedDelimiter{\norm}{\lVert}{\rVert}



%Dirac notation
%TODO: rewrite for variable number of arguments
\DeclarePairedDelimiterX{\braket}[2]{\langle}{\rangle}{#1 \delimsize\vert #2}
\DeclarePairedDelimiterX{\braketthree}[3]{\langle}{\rangle}{#1 \delimsize\vert #2 \delimsize\vert #3}

\DeclarePairedDelimiter{\bra}{\langle}{\rvert}
\DeclarePairedDelimiter{\ket}{\lvert}{\rangle}




%macros

%general

%divide, not divide
\newcommand*{\divides}{\mid}
\newcommand*{\ndivides}{\nmid}
%vector, i.e. mathbf
%https://tex.stackexchange.com/a/45746/26707
\newcommand*{\V}[1]{{\ensuremath{\symbf{#1}}}}
%closure
\newcommand*{\cl}[1]{\overline{#1}}
%conjugate
\newcommand*{\conj}[1]{\overline{#1}}
%set complement
\newcommand*{\stcomp}[1]{\overline{#1}}
\newcommand*{\compose}{\circ}
\newcommand*{\nto}{\nrightarrow}
\newcommand*{\p}{\partial}
%embed
\newcommand*{\embed}{\hookrightarrow}
%surjection
\newcommand*{\surj}{\twoheadrightarrow}
%power set
\newcommand*{\powerset}{\mathcal{P}}

%matrix
\newcommand*{\matrixring}{\mathcal{M}}

%groups
\newcommand*{\normal}{\trianglelefteq}
%rings
\newcommand*{\ideal}{\trianglelefteq}

%fields
\renewcommand*{\C}{{\mathbb{C}}}
\newcommand*{\R}{{\mathbb{R}}}
\newcommand*{\Q}{{\mathbb{Q}}}
\newcommand*{\Z}{{\mathbb{Z}}}
\newcommand*{\N}{{\mathbb{N}}}
\newcommand*{\F}{{\mathbb{F}}}
%not really but I think this belongs here
\newcommand*{\A}{{\mathbb{A}}}

%asymptotic
\newcommand*{\bigO}{O}
\newcommand*{\smallo}{o}

%probability
\newcommand*{\prob}{\mathbb{P}}
\newcommand*{\E}{\mathbb{E}}

%vector calculus
\newcommand*{\gradient}{\V \nabla}
\newcommand*{\divergence}{\gradient \cdot}
\newcommand*{\curl}{\gradient \cdot}

%logic
\newcommand*{\yields}{\vdash}
\newcommand*{\nyields}{\nvdash}

%differential geometry
\renewcommand*{\H}{\mathbb{H}}
\newcommand*{\transversal}{\pitchfork}
\renewcommand{\d}{\mathrm{d}} % exterior derivative

%number theory
\newcommand*{\legendre}[2]{\genfrac{(}{)}{}{}{#1}{#2}}%Legendre symbol

%algebraic geometry
\DeclareMathOperator{\Spec}{Spec}
\DeclareMathOperator{\Proj}{Proj}

\DeclareMathOperator{\Isom}{Isom}

\makeindex

\begin{document}

\begin{titlepage}
  \begin{center}
    \includegraphics[width=0.6\textwidth]{logo.jpg}\par
    \vspace{1cm}
    {\scshape\huge Mathamatics Tripos \par}
    \vspace{2cm}
    {\huge Part \npart \par}
    \vspace{0.6cm}
    {\Huge \bfseries \ntitle \par}
    \vspace{1.2cm}
    {\Large\nterm, \nyear \par}
    \vspace{2cm}
    
    {\large \emph{Lectures by } \par}
    \vspace{0.2cm}
    {\Large \scshape \nlecturer}
    
    \vspace{0.5cm}
    {\large \emph{Notes by }\par}
    \vspace{0.2cm}
    {\Large \scshape \href{mailto:\nauthoremail}{\nauthor}}
 \end{center}
\end{titlepage}

\tableofcontents

\section{Eucldiean Geometry}

Let \((\cdot, \cdot)\) be the standard inner product, a.k.a.\ dot product on the Euclidean space \(\R^n\) where for \(x, y \in \R^n\),
\[
  (x, y) = x \cdot y = \sum_{i = 1}^{n}x_iy_i.
\]
This induces the Euclidean norm
\[
  \norm x = \sqrt{(x, x)}.
\]
Also define the Euclidean distance function
\[
  d(x, y) = \norm{x - y}
\]
which makes \(\R^n\) a metric space.

\begin{definition}[Isometry]\index{isometry}
  A map \(f: \R^n \to \R^n\) is an \emph{isometry} of \(\R^n\) if
  \[
    \forall P, Q \in \R^n,\, d(f(P), f(Q)) = d(P, Q).
  \]
\end{definition}

Recall that an \(n \times n\) matrix \(A\) is \emph{orthogonal} if
\[
  A^TA = AA^T = I.
\]
For any square matrix \(A\) we have
\[
  (Ax, Ay) = (Ax)^T(Ay) = x^TA^TAy = (x, A^TAy)
\]
so we find that \(A\) is orthogonal if and only if \((Ax, Ay) = (x, y)\) for all \(x, y \in \R^n\).

Another point of view:
\[
  (x, y) = \frac{1}{2}(\norm{x + y}^2 -  \norm x^2 - \norm y^2)
\]
so \(A\) is orthogonal if and only if \(\norm{Ax} = \norm x\) for all \(x \in \R^n\).

An example of isometry: let \(f(x) = Ax + b\) where \(b \in \R^n\), then
\[
  d(f(x), f(y)) = \norm{A(x - y)}.
\]
So \(f\) is an isometry if and only if \(A\) is orthogonal.

Surprisingly, it turns out all isometries have this form:

\begin{theorem}
  Every isometry \(f: \R^n \to \R^n\) is of the form \(f(x) = Ax + b\) for some orthogonal matrix \(A\) and some vector \(b \in \R^n\).
\end{theorem}

\begin{proof}
  Let \(e_1, \dots, e_n \in \R^n\) be the standard basis of \(\R^n\). Let \(b = f(0)\), \(a_i = f(e_i) - b\) for \(i = 1, \dots, n\). We want to show that \(a_i\)'s form an orthonormal basis. Firstly
  \[
    \norm{a_i} = \norm{f(e_i) - f(0)} = d(f(e_i), f(0)) = d(e_i, 0) = \norm{e_i} = 1
  \]
  so they have unit length. For \(i \neq j\),
  \begin{align*}
    (a_i, a_j) &= -\frac{1}{2}(\norm{a_i - a_j}^2 - \norm{a_i}^2 - \norm{a_j}^2) \\
               &= -\frac{1}{2}(\norm{f(e_i) - f(e_j)}^2 - 2 \\
               &= -\frac{1}{2}(\norm{e_i - e_j}^2 - 2) \\
               &= 0
  \end{align*}
  Thus \(a_i\)'s form an orthonormal basis of \(\R^n\) and it follows that the matrix \(A\) with columns \(a_i, \dots, a_n\) is orthogonal.

  Let \(g(x) = Ax + b\) which is an isometry. We have \(g(x) = f(x)\) for \(x = 0, e_1, \dots, e_n\). In addition,
  \[
    g^{-1}(x) = A^{-1}(x - b) = A^T(x - b)
  \]
  is an isometry so the composition \(h = g^{-1} \compose f\) is an isometry fixing \(0, e_1, \dots, e_n\). It then suffices to show \(h = \id\). Consider \(x = \sum_{i = 1}^n x_ie_i \in \R^n\). Let \(y = h(x) = \sum_{i = 1}^n y_ie_i\). Then
\begin{align*}
  d(x, e_i)^2 &= \norm x^2 + 1 - 2x_i \\
  d(x, 0)^2 &= \norm x^2 \\
  d(y, e_i)^2 &= \norm y^2 + 1 - 2y_i \\
  d(y, 0)^2 &= \norm y^2
\end{align*}
Since \(h\) is an isometry, \(h(0) = 0\), \(h(e_i) = e_i\) and \(h(x) = y\), we have \(\norm x = \norm y\) so \(x_i = y_i\) for all \(i\). Thus \(h(x) = x\) for all \(x \in \R^n\).
\end{proof}

\begin{remark}
  \[
    \Isom(\R^n) = \{\text{all isometries of } \R^n\}
  \]
  is a group by composition. This is also known as the group of rigid motions of \(\R^n\).
\end{remark}

\begin{eg}[Reflections in an affine hyperplane \(H \subset \R^n\)]
  Let
  \[
    H = \{x \in \R^n: u \cdot x = c\}
  \]
  where \(\norm u = 1\) and \(c \in \R\). Observe that \(u\) is perpendicular to \(H\) and so is a normal vector. The reflection in \(H\) is defined to be
  \[
    R_H: x \mapsto x - 2(x \cdot u - c)u.
  \]
  It is an exercise in example sheet to show that this is an isometry. Observe that if \(x \in H\) then \(R_H(x) = x\). If \(a \in H, t \in \R\) then
  \[
    R_H(a + tu) = (a + tu) - 2tu = a - tu.
  \]
  Thus \(R_h\) fixes exactly the points in \(H\).

  Conversely, suppose \(S \in \Isom(\R^n)\) and \(S\) fixes every point in \(H\). Let \(a \in H\) and defind translation by \(a\) as
  \[
    T_a(x) = x + a
  \]
  which is clearly an isometry. Conjugate \(S\) by \(T_a\), we get
  \[
    R = T_{-a}ST_a \in \Isom(\R^n)
  \]
  and \(R\) fixes \(H' = T_{-a}(H)\). We choose to work with \(H'\) since \(0 \in H'\), making it a subspace of \(\R^n\). Explicitly, if \(H = \{x: x\cdot u = c\}\) then \(H' = \{x: x \cdot u = 0\}\). Then for all \(x \in H'\),
  \[
    (Ru, x) = (Ru, Rx) = (u, x) = 0.
  \]
  Thus \(Ru\) is orthogonal to \(H'\), i.e.\ lies in the orthogonal complement of \(H'\) in \(\R^n\). Thus \(Ru = \lambda u\) for some \(\lambda \in R\) such that \(\lambda^2 = 1\). So \(\lambda = \pm 1\).

  Since \(R\) fixes \(0 \in \R^n\), \(R\) is linear by the previous theorem and either \(R = \id\) or \(R\) is given by the matrix
  \[
    \begin{pmatrix}
      -1 & & \\
      & 1 & \\
      & & \ddots \\
      & & & 1
    \end{pmatrix}
  \]
  i.e.\ \(R_{H'}\). If \(R = \id\) then \(S = \id\). If \(R = R_{H'}\) then \(S = T_aR_{H'}T_{-a}\). Check that
  \[
    S: x \mapsto x - a \mapsto (x - a) - 2(x \cdot u - a \cdot u) u \mapsto x - 2(x \cdot u - c)u
  \]
  is a reflection. Thus if \(S \in \Isom(\R^n)\) fixing \(H\) and \(S \neq \id \) then \(R\) is the reflection in \(H\).
\end{eg}



\printindex

\iffalse
Other courses that might be useful: topology, part of analysis II (differentiability in R^n and inverse function theorem)

Leads to: IID Differential Geometry

Reading List

P.\ Wilson, Curverd Spaces, CUP 2008
From classical geometries to elementary differential geometry
\fi


\end{document}
