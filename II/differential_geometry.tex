\documentclass[a4paper]{article}

\def\npart{II}

\def\ntitle{Differential Geometry}
\def\nlecturer{M.\ Dafermos}

\def\nterm{Lent}
\def\nyear{2018}

\ifx \nauthor\undefined
  \def\nauthor{Qiangru Kuang}
\else
\fi

\ifx \ntitle\undefined
  \def\ntitle{Template}
\else
\fi

\ifx \nauthoremail\undefined
  \def\nauthoremail{qk206@cam.ac.uk}
\else
\fi

\ifx \ndate\undefined
  \def\ndate{\today}
\else
\fi

\title{\ntitle}
\author{\nauthor}
\date{\ndate}

%\usepackage{microtype}
\usepackage{mathtools}
\usepackage{amsthm}
\usepackage{stmaryrd}%symbols used so far: \mapsfrom
\usepackage{empheq}
\usepackage{amssymb}
\let\mathbbalt\mathbb
\let\pitchforkold\pitchfork
\usepackage{unicode-math}
\let\mathbb\mathbbalt%reset to original \mathbb
\let\pitchfork\pitchforkold

\usepackage{imakeidx}
\makeindex[intoc]

%to address the problem that Latin modern doesn't have unicode support for setminus
%https://tex.stackexchange.com/a/55205/26707
\AtBeginDocument{\renewcommand*{\setminus}{\mathbin{\backslash}}}
\AtBeginDocument{\renewcommand*{\models}{\vDash}}%for \vDash is same size as \vdash but orginal \models is larger
\AtBeginDocument{\let\Re\relax}
\AtBeginDocument{\let\Im\relax}
\AtBeginDocument{\DeclareMathOperator{\Re}{Re}}
\AtBeginDocument{\DeclareMathOperator{\Im}{Im}}
\AtBeginDocument{\let\div\relax}
\AtBeginDocument{\DeclareMathOperator{\div}{div}}

\usepackage{tikz}
\usetikzlibrary{automata,positioning}
\usepackage{pgfplots}
%some preset styles
\pgfplotsset{compat=1.15}
\pgfplotsset{centre/.append style={axis x line=middle, axis y line=middle, xlabel={$x$}, ylabel={$y$}, axis equal}}
\usepackage{tikz-cd}
\usepackage{graphicx}
\usepackage{newunicodechar}

\usepackage{fancyhdr}

\fancypagestyle{mypagestyle}{
    \fancyhf{}
    \lhead{\emph{\nouppercase{\leftmark}}}
    \rhead{}
    \cfoot{\thepage}
}
\pagestyle{mypagestyle}

\usepackage{titlesec}
\newcommand{\sectionbreak}{\clearpage} % clear page after each section
\usepackage[perpage]{footmisc}
\usepackage{blindtext}

%\reallywidehat
%https://tex.stackexchange.com/a/101136/26707
\usepackage{scalerel,stackengine}
\stackMath
\newcommand\reallywidehat[1]{%
\savestack{\tmpbox}{\stretchto{%
  \scaleto{%
    \scalerel*[\widthof{\ensuremath{#1}}]{\kern-.6pt\bigwedge\kern-.6pt}%
    {\rule[-\textheight/2]{1ex}{\textheight}}%WIDTH-LIMITED BIG WEDGE
  }{\textheight}% 
}{0.5ex}}%
\stackon[1pt]{#1}{\tmpbox}%
}

%\usepackage{braket}
\usepackage{thmtools}%restate theorem
\usepackage{hyperref}

% https://en.wikibooks.org/wiki/LaTeX/Hyperlinks
\hypersetup{
    %bookmarks=true,
    unicode=true,
    pdftitle={\ntitle},
    pdfauthor={\nauthor},
    pdfsubject={Mathematics},
    pdfcreator={\nauthor},
    pdfproducer={\nauthor},
    pdfkeywords={math maths \ntitle},
    colorlinks=true,
    linkcolor={red!50!black},
    citecolor={blue!50!black},
    urlcolor={blue!80!black}
}

\usepackage{cleveref}



% TODO: mdframed often gives bad breaks that cause empty lines. Would like to switch to tcolorbox.
% The current workaround is to set innerbottommargin=0pt.

%\usepackage[theorems]{tcolorbox}





\usepackage[framemethod=tikz]{mdframed}
\mdfdefinestyle{leftbar}{
  %nobreak=true, %dirty hack
  linewidth=1.5pt,
  linecolor=gray,
  hidealllines=true,
  leftline=true,
  leftmargin=0pt,
  innerleftmargin=5pt,
  innerrightmargin=10pt,
  innertopmargin=-5pt,
  % innerbottommargin=5pt, % original
  innerbottommargin=0pt, % temporary hack 
}
%\newmdtheoremenv[style=leftbar]{theorem}{Theorem}[section]
%\newmdtheoremenv[style=leftbar]{proposition}[theorem]{proposition}
%\newmdtheoremenv[style=leftbar]{lemma}[theorem]{Lemma}
%\newmdtheoremenv[style=leftbar]{corollary}[theorem]{corollary}

\newtheorem{theorem}{Theorem}[section]
\newtheorem{proposition}[theorem]{Proposition}
\newtheorem{lemma}[theorem]{Lemma}
\newtheorem{corollary}[theorem]{Corollary}
\newtheorem{axiom}[theorem]{Axiom}
\newtheorem*{axiom*}{Axiom}

\surroundwithmdframed[style=leftbar]{theorem}
\surroundwithmdframed[style=leftbar]{proposition}
\surroundwithmdframed[style=leftbar]{lemma}
\surroundwithmdframed[style=leftbar]{corollary}
\surroundwithmdframed[style=leftbar]{axiom}
\surroundwithmdframed[style=leftbar]{axiom*}

\theoremstyle{definition}

\newtheorem*{definition}{Definition}
\surroundwithmdframed[style=leftbar]{definition}

\newtheorem*{slogan}{Slogan}
\newtheorem*{eg}{Example}
\newtheorem*{ex}{Exercise}
\newtheorem*{remark}{Remark}
\newtheorem*{notation}{Notation}
\newtheorem*{convention}{Convention}
\newtheorem*{assumption}{Assumption}
\newtheorem*{question}{Question}
\newtheorem*{answer}{Answer}
\newtheorem*{note}{Note}
\newtheorem*{application}{Application}

%operator macros

%basic
\DeclareMathOperator{\lcm}{lcm}

%matrix
\DeclareMathOperator{\tr}{tr}
\DeclareMathOperator{\Tr}{Tr}
\DeclareMathOperator{\adj}{adj}

%algebra
\DeclareMathOperator{\Hom}{Hom}
\DeclareMathOperator{\End}{End}
\DeclareMathOperator{\id}{id}
\DeclareMathOperator{\im}{im}
\DeclareMathOperator{\coker}{coker}
\DeclarePairedDelimiter{\generation}{\langle}{\rangle}

%groups
\DeclareMathOperator{\sym}{Sym}
\DeclareMathOperator{\sgn}{sgn}
\DeclareMathOperator{\inn}{Inn}
\DeclareMathOperator{\aut}{Aut}
\DeclareMathOperator{\GL}{GL}
\DeclareMathOperator{\SL}{SL}
\DeclareMathOperator{\PGL}{PGL}
\DeclareMathOperator{\PSL}{PSL}
\DeclareMathOperator{\SU}{SU}
\DeclareMathOperator{\UU}{U}
\DeclareMathOperator{\SO}{SO}
\DeclareMathOperator{\OO}{O}
\DeclareMathOperator{\PSU}{PSU}
\DeclareMathOperator{\Sp}{Sp}


%hyperbolic
\DeclareMathOperator{\sech}{sech}

%field, galois heory
\DeclareMathOperator{\ch}{ch}
\DeclareMathOperator{\gal}{Gal}
\DeclareMathOperator{\emb}{Emb}



%ceiling and floor
%https://tex.stackexchange.com/a/118217/26707
\DeclarePairedDelimiter\ceil{\lceil}{\rceil}
\DeclarePairedDelimiter\floor{\lfloor}{\rfloor}


\DeclarePairedDelimiter{\innerproduct}{\langle}{\rangle}

%\DeclarePairedDelimiterX{\norm}[1]{\lVert}{\rVert}{#1}
\DeclarePairedDelimiter{\norm}{\lVert}{\rVert}



%Dirac notation
%TODO: rewrite for variable number of arguments
\DeclarePairedDelimiterX{\braket}[2]{\langle}{\rangle}{#1 \delimsize\vert #2}
\DeclarePairedDelimiterX{\braketthree}[3]{\langle}{\rangle}{#1 \delimsize\vert #2 \delimsize\vert #3}

\DeclarePairedDelimiter{\bra}{\langle}{\rvert}
\DeclarePairedDelimiter{\ket}{\lvert}{\rangle}




%macros

%general

%divide, not divide
\newcommand*{\divides}{\mid}
\newcommand*{\ndivides}{\nmid}
%vector, i.e. mathbf
%https://tex.stackexchange.com/a/45746/26707
\newcommand*{\V}[1]{{\ensuremath{\symbf{#1}}}}
%closure
\newcommand*{\cl}[1]{\overline{#1}}
%conjugate
\newcommand*{\conj}[1]{\overline{#1}}
%set complement
\newcommand*{\stcomp}[1]{\overline{#1}}
\newcommand*{\compose}{\circ}
\newcommand*{\nto}{\nrightarrow}
\newcommand*{\p}{\partial}
%embed
\newcommand*{\embed}{\hookrightarrow}
%surjection
\newcommand*{\surj}{\twoheadrightarrow}
%power set
\newcommand*{\powerset}{\mathcal{P}}

%matrix
\newcommand*{\matrixring}{\mathcal{M}}

%groups
\newcommand*{\normal}{\trianglelefteq}
%rings
\newcommand*{\ideal}{\trianglelefteq}

%fields
\renewcommand*{\C}{{\mathbb{C}}}
\newcommand*{\R}{{\mathbb{R}}}
\newcommand*{\Q}{{\mathbb{Q}}}
\newcommand*{\Z}{{\mathbb{Z}}}
\newcommand*{\N}{{\mathbb{N}}}
\newcommand*{\F}{{\mathbb{F}}}
%not really but I think this belongs here
\newcommand*{\A}{{\mathbb{A}}}

%asymptotic
\newcommand*{\bigO}{O}
\newcommand*{\smallo}{o}

%probability
\newcommand*{\prob}{\mathbb{P}}
\newcommand*{\E}{\mathbb{E}}

%vector calculus
\newcommand*{\gradient}{\V \nabla}
\newcommand*{\divergence}{\gradient \cdot}
\newcommand*{\curl}{\gradient \cdot}

%logic
\newcommand*{\yields}{\vdash}
\newcommand*{\nyields}{\nvdash}

%differential geometry
\renewcommand*{\H}{\mathbb{H}}
\newcommand*{\transversal}{\pitchfork}
\renewcommand{\d}{\mathrm{d}} % exterior derivative

%number theory
\newcommand*{\legendre}[2]{\genfrac{(}{)}{}{}{#1}{#2}}%Legendre symbol

%algebraic geometry
\DeclareMathOperator{\Spec}{Spec}
\DeclareMathOperator{\Proj}{Proj}

\DeclareMathOperator{\codim}{codim}

\makeindex

\begin{document}

\begin{titlepage}
  \begin{center}
    \includegraphics[width=0.6\textwidth]{logo.jpg}\par
    \vspace{1cm}
    {\scshape\huge Mathamatics Tripos \par}
    \vspace{2cm}
    {\huge Part \npart \par}
    \vspace{0.6cm}
    {\Huge \bfseries \ntitle \par}
    \vspace{1.2cm}
    {\Large\nterm, \nyear \par}
    \vspace{2cm}
    
    {\large \emph{Lectures by } \par}
    \vspace{0.2cm}
    {\Large \scshape \nlecturer}
    
    \vspace{0.5cm}
    {\large \emph{Notes by }\par}
    \vspace{0.2cm}
    {\Large \scshape \href{mailto:\nauthoremail}{\nauthor}}
 \end{center}
\end{titlepage}

\tableofcontents

\section{Smooth Manifolds}

\subsection{Definitions}

\begin{definition}[Smooth]\index{smooth}
  Let \(U \subseteq \R^n\) be an open set and \(f: U \to \R^m\). \(f\) is said to be \emph{smooth} if
  \[
    \frac{\p^{|\alpha|} f}{\p x^\alpha}
  \]
  exists for all multi-indices \(\alpha\).
\end{definition}

\begin{notation}
  A multi-index \(\alpha = (\alpha_1, \dots, \alpha_n)\) is a tuple of non-negative integers and \(|\alpha| = \sum_{i = 1}^n \alpha_i\).
\end{notation}

\begin{definition}[Smooth]\index{smooth}
  Let \(X \subseteq \R^n\) be a subset. \(f: X \to \R^m\) is \emph{smooth} if for all \(x \in X\), there exists \(U \subseteq \R^n\), \(x \in U\) such that there exists smooth \(\tilde f: U \to \R^m\) extending \(f|_{X \cap U}\). i.e.\ the following diagram commutes
  \[
    \begin{tikzcd}
      U \ar[r, "\tilde f"] & \R^m \\
      U \cap X \ar[u, "\iota"] \ar[ur, "f"']
    \end{tikzcd}
  \]
\end{definition}

\begin{remark}\leavevmode
  \begin{enumerate}
    \item This is a local property.
    \item \(X\) is a topological space with subspace topology induced from \(\R^n\). \(f\) is smooth implies that \(f\) is continuous.
    \end{enumerate}
\end{remark}

\begin{definition}[Diffeomorphism]\index{diffeomorphism}
  Let \(X \subseteq \R^n\) and \(Y \subseteq \R^m\). A map \(f: X \to Y\) is a \emph{diffeomorphism} if \(f\) is smooth and bijective and its inverse \(f^{-1}: Y \to X\) is also smooth.
\end{definition}

\begin{remark}\leavevmode
  \begin{enumerate}
  \item Diffeomorphism implies homeomorphism (with repsect to the subspace topology).
  \item Diffeomorphism is an equivalence relation.
  \end{enumerate}
\end{remark}

\begin{definition}[Manifold]\index{manifold}
  A \emph{\(k\)-dimensional manifold} is a set \(X \subseteq \R^N\) such that for all \(x \in X\) there exists \(V \subseteq X\) open, \(x \in V\) such that \(V\) is diffeomorphic to an open subset \(U \subseteq \R^k\), i.e.\ \(X\) is locally diffeomorphic to \(\R^k\).
\end{definition}

\begin{remark}
  The diffeomorphism \(\varphi: U \to V\) is called a \emph{local parameterisation} and its inverse \(\varphi^{-1}: V \to U\) is called \emph{coordinate charts}. More specifically, let \(x_i: \R^k \to \R\) be the projection of the \(i\)th coordinate and we have \emph{local coordinate} \(x_i \compose \varphi^{-1}: V \to \R\). However, sometimes we abuse the notation and just write \(x_i: V \to \R\) for the above map.
\end{remark}

\begin{notation}
  Write \(\dim X = k\) if \(X\) is a \(k\)-dimensional manifold.
\end{notation}

\begin{eg}\leavevmode
  \begin{enumerate}
  \item \(\R^N\) is an \(N\)-dim manifold. For example, \(\varphi: \R^N \stackrel{\id}{\to} \R^N\).
  \item An open subset \(V \subseteq \R^N\) is an \(N\)-dim manifold, for example by taking \(\varphi\) to be the restriction \(\id|_V\).
  \item In general, if \(X\) is a manifold and \(V \subseteq X\) is an open subset then \(V\) is also a manifold.
  \item The \(n\)-sphere, \(S^n \subseteq \R^{n + 1} = \{x_1^2 + \dots + x_{n + 1}^2 = 1\}\) is an \(n\)-dim manifold. Suppose \(x \in S^n\) lies in the upper half hyperplane define by \(x_{n + 1} > 0\). Take
    \[
      V = S^n \cap \{x_{n + 1} > 0\},\, U = \{(x_1, \dots, x_n): \sum_{i = 1}^n x_i^2 < 1\} = B(\V 0, 1) \subseteq \R^n
    \]
    and define
    \begin{align*}
      \varphi: U &\to V \\
      (x_1, \dots, x_n) &\mapsto \left(x_1, \dots, x_n, +\sqrt{1 - (x_1^2 + \dots + x_n^2)}\right)
    \end{align*}
    Let's do a reality check: the map consists of algebraic operations so is smooth. It is easy to check that the image lies in \(V\). Its inverese is the projection onto the first \(n\) coordinates, which is the restriction of a smooth function so also smooth.

    Similarly, if \(x\) lies in \(\{x_{n + 1} < 0\}\), change \(+\) to \(-\) in the last coordinate of \(\varphi\) will do. This accounts for ``most'' points, with the exception of those on the equator. However, there is nothing special about \(x_{n + 1}\) so we may repeat for the other coordinates. As
    \[
      S^n = \bigcup_{i = 1}^{n + 1} \left( (S^n \cap \{x_i > 0\}) \cup (S^n \cap \{x_i < 0\}) \right)
    \]
    (which in English says that at least one coordinate is non-zero), we have covered \(S^n\) so done.
  \end{enumerate}
\end{eg}

\begin{ex}
  Suppose \(X \subseteq \R^n\) and \(Y \subseteq \R^m\) are manifolds. Show that \(X \times Y \subseteq \R^n \times \R^m = \R^{n + m}\) is a manifold of dimension \(\dim X + \dim Y\).
\end{ex}

\begin{definition}[Submanifold]\index{manifold!submanifold}
  Let \(X \subseteq \R^N\) be a manifold. A manifold \(Y \subseteq \R^N\) such that \(Y \subseteq X\) is a \emph{submanifold} of \(X\).
\end{definition}

\begin{remark}
  By definition, all manifolds are submanifolds of \(\R^N\) for some \(N\).
\end{remark}

Note that we have yet shown that the dimension of a manifold is well-defined but we will do so in a while. Assuming so, we define

\begin{definition}[Codimension]\index{codimension}
  The \emph{codimension} of \(Y\) in \(X\) is
  \[
    \codim_X(Y) = \dim X - \dim Y.
  \]
\end{definition}

\subsection{Tangent space}

\begin{definition}[Differential]\index{differential}
  Let \(U \subseteq \R^n\) and \(f: U \to \R^m\) be a smooth map. The \emph{differential} of \(f\) at \(x \in U\), \(df_x\) is a linear map
  \begin{align*}
    df_x: \R^n &\to \R^m \\
    h &\mapsto \lim_{t \to 0} \frac{f(x + th) - f(x)}{t}
  \end{align*}
\end{definition}

Explicitly, \(df_x\) is represented by the matrix
\[
  \left( \frac{\p f}{\p x_j} \right) \text{ where } f =
  \begin{pmatrix}
    f_1 \\
    \vdots \\
    f_m
  \end{pmatrix}
  .
\]
It then follows that
\[
  df_x(h) = \left(\frac{\p f}{\p x_j}\right)
  \begin{pmatrix}
    h_1 \\
    \vdots \\
    h_n
  \end{pmatrix}
  .
\]

\begin{proposition}[Chain rule]
  Let \(f: U \to V, g: V \to \R^p\) be smooth maps where \(U \subseteq \R^n, V \subseteq \R^m\). Let \(x \in U\) and \(f(x) \in V\). Then
  \[
    d(g \compose f)_x = dg_{f(x)} \compose df_x.
  \]
\end{proposition}

The aim of this section is to define differentials of maps between manifolds. Before that we have to find where differential lives. Certainly in \(\R^n\) it is a linear map living in a linear space.

Let \(X \subseteq \R^N\) be a \(k\)-dim manifold and \(x \in X\). Let \(\varphi\) be a local parameterisation around \(x\). Wlog assume \(\varphi^{-1}(x) = 0\). Note that \(\varphi\) is defined on an open subset of \(\R^k\) so we can do calculus on it (this is false for \(\varphi^{-1}\) as \(V \subseteq \R^N\) may not be open). We can define
\[
  d\varphi_0: \R^k \to \R^N.
\]

\begin{definition}[Tangent space]\index{tangent space}
  The \emph{tangent space} of \(X\) at \(x\), denoted \(T_xX\), is \(d\varphi_0(\R^k)\), a subspace of \(\R^N\).
\end{definition}

For this to be a good definition, we need to show that it is independent of \(\varphi\) and \(\dim d\varphi_0(\R^k) = k\).

Suppose \(\varphi: U \to V\) and \(\tilde \varphi: \tilde U \to \tilde V\) are parameterisations around \(x\), again assumming wlog \(\varphi(0) = x = \tilde \varphi(0)\). By taking intersection we can assume \(V = \tilde V\). Write
\[
  \varphi = \tilde \varphi \compose \underbrace{(\tilde \varphi^{-1} \compose \varphi)}_{U \to \tilde U}.
\]
The term in the parenthesis is a map between open subsets of \(\R^k\) so we can do calculus on it. By chain rule
\[
  d\varphi_0 = d\tilde \varphi_0 \compose d(\tilde \varphi^{-1} \compose \varphi)_0.
\]
Note that \(\varphi^{-1} \compose \tilde \varphi\) is the inverse of \(\tilde \varphi^{-1} \compose \varphi\) so by applying chain rule to \(\id\), we get
\[
  \id = d(\varphi^{-1} \compose \tilde \varphi)_0 \compose d(\tilde \varphi^{-1} \compose \varphi)_0
\]
so both of them are invertible. Thus
\[
  d\varphi_0(\R^k) = d\tilde \varphi_0(\R^k).
\]

Next we want to show \(\dim d\varphi_0(\R^k) = k\). Certainly linear algebra tells us that it is at most \(k\). By definition there exists a smooth \(\psi: W \to \R^k\), where \(W \subseteq \R^N\) is an open subset containing \(x\) such that
\[
  \psi|_{X \cap W} = \varphi^{-1}.
\]
Assume wlog \(V \subseteq X \cap W\). Then
\[
  \id_U = \psi \compose \varphi
\]
so by chain rule
\[
  \id = d\psi_x \compose d\varphi_0.
\]
Thus \(d\varphi_0(\R^k)\) is \(k\)-dimensional.

\begin{remark}
  The functions \(\varphi^{-1} \compose \tilde \varphi\) are called \emph{transition functions} and will appear later in the course.
\end{remark}

\begin{corollary}
  The dimension of a manifold is well-defined.
\end{corollary}

\begin{ex}
  If \(X \subseteq \R^N\) is a manifold and \(Y\) is a submanifold of \(X\) then
  \[
    T_yY \leq T_yX
  \]
  for all \(y \in Y\). In particular \(\codim_X(Y) \geq 0\).
\end{ex}

\begin{eg}\leavevmode
  \begin{enumerate}
  \item \(T_x\R^N = \R^N\).
  \item Let \(X\) be an open subset of \(\R^N\). Then \(T_xX = \R^N\).
  \item Let
    \[
      X = \{(x_1, x_2, \dots, x_k, 0, \dots, 0)\} \subseteq \R^N
    \]
    which can be seen as the image of the embedding \(\varphi: \R^k \to \R^N\). This is a \(k\)-dim manifold. This is a linear map so \(d\varphi_x = \varphi\). Thus \(T_xX = X\). In general, this holds if \(\varphi\) is an injective linear map.
  \item Let \(S^n \subseteq \R^{n + 1}\). Given a point \(x \in S^n\) with \(x_{n + 1} > 0\), use the parameterisation
    \begin{align*}
      \varphi: B(\V 0, 1) &\to \R^{n + 1} \\
      (x_1, \dots, x_n) &\mapsto \left(x_1, \dots, x_n, + \sqrt{1 - x_1^2 - \dots - x_n^2}\right)
    \end{align*}
    Then
    \[
      \im d\varphi_{(x_1, \dots, x_n)} = \text{span}\left\{\frac{\p \varphi}{\p x_1}, \dots, \frac{\p \varphi}{\p x_n}\right\}.
    \]
    We have
    \[
      \frac{\p \varphi}{\p x_i} = \left(0, \dots, 1, \dots, 0, -\frac{x_i}{x_{n + 1}}\right)
    \]
    and we can verify that for all \(i\),
    \[
      \frac{\p \varphi}{\p x_i} \cdot (x_1, \dots, x_n, x_{n + 1}) = 0
    \]
    so \(T_xS^n\) is \(\{v: v \cdot x = 0\} \subseteq \R^{n + 1}\).
  \end{enumerate}
\end{eg}

Now we go on to define differential of maps between manifolds. Let \(f: X \to Y\) be a smooth map of manifolds where \(X \subseteq \R^n, Y \subseteq \R^m\). Let \(\varphi:U \to V\) and \(\tilde \varphi: \tilde U \to \tilde V\) be local parameterisations around \(x\) and \(f(x)\), where \(U \subseteq \R^k, \tilde U \subseteq \R^\ell\). wlog we assume \(\varphi(0) = x, \tilde \varphi(0) = f(x)\). We also assume \(f(V) \subseteq \tilde V\).

We can thus define a map \(\tilde \varphi^{-1} \compose f \compose \varphi: U \to \tilde U\) and do calculus on it. As \(T_xX = \im d\varphi_0\) and \(d\varphi_0\) is an injective map, we can define the inverse of its restriction to the image, denoted by
\[
  (d\varphi_0)^{-1}: T_xX \to \R^k.
\]

\begin{definition}[Differential]\index{differential}
  The \emph{differential map} of \(f\) at \(x \in X\), denoted \(df_x\), is a linear map \(df_x: T_xX \to T_{f(x)}Y\), defined as the composition
  \[
    d\tilde \varphi_0 \compose d(\tilde \varphi^{-1} \compose f \compose \varphi)_0 \compose (d\varphi_0)^{-1}.
  \]

\[
  \begin{tikzcd}
    T_xX \ar[r, "df_0"] & T_{f(x)}Y \\
    \R^k \ar[u, "d\varphi_0"] \ar[r, "d(\tilde \varphi^{-1} \compose f \compose \varphi)_0"'] & \R^\ell \ar[u, "d\tilde\varphi_0"']
  \end{tikzcd}
\]
\end{definition}

Now as everything else we defined in differential geometry, we have to check that it is independent of parameterisation. It will be a horrible job, and one that you would want to do only (at most) once in your lifetime.

Let \(\psi, \tilde \psi\) be another pair of parameterisations. Define \(\varphi\) in terms of transition maps
\[
  \varphi = \psi \compose (\psi^{-1} \compose \varphi).
\]
By chain rule,
\[
  d\varphi_0 = d\psi_0 \compose d(\psi^{-1} \compose \varphi)_0
\]
so
\[
  (d\varphi_0)^{-1} = (d(\psi^{-1} \compose \varphi)_0)^{-1} \compose (d\psi_0)^{-1} = d(\varphi^{-1} \compose \psi)_0 \compose (d\psi_0)^{-1}
\]
where the last equality comes from applying chain rule to 
\[
  (\varphi^{-1} \compose \psi) \compose (\psi^{-1} \compose \varphi) = \id.
\]
Thus
\begin{align*}
  df_x &= d\tilde \varphi_0 \compose d(\tilde \varphi^{-1} \compose f \compose \varphi)_0 \compose (d\varphi_0)^{-1} \\
       &= d\tilde \varphi_0 \compose d(\tilde \varphi^{-1} \compose f \compose \varphi)_0 \compose d(\varphi^{-1} \compose \psi)_0 \compose (d\psi_0)^{-1} \\
       &= d\tilde \varphi_0 \compose d(\tilde \varphi^{-1} \compose f \compose \psi)_0 \compose (d\varphi_0)^{-1} \\
       &= \cdots \\
       &= d\tilde \psi_0 \compose d(\tilde \psi^{-1} \compose f \compose \psi)_0 \compose (d\psi_0)^{-1}
\end{align*}
where the omitted lines are similar and are left as an exercise.

\begin{proposition}[Chain rule]
  Let \(X, Y, Z\) be manifolds, \(f: X \to Y, g: Y \to Z\) smooth. Then for all \(x \in X\),
  \[
    d(g \compose f)_x = dg_{f(x)} \compose df_x.
  \]
\end{proposition}

\begin{proof}
  Tedious exercise using transition maps and chain rule on maps between Euclidean spaces.
\end{proof}

\begin{theorem}[Inverse Function Theorem]\index{inverse function theorem}
  Let \(f: X \to Y\) be a smooth map between manifolds. If \(df_x: T_xX \to T_{f(x)}Y\) is an isomorphism (which implies that in particular \(\dim X = \dim Y\)), then \(f\) is a local diffeomorphism. In other words, there exists an open subset \(V \subseteq X\) such that \(f|_V: V \to Y\) is a diffeomorphism onto its image.
\end{theorem}

\begin{proof}
  This is easy using Inverse Function Theorem from analysis. Let \(\varphi, \tilde \varphi\) be local parameterisations around \(x\) and \(f(x)\) respectively. Then \(d(\tilde \varphi^{-1} \compose f \compose \varphi)_0\) is surjection \(\R^k \to \R^k\) so \(\tilde \varphi^{-1} \compose f \compose \varphi\) is a local diffeomorphism of Euclidean spaces. Thus \(f\) is a local diffeomorphism of manifolds.
\end{proof}

\begin{ex}
  Suppose \(Y\) is a submanifold of \(X\) and \(f: X \to Z\) is a smooth map between manifolds. Then for all \(y \in Y\), show
  \[
    d(f|_Y)_y = df_y|_{T_yY}.
  \]
\end{ex}

\begin{ex}
  Let \(f: X \to Y, g: X \to Z\) be smooth maps between manifolds. We can define map
  \begin{align*}
    (f, g): X &\to Y \times Z \\
    x &\mapsto (f(x), g(x))
  \end{align*}
  Show that this map is smooth and for all \(x \in X\),
  \[
    d(f, g)|_x = (df_x, dg_x).
  \]
  Thus if differential is thought as a matrix, the map to the product space has differential made of blocks of matrices.
\end{ex}

\subsection{Pre-image theorem}

We defined manifolds in terms of charts but in reality, the pre-image theorem actually gives the most convenient way to define a manifold. First define some terminologies:

\begin{definition}[Critial point, critial value, regular value]\index{critical point}\index{critical value}\index{regular value}
  Let \(f: X \to Y\) be a smooth map between manifolds.
  \begin{itemize}
  \item A \emph{critical point} of \(f\) is a point \(x\) such that \(df_x: T_xX \to T_{f(x)}Y\) fails to be surjective. We denote the set of all critical points \(C\).
  \item A \emph{critical value} of \(f\) is a point \(y \in Y\) such that there exists \(x \in X\) critical and \(f(x) = y\), i.e.\ the set of critical values is \(f(C) \subseteq Y\).
  \item \(y \in Y\) is a \emph{regular value} if \(y\) is not a critical value, i.e.\ \(y \in Y \setminus f(C)\).
  \end{itemize}
\end{definition}

\begin{remark}\leavevmode
  \begin{enumerate}
  \item If \(\dim X < \dim Y\) then \(C = X\).
  \item If \(f^{-1}(y) = \emptyset\), i.e.\ \(y \notin f(X)\) then \(y\) is a regular value.
  \end{enumerate}
\end{remark}

\begin{theorem}[Pre-image theorem]
  Let \(f: X \to Y\) be a smooth map between manifolds. Given \(y\) a regular value of \(f\), \(f^{-1}(y) \subseteq X\) is a submanifold of \(X\) with codimension equal to the dimension of \(Y\), i.e.
  \[
    \dim X - \dim (f^{-1}(y)) = \dim Y.
  \]
\end{theorem}

\begin{remark}
  Suppose \(X\) is non-empty (which we may add to our definition of manifolds). Then this theorem gives a proof that \(\codim \geq 0\) for this particular case.
\end{remark}

\begin{proof}
  Let \(x \in f^{-1}(y)\). Suppose \(X, Y \subseteq \R^N\) and \(q = \dim X - \dim Y\). By surjectivity and rank-nullity, \(\dim \ker df_x = g\). It is an exercise in linear algebra to show that there exists a linear map \(T: \R^N \to \R^q\) such that \(\ker T \cap \ker df_x = 0\). Now define
  \begin{align*}
    F: X &\to Y \times \R^q \\
    x &\mapsto (f(x), T(x))
  \end{align*}
  Then by the exercises above \(F\) is smooth with differential \(dF_x = (df_x, dT_x)\) and \(d(T|_X)_x = dT|_{T_xX} = T|_{T_xX}\) so \(dF_x = (df_x, T)\) which is an isomorphism. Thus by Inverse Function Theorem \(F\) is a diffeomorphism around \(x\). There exists an open neighbourhood of \(x\) \(V \subseteq X\) such that
  \[
    F|_V: V \to f(V) \times \underbrace{T(V)}_{= U \subseteq \R^q}
  \]
  is a diffeomorphism. Then
  \[
    \varphi = (F|_V)^{-1}|_{\{y\} \times U}: U \to V
  \]
  is a local parameterisation of \(f^{-1}(y)\).
\end{proof}

\begin{corollary}
  Suppose \(\dim X = \dim Y\) and \(y\) is a regular value of some \(f: X \to Y\). Then \(f^{-1}(y)\) is a manifold of dimension \(0\).
\end{corollary}

If \(Z \subseteq \R^N\) is a manifold of dimension \(0\), then it is a collection of discrete points, i.e.\ for each \(z \in Z\), there exists an open neighbourhood of \(z\) whose intersection with \(Z\) is \(\{z\}\). Thus if \(X\) is in addition compact, then \(f^{-1}(y)\) is a finite set of points.

\begin{theorem}[Stack of records theorem]
  Under the above assumptions, there exist open neighbourhoods \(W_i\) of \(x_i\)'s such that
  \[
    f^{-1}\left(\bigcap_{i = 1}^n f(W_i) \setminus f(X \setminus \bigcup_{i = 1}^n W_i)\right)
  \]
  is a disjoint union of open neighbourhoods of \(x_i\)'s and the restriction of \(f\) on each of which is a diffeomorphism.
\end{theorem}

Given a regular value \(y\), \(T_xf^{-1}(y) = \ker df_x\). See example sheet 1.

\begin{eg}\leavevmode
  \begin{enumerate}
  \item Let \(S^n \subseteq \R^{n + 1}\) and
  \begin{align*}
    f: S^n &\to \R \\
    (x_1, \dots, x_{n + 1}) &\mapsto x_1^2 + \dots + x_{n + 1}^2
  \end{align*}
  which is smooth. We have
  \[
    df = (2x_1, \dots, 2x_{n + 1})
  \]
  which is surjective everywhere on \(S^n\). Thus by pre-image theorem \(f^{-1}(1) \subseteq \R^{n + 1}\) is a manifold of dimension \(1\), which is precisely \(S^n\).
\item Let \(O(n)\) be the set of \(n \times n\) orthogonal matrices, i.e.\ matrices \(A\) such that \(AA^T = I\). \(O(n)\) can be seen as a subset of \(\matrixring_n(\R) = \R^{n^2}\). Claim that \(O(n)\) is a manifold of dimension \(\frac{n(n - 1)}{2}\).

  Let \(S(n) \subseteq \matrixring_n(\R)\) be the subset of symmetric matrices. This is a subspace of \(\R^{n^2}\) so clearly a submanifold of dimension \(\frac{n(n + 1)}{2}\). Consider that map
  \begin{align*}
    f: \matrixring_n &\to S(n) \\
    A &\mapsto AA^T
  \end{align*}
  which is a smooth map. As \(f^{-1}(I) = O(n)\), if \(I\) is a regular value of \(f\) then \(O(n)\) is a manifold of dimension \(\frac{n(n - 1)}{2}\).
  \begin{align*}
    df_A(H) &= \lim_{s \to 0} \frac{(A + sH)(A + sH)^T - AA^T}{s} \\
            &= AH^T + HA^T
  \end{align*}
  Given \(C \in S(n)\), \(H = \frac{CA}{2}\) satisfies \(df_A(H) = C\) so \(df_A\) is surjective.
  \end{enumerate}
\end{eg}

\begin{remark}
  In addition to being a manifold, \(O(n)\) is a group under matrix multiplication. The group operations are smooth. We call it a \emph{Lie group}.

  It is also worthing pointing out that \(O(n)\) is not connected, as seen from the continuous map \(\det: O(n) \to \{\pm 1\}\). The connected component of identity is called \emph{special orthogonal group}
  \[
    SO(n) = \{A \in O(n): \det A = 1\}.
  \]
  Objects that preserve geometric objects are themselves geometric objects.
\end{remark}

\begin{definition}
  A set \(S \subseteq \R^n\) is said to be \emph{measure \(0\)} if for all \(\varepsilon > 0\) there exists a countable collection of cubes \(C_m\) such that \(S \subseteq \bigcup C_m\) and \(\sum \operatorname{volume} C_m < \varepsilon\).
\end{definition}

\begin{definition}
  Let \(X\) be an \(n\)-dim manfiold and \(S \subseteq X\). \(S\) has \emph{measure \(0\)} if for all local parameterisations \(\varphi: U \to V \subseteq X\), \(\varphi^{-1}(V \cap S) \subseteq \R^n\) is of measure \(0\).
\end{definition}

This is well-defined since we can show any manifold can be parameterised by a chart of countably many parameterisations and the countable union of measure \(0\) subsets in \(\R^n\) has measure \(0\).

\begin{remark}\leavevmode
  \begin{enumerate}
  \item An open subset \(V \subseteq X\) is not of measure \(0\).
  \item Suppose \(Y \subseteq X\) is a submanifold of \(X\) and \(\dim Y < \dim Y\). Then \(Y\) is of measure \(0\) in \(X.\)
  \end{enumerate}
\end{remark}

\begin{theorem}[Sard's Theorem]
  Let \(f: X \to Y\) be a smooth map between manifolds. Then the set of critical values of \(f\) is of measure \(0\) in \(Y\).
\end{theorem}

\begin{corollary}
  Regular values are dense.
\end{corollary}

\subsection{Transversality}

Suppose \(f: X \to Y\) is a smooth map between manifolds and \(Z \subseteq Y\) is a submanifold.

\begin{definition}[Transversality]\index{transversality}
  \(f\) is \emph{transversal to \(\Z\)}, denoted \(f \pitchfork Z\), if for all \(x \in f^{-1}(Z)\),
  \[
    \im df_x + T_{f(x)}Z = T_{f(x)}Y.
  \]
\end{definition}

\begin{remark}\leavevmode
  \begin{enumerate}
  \item If \(f(X) \cup Z = \emptyset\) then \(f\) is vacuously transversal.
  \item This is a generalisation of regular value: if \(Z = \{y\}\) then \(f \pitchfork Z\) if and only if \(y\) is a regular value of \(f\) since the tangent space of a dimension \(0\) manifold is \(0\).
  \end{enumerate}
\end{remark}

\begin{theorem}[Transversality theorem]\index{transversality theorem}
  Suppose \(f: X \to Y\) is a smooth map between manifolds and \(Z \subseteq Y\) is a submanifold. If \(f \pitchfork Z\) then \(f^{-1}(Z) \subseteq X\) is a submanifold of \(X\) with codimension equal to \(\codim_Y(Z)\).
\end{theorem}



\printindex

\iffalse
Classical differential geometry concerning geometries of curves and surfaces, from a modern point of view

Contents:
I: notions of smmoth \(k\)-dim manifolds. We study differntial topology.

Geometry is concerned with the study of rigid motions

invariants of curves: \(k\) curvature, \(\tau\) torsion
invariant of surfaces: \(K\) mean curvature, \(K\) Gaussian curvature
\fi

\end{document}
