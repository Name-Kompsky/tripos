\documentclass[a4paper]{article}

\def\npart{II}

\def\ntitle{Representation Theory}
\def\nlecturer{S.\ Martin}

\def\nterm{Lent}
\def\nyear{2019}

\ifx \nauthor\undefined
  \def\nauthor{Qiangru Kuang}
\else
\fi

\ifx \ntitle\undefined
  \def\ntitle{Template}
\else
\fi

\ifx \nauthoremail\undefined
  \def\nauthoremail{qk206@cam.ac.uk}
\else
\fi

\ifx \ndate\undefined
  \def\ndate{\today}
\else
\fi

\title{\ntitle}
\author{\nauthor}
\date{\ndate}

%\usepackage{microtype}
\usepackage{mathtools}
\usepackage{amsthm}
\usepackage{stmaryrd}%symbols used so far: \mapsfrom
\usepackage{empheq}
\usepackage{amssymb}
\let\mathbbalt\mathbb
\let\pitchforkold\pitchfork
\usepackage{unicode-math}
\let\mathbb\mathbbalt%reset to original \mathbb
\let\pitchfork\pitchforkold

\usepackage{imakeidx}
\makeindex[intoc]

%to address the problem that Latin modern doesn't have unicode support for setminus
%https://tex.stackexchange.com/a/55205/26707
\AtBeginDocument{\renewcommand*{\setminus}{\mathbin{\backslash}}}
\AtBeginDocument{\renewcommand*{\models}{\vDash}}%for \vDash is same size as \vdash but orginal \models is larger
\AtBeginDocument{\let\Re\relax}
\AtBeginDocument{\let\Im\relax}
\AtBeginDocument{\DeclareMathOperator{\Re}{Re}}
\AtBeginDocument{\DeclareMathOperator{\Im}{Im}}
\AtBeginDocument{\let\div\relax}
\AtBeginDocument{\DeclareMathOperator{\div}{div}}

\usepackage{tikz}
\usetikzlibrary{automata,positioning}
\usepackage{pgfplots}
%some preset styles
\pgfplotsset{compat=1.15}
\pgfplotsset{centre/.append style={axis x line=middle, axis y line=middle, xlabel={$x$}, ylabel={$y$}, axis equal}}
\usepackage{tikz-cd}
\usepackage{graphicx}
\usepackage{newunicodechar}

\usepackage{fancyhdr}

\fancypagestyle{mypagestyle}{
    \fancyhf{}
    \lhead{\emph{\nouppercase{\leftmark}}}
    \rhead{}
    \cfoot{\thepage}
}
\pagestyle{mypagestyle}

\usepackage{titlesec}
\newcommand{\sectionbreak}{\clearpage} % clear page after each section
\usepackage[perpage]{footmisc}
\usepackage{blindtext}

%\reallywidehat
%https://tex.stackexchange.com/a/101136/26707
\usepackage{scalerel,stackengine}
\stackMath
\newcommand\reallywidehat[1]{%
\savestack{\tmpbox}{\stretchto{%
  \scaleto{%
    \scalerel*[\widthof{\ensuremath{#1}}]{\kern-.6pt\bigwedge\kern-.6pt}%
    {\rule[-\textheight/2]{1ex}{\textheight}}%WIDTH-LIMITED BIG WEDGE
  }{\textheight}% 
}{0.5ex}}%
\stackon[1pt]{#1}{\tmpbox}%
}

%\usepackage{braket}
\usepackage{thmtools}%restate theorem
\usepackage{hyperref}

% https://en.wikibooks.org/wiki/LaTeX/Hyperlinks
\hypersetup{
    %bookmarks=true,
    unicode=true,
    pdftitle={\ntitle},
    pdfauthor={\nauthor},
    pdfsubject={Mathematics},
    pdfcreator={\nauthor},
    pdfproducer={\nauthor},
    pdfkeywords={math maths \ntitle},
    colorlinks=true,
    linkcolor={red!50!black},
    citecolor={blue!50!black},
    urlcolor={blue!80!black}
}

\usepackage{cleveref}



% TODO: mdframed often gives bad breaks that cause empty lines. Would like to switch to tcolorbox.
% The current workaround is to set innerbottommargin=0pt.

%\usepackage[theorems]{tcolorbox}





\usepackage[framemethod=tikz]{mdframed}
\mdfdefinestyle{leftbar}{
  %nobreak=true, %dirty hack
  linewidth=1.5pt,
  linecolor=gray,
  hidealllines=true,
  leftline=true,
  leftmargin=0pt,
  innerleftmargin=5pt,
  innerrightmargin=10pt,
  innertopmargin=-5pt,
  % innerbottommargin=5pt, % original
  innerbottommargin=0pt, % temporary hack 
}
%\newmdtheoremenv[style=leftbar]{theorem}{Theorem}[section]
%\newmdtheoremenv[style=leftbar]{proposition}[theorem]{proposition}
%\newmdtheoremenv[style=leftbar]{lemma}[theorem]{Lemma}
%\newmdtheoremenv[style=leftbar]{corollary}[theorem]{corollary}

\newtheorem{theorem}{Theorem}[section]
\newtheorem{proposition}[theorem]{Proposition}
\newtheorem{lemma}[theorem]{Lemma}
\newtheorem{corollary}[theorem]{Corollary}
\newtheorem{axiom}[theorem]{Axiom}
\newtheorem*{axiom*}{Axiom}

\surroundwithmdframed[style=leftbar]{theorem}
\surroundwithmdframed[style=leftbar]{proposition}
\surroundwithmdframed[style=leftbar]{lemma}
\surroundwithmdframed[style=leftbar]{corollary}
\surroundwithmdframed[style=leftbar]{axiom}
\surroundwithmdframed[style=leftbar]{axiom*}

\theoremstyle{definition}

\newtheorem*{definition}{Definition}
\surroundwithmdframed[style=leftbar]{definition}

\newtheorem*{slogan}{Slogan}
\newtheorem*{eg}{Example}
\newtheorem*{ex}{Exercise}
\newtheorem*{remark}{Remark}
\newtheorem*{notation}{Notation}
\newtheorem*{convention}{Convention}
\newtheorem*{assumption}{Assumption}
\newtheorem*{question}{Question}
\newtheorem*{answer}{Answer}
\newtheorem*{note}{Note}
\newtheorem*{application}{Application}

%operator macros

%basic
\DeclareMathOperator{\lcm}{lcm}

%matrix
\DeclareMathOperator{\tr}{tr}
\DeclareMathOperator{\Tr}{Tr}
\DeclareMathOperator{\adj}{adj}

%algebra
\DeclareMathOperator{\Hom}{Hom}
\DeclareMathOperator{\End}{End}
\DeclareMathOperator{\id}{id}
\DeclareMathOperator{\im}{im}
\DeclareMathOperator{\coker}{coker}
\DeclarePairedDelimiter{\generation}{\langle}{\rangle}

%groups
\DeclareMathOperator{\sym}{Sym}
\DeclareMathOperator{\sgn}{sgn}
\DeclareMathOperator{\inn}{Inn}
\DeclareMathOperator{\aut}{Aut}
\DeclareMathOperator{\GL}{GL}
\DeclareMathOperator{\SL}{SL}
\DeclareMathOperator{\PGL}{PGL}
\DeclareMathOperator{\PSL}{PSL}
\DeclareMathOperator{\SU}{SU}
\DeclareMathOperator{\UU}{U}
\DeclareMathOperator{\SO}{SO}
\DeclareMathOperator{\OO}{O}
\DeclareMathOperator{\PSU}{PSU}
\DeclareMathOperator{\Sp}{Sp}


%hyperbolic
\DeclareMathOperator{\sech}{sech}

%field, galois heory
\DeclareMathOperator{\ch}{ch}
\DeclareMathOperator{\gal}{Gal}
\DeclareMathOperator{\emb}{Emb}



%ceiling and floor
%https://tex.stackexchange.com/a/118217/26707
\DeclarePairedDelimiter\ceil{\lceil}{\rceil}
\DeclarePairedDelimiter\floor{\lfloor}{\rfloor}


\DeclarePairedDelimiter{\innerproduct}{\langle}{\rangle}

%\DeclarePairedDelimiterX{\norm}[1]{\lVert}{\rVert}{#1}
\DeclarePairedDelimiter{\norm}{\lVert}{\rVert}



%Dirac notation
%TODO: rewrite for variable number of arguments
\DeclarePairedDelimiterX{\braket}[2]{\langle}{\rangle}{#1 \delimsize\vert #2}
\DeclarePairedDelimiterX{\braketthree}[3]{\langle}{\rangle}{#1 \delimsize\vert #2 \delimsize\vert #3}

\DeclarePairedDelimiter{\bra}{\langle}{\rvert}
\DeclarePairedDelimiter{\ket}{\lvert}{\rangle}




%macros

%general

%divide, not divide
\newcommand*{\divides}{\mid}
\newcommand*{\ndivides}{\nmid}
%vector, i.e. mathbf
%https://tex.stackexchange.com/a/45746/26707
\newcommand*{\V}[1]{{\ensuremath{\symbf{#1}}}}
%closure
\newcommand*{\cl}[1]{\overline{#1}}
%conjugate
\newcommand*{\conj}[1]{\overline{#1}}
%set complement
\newcommand*{\stcomp}[1]{\overline{#1}}
\newcommand*{\compose}{\circ}
\newcommand*{\nto}{\nrightarrow}
\newcommand*{\p}{\partial}
%embed
\newcommand*{\embed}{\hookrightarrow}
%surjection
\newcommand*{\surj}{\twoheadrightarrow}
%power set
\newcommand*{\powerset}{\mathcal{P}}

%matrix
\newcommand*{\matrixring}{\mathcal{M}}

%groups
\newcommand*{\normal}{\trianglelefteq}
%rings
\newcommand*{\ideal}{\trianglelefteq}

%fields
\renewcommand*{\C}{{\mathbb{C}}}
\newcommand*{\R}{{\mathbb{R}}}
\newcommand*{\Q}{{\mathbb{Q}}}
\newcommand*{\Z}{{\mathbb{Z}}}
\newcommand*{\N}{{\mathbb{N}}}
\newcommand*{\F}{{\mathbb{F}}}
%not really but I think this belongs here
\newcommand*{\A}{{\mathbb{A}}}

%asymptotic
\newcommand*{\bigO}{O}
\newcommand*{\smallo}{o}

%probability
\newcommand*{\prob}{\mathbb{P}}
\newcommand*{\E}{\mathbb{E}}

%vector calculus
\newcommand*{\gradient}{\V \nabla}
\newcommand*{\divergence}{\gradient \cdot}
\newcommand*{\curl}{\gradient \cdot}

%logic
\newcommand*{\yields}{\vdash}
\newcommand*{\nyields}{\nvdash}

%differential geometry
\renewcommand*{\H}{\mathbb{H}}
\newcommand*{\transversal}{\pitchfork}
\renewcommand{\d}{\mathrm{d}} % exterior derivative

%number theory
\newcommand*{\legendre}[2]{\genfrac{(}{)}{}{}{#1}{#2}}%Legendre symbol

%algebraic geometry
\DeclareMathOperator{\Spec}{Spec}
\DeclareMathOperator{\Proj}{Proj}

\newcommand{\ccl}{{\mathcal C}} % conjugacy class
\newcommand*{\ip}{\innerproduct} % inner product

\begin{document}

\begin{titlepage}
  \begin{center}
    \includegraphics[width=0.6\textwidth]{logo.jpg}\par
    \vspace{1cm}
    {\scshape\huge Mathamatics Tripos \par}
    \vspace{2cm}
    {\huge Part \npart \par}
    \vspace{0.6cm}
    {\Huge \bfseries \ntitle \par}
    \vspace{1.2cm}
    {\Large\nterm, \nyear \par}
    \vspace{2cm}
    
    {\large \emph{Lectures by } \par}
    \vspace{0.2cm}
    {\Large \scshape \nlecturer}
    
    \vspace{0.5cm}
    {\large \emph{Notes by }\par}
    \vspace{0.2cm}
    {\Large \scshape \href{mailto:\nauthoremail}{\nauthor}}
 \end{center}
\end{titlepage}

\tableofcontents

\setcounter{section}{-1}

\section{Introduction}

Representation theory is the theory of how \emph{groups} act as groups on \emph{vector spaces}. Here
\begin{enumerate}
\item groups are either finite or compact topological groups,
\item vector spaces are finite-diemnsional and usually over \(\C\),
\item actions are linear. 
\end{enumerate}

\section{Group actions}

\begin{notation}\leavevmode
  \begin{enumerate}
  \item \(\F\) is a field, usually \(\C\), \(\R\) or \(\Q\). In particular \(\F\) is a field of characteristic zero. Thus in this course we mostly deal with what is known as \emph{ordinary representation theory}. Sometimes \(\F = \F_p\) or \(\cl{\F_p}\), and the study of which is known as \emph{modular representation theory}.
  \item \(V\) is a vector space over \(\F\) and will always be finite-dimensional.
  \item \(\GL(V) = \{\theta: V \to V \text{ linear invertible}\}\).
  \end{enumerate}
\end{notation}

\subsection{Review of linear algebra}

If \(\dim_\F V = n\), choose basis \(e_1, \dots, e_n\) over \(\F\) so we can identify it with \(\F^n\). Then \(\theta \in \GL(V)\) correponds to an \(n \times n\) matrix \(A_\theta = (a_{ij})\), where
\[
  \theta(e_j) = \sum_i a_{ij} e_i
\]
for \(1 \leq j \leq n\). In fact we have \(A_\theta \in \GL_n(\F)\), the \emph{general linear group}. Thus

\begin{proposition}
  The map
  \begin{align*}
    \GL(V) &\to \GL_n(\F) \\
    \theta &\mapsto A_\theta
  \end{align*}
  is a group isomorphism.
\end{proposition}

\begin{proof}
  Check \(A_{\theta_1\theta_2} = A_{\theta_1}A_{\theta_2}\) and bijectivity.
\end{proof}

Choosing a different basis gives different isomorphism to \(\GL_n(\F)\), but

\begin{proposition}
  Matrices \(A_1, A_2\) represent the same element of \(\GL(V)\) with respect to different basis if and only if they are \emph{conjugate} or \emph{similar}, i.e.\ exists \(X \in \GL_n(\F)\) such that \(A_2 = XA_1X^{-1}\).
\end{proposition}

Recall that the \emph{trace} of a matrix \(A\) is
\[
  \tr A = \sum_i a_{ii}.
\]

\begin{proposition}
  As \(\tr(XAX^{-1}) = \tr A\) we can define
  \[
    \tr \theta = \tr(A_\theta)
  \]
  which is \emph{independent} of the basis chosen.
\end{proposition}

Some notes on diagonalisation:

\begin{eg}
  Let \(\alpha \in \GL(V)\) where \(V\) is a finite-dimensioanl vector space over \(\C\) with \(\alpha^m = \id\) for some \(m\). Then \(\alpha\) is diagonalisable.
\end{eg}

\begin{proposition}
  Let \(V\) a finite-dimensional vector space over \(\C\) and \(\alpha \in \End(V)\). Then \(\alpha\) is diagonalisable if and only if there exists a polynomial \(f\) with distinct linear factors with \(f(\alpha) = 0\).
\end{proposition}

\begin{remark}
  In the previous example take \(f(X) = X^m - 1 = \prod_{j = 0}^{m - 1} (X - \omega^j)\) where \(\omega = e^{\frac{2\pi i}{m}}\).
\end{remark}

\begin{proposition}
  \label{prop:simultaneous diagonalisable}
  A finite family of commuting separately diagonalisable non-singular transformations of a \(\C\)-vector space can be simultaneously diagonalised.
\end{proposition}

\subsection{Basic group theory}

We have an ample supply of basic groups:
\begin{enumerate}
\item symmetric group \(S_n = \sym(X)\) on a set \(X = \{1, \dots, n\}\) is the set of all permutations of \(X\). \(|S_n| = n!\).
\item alternating group \(A_n\) with \(|A_n| = \frac{n!}{2}\) consists of all even permutations.
\item cyclic group of order \(n\): \(C_n = \langle x: x^m = 1\rangle\). For example \((\Z/m\Z, +)\). It's also
  \begin{itemize}
  \item the group of \(m\)th root of unity in \(\C\) (which embeds to \(\GL_1(\C) = \C^*\)),
  \item the group of rotations, centre \(0\) of a regular \(m\)-gon in \(\R^2\) (which embeds to \(\GL_2(\R)\)).
  \end{itemize}
\item diahedral groups: \(D_{2m} = \langle x, y: x^m = y^2 = 1, yxy^{-1} = x^{-1} \rangle\) of order \(2m\). Think of this as set of rotations and reflections preserving a regular \(m\)-gon.
\item quaternion group: \(Q_8 = \langle x, y: x^4 = 1, y^2 = x^2, yxy^{-1} = x^{-1} \rangle\) of order \(8\). In \(\GL_2(\C)\), can put
  \[
    i =
    \begin{pmatrix}
      i & 0 \\
      0 & -i
    \end{pmatrix}
    \quad
    j =
    \begin{pmatrix}
      0 & 1 \\
      -1 & 0
    \end{pmatrix}
    \quad
     k =
    \begin{pmatrix}
      0 & i \\
      i & 0
    \end{pmatrix}
  \]
  then \(Q_8 = \{\pm i, \pm j, \pm k, \pm I_2\}\).
\end{enumerate}

\begin{definition}[conjugacy class, centraliser]\index{conjugacy class}\index{centraliser}
  The \emph{conjugacy class} of \(g \in G\) is
  \[
    \ccl_G(g) = \{xgx^{-1}: x \in G\}.
  \]

  Then
  \[
    |\ccl_G(g)| = |G: C_G(g)|
  \]
  where \(C_g(g) = \{x \in g: xg = gx\}\) is the \emph{centraliser} of \(g\) in \(G\).
\end{definition}

\begin{definition}[group action]\index{group action}
  Let \(G\) be a group and \(X\) be a set. \(G\) \emph{acts} on \(X\) if there exists a map
  \begin{align*}
    G \times X &\to X \\
    (g, x) &\mapsto gx
  \end{align*}
  such that
  \begin{align*}
    1x &= x \text{ for all } x \in X \\
    (gh) x &= g (hx) \text{ for all } g, h \in G, x \in X
  \end{align*}
\end{definition}

\begin{proposition}[permutation representation]\index{representation!permutation}
  Given an action of \(G\) on \(X\), we obtain a homomorphism \(\theta: G \to \sym(X)\), called the \emph{permutation representation} of \(G\).
\end{proposition}

\begin{proof}
  For \(g \in G\) the function \(\theta_g: X \to X, x \mapsto gx\) is a permutation of \(X\) (with inverse \(\theta_{g^{-1}}\). Moreover for all \(g_1, g_2 \in G\),
  \[
    \theta_{g_1 g_2} = \theta_{g_1} \theta_{g_2}
  \]
  since \((g_1g_2)x = g_1(g_2x)\) for all \(x \in X\).
\end{proof}

In this course \(X\) is often a finite-dimensional vector space over \(\F\)and the action is required to be \emph{linear}, namely
\begin{align*}
  g(v_1 + v_2) &= gv_1 + gv_2 \\
  g(\lambda v) &= \lambda g(v)
\end{align*}
for all \(v_1, v_2 \in V, g \in G, \lambda \in \F\).

\section{Basic definitions}

Let \(G\) be a finite group, \(\F\) a field.

\begin{definition}[representation]\index{representation}
  Let \(V\) be a finite-dimensional vector space over \(\F\). A \emph{(linear) representation} of \(G\) on \(V\) is a group homomorphism
  \[
    \rho = \rho_V: G \to \GL(V).
  \]
  Write \(\rho_g\) for \(\rho_V(g)\).
\end{definition}
So for each \(g \in G, \rho_g \in \GL(V), \rho_1 = \id\) and \(\rho_{g_1g_2} = \rho_{g_1}\rho_{g_2}, \rho_{g_1^{-1}} = \rho_{g_1}^{-1}\).

The \emph{dimension}\index{dimension} or \emph{degree}\index{degree} of \(\rho\) is \(\dim_\F V\).

Reall that \(\ker \rho \normal G\) and \(G / \ker \rho \cong \rho(G) \leq \GL(V)\). We say \(\rho\) is \emph{faithful}\index{faithful} if \(\ker \rho = \{1\}\).

We repeat what we said in introduction, namely the correspondence between group representation and group action:

\begin{definition}[linear action]\index{action!linear}
  \(G\) acts \emph{linearly} on \(V\) if ther exists a linear action \(G \times V \to V, (g, v) \mapsto gv\) such that
  \begin{align*}
    (g_1g_2) v &= g_1(g_2v), 1 v = v \\
    g(v_1 + v_2) &= gv_1 gv_2, g(\lambda v) = \lambda g(v)
  \end{align*}
\end{definition}

Now if \(G\) acts on \(V\), the map
\begin{align*}
  G &\to \GL(V) \\
  g &\mapsto \rho_g
\end{align*}
with \(\rho_g: v \mapsto gv\) is a representation. Conversely, given a representation \(G \to \GL(V)\) we have a linear action of \(G\) on \(V\) via
\[
  gv = \rho(g)(v).
\]

\begin{remark}
  We also say that \(V\) is a \emph{\(G\)-space}\index{\(G\)-space} or that \(V\) is a \emph{\(G\)-module}\index{\(G\)-module}. This use of ``module'' might seen unconventional but if fact if you define the \emph{group algebra}\index{group algebra}
  \[
    \F G = \left\{ \sum_{g\in G} \alpha_g g: \alpha_g \in \F \right\}
  \]
  with natural addition an multiplication, then \(V\) is an \(\F G\)-module. \(\F G\) is an example of \emph{\(\F\)-algebra}, i.e.\ a ring which is also an \(\F\)-module such that multiplication is bilinear.
\end{remark}

If we bring in a basis for \(V\), we get yet another equivalent definition:

\begin{definition}[matrix representation]\index{matrix representation}
  \(R\) is a \emph{matrix representation} of \(G\) of degree \(n\) if \(R\) is a homomorphism \(G \to \GL_n(\F)\).
\end{definition}

Given a linear representation \(\rho: G \to \GL(V)\) with \(\dim_FV = n\), fix a basis \(\mathcal B\) then we get a matrix representation
\begin{align*}
  G &\to \GL_n(\F) \\
  g &\mapsto [\rho(g)]_{\mathcal B}
\end{align*}
Conversely, given a matrix representation \(R: G \to \GL_n(\F)\), you get a linear representation
\begin{align*}
  \rho: G &\to \GL(\F^n) \\
  g &\mapsto \rho_g
\end{align*}
via \(\rho_g(v) = R_g(v)\).

\begin{eg}
  Given any group \(G\), take \(V = \F\) (the \(1\) dimensional space) and
  \begin{align*}
    \rho: G &\to \GL(V) \\
    g &\mapsto \id_V
  \end{align*}
  is known as the \emph{trivial representation}\index{representation!trivial}. \(\deg \rho = 1\).
\end{eg}

\begin{eg}
  Let \(G = C_4 = \langle x: x^4 = 1 \rangle\). Take \(\F = \C\) and let \(n = 2\). Then \(R: x \mapsto X\) will determine \(x^j \mapsto X^j\) and thus the matrix representation \(R\). We need \(X^4 = I\). We can take
  \begin{itemize}
  \item either \(X\) diagonal: any such with diagonal entries in \(\{\pm 1, \pm i\}\) (16 choices),
  \item or \(X\) is not diagonal: then it will be conjugate to a diagonal (by diagonalisability criterion).
  \end{itemize}
\end{eg}

\subsection{Equivalent representations}

\begin{definition}[\(G\)-homomorphism, \(G\)-isomorphism]\index{\(G\)-homomorphism}\index{interwining homomorphism}\index{isomorphic}\index{equivalent}
  Fix \(G\) and \(\F\). Let \(V\) and \(V'\) be \(\F\)-vector spaces and \(\rho: G \to \GL(V), \rho': G \to \GL(V')\) be representations of \(G\). The linear map \(\varphi: V \to V'\) is a \emph{\(G\)-homomorphism} or \emph{interwining homomorphism} if
  \[
    \varphi \rho(g) = \rho'(g) \varphi.
  \]
  In other words, the following diagram commutes:
  \[
    \begin{tikzcd}
      V \ar[r, "\rho_g"] \ar[d, "\varphi"] & V \ar[d, "\varphi"] \\
      V' \ar[r, "\rho_g'"] & V'
    \end{tikzcd}
  \]

  We say \(\varphi\) \emph{intertwines} \(\rho\) and \(\rho'\). Write \(\Hom_G(V, V')\) for the \(\F\)-space of all such.

  \(\varphi\) is a \emph{\(G\)-isomorphism} if \(\varphi\) is also bijective. If such a \(\varphi\) exists, say \(\rho\) and \(\rho'\) are \emph{isomorphic} or \emph{equivalent}. If \(\varphi\) is a \(G\)-isomorphism we can write the interwinding condition as
  \[
    \rho' = \varphi \rho \varphi^{-1}.
  \]
\end{definition}

\begin{lemma}
  Being isomorphic is an equivalence relation on the set of all representations of \(G\) over \(\F\).
\end{lemma}

\begin{proof}
  Exercise.
\end{proof}

\begin{remark}
  If \(\rho\) and \(\rho'\) are isomorphic representation then they have the same dimension. The converse is false:  \(C_4\) has four non-isomorphic \(1\) dimensional representations.
\end{remark}

\begin{remark}
  Given \(G, V, \F\) with \(\dim_\F V = n\) and \(\rho: G \to \GL(V)\), fix a basis \(\mathcal B\) of \(V\). We get an isomorphism
  \begin{align*}
    \varphi: V &\to \F^n \\
    v &\mapsto [v]_{\mathcal B}
  \end{align*}
  And \(\varphi\) gives a representation \(\rho': G \to \GL(\F^n)\) isomorphic to \(\rho\).
\end{remark}

\begin{proposition}\leavevmode
  \begin{enumerate}
  \item Transformations in terms of matrix representatives: \(R: G \to \GL_n(\F), R': G \to \GL_n(\F)\) are \(G\)-isomorphic or \(G\)-equivalent if exists \(X \in \GL_n(\F)\) with
    \[
      R'(g) = XR(g)X^{-1}
    \]
    for all \(g \in G\).
  \item In terms of linear \(G\)-actions, the action of \(G\) on \(V, V'\) are \(G\)-isomorphic if there exists \(\varphi: V \to V'\) such that
    \[
      g\varphi(v) = \varphi(gv)
    \]
    for all \(g \in G, v \in V\).
\end{enumerate}
\end{proposition}

\subsection{Subrepresentation}

\begin{definition}[\(G\)-subspace]\index{\(G\)-subspace}
  Let \(\rho: G \to \GL(V)\) be a representation of \(G\). We say that \(W \leq V\) is a \emph{\(G\)-subspace} if it is a subspace and it is \(\rho(G)\)-invariant, i.e.\ \(\rho_g(W) \subseteq W\) for all \(g \in G\).
\end{definition}

Obviously \(\{0\}\) and \(V\) are \(G\)-subspaces. On the other hand,

\begin{definition}[irreducible/simple representation]\index{representation!irreducible}\index{representation!simple}
  \(\rho\) is said to be \emph{irreducible} or \emph{simple} representation if there are no proper \(G\)-subspaces.
\end{definition}

\begin{eg}
  Any 1 dimensional representation of \(G\) is irreducible. The converse is not true. For example \(D_8\) has a 2 dimensional irreducible representation.
\end{eg}

\begin{definition}[subrepresentation]\index{subrepresentation}
  If \(W\) is a \(G\)-subspace then the corresponding map
  \begin{align*}
    G &\to \GL(W) \\
    g &\mapsto \rho(g)|_W
  \end{align*}
  is a representation of \(G\), known as a \emph{subrepresentation} of \(\rho\).
\end{definition}

\begin{lemma}
  If \(\rho: G \to \GL(V)\) is a representation, \(W\) is a \(G\)-subspace of \(V\) and \(\mathcal B = \{v_1, \dots, v_n\}\) is a basis containing a basis \(\{v_1, \dots, v_m\}\) of \(W\), where \(0 < m \leq n\), then the matrix of \(\rho(g)\) with respect to \(\mathcal B\) has block upper triangular form
  \[
    \begin{pmatrix}
      * & * \\
      0 & *
    \end{pmatrix}
  \]
  for each \(g \in G\).
\end{lemma}

\begin{eg}
  Let \(\F = \C\).
  \begin{enumerate}
  \item Irreducible representation of \(C_4 = \langle x: x^4 = 1\rangle\) are all 1 dimensional and four of them are
    \[
      x \mapsto i, x \mapsto -1, x \mapsto -i, x \mapsto 1.
    \]

    In general \(C_m\) has precisely \(m\) inequivalent complex irreducible representations, all of degree 1. Actually all complex irreducible representations of a \emph{finite abelian group} are 1 dimensional, by simultaneous diagonalisation and primary decomposition. Alternatively, this follows from Schur's lemma\index{Schur's lemma}.
  \item \(G = D_6\): every irreducible \(\C\)-representation has dimension \(\leq 2\). Let \(\rho: G \to \GL(V)\) be an irreducible representation of \(G\). Let \(r\) be a rotation and \(s\) be reflection. Take an eigenvector \(v\) of \(\rho(r)\) so \(\rho(r)v = \lambda v\) for some \(\lambda \in \C, \lambda \neq 0\). Let
    \[
      W = \langle v, \rho(s) v \rangle \leq V.
    \]
    Since
    \begin{align*}
      \rho(s)\rho(s) v &= v \\
      \rho(r)\rho(s) v &= \rho(s)\rho(r)^{-1}v = \lambda^{-1}\rho(s)v
    \end{align*}
    so \(W\) is \(G\)-invariant. Since \(V\) is irreducible \(W = V\).
  \end{enumerate}
\end{eg}

\begin{definition}[(in)decomposable representation, direct sum]\index{representation!decomposable}\index{representation!direct sum}\index{representation!indecomposable}
  We say that \(\rho: G \to \GL(V)\) is \emph{decomposable} if there are proper \(G\)-invariant subspaces \(U, W\) with \(V = U \oplus W\). Say \(\rho\) is the \emph{direct sum} \(\rho_U \oplus \rho_W\). If no such subspaces exist we say \(\rho\) is \emph{indecomposable}.
\end{definition}

\begin{lemma}
  If \(\rho: G \to \GL(V)\) is decomposable, \(\mathcal B = \{v_1, \dots, v_k, w_1, \dots, w_\ell\}\) is a basis of \(V\) consisting of a basis of \(U\) and a basis of \(W\), then \(\rho(g)\) with respect to \(\mathcal B\) is block diagonal for all \(g \in G\).
\end{lemma}

\begin{definition}[direct sum]\index{representation!direct sum}
  Let \(\rho: G \to \GL(V), \rho': G \to \GL(V')\) be two representations. The \emph{direct sum} of \(\rho, \rho'\) is
  \begin{align*}
    \rho \oplus \rho': G &\to \GL(V \oplus V') \\
    (\rho \oplus \rho') (g) (v + v') &= \rho(g) v + \rho'(g) v'
  \end{align*}

  For matrix representations \(R: G \to \GL_n(\F), R': G \to \GL_{n'}(\F)\), define \(R \oplus R': G \to \GL_{n + n'}(\F)\) is given by
  \[
    g \mapsto
    \begin{pmatrix}
      R(g) & 0 \\
      0 & R'(g)
    \end{pmatrix}
  \]
  for all \(g\).
\end{definition}

\section{Complete reducibility and Maschke's theorem}

Given \(G, \F\) as usual.

\begin{definition}[completely reducible/semisimple representation]\index{representation!completely reducible}\index{representation!semisimple}
  A representation \(\rho: G \to \GL(V)\) is \emph{completely reducible} or \emph{semisimple} if it is a direct sum of irreducible representations.
\end{definition}

\begin{remark}
  Irreducible implies completely reducible. The converse is not true. See example sheet 1 question 3.
\end{remark}

From now on take \(G\) to be finite and \(\ch F = 0\) throughout this chapter.

\begin{theorem}[complete reducibility theorem]\index{complete reduciblity theorem}
  \label{thm:complete reducibility theorem}
  Every finite-dimensional representation \(V\) of a finite group over a field of characteristic \(0\) is completely reducible, i.e.\ \(V = V_1 \oplus \dots \oplus V_r\) is a direct sum of representations with each \(V_i\) irreducible.
\end{theorem}

In fact it is enough to prove

\begin{theorem}[Maschke]\index{Maschke's theorem}
  \label{thm:Maschke}
  Suppose \(G\) is finite and \(\rho: G \to \GL(V)\) is a representation with \(V\) finite-dimensional, \(\ch F = 0\). If \(W\) is a \(G\)-subspace of \(V\) then there exists a \(G\)-subspace \(U\) of \(V\) such that \(V = W \oplus U\), a direct sum of \(G\)-subspaces.
\end{theorem}

\begin{proof}
  Let \(W'\) be any complementary subspace of \(W\) in \(V\), i.e.\ \(V = W \oplus W'\). Let \(q: V \to W\) be the projection of \(V\) onto \(W\) along \(W'\), i.e.\ if \(v = w + w'\) then \(q(v) = w\). Define
  \[
    \overline q: v \mapsto \frac{1}{|G|} \sum_{g \in G} g q(g^{-1}(v)),
  \]
  the ``average of \(q\) over \(G\)''. Note that we've dropped the \(\rho\) in \(\rho(g)\) and \(\rho(g^{-1})\) to avoid excessive notations.

  Claim that \(\overline q: V \to W\): for \(v \in V\), \(q(g^{-1}(v)) \in W\) and \(g(W) \subseteq W\). Also \(\overline q(w) = w\) for \(w \in W\) as
  \[
    \overline q(w)
    = \frac{1}{|G|} \sum_{g \in G} g q(g^{-1}w)
    = \frac{1}{|G|} \sum_{g \in G} g (g^{-1}w)
    = \frac{1}{|G|} \sum_{g \in G} w
    = w
  \]
  Thus \(\overline q\) projects \(V\) onto \(W\).

  As \(\overline q\) is a projection we can write \(V = \im \overline q \oplus \ker \overline q = W \oplus \ker \overline q\). Need to show \(\ker \overline q\) is \(G\)-invariant. Note that if \(h \in G\)
  \begin{align*}
    h \overline q(v)
    &= h \frac{1}{|G|} \sum_g g q(g^{-1}v) \\
    &= \frac{1}{|G|} \sum_g hg q(g^{-1}v) \\
    &= \frac{1}{|G|} \sum_g (hg) q((hg)^{-1} hv) \\
    &= \frac{1}{|G|} \sum_g g q(g^{-1}(hv)) \\
    &= \overline q(hv).
  \end{align*}

  Thus if \(v \in \ker \overline q, h \in G\) then
  \[
    h\overline q(v) = 0 = \overline q(hv)
  \]
  so \(hv \in \ker \overline q\). Therefore
  \[
    V = \im \overline q \oplus \ker \overline q = W \oplus \ker \overline q
  \]
  which is a \(G\)-subspace decomposition.
\end{proof}

In fact, we only need \(\ch \F \ndivides |G|\).

\begin{remark}
  Complements are not unique. For example, take \(G = 1\). Then a representation of \(G\) is just a vector space. Take \(V = \C^2\). Then any proper subspace \(W \leq V\) will do.
\end{remark}

\begin{ex}
  Deduce \nameref{thm:complete reducibility theorem} from \nameref{thm:Maschke} by induction on dimension.
\end{ex}

We'll present another proof using inner product. This will generalise easily to compact Lie groups. Take \(\F = \C\).

Recall that for \(V\) a \(\C\)-vector space. \(\ip{\cdot, \cdot}\) is a \emph{Hermitian inner product}\index{inner product} if
\begin{enumerate}
\item \(\ip{w, v} = \conj{\ip{v, w}}\) for all \(v, w\).
\item sesquilinear: linear in second argument.
\item positive definite \(\ip{v, v} > 0\) if \(v = 0\).
\end{enumerate}
Furthermore \(\ip{\cdot, \cdot}\) is \emph{\(G\)-invariant} if
\[
  \ip{gv, gw} = {v, w}
\]
for all \(v, w \in V, g \in G\).

If \(W\) is a \(G\)-invariant subspace of \(V\) (with a \(G\)-invariant inner product) then \(W^\perp\) is also \(G\)-invariant and \(W = W \oplus W^\perp\): enough to show for all \(v \in W^\perp, g \in G\), have \(gv \in W^\perp\). But by definition \(\ip{v, w} = 0\) for all \(w \in W\). Thus by \(G\)-invariance \(\ip{gv, gw} = 0\) for all \(g\). Certainly \(\ip{gv, w'} = 0\) for all \(w' \in W\) as we can choose \(w = g^{-1}w' \in W\). The result thus follows.

Therefore if there is a \(G\)-invariant inner product on any complex \(G\)-space then we get another proof of Maschke's theorem.

\begin{lemma}[Weyl's unitary trick]\index{Weyl's unitary trick}
  Let \(\rho\) be a complex representation of a finite group \(G\) on the \(\C\)-vector space \(V\). Then there is a \(G\)-invariant inner product on \(V\).
\end{lemma}

\begin{proof}
  There exists an inner product on \(V\): take basis \(e_1, \dots, e_n\) and define \((e_i, e_j) = \delta_{ij}\). Extend sesquilinearly. Now define
  \[
    \ip{v, w} = \frac{1}{|G|} \sum_g (gv, gw).
  \]
  Easy exercise that \(\ip{\cdot, \cdot}\) is a \(G\)-invariant inner product. For example for \(G\)-invariance, for all \(h \in G\),
  \begin{align*}
    \ip{hv, hw}
    &= \frac{1}{|G|} \sum_g ((gh)v, (gh)w) \\
    &= \frac{1}{|G|} \sum_{g'} (g'v, g'w) \\
    &= \ip{v, w}
  \end{align*}
\end{proof}

\begin{corollary}
  Every finite subgroup of \(\GL_n(\C)\) is conjugate to a subgroup of \(U(n)\).
\end{corollary}

\begin{proof}
  Example sheet 1 Q5, Q12.
\end{proof}

\begin{definition}[regular representation]\index{representation!regular}
  Recall group algebra of \(G\) is the \(\F\)-space
  \[
    \F G = \text{span} \{e_g: g \in G\}.
  \]
  There is a linear \(G\)-action
  \[
    h . \sum_g a_ge_g = \sum_g a_g e_{hg} = \sum_{g'} a_{h^{-1}g'} e_{g'}.
  \]
  This is known as \emph{regular representation} of \(G\), denoted \(\rho_{\text{reg}}\).
\end{definition}

This is a faithful representation of dimension \(|G|\). We call \(V = \F G\) (sometimes also written \(\F[G]\)) the \emph{regular module}\index{regular module}.

It turns out that every irreducible representation of \(G\) is a subrepresentation of \(\rho_{\text{reg}}\):

\begin{proposition}
  Let \(\rho\) be an irreducible representation of \(G\) over a field of characteristic \(0\). Then \(\rho\) is isomorphic to a subrepresentation of \(\rho_{\text{reg}}\).
\end{proposition}

\begin{proof}
  Let \(\rho: G \to \GL(V)\) be irreducible and let \(v \in V\) nonzero. Consider
  \begin{align*}
    \theta: \F G &\to V \\
    \sum_g a_g e_g &\mapsto \sum_g a_g gv
  \end{align*}
  This is a \(G\)-homomorphism. Now \(V\) is irreducible and \(\im \theta = V\) since \(\im \theta\) is a \(G\)-subspace. Then \(\ker \theta\) is a \(G\)-subspace of \(\F G\). Let \(W\) be a \(G\)-complement of \(\ker \theta\) in \(\F G\). Thus
  \[
    W \cong \F G/ \ker \theta \cong \im \theta = V.
  \]
\end{proof}

More generally,

\begin{definition}[permutation representation]\index{representation!permutation}
  Let \(G\) act on a set \(X\). Let \(\F X = \text{span}\{e_x: x \in X\}\) with \(G\) action
  \[
    g . \sum_x a_x e_x = \sum_x a_x e_{gx}
  \]
  so we have a \(G\)-space \(\F X\). The representation \(G \to \GL(\F X)\) is the corresponding \emph{permutation representation}.
\end{definition}

\section{Schur's lemma}

\begin{theorem}[Schur's lemma]\index{Schur's lemma}\leavevmode
  \begin{enumerate}
  \item Assume \(V\) and \(W\) are irreducible \(G\)-spaces (over field \(\F\)). Then any \(G\)-homomorphism \(\theta: V \to W\) is either \(0\) or a \(G\)-isomorphism.
  \item Assume \(\F\) is algebraically closed and let \(V\) be an irreducible \(G\)-space. Then any \(G\)-endomorphism \(V \to V\) is a scalar multiple of the identity map \(1_V\) (a \emph{homothety}).
  \end{enumerate}
\end{theorem}

\begin{proof}\leavevmode
  \begin{enumerate}
  \item Let \(\theta: V \to W\) be a \(G\)-homomorphism. Then \(\ker \theta\) is a \(G\)-subspace of \(V\). Since \(V\) is irreducible either \(\ker \theta = 0\) or \(\ker \theta = V\). Similarly \(\im \theta = 0\) or \(\im \theta = W\). Hence either \(\theta = 0\) or \(\theta\) is injective and surjective.
  \item Since \(\F\) is algebraically closd, \(\theta\) has an eigenvalue \(\lambda\). Then \(\theta - \lambda 1_V\) is a singular \(G\)-endomorphsim on \(V\), so must be \(0\).
  \end{enumerate}
\end{proof}

Recall the \(\F\)-space \(\Hom_G(V, W)\) of all \(G\)-homomorphisms \(V \to W\), we can restate Schur's lemma
\begin{corollary}
  If \(V\) and \(W\) are irreducible complex \(G\)-spaces then
  \[
    \dim_\C \Hom_G(V, W) =
    \begin{cases}
      1 & \text{if \(V, W\) are \(G\)-isomorphic} \\
      0 & \text{otherwise}
    \end{cases}
  \]
\end{corollary}

\begin{proof}
  If \(V\) and \(W\) are not isomorphic then the only \(G\)-homomorphism \(V \to W\) is \(0\). Assume \(V \cong_G W\) and \(\theta_1, \theta_2 \in \Hom_G(V, W)\), both nonzero. Then \(\theta_2\) is invertible and \(\theta_2^{-1}\theta_1 \in \End_G(V)\) and nonzero, so \(\theta_2^{-1}\theta_1 = \lambda 1_V\). Then \(\theta_1 = \lambda \theta_2\).
\end{proof}

\begin{corollary}
  If \(G\) has a faithful complex irreducible representation then \(Z(G)\) is cyclic.
\end{corollary}

\begin{remark}
  The converse is false. See example sheet Q10.
\end{remark}

\begin{proof}
  Let \(\rho: G \to \GL(V)\) be a faithful representation over \(\C\). Let \(z \in Z(G)\), then \(\varphi_z: v \mapsto zv\) is a \(G\)-endomorphism, hence multiplication by a scalar, say \(\mu_z\). Then
  \begin{align*}
    Z(G) &\to \C^\times \\
    g &\mapsto \mu_g
  \end{align*}
  is a representation of \(Z(G)\) and is faithful since \(\rho\) is. Thus \(Z(G)\) is isomorphic to a finite subgroup of \(\C^\times\) so cyclic.
\end{proof}

This is our first group theoretic result based on representation theory. This is a recurring theme in representation theory.

\begin{corollary}
  The irreducible \(\C\)-representations of a finite abelian group \(G\) are all 1 dimensional.
\end{corollary}

\begin{proof}
  One can use \Cref{prop:simultaneous diagonalisable} to invoke simultaneous diagonalisation: if \(v\) is an eigenvector for each \(g \in G\) and if \(V\) is irreducible then \(V = \langle v \rangle\).

  Alternatively, let \(V\) be an irreducible representation. Given \(g \in G\), the map
  \begin{align*}
    \theta_g: V &\to V \\
    v &\mapsto gv
  \end{align*}
  is a \(G\)-endomorphism of \(V\). Hence \(\theta_g = \lambda_g 1_V\) for some \(\lambda_g \in \C\). Thus \(gv = \lambda_g v\) for any \(g \in G\). Thus as \(V \neq 0\) is irreducible, \(V = \langle v \rangle\).
\end{proof}

\begin{remark}
  This fails for \(\R\). For example \(C_3\) has two irreducible \(\R\)-representations, one of dimension 1 and one of dimension 2.
\end{remark}

Recall that every finite abelian group \(G\) is isomorphic to a product of cyclic groups. In fact it can be written as product of \(C_{p^\alpha}\) for various primes \(p\) and \(\alpha \geq 1\). The elements are uniquely determined up to order.

\begin{proposition}
  The finite abelian group \(G \cong C_{n_1} \times \dots \times C_{n_r}\) has precisely \(|G|\) irreducible \(\C\)-representations as described below.
\end{proposition}

\begin{proof}
  Write \(G = \langle x_1 \rangle \times \dots \times \langle x_r \rangle\) where \(|x_j| = n_j\). Suppose \(\rho\) is irreducible so it is 1 dimensional. Let \(\rho(1, \dots, x_j, \dots, 1) = \lambda_j\). Then \(\lambda_j^{n_j} = 1\) so \(\lambda_j\) is an \(n_j\)th root of unity. Now the values \((\lambda_1, \dots, \lambda_r)\) determine \(\rho\), and no two are equivalent.
\end{proof}

Note that however, there is no canonical bijective correspondence between the elements of \(G\) and the representations of \(G\). If you choose an isomorphism \(G \cong C_{a_1} \times \dots C_{a_r}\) then we can identify the two sets, but it depends on the choice of isomorphism.

\subsection{Isotypical decompositions}

We know that in characteristic \(0\), every representation \(V\) of \(G\) decomposes as \(\bigoplus V_i\) where each \(V_i\) is irreducible. How unique is this?

A wishlist of properties:
\begin{enumerate}
\item uniqueness: for each \(V\) there is only one way to decompose \(V = \bigoplus V_i\) with \(V_i\) irreducible.
\item uniqueness of isotypes: for each \(V\) there exist unique subrepresentations \(U_1, \dots, U_k\) such that \(V = \bigoplus U_i\) and if \(V_i \leq U_i, V_j' \leq U_j\) irreducible subrepresentations then \(V_i \cong V_j'\) if and only if \(i = j\).
\item uniqueness of factors: if \(\bigoplus_{i = 1}^k V_i \cong \bigoplus_{i = 1}^{k'} V_i'\) and \(V_i, V_i'\) are irreducible then \(k = k'\) and there exists \(\pi \in S_k\) such that \(V_{\pi(i)}' \cong V_i\).
\end{enumerate}

Evidently 1 is too strong (\(G = 1\) acting on any \(V\) with dimension \(> 1\)). However 2 and 3 do work. We will skip the proof and refer the reader to Teleman \textsection 5. However, we shall discuss how to calculate multiplicities of simples in the isotypes.

\begin{lemma}
  Let \(V, V_1, V_2\) be \(G\)-spaces.
  \begin{enumerate}
  \item \(\Hom_G(V, V_1 \oplus V_2) \cong \Hom_G(V, V_1) \oplus \Hom_G(V, V_2)\).
  \item \(\Hom_G(V_1 \oplus V_2, V) \cong \Hom_G(V_1, V) \oplus \Hom_G(V_2, V)\).
  \end{enumerate}
\end{lemma}

\begin{proof}
  Let \(\pi_i: V_1 \oplus V_2 \to V_i\) be the \(G\)-linear projections in \(V_i\) with kernel \(V_{3 - i}\). Then
  \begin{align*}
    \Hom_G(V, V_1 \oplus V_2) &\to \Hom_G(V, V_1) \oplus \Hom_G(V, V_2) \\
    \varphi &\mapsto (\pi_1 \varphi, \pi_2 \varphi)
  \end{align*}
  has inverse \((\psi_1, \psi_2) \mapsto \psi_1 + \psi_2\).

  Also the map
  \begin{align*}
    \Hom_G(V_1 \oplus V_2, V) &\to \Hom_G(V_1, V) \oplus \Hom_G(V_2, V) \\
    \varphi &\mapsto (\varphi|_{V_1}, \varphi|_{V_2})
  \end{align*}
  has inverse \((\psi_1, \psi_2) \mapsto \psi_1 \pi_1 + \psi_2 \pi_2\).
\end{proof}

\begin{corollary}
  Suppose \(\F\) is algebraically closed and \(V = \bigoplus_{i = 1}^n V_i\) is a decomposition into irreducibles. Then for each irreducible representation \(S\) of \(G\),
  \[
    \# \{j: V_j \cong S\} \cong \dim \Hom_G(S, V).
  \]
  This is known as the \emph{multiplicity}\index{multiplicity} of \(S\) in \(V\).
\end{corollary}

\begin{proof}
  By induction on \(n\). Obvious for \(n = 0, 1\). For \(n > 1\), write
  \[
    V = (\bigoplus_{i = 1}^{n - 1} V_i) \oplus V_n.
  \]
  Then
  \[
    \dim \Hom_G(S, (\bigoplus_{i = 1}^{n - 1} V_i) \oplus V_n)
    = \dim \Hom_G(S, \bigoplus_{i = 1}^{n - 1} V_i) + \dim \Hom_G(S, V_n)
  \]
  and use Schur's lemma.
\end{proof}

\begin{definition}[canonical decomposition]\index{canonical decomposition}
  A decomposition \(V = \bigoplus W_i\) where each \(W_j\) is isomorphic to \(n_j\) copies of irreducible representation \(S_j\) (each non-isomorphic for each \(j\)) is the \emph{canonical decomposition} or the \emph{decomposition into isotypical components} \(W_j\).
\end{definition}

For \(\F\) closed, the above lemma says that \(n_j = \dim \Hom_G(S_j, V)\), i.e.\ \(n_j\) is detectable at \(G\)-homomorphism level.

\begin{eg}
  Teleman \textsection 5 gives an example on \(D_6\).

  If \(G\) is finite abelian then every complex representation \(V\) of \(G\) has unique isotypical decomposition.
\end{eg}

\section{Character theory}

We want to attach invariants to a representation \(\rho\) of a finite group \(G\) on \(V\). Matrix coefficients of \(\rho(g)\) are basis-dependent so not true invariants.

\(\det\) is an invariant but not a very useful one, as lots of (inequivalent) representations have determinant \(1\). Instead we'll use trace.

Let \(\F = \C\) and let \(\rho = \rho_V: G \to \GL(V)\) be a representation.

\begin{definition}[character]\index{character}\index{character!irreducible}\index{character!faithful}\index{character!faithful, trivial}
  The \emph{character} \(\chi_\rho = \chi_V = \chi\) is defined as
  \begin{align*}
    \chi: G &\to \C \\
    g &\mapsto \tr \rho(g)
  \end{align*}
  The \emph{degree} of \(\chi_V\) is \(\dim V\).

  \(\chi\) is \emph{linear} if \(\dim V = 1\), in which case \(\chi\) is a homomorphism \(G \to \C^\times\). \(\chi\) is \emph{irreducible/faithful/trivial (or principal)} if \(\rho\) is. In the last case we also write \(\chi = 1_G\).
\end{definition}

It turns out that \(\chi\) is a complete invariant in the sense that it determines \(\rho\) up to isomorphism. We'll prove this later.

\begin{theorem}\leavevmode
  \begin{enumerate}
  \item \(\chi_V(1) = \dim V\).
  \item \(\chi_V\) is a \emph{class function}\index{class function}, namely it is conjugation invariant. Thus \(\chi_V\) is constant on the conjugacy class of \(G\).
  \item \(\chi_V(g^{-1}) = \conj{\chi_V(g)}\).
  \item For two representations \(V\) and \(W\),
    \[
      \chi_{V \oplus W} = \chi_V + \chi_W.
    \]
  \item 
  \end{enumerate}
\end{theorem}






\printindex
\end{document}

% Books: James-Liebeck [JL]
% Alperin & Bell
% Thomas: Representations of finite and Lie groups
% online notes: SM's notes, C. Teleman
% Webb: A course in finite group representation theory
% Curtis: Pioneers of representation theory