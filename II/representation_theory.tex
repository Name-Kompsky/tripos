\documentclass[a4paper]{article}

\def\npart{II}

\def\ntitle{Representation Theory}
\def\nlecturer{S.\ Martin}

\def\nterm{Lent}
\def\nyear{2019}

\ifx \nauthor\undefined
  \def\nauthor{Qiangru Kuang}
\else
\fi

\ifx \ntitle\undefined
  \def\ntitle{Template}
\else
\fi

\ifx \nauthoremail\undefined
  \def\nauthoremail{qk206@cam.ac.uk}
\else
\fi

\ifx \ndate\undefined
  \def\ndate{\today}
\else
\fi

\title{\ntitle}
\author{\nauthor}
\date{\ndate}

%\usepackage{microtype}
\usepackage{mathtools}
\usepackage{amsthm}
\usepackage{stmaryrd}%symbols used so far: \mapsfrom
\usepackage{empheq}
\usepackage{amssymb}
\let\mathbbalt\mathbb
\let\pitchforkold\pitchfork
\usepackage{unicode-math}
\let\mathbb\mathbbalt%reset to original \mathbb
\let\pitchfork\pitchforkold

\usepackage{imakeidx}
\makeindex[intoc]

%to address the problem that Latin modern doesn't have unicode support for setminus
%https://tex.stackexchange.com/a/55205/26707
\AtBeginDocument{\renewcommand*{\setminus}{\mathbin{\backslash}}}
\AtBeginDocument{\renewcommand*{\models}{\vDash}}%for \vDash is same size as \vdash but orginal \models is larger
\AtBeginDocument{\let\Re\relax}
\AtBeginDocument{\let\Im\relax}
\AtBeginDocument{\DeclareMathOperator{\Re}{Re}}
\AtBeginDocument{\DeclareMathOperator{\Im}{Im}}
\AtBeginDocument{\let\div\relax}
\AtBeginDocument{\DeclareMathOperator{\div}{div}}

\usepackage{tikz}
\usetikzlibrary{automata,positioning}
\usepackage{pgfplots}
%some preset styles
\pgfplotsset{compat=1.15}
\pgfplotsset{centre/.append style={axis x line=middle, axis y line=middle, xlabel={$x$}, ylabel={$y$}, axis equal}}
\usepackage{tikz-cd}
\usepackage{graphicx}
\usepackage{newunicodechar}

\usepackage{fancyhdr}

\fancypagestyle{mypagestyle}{
    \fancyhf{}
    \lhead{\emph{\nouppercase{\leftmark}}}
    \rhead{}
    \cfoot{\thepage}
}
\pagestyle{mypagestyle}

\usepackage{titlesec}
\newcommand{\sectionbreak}{\clearpage} % clear page after each section
\usepackage[perpage]{footmisc}
\usepackage{blindtext}

%\reallywidehat
%https://tex.stackexchange.com/a/101136/26707
\usepackage{scalerel,stackengine}
\stackMath
\newcommand\reallywidehat[1]{%
\savestack{\tmpbox}{\stretchto{%
  \scaleto{%
    \scalerel*[\widthof{\ensuremath{#1}}]{\kern-.6pt\bigwedge\kern-.6pt}%
    {\rule[-\textheight/2]{1ex}{\textheight}}%WIDTH-LIMITED BIG WEDGE
  }{\textheight}% 
}{0.5ex}}%
\stackon[1pt]{#1}{\tmpbox}%
}

%\usepackage{braket}
\usepackage{thmtools}%restate theorem
\usepackage{hyperref}

% https://en.wikibooks.org/wiki/LaTeX/Hyperlinks
\hypersetup{
    %bookmarks=true,
    unicode=true,
    pdftitle={\ntitle},
    pdfauthor={\nauthor},
    pdfsubject={Mathematics},
    pdfcreator={\nauthor},
    pdfproducer={\nauthor},
    pdfkeywords={math maths \ntitle},
    colorlinks=true,
    linkcolor={red!50!black},
    citecolor={blue!50!black},
    urlcolor={blue!80!black}
}

\usepackage{cleveref}



% TODO: mdframed often gives bad breaks that cause empty lines. Would like to switch to tcolorbox.
% The current workaround is to set innerbottommargin=0pt.

%\usepackage[theorems]{tcolorbox}





\usepackage[framemethod=tikz]{mdframed}
\mdfdefinestyle{leftbar}{
  %nobreak=true, %dirty hack
  linewidth=1.5pt,
  linecolor=gray,
  hidealllines=true,
  leftline=true,
  leftmargin=0pt,
  innerleftmargin=5pt,
  innerrightmargin=10pt,
  innertopmargin=-5pt,
  % innerbottommargin=5pt, % original
  innerbottommargin=0pt, % temporary hack 
}
%\newmdtheoremenv[style=leftbar]{theorem}{Theorem}[section]
%\newmdtheoremenv[style=leftbar]{proposition}[theorem]{proposition}
%\newmdtheoremenv[style=leftbar]{lemma}[theorem]{Lemma}
%\newmdtheoremenv[style=leftbar]{corollary}[theorem]{corollary}

\newtheorem{theorem}{Theorem}[section]
\newtheorem{proposition}[theorem]{Proposition}
\newtheorem{lemma}[theorem]{Lemma}
\newtheorem{corollary}[theorem]{Corollary}
\newtheorem{axiom}[theorem]{Axiom}
\newtheorem*{axiom*}{Axiom}

\surroundwithmdframed[style=leftbar]{theorem}
\surroundwithmdframed[style=leftbar]{proposition}
\surroundwithmdframed[style=leftbar]{lemma}
\surroundwithmdframed[style=leftbar]{corollary}
\surroundwithmdframed[style=leftbar]{axiom}
\surroundwithmdframed[style=leftbar]{axiom*}

\theoremstyle{definition}

\newtheorem*{definition}{Definition}
\surroundwithmdframed[style=leftbar]{definition}

\newtheorem*{slogan}{Slogan}
\newtheorem*{eg}{Example}
\newtheorem*{ex}{Exercise}
\newtheorem*{remark}{Remark}
\newtheorem*{notation}{Notation}
\newtheorem*{convention}{Convention}
\newtheorem*{assumption}{Assumption}
\newtheorem*{question}{Question}
\newtheorem*{answer}{Answer}
\newtheorem*{note}{Note}
\newtheorem*{application}{Application}

%operator macros

%basic
\DeclareMathOperator{\lcm}{lcm}

%matrix
\DeclareMathOperator{\tr}{tr}
\DeclareMathOperator{\Tr}{Tr}
\DeclareMathOperator{\adj}{adj}

%algebra
\DeclareMathOperator{\Hom}{Hom}
\DeclareMathOperator{\End}{End}
\DeclareMathOperator{\id}{id}
\DeclareMathOperator{\im}{im}
\DeclareMathOperator{\coker}{coker}
\DeclarePairedDelimiter{\generation}{\langle}{\rangle}

%groups
\DeclareMathOperator{\sym}{Sym}
\DeclareMathOperator{\sgn}{sgn}
\DeclareMathOperator{\inn}{Inn}
\DeclareMathOperator{\aut}{Aut}
\DeclareMathOperator{\GL}{GL}
\DeclareMathOperator{\SL}{SL}
\DeclareMathOperator{\PGL}{PGL}
\DeclareMathOperator{\PSL}{PSL}
\DeclareMathOperator{\SU}{SU}
\DeclareMathOperator{\UU}{U}
\DeclareMathOperator{\SO}{SO}
\DeclareMathOperator{\OO}{O}
\DeclareMathOperator{\PSU}{PSU}
\DeclareMathOperator{\Sp}{Sp}


%hyperbolic
\DeclareMathOperator{\sech}{sech}

%field, galois heory
\DeclareMathOperator{\ch}{ch}
\DeclareMathOperator{\gal}{Gal}
\DeclareMathOperator{\emb}{Emb}



%ceiling and floor
%https://tex.stackexchange.com/a/118217/26707
\DeclarePairedDelimiter\ceil{\lceil}{\rceil}
\DeclarePairedDelimiter\floor{\lfloor}{\rfloor}


\DeclarePairedDelimiter{\innerproduct}{\langle}{\rangle}

%\DeclarePairedDelimiterX{\norm}[1]{\lVert}{\rVert}{#1}
\DeclarePairedDelimiter{\norm}{\lVert}{\rVert}



%Dirac notation
%TODO: rewrite for variable number of arguments
\DeclarePairedDelimiterX{\braket}[2]{\langle}{\rangle}{#1 \delimsize\vert #2}
\DeclarePairedDelimiterX{\braketthree}[3]{\langle}{\rangle}{#1 \delimsize\vert #2 \delimsize\vert #3}

\DeclarePairedDelimiter{\bra}{\langle}{\rvert}
\DeclarePairedDelimiter{\ket}{\lvert}{\rangle}




%macros

%general

%divide, not divide
\newcommand*{\divides}{\mid}
\newcommand*{\ndivides}{\nmid}
%vector, i.e. mathbf
%https://tex.stackexchange.com/a/45746/26707
\newcommand*{\V}[1]{{\ensuremath{\symbf{#1}}}}
%closure
\newcommand*{\cl}[1]{\overline{#1}}
%conjugate
\newcommand*{\conj}[1]{\overline{#1}}
%set complement
\newcommand*{\stcomp}[1]{\overline{#1}}
\newcommand*{\compose}{\circ}
\newcommand*{\nto}{\nrightarrow}
\newcommand*{\p}{\partial}
%embed
\newcommand*{\embed}{\hookrightarrow}
%surjection
\newcommand*{\surj}{\twoheadrightarrow}
%power set
\newcommand*{\powerset}{\mathcal{P}}

%matrix
\newcommand*{\matrixring}{\mathcal{M}}

%groups
\newcommand*{\normal}{\trianglelefteq}
%rings
\newcommand*{\ideal}{\trianglelefteq}

%fields
\renewcommand*{\C}{{\mathbb{C}}}
\newcommand*{\R}{{\mathbb{R}}}
\newcommand*{\Q}{{\mathbb{Q}}}
\newcommand*{\Z}{{\mathbb{Z}}}
\newcommand*{\N}{{\mathbb{N}}}
\newcommand*{\F}{{\mathbb{F}}}
%not really but I think this belongs here
\newcommand*{\A}{{\mathbb{A}}}

%asymptotic
\newcommand*{\bigO}{O}
\newcommand*{\smallo}{o}

%probability
\newcommand*{\prob}{\mathbb{P}}
\newcommand*{\E}{\mathbb{E}}

%vector calculus
\newcommand*{\gradient}{\V \nabla}
\newcommand*{\divergence}{\gradient \cdot}
\newcommand*{\curl}{\gradient \cdot}

%logic
\newcommand*{\yields}{\vdash}
\newcommand*{\nyields}{\nvdash}

%differential geometry
\renewcommand*{\H}{\mathbb{H}}
\newcommand*{\transversal}{\pitchfork}
\renewcommand{\d}{\mathrm{d}} % exterior derivative

%number theory
\newcommand*{\legendre}[2]{\genfrac{(}{)}{}{}{#1}{#2}}%Legendre symbol

%algebraic geometry
\DeclareMathOperator{\Spec}{Spec}
\DeclareMathOperator{\Proj}{Proj}

\newcommand{\ccl}{{\mathcal C}}

\begin{document}

\begin{titlepage}
  \begin{center}
    \includegraphics[width=0.6\textwidth]{logo.jpg}\par
    \vspace{1cm}
    {\scshape\huge Mathamatics Tripos \par}
    \vspace{2cm}
    {\huge Part \npart \par}
    \vspace{0.6cm}
    {\Huge \bfseries \ntitle \par}
    \vspace{1.2cm}
    {\Large\nterm, \nyear \par}
    \vspace{2cm}
    
    {\large \emph{Lectures by } \par}
    \vspace{0.2cm}
    {\Large \scshape \nlecturer}
    
    \vspace{0.5cm}
    {\large \emph{Notes by }\par}
    \vspace{0.2cm}
    {\Large \scshape \href{mailto:\nauthoremail}{\nauthor}}
 \end{center}
\end{titlepage}

\tableofcontents

\setcounter{section}{-1}

\section{Introduction}

Representation theory is the theory of how \emph{groups} act as groups on \emph{vector spaces}. Here
\begin{enumerate}
\item groups are either finite or compact topological groups,
\item vector spaces are finite-diemnsional and usually over \(\C\),
\item actions are linear. 
\end{enumerate}

\section{Group actions}

\begin{notation}\leavevmode
  \begin{enumerate}
  \item \(\F\) is a field, usually \(\C\), \(\R\) or \(\Q\). In particular \(\F\) is a field of characteristic zero. Thus in this course we mostly deal with what is known as \emph{ordinary representation theory}. Sometimes \(\F = \F_p\) or \(\cl{\F_p}\), and the study of which is known as \emph{modular representation theory}.
  \item \(V\) is a vector space over \(\F\) and will always be finite-dimensional.
  \item \(\GL(V) = \{\theta: V \to V \text{ linear invertible}\}\).
  \end{enumerate}
\end{notation}

\subsection{Review of linear algebra}

If \(\dim_\F V = n\), choose basis \(e_1, \dots, e_n\) over \(\F\) so we can identify it with \(\F^n\). Then \(\theta \in \GL(V)\) correponds to an \(n \times n\) matrix \(A_\theta = (a_{ij})\), where
\[
  \theta(e_j) = \sum_i a_{ij} e_i
\]
for \(1 \leq j \leq n\). In fact we have \(A_\theta \in \GL_n(\F)\), the \emph{general linear group}. Thus

\begin{proposition}
  The map
  \begin{align*}
    \GL(V) &\to \GL_n(\F) \\
    \theta &\mapsto A_\theta
  \end{align*}
  is a group isomorphism.
\end{proposition}

\begin{proof}
  Check \(A_{\theta_1\theta_2} = A_{\theta_1}A_{\theta_2}\) and bijectivity.
\end{proof}

Choosing a different basis gives different isomorphism to \(\GL_n(\F)\), but

\begin{proposition}
  Matrices \(A_1, A_2\) represent the same element of \(\GL(V)\) with respect to different basis if and only if they are \emph{conjugate} or \emph{similar}, i.e.\ exists \(X \in \GL_n(\F)\) such that \(A_2 = XA_1X^{-1}\).
\end{proposition}

Recall that the \emph{trace} of a matrix \(A\) is
\[
  \tr A = \sum_i a_{ii}.
\]

\begin{proposition}
  As \(\tr(XAX^{-1}) = \tr A\) we can define
  \[
    \tr \theta = \tr(A_\theta)
  \]
  which is \emph{independent} of the basis chosen.
\end{proposition}

Some notes on diagonalisation:

\begin{eg}
  Let \(\alpha \in \GL(V)\) where \(V\) is a finite-dimensioanl vector space over \(\C\) with \(\alpha^m = \id\) for some \(m\). Then \(\alpha\) is diagonalisable.
\end{eg}

\begin{proposition}
  Let \(V\) a finite-dimensional vector space over \(\C\) and \(\alpha \in \End(V)\). Then \(\alpha\) is diagonalisable if and only if there exists a polynomial \(f\) with distinct linear factors with \(f(\alpha) = 0\).
\end{proposition}

\begin{remark}
  In the previous example take \(f(X) = X^m - 1 = \prod_{j = 0}^{m - 1} (X - \omega^j)\) where \(\omega = e^{\frac{2\pi i}{m}}\).
\end{remark}

\begin{proposition}
  A finite family of commuting separately diagonalisable non-singular transformations of a \(\C\)-vector space can be simultaneously diagonalised.
\end{proposition}

\subsection{Basic group theory}

We have an ample supply of basic groups:
\begin{enumerate}
\item symmetric group \(S_n = \sym(X)\) on a set \(X = \{1, \dots, n\}\) is the set of all permutations of \(X\). \(|S_n| = n!\).
\item alternating group \(A_n\) with \(|A_n| = \frac{n!}{2}\) consists of all even permutations.
\item cyclic group of order \(n\): \(C_n = \langle x: x^m = 1\rangle\). For example \((\Z/m\Z, +)\). It's also
  \begin{itemize}
  \item the group of \(m\)th root of unity in \(\C\) (which embeds to \(\GL_1(\C) = \C^*\)),
  \item the group of rotations, centre \(0\) of a regular \(m\)-gon in \(\R^2\) (which embeds to \(\GL_2(\R)\)).
  \end{itemize}
\item diahedral groups: \(D_{2m} = \langle x, y: x^m = y^2 = 1, yxy^{-1} = x^{-1} \rangle\) of order \(2m\). Think of this as set of rotations and reflections preserving a regular \(m\)-gon.
\item quaternion group: \(Q_8 = \langle x, y: x^4 = 1, y^2 = x^2, yxy^{-1} = x^{-1} \rangle\) of order \(8\). In \(\GL_2(\C)\), can put
  \[
    i =
    \begin{pmatrix}
      i & 0 \\
      0 & -i
    \end{pmatrix}
    \quad
    j =
    \begin{pmatrix}
      0 & 1 \\
      -1 & 0
    \end{pmatrix}
    \quad
     k =
    \begin{pmatrix}
      0 & i \\
      i & 0
    \end{pmatrix}
  \]
  then \(Q_8 = \{\pm i, \pm j, \pm k, \pm I_2\}\).
\end{enumerate}

\begin{definition}
  The \emph{conjugacy class} of \(g \in G\) is
  \[
    \ccl_G(g) = \{xgx^{-1}: x \in G\}.
  \]

  Then
  \[
    |\ccl_G(g)| = |G: C_G(g)|
  \]
  where \(C_g(g) = \{x \in g: xg = gx\}\) is the \emph{centraliser} of \(g\) in \(G\).
\end{definition}




\printindex
\end{document}

% Books: James-Liebeck [JL]
% Alperin & Bell
% Thomas: Representations of finite and Lie groups
% online notes: SM's notes, C. Teleman
% Webb: A course in finite group representation theory
% Curtis: Pioneers of representation theory