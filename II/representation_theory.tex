\documentclass[a4paper]{article}

\def\npart{II}

\def\ntitle{Representation Theory}
\def\nlecturer{S.\ Martin}

\def\nterm{Lent}
\def\nyear{2019}

\ifx \nauthor\undefined
  \def\nauthor{Qiangru Kuang}
\else
\fi

\ifx \ntitle\undefined
  \def\ntitle{Template}
\else
\fi

\ifx \nauthoremail\undefined
  \def\nauthoremail{qk206@cam.ac.uk}
\else
\fi

\ifx \ndate\undefined
  \def\ndate{\today}
\else
\fi

\title{\ntitle}
\author{\nauthor}
\date{\ndate}

%\usepackage{microtype}
\usepackage{mathtools}
\usepackage{amsthm}
\usepackage{stmaryrd}%symbols used so far: \mapsfrom
\usepackage{empheq}
\usepackage{amssymb}
\let\mathbbalt\mathbb
\let\pitchforkold\pitchfork
\usepackage{unicode-math}
\let\mathbb\mathbbalt%reset to original \mathbb
\let\pitchfork\pitchforkold

\usepackage{imakeidx}
\makeindex[intoc]

%to address the problem that Latin modern doesn't have unicode support for setminus
%https://tex.stackexchange.com/a/55205/26707
\AtBeginDocument{\renewcommand*{\setminus}{\mathbin{\backslash}}}
\AtBeginDocument{\renewcommand*{\models}{\vDash}}%for \vDash is same size as \vdash but orginal \models is larger
\AtBeginDocument{\let\Re\relax}
\AtBeginDocument{\let\Im\relax}
\AtBeginDocument{\DeclareMathOperator{\Re}{Re}}
\AtBeginDocument{\DeclareMathOperator{\Im}{Im}}
\AtBeginDocument{\let\div\relax}
\AtBeginDocument{\DeclareMathOperator{\div}{div}}

\usepackage{tikz}
\usetikzlibrary{automata,positioning}
\usepackage{pgfplots}
%some preset styles
\pgfplotsset{compat=1.15}
\pgfplotsset{centre/.append style={axis x line=middle, axis y line=middle, xlabel={$x$}, ylabel={$y$}, axis equal}}
\usepackage{tikz-cd}
\usepackage{graphicx}
\usepackage{newunicodechar}

\usepackage{fancyhdr}

\fancypagestyle{mypagestyle}{
    \fancyhf{}
    \lhead{\emph{\nouppercase{\leftmark}}}
    \rhead{}
    \cfoot{\thepage}
}
\pagestyle{mypagestyle}

\usepackage{titlesec}
\newcommand{\sectionbreak}{\clearpage} % clear page after each section
\usepackage[perpage]{footmisc}
\usepackage{blindtext}

%\reallywidehat
%https://tex.stackexchange.com/a/101136/26707
\usepackage{scalerel,stackengine}
\stackMath
\newcommand\reallywidehat[1]{%
\savestack{\tmpbox}{\stretchto{%
  \scaleto{%
    \scalerel*[\widthof{\ensuremath{#1}}]{\kern-.6pt\bigwedge\kern-.6pt}%
    {\rule[-\textheight/2]{1ex}{\textheight}}%WIDTH-LIMITED BIG WEDGE
  }{\textheight}% 
}{0.5ex}}%
\stackon[1pt]{#1}{\tmpbox}%
}

%\usepackage{braket}
\usepackage{thmtools}%restate theorem
\usepackage{hyperref}

% https://en.wikibooks.org/wiki/LaTeX/Hyperlinks
\hypersetup{
    %bookmarks=true,
    unicode=true,
    pdftitle={\ntitle},
    pdfauthor={\nauthor},
    pdfsubject={Mathematics},
    pdfcreator={\nauthor},
    pdfproducer={\nauthor},
    pdfkeywords={math maths \ntitle},
    colorlinks=true,
    linkcolor={red!50!black},
    citecolor={blue!50!black},
    urlcolor={blue!80!black}
}

\usepackage{cleveref}



% TODO: mdframed often gives bad breaks that cause empty lines. Would like to switch to tcolorbox.
% The current workaround is to set innerbottommargin=0pt.

%\usepackage[theorems]{tcolorbox}





\usepackage[framemethod=tikz]{mdframed}
\mdfdefinestyle{leftbar}{
  %nobreak=true, %dirty hack
  linewidth=1.5pt,
  linecolor=gray,
  hidealllines=true,
  leftline=true,
  leftmargin=0pt,
  innerleftmargin=5pt,
  innerrightmargin=10pt,
  innertopmargin=-5pt,
  % innerbottommargin=5pt, % original
  innerbottommargin=0pt, % temporary hack 
}
%\newmdtheoremenv[style=leftbar]{theorem}{Theorem}[section]
%\newmdtheoremenv[style=leftbar]{proposition}[theorem]{proposition}
%\newmdtheoremenv[style=leftbar]{lemma}[theorem]{Lemma}
%\newmdtheoremenv[style=leftbar]{corollary}[theorem]{corollary}

\newtheorem{theorem}{Theorem}[section]
\newtheorem{proposition}[theorem]{Proposition}
\newtheorem{lemma}[theorem]{Lemma}
\newtheorem{corollary}[theorem]{Corollary}
\newtheorem{axiom}[theorem]{Axiom}
\newtheorem*{axiom*}{Axiom}

\surroundwithmdframed[style=leftbar]{theorem}
\surroundwithmdframed[style=leftbar]{proposition}
\surroundwithmdframed[style=leftbar]{lemma}
\surroundwithmdframed[style=leftbar]{corollary}
\surroundwithmdframed[style=leftbar]{axiom}
\surroundwithmdframed[style=leftbar]{axiom*}

\theoremstyle{definition}

\newtheorem*{definition}{Definition}
\surroundwithmdframed[style=leftbar]{definition}

\newtheorem*{slogan}{Slogan}
\newtheorem*{eg}{Example}
\newtheorem*{ex}{Exercise}
\newtheorem*{remark}{Remark}
\newtheorem*{notation}{Notation}
\newtheorem*{convention}{Convention}
\newtheorem*{assumption}{Assumption}
\newtheorem*{question}{Question}
\newtheorem*{answer}{Answer}
\newtheorem*{note}{Note}
\newtheorem*{application}{Application}

%operator macros

%basic
\DeclareMathOperator{\lcm}{lcm}

%matrix
\DeclareMathOperator{\tr}{tr}
\DeclareMathOperator{\Tr}{Tr}
\DeclareMathOperator{\adj}{adj}

%algebra
\DeclareMathOperator{\Hom}{Hom}
\DeclareMathOperator{\End}{End}
\DeclareMathOperator{\id}{id}
\DeclareMathOperator{\im}{im}
\DeclareMathOperator{\coker}{coker}
\DeclarePairedDelimiter{\generation}{\langle}{\rangle}

%groups
\DeclareMathOperator{\sym}{Sym}
\DeclareMathOperator{\sgn}{sgn}
\DeclareMathOperator{\inn}{Inn}
\DeclareMathOperator{\aut}{Aut}
\DeclareMathOperator{\GL}{GL}
\DeclareMathOperator{\SL}{SL}
\DeclareMathOperator{\PGL}{PGL}
\DeclareMathOperator{\PSL}{PSL}
\DeclareMathOperator{\SU}{SU}
\DeclareMathOperator{\UU}{U}
\DeclareMathOperator{\SO}{SO}
\DeclareMathOperator{\OO}{O}
\DeclareMathOperator{\PSU}{PSU}
\DeclareMathOperator{\Sp}{Sp}


%hyperbolic
\DeclareMathOperator{\sech}{sech}

%field, galois heory
\DeclareMathOperator{\ch}{ch}
\DeclareMathOperator{\gal}{Gal}
\DeclareMathOperator{\emb}{Emb}



%ceiling and floor
%https://tex.stackexchange.com/a/118217/26707
\DeclarePairedDelimiter\ceil{\lceil}{\rceil}
\DeclarePairedDelimiter\floor{\lfloor}{\rfloor}


\DeclarePairedDelimiter{\innerproduct}{\langle}{\rangle}

%\DeclarePairedDelimiterX{\norm}[1]{\lVert}{\rVert}{#1}
\DeclarePairedDelimiter{\norm}{\lVert}{\rVert}



%Dirac notation
%TODO: rewrite for variable number of arguments
\DeclarePairedDelimiterX{\braket}[2]{\langle}{\rangle}{#1 \delimsize\vert #2}
\DeclarePairedDelimiterX{\braketthree}[3]{\langle}{\rangle}{#1 \delimsize\vert #2 \delimsize\vert #3}

\DeclarePairedDelimiter{\bra}{\langle}{\rvert}
\DeclarePairedDelimiter{\ket}{\lvert}{\rangle}




%macros

%general

%divide, not divide
\newcommand*{\divides}{\mid}
\newcommand*{\ndivides}{\nmid}
%vector, i.e. mathbf
%https://tex.stackexchange.com/a/45746/26707
\newcommand*{\V}[1]{{\ensuremath{\symbf{#1}}}}
%closure
\newcommand*{\cl}[1]{\overline{#1}}
%conjugate
\newcommand*{\conj}[1]{\overline{#1}}
%set complement
\newcommand*{\stcomp}[1]{\overline{#1}}
\newcommand*{\compose}{\circ}
\newcommand*{\nto}{\nrightarrow}
\newcommand*{\p}{\partial}
%embed
\newcommand*{\embed}{\hookrightarrow}
%surjection
\newcommand*{\surj}{\twoheadrightarrow}
%power set
\newcommand*{\powerset}{\mathcal{P}}

%matrix
\newcommand*{\matrixring}{\mathcal{M}}

%groups
\newcommand*{\normal}{\trianglelefteq}
%rings
\newcommand*{\ideal}{\trianglelefteq}

%fields
\renewcommand*{\C}{{\mathbb{C}}}
\newcommand*{\R}{{\mathbb{R}}}
\newcommand*{\Q}{{\mathbb{Q}}}
\newcommand*{\Z}{{\mathbb{Z}}}
\newcommand*{\N}{{\mathbb{N}}}
\newcommand*{\F}{{\mathbb{F}}}
%not really but I think this belongs here
\newcommand*{\A}{{\mathbb{A}}}

%asymptotic
\newcommand*{\bigO}{O}
\newcommand*{\smallo}{o}

%probability
\newcommand*{\prob}{\mathbb{P}}
\newcommand*{\E}{\mathbb{E}}

%vector calculus
\newcommand*{\gradient}{\V \nabla}
\newcommand*{\divergence}{\gradient \cdot}
\newcommand*{\curl}{\gradient \cdot}

%logic
\newcommand*{\yields}{\vdash}
\newcommand*{\nyields}{\nvdash}

%differential geometry
\renewcommand*{\H}{\mathbb{H}}
\newcommand*{\transversal}{\pitchfork}
\renewcommand{\d}{\mathrm{d}} % exterior derivative

%number theory
\newcommand*{\legendre}[2]{\genfrac{(}{)}{}{}{#1}{#2}}%Legendre symbol

%algebraic geometry
\DeclareMathOperator{\Spec}{Spec}
\DeclareMathOperator{\Proj}{Proj}

\newcommand{\ccl}{{\mathcal C}} % conjugacy class
\newcommand*{\ip}{\innerproduct} % inner product
\DeclareMathOperator{\Res}{Res} % restriction
\DeclareMathOperator{\Ind}{Ind} % induction

\theoremstyle{definition}
\newtheorem*{fact}{Fact}


\begin{document}

\begin{titlepage}
  \begin{center}
    \includegraphics[width=0.6\textwidth]{logo.jpg}\par
    \vspace{1cm}
    {\scshape\huge Mathamatics Tripos \par}
    \vspace{2cm}
    {\huge Part \npart \par}
    \vspace{0.6cm}
    {\Huge \bfseries \ntitle \par}
    \vspace{1.2cm}
    {\Large\nterm, \nyear \par}
    \vspace{2cm}
    
    {\large \emph{Lectures by } \par}
    \vspace{0.2cm}
    {\Large \scshape \nlecturer}
    
    \vspace{0.5cm}
    {\large \emph{Notes by }\par}
    \vspace{0.2cm}
    {\Large \scshape \href{mailto:\nauthoremail}{\nauthor}}
 \end{center}
\end{titlepage}

\tableofcontents

\setcounter{section}{-1}

\section{Introduction}

Representation theory is the theory of how \emph{groups} act as groups on \emph{vector spaces}. Here
\begin{enumerate}
\item groups are either finite or compact topological groups,
\item vector spaces are finite-diemnsional and usually over \(\C\),
\item actions are linear. 
\end{enumerate}

\section{Group actions}

\begin{notation}\leavevmode
  \begin{enumerate}
  \item \(\F\) is a field, usually \(\C\), \(\R\) or \(\Q\). In particular \(\F\) is a field of characteristic zero. Thus in this course we mostly deal with what is known as \emph{ordinary representation theory}. Sometimes \(\F = \F_p\) or \(\cl{\F_p}\), and the study of which is known as \emph{modular representation theory}.
  \item \(V\) is a vector space over \(\F\) and will always be finite-dimensional.
  \item \(\GL(V) = \{\theta: V \to V \text{ linear invertible}\}\).
  \end{enumerate}
\end{notation}

\subsection{Review of linear algebra}

If \(\dim_\F V = n\), choose basis \(e_1, \dots, e_n\) over \(\F\) so we can identify it with \(\F^n\). Then \(\theta \in \GL(V)\) correponds to an \(n \times n\) matrix \(A_\theta = (a_{ij})\), where
\[
  \theta(e_j) = \sum_i a_{ij} e_i
\]
for \(1 \leq j \leq n\). In fact we have \(A_\theta \in \GL_n(\F)\), the \emph{general linear group}. Thus

\begin{proposition}
  The map
  \begin{align*}
    \GL(V) &\to \GL_n(\F) \\
    \theta &\mapsto A_\theta
  \end{align*}
  is a group isomorphism.
\end{proposition}

\begin{proof}
  Check \(A_{\theta_1\theta_2} = A_{\theta_1}A_{\theta_2}\) and bijectivity.
\end{proof}

Choosing a different basis gives different isomorphism to \(\GL_n(\F)\), but

\begin{proposition}
  Matrices \(A_1, A_2\) represent the same element of \(\GL(V)\) with respect to different basis if and only if they are \emph{conjugate} or \emph{similar}, i.e.\ exists \(X \in \GL_n(\F)\) such that \(A_2 = XA_1X^{-1}\).
\end{proposition}

Recall that the \emph{trace} of a matrix \(A\) is
\[
  \tr A = \sum_i a_{ii}.
\]

\begin{proposition}
  As \(\tr(XAX^{-1}) = \tr A\) we can define
  \[
    \tr \theta = \tr(A_\theta)
  \]
  which is \emph{independent} of the basis chosen.
\end{proposition}

Some notes on diagonalisation:

\begin{eg}
  Let \(\alpha \in \GL(V)\) where \(V\) is a finite-dimensioanl vector space over \(\C\) with \(\alpha^m = \id\) for some \(m\). Then \(\alpha\) is diagonalisable.
\end{eg}

\begin{proposition}
  Let \(V\) a finite-dimensional vector space over \(\C\) and \(\alpha \in \End(V)\). Then \(\alpha\) is diagonalisable if and only if there exists a polynomial \(f\) with distinct linear factors with \(f(\alpha) = 0\).
\end{proposition}

\begin{remark}
  In the previous example take \(f(X) = X^m - 1 = \prod_{j = 0}^{m - 1} (X - \omega^j)\) where \(\omega = e^{\frac{2\pi i}{m}}\).
\end{remark}

\begin{proposition}
  \label{prop:simultaneous diagonalisable}
  A finite family of commuting separately diagonalisable non-singular transformations of a \(\C\)-vector space can be simultaneously diagonalised.
\end{proposition}

\subsection{Basic group theory}

We have an ample supply of basic groups:
\begin{enumerate}
\item symmetric group \(S_n = \sym(X)\) on a set \(X = \{1, \dots, n\}\) is the set of all permutations of \(X\). \(|S_n| = n!\).
\item alternating group \(A_n\) with \(|A_n| = \frac{n!}{2}\) consists of all even permutations.
\item cyclic group of order \(n\): \(C_n = \langle x: x^m = 1\rangle\). For example \((\Z/m\Z, +)\). It's also
  \begin{itemize}
  \item the group of \(m\)th root of unity in \(\C\) (which embeds to \(\GL_1(\C) = \C^\times\)),
  \item the group of rotations, centre \(0\) of a regular \(m\)-gon in \(\R^2\) (which embeds to \(\GL_2(\R)\)).
  \end{itemize}
\item diahedral groups: \(D_{2m} = \langle x, y: x^m = y^2 = 1, yxy^{-1} = x^{-1} \rangle\) of order \(2m\). Think of this as set of rotations and reflections preserving a regular \(m\)-gon.
\item quaternion group: \(Q_8 = \langle x, y: x^4 = 1, y^2 = x^2, yxy^{-1} = x^{-1} \rangle\) of order \(8\). In \(\GL_2(\C)\), can put
  \[
    i =
    \begin{pmatrix}
      i & 0 \\
      0 & -i
    \end{pmatrix}
    \quad
    j =
    \begin{pmatrix}
      0 & 1 \\
      -1 & 0
    \end{pmatrix}
    \quad
     k =
    \begin{pmatrix}
      0 & i \\
      i & 0
    \end{pmatrix}
  \]
  then \(Q_8 = \{\pm i, \pm j, \pm k, \pm I_2\}\).
\end{enumerate}

\begin{definition}[conjugacy class, centraliser]\index{conjugacy class}\index{centraliser}
  The \emph{conjugacy class} of \(g \in G\) is
  \[
    \ccl_G(g) = \{xgx^{-1}: x \in G\}.
  \]

  Then
  \[
    |\ccl_G(g)| = |G: C_G(g)|
  \]
  where \(C_g(g) = \{x \in g: xg = gx\}\) is the \emph{centraliser} of \(g\) in \(G\).
\end{definition}

\begin{definition}[group action]\index{group action}
  Let \(G\) be a group and \(X\) be a set. \(G\) \emph{acts} on \(X\) if there exists a map
  \begin{align*}
    G \times X &\to X \\
    (g, x) &\mapsto gx
  \end{align*}
  such that
  \begin{align*}
    1x &= x \text{ for all } x \in X \\
    (gh) x &= g (hx) \text{ for all } g, h \in G, x \in X
  \end{align*}
\end{definition}

\begin{proposition}[permutation representation]\index{permutation representation}
  Given an action of \(G\) on \(X\), we obtain a homomorphism \(\theta: G \to \sym(X)\), called the \emph{permutation representation} of \(G\).
\end{proposition}

\begin{proof}
  For \(g \in G\) the function \(\theta_g: X \to X, x \mapsto gx\) is a permutation of \(X\) (with inverse \(\theta_{g^{-1}}\). Moreover for all \(g_1, g_2 \in G\),
  \[
    \theta_{g_1 g_2} = \theta_{g_1} \theta_{g_2}
  \]
  since \((g_1g_2)x = g_1(g_2x)\) for all \(x \in X\).
\end{proof}

In this course \(X\) is often a finite-dimensional vector space over \(\F\)and the action is required to be \emph{linear}, namely
\begin{align*}
  g(v_1 + v_2) &= gv_1 + gv_2 \\
  g(\lambda v) &= \lambda g(v)
\end{align*}
for all \(v_1, v_2 \in V, g \in G, \lambda \in \F\).

\section{Basic definitions}

Let \(G\) be a finite group, \(\F\) a field.

\begin{definition}[representation]\index{representation}
  Let \(V\) be a finite-dimensional vector space over \(\F\). A \emph{(linear) representation} of \(G\) on \(V\) is a group homomorphism
  \[
    \rho = \rho_V: G \to \GL(V).
  \]
  Write \(\rho_g\) for \(\rho_V(g)\).
\end{definition}
So for each \(g \in G, \rho_g \in \GL(V), \rho_1 = \id\) and \(\rho_{g_1g_2} = \rho_{g_1}\rho_{g_2}, \rho_{g_1^{-1}} = \rho_{g_1}^{-1}\).

The \emph{dimension}\index{dimension} or \emph{degree}\index{degree} of \(\rho\) is \(\dim_\F V\).

Reall that \(\ker \rho \normal G\) and \(G / \ker \rho \cong \rho(G) \leq \GL(V)\). We say \(\rho\) is \emph{faithful}\index{faithful} if \(\ker \rho = \{1\}\).

We repeat what we said in introduction, namely the correspondence between group representation and group action:

\begin{definition}[linear action]\index{action!linear}
  \(G\) acts \emph{linearly} on \(V\) if ther exists a linear action \(G \times V \to V, (g, v) \mapsto gv\) such that
  \begin{align*}
    (g_1g_2) v &= g_1(g_2v), 1 v = v \\
    g(v_1 + v_2) &= gv_1 gv_2, g(\lambda v) = \lambda g(v)
  \end{align*}
\end{definition}

Now if \(G\) acts on \(V\), the map
\begin{align*}
  G &\to \GL(V) \\
  g &\mapsto \rho_g
\end{align*}
with \(\rho_g: v \mapsto gv\) is a representation. Conversely, given a representation \(G \to \GL(V)\) we have a linear action of \(G\) on \(V\) via
\[
  gv = \rho(g)(v).
\]

\begin{remark}
  We also say that \(V\) is a \emph{\(G\)-space}\index{\(G\)-space} or that \(V\) is a \emph{\(G\)-module}\index{\(G\)-module}. This use of ``module'' might seen unconventional but if fact if you define the \emph{group algebra}\index{group algebra}
  \[
    \F G = \left\{ \sum_{g\in G} \alpha_g g: \alpha_g \in \F \right\}
  \]
  with natural addition an multiplication, then \(V\) is an \(\F G\)-module. \(\F G\) is an example of \emph{\(\F\)-algebra}, i.e.\ a ring which is also an \(\F\)-module such that multiplication is bilinear.
\end{remark}

If we bring in a basis for \(V\), we get yet another equivalent definition:

\begin{definition}[matrix representation]\index{matrix representation}
  \(R\) is a \emph{matrix representation} of \(G\) of degree \(n\) if \(R\) is a homomorphism \(G \to \GL_n(\F)\).
\end{definition}

Given a linear representation \(\rho: G \to \GL(V)\) with \(\dim_FV = n\), fix a basis \(\mathcal B\) then we get a matrix representation
\begin{align*}
  G &\to \GL_n(\F) \\
  g &\mapsto [\rho(g)]_{\mathcal B}
\end{align*}
Conversely, given a matrix representation \(R: G \to \GL_n(\F)\), you get a linear representation
\begin{align*}
  \rho: G &\to \GL(\F^n) \\
  g &\mapsto \rho_g
\end{align*}
via \(\rho_g(v) = R_g(v)\).

\begin{eg}
  Given any group \(G\), take \(V = \F\) (the \(1\) dimensional space) and
  \begin{align*}
    \rho: G &\to \GL(V) \\
    g &\mapsto \id_V
  \end{align*}
  is known as the \emph{trivial representation}\index{representation!trivial}. \(\deg \rho = 1\).
\end{eg}

\begin{eg}
  Let \(G = C_4 = \langle x: x^4 = 1 \rangle\). Take \(\F = \C\) and let \(n = 2\). Then \(R: x \mapsto X\) will determine \(x^j \mapsto X^j\) and thus the matrix representation \(R\). We need \(X^4 = I\). We can take
  \begin{itemize}
  \item either \(X\) diagonal: any such with diagonal entries in \(\{\pm 1, \pm i\}\) (16 choices),
  \item or \(X\) is not diagonal: then it will be conjugate to a diagonal (by diagonalisability criterion).
  \end{itemize}
\end{eg}

\subsection{Equivalent representations}

\begin{definition}[\(G\)-homomorphism, \(G\)-isomorphism]\index{\(G\)-homomorphism}\index{intertwining homomorphism}\index{isomorphic}\index{equivalent}
  Fix \(G\) and \(\F\). Let \(V\) and \(V'\) be \(\F\)-vector spaces and \(\rho: G \to \GL(V), \rho': G \to \GL(V')\) be representations of \(G\). The linear map \(\varphi: V \to V'\) is a \emph{\(G\)-homomorphism} or \emph{intertwining homomorphism} if
  \[
    \varphi \rho(g) = \rho'(g) \varphi.
  \]
  In other words, the following diagram commutes:
  \[
    \begin{tikzcd}
      V \ar[r, "\rho_g"] \ar[d, "\varphi"] & V \ar[d, "\varphi"] \\
      V' \ar[r, "\rho_g'"] & V'
    \end{tikzcd}
  \]

  We say \(\varphi\) \emph{intertwines} \(\rho\) and \(\rho'\). Write \(\Hom_G(V, V')\) for the \(\F\)-space of all such.

  \(\varphi\) is a \emph{\(G\)-isomorphism} if \(\varphi\) is also bijective. If such a \(\varphi\) exists, say \(\rho\) and \(\rho'\) are \emph{isomorphic} or \emph{equivalent}. If \(\varphi\) is a \(G\)-isomorphism we can write the intertwining condition as
  \[
    \rho' = \varphi \rho \varphi^{-1}.
  \]
\end{definition}

\begin{lemma}
  Being isomorphic is an equivalence relation on the set of all representations of \(G\) over \(\F\).
\end{lemma}

\begin{proof}
  Exercise.
\end{proof}

\begin{remark}
  If \(\rho\) and \(\rho'\) are isomorphic representation then they have the same dimension. The converse is false:  \(C_4\) has four non-isomorphic \(1\) dimensional representations.
\end{remark}

\begin{remark}
  Given \(G, V, \F\) with \(\dim_\F V = n\) and \(\rho: G \to \GL(V)\), fix a basis \(\mathcal B\) of \(V\). We get an isomorphism
  \begin{align*}
    \varphi: V &\to \F^n \\
    v &\mapsto [v]_{\mathcal B}
  \end{align*}
  And \(\varphi\) gives a representation \(\rho': G \to \GL(\F^n)\) isomorphic to \(\rho\).
\end{remark}

\begin{proposition}\leavevmode
  \begin{enumerate}
  \item Transformations in terms of matrix representatives: \(R: G \to \GL_n(\F), R': G \to \GL_n(\F)\) are \(G\)-isomorphic or \(G\)-equivalent if exists \(X \in \GL_n(\F)\) with
    \[
      R'(g) = XR(g)X^{-1}
    \]
    for all \(g \in G\).
  \item In terms of linear \(G\)-actions, the action of \(G\) on \(V, V'\) are \(G\)-isomorphic if there exists \(\varphi: V \to V'\) such that
    \[
      g\varphi(v) = \varphi(gv)
    \]
    for all \(g \in G, v \in V\).
\end{enumerate}
\end{proposition}

\subsection{Subrepresentation}

\begin{definition}[\(G\)-subspace]\index{\(G\)-subspace}
  Let \(\rho: G \to \GL(V)\) be a representation of \(G\). We say that \(W \leq V\) is a \emph{\(G\)-subspace} if it is a subspace and it is \(\rho(G)\)-invariant, i.e.\ \(\rho_g(W) \subseteq W\) for all \(g \in G\).
\end{definition}

Obviously \(\{0\}\) and \(V\) are \(G\)-subspaces. On the other hand,

\begin{definition}[irreducible/simple representation]\index{representation!irreducible}\index{representation!simple}
  \(\rho\) is said to be \emph{irreducible} or \emph{simple} representation if there are no proper \(G\)-subspaces.
\end{definition}

\begin{eg}
  Any 1 dimensional representation of \(G\) is irreducible. The converse is not true. For example \(D_8\) has a 2 dimensional irreducible representation.
\end{eg}

\begin{definition}[subrepresentation]\index{subrepresentation}
  If \(W\) is a \(G\)-subspace then the corresponding map
  \begin{align*}
    G &\to \GL(W) \\
    g &\mapsto \rho(g)|_W
  \end{align*}
  is a representation of \(G\), known as a \emph{subrepresentation} of \(\rho\).
\end{definition}

\begin{lemma}
  If \(\rho: G \to \GL(V)\) is a representation, \(W\) is a \(G\)-subspace of \(V\) and \(\mathcal B = \{v_1, \dots, v_n\}\) is a basis containing a basis \(\{v_1, \dots, v_m\}\) of \(W\), where \(0 < m \leq n\), then the matrix of \(\rho(g)\) with respect to \(\mathcal B\) has block upper triangular form
  \[
    \begin{pmatrix}
      * & * \\
      0 & *
    \end{pmatrix}
  \]
  for each \(g \in G\).
\end{lemma}

\begin{eg}
  Let \(\F = \C\).
  \begin{enumerate}
  \item Irreducible representation of \(C_4 = \langle x: x^4 = 1\rangle\) are all 1 dimensional and four of them are
    \[
      x \mapsto i, x \mapsto -1, x \mapsto -i, x \mapsto 1.
    \]

    In general \(C_m\) has precisely \(m\) inequivalent complex irreducible representations, all of degree 1. Actually all complex irreducible representations of a \emph{finite abelian group} are 1 dimensional, by simultaneous diagonalisation and primary decomposition. Alternatively, this follows from Schur's lemma\index{Schur's lemma}.
  \item \(G = D_6\): every irreducible \(\C\)-representation has dimension \(\leq 2\). Let \(\rho: G \to \GL(V)\) be an irreducible representation of \(G\). Let \(r\) be a rotation and \(s\) be reflection. Take an eigenvector \(v\) of \(\rho(r)\) so \(\rho(r)v = \lambda v\) for some \(\lambda \in \C, \lambda \neq 0\). Let
    \[
      W = \langle v, \rho(s) v \rangle \leq V.
    \]
    Since
    \begin{align*}
      \rho(s)\rho(s) v &= v \\
      \rho(r)\rho(s) v &= \rho(s)\rho(r)^{-1}v = \lambda^{-1}\rho(s)v
    \end{align*}
    so \(W\) is \(G\)-invariant. Since \(V\) is irreducible \(W = V\).
  \end{enumerate}
\end{eg}

\begin{definition}[(in)decomposable representation, direct sum]\index{representation!decomposable}\index{representation!direct sum}\index{representation!indecomposable}
  We say that \(\rho: G \to \GL(V)\) is \emph{decomposable} if there are proper \(G\)-invariant subspaces \(U, W\) with \(V = U \oplus W\). Say \(\rho\) is the \emph{direct sum} \(\rho_U \oplus \rho_W\). If no such subspaces exist we say \(\rho\) is \emph{indecomposable}.
\end{definition}

\begin{lemma}
  If \(\rho: G \to \GL(V)\) is decomposable, \(\mathcal B = \{v_1, \dots, v_k, w_1, \dots, w_\ell\}\) is a basis of \(V\) consisting of a basis of \(U\) and a basis of \(W\), then \(\rho(g)\) with respect to \(\mathcal B\) is block diagonal for all \(g \in G\).
\end{lemma}

\begin{definition}[direct sum]\index{representation!direct sum}
  Let \(\rho: G \to \GL(V), \rho': G \to \GL(V')\) be two representations. The \emph{direct sum} of \(\rho, \rho'\) is
  \begin{align*}
    \rho \oplus \rho': G &\to \GL(V \oplus V') \\
    (\rho \oplus \rho') (g) (v + v') &= \rho(g) v + \rho'(g) v'
  \end{align*}

  For matrix representations \(R: G \to \GL_n(\F), R': G \to \GL_{n'}(\F)\), define \(R \oplus R': G \to \GL_{n + n'}(\F)\) is given by
  \[
    g \mapsto
    \begin{pmatrix}
      R(g) & 0 \\
      0 & R'(g)
    \end{pmatrix}
  \]
  for all \(g\).
\end{definition}

\section{Complete reducibility and Maschke's theorem}

Given \(G, \F\) as usual.

\begin{definition}[completely reducible/semisimple representation]\index{representation!completely reducible}\index{representation!semisimple}
  A representation \(\rho: G \to \GL(V)\) is \emph{completely reducible} or \emph{semisimple} if it is a direct sum of irreducible representations.
\end{definition}

\begin{remark}
  Irreducible implies completely reducible. The converse is not true. See example sheet 1 question 3.
\end{remark}

From now on take \(G\) to be finite and \(\ch F = 0\) throughout this chapter.

\begin{theorem}[complete reducibility theorem]\index{complete reduciblity theorem}
  \label{thm:complete reducibility theorem}
  Every finite-dimensional representation \(V\) of a finite group over a field of characteristic \(0\) is completely reducible, i.e.\ \(V = V_1 \oplus \dots \oplus V_r\) is a direct sum of representations with each \(V_i\) irreducible.
\end{theorem}

In fact it is enough to prove

\begin{theorem}[Maschke]\index{Maschke's theorem}
  \label{thm:Maschke}
  Suppose \(G\) is finite and \(\rho: G \to \GL(V)\) is a representation with \(V\) finite-dimensional, \(\ch F = 0\). If \(W\) is a \(G\)-subspace of \(V\) then there exists a \(G\)-subspace \(U\) of \(V\) such that \(V = W \oplus U\), a direct sum of \(G\)-subspaces.
\end{theorem}

\begin{proof}
  Let \(W'\) be any complementary subspace of \(W\) in \(V\), i.e.\ \(V = W \oplus W'\). Let \(q: V \to W\) be the projection of \(V\) onto \(W\) along \(W'\), i.e.\ if \(v = w + w'\) then \(q(v) = w\). Define
  \[
    \overline q: v \mapsto \frac{1}{|G|} \sum_{g \in G} g q(g^{-1}(v)),
  \]
  the ``average of \(q\) over \(G\)''. Note that we've dropped the \(\rho\) in \(\rho(g)\) and \(\rho(g^{-1})\) to avoid excessive notations.

  Claim that \(\overline q: V \to W\): for \(v \in V\), \(q(g^{-1}(v)) \in W\) and \(g(W) \subseteq W\). Also \(\overline q(w) = w\) for \(w \in W\) as
  \[
    \overline q(w)
    = \frac{1}{|G|} \sum_{g \in G} g q(g^{-1}w)
    = \frac{1}{|G|} \sum_{g \in G} g (g^{-1}w)
    = \frac{1}{|G|} \sum_{g \in G} w
    = w
  \]
  Thus \(\overline q\) projects \(V\) onto \(W\).

  As \(\overline q\) is a projection we can write \(V = \im \overline q \oplus \ker \overline q = W \oplus \ker \overline q\). Need to show \(\ker \overline q\) is \(G\)-invariant. Note that if \(h \in G\)
  \begin{align*}
    h \overline q(v)
    &= h \frac{1}{|G|} \sum_g g q(g^{-1}v) \\
    &= \frac{1}{|G|} \sum_g hg q(g^{-1}v) \\
    &= \frac{1}{|G|} \sum_g (hg) q((hg)^{-1} hv) \\
    &= \frac{1}{|G|} \sum_g g q(g^{-1}(hv)) \\
    &= \overline q(hv).
  \end{align*}

  Thus if \(v \in \ker \overline q, h \in G\) then
  \[
    h\overline q(v) = 0 = \overline q(hv)
  \]
  so \(hv \in \ker \overline q\). Therefore
  \[
    V = \im \overline q \oplus \ker \overline q = W \oplus \ker \overline q
  \]
  which is a \(G\)-subspace decomposition.
\end{proof}

In fact, we only need \(\ch \F \ndivides |G|\).

\begin{remark}
  Complements are not unique. For example, take \(G = 1\). Then a representation of \(G\) is just a vector space. Take \(V = \C^2\). Then any proper subspace \(W \leq V\) will do.
\end{remark}

\begin{ex}
  Deduce \nameref{thm:complete reducibility theorem} from \nameref{thm:Maschke} by induction on dimension.
\end{ex}

We'll present another proof using inner product. This will generalise easily to compact Lie groups. Take \(\F = \C\).

Recall that for \(V\) a \(\C\)-vector space. \(\ip{\cdot, \cdot}\) is a \emph{Hermitian inner product}\index{inner product} if
\begin{enumerate}
\item \(\ip{w, v} = \conj{\ip{v, w}}\) for all \(v, w\).
\item sesquilinear: linear in second argument.
\item positive definite: \(\ip{v, v} > 0\) if \(v \neq 0\).
\end{enumerate}
Furthermore \(\ip{\cdot, \cdot}\) is \emph{\(G\)-invariant} if
\[
  \ip{gv, gw} = {v, w}
\]
for all \(v, w \in V, g \in G\).

If \(W\) is a \(G\)-invariant subspace of \(V\) (with a \(G\)-invariant inner product) then \(W^\perp\) is also \(G\)-invariant and \(W = W \oplus W^\perp\): enough to show for all \(v \in W^\perp, g \in G\), have \(gv \in W^\perp\). But by definition \(\ip{v, w} = 0\) for all \(w \in W\). Thus by \(G\)-invariance \(\ip{gv, gw} = 0\) for all \(g\). Certainly \(\ip{gv, w'} = 0\) for all \(w' \in W\) as we can choose \(w = g^{-1}w' \in W\). The result thus follows.

Therefore if there is a \(G\)-invariant inner product on any complex \(G\)-space then we get another proof of Maschke's theorem.

\begin{lemma}[Weyl's unitary trick]\index{Weyl's unitary trick}
  Let \(\rho\) be a complex representation of a finite group \(G\) on the \(\C\)-vector space \(V\). Then there is a \(G\)-invariant inner product on \(V\).
\end{lemma}

\begin{proof}
  There exists an inner product on \(V\): take basis \(e_1, \dots, e_n\) and define \((e_i, e_j) = \delta_{ij}\). Extend sesquilinearly. Now define
  \[
    \ip{v, w} = \frac{1}{|G|} \sum_g (gv, gw).
  \]
  Easy exercise that \(\ip{\cdot, \cdot}\) is a \(G\)-invariant inner product. For example for \(G\)-invariance, for all \(h \in G\),
  \begin{align*}
    \ip{hv, hw}
    &= \frac{1}{|G|} \sum_g ((gh)v, (gh)w) \\
    &= \frac{1}{|G|} \sum_{g'} (g'v, g'w) \\
    &= \ip{v, w}
  \end{align*}
\end{proof}

\begin{corollary}
  Every finite subgroup of \(\GL_n(\C)\) is conjugate to a subgroup of \(U(n)\).
\end{corollary}

\begin{proof}
  Example sheet 1 Q5, Q12.
\end{proof}

\begin{definition}[regular representation]\index{representation!regular}
  Recall group algebra of \(G\) is the \(\F\)-space
  \[
    \F G = \text{span} \{e_g: g \in G\}.
  \]
  There is a linear \(G\)-action
  \[
    h . \sum_g a_ge_g = \sum_g a_g e_{hg} = \sum_{g'} a_{h^{-1}g'} e_{g'}.
  \]
  This is known as \emph{regular representation} of \(G\), denoted \(\rho_{\text{reg}}\).
\end{definition}

This is a faithful representation of dimension \(|G|\). We call \(V = \F G\) (sometimes also written \(\F[G]\)) the \emph{regular module}\index{regular module}.

It turns out that every irreducible representation of \(G\) is a subrepresentation of \(\rho_{\text{reg}}\):

\begin{proposition}
  Let \(\rho\) be an irreducible representation of \(G\) over a field of characteristic \(0\). Then \(\rho\) is isomorphic to a subrepresentation of \(\rho_{\text{reg}}\).
\end{proposition}

\begin{proof}
  Let \(\rho: G \to \GL(V)\) be irreducible and let \(v \in V\) nonzero. Consider
  \begin{align*}
    \theta: \F G &\to V \\
    \sum_g a_g e_g &\mapsto \sum_g a_g gv
  \end{align*}
  This is a \(G\)-homomorphism. Now \(V\) is irreducible and \(\im \theta = V\) since \(\im \theta\) is a \(G\)-subspace. Then \(\ker \theta\) is a \(G\)-subspace of \(\F G\). Let \(W\) be a \(G\)-complement of \(\ker \theta\) in \(\F G\). Thus
  \[
    W \cong \F G/ \ker \theta \cong \im \theta = V.
  \]
\end{proof}

More generally,

\begin{definition}[permutation representation]\index{permutation representation}
  Let \(G\) act on a set \(X\). Let \(\F X = \text{span}\{e_x: x \in X\}\) with \(G\) action
  \[
    g . \sum_x a_x e_x = \sum_x a_x e_{gx}
  \]
  so we have a \(G\)-space \(\F X\). The representation \(G \to \GL(\F X)\) is the corresponding \emph{permutation representation}.
\end{definition}

\section{Schur's lemma}

\begin{theorem}[Schur's lemma]\index{Schur's lemma}\leavevmode
  \begin{enumerate}
  \item Assume \(V\) and \(W\) are irreducible \(G\)-spaces (over field \(\F\)). Then any \(G\)-homomorphism \(\theta: V \to W\) is either \(0\) or a \(G\)-isomorphism.
  \item Assume \(\F\) is algebraically closed and let \(V\) be an irreducible \(G\)-space. Then any \(G\)-endomorphism \(V \to V\) is a scalar multiple of the identity map \(1_V\) (a \emph{homothety}).
  \end{enumerate}
\end{theorem}

\begin{proof}\leavevmode
  \begin{enumerate}
  \item Let \(\theta: V \to W\) be a \(G\)-homomorphism. Then \(\ker \theta\) is a \(G\)-subspace of \(V\). Since \(V\) is irreducible either \(\ker \theta = 0\) or \(\ker \theta = V\). Similarly \(\im \theta = 0\) or \(\im \theta = W\). Hence either \(\theta = 0\) or \(\theta\) is injective and surjective.
  \item Since \(\F\) is algebraically closd, \(\theta\) has an eigenvalue \(\lambda\). Then \(\theta - \lambda 1_V\) is a singular \(G\)-endomorphism on \(V\), so must be \(0\).
  \end{enumerate}
\end{proof}

Recall the \(\F\)-space \(\Hom_G(V, W)\) of all \(G\)-homomorphisms \(V \to W\), we can restate Schur's lemma
\begin{corollary}
  If \(V\) and \(W\) are irreducible complex \(G\)-spaces then
  \[
    \dim_\C \Hom_G(V, W) =
    \begin{cases}
      1 & \text{if \(V, W\) are \(G\)-isomorphic} \\
      0 & \text{otherwise}
    \end{cases}
  \]
\end{corollary}

\begin{proof}
  If \(V\) and \(W\) are not isomorphic then the only \(G\)-homomorphism \(V \to W\) is \(0\). Assume \(V \cong_G W\) and \(\theta_1, \theta_2 \in \Hom_G(V, W)\), both nonzero. Then \(\theta_2\) is invertible and \(\theta_2^{-1}\theta_1 \in \End_G(V)\) and nonzero, so \(\theta_2^{-1}\theta_1 = \lambda 1_V\). Then \(\theta_1 = \lambda \theta_2\).
\end{proof}

\begin{corollary}
  If \(G\) has a faithful complex irreducible representation then \(Z(G)\) is cyclic.
\end{corollary}

\begin{remark}
  The converse is false. See example sheet Q10.
\end{remark}

\begin{proof}
  Let \(\rho: G \to \GL(V)\) be a faithful representation over \(\C\). Let \(z \in Z(G)\), then \(\varphi_z: v \mapsto zv\) is a \(G\)-endomorphism, hence multiplication by a scalar, say \(\mu_z\). Then
  \begin{align*}
    Z(G) &\to \C^\times \\
    g &\mapsto \mu_g
  \end{align*}
  is a representation of \(Z(G)\) and is faithful since \(\rho\) is. Thus \(Z(G)\) is isomorphic to a finite subgroup of \(\C^\times\) so cyclic.
\end{proof}

This is our first group theoretic result based on representation theory. This is a recurring theme in representation theory.

\begin{corollary}
  The irreducible \(\C\)-representations of a finite abelian group \(G\) are all 1 dimensional.
\end{corollary}

\begin{proof}
  One can use \Cref{prop:simultaneous diagonalisable} to invoke simultaneous diagonalisation: if \(v\) is an eigenvector for each \(g \in G\) and if \(V\) is irreducible then \(V = \langle v \rangle\).

  Alternatively, let \(V\) be an irreducible representation. Given \(g \in G\), the map
  \begin{align*}
    \theta_g: V &\to V \\
    v &\mapsto gv
  \end{align*}
  is a \(G\)-endomorphism of \(V\). Hence \(\theta_g = \lambda_g 1_V\) for some \(\lambda_g \in \C\). Thus \(gv = \lambda_g v\) for any \(g \in G\). Thus as \(V \neq 0\) is irreducible, \(V = \langle v \rangle\).
\end{proof}

\begin{remark}
  This fails for \(\R\). For example \(C_3\) has two irreducible \(\R\)-representations, one of dimension 1 and one of dimension 2.
\end{remark}

Recall that every finite abelian group \(G\) is isomorphic to a product of cyclic groups. In fact it can be written as product of \(C_{p^\alpha}\) for various primes \(p\) and \(\alpha \geq 1\). The elements are uniquely determined up to order.

\begin{proposition}
  \label{prop:representation of abelian group}
  The finite abelian group \(G \cong C_{n_1} \times \dots \times C_{n_r}\) has precisely \(|G|\) irreducible \(\C\)-representations as described below.
\end{proposition}

\begin{proof}
  Write \(G = \langle x_1 \rangle \times \dots \times \langle x_r \rangle\) where \(|x_j| = n_j\). Suppose \(\rho\) is irreducible so it is 1 dimensional. Let \(\rho(1, \dots, x_j, \dots, 1) = \lambda_j\). Then \(\lambda_j^{n_j} = 1\) so \(\lambda_j\) is an \(n_j\)th root of unity. Now the values \((\lambda_1, \dots, \lambda_r)\) determine \(\rho\), and no two are equivalent.
\end{proof}

Note that however, there is no canonical bijective correspondence between the elements of \(G\) and the representations of \(G\). If you choose an isomorphism \(G \cong C_{a_1} \times \dots C_{a_r}\) then we can identify the two sets, but it depends on the choice of isomorphism.

\subsection{Isotypical decompositions}

We know that in characteristic \(0\), every representation \(V\) of \(G\) decomposes as \(\bigoplus V_i\) where each \(V_i\) is irreducible. How unique is this?

A wishlist of properties:
\begin{enumerate}
\item uniqueness: for each \(V\) there is only one way to decompose \(V = \bigoplus V_i\) with \(V_i\) irreducible.
\item uniqueness of isotypes: for each \(V\) there exist unique subrepresentations \(U_1, \dots, U_k\) such that \(V = \bigoplus U_i\) and if \(V_i \leq U_i, V_j' \leq U_j\) irreducible subrepresentations then \(V_i \cong V_j'\) if and only if \(i = j\).
\item uniqueness of factors: if \(\bigoplus_{i = 1}^k V_i \cong \bigoplus_{i = 1}^{k'} V_i'\) and \(V_i, V_i'\) are irreducible then \(k = k'\) and there exists \(\pi \in S_k\) such that \(V_{\pi(i)}' \cong V_i\).
\end{enumerate}

Evidently 1 is too strong (\(G = 1\) acting on any \(V\) with dimension \(> 1\)). However 2 and 3 do work. We will skip the proof and refer the reader to Teleman \textsection 5. However, we shall discuss how to calculate multiplicities of simples in the isotypes.

\begin{lemma}
  Let \(V, V_1, V_2\) be \(G\)-spaces.
  \begin{enumerate}
  \item \(\Hom_G(V, V_1 \oplus V_2) \cong \Hom_G(V, V_1) \oplus \Hom_G(V, V_2)\).
  \item \(\Hom_G(V_1 \oplus V_2, V) \cong \Hom_G(V_1, V) \oplus \Hom_G(V_2, V)\).
  \end{enumerate}
\end{lemma}

\begin{proof}
  Let \(\pi_i: V_1 \oplus V_2 \to V_i\) be the \(G\)-linear projections in \(V_i\) with kernel \(V_{3 - i}\). Then
  \begin{align*}
    \Hom_G(V, V_1 \oplus V_2) &\to \Hom_G(V, V_1) \oplus \Hom_G(V, V_2) \\
    \varphi &\mapsto (\pi_1 \varphi, \pi_2 \varphi)
  \end{align*}
  has inverse \((\psi_1, \psi_2) \mapsto \psi_1 + \psi_2\).

  Also the map
  \begin{align*}
    \Hom_G(V_1 \oplus V_2, V) &\to \Hom_G(V_1, V) \oplus \Hom_G(V_2, V) \\
    \varphi &\mapsto (\varphi|_{V_1}, \varphi|_{V_2})
  \end{align*}
  has inverse \((\psi_1, \psi_2) \mapsto \psi_1 \pi_1 + \psi_2 \pi_2\).
\end{proof}

\begin{corollary}
  Suppose \(\F\) is algebraically closed and \(V = \bigoplus_{i = 1}^n V_i\) is a decomposition into irreducibles. Then for each irreducible representation \(S\) of \(G\),
  \[
    \# \{j: V_j \cong S\} = \dim \Hom_G(S, V).
  \]
  This is known as the \emph{multiplicity}\index{multiplicity} of \(S\) in \(V\).
\end{corollary}

\begin{proof}
  By induction on \(n\). Obvious for \(n = 0, 1\). For \(n > 1\), write
  \[
    V = (\bigoplus_{i = 1}^{n - 1} V_i) \oplus V_n.
  \]
  Then
  \[
    \dim \Hom_G(S, (\bigoplus_{i = 1}^{n - 1} V_i) \oplus V_n)
    = \dim \Hom_G(S, \bigoplus_{i = 1}^{n - 1} V_i) + \dim \Hom_G(S, V_n)
  \]
  and use Schur's lemma.
\end{proof}

\begin{definition}[canonical decomposition]\index{canonical decomposition}\index{isotypical component}
  A decomposition \(V = \bigoplus W_i\) where each \(W_j\) is isomorphic to \(n_j\) copies of irreducible representation \(S_j\) (each non-isomorphic for each \(j\)) is the \emph{canonical decomposition} or the \emph{decomposition into isotypical components} \(W_j\).
\end{definition}

For \(\F\) closed, the above lemma says that \(n_j = \dim \Hom_G(S_j, V)\), i.e.\ \(n_j\) is detectable at \(G\)-homomorphism level.

\begin{eg}
  Teleman \textsection 5 gives an example on \(D_6\).

  If \(G\) is finite abelian then every complex representation \(V\) of \(G\) has unique isotypical decomposition.
\end{eg}

\section{Character theory}

We want to attach invariants to a representation \(\rho\) of a finite group \(G\) on \(V\). Matrix coefficients of \(\rho(g)\) are basis-dependent so not true invariants. \(\det\) is an invariant but not a very useful one, as lots of inequivalent representations have determinant \(1\). Instead we'll use trace.

Let \(\F = \C\) and let \(\rho = \rho_V: G \to \GL(V)\) be a representation.

\begin{definition}[character]\index{character}\index{character!irreducible}\index{character!faithful}\index{character!faithful, trivial}
  The \emph{character} \(\chi_\rho = \chi_V = \chi\) is defined as
  \begin{align*}
    \chi: G &\to \C \\
    g &\mapsto \tr \rho(g)
  \end{align*}
  The \emph{degree} of \(\chi_V\) is \(\dim V\).

  \(\chi\) is \emph{linear} if \(\dim V = 1\), in which case \(\chi\) is a homomorphism \(G \to \C^\times\). \(\chi\) is \emph{irreducible/faithful/trivial (or principal)} if \(\rho\) is. In the last case we also write \(\chi = 1_G\).
\end{definition}

It turns out that \(\chi\) is a complete invariant in the sense that it determines \(\rho\) up to isomorphism. We'll prove this later.

\begin{theorem}\leavevmode
  \begin{enumerate}
  \item \(\chi_V(1) = \dim V\).
  \item \(\chi_V\) is a \emph{class function}\index{class function}, namely it is conjugation invariant. Thus \(\chi_V\) is constant on the conjugacy class of \(G\).
  \item \(\chi_V(g^{-1}) = \conj{\chi_V(g)}\).
  \item For two representations \(V\) and \(W\),
    \[
      \chi_{V \oplus W} = \chi_V + \chi_W.
    \]
  \end{enumerate}
\end{theorem}

\begin{proof}\leavevmode
  \begin{enumerate}
  \item Clearly \(\tr I_n = n\).
  \item \(\chi(hgh^{-1}) = \tr(R_hR_gR_h^{-1}) = \tr R_g = \chi(g)\).
  \item \(g \in G\) has finite order so diagonalisable so can assume \(\rho(g)\) is represented by diagonal matrix
    \[
      \begin{pmatrix}
        \lambda_1 & \cdots & 0 \\
        & \ddots \\
        0 & \cdots & \lambda_n
      \end{pmatrix}
    \]
    so \(\chi(g) = \sum \lambda_i\). Now \(g^{-1}\) is represented by
    \[
      \begin{pmatrix}
        \lambda_1^{-1} & \cdots & 0 \\
        & \ddots \\
        0 & \cdots & \lambda_n^{-1}
      \end{pmatrix}
    \] 
    hence
    \[
      \chi(g^{-1}) = \sum \lambda_i^{-1} = \sum \conj \lambda_i = \conj{\sum \lambda_i} = \conj{\chi(g)}.
    \]
  \item Suppose \(V = V_1 \oplus V_2, \rho_i: G \to \GL(V_i), \rho: G \to \GL(V)\). Take basis \(\mathcal B = \mathcal B_1 \cup \mathcal B_2\), where \(\mathcal B_1\) and \(\mathcal B_2\) are basis for \(V_1\) and \(V_2\) respectively, of \(V\). With respect to \(\mathcal B\) \(\rho(g)\) has matrix
    \[
      \begin{pmatrix}
        [\rho_1(g)]_{\mathcal B_1} & 0 \\
        0 & [\rho_2(g)]_{\mathcal B_2}
      \end{pmatrix}
    \]
    and so
    \[
      \chi(g) = \tr \rho_1(g) + \tr \rho_2(g) = \chi_1(g) + \chi_2(g).
    \]
  \end{enumerate}
\end{proof}

\begin{remark}
  We'll see later that if \(\chi_1, \chi_2\) are characters of \(G\) then \(\chi_1\chi_2\) is also a character of \(G\) (spoiler: tensor product).
\end{remark}

\begin{lemma}
  \label{lemma:kernel of character}
  Let \(\rho: G \to \GL(V)\) be a (complex) representation affording the character \(\chi\). Then for \(g \in G\), \(|\chi(g)| \leq \chi(1)\) with equality if and only if \(\rho(g) = \lambda I\) for some \(\lambda \in \C\) a root of unity. Moreover \(\chi(g) = \chi(1)\) if and only if \(g \in \ker \rho\). In other words, the \emph{kernel}\index{character!kernel} of \(\chi\) \(\ker \chi\) is
  \[
    \ker \rho = \{g \in G: \chi(g) = \chi(1)\}.
  \]
\end{lemma}

\begin{proof}
  Example sheet 2 Q1.
\end{proof}

\begin{lemma}\leavevmode
  \begin{enumerate}
  \item If \(\chi\) is a (complex irreducible, respectively) character of \(G\) then so is \(\conj \chi\).
  \item If \(\chi\) is a (complex irreducible, respectively) character of \(G\) then so is \(\varepsilon \chi\) for any linear (i.e.\ \(1\) dimensional) character \(\varepsilon\) of \(G\).
  \end{enumerate}
\end{lemma}

\begin{proof}
  If \(R: G \to \GL_n(\C)\) is a (complex irreducible) representation then so is
  \begin{align*}
    R: G &\to \GL_n(\C) \\
    g &\mapsto \conj{R(g)}
  \end{align*}
  Similarly \(r': g \mapsto \varepsilon(g) R(g)\). Check the details.
\end{proof}

\begin{definition}[class function, class number]\index{class function}\index{class number}
  Define
  \[
    \mathcal C(G) = \{f: G \to \C: f(hgh^{-1}) = f(g) \text{ for all } h, g \in \C\},
  \]
  the complex space of \emph{class functions}. It is a \(\C\)-vector space.

  Let \(k = k(G)\) be the \emph{class number} of \(G\), i.e.\ number of conjugacy classes of \(G\). List conjugacy classes as \(\mathcal C_1 = \{1\}, \mathcal C_2, \dots, \mathcal C_k\). Choose \(g_1 = 1, g_2, \dots, g_k\) representatives of the classes. Note that \(\dim \mathcal C(G) = k\), as the characteristic functions \(\delta_j\) of the conjugacy classes form a basis.
\end{definition}

Define a Hermitian inner product on \(\mathcal C(G)\) as follow:
\begin{align*}
  \langle f, f' \rangle
  &= \frac{1}{|G|} \sum_{g \in G} \conj{f(g)} f'(g) \\
  &= \frac{1}{|G|} \sum_{j = 1}^k |\mathcal C_j| \conj{f(g_j)} f'(g_j) \\
  &= \sum_{j = 1}^k \frac{1}{|C_G(g_j)|} \conj{f(g_j)} f'(g_j)
\end{align*}
For characters we have
\[
  \langle \chi, \chi' \rangle
  = \sum_{j = 1}^k \frac{1}{|C_G(g_j)|} \chi(g_j^{-1}) \chi'(g_j)
\]
which is a real symmetric form (in fact we will show it is an integer).

\begin{theorem}[completeness of characters]\index{completeness of characters}
  \label{thm:completeness of character}
  The \(\C\)-irreducible characters of \(G\) form an orthonormal basis of \(\mathcal C(G)\). More precisely,
  \begin{enumerate}
  \item if \(\rho: G \to \GL(V), \rho': G \to \GL(V')\) are irreducible representations of \(G\), affording characters \(\chi\) and \(\chi'\) then
    \[
      \langle \chi, \chi' \rangle =
      \begin{cases}
        1 & \text{if \(\rho, \rho'\) are isomorphic} \\
        0 & \text{otherwise}
      \end{cases}
    \]
    This is called \emph{row orthogonality}\index{row orthogonality}.
  \item each class function of \(G\) is a linear combination of irreducible characters of \(G\).
  \end{enumerate}
\end{theorem}

\begin{proof}
  See chapter 6.
\end{proof}

\begin{corollary}
  Complex representations of finite groups are characterised by their characters.
\end{corollary}

Note the finiteness condition. For counterexample otherwise take \(G = \Z\), \(1 \mapsto I\) and \(1 \mapsto
\begin{psmallmatrix}
  1 & 1 \\
  0 & 1
\end{psmallmatrix}
\).

\begin{proof}
  Let \(G\) be a finite group and \(\rho: G \to \GL(V)\) be a representation affording \(\chi\). By Maschke's theorem \(\rho = m_1 \rho_1 \oplus \dots \oplus m_k \rho_k\) where \(\rho_1, \dots, \rho_k\) are irreducible and \(m_j \geq 0\). Then \(m_j = \langle \chi_j, \chi \rangle\) where \(\chi_j\) is afforded by \(\rho_j\): for \(\chi = m_1 \chi_1 + \dots + m_k \chi_k\) and thus
  \[
    \langle \chi_j, \chi \rangle
    = \langle \chi_j, m_1 \chi_1 + \dots + m_j \chi_j \rangle
    = m_j
  \]
\end{proof}

\begin{corollary}[irreducibility criterion]\index{irreducibility criterion}
  If \(\rho\) is a \(\C\)-representation of \(G\) affording \(\chi\) then \(\rho\) is irreducible if and only if \(\langle \chi, \chi \rangle = 1\).
\end{corollary}

\begin{proof}
  \(\implies\) is row orthogonality. For \(\impliedby\), suppose \(\langle \chi, \chi \rangle = 1\). Complete reducibility says that \(\chi = \sum m_j \chi_j\) where \(\chi_j\)'s are irreducible and \(m_j \geq 0\). Then \(\sum m_j^2 = 1\) so \(\chi = \chi_j\) for some \(j\), so \(\chi\) is irreducible.
\end{proof}

\begin{theorem}
  If the irreducible \(\C\)-representations of \(G\), \(\rho_1, \dots, \rho_k\) have dimensions \(n_1, \dots, n_k\) then
  \[
    |G| = \sum_{i = 1}^k n_i^2.
  \]
\end{theorem}

\begin{proof}
  Recall \(\rho_{\text{reg}}: G \to \GL(\C G)\), the regular representation of \(G\) of dimension \(|G|\). Let \(\pi_{\text{reg}}\) be its character, the \emph{regular character} of \(G\). Note that
  \[
    \pi_{\text{reg}}(g) =
    \begin{cases}
      |G| & g = 1 \\
      0 & \text{otherwise}
    \end{cases}
  \]
  Also claim that \(\pi_{\text{reg}} = \sum_j n_j \chi_j\) with \(n_j = \chi_j(1)\):
  \[
    n_j
    = \langle \pi_{\text{reg}}, \chi_j \rangle
    = \frac{1}{|G|} \sum_{g \in G} \conj{\pi_{\text{reg}}(g)} \chi_j(g)
    = \frac{1}{|G|} |G| \chi_j(1)
    = \chi_j(1).
  \]
\end{proof}

\begin{corollary}
  \label{cor:number of irreducible equals to class number}
  The number of irreducible characters of \(G\) (up to equivalence) equals to the class number.
\end{corollary}

\begin{corollary}
  Elements \(g_1, g_2 \in G\) are conjugate if and only if \(\chi(g_1) = \chi(g_2)\) for all irreducible characters \(\chi\) of \(G\).
\end{corollary}

\begin{proof}
  \(\implies\): characters are class functions. \(\impliedby\): if \(\chi(g_1) = \chi(g_2)\) for all irreducible characters \(\chi\) then \(f(g_1) = f(g_2)\) for all class fucntions of \(G\). In particular this is true for the characteristic function \(\delta\) taking \(1\) on conjugacy class of \(g_1\) and \(0\) otherwise.
\end{proof}

Recall the inner prodcut on \(\ccl(G)\) and the real symmetric form \(\langle \cdot, \cdot \rangle\) for characters.

\begin{definition}\index{character table}\index{character table}
  Let \(G\) be a finite group and \(\F = \C\). The \emph{character table} of \(G\) is the \(k \times k\) matrix \(X = [\chi_i(g_j)]\) where \(1 = \chi_1, \dots, \chi_k\) are the irreducible characters of \(G\) and \(\ccl_1 = \{1\}, \dots, \ccl_k\) are the conjugacy classes with \(g_j \in \ccl_j\).
\end{definition}

\begin{eg}
  \(G = S_3 = D_6 = \langle r, s: r^3 = s^2 = 1, srs^{-1} = r^{-1} \rangle\). The conjugacy classes are
  \[
    \ccl_1 = \{1\}, \ccl_2 = \{s, sr, sr^2\}, \ccl_3 = \{r, r^{-1}\}.
  \]
  Thus from the corollary there are three representations. It is easy to write down two of them: the trivial representation \(1\) and the sign \(S\) of permutation. Think geometrically, it's not hard to come up with a \(2\) dimensional irreducible representation \(W\) of symmetry of an equilateral triangle. \(sr^j\) acts by matrix with eigenvalues \(\pm 1\) so \(\chi(sr^j) = 0\) for all \(j\). \(r^k\) acts by the matrix
  \[
    \begin{pmatrix}
      \cos \frac{2k\pi}{3} & - \sin \frac{2k\pi}{3} \\
      \sin \frac{2k\pi}{3} & \cos \frac{2k\pi}{3}
    \end{pmatrix}
  \]
  so \(\chi(r^k) = 2 \cos \frac{2k\pi}{3} = - 1\) for all \(k\). Thus we have character table
  \[
    \begin{array}{c|ccc}
      & 1 & \ccl_1 & \ccl_2 \\ \hline
      1 & 1 & 1 & 1 \\
      S & 1 & -1 & 1 \\
      W & 2 & 0 & -1
    \end{array}
  \]
  We can do a few sanity checks: the sum of squares of the first column is \(6\), which equals to the order of \(G\). Also
  \[
    \langle \chi_W, \chi_W \rangle
    = \frac{2^2}{6} + \frac{0^2}{2} + \frac{(-1)^2}{3}
    = 1
  \]
  so indeed it is irreducible.
\end{eg}

\section{Proofs of orthogonality}

\index{completeness of characters}

\begin{proof}[Proof of \nameref{thm:completeness of character} 1]
  Fix bases of \(V\) and \(V'\). Write \(R(g), R'(g)\) for matrices of \(\rho(g)\) and \(\rho'(g)\) with respect to these bases respectively. Then
  \[
    \langle \chi', \chi \rangle
    = \frac{1}{|G|} \sum_{g \in G} \chi'(g^{-1}) \chi(g)
    = \frac{1}{|G|} \sum_{\substack{g \in G \\ 1 \leq i, j \leq n}} R'(g^{-1})_{ii} R(g)_{jj}.
  \]
  Let \(\varphi: V \to V'\) be linear and define its ``average''
  \begin{align*}
    \tilde \varphi: V &\to V' \\
    v &\mapsto \frac{1}{|G|} \sum_{g \in G} \rho'(g^{-1}) \varphi \rho(g) v
  \end{align*}
  then \(\tilde \varphi\) is a \(G\)-homomorphism. To see this, if \(h \in G\) then
  \begin{align*}
    \rho'(h^{-1}) \tilde \varphi \rho(h) (v)
    &= \frac{1}{|G|} \sum_{g \in G} \rho'((gh)^{-1}) \varphi \rho(gh) (v) \\
    &= \frac{1}{|G|} \sum_{g' \in G} \rho'(g'^{-1}) \varphi \rho(g') (v) \\
    &= \tilde \varphi(v)
  \end{align*}
  
\paragraph{Case 1: \(\rho, \rho'\) are not isomorphic}

Schur's lemma says \(\tilde \varphi = 0\) for any \(\varphi: V \to V'\) linear. Take \(\varphi = \varepsilon_{\alpha\beta}\), having matrix \(E_{\alpha\beta}\) with respect to our basis with \(0\) everywhere except \(1\) in \((\alpha, \beta)\)th entry. Then
\[
  0 = \tilde \varepsilon_{\alpha\beta}
  = \frac{1}{|G|} \sum_{g \in G} (R'(g^{-1}) E_{\alpha\beta} R(g))_{ij}
\]
so
\[
  \frac{1}{|G|} \sum_{g \in G} R(g^{-1})_{i\alpha} R(g)_{\beta j} = 0
\]
for all \(i, j\). Specialise to \(i = \alpha, j = \beta\) and sum over \(i, j\) to get
\[
  \langle \chi', \chi \rangle = 0.
\]

\paragraph{Case 2: \(\rho, \rho'\) are isomorphic}

\(\chi = \chi'\). Take \(V = V', \rho = \rho'\). If \(\varphi: V \to V\) is linear endomorphism then \(\tilde \varphi \in \End_G(V)\). Now \(\tr \varphi = \tr \tilde \varphi\):
\[
  \tr \tilde \varphi
  = \frac{1}{|G|} \sum_g \tr(\rho(g^{-1}) \varphi \rho(g))
  = \frac{1}{|G|} \sum_g \tr \varphi
  = \tr \varphi
\]
By Schur, \(\tilde \varphi = \lambda \id_V\) for some \(\lambda \in \C\). Then \(\lambda = \frac{1}{n} \tr \varphi\) where \(n\) is the dimension of \(V\).

Let \(\varphi = \varepsilon_{\alpha\beta}\) so \(\tr \varphi = \delta_{\alpha\beta}\). Hence
\[
  \tilde \varphi_{\alpha\beta}
  = \frac{1}{n} \delta_{\alpha\beta} \id
  = \frac{1}{|G|} \sum_g \rho(g^{-1}) \varepsilon_{\alpha\beta} \rho(g)
\]
In terms of matrices, take \((i, j)\)th entry:
\[
  \frac{1}{|G|} \sum_g R(g^{-1})_{i\alpha} R(g)_{\beta j} = \frac{1}{n} \delta_{\alpha\beta} \delta_{ij}
\]
and put \(\alpha = i, \beta = j\) to get
\[
  \frac{1}{|G|} \sum_g R(g^{-1})_{ii} R(g)_{jj} = \frac{1}{n} \delta_{ij}.
\]
Finally sum over \(i, j\) to get
\[
  \langle \chi, \chi \rangle = 1.
\]
\end{proof}

Before prove 2, let's prove column orthogonality, assuming \Cref{cor:number of irreducible equals to class number}.

\begin{corollary}[column orthogonality relations]\index{column orthogonality}
  \[
    \sum_{i = 1}^k \conj{\chi_i(g_j)} \chi_i(g_\ell) = \delta_{j\ell} |C_G(g_j)|.
  \]
\end{corollary}

This has an easy corollary:

\begin{theorem}
  \[
    |G| = \sum_{i = 1}^k \chi_i^2(1).
  \]
\end{theorem}

\begin{proof}
  \[
    \delta_{ij}
    = \langle \chi_i, \chi_j \rangle
    = \sum_{\ell = 1}^k \frac{1}{|C_G(g_\ell)|} \conj{\chi_i(g_\ell)} \chi_j(g_\ell)
  \]
  Consider the character table \(X = (\chi_i(g_j))\). Then
  \[
    \conj X D^{-1} X^t = I_k
  \]
  where
  \[
    D =
    \begin{pmatrix}
      |C_G(g_1)| & & 0 \\
      & \ddots \\
      0 & & |C_G(g_k)|
    \end{pmatrix}
  \]
  Since \(X\) is square, it follows that \(D^{-1} \conj X^t\) is the inverse of \(X\) so \(\conj X^t X = D\).
\end{proof}

\begin{proof}[Proof of \nameref{thm:completeness of character} 2]
  List all the irreducible characters \(\chi_1, \dots, \chi_\ell\) of \(G\). Claim these generate \(\mathcal C(G)\), the \(\C\)-space of class functions on \(G\). It's enough to show that the orthogonal complement to \(\text{span} (\chi_1, \dots, \chi_\ell)\) in \(\mathcal C(G)\) is \(0\). To see this let \(f \in \mathcal C(G)\) with \(\langle f, \chi_j \rangle = 0\) for all \(\chi_j\) irreducible. Let \(\rho: G \to \GL(V)\) be irreducible representation affording \(\chi \in \{\chi_1, \dots, \chi_\ell\}\). Then \(\langle f, \chi \rangle = 0\).

  Consider the \(G\)-endormophism
  \[
    \frac{1}{|G|} \sum_g \conj{f(g)} \rho(g): V \to V
  \]
  so as \(\rho\) is irreducible it must be \(\lambda \id_V\) for some \(\lambda \in \C\). Take trace,
  \[
    n \lambda
    = \tr \frac{1}{|G|} \sum_g \conj{f(g)} \rho(g)
    = \frac{1}{|G|} \sum_g \conj{f(g)} \chi(g)
    = \langle f, \chi \rangle
    = 0
  \]
  so \(\lambda = 0\). Hence \(\sum \conj{f(g)} \rho(g) = 0\) for all representation \(\rho\) by complete reducibility. Take \(\rho = \rho_{\text{reg}}\) so
  \[
    \sum_g \conj{f(g)} \rho_{\text{reg}}(g)(e_1) = \sum_g \conj{f(g)} e_g = 0
  \]
  so \(f(g) = 0\) for all \(g\).
\end{proof}

\section{Permutation representations}

Let \(G\) be a finite group acting on \(X = \{x_1, \dots, x_n\}\). Recall that \(\C X\) is the free \(\C\)-space generated by \(X\). The corresponding permutation representation
\begin{align*}
  \rho_X: G &\to \GL(\C X) \\
  g &\mapsto \rho(g)
\end{align*}
is given by \(\rho(g): e_{x_j} \mapsto e_{gx_j}\). We call \(\rho_X\) the \emph{permutation representation}\index{permutation representation} corresponding to the action of \(G\) on \(X\). Matrices of \(\rho_X(g)\) with respect to basis \(\{e_x\}_{x \in X}\) are permutation matices: \(0\) except for one \(1\) in each row and column and \((\rho(g))_{ij} = 1\) when \(gx_j = x_i\). The corresponding \emph{permutation character}\index{permutation character} \(\pi_X\) is
\[
  \pi_X(g) = |\text{fix}_X(g)| = |\{x \in X: gx = x\}|.
\]

\begin{lemma}
  \(\pi_X\) always contains \(1_G\).
\end{lemma}

\begin{proof}
  \(\text{span}(e_{x_1} + \dots + e_{x_n})\) is a trivial \(G\)-subspace of \(\C X\) with \(G\)-invariant complement \(\text{span}(\sum_{x \in X} a_x e_x: \sum a_x = 0)\).
\end{proof}

\begin{lemma}
  \[
    \langle \pi_X, 1_G \rangle = \# G\text{-orbits of \(G\) on \(X\)}.
  \]
\end{lemma}

\begin{proof}
  If \(X = X_1 \cup \dots \cup X_\ell\) is the disjoint union of orbits then
  \[
    \pi_X = \pi_{X_1} + \dots + \pi_{X_\ell}
  \]
  with \(\pi_{X_j}\) the permutation character of \(G\) on \(X_j\). So prove the lemma, it is enough to show that if \(G\) acts transitively on \(X\) then \(\langle \pi_X, 1 \rangle = 1\). Assume \(G\) is transitive on \(X\),
  \begin{align*}
    \langle \pi_X, 1 \rangle
    &= \frac{1}{|G|} \sum_g \pi_X(g) \\
    &= \frac{1}{|G|} |\{(g, x) \in G \times X: gx = x\}| \\
    &= \frac{1}{|G|} \sum_{x \in X} |G_x| \\
    &= \frac{1}{|G|} |X| |G_X| \\
    &= \frac{1}{|G|} |G| \quad \text{orbit stabiliser} \\
    &= 1
  \end{align*}
  The whole proof can be seen as different ways to write fixed points of \(G\).
\end{proof}

\begin{lemma}
  Let \(G\) act on the sets \(X_1, X_2\). Then \(G\) acts on \(X_1 \times X_2\) via \(g(x_1, x_2) = (gx_1, gx_2)\). The character \(\pi_{X_1 \times X_2} = \pi_{X_1} \pi_{X_2}\) and so
  \[
    \langle \pi_{X_1}, \pi_{X_2} \rangle
    = \#\{\text{orbits of \(G\) on \(X_1 \times X_2\)}\}
  \]
\end{lemma}

\begin{proof}
  If \(g \in G\) then \(\pi_{X_1 \times X_2}(g) = \pi_{X_1}(g) \pi_{X_2}(g)\). And
  \[
    \langle \pi_{X_1}, \pi_{X_2} \rangle
    = \langle \pi_{X_1} \pi_{X_2}, 1 \rangle
    = \langle \pi_{X_1 \times X_2}, 1 \rangle
    = \#\{\text{orbits of \(G\) on \(X_1 \times X_2\)}\}
  \]
\end{proof}

\begin{definition}[\(2\)-transitive]\index{\(2\)-transitive}
  Let \(G\) act on \(X\), \(|X| > 2\). Then \(G\) is \emph{\(2\)-transitive} on \(X\) if \(G\) has exactly two orbits on \(X \times X\): \(\{(x, x): x \in X\}\) and \(\{(x_1, x_2): x_i \in X, x_1 \neq x_2\}\).
\end{definition}

\begin{lemma}
  Let \(G\) act on \(X\) with \(|X| > 2\). Then
  \[
    \pi_X = 1 + \chi
  \]
  with \(\chi\) irreducible if and only if \(G\) is \(2\)-transitive on \(X\).
\end{lemma}

\begin{proof}
  Write
  \[
    \pi_X = m_1 1 + m_2 \chi_2 + \dots + m_\ell \chi_\ell
  \]
  with \(1, \chi_2, \dots, \chi_\ell\) distinct irreducibles and \(m_i \in \N\). Then
  \[
    \langle \pi_X, \pi_X \rangle = \sum_{i = 1}^\ell m_i^2.
  \]
  Hence \(G\) is \(2\)-transitive if and only if \(\ell = 2, m_1 = m_2 = 1\).
\end{proof}

\begin{eg}
  \(S_n\) acting on \(X = \{1, \dots, n\}\) is \(2\)-transitive. Hence \(\pi_X = 1 + \chi\) with \(\chi\) irreducible of degree \(n - 1\). Similar for \(A_n\), \(n > 3\).
\end{eg}

\begin{eg}
  Let's write down the table of \(G = S_4\):
  \[
    \begin{array}{r|c|c|c|c|c}
      & 1 & 3 & 8 & 6 & 6 \\
      & 1 & (12)(34) & (123) & (1234) & (12) \\ \hline
      \chi_1 & 1 & 1 & 1 & 1 & 1 \\
      \sgn = \chi_2 & 1 & 1 & 1 & -1 & -1  \\
      \pi_X - 1 = \chi_3 & 3 & -1 & 0 & -1 & 1 \\
      \chi_3 \chi_2 = \chi_4 & 3 & -1 & 0 & 1 & -1 \\
      \chi_5 & 2 & x & y & z & w
    \end{array}
  \]
  By column orthogonality, \(x = 2, y = -1, z = w = 0\). Alternatively, we can use
  \[
    \chi_{\text{reg}} = \chi_1 + \chi_2 + 3 \chi_3 + 3 \chi_4 + 2 \chi_5
  \]
  to deduce \(\chi_5\). It is the lifting character of \(S_4/V_4 \cong S_3\). See next chapter.
\end{eg}

\subsection{Alternating groups}

Suppose \(g \in A_n\) then
\begin{align*}
  |\mathcal C_{S_n}(g)| &= |S_n : C_{S_n}(g)| \\
  |\mathcal C_{A_n}(g)| &= |A_n : C_{A_n}(g)|
\end{align*}
\(C_{A_n}(g)\) is contained in \(C_{S_n}(g)\) but they are not necessarily equal. For example, let \(g = (123) \in A_3\). \(\mathcal C_{A_3} (g) = g\) but \(\mathcal C_{S_3}(g) = \{g, g^{-1}\}\). Recall from IA Groups

\begin{lemma}\leavevmode
  \begin{enumerate}
  \item If \(g\) commutes with some odd permutation in \(S_n\) then \(\mathcal C_{S_n}(g) = \mathcal C_{A_n}(g)\).
  \item If \(g\) does not commute with any odd permutation then \(\mathcal C_{S_n}(g)\) splits into two conjugacy classes in \(A_n\) of equal size.
  \end{enumerate}
\end{lemma}

\begin{ex}
  Character table for \(A_5\). See Teleman \textsection 12.
\end{ex}

\section{Normal subgroups and lifting characters}

\begin{lemma}[lifting]\index{lifting}
  \label{lemma:lifting}
  Let \(N \normal G\) and let \(\tilde \rho: G/N \to \GL(V)\) be a representation of \(G/N\). Then
  \[
    \rho: G \to G/N \xrightarrow{\tilde \rho} \GL(V)
  \]
  is a representation of \(G\) where \(\rho(g) = \tilde \rho (gN)\). Moreover \(\rho\) is irreducible if \(\tilde \rho\) is. The corresponding characters satisfy
  \[
    \chi(g) = \tilde \chi(gN)
  \]
  and \(\deg \chi = \deg \tilde \chi\). We say that \(\tilde \chi\) \emph{lifts} to \(\chi\).

  Lifting \(\tilde \chi \to \chi\) is a bijection between
  \[
    \{\text{irreducible reps of \(G/N\)}\} \leftrightarrow \{\text{irreducible reps of \(G\) with \(N\) lying in kernel}\}.
  \]
\end{lemma}

\begin{proof}
  Example sheet 1 Q4.
\end{proof}

\begin{lemma}\index{derived subgroup}
  The derived subgroup \(G' = \langle [a, b]: a, b \in G \rangle\) of \(G\) is the unique minimal normal subgroup of \(G\) such that \(G/G'\) is abelian. \(G\) has precisely \(\ell = |G/G'|\) representations of dimension \(1\), all with kernel containing \(G'\) and obtained by lifting from \(G/G'\). In particular \(\ell \divides |G|\).
\end{lemma}

\begin{proof}
  Easy to check \(G' \normal G\) and given \(N \normal G\), \(G' \leq N\) if and only if \(G/N\) is abelian. By \Cref{prop:representation of abelian group}, \(G/G'\) has exactly \(\ell\) characters \(\tilde \chi_1, \dots, \tilde \chi_\ell\), all of degree \(1\). The lifts of these to \(G\) also have degree \(1\) and thus by \Cref{lemma:lifting} these are precisely the irreducible characters \(\chi\) of \(G\) such that \(G' \leq \ker \chi\). But any linear character of \(G\) is a homomorphism \(\chi: G \to \C^\times\), hence \(\chi(g^{-1}h^{-1}gh) = 1\). Thus \(G' \leq \ker \chi\). Thus \(\chi_1, \dots, \chi_\ell\) are all the linear characters of \(G\).
\end{proof}

\begin{eg}\leavevmode
  \begin{enumerate}
  \item \(G = S_n\). Show \(S_n' = A_n\). Since \(G/G' \cong C_2\), \(S_n\) must have exactly 2 linear characters.
  \item \(G = A_4\). Let \(V = \{1, (12)(34), (13)(24), (14)(23)\} \normal G\) and \(G^{\text{ab}} = G/V \cong C_3\). Hence there are three linear characters, all of them trivial on \(V\). Thus \(A_4\) has character table
    \[
      \begin{array}{r|c|c|c|c}
        & 1 & 3 & 4 & 4 \\
        & 1 & (12)(34) & (123) & (132) \\ \hline
        1_G & 1 & 1 & 1 & 1 \\
        \chi_2 & 1 & 1 & \omega & \omega^2 \\
        \chi_3 & 1 & 1 & \omega^2 & \omega \\
        \chi_4 & 3 & -1 & 0 & 0
      \end{array}
    \]
    where the last row is from orthogonality.
  \end{enumerate}
\end{eg}

\begin{lemma}
  \(G\) is not simple if and only if \(\chi(g) = \chi(1)\) for some irreducible character \(\chi \neq 1_G\) and some \(1 \neq g \in G\). Moreover any normal subgroup of \(G\) is the intersection of the kernels of some of the irreducible characters of \(G\).
\end{lemma}

\begin{proof}
  If \(\chi(g) = \chi(1)\) for some non-principal character \(\chi\) (afforded by \(\rho\)) then \(g \in \ker \rho\) by \Cref{lemma:kernel of character}. So if \(g \neq 1\) then \(\ker \rho\) is a nontrivial proper normal subgroup of \(G\). If \(N\) is a nontrivial proper normal subgroup, take non-principal irreducible \(\tilde \chi\) of \(G/N\). Lift to get an irreducible \(\chi\), afforded by \(\rho\) of \(G\), then \(N \leq \ker \rho \normal G\). Hence \(\chi(g) = \chi(1)\) for all \(g \in N\).

  Claim that if \(1 \neq N \normal G\) then \(N\) is the intersection of the kernels of the lifts of all the irreducibles of \(G/N\): \(\leq\) is clear. For \(\geq\), if \(g \in G \setminus N\) then \(gN \neq N\) so \(\tilde \chi(gN) \neq \tilde \chi(N)\) for some irreducible \(\tilde \chi\) of \(G/N\). Lifting \(\tilde \chi\) to \(\chi\) we have \(\chi(g) \neq \chi(1)\).
\end{proof}

\section{Dual spaces \& tensor products}

Recall that \(\mathcal C(G)\) is the \(\C\)-space of class functions with dimension \(k\). It has an orthonormal basis \(\chi_1, \dots, \chi_k\) of irreducible characters of \(G\). There exists an involution \(f \mapsto f^*\) where \(f^*(g) = f(g^{-1})\).

\subsection{Duality}

\begin{lemma}[dual representation]\index{dual representation}
  Let \(\rho: G \to \GL(V)\) be a representation over \(\F\) and let \(V^* = \Hom_\F(V, \F)\), the dual space of \(V\). Then \(V^*\) is a \(G\)-space under
  \[
    (\rho^*(g)\varphi)(v) = \varphi(\rho(g^{-1})),
  \]
  the \emph{dual representation} to \(\rho\). Its character is
  \[
    \chi_{\rho^*}(g) = \chi_\rho(g^{-1}).
  \]
\end{lemma}

\begin{proof}
  First show \(\rho^*: G \to \GL(V^*)\) is indeed a representation:
  \begin{align*}
    \rho^*(g_1) (\rho^*(g_2) \varphi)(v)
    &= (\rho^*(g_2\varphi)) (\rho(g_1^{-1}) (v)) \\
    &= \varphi(\rho(g_2^{-1}) \varphi(g_1^{-1})(v)) \\
    &= \varphi(\rho(g_1g_2)^{-1} (v)) \\
    &= (\rho^*(g_1g_2) \varphi)(v)
  \end{align*}

  For the character, fix \(g \in G\) and let \(e_1, \dots, e_n\) be a basis of \(V\) of eigenvectors of \(\rho(g)\), say
  \[
    \rho(g) e_j = \lambda_j e_j.
  \]
  Let \(\varepsilon_1, \dots, \varepsilon_n\) be the dual basis. Then
  \[
    (\rho^*(g) \varepsilon_j)(e_i)
    = \varepsilon_j (\rho(g^{-1}) e_i)
    = \varepsilon_j \lambda_i^{-1} e_i
    = \lambda_j^{-1} \varepsilon_j e_i
  \]
  for all \(i\) so \(\rho^*(g) \varepsilon_j = \lambda_j^{-1} \varepsilon_j\). Thus
  \[
    \chi_{\rho^*}(g) = \sum \lambda_j^{-1} = \chi_\rho (g^{-1}).
  \]
\end{proof}

\begin{definition}[self-dual]\index{dual representation!self-dual}
  \(\rho: G \to \GL(V)\) is \emph{self-dual} if \(V \cong_G V^*\). Over \(\F = \C\), this holds if and only if
  \[
    \chi_\rho(g) = \chi_\rho(g^{-1}) = \conj{\chi_\rho(g)}
  \]
  if and only if \(\chi_\rho(g) \in \R\) for all \(g\).
\end{definition}

\begin{eg}\leavevmode
  \begin{enumerate}
  \item All irreducible representations of \(S_n\) are self dual: the conjugacy classes are determined by cycle types so \(g, g^{-1}\) are always \(S_n\)-conjugate. Not always true for \(A_n\).
  \item Permutation representations \(\C X\) are always self-dual.
  \end{enumerate}
\end{eg}

\subsection{Tensor products}

\begin{definition}[tensor product]\index{tensor product}
  Let \(V, W\) be \(\F\)-spaces with \(\dim V = m, \dim W = n\). Fix basis \(v_1, \dots, v_m\) of \(V\), \(w_1 \dots, w_n\) of \(W\). The \emph{tensor product} space \(V \otimes_\F W\) or \(V \otimes W\) is an \(nm\)-dimensional \(\F\)-space with basis
  \[
    \{v_i \otimes w_j: 1 \leq i \leq m, 1 \leq j \leq n \}.
  \]
\end{definition}

Thus
\begin{enumerate}
\item
  \[
    V \otimes W = \left\{\sum \lambda_{ij} v_i \otimes w_j: \lambda_{ij} \in \F \right\}
  \]
  with obvious addition and multiplication.
\item If \(v = \sum \alpha_i v_i \in V, w = \sum \beta_j w_j \in W\) define
  \[
    v \otimes w = \sum \alpha_i \beta_j (v_i \otimes w_j).
  \]
\end{enumerate}

\begin{remark}
  Note not all elements of \(V \otimes W\) are of this form --- some are combinations, e.g.\ \(v_1 \otimes w_1 + v_2 \otimes w_2\), which can't be further simplified.
\end{remark}

\begin{lemma}\leavevmode
  \begin{enumerate}
  \item For \(v \in V, w \in W, \lambda \in \F\),
    \[
      (\lambda v) \otimes w = \lambda (v \otimes w) = v \otimes (\lambda w).
    \]
  \item If \(x, x_1, x_2 \in V, y, y_1, y_2 \in W\) then
    \begin{align*}
      (x_1 + x_2) \otimes y &= x_1 \otimes y + x_2 \otimes y \\
      x \otimes (y_1 + y_2) &= x \otimes y_1 + x \otimes y_2
    \end{align*}
  \end{enumerate}
\end{lemma}

\begin{proof}
  Easy verifications:
  \begin{enumerate}
  \item if \(v = \sum \alpha_i v_i, w = \sum \beta_j v_j\) then
    \begin{align*}
      \lambda v \otimes w &= \sum_{i, j} (\lambda \alpha_i) \beta_j v_i \otimes w_j \\
      \lambda (v \otimes w) &= \sum_{i, j} (\lambda \alpha_i) \beta_j v_i \otimes w_j \\
      v \otimes \lambda w &= \sum_{i, j} (\lambda \alpha_i) \beta_j v_i \otimes w_j
    \end{align*}
  \item exercise.
  \end{enumerate}
\end{proof}

It follows that the map
\begin{align*}
  V \times W &\to V \otimes W \\
  (v, w) &\mapsto v \otimes w
\end{align*}
is bilinear.

\begin{lemma}
  If \(\{e_1, \dots, e_m\}\) is any basis of \(V\), \(\{f_1, \dots, f_n\}\) any basis of \(W\) then \(\{e_i \otimes f_j: 1 \leq i \leq m, 1 \leq j \leq n\}\) is a basis of \(V \otimes W\).
\end{lemma}

\begin{proof}
  Writing \(v_k = \sum_i \alpha_{ik} e_i, w_\ell = \sum_j \beta_{j\ell} f_j\), we have
  \[
    v_k \otimes w_\ell = \sum_{i, j} \alpha_{ik} \beta_{j\ell} e_i \otimes f_j,
  \]
  hence \(\{e_i \otimes f_j\}\) spans \(V \otimes W\). And since there are \(nm\) of them, they are a basis.
\end{proof}

\begin{remark}
  One can define \(V \otimes W\) in a basis independent way in the first place. See Teleman \textsection 6.
\end{remark}

\begin{proposition}
  Let \(\rho: G \to \GL(V), \rho': G \to \GL(V')\) be complex representations of \(G\). Define
  \[
    (\rho \otimes \rho') (g): \sum \lambda_{ij} v_i \otimes w_j \mapsto \sum \lambda_{ij} \rho(g) v_i \otimes \rho'(g) w_j.
  \]
  Then \(\rho \otimes \rho'\) is a representation with character
  \[
    \chi_{\rho \otimes \chi'} (g) = \chi_\rho(g) \chi_{\rho'} (g)
  \]
  for all \(g\). Hence product of two characters of \(G\) is also a character.
\end{proposition}

\begin{remark}
  On example sheet 1, we saw \(\rho\) irreducible, \(\rho'\) of degree \(1\) implies that \(\rho \otimes \rho'\) is irreducible. If \(\rho'\) is not of degree \(1\) this is usually false.
\end{remark}

\begin{proof}
  It is clear that \((\rho \otimes \rho') (g) \in \GL(V \otimes V')\) for all \(g \in G\) and so \(\rho \otimes \rho'\) is a homomorphism \(G \to \GL(V \otimes V')\). Let \(g \in G\). Let \(v_1, \dots, v_m\) be a basis of \(V\) of eigenvectors of \(\rho(g)\), \(w_1, \dots, w_n\) be a basis of \(V'\) of eigenvectors of \(\rho'(g)\), say
  \[
    \rho(g) v_j = \lambda_j v_j, \rho'(g) w_j = \mu_j w_j.
  \]
  Then
  \[
    (\rho \otimes \rho')(g) (v_i \otimes w_j)
    = \rho(g) v_i \otimes \rho'(g) w_j
    = \lambda_i v_i \otimes \mu_j w_j
    = (\lambda_i \mu_j) (v_i \otimes w_j)
  \]
  so
  \[
    \chi_{\rho \otimes \rho'} (g)
    = \sum_{i, j} \lambda_i \mu_j
    = \sum \lambda_i \sum \mu_j
    = \chi_\rho(g) \chi_{\rho'}(g).
  \]
\end{proof}

Let \(\F = \C\). Take \(V = V'\) and define
\[
  V^{\otimes 2} = V \otimes V.
\]
Let \(\tau: \sum \lambda_{ij} v_i \otimes v_j \mapsto \sum \lambda_{ij} v_j \otimes v_i\), a linear \(G\)-endomorphism of \(V^{\otimes 2}\) such that \(\tau^2 = 1\), so has eigenvalues \(\pm 1\).

\begin{definition}[symmetric/exterior square]\index{symmetric power}\index{exterior power}\index{symmetric square}\index{exterior square}
  Define
  \begin{align*}
    S^2V &= \{x \in V^{\otimes 2}: \tau(x) = x\} \\
    \Lambda^2 V &= \{x \in V^{\otimes 2}: \tau(x) = -x\}
  \end{align*}
  the \emph{symmetric} and \emph{exterior square} of \(V\).
\end{definition}

\begin{lemma}
  \(S^2V, \Lambda^2 V\) are \(G\)-subspaces of \(V^{\otimes 2}\) and
  \[
    V^{\otimes 2} = S^2 V \oplus \Lambda^2V.
  \]
  \(S^2V\) has basis
  \[
    \{v_iv_j = v_i \otimes v_j + v_j \otimes v_i: 1 \leq i \leq j \leq n\}
  \]
  \(\Lambda^2V\) has basis
  \[
    \{v_i \wedge v_j = v_i \otimes v_j - v_j \otimes v_i: 1 \leq i < j \leq n\}.
  \]
  (Note that in some conventions the definition of \(v_iv_j\) and \(v_i \wedge v_j\) is half what we defined here.) Hence
  \begin{align*}
    \dim S^2 V &= \frac{n (n + 1)}{2} \\
    \dim \Lambda^2 V &= \frac{n (n - 1)}{2}
  \end{align*}
\end{lemma}

\begin{proof}
  Easy exercise by noting that for any \(x \in V^{\otimes 2}\),
  \[
    x = \frac{1}{2} (x + \tau(x)) + \frac{1}{2} (x - \tau(x)).
  \]
\end{proof}

\begin{lemma}
  If \(\rho: G \to \GL(V)\) is a representation affording character \(\chi\), then
  \[
    \chi^2 = \chi_S + \chi_\Lambda
  \]
  where \(\chi_S, \chi_\Lambda\) are the characters of \(G\) in the subrepresentations \(S^2V\) and \(\Lambda^2V\). Moreover
  \begin{align*}
    \chi_S(g) &= \frac{1}{2} (\chi^2(g) + \chi(g^2)) \\
    \chi_\Lambda(g) &= \frac{1}{2} (\chi^2(g) - \chi(g^2))
  \end{align*}
\end{lemma}

\begin{proof}
  Compute the characters \(\chi_S, \chi_\Lambda\) in the usual way: fix an element and choose an eigenbasis.
\end{proof}

\begin{eg}
  \(G = S_4\): We have worked out the character table before
  \[
    \begin{array}{r|c|c|c|c|c}
      & 1 & 3 & 8 & 6 & 6 \\
      & 1 & (12)(34) & (123) & (1234) & (12) \\ \hline
      \chi_1 & 1 & 1 & 1 & 1 & 1 \\
      \sgn = \chi_2 & 1 & 1 & 1 & -1 & -1  \\
      \pi_X - 1 = \chi_3 & 3 & -1 & 0 & -1 & 1 \\
      \chi_3 \chi_2 = \chi_4 & 3 & -1 & 0 & 1 & -1 \\
      \chi_5 & 2 & 2 & -1 & 0 & 0
    \end{array}
  \]
  Take \(\chi_3\), we can work out its symmetric and exterior square
  \[
    \begin{array}{r|c|c|c|c|c}
      & 1 & 3 & 8 & 6 & 6 \\
      & 1 & (12)(34) & (123) & (1234) & (12) \\ \hline
      \chi_3^2 & 9 & 1 & 0 & 1 & 1 \\
      \chi_3(g^2) & 3 & 3 & 0 & 3 & -1 \\
      S^2 \chi_3 & 6 & 2 & 0 & 2 & 0 \\
      \Lambda^2 \chi_3 & 3 & -1 & 0 & -1 & 1
    \end{array}
  \]
  By simply calculating the inner prodcuct, we see that \(\chi_4 = \Lambda^2\chi_3\) is irreducible.

  We also see
  \[
    S^2 \chi_3 = 1 + \chi_3 + \chi_5.
  \]
\end{eg}

\subsection{Characters of product groups}

\begin{proposition}
  If \(G, H\) are finite groups, with their irreducible characters \(\chi_1, \dots, \chi_k\) and \(\psi_1, \dots, \psi_r\) respectively, then the irreducible characters of their direct product \(G \times H\) are precisely \(\{\chi_i \psi_j: 1 \leq i \leq k, 1 \leq j \leq r\}\), where
  \[
    \chi_i \psi_j (g, h) = \chi_i(g) \psi_j(h).
  \]
\end{proposition}

\begin{proof}
  If \(\rho: G \to \GL(V)\) affords \(\chi\), \(\rho': H \to \GL(W)\) affords \(\psi\) then
  \begin{align*}
    \rho \otimes \rho': G \times H &\to \GL(V \otimes W) \\
    (g, h) &\mapsto \rho(g) \otimes \rho'(h)
  \end{align*}
  is a representation of \(G \times H\) on \(V \otimes W\) and \(\chi_{\rho \otimes \rho'} = \chi\psi\).

  Claim that \(\chi_i \psi_j\)'s are distinct and irreducible:
  \begin{align*}
    \langle \chi_i \psi_j, \chi_r \psi_s \rangle_{G \times H}
    &= \frac{1}{|G \times H|} \sum_{(g, h)} \conj{\chi_i\psi_j (g, h)} \chi_r \psi_s(g, h) \\
    &= \left( \frac{1}{|G} \sum_g \conj{\chi_i(g)} \chi_r(g) \right) \left( \frac{1}{|H|} \sum_h \conj{\psi_j(h)} \psi_s(h) \right) \\
    &= \delta_{ir} \delta_{js}
  \end{align*}

  To show that they are complete, we take their squares at identity:
  \[
    \sum_{i, j} \chi_i \psi_j (1)^2
    = \sum_i \chi_i^1(1) \sum_j \psi_j^2(1)
    = |G| |H|
    = |G \times H|.
  \]
\end{proof}

\subsection{Symmetric and exterior powers}

Let \(V\) be an \(\F\)-space with \(\dim V = d\). Choose a basis \(\{v_1, \dots, v_d\}\). Let
\[
  V^{\otimes n} = \underbrace{V \otimes \dots \otimes V}_n
\]
which has a basis \(\{v_{i_1} \otimes \dots \otimes v_{i_n}: i_1, \dots, i_n \in \{1, \dots, d\}\}\) so \(\dim V^{\otimes n} = d^n\).

There is an \(S_n\)-action on the space \(V\): for each \(\sigma \in S_n\), we can define a linear map
\begin{align*}
  \sigma: V^{\otimes n} &\to V^{\otimes n} \\
  v_1 \otimes \dots \otimes v_n &\mapsto v_{\sigma(1)} \otimes \dots \otimes v_{\sigma(n)}
\end{align*}
for \(v_1, \dots, v_n \in V\), which induces a (right) action of \(S_n\) on \(V^{\otimes n}\).

Given a representation \(\rho: G \to \GL(V)\), define a (left) action of \(G\) on \(V^{\otimes n}\) by
\[
  \rho^{\otimes n}: v_1 \otimes \dots \otimes v_n \mapsto \rho(g) v_1 \otimes \dots \otimes \rho(g) v_n
\]
which commutes with the \(S_n\)-action. So we can decompose \(V^{\otimes n}\) as \(S_n\)-spaces, and each isotypical component\index{isotypical component} is a \(G\)-invariant subspace of \(V^{\otimes n}\). In particular

\begin{definition}[symmetric/exterior power]\index{symmetric power}\index{exterior power}
  For \(G\)-space \(V\), define
  \begin{enumerate}
  \item the \emph{\(n\)th symmetric power} of \(V\)
    \[
      S^nV = \{x \in V^{\otimes n}: \sigma(x) = x \text{ for all } \sigma \in S_n\},
    \]
  \item the \emph{\(n\)th exterior power} of \(V\)
    \[
      \Lambda^n V = \{x \in V^{\otimes n}: \sigma(x) = (\sgn \sigma) x \text{ for all } \sigma \in S_n\}.
    \]
  \end{enumerate}
\end{definition}
Both are \(G\)-subspaces of \(V^{\otimes n}\), but for \(n > 2\), \(S^n V \oplus \Lambda^nV\) is a proper subspace of \(V^{\otimes n}\). See example sheet 3 Q7 for bases of \(S^nV, \Lambda^nV\).

\subsection{Tensor algebra}

Take \(\ch \F = 0\).

\begin{definition}[tensor algebra]\index{tensor algebra}\index{symmetric algebra}\index{exterior algebra}
  Let \(T^nV = V^{\otimes n}\). The \emph{tensor algebra} of \(V\) is
  \[
    T(V) = \bigoplus_{n \geq 0} T^nV
  \]
  with \(T^0(V) = \F\). This is an \(\F\)-algebra. \(T(V)\) is a graded ring with product
  \[
    x \in T^n(V), y \in T^m(V) \implies x \cdot y = x \otimes y \in T^{n + m}(V).
  \]
  There are two graded quotient rings
  \begin{align*}
    S(V) &= T(V)/(u \otimes v - v \otimes u) \\
    \Lambda(V) &= T(V)/(v \otimes v)
  \end{align*}
  the \emph{symmetric} and \emph{exterior algebra} respectively. Have
  \begin{align*}
    S(V) &= \bigoplus_{n \geq 0} S^n V \\
    \Lambda(V) &= \bigoplus_{n \geq 0} \Lambda^n V
  \end{align*}
\end{definition}

\subsection{Character ring}

\(\mathcal C(G)\) is a commutative ring.

\begin{definition}[character ring, virtual character]\index{character ring}\index{Grothendieck ring}\index{virtual character}
  The \(\Z\)-submodule of \(\mathcal C(G)\) spanned by irreducible characters of \(G\) is called the \emph{character ring} of \(G\), sometimes also known as \emph{Grothendieck ring}, denoted \(R(G)\). Elements of \(R(G)\) are called \emph{generalised characters} or \emph{virtual characters}.
\end{definition}

Some properties:
\begin{enumerate}
\item \(R(G)\) is a ring. Any generalised character is a difference of two ordinary characters. \(\{\chi_i\}\) form a \(\Z\)-basis for \(R(G)\) as a free \(\Z\)-module.
\end{enumerate}

\section{Induction and restriction}

Throughout the chapter let \(\F = \C\) and \(H \leq G\).

\begin{definition}[restriction]\index{restriction}
  Let \(\rho: G \to \GL(V)\) be a representation affording \(\chi\). We can think of \(V\) as a \(H\)-space by restricting attention to \(h \in H\). We get \(\Res^G_H \rho = \rho \downarrow_H = r_H\), the \emph{restriction} of \(\rho\) to \(H\). It affords the character \(\Res^G_H \chi = \chi \downarrow_H = \chi_H\).
\end{definition}

\begin{lemma}
  If \(\psi\) is any nonzero character of \(H\) then there exists an irreducible character \(\chi\) of \(G\) such that \(\psi\) is a \emph{constituent}\index{constituent} of \(\Res_H^G \chi\), i.e.
  \[
    \ip{\Res_H^G \chi, \psi} \neq 0.
  \]
\end{lemma}

\begin{proof}
  List the irreducible characters of \(G\) as \(\chi_1, \dots, \chi_k\). Recall \(\pi_{\text{reg}}\). Have
  \[
    \sum_{i = 1}^k \deg \chi_i \ip{\Res_H^G \chi_i, \psi}
    = \ip{\Res_H^G \pi_{\text{reg}}, \psi}
    = \frac{|G|}{|H|} \psi(1)
    \neq 0
  \]
  so \(\ip{\Res_H^G \chi_i, \psi} \neq 0\) for some \(i\).
\end{proof}

\begin{lemma}
  Let \(\chi\) be an irreducible character of \(G\) and write
  \[
    \Res_H^G \chi = \sum_i c_i \chi_i
  \]
  where \(\chi_i\)'s are irreducible characters of \(H\). Then
  \[
    \sum_i c_i^2 \leq |G:H|
  \]
  with equality if and only if \(\chi(g) = 0\) for all \(g \in G \setminus H\).
\end{lemma}

\begin{proof}
  We have
  \begin{align*}
    1
    &= \ip{\chi, \chi} \\
    &= \frac{1}{|G|} \sum_{g \in G} |\chi(g)|^2 \\
    &= \frac{1}{|G|} \left(\sum_{g \in H} |\chi(g)|^2 + \sum_{g \in G \setminus H} |\chi(g)|^2 \right) \\
    &= \frac{|H|}{|G|} \ip{\Res_H^G \chi, \Res_H^G \chi} + \frac{1}{|G|} \sum_{g \in G \setminus H} |\chi(g)|^2 \\
    &\geq \frac{1}{|G:H|} \sum_i c_i^2
  \end{align*}
  with equality if and only if \(\chi(g) = 0\) for all \(g \in G \setminus H\).
\end{proof}

\begin{definition}[induction]\index{induction}
  If \(\psi\) is a class function of \(H\), define the \emph{induced class function} \(\Ind_H^G \psi = \psi \uparrow^G = \psi^G\) by
  \[
    \Ind_H^G \psi (g) = \frac{1}{|H|} \sum_{x \in G} \ocirc \psi(x^{-1} gx)
  \]
  where
  \[
    \ocirc \psi(y) =
    \begin{cases}
      \psi(y) & y \in H \\
      0 & \text{otherwise}
    \end{cases}
  \]
\end{definition}

\begin{lemma}
  If \(\psi \in \mathcal C(H)\) then \(\Ind_H^G \psi \in \mathcal C(G)\) is a class function of \(G\) and
  \[
    \Ind_H^G \psi(1) = |G:H| \psi(1).
  \]
\end{lemma}

\begin{proof}
  Obvious.
\end{proof}

Let \(n = |G:H|\). Let \(t_1 = 1, t_2, \dots, t_n\) be a \emph{left transversal}\index{left transversal} of \(H\) in \(G\), i.e.\ \(t_1H = H, t_2H, \dots, t_nH\) are precisely the left cosets of \(H\) in \(G\).

\begin{lemma}
  Given \(\psi \in \mathcal C(H)\) and a left transversal \(t_1, \dots, t_n\), have
  \[
    \Ind_H^G \psi(g) = \sum_{i = 1}^n \ocirc \psi(t_i^{-1} g t_i).
  \]
\end{lemma}

\begin{proof}
  Note that every \(x \in G\) can be written as \(t_i h\) where \(h \in H\) and
  \[
    \ocirc \psi(x^{-1} g x)
    = \ocirc \psi(h^{-1}(t_i^{-1} g t_i) h)
    = \ocirc \psi(t_i^{-1} gt_i)
  \]
  as \(\psi\) is a class function of \(H\).
\end{proof}

\begin{theorem}[Frobenius reciprocity]\index{frobenius reciprocity}
  Let \(\psi \in \mathcal C(H), \varphi \in \mathcal C(G)\). Then
  \[
    \ip{\Res_H^G \varphi, \psi} = \ip{\varphi, \Ind_H^G \psi}.
  \]
\end{theorem}

\begin{proof}
  \begin{align*}
    \ip{\varphi, \Ind_H^G \psi}
    &= \frac{1}{|G|} \sum_{g \in G} \conj{\varphi(g)} \Ind_H^G \psi(g) \\
    &= \frac{1}{|G| |H|} \sum_{x, g \in G} \conj{\varphi(g)} \ocirc \psi(x^{-1}gx) \\
    &= \frac{1}{|G| |H|} \sum_{x, y \in G} \conj{\varphi(y)} \ocirc \psi(y) \quad \text{set } x^{-1}gx = y \\
    &= \frac{1}{|H|} \sum_{y \in G} \conj{\varphi(y)} \ocirc \psi(y) \\
    &= \frac{1}{|H|} \sum_{y \in G} \conj{\varphi(y)} \psi(y) \\
    &= \ip{\Res_H^G \varphi, \psi}
  \end{align*}
\end{proof}

\begin{corollary}
  If \(\psi\) is a character of \(H\) then \(\Ind_H^G \psi\) is a character of \(G\).
\end{corollary}

\begin{proof}
  If \(\chi\) is an irreducible character of \(G\) then by Frobenius reciprocity
  \[
    \ip{\chi, \Ind_H^G \psi} = \ip{\Res_H^G \chi, \psi} \in \Z_{\geq 0}
  \]
  since \(\psi, \Res_H^G \chi\) are characters. Hence \(\Ind_H^G \psi\) is a linear combination of irreducible characters with nonnegative coefficients, hence a character.
\end{proof}

\begin{proposition}
  Let \(\psi\) be a character of \(H \leq G\) and let \(g \in G\). Let
  \[
    \mathcal C_G(g) \cap H = \bigcup_{i = 1}^m \mathcal C_H(x_i)
  \]
  where \(x_i\)'s are representatives of the \(m\) \(H\)-conjugacy classes of elements of \(H\) conjugate to \(g\). Then if \(m = 0\) then \(\Ind_H^G \psi(g) = 0\). Otherwise
  \[
    \Ind_H^G \psi(g) = |C_G(g)| \sum_{i = 1}^m \frac{\psi(x_i)}{|C_H(x_i)|}.
  \]
\end{proposition}

\begin{proof}
  If \(m = 0\) then \(\{x \in G: x^{-1} g x \in H\} = \emptyset\)  and so \(\Ind_H^G \psi(g) = 0\). Assume that \(m > 0\) and let
  \[
    X_i = \{x \in G: x^{-1} g x \in H \text{ and conjugate in \(H\) to \(x_i\)}\}.
  \]
  The \(X_i\)'s are pairwise disjoint and their union is \(\{x \in G: x^{-1} g x \in H\}\). By definition
  \begin{align*}
    \Ind_H^G \psi(g)
    &= \frac{1}{|H|} \sum_{x \in G} \ocirc \psi(x^{-1} g x) \\
    &= \frac{1}{|H|} \sum_{\substack{x \in G \\ x^{-1} g x \in H}} \psi(x^{-1} gx) \\
    &= \frac{1}{|H|} \sum_{i = 1}^m \sum_{x \in X_i} \psi(x^{-1} g x) \\
    &= \frac{1}{|H|} \sum_{i = 1}^m \sum_{x \in X_i} \psi(x_i) \\
    &= \sum_{i = 1}^m \frac{|X_i|}{|H|} \psi(x_i)
  \end{align*}

  Need to understand the quotient \(\frac{|X_i|}{|H|}\). Fix some \(1 \leq i \leq m\) and choose some \(g_i \in G\) such that \(g_i^{-1} g g_i = x_i\). So for all \(c \in C_G(g)\) and \(h \in H\),
  \[
    (cg_ih)^{-1} g (cg_ih)
    = h^{-1} g_i^{-1} c^{-1} g c g_i h
    = h^{-1} g_i^{-1} g g_i h
    = h^{-1} x_i h
    \in H
  \]
  i.e.\ \(cg_ih \in X_i\), hence \(C_G(g) g_i H \subseteq X_i\).

  Conversely, if \(x \in X_i\) then
  \[
    x^{-1} g x
    = h^{-1} x_i h 
    = h^{-1} (g_i^{-1} g g_i) h
  \]
  for some \(h \in H\). Thus \(x h^{-1} g_i^{-1} \in C_G(g)\) and
  \[
    x \in C_G(g) g_i h \subseteq C_G(g) g_i H
  \]
  so we have equality
  \[
    X_i = C_G(g) g_i H.
  \]
  Thus
  \begin{align*}
    |X_i|
    &= |C_G(g) g_i H| \\
    &= \frac{|C_G(g)| |H|}{|H \cap g_i^{-1} C_G(g) g_i|} \\
    \intertext{Note that \(g_i^{-1} C_G(g) g_i = C_G(g_i^{-1} g g_i) = C_G(x_i)\),}
    &= |H: H \cap C_G(x_i)| |C_G(g)| \\
    &= |H: C_H(x_i)| |C_G(g)|
  \end{align*}
  where we used a formula for double coset size. The result thus follows.
\end{proof}

\begin{remark}\leavevmode
  \begin{enumerate}
  \item If \(H, K \leq G\), an \emph{\((H, K)\)-double coset}\index{double coset} of \(H\) and \(K\) in \(G\) is a set
    \[
      HgK = \{hgk: h \in H, k \in K\}
    \]
    for some \(g \in G\). Facts:
    \begin{enumerate}
    \item two double cosets are either disjoint or equal.
    \item for finite \(|HK|\),
      \[
        |H g K|
        = \frac{|H| |K|}{|H \cap g K g^{-1}|}
        = \frac{|H| |K|}{|g^{-1} H g  \cap K|}
      \]
    \end{enumerate}
    See chapter 12 for more on double cosets.
  \item An alternative proof can be founded in James and Liebeck, chapter 21, 23.
  \end{enumerate}
\end{remark}

\begin{eg}
  \(H = C_4 = \langle (1234) \rangle \leq G = S_4\) with index \(6\). We calculate the character of induced representations \(\Ind_H^G(\alpha)\), where \(\alpha\) is a 1 dimensional faithful representation of \(C_4\).

  If \(\alpha(1234) = i\) then character of \(\alpha\) is
  \[
    \begin{array}{r|cccc}
      & 1 & (1234) & (13)(24) & (1432) \\ \hline
      \chi_\alpha & 1 & i & -1 & -i
    \end{array}
  \]
  The induced representation of \(S_4\) is
  \[
    \begin{array}{r|cccccc}
      & 1 & 6 & 8 & 3 & 6 \\
      & 1 & (12) & (123) & (12)(24) & (1234) \\ \hline
      \Ind_H^G \chi_\alpha  & 6 & 0 & 0 & -2 & 0
    \end{array}
  \]
  The first three entries are easy. For \((12)(34)\), only one of the three elements in \(C_4\) it's conjugate to lies in \(H\), namely \((13)(24)\) so
  \[
    \Ind_H^G \chi_\alpha ((12)(34)) = 8 \cdot \frac{-1}{4} = -2.
  \]
  For \((1234)\) its conjugate to six elements of \(S_4\), of which two are in \(C_4\): \((1234)\) and \((1432)\). So
  \[
    \Ind_H^G \chi_\alpha (1234) = 4 \cdot \left(\frac{i}{4} - \frac{i}{4} \right) = 0.
  \]
\end{eg}

\begin{lemma}
  If \(\psi = 1_H\) then
  \[
    \Ind_H^G 1_H = \pi_X,
  \]
  the permutation character of \(G\) on the set \(X\) of left cosets of \(H\) in \(G\).
\end{lemma}

\begin{proof}
  \begin{align*}
    \Ind_H^G 1_H(g)
    &= \sum_{i = 1}^n \ocirc 1_H(t_i^{-1} g t_i) \\
    &= |\{i: t_i^{-1} g t_i \in H\}| \\
    &= |\{i: g \in t_iHt_i^{-1}\}| \\
    &= |\text{fix}_X(g)| \\
    &= \pi_X (g)
  \end{align*}
\end{proof}

\begin{remark}
  It follows from Frobenius reciprocity
  \[
    \ip{\pi_X, 1_G}_G
    = \ip{\Ind_H^G 1_H, 1_G}_G
    = \ip{1_H, 1_H}_H
    = 1
  \]
  as predicted in chapter 7.
\end{remark}

\subsection{Induced representations}

What are the representations affording induced characters? Let \(H \leq G\) with index \(n\). Let \(1 = t_1, \dots, t_n\) be transversals. Let \(W\) be an \(H\)-space.

\begin{definition}
  Let
  \[
    V = \Ind_H^G W = \bigoplus_i t_i \otimes W
  \]
  where \(t_i \otimes W = \{t_i \otimes w: w \in W\}\).
\end{definition}
Have \(\dim V = n \dim W\).

We can define a \(G\)-action on \(V\). If \(g \in G\) then for all \(i\) there exists a unique \(j\) with \(t_j^{-1} g t_i \in H\) (namely \(t_j H\) is the coset containing \(gt_i\)). Define
\[
  g (t_i \otimes w)
  = t_j \otimes (\underbrace{(t_j^{-1} g t_i)}_{\in H} w)
  =  t_j ((t_j^{-1} g t_i) w).
\]
where we omit the tensor symbol in the last expression. Check this is a \(G\)-action:
\begin{align*}
  g_1(g_2 t_i w)
  &= g_1 (t_j (t_j^{-1} g_2 t_i)w) \\
  &= t_\ell ((t_\ell^{-1} g_1 t_j) (t_j^{-1} g_2 t_i) w) \\
  &= t_\ell (t_\ell^{-1} (g_1g_2)t_i)w \\
  &= (g_1g_2) (t_i w)
\end{align*}
where \(j, \ell\) is the unique such that \(g_2 t_i H = t_j H\) and \(g_1 t_jH = t_\ell H\). It follows that \(\ell\) is the unique such that \((g_1g_2) t_i H = t_\ell H\). Note that \(g\) permutes the cosets as
\[
  g: t_i w \mapsto t_j(t_j^{-1} g t_i)w
\]
so the contribution to the character is \(0\) unless \(j = i\), i.e.\ \(t_i^{-1}g t_i \in H\), then it contributes \(\psi(t_i^{-1}gt_i)\) so
\[
  \Ind_H^G \psi(g) = \sum_{i = 1}^n \ocirc \psi(t_i^{-1} gt_i).
\]

\begin{proposition}[properties of induced modules]\leavevmode
  \begin{enumerate}
  \item \(\Ind_H^G (A \oplus B) = \Ind_H^G A \oplus \Ind_H^G B\) where \(A, B\) are \(H\)-spaces.
  \item \(\dim \Ind_H^G W = |G: H| \dim W\).
  \item \(\Ind_{\{1\}}^G 1 = \rho_{\text{reg}}\).
  \item If \(H \leq K \leq G\) then
    \[
      \Ind^G_K \Ind^K_H W \cong \Ind^G_H W.
    \]
  \item (Frobenius reciprocity)
    \[
      \Hom_H(W, \Res_H^G V) \cong \Hom_G(\Ind_H^G W, V)
    \]
    naturally.
  \end{enumerate}
\end{proposition}

\begin{proof}
  Exercises. For 4 see exmaple sheet 3. For 5 see Teleman 15.6.
\end{proof}

\section{Frobenius groups}

\begin{theorem}
  \label{thm:Frobenius theorem}
  Let \(G\) be a transitive permutation group on finite set \(X\), say \(|X| = n\). Assume that each non-identity element fixes at most one element of \(X\). Then
  \[
    K = \{1\} \cup \{g \in G: g \alpha \neq \alpha \text{ for all } \alpha \in X\}
  \]
  is a normal subgroup of \(G\) of order \(n\).
\end{theorem}
Note that \(G\) is necessarily finite, being isomorphic to a subgroup of \(\Sigma_X\).

\begin{proof}[Proof Due to I.\ M.\ Issacs]
  Required to show that \(K \normal G\). Let \(H = G_\alpha\), the stabiliser of \(\alpha\) for some \(\alpha \in X\), so conjugates of \(H\) are the stabilisers of single elements of \(X\), i.e.
  \[
    G_{g\alpha} = g G_\alpha g^{-1}.
  \]
  No two conjugates can share a non-identity element by hypothesis so \(H\) has \(n\) distinct conjugates and \(G\) has \(n (|H| - 1)\) elements that fix exactly one element of \(X\). Now
  \[
    |G| = |X| |H| = n |H|
  \]
  because \(X\) and \(G/H\) are isomorphic as \(G\)-sets by transitivity. Hence
  \[
    |K| = |G| - n (|H| - 1) = n
  \]
  If \(1 \neq h \in H\) and suppose \(h = g h' g^{-1}\) for some \(g \in G, h' \in H\), then \(h\) lies in both \(H = G_\alpha\) and \(gHg^{-1} = G_{g\alpha}\), by hypothesis \(g\alpha = \alpha\), hence \(g \in H\). Therefore the conjugacy class in \(G\) of \(h\) is precisely the conjugacy class in \(H\) of \(h\).

  Similarly if \(g \in C_G(h)\) then
  \[
    h = ghg^{-1} \in G_{g\alpha}
  \]
  and hence \(g \in H\), which implies
  \[
    C_G(h) = C_H(h).
  \]

  Every element of \(G\) is either an element of \(K\) or lies in one of the \(n\) stabilisers, each of which is conjugate to \(H\). Thus every element of \(G \setminus K\) is conjugate to a non-identity element of \(H\). Hence
  \[
    \{1, h_2, \dots, h_t, y_1, \dots, y_u\}
  \]
  is a set of conjugacy class representatives for \(G\), with \(1, \dots, h_t\) representatives of \(H\)-conjugacy classes and \(y_1, \dots, y_u\) representatives of conjugacy classes of \(G\) comprises \(K \setminus \{1\}\).

  Let \(1_H = \psi_1, \psi_2, \dots, \psi_t\) be irreducible characters of \(H\). Fix \(1 \leq i \leq t\). Then if \(g \in G\),
  \[
    \Ind_H^G \psi_i =
    \begin{cases}
      |G: H| \psi_i(1) = n \psi_i(1) & g = 1 \\
      \psi_i(h_j) & g = h_j, 2 \leq j \leq t \\
      0 & g = y_k, 1 \leq k \leq u
    \end{cases}
  \]
  Let \(\theta_1 = 1_G\). Fix some \(2 \leq i \leq t\) and define virtual characters\index{virtual character}
  \[
    \theta_i = \psi_i^G - \psi_i(1) \psi_1^G + \psi_i(1) \theta_1 \in R(G)
  \]
  Write down a table
  \[
    \begin{array}{r|cccc}
      & 1 & h_j & y_k \\ \hline
      \psi_i^G & n \psi_i(1) & \psi_i(h_j) & 0 \\
      \psi_i(1) \psi_1(G) & n\psi(1) & \psi_i(1) & 0 \\
      \psi_i(1) \theta_1 & \psi_i(1) & \psi_i(1) & \psi_i(1) \\ \hline
      \theta_i & \psi_i(1) & \psi_i(h_j) & \psi_i(1)
    \end{array}
  \]
  Check the inner product:
  \begin{align*}
    \ip{\theta_i, \theta_i}
    &= \frac{1}{|G|} \sum_{g \in G} |\theta_i(g)|^2 \\
    &= \frac{1}{|G|} \left( \sum_{g \in K} |\theta_i(g)|^2 + \sum_{\alpha \in X} \sum_{1 \neq g \in G_\alpha} |\theta_i(g)|^2 \right) \\
    &= \frac{1}{|G|} \left( n \psi_i(1)^2 + n \sum_{1 \neq h \in H} |\theta_i(h)|^2 \right) \\
    &= \frac{1}{|H|} \sum_{h \in H} |\psi_i(h)|^2 \\
    &= \ip{\psi_i, \psi_i} \\
    &= 1
  \end{align*}
  Thus (cf 9.15) either \(\theta_i\) or \(-\theta_i\) is an irreducible character of \(G\). But since \(\theta_i(1) > 0\), it must be \(\theta_i\) is an actual character.

  Now define \(\theta = \sum_{i = 1}^t \theta_i(1) \theta_i\). By column orthogonality, for \(1 \neq h \in H\)
  \[
    \theta(h) = \sum_{i = 1}^t \psi_i(1) \psi_i(h) = 0,
  \]
  and for any \(y \in K\),
  \[
    \theta(y) = \sum_{i = 1}^t \psi_i(1)^2 = |H|.
  \]
  Hence \(\theta(g) =
  \begin{cases}
    |H| & g \in K \\
    0 & g \notin K
  \end{cases}
  \) so
  \[
    K = \{g \in G: \theta(g) = \theta(1)\} = \ker \theta \normal G.
  \]
\end{proof}

\begin{definition}[Frobenius group]\index{Frobenius group}\index{Frobenius complement}
  A \emph{Frobenius group} is a group \(G\) having subgroup \(H\) such that \(H \cap gHg^{-1} = 1\) for all \(g \notin H\). \(H\) is the \emph{Frobenius complement} of \(G\).
\end{definition}

\begin{proposition}
  Any finite Frobenius group satisfies the hypothesis of \Cref{thm:Frobenius theorem}. The normal subgroup \(K\) is a \emph{Frobenius kernel}\index{Frobenius kernel} of \(G\).
\end{proposition}

\begin{proof}
  Suppose \(G\) is Frobenius with complement \(H\). Then the action of \(G\) on \(G/H\) is transitive and faithful. Furthermore, if \(1 \neq g \in G\) fixes both \(xH\) and \(yH\) then \(g \in xHx^{-1} \cap yHy^{-1}\) and hence
  \[
    H \cap (y^{-1}x) H (y^{-1} x)^{-1} \neq 1.
  \]
  Hence \(xH = yH\).
\end{proof}

\begin{eg}\leavevmode
  \begin{enumerate}
  \item If \(p, q\) are distinct primes and \(p = 1 \pmod q\), the unique non-abelian group of order \(pq\) is a Frobenius group. See JL \textsection 25 and Teleman \textsection 11.
  \item If \(n\) is odd, \(D_{2n}\) is a Frobenius group with complement \(C_2\). The smallest example is \(S_3\) with \(K = C_3, H = C_2\).
  \end{enumerate}
\end{eg}

\begin{remark}\leavevmode
  \begin{enumerate}
  \item J.\ Thompson (thesis, 1959) proved that any finite group having a fixed-point-free automorphism of prime power order is nilpotent. This implies that the Frobenius kernel of a Frobenius group is nilpotent (which is equivalent to \(K\) being the direct product of its Sylow subgroups).
  \item There is no known proof of \Cref{thm:Frobenius theorem} in which character theory is not used.
  \end{enumerate}
\end{remark}

\section{Mackey theory}

Let \(\F = \C\). Mackey theory describes restriction to a subgroup \(K \leq G\) of an induced representation \(\Ind_H^G W\). \(K\) and \(H\) are unrelated, but usually we take \(K = H\), in which case we can characterise when \(\Ind_H^G W\) is irreducible.

We'll work with the special case \(W = 1_H\) first. Then \(\Ind_H^G 1_H\) is the permutation representation of \(G\) on \(G/H\). Recall that if \(G\) is transitive on a set \(X\) and \(H = G_\alpha\) for some \(a \in X\) then the action of \(G\) on \(X\) is isomorphic to the action of \(G\) on \(G/H\), namely
\[
  g . \alpha \leftrightarrow gH
  \tag{*, 12. 1}
\]
is a well-defined bijection and commutes with the \(G\)-action
\[
  x(g\alpha) = (xg) \alpha \leftrightarrow x(gH) = (xg)H.
\]
Consider the action of \(G\) on \(G/H\) and let \(K \leq G\). Then \(G/H\) splits into \(K\)-orbits: those correspond to \emph{double cosets}\index{double coset}
\[
  KgH = \{kgh: k \in K, h \in H\},
\]
namely the \(K\)-orbit containing \(gH\), contains precisely all \(kgH\) for \(k \in K\).

\begin{notation}
  Denote by \(K \setminus G/H\) the set of \((K, H)\)-double cosets. They paritition \(G\). Let \(S\) be the set of representatives. Note
  \[
    \# K \setminus G /H = \ip{\pi_{G/K}, \pi_{G/H}}
  \]
  by 7.3.
\end{notation}

  Clearly \(G_{gH} = gHg^{-1}\). Restricting to \(K\), we get
  \[
    H_g := K_{gH} = gHg^{-1} \cap K.
  \]
  %TODO: check this
  % (?) By equation 12.1 the action of \(G\) on the orbit containing \(gH\) is isomorphic to the action of \(K\) on
  % \[
  %   K/(gHg^{-1} \cap K) = K/H_g.
  % \]
  So as a set with \(K\)-action, \(KgH \cong K/K \cap gHg^{-1}\). By 10.10,
  \[
    \Ind_H^G 1_H = \C X
  \]
where \(G/H\), if \(X = \bigcup X_i\) then decomposes into orbits \(\C X = \bigcup \C X_i\).

\begin{proposition}
  If \(G\) is finite, \(H, K \leq G\) then
  \[
    \Res_K^G \Ind_H^G 1 \cong \bigoplus_g \Ind_{K \cap gHg^{-1}}^K 1
  \]
  where the summation is over all representatives \(g \in K \setminus G/H\).
\end{proposition}

Let \(S = \{1 = g_1, \dots, g_r\}\) be the such that \(G = \bigcup_i Kg_iH\) as a union of disjoint set. Let \(H_g = gHg^{-1} \cap K \leq K\). Take a representation \((\rho, W)\) of \(H\). For \(g \in G\) define \((\rho_g, W_g)\) to be the representation of \(H_g\) with the same underlying vector space \(W\) but now the \(H_g\)-action is
\[
  \rho_g(x) = \rho(h) = \rho(g^{-1}xg)
\]
where \(x = ghg^{-1}\). This is well-defined because \(g^{-1}xg \in H\) for \(x \in gHg^{-1}\). Since \(H_g \leq K\) we obtain an induced representation \(\Ind_{H_g}^K W_g\).

\begin{theorem}[Mackey's restriction formula]\index{Mackey's restriction formula}
  \label{thm:Mackey's restriction formula}
  Let \(W\) be an \(H\)-space. Then
  \[
    \Res_K^G \Ind_H^G W \cong \bigoplus_{g\in S} \Ind_{H_g}^K W_g
  \]
  as representations of \(K\).
\end{theorem}

\begin{corollary}[character version of Mackey's restriction formula]
  If \(\psi\) is a character of a representation of \(H\) then
  \[
    \Res_K^G \Ind_H^G \psi = \sum_{g \in S} \Ind_{H_g}^K \psi_g
  \]
  where \(\psi_g\) is the character of \(H_g\) given as \(\psi_g(x) = \psi(g^{-1}xg)\).
\end{corollary}

\begin{corollary}[Mackey's irreducibility criterion]\index{Mackey's irreducibility criterion}
  \label{cor:Mackey's irreducibility criterion}
  Let \(H \leq G\) and \(W\) an \(H\)-space. Then \(V = \Ind_H^G W\) is irreducible if and only if
  \begin{enumerate}
  \item \(W\) is irreducible,
  \item and for each \(g \in S \setminus H\) the two \(H_g\)-spaces \(W_g\) and \(\Res_{H_g}^H W\) have no irreducible constituents in common.
  \end{enumerate}
\end{corollary}

\begin{remark}
  The set \(f\) of representatives was arbitrary so we could just as easily demand in 2 that \(g \in G \setminus H\). However it suffices to check for \(g \in S \setminus H\).
\end{remark}

\begin{proof}[Proof of \nameref{cor:Mackey's irreducibility criterion}]
  Use characters and recall that \(W\) is irreducbile if and only if \(\ip{\psi, \psi} = 1\) where \(W\) affords the character \(\psi\). Take \(K = H\) in Mckey's restriction formula. Note \(H_g = gHg^{-1} \cap H\). Use Frobenius reciprocity,
  \begin{align*}
    \ip{\Ind_H^G \psi, \Ind_H^G \psi}_G
    &= \ip{\psi, \Res_H^G \Ind_H^G \psi}_H \\
    &= \sum_{g \in S} \ip{\psi, \Ind_{H_g}^H \psi_g}_H \\
    &= \sum_{g \in S} \ip{\Res_{H_g}^H \psi, \psi_g}_{H_g} \\
    &= \ip{\psi, \psi}_H + \sum_{\substack{g \in S \\ g \notin H}} d_g
  \end{align*}
  where \(d_g = \ip{\Res_{H_g}^G \psi, \psi_g}_{H_g}\). For \(g = 1\) we have \(H_g = H\). Hence this is a sum of nonnegative integers which is \(\geq 1\), so \(\Ind_H^G \psi\) is irreducible if and only if \(\ip{\psi, \psi} = 1\) and all the other terms are \(0\). In other words \(W\) is irreducible and for all \(g \notin H\), \(W\) and \(W_g\) are disjoint representations (of \(H \cap gHg^{-1}\)).
\end{proof}

\begin{corollary}
  If \(H \normal G\), assume \(\psi\) is an irreducible character of \(G\). Then \(\Ind_H^G \psi\) is irreducible if and only if \(\psi\) is disjoint from all its conjugates \(\psi_g\) for \(g \in G \setminus H\).
\end{corollary}

\begin{proof}
  Take \(K = H\). Double cosets are left or right cosets and \(H_g = gHg^{-1} \cap H = H\) for all \(g\). Moreover \(W_g\) is irreducible since \(W\) is irreducible. Thus \(\Ind_H^G\) is irreducible precisely if \(W \ncong W_g\) for all \(g \in G \setminus H\). This is equivalent to \(\psi \neq \psi_g\). [Again could check condiition on set of representatives: actually the isomorphism class of \(W_g\), where \(g \in G\) depends only on \(gH\)]
\end{proof}

\begin{proof}[Proof of \nameref{thm:Mackey's restriction formula}]
  Write \(V = \Ind_H^G W\). Fix \(g \in G\). Now \(V\) is direct sum of \(x \otimes W\) with \(x\) running through set of representatives of left cosets of \(H\) in \(G\). Consider a particular double coset \(KgH \in K \setminus G/H\). The terms
  \[
    \mathcal V(g) = \bigoplus_{\substack{x \text{ rep} \\ x \in KgH}} x \otimes W
  \]
  form a subspace invariant under the action of \(K\) (it is the direct sum of an orbit of subspaces permuted by \(K\) as \(kx \in KgH\) for all \(x \in KgH\).

  Now viewing \(V\) as a \(K\)-space, \(\Res_K^G V = \bigoplus_{g \in S} \mathcal V(g)\). Thus need to show \(\mathcal V(g) \cong \Ind_{H_g}^K W_g\) as \(K\)-spaces for each \(g \in S\).

  Now
  \begin{align*}
    \operatorname{Stab}_K (g \otimes W)
    &= \{k \in K: kg \otimes W = g \otimes W\} \\
    &= \{k \in : g^{-1}kg \in \operatorname{Stab}_G(1 \otimes W) = H\} \\
    &= K \cap gHg^{-1} \\
    &= H_g
  \end{align*}
  This implies that if \(x = kgh, x' = k'gh'\) then \(x \otimes W = x' \otimes W\) if and only if \(k, k'\) lie in the same coset in \(K/H_g\). Hence \(\mathcal V(g)\) is the direct sum \(\bigoplus_{\text{rep } k \in K/H_g} k \otimes (g \otimes W)\).

  Therefore as a representation for \(K\), this space is
  \[
    \mathcal V(g) \cong \Ind_{H_g}^K (g \otimes W).
  \]
  But \(W_g \cong g \otimes W\) as representations of \(H_g\) using linear isomorphism \(w \mapsto g \otimes w\). Putting all these expressions together gives the result.
\end{proof}

\section{Integrality and group algebra}

\begin{definition}[algebraic integer]\index{algebraic integer}
  \(a \in \C\) is an \emph{algebraic integer} if \(a\) is a root of a monic polynomial in \(\Z[x]\). Equivalently, the subring of \(\C\)
  \[
    \Z[a] = \{f(a): f(x) \in \Z[x]\}
  \]
  is a finitely-generated \(\Z\)-algebra.
\end{definition}

\begin{fact}\leavevmode
  \begin{enumerate}
  \item The algebraic integers form a subring of \(\C\).
  \item If \(a \in \C\) is both an algebraic integer and a rational number then \(a \in \Z\).
  \item Any subring \(S\) of \(\C\) which is a finitely-generately \(\Z\)-module consists of algebraic integers. (suppose \(s_1, \dots, s_n\) are generators of \(S\) as \(\Z\)-module and \(a \in S\). Then for all \(i\) exists \(a_{ij} \in \Z\) such that \(as_i = \sum_j a_{ij} s_j\). Put \(A = (a_{ij})\) then \(Av = av\) where \(v = (s_1, \dots, s_n)\), so \(a\) is the root of the characteristic polynomial of \(A\), and is thus an algebraic integer.
  \end{enumerate}
\end{fact}

\begin{proposition}
  If \(\chi\) is a character of \(G\) and \(g \in G\) then \(\chi(g)\) is an algebraic integer.
\end{proposition}

\begin{proof}
  \(\chi(g)\) is the sum of \(n\)th roots of unity, where \(n\) is the order of \(g\). Each root of unity is an algebraic integer.
\end{proof}

\begin{corollary}
  There are no entries in the character table of any finite group which are rational but not integers.
\end{corollary}

\subsection{The centre of \(C G\)} % TODO

Recall that the \emph{group algebra}\index{group algebra} \(\C G\) of a finite group \(G\), the \(\C\)-space with basis \(G\) and dimension \(|G|\). Also a ring and a \(\C\)-algebra.

List \(\{1\} = \ccl_1, \ccl_2, \dots, \ccl_k\) be the \(G\)-conjugacy classes. Define the \emph{class sums}\index{class sums}
\[
  C_j = \sum_{g \in \mathcal C_j} g \in \C G.
\]
Now each \(C_j \in Z(\C G)\), the centre of \(\C G\). Moreover
\begin{proposition}
  \(C_1, \dots, C_k\) is a basis of \(Z(\C G)\). There exist non-negative integers \(a_{ij\ell}\), \(1 \leq i, j, \ell \leq k\) with
  \[
    C_iC_j = \sum_\ell a_{ij\ell} C_\ell.
  \]
  These are the \emph{class algebra constants}\index{class algebra constant} or \emph{structure constants}\index{structure constant} for \(Z(\C G)\).
\end{proposition}

\begin{proof}
  Check \(g C_j g^{-1} = C_j\) for all \(g \in G\) so \(C_j \in Z(\C G)\). Clearly \(C_j\)'s are linearly independent because \(\ccl_j\)'s are disjoint. For spanning, suppose \(z = \sum_{g \in G} a_g g \in Z(\C G)\). Then for all \(h \in G\), \(a_{h^{-1}gh} = \alpha_g\) so the function \(g \mapsto a_g\) is constant on \(G\)-conjugacy classes. Writing \(a_g = \alpha_j\) if \(g \in \ccl C_j\). Then
  \[
    z = \sum_{j = 1}^k \alpha_j C_j.
  \]
  Finally \(Z(\C G)\) is a \(\C\)-algebra so \(C_iC_j = \sum_{\ell = 1}^k a_{ij\ell} C_\ell\) as the \(C_\ell\)'s span. We claim that \(a_{ij\ell} \in \Z_{\geq 0}\): fix \(g_\ell \in \ccl_\ell\) then
    \[
      a_{ij\ell} = |\{(x, y) \in \ccl_i \times \ccl_j: xy = g_\ell\}| \in \Z_{\geq 0}.
    \]
\end{proof}

\begin{definition}[representation of algebra]\index{representation of algebra}
  Let \(\rho: G \to \GL(V)\) be an irreducible representation over \(\C\) affording character \(\chi\). Extend linearly to a map \(\rho: A = \C G \to \End(V)\), an algebra homomorphism. Such a homomorphism of algebra \(A\) into \(\End(V)\) is called a \emph{representation} of \(A\).

  A \emph{central homomorphism} is a ring homomorphism \(Z(A) \to \C\).
\end{definition}

Let \(z \in Z(\C G)\). Then \(\rho(z)\) commutes with \(\rho(g)\) for all \(g \in G\), so by Schur's lemma \(\rho(z) = \lambda_z I\) for some \(\lambda_z \in \C\). Consider the central homomorphism
\begin{align*}
  \omega_\chi = \omega: Z(\C G) &\to \C \\
  z &\mapsto \lambda_z
\end{align*}
Now \(\rho(C_i) = \omega_\chi(C_i) I\) so taking traces,
\[
  \chi(1) \omega_\chi(C_i)
  = \sum_{g \in \ccl_i} \chi(g)
  = |\ccl_i| \chi(g_i).
\]
Thus
\[
  \omega_\chi(C_i) = \frac{\chi(g_i)}{\chi(1)} |\ccl_i|.
\]

\begin{lemma}
  The values of \(\omega_\chi(C_i)\) are algebraic integers.
\end{lemma}

\begin{proof}
  Since \(\omega_\chi\) is a homomorphism
  \[
    \omega_\chi(C_i) \omega_\chi(C_j) = \sum_{\ell = 1}^k a_{ij\ell} \omega_\chi (C_\ell)
  \]
  where \(a_{ij\ell} \in \Z_{\geq 0}\). Thus the span of \(\{\omega_\chi(C_j): 1 \leq j \leq k\}\) is a subring of \(\C\), and is a finitely-generated abelian group, so consists of algebraic integers.
\end{proof}

\begin{ex}
  Show that \(a_{ij\ell}\) can be obtained from the character table. In fact,
  \[
    a_{ij\ell} = \frac{|G|}{|C_G(g_i) |C_G(g_j)|} \sum_{s = 1}^k \frac{\chi_s(g_i) \chi_s(g_j) \chi_s(g_\ell^{-1})}{\chi_s(1)}.
  \]
  See JL 30.4.
\end{ex}

\begin{theorem}
  The degree of any irreducible complex character of \(G\) divides \(|G|\).
\end{theorem}

\begin{proof}
  Given an irreducible character \(\chi\),
  \begin{align*}
    \frac{|G|}{\chi(1)}
    &= \frac{1}{\chi(1)} \sum_{g \in G} \chi(g) \chi(g^{-1}) \\
    &= \frac{1}{\chi(1)} \sum_{i = 1}^k |\ccl_i| \chi(g_i) \chi(g_i^{-1}) \\
    &= \sum_{i = 1}^k \underbrace{\frac{|\ccl_i| \chi(g_i)}{\chi(1)}}_{\text{alg integer}} \chi(g_i^{-1})
  \end{align*}
  which is algebraic integer. LHS is rational.
\end{proof}

\begin{eg}\leavevmode
  \begin{enumerate}
  \item If \(G\) is a \(p\)-group then \(\chi(1)\) is a \(p\)-power. In particular if \(|G| = p^2\) then \(\chi(1) = 1\) for all \(\chi\), hence \(G\) must be abelian.
  \item If \(G = S_n\) then every prime \(p\) dividing the degree of an irreducible character also divides \(n!\), so in particular \(p \leq n\).
  \item No simple group has an irreducible character of degree 2. See James and Liebeck 22.13.
  \end{enumerate}
\end{eg}

\begin{theorem}
  If \(\chi\) is irreducible then \(\chi(1)\) divides \(|G:Z(G)|\).
\end{theorem}

\begin{proof}
  Exercise.
\end{proof}

\section{Burnside's theorem}

\begin{theorem}[Burnside]\index{Burnside's theorem}
  \label{thm:Burnside}
  Let \(p, q\) be primes. Let \(|G| = p^a q^b\) where \(a, b \in \Z_{\geq 0}\), with \(a + b \geq 2\). Then \(|G|\) is not simple.
\end{theorem}

\begin{remark}\leavevmode
  \begin{enumerate}
  \item If fact more is true: \(G\) is soluble.
  \item This is the best possible in the sense that \(|A_5| = 2^2 \cdot 3 \cdot 4\) has exactly \(3\) prime factors.
  \item If either \(a\) or \(b = 0\) then \(G\) is \(p\)-group, so nilpotent so soluble.
  \item Feit and Thompson proved in 1963 that any group of odd order is soluble.
  \item H.\ Bender and D.\ Goldschmidt independently found the first proof without the use of representation.
  \end{enumerate}
\end{remark}

The theorem follows from two lemmas, one of which is starred.

\begin{lemma}
  Suppose \(0 \neq \alpha = \frac{1}{m} \sum_{j = 1}^m \lambda_j\) with \(\lambda_j^n = 1\) is an algebraic integer. Then \(|\alpha| = 1\).
\end{lemma}

\begin{proof}[Proof*]
  Clearly \(0 < |\alpha| \leq 1\). Observe that \(\alpha \in F = \Q(\varepsilon)\) where \(\varepsilon = e^{\frac{2\pi i}{n}}\). Let \(G = \gal(F/\Q)\). We know
  \[
    \{\beta \in F: \sigma(\beta) = \beta \text{ for all } \sigma \in G\} = \Q.
  \]
  Define norm
  \[
    N(\alpha) = \prod_{\sigma \in G} \sigma(\alpha).
  \]
  Then \(N(\alpha)\) is fixed by every element of \(G\) so \(N(\alpha) \in \Q\). Now \(N(\alpha)\) is an algebraic integer since Galois conjugates of algebraic integers are algebraic integers. Thus \(N(\alpha) \in \Z\). But for \(\sigma \in G\),
  \[
    |\sigma(\alpha)| = \left| \frac{1}{m} \sum \sigma(\lambda_j) \right| \leq 1.
  \]
  Thus \(N(\alpha) = \pm 1\), which implies that \(|\alpha| = 1\).
\end{proof}

\begin{lemma}
  Suppose \(\chi\) is an irreducible character of \(G\) and \(\ccl\) is a conjugacy class in \(G\) such that \(\chi(1)\) and \(|\ccl|\) are coprime. Then for all \(g \in \ccl\), \(|\chi(g)| = \chi(1)\) or \(0\).
\end{lemma}

\begin{proof}
  By Bézout's theorem exist \(a, b \in \Z\) with \(a \chi(1) + b |\ccl| = 1\). Define
  \[
    \alpha = \frac{\chi(g)}{\chi(1)} = a \chi(g) + b \frac{\chi(g)}{\chi(1)} |\ccl|
    \]
    which is an algebraic integer. Thus \(\alpha\) satisfies the conditions of the previous lemma.
\end{proof}

\begin{proposition}
  If in a finite group \(G\), the number of elements in a conjugacy class \(\ccl_i \neq 1\) is of prime power order then \(G\) is not (non-abelian) simple.
\end{proposition}

Granted this, we can prove \nameref{thm:Burnside}: if \(a, b > 0\) let \(Q\) be a Sylow \(q\)-subgroup, so \(Q \neq 1\) (otherwise \(G\) is \(p\)-group). Now \(1 \neq Z(Q)\) so exists \(1 \neq g \in Z(Q)\). Then as \(C_G(g) \geq Q\), we have
\[
  |\mathcal C_G(g)| = |G: C_G(g)| = p^r
\]
for some \(0 \leq r \leq a\).

\begin{proof}
  Suppose \(G\) is non-abelian simple, and there exists \(1 \neq g \in G\) lying in the conjugacy class \(\ccl\) of order \(p^r\). If \(\chi \neq 1_G\) is a non-trivial irreducible character of \(G\) then \(|\chi(g)| < \chi(1)\) (otherwise \(G\) not simple). Then for every non-trivial irreducible character, either \(p \divides \chi(1)\) or \(|\chi(g)| = 0\). By column orthogonality applied to \(\{1\}\) and \(\ccl\),
  \[
    0 = 1 + \sum_{\substack{\chi \neq 1_G \\ p \divides \chi(1)}} \chi(1) \chi(g)
  \]
  so
  \[
    -\frac{1}{p} = \sum_{\chi \neq 1} \frac{\chi(1)}{p} \chi(g)
  \]
  is an algebraic integer in \(\Q\). Absurd.
\end{proof}

\section{Representations of compact groups}

See Teleman \textsection 19 - 22 and C.\ Thomas \textsection 6 for more detailed treatment of this chapter.

\begin{definition}[topological group]\index{topological group}
  A \emph{topological group} is a group \(G\) is a group \(G\) which is also a topological space and for which multiplication \(G \times G \to G\) and inversion \(G \to G\) are continuous. It is \emph{compact} if it is so as a topological space.
\end{definition}

\begin{eg}\leavevmode
  \begin{enumerate}
  \item Any finite group \(G\) with discrete topology.
  \item \(\GL_n(\C)\) and \(\GL_n(\R)\) are topological groups (as open subsets of \(\C^{n^2}\) or \(\R^{n^2}\)).
  \item Examples of compact groups:
    \begin{enumerate}
    \item finite groups,
    \item \(S^1 = \{z \in \C: |z| = 1\}\) under multiplication, the \emph{circle group},
    \item torus: finite product \(S^1 \times \cdots \times S^1\),
    \item \(O(n) = \{A \in \GL_n(\R): AA^t = I_n\}\), orthogonal group,
    \item \(\SO(n) = \{A \in O(n): \det A = 1\}\), special orthogonal group,
    \item \(U(n) = \{A \in \GL_n(\C): A \conj A^t = I_n\}\), unitary group,
    \item \(\SU(n) = \{A \in U(n): \det A = 1\}\), speical unitary group.
    \end{enumerate}
  \end{enumerate}
\end{eg}

\begin{remark}\leavevmode
  \begin{enumerate}
  \item \(U(1) \cong \SO(2) \cong S^1\). Note that the second isomorphism is also a homeomorphism.
  \item \(\SU(2) = \{(z_1, z_2) \in \C^2: z_1 \conj z_1 + z_2 \conj z_2 = 1\} \subseteq \R^4 \cong \C^2\) is isomorphic and homeomorphic to \(S^3\).
  \end{enumerate}
\end{remark}


  




\printindex
\end{document}

% Books: James-Liebeck [JL]
% Alperin & Bell
% Thomas: Representations of finite and Lie groups
% online notes: SM's notes, C. Teleman
% Webb: A course in finite group representation theory
% Curtis: Pioneers of representation theory