\documentclass[a4paper]{article}

\def\npart{II}

\def\ntitle{Linear Analysis}
\def\nlecturer{R.\ Bauerschmidt}

\def\nterm{Michaelmas}
\def\nyear{2018}

\ifx \nauthor\undefined
  \def\nauthor{Qiangru Kuang}
\else
\fi

\ifx \ntitle\undefined
  \def\ntitle{Template}
\else
\fi

\ifx \nauthoremail\undefined
  \def\nauthoremail{qk206@cam.ac.uk}
\else
\fi

\ifx \ndate\undefined
  \def\ndate{\today}
\else
\fi

\title{\ntitle}
\author{\nauthor}
\date{\ndate}

%\usepackage{microtype}
\usepackage{mathtools}
\usepackage{amsthm}
\usepackage{stmaryrd}%symbols used so far: \mapsfrom
\usepackage{empheq}
\usepackage{amssymb}
\let\mathbbalt\mathbb
\let\pitchforkold\pitchfork
\usepackage{unicode-math}
\let\mathbb\mathbbalt%reset to original \mathbb
\let\pitchfork\pitchforkold

\usepackage{imakeidx}
\makeindex[intoc]

%to address the problem that Latin modern doesn't have unicode support for setminus
%https://tex.stackexchange.com/a/55205/26707
\AtBeginDocument{\renewcommand*{\setminus}{\mathbin{\backslash}}}
\AtBeginDocument{\renewcommand*{\models}{\vDash}}%for \vDash is same size as \vdash but orginal \models is larger
\AtBeginDocument{\let\Re\relax}
\AtBeginDocument{\let\Im\relax}
\AtBeginDocument{\DeclareMathOperator{\Re}{Re}}
\AtBeginDocument{\DeclareMathOperator{\Im}{Im}}
\AtBeginDocument{\let\div\relax}
\AtBeginDocument{\DeclareMathOperator{\div}{div}}

\usepackage{tikz}
\usetikzlibrary{automata,positioning}
\usepackage{pgfplots}
%some preset styles
\pgfplotsset{compat=1.15}
\pgfplotsset{centre/.append style={axis x line=middle, axis y line=middle, xlabel={$x$}, ylabel={$y$}, axis equal}}
\usepackage{tikz-cd}
\usepackage{graphicx}
\usepackage{newunicodechar}

\usepackage{fancyhdr}

\fancypagestyle{mypagestyle}{
    \fancyhf{}
    \lhead{\emph{\nouppercase{\leftmark}}}
    \rhead{}
    \cfoot{\thepage}
}
\pagestyle{mypagestyle}

\usepackage{titlesec}
\newcommand{\sectionbreak}{\clearpage} % clear page after each section
\usepackage[perpage]{footmisc}
\usepackage{blindtext}

%\reallywidehat
%https://tex.stackexchange.com/a/101136/26707
\usepackage{scalerel,stackengine}
\stackMath
\newcommand\reallywidehat[1]{%
\savestack{\tmpbox}{\stretchto{%
  \scaleto{%
    \scalerel*[\widthof{\ensuremath{#1}}]{\kern-.6pt\bigwedge\kern-.6pt}%
    {\rule[-\textheight/2]{1ex}{\textheight}}%WIDTH-LIMITED BIG WEDGE
  }{\textheight}% 
}{0.5ex}}%
\stackon[1pt]{#1}{\tmpbox}%
}

%\usepackage{braket}
\usepackage{thmtools}%restate theorem
\usepackage{hyperref}

% https://en.wikibooks.org/wiki/LaTeX/Hyperlinks
\hypersetup{
    %bookmarks=true,
    unicode=true,
    pdftitle={\ntitle},
    pdfauthor={\nauthor},
    pdfsubject={Mathematics},
    pdfcreator={\nauthor},
    pdfproducer={\nauthor},
    pdfkeywords={math maths \ntitle},
    colorlinks=true,
    linkcolor={red!50!black},
    citecolor={blue!50!black},
    urlcolor={blue!80!black}
}

\usepackage{cleveref}



% TODO: mdframed often gives bad breaks that cause empty lines. Would like to switch to tcolorbox.
% The current workaround is to set innerbottommargin=0pt.

%\usepackage[theorems]{tcolorbox}





\usepackage[framemethod=tikz]{mdframed}
\mdfdefinestyle{leftbar}{
  %nobreak=true, %dirty hack
  linewidth=1.5pt,
  linecolor=gray,
  hidealllines=true,
  leftline=true,
  leftmargin=0pt,
  innerleftmargin=5pt,
  innerrightmargin=10pt,
  innertopmargin=-5pt,
  % innerbottommargin=5pt, % original
  innerbottommargin=0pt, % temporary hack 
}
%\newmdtheoremenv[style=leftbar]{theorem}{Theorem}[section]
%\newmdtheoremenv[style=leftbar]{proposition}[theorem]{proposition}
%\newmdtheoremenv[style=leftbar]{lemma}[theorem]{Lemma}
%\newmdtheoremenv[style=leftbar]{corollary}[theorem]{corollary}

\newtheorem{theorem}{Theorem}[section]
\newtheorem{proposition}[theorem]{Proposition}
\newtheorem{lemma}[theorem]{Lemma}
\newtheorem{corollary}[theorem]{Corollary}
\newtheorem{axiom}[theorem]{Axiom}
\newtheorem*{axiom*}{Axiom}

\surroundwithmdframed[style=leftbar]{theorem}
\surroundwithmdframed[style=leftbar]{proposition}
\surroundwithmdframed[style=leftbar]{lemma}
\surroundwithmdframed[style=leftbar]{corollary}
\surroundwithmdframed[style=leftbar]{axiom}
\surroundwithmdframed[style=leftbar]{axiom*}

\theoremstyle{definition}

\newtheorem*{definition}{Definition}
\surroundwithmdframed[style=leftbar]{definition}

\newtheorem*{slogan}{Slogan}
\newtheorem*{eg}{Example}
\newtheorem*{ex}{Exercise}
\newtheorem*{remark}{Remark}
\newtheorem*{notation}{Notation}
\newtheorem*{convention}{Convention}
\newtheorem*{assumption}{Assumption}
\newtheorem*{question}{Question}
\newtheorem*{answer}{Answer}
\newtheorem*{note}{Note}
\newtheorem*{application}{Application}

%operator macros

%basic
\DeclareMathOperator{\lcm}{lcm}

%matrix
\DeclareMathOperator{\tr}{tr}
\DeclareMathOperator{\Tr}{Tr}
\DeclareMathOperator{\adj}{adj}

%algebra
\DeclareMathOperator{\Hom}{Hom}
\DeclareMathOperator{\End}{End}
\DeclareMathOperator{\id}{id}
\DeclareMathOperator{\im}{im}
\DeclareMathOperator{\coker}{coker}
\DeclarePairedDelimiter{\generation}{\langle}{\rangle}

%groups
\DeclareMathOperator{\sym}{Sym}
\DeclareMathOperator{\sgn}{sgn}
\DeclareMathOperator{\inn}{Inn}
\DeclareMathOperator{\aut}{Aut}
\DeclareMathOperator{\GL}{GL}
\DeclareMathOperator{\SL}{SL}
\DeclareMathOperator{\PGL}{PGL}
\DeclareMathOperator{\PSL}{PSL}
\DeclareMathOperator{\SU}{SU}
\DeclareMathOperator{\UU}{U}
\DeclareMathOperator{\SO}{SO}
\DeclareMathOperator{\OO}{O}
\DeclareMathOperator{\PSU}{PSU}
\DeclareMathOperator{\Sp}{Sp}


%hyperbolic
\DeclareMathOperator{\sech}{sech}

%field, galois heory
\DeclareMathOperator{\ch}{ch}
\DeclareMathOperator{\gal}{Gal}
\DeclareMathOperator{\emb}{Emb}



%ceiling and floor
%https://tex.stackexchange.com/a/118217/26707
\DeclarePairedDelimiter\ceil{\lceil}{\rceil}
\DeclarePairedDelimiter\floor{\lfloor}{\rfloor}


\DeclarePairedDelimiter{\innerproduct}{\langle}{\rangle}

%\DeclarePairedDelimiterX{\norm}[1]{\lVert}{\rVert}{#1}
\DeclarePairedDelimiter{\norm}{\lVert}{\rVert}



%Dirac notation
%TODO: rewrite for variable number of arguments
\DeclarePairedDelimiterX{\braket}[2]{\langle}{\rangle}{#1 \delimsize\vert #2}
\DeclarePairedDelimiterX{\braketthree}[3]{\langle}{\rangle}{#1 \delimsize\vert #2 \delimsize\vert #3}

\DeclarePairedDelimiter{\bra}{\langle}{\rvert}
\DeclarePairedDelimiter{\ket}{\lvert}{\rangle}




%macros

%general

%divide, not divide
\newcommand*{\divides}{\mid}
\newcommand*{\ndivides}{\nmid}
%vector, i.e. mathbf
%https://tex.stackexchange.com/a/45746/26707
\newcommand*{\V}[1]{{\ensuremath{\symbf{#1}}}}
%closure
\newcommand*{\cl}[1]{\overline{#1}}
%conjugate
\newcommand*{\conj}[1]{\overline{#1}}
%set complement
\newcommand*{\stcomp}[1]{\overline{#1}}
\newcommand*{\compose}{\circ}
\newcommand*{\nto}{\nrightarrow}
\newcommand*{\p}{\partial}
%embed
\newcommand*{\embed}{\hookrightarrow}
%surjection
\newcommand*{\surj}{\twoheadrightarrow}
%power set
\newcommand*{\powerset}{\mathcal{P}}

%matrix
\newcommand*{\matrixring}{\mathcal{M}}

%groups
\newcommand*{\normal}{\trianglelefteq}
%rings
\newcommand*{\ideal}{\trianglelefteq}

%fields
\renewcommand*{\C}{{\mathbb{C}}}
\newcommand*{\R}{{\mathbb{R}}}
\newcommand*{\Q}{{\mathbb{Q}}}
\newcommand*{\Z}{{\mathbb{Z}}}
\newcommand*{\N}{{\mathbb{N}}}
\newcommand*{\F}{{\mathbb{F}}}
%not really but I think this belongs here
\newcommand*{\A}{{\mathbb{A}}}

%asymptotic
\newcommand*{\bigO}{O}
\newcommand*{\smallo}{o}

%probability
\newcommand*{\prob}{\mathbb{P}}
\newcommand*{\E}{\mathbb{E}}

%vector calculus
\newcommand*{\gradient}{\V \nabla}
\newcommand*{\divergence}{\gradient \cdot}
\newcommand*{\curl}{\gradient \cdot}

%logic
\newcommand*{\yields}{\vdash}
\newcommand*{\nyields}{\nvdash}

%differential geometry
\renewcommand*{\H}{\mathbb{H}}
\newcommand*{\transversal}{\pitchfork}
\renewcommand{\d}{\mathrm{d}} % exterior derivative

%number theory
\newcommand*{\legendre}[2]{\genfrac{(}{)}{}{}{#1}{#2}}%Legendre symbol

%algebraic geometry
\DeclareMathOperator{\Spec}{Spec}
\DeclareMathOperator{\Proj}{Proj}

\newtheorem*{fact}{Fact}

\newcommand{\K}{{\mathbb{K}}} % field

\begin{document}

\begin{titlepage}
  \begin{center}
    \includegraphics[width=0.6\textwidth]{logo.jpg}\par
    \vspace{1cm}
    {\scshape\huge Mathamatics Tripos \par}
    \vspace{2cm}
    {\huge Part \npart \par}
    \vspace{0.6cm}
    {\Huge \bfseries \ntitle \par}
    \vspace{1.2cm}
    {\Large\nterm, \nyear \par}
    \vspace{2cm}
    
    {\large \emph{Lectures by } \par}
    \vspace{0.2cm}
    {\Large \scshape \nlecturer}
    
    \vspace{0.5cm}
    {\large \emph{Notes by }\par}
    \vspace{0.2cm}
    {\Large \scshape \href{mailto:\nauthoremail}{\nauthor}}
 \end{center}
\end{titlepage}

\tableofcontents

\section{Normed spaces and linear operators}

Unless stated, vector spaces are real or complex, and \(\K\) stands for \(\R\) or \(\C\).

\subsection{Normed vector spaces}

\begin{definition}[normed vector space]\index{normed vector space}
  A \emph{normed vector space} \((X, \norm{\cdot})\) is a vector space \(X\) with a norm \(\norm{\cdot}: X \to \R, x \mapsto \norm x\) satisfying
  \begin{enumerate}
  \item positive-definite: \(\norm x \geq 0\) for all \(x \in X\) and \(\norm x = 0\) if and only if \(x = 0\),
  \item positive homogeneity: \(\norm{\lambda x} = |\lambda| \norm x\) for all \(\lambda \in K\) and \(x \in X\),
  \item triangle inequality: \(\norm{x + y} \leq \norm x + \norm y\) for all \(x, y \in X\).
  \end{enumerate}
\end{definition}

In particular, every norm induces a \emph{metric} by \(d(x, y) = \norm{x - y}\).

\begin{fact}
Vector space operations and the norm are continuous maps, i.e.\ the following maps
\begin{align*}
  \K \times X &\to X \\
  (\lambda, x) &\mapsto \lambda x \\
  X \times X &\to X \\
  (x, y) &\mapsto x + y \\
  X &\to \R \\
  x &\mapsto \norm x
\end{align*}
are continuous and the metric is translation invariant: \(d(x, y) = d(x + z, y + z)\) for all \(x, y, z \in X\).
\end{fact}

\begin{proof}
  We only check scalar multiplication. The others are left as exercises. Since \(\K\) and \(X\) are both metric spaces, it suffices to check that \(\lambda_j \to \lambda\) in \(\K\) and \(x_j \to x\) in \(X\) implies \(\lambda_j x_j \to \lambda x\).

  Indeed,
  \begin{align*}
    &\norm{\lambda_j x_j - \lambda x} \\
    =& \norm{(\lambda_j - \lambda) x_j + \lambda(x_j - x)} \\
    \leq& \norm{(\lambda_j - \lambda) x_j} + \norm{\lambda(x_j - x)} \\
    =& |\lambda_j - \lambda| \norm{x_j} +|\lambda| \norm{x_j - x} \\
    \to& 0
  \end{align*}
\end{proof}

\begin{eg}\leavevmode
  \begin{enumerate}
  \item \(\ell_n^2 = (\R^n, \norm \cdot_2)\) where \(\norm x_2 = (\sum_{i = 1}^n |x_i|^2)^{1/2}\), i.e.\ Euclidean norm.
  \item \(\ell_n^1 = (\R^n, \norm \cdot_1)\) where \(\norm x_1 = \sum_{i = 1}^n |x_i|\).
  \item \(\ell_n^\infty = (\R^n, \norm \cdot_\infty)\) where \(\norm x_\infty = \max_i |x_i|\).
  \end{enumerate}
\end{eg}

It is often useful to consider the \emph{unit ball} \(B = B(X) = \{x \in X: \norm x \leq 1\}\). (Pictures)

\begin{fact}\leavevmode
\begin{enumerate}
\item \(B\) determines the norm through \(\norm x = \inf \{t > 0: x \in tB\}\).
\item \(B\) is \emph{convex}: for all \(x, y \in B, \lambda \in (0, 1), \lambda x + (1 - \lambda) y \in B\).
\end{enumerate}
\end{fact}

\begin{remark}
  Any set \(B \subseteq \R^n\) which is a closed, bounded, symmetric (\(x \in B \implies -x \in B\)) neighbourhood of \(0\) defines a norm by the same formula as above and \(B\) is the unit ball of that norm, although we will not use this fact in the course.
\end{remark}

\subsection{The space \(\mathcal l^p\)}

Let \(S = \{x = (x_i)_{i = 1}^\infty \subseteq \K\}\) be the set of scalar sequences with
\begin{align*}
  x + y &= (x_i)_i + (y_i)_i = (x_i + y_i)_i, \\
  \lambda x &= \lambda (x_i)_i = (\lambda x_i)_i.
\end{align*}

\begin{definition}
  For \(1 \leq p < \infty\), let \(\ell^p = \{x \in S: \sum_n |x_n|^p < \infty\}\) with norm \(\norm x_p = (\sum_n |x_n|^p)^{1/p}\). let \(\ell^\infty = \{x \in S: \sup_n |x_n| < \infty\}\) with norm \(\norm x_\infty = \sup_n |x_n|\). Finally, \(c_0 = \{x \in S: x_n \to 0\}\) with norm \(\norm x_\infty = \sup_n |x_n|\).
\end{definition}

We have yet proved \(\norm \cdot_p\) is a norm for general \(p\). The triangle inequality follows from Minkowski's inequality, discussed next.

Recall that \(f: \R^+ \to \R\) is \emph{convex} if
\[
  f(\lambda t + (1 - \lambda) s) \leq \lambda f(t) + (1 - \lambda) f(s)
\]
for all \(s, t \in \R^+, \lambda \in (0, 1)\). Graphically, the graph of \(f\) lies below the secant between any two points on the graph. \(f\) is concave if \(-f\) is convex. Note that \(\log: \R^+ \to \R\) is a concave function.

\begin{corollary}
  Let \(1 < p, q < \infty\) with \(\frac{1}{p} + \frac{1}{q} = 1\). Then
  \[
    \frac{1}{p} |x|^p + \frac{1}{q} |y|^q \geq |x| \cdot |y|
  \]
  for all \(x, y \in \K\).
\end{corollary}

\begin{proof}
  Set \(t = |x|^p, s = |y|^q, \lambda = \frac{1}{p}\). Then
  \begin{align*}
    &\frac{1}{p} |x|^p + \frac{1}{q} |y|^q \geq |x| |y| \\
    \iff& \lambda t + (1 - \lambda) s \geq t^\lambda s^{1 - \lambda} \\
    \iff& \log (\lambda t + (1 - \lambda) s) \geq \lambda \log t + (1 - \lambda) \log s
  \end{align*}
  which holds by concavity of \(\log\).
\end{proof}

\begin{theorem}[Hölder's inequality]\index{Hölder's inequality}
  Let \(1 < p, q < \infty\) with \(\frac{1}{p} + \frac{1}{q} = 1\), let \(x \in \ell^p, y \in \ell^q\). Then \(xy = (x_ny_n)_n \in \ell^1\) and
  \[
    \norm{xy}_1 \leq \norm x_p \norm x_q.
  \]
\end{theorem}

\begin{proof}
  It suffcies to assumes that \(\norm x_p = 1 = \norm y_q\). By Hölder's inequality,
  \[
    \sum_{n = 1}^N |x_n y_n| \leq \frac{1}{p} \sum_{n = 1}^N |x_n|^p + \frac{1}{q} \sum_{n = 1}^N |y_n|^q.
  \]
  Take \(N \to \infty\),
  \[
    \norm{xy}_1 \leq \frac{1}{p} + \frac{1}{q} = 1 = \norm x_p \norm y_q.
  \]
\end{proof}

\begin{theorem}[Minkowski's inequality]\index{Minkowski's inequality}
  Let \(1 < p < \infty\) and let \(x, y \in \ell^p\). Then \(x + y \in \ell^p\) and \(\norm{x + y}_p \leq \norm x_p + \norm y_p\).
\end{theorem}

\begin{proof}
  We call the power \(r\). Have
  \begin{align*}
    & \sum_n |x_n + y_n|^r \\
    =& \sum_n |x_n + y_n|^{r - 1}|x_n + y_n| \\
    \leq& \sum_n |x_n + y_n|^{r - 1} |x_n| + \sum_n |x_n + y_n|^{r - 1} |y_n| \\
    \intertext{Applying Hölder's inequality for \(p = \frac{r}{r - 1}, q = r\) to the first term and similarly to the second term,}
    \leq& \left( \sum_n |x_n + y_n|^r \right) ^{\frac{r - 1}{r}} \left( \sum_n |x_n|^r \right)^{\frac{1}{r}} + \left( \sum_n |x_n + y_n|^r \right) ^{\frac{r - 1}{r}} \left( \sum_n |y_n|^r \right)^{\frac{1}{r}}
  \end{align*}
  Divide by both sides by a common factor, get
  \[
    \norm{x + y}_r \leq \norm x_r + \norm y_r.
  \]
\end{proof}











\printindex
\end{document}

% http://www.statslab.cam.ac.uk/~rb812/teaching/la2018/index.html