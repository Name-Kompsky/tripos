\documentclass[a4paper]{article}

\def\npart{II}

\def\ntitle{Number Theory}
\def\nlecturer{A.\ Scholl}

\def\nterm{Michaelmas}
\def\nyear{2018}

\ifx \nauthor\undefined
  \def\nauthor{Qiangru Kuang}
\else
\fi

\ifx \ntitle\undefined
  \def\ntitle{Template}
\else
\fi

\ifx \nauthoremail\undefined
  \def\nauthoremail{qk206@cam.ac.uk}
\else
\fi

\ifx \ndate\undefined
  \def\ndate{\today}
\else
\fi

\title{\ntitle}
\author{\nauthor}
\date{\ndate}

%\usepackage{microtype}
\usepackage{mathtools}
\usepackage{amsthm}
\usepackage{stmaryrd}%symbols used so far: \mapsfrom
\usepackage{empheq}
\usepackage{amssymb}
\let\mathbbalt\mathbb
\let\pitchforkold\pitchfork
\usepackage{unicode-math}
\let\mathbb\mathbbalt%reset to original \mathbb
\let\pitchfork\pitchforkold

\usepackage{imakeidx}
\makeindex[intoc]

%to address the problem that Latin modern doesn't have unicode support for setminus
%https://tex.stackexchange.com/a/55205/26707
\AtBeginDocument{\renewcommand*{\setminus}{\mathbin{\backslash}}}
\AtBeginDocument{\renewcommand*{\models}{\vDash}}%for \vDash is same size as \vdash but orginal \models is larger
\AtBeginDocument{\let\Re\relax}
\AtBeginDocument{\let\Im\relax}
\AtBeginDocument{\DeclareMathOperator{\Re}{Re}}
\AtBeginDocument{\DeclareMathOperator{\Im}{Im}}
\AtBeginDocument{\let\div\relax}
\AtBeginDocument{\DeclareMathOperator{\div}{div}}

\usepackage{tikz}
\usetikzlibrary{automata,positioning}
\usepackage{pgfplots}
%some preset styles
\pgfplotsset{compat=1.15}
\pgfplotsset{centre/.append style={axis x line=middle, axis y line=middle, xlabel={$x$}, ylabel={$y$}, axis equal}}
\usepackage{tikz-cd}
\usepackage{graphicx}
\usepackage{newunicodechar}

\usepackage{fancyhdr}

\fancypagestyle{mypagestyle}{
    \fancyhf{}
    \lhead{\emph{\nouppercase{\leftmark}}}
    \rhead{}
    \cfoot{\thepage}
}
\pagestyle{mypagestyle}

\usepackage{titlesec}
\newcommand{\sectionbreak}{\clearpage} % clear page after each section
\usepackage[perpage]{footmisc}
\usepackage{blindtext}

%\reallywidehat
%https://tex.stackexchange.com/a/101136/26707
\usepackage{scalerel,stackengine}
\stackMath
\newcommand\reallywidehat[1]{%
\savestack{\tmpbox}{\stretchto{%
  \scaleto{%
    \scalerel*[\widthof{\ensuremath{#1}}]{\kern-.6pt\bigwedge\kern-.6pt}%
    {\rule[-\textheight/2]{1ex}{\textheight}}%WIDTH-LIMITED BIG WEDGE
  }{\textheight}% 
}{0.5ex}}%
\stackon[1pt]{#1}{\tmpbox}%
}

%\usepackage{braket}
\usepackage{thmtools}%restate theorem
\usepackage{hyperref}

% https://en.wikibooks.org/wiki/LaTeX/Hyperlinks
\hypersetup{
    %bookmarks=true,
    unicode=true,
    pdftitle={\ntitle},
    pdfauthor={\nauthor},
    pdfsubject={Mathematics},
    pdfcreator={\nauthor},
    pdfproducer={\nauthor},
    pdfkeywords={math maths \ntitle},
    colorlinks=true,
    linkcolor={red!50!black},
    citecolor={blue!50!black},
    urlcolor={blue!80!black}
}

\usepackage{cleveref}



% TODO: mdframed often gives bad breaks that cause empty lines. Would like to switch to tcolorbox.
% The current workaround is to set innerbottommargin=0pt.

%\usepackage[theorems]{tcolorbox}





\usepackage[framemethod=tikz]{mdframed}
\mdfdefinestyle{leftbar}{
  %nobreak=true, %dirty hack
  linewidth=1.5pt,
  linecolor=gray,
  hidealllines=true,
  leftline=true,
  leftmargin=0pt,
  innerleftmargin=5pt,
  innerrightmargin=10pt,
  innertopmargin=-5pt,
  % innerbottommargin=5pt, % original
  innerbottommargin=0pt, % temporary hack 
}
%\newmdtheoremenv[style=leftbar]{theorem}{Theorem}[section]
%\newmdtheoremenv[style=leftbar]{proposition}[theorem]{proposition}
%\newmdtheoremenv[style=leftbar]{lemma}[theorem]{Lemma}
%\newmdtheoremenv[style=leftbar]{corollary}[theorem]{corollary}

\newtheorem{theorem}{Theorem}[section]
\newtheorem{proposition}[theorem]{Proposition}
\newtheorem{lemma}[theorem]{Lemma}
\newtheorem{corollary}[theorem]{Corollary}
\newtheorem{axiom}[theorem]{Axiom}
\newtheorem*{axiom*}{Axiom}

\surroundwithmdframed[style=leftbar]{theorem}
\surroundwithmdframed[style=leftbar]{proposition}
\surroundwithmdframed[style=leftbar]{lemma}
\surroundwithmdframed[style=leftbar]{corollary}
\surroundwithmdframed[style=leftbar]{axiom}
\surroundwithmdframed[style=leftbar]{axiom*}

\theoremstyle{definition}

\newtheorem*{definition}{Definition}
\surroundwithmdframed[style=leftbar]{definition}

\newtheorem*{slogan}{Slogan}
\newtheorem*{eg}{Example}
\newtheorem*{ex}{Exercise}
\newtheorem*{remark}{Remark}
\newtheorem*{notation}{Notation}
\newtheorem*{convention}{Convention}
\newtheorem*{assumption}{Assumption}
\newtheorem*{question}{Question}
\newtheorem*{answer}{Answer}
\newtheorem*{note}{Note}
\newtheorem*{application}{Application}

%operator macros

%basic
\DeclareMathOperator{\lcm}{lcm}

%matrix
\DeclareMathOperator{\tr}{tr}
\DeclareMathOperator{\Tr}{Tr}
\DeclareMathOperator{\adj}{adj}

%algebra
\DeclareMathOperator{\Hom}{Hom}
\DeclareMathOperator{\End}{End}
\DeclareMathOperator{\id}{id}
\DeclareMathOperator{\im}{im}
\DeclareMathOperator{\coker}{coker}
\DeclarePairedDelimiter{\generation}{\langle}{\rangle}

%groups
\DeclareMathOperator{\sym}{Sym}
\DeclareMathOperator{\sgn}{sgn}
\DeclareMathOperator{\inn}{Inn}
\DeclareMathOperator{\aut}{Aut}
\DeclareMathOperator{\GL}{GL}
\DeclareMathOperator{\SL}{SL}
\DeclareMathOperator{\PGL}{PGL}
\DeclareMathOperator{\PSL}{PSL}
\DeclareMathOperator{\SU}{SU}
\DeclareMathOperator{\UU}{U}
\DeclareMathOperator{\SO}{SO}
\DeclareMathOperator{\OO}{O}
\DeclareMathOperator{\PSU}{PSU}
\DeclareMathOperator{\Sp}{Sp}


%hyperbolic
\DeclareMathOperator{\sech}{sech}

%field, galois heory
\DeclareMathOperator{\ch}{ch}
\DeclareMathOperator{\gal}{Gal}
\DeclareMathOperator{\emb}{Emb}



%ceiling and floor
%https://tex.stackexchange.com/a/118217/26707
\DeclarePairedDelimiter\ceil{\lceil}{\rceil}
\DeclarePairedDelimiter\floor{\lfloor}{\rfloor}


\DeclarePairedDelimiter{\innerproduct}{\langle}{\rangle}

%\DeclarePairedDelimiterX{\norm}[1]{\lVert}{\rVert}{#1}
\DeclarePairedDelimiter{\norm}{\lVert}{\rVert}



%Dirac notation
%TODO: rewrite for variable number of arguments
\DeclarePairedDelimiterX{\braket}[2]{\langle}{\rangle}{#1 \delimsize\vert #2}
\DeclarePairedDelimiterX{\braketthree}[3]{\langle}{\rangle}{#1 \delimsize\vert #2 \delimsize\vert #3}

\DeclarePairedDelimiter{\bra}{\langle}{\rvert}
\DeclarePairedDelimiter{\ket}{\lvert}{\rangle}




%macros

%general

%divide, not divide
\newcommand*{\divides}{\mid}
\newcommand*{\ndivides}{\nmid}
%vector, i.e. mathbf
%https://tex.stackexchange.com/a/45746/26707
\newcommand*{\V}[1]{{\ensuremath{\symbf{#1}}}}
%closure
\newcommand*{\cl}[1]{\overline{#1}}
%conjugate
\newcommand*{\conj}[1]{\overline{#1}}
%set complement
\newcommand*{\stcomp}[1]{\overline{#1}}
\newcommand*{\compose}{\circ}
\newcommand*{\nto}{\nrightarrow}
\newcommand*{\p}{\partial}
%embed
\newcommand*{\embed}{\hookrightarrow}
%surjection
\newcommand*{\surj}{\twoheadrightarrow}
%power set
\newcommand*{\powerset}{\mathcal{P}}

%matrix
\newcommand*{\matrixring}{\mathcal{M}}

%groups
\newcommand*{\normal}{\trianglelefteq}
%rings
\newcommand*{\ideal}{\trianglelefteq}

%fields
\renewcommand*{\C}{{\mathbb{C}}}
\newcommand*{\R}{{\mathbb{R}}}
\newcommand*{\Q}{{\mathbb{Q}}}
\newcommand*{\Z}{{\mathbb{Z}}}
\newcommand*{\N}{{\mathbb{N}}}
\newcommand*{\F}{{\mathbb{F}}}
%not really but I think this belongs here
\newcommand*{\A}{{\mathbb{A}}}

%asymptotic
\newcommand*{\bigO}{O}
\newcommand*{\smallo}{o}

%probability
\newcommand*{\prob}{\mathbb{P}}
\newcommand*{\E}{\mathbb{E}}

%vector calculus
\newcommand*{\gradient}{\V \nabla}
\newcommand*{\divergence}{\gradient \cdot}
\newcommand*{\curl}{\gradient \cdot}

%logic
\newcommand*{\yields}{\vdash}
\newcommand*{\nyields}{\nvdash}

%differential geometry
\renewcommand*{\H}{\mathbb{H}}
\newcommand*{\transversal}{\pitchfork}
\renewcommand{\d}{\mathrm{d}} % exterior derivative

%number theory
\newcommand*{\legendre}[2]{\genfrac{(}{)}{}{}{#1}{#2}}%Legendre symbol

%algebraic geometry
\DeclareMathOperator{\Spec}{Spec}
\DeclareMathOperator{\Proj}{Proj}

\DeclareMathOperator{\li}{li}

\begin{document}

\begin{titlepage}
  \begin{center}
    \includegraphics[width=0.6\textwidth]{logo.jpg}\par
    \vspace{1cm}
    {\scshape\huge Mathamatics Tripos \par}
    \vspace{2cm}
    {\huge Part \npart \par}
    \vspace{0.6cm}
    {\Huge \bfseries \ntitle \par}
    \vspace{1.2cm}
    {\Large\nterm, \nyear \par}
    \vspace{2cm}
    
    {\large \emph{Lectures by } \par}
    \vspace{0.2cm}
    {\Large \scshape \nlecturer}
    
    \vspace{0.5cm}
    {\large \emph{Notes by }\par}
    \vspace{0.2cm}
    {\Large \scshape \href{mailto:\nauthoremail}{\nauthor}}
 \end{center}
\end{titlepage}

\tableofcontents

\setcounter{section}{-1}

\section{Introduction}

Number theory is the study of \(\Z\) and \(\Q\). It differs from other areas of mathematics of being an experimental science. We list three conjectures:

\begin{enumerate}
\item congruent number problem: suppose \(N \geq 1\) is an integer. Does there exist a right angle triangle whose sides are rational whose area is \(N\)? This is equivalent to find \(a, b, c \in \Q\) positive such that \(a^2 + b^2 = c^2, ab = 2N\).

  We expect if \(N\) is of the form \(8n + 5, 8n + 6\) or \(8n + 7\) then there exist such \(a, b, c\). It turns out to be equivalen to an unsolved problem about elliptic curves, i.e.\ Birch-Swinnerton-Dyer conjecutre.
\item twin prime conjecture: do there exist infinitely many primes \(p\) such that \(p + 2\) is prime? By a result of Chen in the 1950s, there exist infinitely many \(p\) such that \(p + 2\) is either prime or is the product of two primes. In 2014 it is proven that there exist infinitely many primes \(p\) such that \(\{p + 2, p + 4, \dots, p + 246\}\) contains at least one prime.
\item let \(\pi(x) = \# \{\text{primes } p \leq x\}\), the prime counting function. Define \(\li (x) = \int_2^x \frac{dt}{\log t}\) which grow asymptotically as \(\frac{x}{\log x}\). Then prime number theory asserts that \(\frac{\pi(x)}{\li(x)} \to 1\) as \(x \to \infty\). The question: is it true that \(|\pi(x) - \li(x)| \leq \sqrt x \log x\)? It turns out it would follow from Riemann Hypothesis.
\end{enumerate}

Number theory is related to algebra, geometry and many other aspects of mathematics. However in this course, we will focus on elementary number theory. But that doesn't mean we won't be able to prove results with great importance or implication. In fact, by the end of the course, we will be able to prove Tchebychev's theorem: there exist \(c_1 > c_2 > 0\) such that
\[
  c_2 \frac{2}{\log x} < \pi(x) < c_1 \frac{x}{\log x}
\]
for all \(x > 2\), which is an approximation to the prime number theorem.

\section{Euclid's algorithm and factoring}

\begin{proposition}[division algorithm]
  Given \(a, b \in \Z, b > 0\), then there exist \(q, r \in \Z\) with \(a = bq + r\) and \(0 \leq r < b\).
\end{proposition}

\begin{proof}
  Let \(S = \{a - nb: n \in \Z\}\). Then \(S\) contains some nonnegative integers. Let \(r\) be the least element of \(S\), which is nonnegative. Then claim \(r < b\): if not, \(a - (n + 1) b = r - b \in S\) and \( 0 \leq r - b\). So \(a = nb + r, 0 \leq r < b\).
\end{proof}

\begin{notation}
  If \(r = 0\), i.e.\ \(a = bq\) for some \(q \in \Z\), write \(b \divides a\), read as ``\(b\) is a divisor of \(a\)'' or ``\(b\) divides \(a\)''. Otherwise write \(b \ndivides a\).
\end{notation}

Let \(a_1, \dots, a_n \in \Z\), not all zero. Let
\[
  I = \{\lambda_1 a_1 + \dots + \lambda_n a_n: \lambda_i \in \Z\}.
\]
Note that \(I\) is an ideal of \(\Z\).

\begin{lemma}
  There exists a unique \(d \in \Z, d > 0\) such that \(I = \{md: m \in \Z\} = d\Z\).
\end{lemma}

\begin{proof}
  Uniqueness is obvious. For existence, let \(d\) be the least positive element of \(I\). Then \(d\Z \subseteq I\). If \(a \in I\), write \(a = dq + r, 0 \leq r < d\). Then \(r = a - dq \in I\) so by minimality of \(d\), \(r = 0\), i.e.\ \(a = dq \in d\Z\).
\end{proof}

\begin{remark}
  \(a_i \in I\) so for all \(a_i\), \(d \divides a_i\). On the other hand, if \(e \divides a_i\) for all \(i\) then \(e\) divides \(d = \lambda_1 a_1 + \dots \lambda_n a_n\) for some \(\lambda_i\)'s.
\end{remark}

\begin{notation}[greatest commmon divisor]\index{greatest common divisor}
  Write \(d = (a_1, \dots, a_n) = \gcd(a_1, \dots, a_n)\), the \emph{greatest common divisor} of \(\{a_i\}\), justified by the previous remark.
\end{notation}

\begin{corollary}
  Let \(a, b, c \in \Z\), \(a, b\) not both \(0\). Then there exist \(x, y \in \Z\) such that \(a x + b y = c\) if and only if \((a, b) \divides c\). Espeically, \((a, b) = 1\) if and only if there exist \(x, y \in \Z\) such that \(ax + by = 1\).
\end{corollary}

This ``algorithm'' is not very constructive since to find the GCD, we have to write down infinitely many numbers. So how to compute \(\gcd(a, b)\)? Assume \(a > b > 0\). Write
\[
  a = q_1b + r_1, 0 \leq r_1 < b.
\]
If \(r_1 = 0\) then stop, and \(\gcd(a, b) = b\). Otherwise write
\[
  b = q_2r_1 + r_2, 0 \leq r_2 < r_1.
\]
If \(r_2 = 0\) then stop. Iterate by
\begin{align*}
  r_1 &= q_3r_2 + r_3, 0 \leq r_3 < r_2, \\
      &\vdots \\
  r_{k -1} &= q_k r_{k - 1} + r_k , r_k \neq 0 \\
  r_{k - 1} &= q_{k + 1} r_k + 0
\end{align*}
It has to terminate as \(r_1 > r_2 > \dots \geq 0\). Claim that \((a, b) = r_k\): indeed \(r_1 = a - q_1b\) so \((a, b) = (b, r_1) = (r_1, r_2) = \dots = (r_{k - 1}, r_k) = r_k\).

This is know as \emph{Euclid's algorithm}\index{Euclid's algorithm}, which, besides being constructive, is very efficient.

\begin{remark}
  We know \((a, b) = d = ra + sb\) for some \(r, s \in \Z\). Euclid's algorithm also can find \(r\) and \(s\). (table of extended Euclid's algorithm)
\end{remark}

\begin{prime}[prime, composite]
  Recall \(n > 1\) is (a) \emph{prime} if its only positive divisors are \(1\) and \(n\). Otherwise say that \(n\) is \emph{composite}.
\end{prime}

\begin{lemma}
  Let \(p\) be prime and \(a, b \in \Z\) with \(p \divides ab\). Then either \(p \divides a\) or \(p \divides b\).
\end{lemma}

\begin{proof}
  Suppose \(p \ndivides a\), Then \((p, a) \divides p\) so \((p, a) = 1\). Then there exist \(x, y \in \Z\) such that \(ax + py = 1\). Then \(b = (ab) x + p (by)\) is divisible by \(p\).
\end{proof}

\begin{theorem}[Fundamental Theorem of Arithmetic]
  Every integer larger than \(1\) can be written as a product of primes. The expression is unique up to order.
\end{theorem}

\begin{proof}
  Existence is easy by induction on \(n\). For uniqueness, suppose
  \[
    n = p_1 \dots p_r = q_1 \dots q_s
  \]
  where \(p_i\)'s and \(q_i\)'s prime. Then \(p_1 \divides q_1 \dots q_s\) so by the above lemma \(p_1 = q_i\) for some \(i\). Reorder and assume \(p_1 = q_1\) so
  \[
    \frac{n}{p1} = p_2 \dots p_r = q_2 \dots q_s.
  \]
  Proceed by induction.
\end{proof}

\begin{remark}
  If
  \[
    a = \prod_{i = 1}^k p_i^{\alpha_i}, b = \prod_{i = 1}^k p_i^{\beta_i}
  \]
  where \(p_i\)'s are distinct primes, then clearly
  \[
    (a, b) = \prod_{i = 1}^k p_i^{\gamma_i}
  \]
  where \(\gamma(i) = \min(\alpha_i, \beta_i)\). However, this is \emph{not} an efficient way to calculate GCDs --- Euclid's algorithm is much better.
\end{remark}

\begin{definition}
  An algorithm with input an integer \(N > 0\) is \emph{polynomial-time} if it completes after \(\leq c (\log N)^b\) ``elementary operations'' (for example, add or multiply digits in some fixed base) for some constants \(b, c > 0\) (independent of \(N\)).
\end{definition}

\begin{eg}
  Addition and multiplication (in the usual way) and computing GCDs using Euclid's algorithm are obviously polynomial time. It is perhaps slighly surprising that primality testing is also polynomial time.

  On the other hand, factoring \(N\), by naïve way to test divisibility by numbers up to \(\sqrt N\), takes \(\sqrt N\) time. Later in the course, we will discuss better factoring algorithms (but \emph{not} polynomial-time), practial for \(N\) up to \(200\) decimal digits. Is there a polynomial-time algorithm for factoring? It is an unsolved problem and is generally believed to be no.
\end{eg}

\begin{theorem}[Euclid]
  There are infinitely many primes.
\end{theorem}

\begin{proof}
  IA Numbers and Sets.
\end{proof}

\section{Congruences}

Diophantine equaitions, i.e.\ polynomial equations in \(\Z\) whose solutions are also in \(\Z\), such as \(x^n + y^n = z^n\), are often difficult to solve. If \(f(x, y, \dots) = 0\) then for every \(n\), \(f(x, y, \dots) = 0 \pmod n\) so it is useful to study the congruence first.

Fix integer \(n \geq 1\) (the \emph{modulus}, and usually \(n > 1\)), recall that given \(a, b \in \Z\), \(a = b \pmod n\) (``\(a\) is congruent to \(b \mod n\)'') if \(n \divides a - b\). This defines an equivalence on \(\Z\). The equivalence classes are the sets \(a + n\Z\) (taking, say, \(a = 0, \dots, n - 1\)). Write \(\Z/n\Z\) for the set of equivalence classes. Then addition and multiplication are well-defined.

\begin{lemma}
  Let \(a \in \Z\). Then TFAE:
  \begin{enumerate}
  \item \((a, n) = 1\),
  \item there exists \(x \in \Z\) such that \(a x = 1 \pmod n\),
  \item (the equivalence class of) \(a\) is a generator of the group \((\Z/n\Z, +)\).
  \end{enumerate}
\end{lemma}

\begin{proof}\leavevmode
  \begin{itemize}
  \item \(1 \iff 2\): \((a, n) = 1\) if and only if there exists \(x, y \in \Z\) such that \(a x + n y = 1\), if and only if \(a x = 1 \pmod n\).
  \item \(2 \iff 3\): \(a x = 1 \pmod n\) if and only if in the group \((\Z/n\Z, +)\), \(1\) is a multiple of \(a\), if and only if \(a\) is a generator of the group.
  \end{itemize}
\end{proof}

\begin{notation}
  For \(n > 1\), \((\Z/n\Z)^* = \) set of elements of \(\Z/n\Z\) with an inverse under multiplication, i.e.\ \emph{units} of \(\Z/n\Z\).
\end{notation}

By the lemma above,
\[
  (\Z/n\Z)^* = \{a + n\Z: (a, n) = 1\}
\]
and define

\begin{definition}[Euler's \(\varphi\)-function]\index{Euler's \(\varphi\)-function}
  \[
    \varphi(n) = \# (\Z/n\Z)^* = \#\{a \in \Z: 1 \leq a \leq n: (a, n) = 1\}
  \]
  where \(\# S\) is the number of elements of \(S\).

  We put \(\varphi(1) = 1\).
\end{definition}

\begin{theorem}[Fermat-Euler]
  If \((a, n) = 1\) then \(a^{\varphi(n)} = 1 \pmod n\). Especially when \(n = p\) a prime, for all \(a \in \Z\), \(a^p = a \pmod p\).
\end{theorem}

\begin{proof}
  \(\varphi(n) = \#G\) where \(G = ((\Z/n\Z)^*, \times)\). By Lagrange's theorem, the order of \(a\) in \(G\) divides \(\varphi(n)\).
\end{proof}






\printindex
\end{document}