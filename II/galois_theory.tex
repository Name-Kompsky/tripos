\documentclass[a4paper]{article}

\def\npart{II}

\def\ntitle{Galois Theory}
\def\nlecturer{C.\ Brookes}

\def\nterm{Michaelmas}
\def\nyear{2017}

\ifx \nauthor\undefined
  \def\nauthor{Qiangru Kuang}
\else
\fi

\ifx \ntitle\undefined
  \def\ntitle{Template}
\else
\fi

\ifx \nauthoremail\undefined
  \def\nauthoremail{qk206@cam.ac.uk}
\else
\fi

\ifx \ndate\undefined
  \def\ndate{\today}
\else
\fi

\title{\ntitle}
\author{\nauthor}
\date{\ndate}

%\usepackage{microtype}
\usepackage{mathtools}
\usepackage{amsthm}
\usepackage{stmaryrd}%symbols used so far: \mapsfrom
\usepackage{empheq}
\usepackage{amssymb}
\let\mathbbalt\mathbb
\let\pitchforkold\pitchfork
\usepackage{unicode-math}
\let\mathbb\mathbbalt%reset to original \mathbb
\let\pitchfork\pitchforkold

\usepackage{imakeidx}
\makeindex[intoc]

%to address the problem that Latin modern doesn't have unicode support for setminus
%https://tex.stackexchange.com/a/55205/26707
\AtBeginDocument{\renewcommand*{\setminus}{\mathbin{\backslash}}}
\AtBeginDocument{\renewcommand*{\models}{\vDash}}%for \vDash is same size as \vdash but orginal \models is larger
\AtBeginDocument{\let\Re\relax}
\AtBeginDocument{\let\Im\relax}
\AtBeginDocument{\DeclareMathOperator{\Re}{Re}}
\AtBeginDocument{\DeclareMathOperator{\Im}{Im}}
\AtBeginDocument{\let\div\relax}
\AtBeginDocument{\DeclareMathOperator{\div}{div}}

\usepackage{tikz}
\usetikzlibrary{automata,positioning}
\usepackage{pgfplots}
%some preset styles
\pgfplotsset{compat=1.15}
\pgfplotsset{centre/.append style={axis x line=middle, axis y line=middle, xlabel={$x$}, ylabel={$y$}, axis equal}}
\usepackage{tikz-cd}
\usepackage{graphicx}
\usepackage{newunicodechar}

\usepackage{fancyhdr}

\fancypagestyle{mypagestyle}{
    \fancyhf{}
    \lhead{\emph{\nouppercase{\leftmark}}}
    \rhead{}
    \cfoot{\thepage}
}
\pagestyle{mypagestyle}

\usepackage{titlesec}
\newcommand{\sectionbreak}{\clearpage} % clear page after each section
\usepackage[perpage]{footmisc}
\usepackage{blindtext}

%\reallywidehat
%https://tex.stackexchange.com/a/101136/26707
\usepackage{scalerel,stackengine}
\stackMath
\newcommand\reallywidehat[1]{%
\savestack{\tmpbox}{\stretchto{%
  \scaleto{%
    \scalerel*[\widthof{\ensuremath{#1}}]{\kern-.6pt\bigwedge\kern-.6pt}%
    {\rule[-\textheight/2]{1ex}{\textheight}}%WIDTH-LIMITED BIG WEDGE
  }{\textheight}% 
}{0.5ex}}%
\stackon[1pt]{#1}{\tmpbox}%
}

%\usepackage{braket}
\usepackage{thmtools}%restate theorem
\usepackage{hyperref}

% https://en.wikibooks.org/wiki/LaTeX/Hyperlinks
\hypersetup{
    %bookmarks=true,
    unicode=true,
    pdftitle={\ntitle},
    pdfauthor={\nauthor},
    pdfsubject={Mathematics},
    pdfcreator={\nauthor},
    pdfproducer={\nauthor},
    pdfkeywords={math maths \ntitle},
    colorlinks=true,
    linkcolor={red!50!black},
    citecolor={blue!50!black},
    urlcolor={blue!80!black}
}

\usepackage{cleveref}



% TODO: mdframed often gives bad breaks that cause empty lines. Would like to switch to tcolorbox.
% The current workaround is to set innerbottommargin=0pt.

%\usepackage[theorems]{tcolorbox}





\usepackage[framemethod=tikz]{mdframed}
\mdfdefinestyle{leftbar}{
  %nobreak=true, %dirty hack
  linewidth=1.5pt,
  linecolor=gray,
  hidealllines=true,
  leftline=true,
  leftmargin=0pt,
  innerleftmargin=5pt,
  innerrightmargin=10pt,
  innertopmargin=-5pt,
  % innerbottommargin=5pt, % original
  innerbottommargin=0pt, % temporary hack 
}
%\newmdtheoremenv[style=leftbar]{theorem}{Theorem}[section]
%\newmdtheoremenv[style=leftbar]{proposition}[theorem]{proposition}
%\newmdtheoremenv[style=leftbar]{lemma}[theorem]{Lemma}
%\newmdtheoremenv[style=leftbar]{corollary}[theorem]{corollary}

\newtheorem{theorem}{Theorem}[section]
\newtheorem{proposition}[theorem]{Proposition}
\newtheorem{lemma}[theorem]{Lemma}
\newtheorem{corollary}[theorem]{Corollary}
\newtheorem{axiom}[theorem]{Axiom}
\newtheorem*{axiom*}{Axiom}

\surroundwithmdframed[style=leftbar]{theorem}
\surroundwithmdframed[style=leftbar]{proposition}
\surroundwithmdframed[style=leftbar]{lemma}
\surroundwithmdframed[style=leftbar]{corollary}
\surroundwithmdframed[style=leftbar]{axiom}
\surroundwithmdframed[style=leftbar]{axiom*}

\theoremstyle{definition}

\newtheorem*{definition}{Definition}
\surroundwithmdframed[style=leftbar]{definition}

\newtheorem*{slogan}{Slogan}
\newtheorem*{eg}{Example}
\newtheorem*{ex}{Exercise}
\newtheorem*{remark}{Remark}
\newtheorem*{notation}{Notation}
\newtheorem*{convention}{Convention}
\newtheorem*{assumption}{Assumption}
\newtheorem*{question}{Question}
\newtheorem*{answer}{Answer}
\newtheorem*{note}{Note}
\newtheorem*{application}{Application}

%operator macros

%basic
\DeclareMathOperator{\lcm}{lcm}

%matrix
\DeclareMathOperator{\tr}{tr}
\DeclareMathOperator{\Tr}{Tr}
\DeclareMathOperator{\adj}{adj}

%algebra
\DeclareMathOperator{\Hom}{Hom}
\DeclareMathOperator{\End}{End}
\DeclareMathOperator{\id}{id}
\DeclareMathOperator{\im}{im}
\DeclareMathOperator{\coker}{coker}
\DeclarePairedDelimiter{\generation}{\langle}{\rangle}

%groups
\DeclareMathOperator{\sym}{Sym}
\DeclareMathOperator{\sgn}{sgn}
\DeclareMathOperator{\inn}{Inn}
\DeclareMathOperator{\aut}{Aut}
\DeclareMathOperator{\GL}{GL}
\DeclareMathOperator{\SL}{SL}
\DeclareMathOperator{\PGL}{PGL}
\DeclareMathOperator{\PSL}{PSL}
\DeclareMathOperator{\SU}{SU}
\DeclareMathOperator{\UU}{U}
\DeclareMathOperator{\SO}{SO}
\DeclareMathOperator{\OO}{O}
\DeclareMathOperator{\PSU}{PSU}
\DeclareMathOperator{\Sp}{Sp}


%hyperbolic
\DeclareMathOperator{\sech}{sech}

%field, galois heory
\DeclareMathOperator{\ch}{ch}
\DeclareMathOperator{\gal}{Gal}
\DeclareMathOperator{\emb}{Emb}



%ceiling and floor
%https://tex.stackexchange.com/a/118217/26707
\DeclarePairedDelimiter\ceil{\lceil}{\rceil}
\DeclarePairedDelimiter\floor{\lfloor}{\rfloor}


\DeclarePairedDelimiter{\innerproduct}{\langle}{\rangle}

%\DeclarePairedDelimiterX{\norm}[1]{\lVert}{\rVert}{#1}
\DeclarePairedDelimiter{\norm}{\lVert}{\rVert}



%Dirac notation
%TODO: rewrite for variable number of arguments
\DeclarePairedDelimiterX{\braket}[2]{\langle}{\rangle}{#1 \delimsize\vert #2}
\DeclarePairedDelimiterX{\braketthree}[3]{\langle}{\rangle}{#1 \delimsize\vert #2 \delimsize\vert #3}

\DeclarePairedDelimiter{\bra}{\langle}{\rvert}
\DeclarePairedDelimiter{\ket}{\lvert}{\rangle}




%macros

%general

%divide, not divide
\newcommand*{\divides}{\mid}
\newcommand*{\ndivides}{\nmid}
%vector, i.e. mathbf
%https://tex.stackexchange.com/a/45746/26707
\newcommand*{\V}[1]{{\ensuremath{\symbf{#1}}}}
%closure
\newcommand*{\cl}[1]{\overline{#1}}
%conjugate
\newcommand*{\conj}[1]{\overline{#1}}
%set complement
\newcommand*{\stcomp}[1]{\overline{#1}}
\newcommand*{\compose}{\circ}
\newcommand*{\nto}{\nrightarrow}
\newcommand*{\p}{\partial}
%embed
\newcommand*{\embed}{\hookrightarrow}
%surjection
\newcommand*{\surj}{\twoheadrightarrow}
%power set
\newcommand*{\powerset}{\mathcal{P}}

%matrix
\newcommand*{\matrixring}{\mathcal{M}}

%groups
\newcommand*{\normal}{\trianglelefteq}
%rings
\newcommand*{\ideal}{\trianglelefteq}

%fields
\renewcommand*{\C}{{\mathbb{C}}}
\newcommand*{\R}{{\mathbb{R}}}
\newcommand*{\Q}{{\mathbb{Q}}}
\newcommand*{\Z}{{\mathbb{Z}}}
\newcommand*{\N}{{\mathbb{N}}}
\newcommand*{\F}{{\mathbb{F}}}
%not really but I think this belongs here
\newcommand*{\A}{{\mathbb{A}}}

%asymptotic
\newcommand*{\bigO}{O}
\newcommand*{\smallo}{o}

%probability
\newcommand*{\prob}{\mathbb{P}}
\newcommand*{\E}{\mathbb{E}}

%vector calculus
\newcommand*{\gradient}{\V \nabla}
\newcommand*{\divergence}{\gradient \cdot}
\newcommand*{\curl}{\gradient \cdot}

%logic
\newcommand*{\yields}{\vdash}
\newcommand*{\nyields}{\nvdash}

%differential geometry
\renewcommand*{\H}{\mathbb{H}}
\newcommand*{\transversal}{\pitchfork}
\renewcommand{\d}{\mathrm{d}} % exterior derivative

%number theory
\newcommand*{\legendre}[2]{\genfrac{(}{)}{}{}{#1}{#2}}%Legendre symbol

%algebraic geometry
\DeclareMathOperator{\Spec}{Spec}
\DeclareMathOperator{\Proj}{Proj}

\DeclareMathOperator{\n}{N}
\newcommand*{\red}[1]{\overline{#1}}

\begin{document}

\begin{titlepage}
  \begin{center}
    \includegraphics[width=0.6\textwidth]{logo.jpg}\par
    \vspace{1cm}
    {\scshape\huge Mathamatics Tripos \par}
    \vspace{2cm}
    {\huge Part \npart \par}
    \vspace{0.6cm}
    {\Huge \bfseries \ntitle \par}
    \vspace{1.2cm}
    {\Large\nterm, \nyear \par}
    \vspace{2cm}
    
    {\large \emph{Lectures by } \par}
    \vspace{0.2cm}
    {\Large \scshape \nlecturer}
    
    \vspace{0.5cm}
    {\large \emph{Notes by }\par}
    \vspace{0.2cm}
    {\Large \scshape \href{mailto:\nauthoremail}{\nauthor}}
 \end{center}
\end{titlepage}

\tableofcontents

\section{Field Extensions}

\blindtext{10}

\section{Seperable, Normal and Galois Extensions}

\blindtext

\subsection{Trace \& Norm}

\begin{definition}[Trace and norm]
  Let \(K \leq M\) be a finite field extension and \(\alpha \in M\). Multiplication by \(a\) is a \(K\)-linear map \(\theta_\alpha: M \to M\). The \emph{trace of \(\alpha\) over \(K\)} is
  \[
    \Tr_{M/K}(\alpha) = \Tr( \theta_\alpha) \in K
  \]
  and the \emph{norm of \(\alpha\) over \(K\)} is
  \[
    \n_{M/K}(\alpha) = \det \theta_\alpha \in K.
  \]
\end{definition}

\begin{note}
  These dependent on the field extension.
\end{note}

\begin{theorem}
  Suppose \(f_\alpha(t) = t^s + a_{s - 1}t^{s - 1} + \dots + a_0\) is the minimal polynomial of \(\alpha\) over \(K\). Let \(r = |M: K(\alpha)|\). Then the characteristic polynomial of \(\theta_\alpha\) is \((f_\alpha(t))^r\) and
  \begin{align*}
    \Tr_{M/K}(\alpha) &= -r a_{s-1} \\
    \n_{M/K}(\alpha) &= ((-1)^s a_0)^r
  \end{align*}
\end{theorem}

\subsection{Normal Extensions}

We met this definition before:

\begin{definition}[Normal extension]
  An extension \(K \leq M\) is \emph{normal} if for every \(\alpha \in M\), the minimal polynomial \(f_\alpha(t)\) of \(\alpha\) over \(K\) splits over \(M\).
\end{definition}

\begin{theorem}
  Let \(K \leq M\) be a finite field extension. Then \(K \leq M\) is normal if and only if \(M\) is the splitting field for some \(f(t) \in K[t]\).
\end{theorem}

\begin{proof}\leavevmode
  \begin{itemize}
  \item \(\Rightarrow\): suppose \(K \leq M\) isnormal. Pick \(\alpha_1, \dots, \alpha_r \in M\) such that \(M = K(\alpha_1, \dots, \alpha_kr\). Let \(f(_{\alpha_i}(t)\) be the minimal polynomail of \(\alpha_i\) over \(K\). Let \(f(t) = \prod_{n = 1}^{r} f_{\alpha_i}(t)\).

    By normality, each \(f_{\alpha_i}(t)\) splits over \(M\) and so is \(f(t)\). \(M\) is the splitting field of \(f(t)\) over \(K\) since if \(\beta_1, \dots, \beta_m\) are the roots of \(f(t)\) then \(M = K(\beta_1, \dots, \beta_m)\).
  \item \(\Leftarrow\): suppose \(M\) is a splitting field for \(f(t)\) over \(K\). Then \(M = K(\beta_1, \dots, \beta_m)\) where \(\beta_j\) are the roots of \(f(t)\) over \(M\). Take \(\alpha \in M\). Let \(f_\alpha(t)\) be the minimal polynomial of \(\alpha\) over \(K\). Let \(M \leq L\) large enough so that \(f_\alpha(t)\) splits over \(L\)

    Now consdier a \(K\)-homomorphism \(\phi: M \to L\): \(\phi(\beta_j)\) is also a root of \(f(t)\) and is therefore one of the \(\beta_j\). Injectivity of \(K\)-homomorphisms implies that \(\phi\) permutes \(\beta_j\). However, \(M = K(\beta_1, \dots, \beta_m)\) and so \(\phi\) is determined by the images of \(\beta_j\), thus \(\phi(M) = M\).

    However, if \(\alpha_i\) is a root of \(f_\alpha(t)\) in \(L\), then there is a \(K\)-homomorphism
    \begin{align*}
      K(\alpha) &\to K(\alpha_i) \\
      \alpha &\mapsto \alpha_i
    \end{align*}
    This extends by 2.10~\ref{tbf} to a \(K\)-homomorphism \(\phi: M \to L\). But \(\phi(M) = M\) so \(\alpha_i \in M\). Thus \(M\) is normal over \(K\).
  \end{itemize}
\end{proof}

\begin{remark}
  As for separability, being normal is equivalent to being normally generated. We will show in example sheet that \(K \leq L\) is a normal and finite extension if and only if \(L = K(\alpha_1, \dots, \alpha_r)\) with the minimal polynomial of each adjoined element splitting over \(L\).
\end{remark}

\begin{definition}[\(K\)-automorphism group]
  Let \(K \leq M\) be a finite extension. Its \emph{\(K\)-automorphism group} is
  \[
    \aut_K(M) = \Hom_K(M, M).
  \]
\end{definition}

From \ref{tbh} we know that such \(K\)-automorphisms are isormophisms and thus have inverses (well, it name says so).

\begin{lemma}
  \[
    |\aut_K(M)| \leq |M:K|.
  \]
\end{lemma}

\begin{proof}
  \ref{tbh} (extension of homomorphisms)
\end{proof}

\begin{theorem}
  \label{thm:galois criterion}
  Let \(K \leq M\) be a finite field extension, then
  \[
    |\aut_K(M)| = |M:K|
  \]
  if and only if the extension is both normal and separable.
\end{theorem}

\begin{definition}[Galois extension]
  A finite field extension that is normal and separable is a \emph{Galois extension}.
\end{definition}

\begin{definition}[Galois group]
  Let \(K \leq M\) be a Galois extension. Then the \(K\)-automorphism group of \(M\) is the \emph{Galois group} of \(M\) over \(K\), denoted by
  \[
    \gal(M/K).
  \]
\end{definition}

\begin{remark}
  Some authors use the term ``Galois group'' as a synonym for automorphism group even when the extension is not Galois.
\end{remark}

\begin{proof}[Proof of \Cref{thm:galois criterion}]
  Suppose \(|\aut_K(M)| = |M:K| = n\). Let \(M \leq L\) be large enough. The \(n\) distinct \(K\)-automorphisms \(\phi: M \to M\) extend to \(n\) \(K\)-homomorphisms \(\phi: M \to L\) and 2.12~\ref{tbf} says that \(M\) is seperable over \(K\).

  For normality, pick \(\alpha \in M\) with minimal polynomial \(f_\alpha(t)\) over \(K\). Take \(M = K(\alpha_1, \dots, \alpha_m)\) as in the proof of 2.10~\ref{tbf} with \(\alpha = \alpha_1\) and \(L = M\). We only get \(|M:K|\) extensions of the inclusion \(K \hookrightarrow M\) if each inequality in the proof is an equality. In particular, we need the number of \(K\)-homomorphisms \(K(\alpha_1) \to M\) to be \(K(\alpha_1):K|\). But then 2.6~\ref{tbf} says we have \(|K(\alpha):K|\) distinct roots of \(f_\alpha(t)\) i \(M\). Thus \(f_\alpha(t)\) splits over \(M\).

  Conversely, suppose \(K \leq M\) is separable and normal. Then for \(K \leq M \leq L\) with \(L\) large enough, separability implies there are \(|M:K|\) \(K\)-homomorphisms \(\phi: M \to L\) by 2.12~\ref{tbf}. However, \(K \leq M\) is normal implies that it is the splitting field for some polynomial \(f(t) \in K[t]\) and thus \(M = K(\alpha_1, \dots, \alpha_n)\) where \(f(t) = \prod_{i = 1}^n (t - \alpha_i)\). Note that \(\phi(\alpha_j)\) is also a root of \(\phi(f(t)) = f(t)\), and is therefore one of the \(\alpha_j\). Thus \(\phi(M) = M\). Thus \(|\aut_K(M)| = |M:K|\).
\end{proof}

\begin{remark}
  In the previous proof we have shown that if \(K \leq M \leq L\), \(\phi \in \Hom_K(M, L)\) and \(K \leq M\) is normal then \(\phi(M) = M\).
\end{remark}

\begin{eg}\leavevmode
  \begin{enumerate}
  \item Consider \(\Q \leq \Q(\sqrt 2, i)\), which is Galois:
    \[
      \gal(\Q(\sqrt 2, i)/\Q) = \generation{\sigma: \sqrt 2 \mapsto -\sqrt 2, \tau: i \mapsto -1} \cong C_2 \times C_2.
    \]
    All non-identity elements have order \(2\).
  \item Let \(f(t) = t^3 - 2\). The splitting field of \(f(t)\) over \(\Q\) is \(\Q(\sqrt[3]{2}, \omega)\) where \(\omega\) is a primitive cubic root of unity. Thus \(\Q \leq \Q(\sqrt[3]{2}, \omega)\) is Galois with \(|\gal(\Q(\sqrt[3]{2}, \omega)/\Q) = |\Q(\sqrt[3]{2}, \omega):\Q| = 6\). The Galois group contains
    \begin{align*}
      \sigma_1: \sqrt[3]{2} \mapsto \sqrt[3]{2} %%%%%%%%
    \end{align*}
    \texttt{to be filled in}
    %
    %
    %
    %
    %
    %
  \end{enumerate}
\end{eg}
\section{Fundamental Theorem of Galois Theory}

\begin{definition}[Fixed field]
  Let \(K \leq L\) be a field extension and \(H \leq \aut_K(L)\). The \emph{fixed field} of \(H\) is
  \[
    L^H = \{\alpha \in L: \sigma(\alpha) = \alpha \text{ for all } \sigma \in H \}.
  \]
\end{definition}

The fixed field is a field and \(K \leq L^H \leq L\).

\begin{theorem}[Fundamental Theorem of Galois Theory]
  \label{thm:fundamental}
  Let \(K \leq L\) be a finite Galois extension. Then
  \begin{enumerate}
  \item There is a one-to-one correspondence
    \begin{align*}
      \left\{
        \begin{array}[h]{c}
          \text{intermediate subfield} \\
          K \leq M \leq L
        \end{array}
      \right\}
      &\longleftrightarrow
      \left\{
        \begin{array}[h]{c}
          \text{subgroup } H \\
          \text{of } \gal(L/K)
        \end{array}
      \right\} \\
      %
      M &\mapsto \aut_M(L) \\
      L^H &\mapsfrom H
    \end{align*}
  \item \(H\) is a normal subgroup of \(\gal(L/K)\) if and only if \(K \leq L^H\) is normal if and only if \(K \leq L^H\) is Galois.
  \item If \(H \trianglelefteq \gal(L/K)\) then the map
    \[
      \theta: \gal(L/K) \to \gal(L^H/K)
    \]
    given by restriction to \(L^H\) is a surjective group homomorphism with kernel \(H\).
  \end{enumerate}
\end{theorem}

\begin{remark}\leavevmode
  \begin{enumerate}
  \item Observe that \(M \leq L\) is Galois and so we could have written \(\gal(L/M)\) instead of \(\aut_M(L)\). To see this,
    \begin{itemize}
    \item seperability: follows from \ref{tbf}
    \item normality: if \(\alpha \in L\) then the minimal polynomial of \(\alpha\) over \(M\) divides the minimal polynomial of \(\alpha\) over \(K\). But the latter splits over \(L\) (see example sheet).
    \end{itemize}
  \item If \(K \leq M\) is normal then \ref{tbf} says if \(\sigma: L \to L\) then \(\sigma(M) = M\) and so we can talk about the restriction of \(\alpha\) to \(M\) giving an automorphism of \(M\).
  \end{enumerate}
\end{remark}

\begin{eg}\leavevmode
  \begin{enumerate}
  \item \(\Q \leq \Q(\sqrt{2}, i)\). We saw in \ref{tbf} the lattices of intermediate subfields and subgroups \(\gal(\Q(\sqrt 2, i/\Q) \cong C_2 \times C_2\) which is abelian. Thus all subgroups are normal and all intermediate subfields are normal extensions of \(\Q\).
  \item \(\Q \leq \Q(\sqrt[3]{2}, \omega)\).
    \[
      \begin{tikzcd}
        & \Q(\sqrt[3]{2}, \omega) \ar[dl, dash, "3"] \ar[d, dash, "2"] \ar[dr, dash, "2"] \ar[drr, dash, "2"] \\
        \Q(\omega) \ar[dr, dash, "2"] & \Q(\sqrt[3]{2}) \ar[d, dash, "3"] & \Q(\omega\sqrt[3]{2}) \ar[dl, dash, "3"] & \Q(\omega^2\sqrt[3]{2}) \ar[dll, dash, "3"] \\
        & \Q
      \end{tikzcd}
    \]
    \[
      \begin{tikzcd}
        & 1 \ar[dl, dash, "3"] \ar[d, dash, "2"] \ar[dr, dash, "2"] \ar[drr, dash, "2"] \\
        \{\sigma_1, \sigma_2, \sigma_3\} \ar[dr, dash, "2"] & \{\sigma_1, \sigma_4\} \ar[d, dash, "3"] & \{\sigma_1, \sigma_5\} \ar[dl, dash, "3"] & \{\sigma_1, \sigma_6\} \ar[dll, dash, "3"] \\
        & \{\sigma_1, \dots, \sigma_6\} \cong D_6
      \end{tikzcd}
    \]
    The subgroup \(H\) of order \(3\) is normal but those of order \(2\) are not so the map \(\gal(\Q(\sqrt[3]{2}, \omega)/\Q) \to \gal(\Q(\omega)/\Q)\) has kernel \(H\).
    \end{enumerate}
\end{eg}

\begin{theorem}[Artin's]
  \label{thm:artin}
  Let \(K \leq L\) be a field extension and \(H \leq \aut_K(L)\) be a finite subgroup. Let \(M = L^H\). Then \(M \leq L\) is a finite Galois extension and \(H = \gal(L/M)\).
\end{theorem}

\begin{remark}
  This implies some part of the Galois correspondence:
  \[
    H \to L^H \to \gal(L/H^H)
  \]
  where we get back to \(H\).
\end{remark}

\begin{proof}
  Take \(\alpha \in L\). We first show that \(|M(\alpha):M| \leq |H|\). Let \(\{\alpha_1, \dots, \alpha_n\}\) be the distinct images of \(\alpha\) under \(H\), i.e.\ \(\{\phi(\alpha): \alpha \in H\}\). Define \(g(t) = \prod_{i = 1}^{\infty} (t - \alpha_i) \). Each \(\phi\) induces an endomorphism on \(L[t]\) under which \(g(t)\) is invariant since \(\phi\) permutes the \(\alpha_i\). Thus the coefficients lie in \(L^H = M\) and so \(g(t) \in M[t]\). By definition \(g(\alpha) = 0\) since \(\alpha\) is one of the \(\alpha_i\). Hence the minimal polynomial \(f_\alpha(t)\) of \(\alpha\) over \(M\) divides \(g(t)\). Thus
  \[
    |M(\alpha):M| = \deg f_\alpha(t) \leq \deg g(t) \leq |H|.
  \]
  This step shows that \(\alpha\) is algebraic over \(M\) and \(f_\alpha(t)\) is seperable whenver \(g(t)\) is. Thus \((M \leq L\) is a separable extension.

  Next we show that \(M \leq L\) is a simple extension. Pick \(\alpha \in L\) with \(|M(\alpha):m|\) maximal. We will show that \(L = M(\alpha)\) for this \(\alpha\). Suppose \(\beta \in L\). Then \(M \leq M(a=\alpha, \beta)\) is finite and separably generated and hence is a finite separable extension. By \nameref{thm:primitive} \(M(\alpha, \beta) = M(\gamma)\) for some \(\gamma\). But then
  \[
    M \leq M(\alpha) \leq M(\gamma)
  \]
  so by the maximality of \(|M(\alpha):M|\), \(M(\alpha) = M(\gamma)\). Thus \(\beta \in M(\gamma) = M(\alpha)\) so \(L = M(\alpha)\).

  It follows that \(|L:M| \leq |H|\).

  Finally,
  \[
    |L:M| = |M(\alpha):M| \leq |H| \leq |\aut_M(L)| \leq |L:M| \leq |L:M|
  \]
  so we must have equality throughtout, i.e.\ \(|L:M| = |\aut_M(L)| = |H|\) so \(M \leq L\) is a finite Glois extension and \(H = \gal(L/M)\).
\end{proof}

\begin{theorem}
  Let \(K \leq L\) be a finite field extension. TFAE:
  \begin{enumerate}
  \item \(K \leq L\) is Galois,
  \item \(L^H = K\) when \(H = \aut_K(L)\).
  \end{enumerate}
\end{theorem}

\begin{remark}
  The theorem allows some authors to give yet another definition of a Galois extension.
\end{remark}

\begin{proof}\leavevmode
  \begin{enumerate}
  \item \(1 \Rightarrow 2\): Let \(M = L^H\) where \(H = \aut_K(L)\). By \nameref{thm:artin} \(M \leq L\) is a Galois extension. Now we have two Galois extensions, giving the equality
    \begin{align*}
      |H| &= |\gal_K(L)| = |L:K| \\
          &= |\gal_M(L)| = |L:M|
    \end{align*}
    As \(K \leq L^H = M\), equality.
  \item \(2 \Rightarrow 1\): \nameref{thm:artin}.
  \end{enumerate}
\end{proof}

\begin{proof}[Proof of \nameref{thm:fundamental}]\leavevmode
  \begin{enumerate}
  \item Composing the maps
    \begin{align*}
      H &\mapsto L^H \\
      M &\mapsto \gal(L/M)
    \end{align*}
    gives \(H \to H\) by \nameref{thm:artin}. Composition the other way \(M \mapsto \gal(L/M) \mapsto L^H\) where \(H = \gal(L/M)\) gives \(M\) since \(M \leq L^H\) and
    \[
      |L:L^H| = |H| = |\gal(L/M)| = |L:M|.
    \]
  \item Take \(H \leq \gal(L/K)\). Then for \(\phi \in \gal(L/K)\), \(L^{\phi H \phi^{-1}} = \phi(L^H)\) so by 1 \(H \normal \gal(L/K)\) if and only if \(\phi(L^H) = L^H\). Let \(M = L^H\). We will show that \(K \leq M\) is normal if and only if \(\phi(M) = M\) for every \(\phi \in \gal(L/K)\):
    \begin{itemize}
    \item \(\Rightarrow\): To be filled in
      %
      %
      %
      %
      %
      %
    \end{itemize}

  \end{enumerate}
\end{proof}

\section{Galois Group of Polynomials}

\begin{definition}[Galois group]
  Let \(f(t) \in K[t]\) be a separable polynomial and \(K \leq L\) with \(L\) a splitting field for \(f(t)\). Then the \emph{Galois group} of \(f(t)\) over \(K\) is
  \[
    \gal(f) = \gal(L/K).
  \]
\end{definition}

Since \(L\) is a splitting field for \(f(t)\), \(L = K(\alpha_1, \dots, \alpha_n)\) where \(\alpha_1, \dots, \alpha_n\) are the roots of \(f(t)\) in \(L\). Observe that if \(\phi \in \gal(L/K)\) it maps bijectively roots of \(f(t)\) to itself. Thus \(\phi\) permutes the \(\alpha_i\).

Moreover if \(\phi\) fixes each \(\alpha_i\) it also fixes all elements of \(L\) and so it is the identity map. Thus \(\gal(f)\) may be regarded as a permutation group of the \(n\) roots, so in particular admitting a permutation representation in \(S_n\).

\begin{lemma}
  \label{lem:galois orbits}
  Suppose seperable \(f(t) = g_1(t) \cdots g_s(t)\) with \(g_i(t)\) irreducible in \(K[t]\) is a factorisation in \(K[t]\). Then the orbits of \(\gal(f)\) acting on the roots of \(f(t)\) correspond to the factors \(g_j(t)\): two roots are in the same orbit if and only if they are roots of the same \(g_i(t)\).

  In particular if \(f(t) \in K[t]\) is irreducible, there is only one orbit, i.e.\ \(\gal(f)\) acts transitively on the roots of \(f(t)\).
\end{lemma}

\begin{proof}
  Let \(\alpha_k\) and \(\alpha_\ell\) be in the same orbit under \(\gal(f)\). Thus there is \(\phi \in \gal(f)\) with \(\alpha_\ell = \phi(\alpha_k)\). But if \(\alpha_k\) is a root of \(g_j(t)\) then \(\phi(\alpha_k)\) is also a root of \(g_j(t)\).

  Conversely if \(\alpha_k\) and \(\alpha_\ell\) are roots of \(g_j(t)\), then
  \[
    K(\alpha_k) \cong K[t]/(g_j(t)) \cong K(\alpha_\ell) \leq L
  \]
  Denote by \(\phi_0\) the isomorphsim \(K(\alpha_k) \to K(\alpha_\ell)\). \(\phi_0\) extends to \(\phi \in \gal(L/K)\). Thus \(\alpha_k\) and \(\alpha_\ell\) are in the same orbit.
\end{proof}

\begin{lemma}
  The transitive subgroups of \(S_n\) for \(n \leq 5\) are
  \[
  \begin{array}[h]{c|l}
    n & \\ \hline
    2 & S_2 \cong C_2 \\ \hline
    3 & A_3 \cong C_3, S_3 \\ \hline
    4 & C_4, V_4, D_8, A_4, S_4 \\ \hline
    5 & C_5, D_{10}, H_{20}, A_5, S_5 \\ \hline
  \end{array}
  \]
  where \(H_{20}\) is generated by a \(5\)-cycle and a \(4\)-cycle.
\end{lemma}

\begin{theorem}
  Let \(p\) be a prime and \(f(t) \in \Q[t]\) of degree \(p\). Suppose \(f(t)\) has exactly \(2\) non-real roots in \(\C\). Then
  \[
    \gal(f) \cong S_p.
  \]
\end{theorem}

\begin{proof}
  \(\gal(f)\) acts on the \(p\) distincts roots of \(f(t)\) in a splitting field \(L\) of \(f(t)\) (in \(\C\)). By \Cref{lem:galois orbits} the irreducibility of \(f(t)\) implies that \(\gal(f)\) acts transitively on \(p\) roots so by orbit-stabiliser \(p \divides |\gal(f)|\) but
  \[
    |\gal(f)| \leq |S_p| = p!
  \]
  and so \(\gal(f)\) has a Sylow \(p\)-subgroup of order \(p\), necessarily cyclic. Thus \(\gal(f)\) contains a \(p\)-cycle. Supposition that there are exactly \(2\) non-real roots gives that complex conjugation yields a transposition in \(\gal(f)\). The \(p\)-cycle and the transposition generate \(S_p\).
\end{proof}

\begin{eg}
  Let \(f(t) = t^5 - 6t + 3 \in \Q[t]\), then claim \(\gal(f) \cong S_5\):

  \begin{proof}
    \(f(t)\) is irreducible by Einsenstein with \(p = 3\). We want to show that \(f(t)\) has \(3\) real roots (so \(2\) non-real roots) and apply the previous theorem.

    Now we apply knowledge from analysis:
    \[
      f(-2) = -17, f(-1) = 8, f(1) = -2, f(2) = 23
    \]
    and \(f'(t) = 5t^4 - 6\) which has two real roots. Thus by Intermediate Value Theorem there are \(3\) real roots while by Rolle's Theorem there are at most \(3\) real roots.
  \end{proof}
\end{eg}

\begin{definition}[Discriminant]
  Let \(f(t) \in K[t]\) with distinct roots \(\alpha_1, \dots, \alpha_n\) (in a splitting field) (\(f(t)\) need not be irreducible). Set \(\Delta = \prod_{i < j}(\alpha_i - \alpha_j)\). The the \emph{discriminant} \(D = D(f)\) of \(f\) is
  \[
    D(f) = \Delta^2 = \prod_{i < j}(\alpha_i - \alpha_j)^2 = (-1)^{\frac{n(n-1)}{2}} \prod_{i \neq j}(\alpha_i - \alpha_j).
  \]
\end{definition}

\begin{remark}
  We have already met this in the proof of~\ref{tbf}.
\end{remark}

\begin{lemma}
  Let \(f(t) \in K[t]\) be seperable of degree \(n\) with \(\ch K \neq 2\). Then
  \[
    \gal(f) \leq A_n \Leftrightarrow D(f) \text{ is a square in } K.
  \]
\end{lemma}

\begin{proof}
  Let \(L\) be a splitting field of \(f(t)\) over \(K\). Then \(D(f) \neq 0\) and is fixed by all elements of \(G = \gal(L/K)\) as the latter permutes the roots. Thus \(D \in K\) since \(L^G = K\) by Galois correspondence.

  If \(\sigma \in G\) then \(\sigma(\Delta) = (\sgn \sigma) \Delta\) where we are regarding \(G\) as a subgroup of \(S_n\) and \(\sgn\) is the signature (this is where we need \(\ch K \neq 2\)). Thus if \(G \leq A_n\) we got that \(\Delta\) is fixed by all \(\sigma \in G\). Thus \(\Delta \in K = L^G\).

  On the other hand if \(G \nleq A_n\) we get \(\sigma(\Delta) = -\Delta\) if \(\sigma\) is odd and so \(\Delta \notin K = L^G\). Finally note that if \(D\) has square roots they must be \(\pm \Delta\).
\end{proof}

\begin{eg}\leavevmode
  \begin{enumerate}
  \item \(n = 2\): \(f(t) =t^2 + bt + c = (t - \alpha_1)(t - \alpha_2)\) has determinant
    \[
      D(f) = (\alpha_1 - \alpha_2)^2 - (\alpha_1 + \alpha_2)^2 - 4\alpha_1\alpha_2 = b^2 - 4c
    \]
  \item \(n = 3\): \(f(t) = t^3 + ct + d\) has determinant
    \[
      D(t) = -4c^3 - 27d^2
   \]
   \begin{remark}
     Any general monic cubic \(g(t)\) can be put into this form by a suitable substitution \(f(t) = g(t_1 + \lambda)\) for suitable \(\lambda\). Note \(D(f) = D(g)\).
   \end{remark}
 \end{enumerate}
\end{eg}

\begin{eg}\leavevmode
  \begin{itemize}
  \item \(f(t) = t^3 - t - 1 \in \Q[t]\) irreducible in \(\Z[t]\) since irreducible mod \(2\). \(D(f) = -23\) which is not a square in \(\Q\). Thus \(\gal(f) \cong S_3\).
  \item \(f(t) = t^3 - 3t - 1 \in \Q[t]\) irredicible since irreducible mod \(2\). \(D(f) = 81\) which is a square so \(\gal(f) \cong A_3\).
  \end{itemize}
\end{eg}

Now we move on to irreducible quartics. We saw that the possible Galois groups are
\[
  C_4, V_4, D_8, A_4, S_4
\]
with the first two being subgroups of \(A_4\). From looking at the discriminant one gets information as whether the group is one of \(V_4, A_4\) or one of \(C_4, D_8, S_4\). We need further methods to pin down which group we are dealing with.

\begin{theorem}[mod \(p\) reduction]
  \label{thm:mod p reduction}
  Let \(f(t) \in \Z[t]\) be monic of degree \(n\) with \(n\) distinct roots in a splitting field. Let \(p\) be a prime such that \(\red f(t)\), the reduction of \(f(t)\) mod \(p\), also has \(n\) distinct roots in a splitting field. Let \(\red f(t) = \red{g_1}(t) \dots \red{g_s}(t)\) be the factorisation into irreducible in \(\F_p[t]\) with \(n_j =  \deg \red{g_j}(t)\). Then
  \[
    \gal(\red f) \hookrightarrow \gal(f)
  \]
  and has an element of cycle type \((n_1, n_2, \dots, n_s)\).
\end{theorem}

\begin{proof}
  I will talk about last line once we have thought about Galois groups over finite fields. The fact that \(\gal(\bar f) \hookrightarrow \gal(f)\) is from Number Fields. Look at Tony Scholl's teaching page on Galois Theory.
\end{proof}

\begin{eg}
  Given a quartic of the from \(f(t) = t^4 + dt + e\), its determinant is \(D(t) = -27d^4 + 256e^3\). Consider \(f(t) = t^4 - t -1\), irreducible since irreducible mod \(2\) and \(D(f) = -283\) which is not a square.

  Consider mod \(7\),
  \[
    \red f(t) = t^4 - t - 1 = (t + 4)(t^3 + 3t^2 + 2t + 5)
  \]
  the second factor is irreducible over \(\F_7\) since it has no roots in \(\F_7\). By \Cref{thm:mod p reduction} \(\gal(f)\) contains an element of cycle type \((1, 3)\), i.e.\ a \(3\)-cycle. We deduce that \(\gal(f) \cong S_4\) as the other possibilities that contain an odd permutation do not contain a \(3\)-cycle.
\end{eg}

\section{Galois Theory of Finite Fields}

Recall what we already know from \Cref{sec:section 1}: a finite field \(\F\) is of characteristic \(p > 0\) where \(p\) is a prime and \(|\F| = p^r\) for some \(r\). The multiplicative group of \(\F\) is cyclic. It is a splitting field for \(t^n - 1\) over \(\F_p\) where \(n = p^r - 1\). By the uniqueness of splitting fields this is unique. Observe we could also describe \(\F\) as the splitting field of \(t^{p^r} - t\) over \(\F_p\).

What we haven't shown yet is that for any \(p^r\) there is a field with \(|\F| = p^r\).

\begin{definition}[Frobenius automorphism]
  Let \(\F\) be a finite field of characteristic \(p\). The \emph{Frobenius automorphism} of \(\F\) is
  \begin{align*}
    \phi_p: \F &\to \F \\
    \alpha &\mapsto \alpha^p
  \end{align*}
\end{definition}

\begin{remark}
  \((\alpha + \beta)^p = \alpha^p + \beta^p\) since all terms in binomial expansion is divisible by \(p\). Also \(\F_p\) is fixed under this so this is an \(\F_p\)-automorphism.
\end{remark}

Since \(t^{p^r} - t\) splits as a product of linear factors \((t - \alpha)\) in \(\F\), we have that \(\F_p \leq \F\) is a Galois extension and so we consider \(G = \gal(\F/\F_p)\). It is of order \(r\) since \(|\F:\F_p| = r\).

\begin{theorem}[Galois group of finite fields]
  Let \(\F\) be a finite field with \(|\F| = p^r\). Then \(\F_p \leq \F\) is a Galois extension with
  \[
    \gal(\F/\F_p) = \generation{\phi_p} \cong C_r.
  \]
\end{theorem}

\begin{proof}
  It remains to show that the order of Frobenius automorphism is \(r\). Suppose \(\phi_p^s = \id\). Then \(\alpha^{p^s} = \alpha\) for all \(\alpha \in \F\). But \(t^{p^s} - t\) has at most \(p^s\) roots in \(\F\). So we conclude \(s \geq r\). Observe that \(\phi_p^r = \id\) since \(\alpha^{p^r} = \alpha\) for all \(\alpha \in \F\).
\end{proof}

Now apply \nameref{thm:fundamental},
\[
  \{\text{intermediate fields } \F_p \leq M \leq \F\} \leftrightarrow \{\text{subgroups } H \leq G \}
\]
where \(G = \gal(\F/\F_p)\) is cyclic.

But we know all about subgroups of a cyclic group with generator \(\phi_p\) with order \(r\). There is exactly one group of order \(s\) for each \(s \divides r\) generated by \(\phi_p^{r/s}\). The corresponding intermediate subfields are the fixed fields \(\F^{\generation{\phi_p^{r/s}}}\) and \(|\F:\F^{\generation{\phi_p^{r/s}}}| = s\). By Tower Law \(|\F^{\generation{\phi_p^{r/s}}}:\F| = r/s\).

Observe that all subgroups of cyclic groups are normal and therefore all our intermediate fields are normal extensions of \(\F_p\). Thus \(\gal(\F^{\generation{\phi_p^{r/s}}}/\F_p) \cong \gal(\F/\F_p)/H\) where \(H = \generation{\phi_p^{r/s}}\).

\begin{corollary}
  Let \(\F_p \leq M \leq \F\) be finite fields. Then \(\gal(\F/M)\) is cyclic, generated by \(\phi_p^n\) where \(\phi_p\) is the Frobenius map and \(|M| = p^n\) and \(M\) is the fixed field of \(\generation{\phi_p^n}\).
\end{corollary}

\begin{proof}
  Set \(n = r/s\).
\end{proof}

\begin{theorem}[Existence of finite fields]
  Let \(p\) be a prime and \(n \geq 1\). Then there is a field of order \(p^n\) unique up to isomorphism.
\end{theorem}

\begin{proof}
  Consider the splitting field \(L\) of \(f(t) = t^{p^n} - t\) over \(\F_p\). \(\F_p \leq L\) is a finite Galois extension. However the roots of \(f(t)\) form a field, the fixed field of \(\phi_p^n\). Set \(L = \F\) and \(|\F:\F_p| = n\).
\end{proof}
\end{document}