\documentclass[a4paper]{article}

\def\npart{II}

\def\ntitle{Differential Geometry}
\def\nlecturer{M.\ Dafermos}

\def\nterm{Lent}
\def\nyear{2018}

\ifx \nauthor\undefined
  \def\nauthor{Qiangru Kuang}
\else
\fi

\ifx \ntitle\undefined
  \def\ntitle{Template}
\else
\fi

\ifx \nauthoremail\undefined
  \def\nauthoremail{qk206@cam.ac.uk}
\else
\fi

\ifx \ndate\undefined
  \def\ndate{\today}
\else
\fi

\title{\ntitle}
\author{\nauthor}
\date{\ndate}

%\usepackage{microtype}
\usepackage{mathtools}
\usepackage{amsthm}
\usepackage{stmaryrd}%symbols used so far: \mapsfrom
\usepackage{empheq}
\usepackage{amssymb}
\let\mathbbalt\mathbb
\let\pitchforkold\pitchfork
\usepackage{unicode-math}
\let\mathbb\mathbbalt%reset to original \mathbb
\let\pitchfork\pitchforkold

\usepackage{imakeidx}
\makeindex[intoc]

%to address the problem that Latin modern doesn't have unicode support for setminus
%https://tex.stackexchange.com/a/55205/26707
\AtBeginDocument{\renewcommand*{\setminus}{\mathbin{\backslash}}}
\AtBeginDocument{\renewcommand*{\models}{\vDash}}%for \vDash is same size as \vdash but orginal \models is larger
\AtBeginDocument{\let\Re\relax}
\AtBeginDocument{\let\Im\relax}
\AtBeginDocument{\DeclareMathOperator{\Re}{Re}}
\AtBeginDocument{\DeclareMathOperator{\Im}{Im}}
\AtBeginDocument{\let\div\relax}
\AtBeginDocument{\DeclareMathOperator{\div}{div}}

\usepackage{tikz}
\usetikzlibrary{automata,positioning}
\usepackage{pgfplots}
%some preset styles
\pgfplotsset{compat=1.15}
\pgfplotsset{centre/.append style={axis x line=middle, axis y line=middle, xlabel={$x$}, ylabel={$y$}, axis equal}}
\usepackage{tikz-cd}
\usepackage{graphicx}
\usepackage{newunicodechar}

\usepackage{fancyhdr}

\fancypagestyle{mypagestyle}{
    \fancyhf{}
    \lhead{\emph{\nouppercase{\leftmark}}}
    \rhead{}
    \cfoot{\thepage}
}
\pagestyle{mypagestyle}

\usepackage{titlesec}
\newcommand{\sectionbreak}{\clearpage} % clear page after each section
\usepackage[perpage]{footmisc}
\usepackage{blindtext}

%\reallywidehat
%https://tex.stackexchange.com/a/101136/26707
\usepackage{scalerel,stackengine}
\stackMath
\newcommand\reallywidehat[1]{%
\savestack{\tmpbox}{\stretchto{%
  \scaleto{%
    \scalerel*[\widthof{\ensuremath{#1}}]{\kern-.6pt\bigwedge\kern-.6pt}%
    {\rule[-\textheight/2]{1ex}{\textheight}}%WIDTH-LIMITED BIG WEDGE
  }{\textheight}% 
}{0.5ex}}%
\stackon[1pt]{#1}{\tmpbox}%
}

%\usepackage{braket}
\usepackage{thmtools}%restate theorem
\usepackage{hyperref}

% https://en.wikibooks.org/wiki/LaTeX/Hyperlinks
\hypersetup{
    %bookmarks=true,
    unicode=true,
    pdftitle={\ntitle},
    pdfauthor={\nauthor},
    pdfsubject={Mathematics},
    pdfcreator={\nauthor},
    pdfproducer={\nauthor},
    pdfkeywords={math maths \ntitle},
    colorlinks=true,
    linkcolor={red!50!black},
    citecolor={blue!50!black},
    urlcolor={blue!80!black}
}

\usepackage{cleveref}



% TODO: mdframed often gives bad breaks that cause empty lines. Would like to switch to tcolorbox.
% The current workaround is to set innerbottommargin=0pt.

%\usepackage[theorems]{tcolorbox}





\usepackage[framemethod=tikz]{mdframed}
\mdfdefinestyle{leftbar}{
  %nobreak=true, %dirty hack
  linewidth=1.5pt,
  linecolor=gray,
  hidealllines=true,
  leftline=true,
  leftmargin=0pt,
  innerleftmargin=5pt,
  innerrightmargin=10pt,
  innertopmargin=-5pt,
  % innerbottommargin=5pt, % original
  innerbottommargin=0pt, % temporary hack 
}
%\newmdtheoremenv[style=leftbar]{theorem}{Theorem}[section]
%\newmdtheoremenv[style=leftbar]{proposition}[theorem]{proposition}
%\newmdtheoremenv[style=leftbar]{lemma}[theorem]{Lemma}
%\newmdtheoremenv[style=leftbar]{corollary}[theorem]{corollary}

\newtheorem{theorem}{Theorem}[section]
\newtheorem{proposition}[theorem]{Proposition}
\newtheorem{lemma}[theorem]{Lemma}
\newtheorem{corollary}[theorem]{Corollary}
\newtheorem{axiom}[theorem]{Axiom}
\newtheorem*{axiom*}{Axiom}

\surroundwithmdframed[style=leftbar]{theorem}
\surroundwithmdframed[style=leftbar]{proposition}
\surroundwithmdframed[style=leftbar]{lemma}
\surroundwithmdframed[style=leftbar]{corollary}
\surroundwithmdframed[style=leftbar]{axiom}
\surroundwithmdframed[style=leftbar]{axiom*}

\theoremstyle{definition}

\newtheorem*{definition}{Definition}
\surroundwithmdframed[style=leftbar]{definition}

\newtheorem*{slogan}{Slogan}
\newtheorem*{eg}{Example}
\newtheorem*{ex}{Exercise}
\newtheorem*{remark}{Remark}
\newtheorem*{notation}{Notation}
\newtheorem*{convention}{Convention}
\newtheorem*{assumption}{Assumption}
\newtheorem*{question}{Question}
\newtheorem*{answer}{Answer}
\newtheorem*{note}{Note}
\newtheorem*{application}{Application}

%operator macros

%basic
\DeclareMathOperator{\lcm}{lcm}

%matrix
\DeclareMathOperator{\tr}{tr}
\DeclareMathOperator{\Tr}{Tr}
\DeclareMathOperator{\adj}{adj}

%algebra
\DeclareMathOperator{\Hom}{Hom}
\DeclareMathOperator{\End}{End}
\DeclareMathOperator{\id}{id}
\DeclareMathOperator{\im}{im}
\DeclareMathOperator{\coker}{coker}
\DeclarePairedDelimiter{\generation}{\langle}{\rangle}

%groups
\DeclareMathOperator{\sym}{Sym}
\DeclareMathOperator{\sgn}{sgn}
\DeclareMathOperator{\inn}{Inn}
\DeclareMathOperator{\aut}{Aut}
\DeclareMathOperator{\GL}{GL}
\DeclareMathOperator{\SL}{SL}
\DeclareMathOperator{\PGL}{PGL}
\DeclareMathOperator{\PSL}{PSL}
\DeclareMathOperator{\SU}{SU}
\DeclareMathOperator{\UU}{U}
\DeclareMathOperator{\SO}{SO}
\DeclareMathOperator{\OO}{O}
\DeclareMathOperator{\PSU}{PSU}
\DeclareMathOperator{\Sp}{Sp}


%hyperbolic
\DeclareMathOperator{\sech}{sech}

%field, galois heory
\DeclareMathOperator{\ch}{ch}
\DeclareMathOperator{\gal}{Gal}
\DeclareMathOperator{\emb}{Emb}



%ceiling and floor
%https://tex.stackexchange.com/a/118217/26707
\DeclarePairedDelimiter\ceil{\lceil}{\rceil}
\DeclarePairedDelimiter\floor{\lfloor}{\rfloor}


\DeclarePairedDelimiter{\innerproduct}{\langle}{\rangle}

%\DeclarePairedDelimiterX{\norm}[1]{\lVert}{\rVert}{#1}
\DeclarePairedDelimiter{\norm}{\lVert}{\rVert}



%Dirac notation
%TODO: rewrite for variable number of arguments
\DeclarePairedDelimiterX{\braket}[2]{\langle}{\rangle}{#1 \delimsize\vert #2}
\DeclarePairedDelimiterX{\braketthree}[3]{\langle}{\rangle}{#1 \delimsize\vert #2 \delimsize\vert #3}

\DeclarePairedDelimiter{\bra}{\langle}{\rvert}
\DeclarePairedDelimiter{\ket}{\lvert}{\rangle}




%macros

%general

%divide, not divide
\newcommand*{\divides}{\mid}
\newcommand*{\ndivides}{\nmid}
%vector, i.e. mathbf
%https://tex.stackexchange.com/a/45746/26707
\newcommand*{\V}[1]{{\ensuremath{\symbf{#1}}}}
%closure
\newcommand*{\cl}[1]{\overline{#1}}
%conjugate
\newcommand*{\conj}[1]{\overline{#1}}
%set complement
\newcommand*{\stcomp}[1]{\overline{#1}}
\newcommand*{\compose}{\circ}
\newcommand*{\nto}{\nrightarrow}
\newcommand*{\p}{\partial}
%embed
\newcommand*{\embed}{\hookrightarrow}
%surjection
\newcommand*{\surj}{\twoheadrightarrow}
%power set
\newcommand*{\powerset}{\mathcal{P}}

%matrix
\newcommand*{\matrixring}{\mathcal{M}}

%groups
\newcommand*{\normal}{\trianglelefteq}
%rings
\newcommand*{\ideal}{\trianglelefteq}

%fields
\renewcommand*{\C}{{\mathbb{C}}}
\newcommand*{\R}{{\mathbb{R}}}
\newcommand*{\Q}{{\mathbb{Q}}}
\newcommand*{\Z}{{\mathbb{Z}}}
\newcommand*{\N}{{\mathbb{N}}}
\newcommand*{\F}{{\mathbb{F}}}
%not really but I think this belongs here
\newcommand*{\A}{{\mathbb{A}}}

%asymptotic
\newcommand*{\bigO}{O}
\newcommand*{\smallo}{o}

%probability
\newcommand*{\prob}{\mathbb{P}}
\newcommand*{\E}{\mathbb{E}}

%vector calculus
\newcommand*{\gradient}{\V \nabla}
\newcommand*{\divergence}{\gradient \cdot}
\newcommand*{\curl}{\gradient \cdot}

%logic
\newcommand*{\yields}{\vdash}
\newcommand*{\nyields}{\nvdash}

%differential geometry
\renewcommand*{\H}{\mathbb{H}}
\newcommand*{\transversal}{\pitchfork}
\renewcommand{\d}{\mathrm{d}} % exterior derivative

%number theory
\newcommand*{\legendre}[2]{\genfrac{(}{)}{}{}{#1}{#2}}%Legendre symbol

%algebraic geometry
\DeclareMathOperator{\Spec}{Spec}
\DeclareMathOperator{\Proj}{Proj}

\DeclareMathOperator{\codim}{codim}
\DeclareMathOperator{\Int}{Int}
\newcommand*{\inner}{\innerproduct}

\begin{document}

\begin{titlepage}
  \begin{center}
    \includegraphics[width=0.6\textwidth]{logo.jpg}\par
    \vspace{1cm}
    {\scshape\huge Mathamatics Tripos \par}
    \vspace{2cm}
    {\huge Part \npart \par}
    \vspace{0.6cm}
    {\Huge \bfseries \ntitle \par}
    \vspace{1.2cm}
    {\Large\nterm, \nyear \par}
    \vspace{2cm}
    
    {\large \emph{Lectures by } \par}
    \vspace{0.2cm}
    {\Large \scshape \nlecturer}
    
    \vspace{0.5cm}
    {\large \emph{Notes by }\par}
    \vspace{0.2cm}
    {\Large \scshape \href{mailto:\nauthoremail}{\nauthor}}
 \end{center}
\end{titlepage}

\tableofcontents

\section{Smooth Manifolds}

\subsection{Definitions}

\begin{definition}[Smooth]\index{smooth}
  Let \(U \subseteq \R^n\) be an open set and \(f: U \to \R^m\). \(f\) is said to be \emph{smooth} if
  \[
    \frac{\p^{|\alpha|} f}{\p x^\alpha}
  \]
  exists for all multi-indices \(\alpha\).
\end{definition}

\begin{notation}
  A multi-index \(\alpha = (\alpha_1, \dots, \alpha_n)\) is a tuple of non-negative integers and \(|\alpha| = \sum_{i = 1}^n \alpha_i\).
\end{notation}

\begin{definition}[Smooth]\index{smooth}
  Let \(X \subseteq \R^n\) be a subset. \(f: X \to \R^m\) is \emph{smooth} if for all \(x \in X\), there exists \(U \subseteq \R^n\), \(x \in U\) such that there exists smooth \(\tilde f: U \to \R^m\) extending \(f|_{X \cap U}\). i.e.\ the following diagram commutes
  \[
    \begin{tikzcd}
      U \ar[r, "\tilde f"] & \R^m \\
      U \cap X \ar[u, "\iota"] \ar[ur, "f"']
    \end{tikzcd}
  \]
\end{definition}

\begin{remark}\leavevmode
  \begin{enumerate}
    \item This is a local property.
    \item \(X\) is a topological space with subspace topology induced from \(\R^n\). \(f\) is smooth implies that \(f\) is continuous.
    \end{enumerate}
\end{remark}

\begin{definition}[Diffeomorphism]\index{diffeomorphism}
  Let \(X \subseteq \R^n\) and \(Y \subseteq \R^m\). A map \(f: X \to Y\) is a \emph{diffeomorphism} if \(f\) is smooth and bijective and its inverse \(f^{-1}: Y \to X\) is also smooth.
\end{definition}

\begin{remark}\leavevmode
  \begin{enumerate}
  \item Diffeomorphism implies homeomorphism (with repsect to the subspace topology).
  \item Diffeomorphism is an equivalence relation.
  \end{enumerate}
\end{remark}

\begin{definition}[Manifold]\index{manifold}
  A \emph{\(k\)-dimensional manifold} is a set \(X \subseteq \R^N\) such that for all \(x \in X\) there exists \(V \subseteq X\) open, \(x \in V\) such that \(V\) is diffeomorphic to an open subset \(U \subseteq \R^k\), i.e.\ \(X\) is locally diffeomorphic to \(\R^k\).
\end{definition}

\begin{remark}
  The diffeomorphism \(\varphi: U \to V\) is called a \emph{local parameterisation} and its inverse \(\varphi^{-1}: V \to U\) is called \emph{coordinate charts}. More specifically, let \(x_i: \R^k \to \R\) be the projection of the \(i\)th coordinate and we have \emph{local coordinate} \(x_i \compose \varphi^{-1}: V \to \R\). However, sometimes we abuse the notation and just write \(x_i: V \to \R\) for the above map.
\end{remark}

\begin{notation}
  Write \(\dim X = k\) if \(X\) is a \(k\)-dimensional manifold.
\end{notation}

\begin{eg}\leavevmode
  \begin{enumerate}
  \item \(\R^N\) is an \(N\)-dim manifold. For example, \(\varphi: \R^N \stackrel{\id}{\to} \R^N\).
  \item An open subset \(V \subseteq \R^N\) is an \(N\)-dim manifold, for example by taking \(\varphi\) to be the restriction \(\id|_V\).
  \item In general, if \(X\) is a manifold and \(V \subseteq X\) is an open subset then \(V\) is also a manifold.
  \item The \(n\)-sphere, \(S^n \subseteq \R^{n + 1} = \{x_1^2 + \dots + x_{n + 1}^2 = 1\}\) is an \(n\)-dim manifold. Suppose \(x \in S^n\) lies in the upper half hyperplane define by \(x_{n + 1} > 0\). Take
    \[
      V = S^n \cap \{x_{n + 1} > 0\},\, U = \{(x_1, \dots, x_n): \sum_{i = 1}^n x_i^2 < 1\} = B(\V 0, 1) \subseteq \R^n
    \]
    and define
    \begin{align*}
      \varphi: U &\to V \\
      (x_1, \dots, x_n) &\mapsto \left(x_1, \dots, x_n, +\sqrt{1 - (x_1^2 + \dots + x_n^2)}\right)
    \end{align*}
    Let's do a reality check: the map consists of algebraic operations so is smooth. It is easy to check that the image lies in \(V\). Its inverese is the projection onto the first \(n\) coordinates, which is the restriction of a smooth function so also smooth.

    Similarly, if \(x\) lies in \(\{x_{n + 1} < 0\}\), change \(+\) to \(-\) in the last coordinate of \(\varphi\) will do. This accounts for ``most'' points, with the exception of those on the equator. However, there is nothing special about \(x_{n + 1}\) so we may repeat for the other coordinates. As
    \[
      S^n = \bigcup_{i = 1}^{n + 1} \left( (S^n \cap \{x_i > 0\}) \cup (S^n \cap \{x_i < 0\}) \right)
    \]
    (which in English says that at least one coordinate is non-zero), we have covered \(S^n\) so done.
  \end{enumerate}
\end{eg}

\begin{ex}
  Suppose \(X \subseteq \R^n\) and \(Y \subseteq \R^m\) are manifolds. Show that \(X \times Y \subseteq \R^n \times \R^m = \R^{n + m}\) is a manifold of dimension \(\dim X + \dim Y\).
\end{ex}

\begin{definition}[Submanifold]\index{manifold!submanifold}
  Let \(X \subseteq \R^N\) be a manifold. A manifold \(Y \subseteq \R^N\) such that \(Y \subseteq X\) is a \emph{submanifold} of \(X\).
\end{definition}

\begin{remark}
  By definition, all manifolds are submanifolds of \(\R^N\) for some \(N\).
\end{remark}

Note that we have yet shown that the dimension of a manifold is well-defined but we will do so in a while. Assuming so, we define

\begin{definition}[Codimension]\index{codimension}
  The \emph{codimension} of \(Y\) in \(X\) is
  \[
    \codim_X(Y) = \dim X - \dim Y.
  \]
\end{definition}

\subsection{Tangent space}

\begin{definition}[Differential]\index{differential}
  Let \(U \subseteq \R^n\) and \(f: U \to \R^m\) be a smooth map. The \emph{differential} of \(f\) at \(x \in U\), \(df_x\) is a linear map
  \begin{align*}
    df_x: \R^n &\to \R^m \\
    h &\mapsto \lim_{t \to 0} \frac{f(x + th) - f(x)}{t}
  \end{align*}
\end{definition}

Explicitly, \(df_x\) is represented by the matrix
\[
  \left( \frac{\p f}{\p x_j} \right) \text{ where } f =
  \begin{pmatrix}
    f_1 \\
    \vdots \\
    f_m
  \end{pmatrix}
  .
\]
It then follows that
\[
  df_x(h) = \left(\frac{\p f}{\p x_j}\right)
  \begin{pmatrix}
    h_1 \\
    \vdots \\
    h_n
  \end{pmatrix}
  .
\]

\begin{proposition}[Chain rule]
  Let \(f: U \to V, g: V \to \R^p\) be smooth maps where \(U \subseteq \R^n, V \subseteq \R^m\). Let \(x \in U\) and \(f(x) \in V\). Then
  \[
    d(g \compose f)_x = dg_{f(x)} \compose df_x.
  \]
\end{proposition}

The aim of this section is to define differentials of maps between manifolds. Before that we have to find where differential lives. Certainly in \(\R^n\) it is a linear map living in a linear space.

Let \(X \subseteq \R^N\) be a \(k\)-dim manifold and \(x \in X\). Let \(\varphi\) be a local parameterisation around \(x\). Wlog assume \(\varphi^{-1}(x) = 0\). Note that \(\varphi\) is defined on an open subset of \(\R^k\) so we can do calculus on it (this is false for \(\varphi^{-1}\) as \(V \subseteq \R^N\) may not be open). We can define
\[
  d\varphi_0: \R^k \to \R^N.
\]

\begin{definition}[Tangent space]\index{tangent space}
  The \emph{tangent space} of \(X\) at \(x\), denoted \(T_xX\), is \(d\varphi_0(\R^k)\), a subspace of \(\R^N\).
\end{definition}

For this to be a good definition, we need to show that it is independent of \(\varphi\) and \(\dim d\varphi_0(\R^k) = k\).

Suppose \(\varphi: U \to V\) and \(\tilde \varphi: \tilde U \to \tilde V\) are parameterisations around \(x\), again assumming wlog \(\varphi(0) = x = \tilde \varphi(0)\). By taking intersection we can assume \(V = \tilde V\). Write
\[
  \varphi = \tilde \varphi \compose \underbrace{(\tilde \varphi^{-1} \compose \varphi)}_{U \to \tilde U}.
\]
The term in the parenthesis is a map between open subsets of \(\R^k\) so we can do calculus on it. By chain rule
\[
  d\varphi_0 = d\tilde \varphi_0 \compose d(\tilde \varphi^{-1} \compose \varphi)_0.
\]
Note that \(\varphi^{-1} \compose \tilde \varphi\) is the inverse of \(\tilde \varphi^{-1} \compose \varphi\) so by applying chain rule to \(\id\), we get
\[
  \id = d(\varphi^{-1} \compose \tilde \varphi)_0 \compose d(\tilde \varphi^{-1} \compose \varphi)_0
\]
so both of them are invertible. Thus
\[
  d\varphi_0(\R^k) = d\tilde \varphi_0(\R^k).
\]

Next we want to show \(\dim d\varphi_0(\R^k) = k\). Certainly linear algebra tells us that it is at most \(k\). By definition there exists a smooth \(\psi: W \to \R^k\), where \(W \subseteq \R^N\) is an open subset containing \(x\) such that
\[
  \psi|_{X \cap W} = \varphi^{-1}.
\]
Assume wlog \(V \subseteq X \cap W\). Then
\[
  \id_U = \psi \compose \varphi
\]
so by chain rule
\[
  \id = d\psi_x \compose d\varphi_0.
\]
Thus \(d\varphi_0(\R^k)\) is \(k\)-dimensional.

\begin{remark}
  The functions \(\varphi^{-1} \compose \tilde \varphi\) are called \emph{transition functions} and will appear later in the course.
\end{remark}

\begin{corollary}
  The dimension of a manifold is well-defined.
\end{corollary}

\begin{ex}
  If \(X \subseteq \R^N\) is a manifold and \(Y\) is a submanifold of \(X\) then
  \[
    T_yY \leq T_yX
  \]
  for all \(y \in Y\). In particular \(\codim_X(Y) \geq 0\).
\end{ex}

\begin{eg}\leavevmode
  \begin{enumerate}
  \item \(T_x\R^N = \R^N\).
  \item Let \(X\) be an open subset of \(\R^N\). Then \(T_xX = \R^N\).
  \item Let
    \[
      X = \{(x_1, x_2, \dots, x_k, 0, \dots, 0)\} \subseteq \R^N
    \]
    which can be seen as the image of the embedding \(\varphi: \R^k \to \R^N\). This is a \(k\)-dim manifold. This is a linear map so \(d\varphi_x = \varphi\). Thus \(T_xX = X\). In general, this holds if \(\varphi\) is an injective linear map.
  \item Let \(S^n \subseteq \R^{n + 1}\). Given a point \(x \in S^n\) with \(x_{n + 1} > 0\), use the parameterisation
    \begin{align*}
      \varphi: B(\V 0, 1) &\to \R^{n + 1} \\
      (x_1, \dots, x_n) &\mapsto \left(x_1, \dots, x_n, + \sqrt{1 - x_1^2 - \dots - x_n^2}\right)
    \end{align*}
    Then
    \[
      \im d\varphi_{(x_1, \dots, x_n)} = \text{span}\left\{\frac{\p \varphi}{\p x_1}, \dots, \frac{\p \varphi}{\p x_n}\right\}.
    \]
    We have
    \[
      \frac{\p \varphi}{\p x_i} = \left(0, \dots, 1, \dots, 0, -\frac{x_i}{x_{n + 1}}\right)
    \]
    and we can verify that for all \(i\),
    \[
      \frac{\p \varphi}{\p x_i} \cdot (x_1, \dots, x_n, x_{n + 1}) = 0
    \]
    so \(T_xS^n\) is \(\{v: v \cdot x = 0\} \subseteq \R^{n + 1}\).
  \end{enumerate}
\end{eg}

Now we go on to define differential of maps between manifolds. Let \(f: X \to Y\) be a smooth map of manifolds where \(X \subseteq \R^n, Y \subseteq \R^m\). Let \(\varphi:U \to V\) and \(\tilde \varphi: \tilde U \to \tilde V\) be local parameterisations around \(x\) and \(f(x)\), where \(U \subseteq \R^k, \tilde U \subseteq \R^\ell\). wlog we assume \(\varphi(0) = x, \tilde \varphi(0) = f(x)\). We also assume \(f(V) \subseteq \tilde V\).

We can thus define a map \(\tilde \varphi^{-1} \compose f \compose \varphi: U \to \tilde U\) and do calculus on it. As \(T_xX = \im d\varphi_0\) and \(d\varphi_0\) is an injective map, we can define the inverse of its restriction to the image, denoted by
\[
  (d\varphi_0)^{-1}: T_xX \to \R^k.
\]

\begin{definition}[Differential]\index{differential}
  The \emph{differential map} of \(f\) at \(x \in X\), denoted \(df_x\), is a linear map \(df_x: T_xX \to T_{f(x)}Y\), defined as the composition
  \[
    d\tilde \varphi_0 \compose d(\tilde \varphi^{-1} \compose f \compose \varphi)_0 \compose (d\varphi_0)^{-1}.
  \]

\[
  \begin{tikzcd}
    T_xX \ar[r, "df_0"] & T_{f(x)}Y \\
    \R^k \ar[u, "d\varphi_0"] \ar[r, "d(\tilde \varphi^{-1} \compose f \compose \varphi)_0"'] & \R^\ell \ar[u, "d\tilde\varphi_0"']
  \end{tikzcd}
\]
\end{definition}

Now as everything else we defined in differential geometry, we have to check that it is independent of parameterisation. It will be a horrible job, and one that you would want to do only (at most) once in your lifetime.

Let \(\psi, \tilde \psi\) be another pair of parameterisations. Define \(\varphi\) in terms of transition maps
\[
  \varphi = \psi \compose (\psi^{-1} \compose \varphi).
\]
By chain rule,
\[
  d\varphi_0 = d\psi_0 \compose d(\psi^{-1} \compose \varphi)_0
\]
so
\[
  (d\varphi_0)^{-1} = (d(\psi^{-1} \compose \varphi)_0)^{-1} \compose (d\psi_0)^{-1} = d(\varphi^{-1} \compose \psi)_0 \compose (d\psi_0)^{-1}
\]
where the last equality comes from applying chain rule to 
\[
  (\varphi^{-1} \compose \psi) \compose (\psi^{-1} \compose \varphi) = \id.
\]
Thus
\begin{align*}
  df_x &= d\tilde \varphi_0 \compose d(\tilde \varphi^{-1} \compose f \compose \varphi)_0 \compose (d\varphi_0)^{-1} \\
       &= d\tilde \varphi_0 \compose d(\tilde \varphi^{-1} \compose f \compose \varphi)_0 \compose d(\varphi^{-1} \compose \psi)_0 \compose (d\psi_0)^{-1} \\
       &= d\tilde \varphi_0 \compose d(\tilde \varphi^{-1} \compose f \compose \psi)_0 \compose (d\varphi_0)^{-1} \\
       &= \cdots \\
       &= d\tilde \psi_0 \compose d(\tilde \psi^{-1} \compose f \compose \psi)_0 \compose (d\psi_0)^{-1}
\end{align*}
where the omitted lines are similar and are left as an exercise.

\begin{proposition}[Chain rule]
  Let \(X, Y, Z\) be manifolds, \(f: X \to Y, g: Y \to Z\) smooth. Then for all \(x \in X\),
  \[
    d(g \compose f)_x = dg_{f(x)} \compose df_x.
  \]
\end{proposition}

\begin{proof}
  Tedious exercise using transition maps and chain rule on maps between Euclidean spaces.
\end{proof}

\begin{theorem}[Inverse Function Theorem]\index{inverse function theorem}
  Let \(f: X \to Y\) be a smooth map between manifolds. If \(df_x: T_xX \to T_{f(x)}Y\) is an isomorphism (which implies that in particular \(\dim X = \dim Y\)), then \(f\) is a local diffeomorphism. In other words, there exists an open subset \(V \subseteq X\) such that \(f|_V: V \to Y\) is a diffeomorphism onto its image.
\end{theorem}

\begin{proof}
  This is easy using Inverse Function Theorem from analysis. Let \(\varphi, \tilde \varphi\) be local parameterisations around \(x\) and \(f(x)\) respectively. Then \(d(\tilde \varphi^{-1} \compose f \compose \varphi)_0\) is surjection \(\R^k \to \R^k\) so \(\tilde \varphi^{-1} \compose f \compose \varphi\) is a local diffeomorphism of Euclidean spaces. Thus \(f\) is a local diffeomorphism of manifolds.
\end{proof}

\begin{ex}
  Suppose \(Y\) is a submanifold of \(X\) and \(f: X \to Z\) is a smooth map between manifolds. Then for all \(y \in Y\), show
  \[
    d(f|_Y)_y = df_y|_{T_yY}.
  \]
\end{ex}

\begin{ex}
  Let \(f: X \to Y, g: X \to Z\) be smooth maps between manifolds. We can define map
  \begin{align*}
    (f, g): X &\to Y \times Z \\
    x &\mapsto (f(x), g(x))
  \end{align*}
  Show that this map is smooth and for all \(x \in X\),
  \[
    d(f, g)|_x = (df_x, dg_x).
  \]
  Thus if differential is thought as a matrix, the map to the product space has differential made of blocks of matrices.
\end{ex}

\subsection{Pre-image theorem}

We defined manifolds in terms of charts but in reality, the pre-image theorem actually gives the most convenient way to define a manifold. First define some terminologies:

\begin{definition}[Critial point, critial value, regular value]\index{critical point}\index{critical value}\index{regular value}
  Let \(f: X \to Y\) be a smooth map between manifolds.
  \begin{itemize}
  \item A \emph{critical point} of \(f\) is a point \(x\) such that \(df_x: T_xX \to T_{f(x)}Y\) fails to be surjective. We denote the set of all critical points \(C\).
  \item A \emph{critical value} of \(f\) is a point \(y \in Y\) such that there exists \(x \in X\) critical and \(f(x) = y\), i.e.\ the set of critical values is \(f(C) \subseteq Y\).
  \item \(y \in Y\) is a \emph{regular value} if \(y\) is not a critical value, i.e.\ \(y \in Y \setminus f(C)\).
  \end{itemize}
\end{definition}

\begin{remark}\leavevmode
  \begin{enumerate}
  \item If \(\dim X < \dim Y\) then \(C = X\).
  \item If \(f^{-1}(y) = \emptyset\), i.e.\ \(y \notin f(X)\) then \(y\) is a regular value.
  \end{enumerate}
\end{remark}

\begin{theorem}[Pre-image theorem]
  Let \(f: X \to Y\) be a smooth map between manifolds. Given \(y\) a regular value of \(f\), \(f^{-1}(y) \subseteq X\) is a submanifold of \(X\) with codimension equal to the dimension of \(Y\), i.e.
  \[
    \dim X - \dim (f^{-1}(y)) = \dim Y.
  \]
\end{theorem}

\begin{remark}
  Suppose \(X\) is non-empty (which we may add to our definition of manifolds). Then this theorem gives a proof that \(\codim \geq 0\) for this particular case.
\end{remark}

\begin{proof}
  Let \(x \in f^{-1}(y)\). Suppose \(X, Y \subseteq \R^N\) and \(q = \dim X - \dim Y\). By surjectivity and rank-nullity, \(\dim \ker df_x = g\). It is an exercise in linear algebra to show that there exists a linear map \(T: \R^N \to \R^q\) such that \(\ker T \cap \ker df_x = 0\). Now define
  \begin{align*}
    F: X &\to Y \times \R^q \\
    x &\mapsto (f(x), T(x))
  \end{align*}
  Then by the exercises above \(F\) is smooth with differential \(dF_x = (df_x, dT_x)\) and \(d(T|_X)_x = dT|_{T_xX} = T|_{T_xX}\) so \(dF_x = (df_x, T)\) which is an isomorphism. Thus by Inverse Function Theorem \(F\) is a diffeomorphism around \(x\). There exists an open neighbourhood of \(x\) \(V \subseteq X\) such that
  \[
    F|_V: V \to f(V) \times \underbrace{T(V)}_{= U \subseteq \R^q}
  \]
  is a diffeomorphism. Then
  \[
    \varphi = (F|_V)^{-1}|_{\{y\} \times U}: U \to V
  \]
  is a local parameterisation of \(f^{-1}(y)\).
\end{proof}

\begin{corollary}
  Suppose \(\dim X = \dim Y\) and \(y\) is a regular value of some \(f: X \to Y\). Then \(f^{-1}(y)\) is a manifold of dimension \(0\).
\end{corollary}

If \(Z \subseteq \R^N\) is a manifold of dimension \(0\), then it is a collection of discrete points, i.e.\ for each \(z \in Z\), there exists an open neighbourhood of \(z\) whose intersection with \(Z\) is \(\{z\}\). Thus if \(X\) is in addition compact, then \(f^{-1}(y)\) is a finite set of points.

\begin{theorem}[Stack of records theorem]
  Under the above assumptions, there exist open neighbourhoods \(W_i\) of \(x_i\)'s such that
  \[
    f^{-1}\left(\bigcap_{i = 1}^n f(W_i) \setminus f(X \setminus \bigcup_{i = 1}^n W_i)\right)
  \]
  is a disjoint union of open neighbourhoods of \(x_i\)'s and the restriction of \(f\) on each of which is a diffeomorphism.
\end{theorem}

Given a regular value \(y\), \(T_xf^{-1}(y) = \ker df_x\). See example sheet 1.

\begin{eg}\leavevmode
  \begin{enumerate}
  \item Let \(S^n \subseteq \R^{n + 1}\) and
  \begin{align*}
    f: S^n &\to \R \\
    (x_1, \dots, x_{n + 1}) &\mapsto x_1^2 + \dots + x_{n + 1}^2
  \end{align*}
  which is smooth. We have
  \[
    df = (2x_1, \dots, 2x_{n + 1})
  \]
  which is surjective everywhere on \(S^n\). Thus by pre-image theorem \(f^{-1}(1) \subseteq \R^{n + 1}\) is a manifold of dimension \(1\), which is precisely \(S^n\).
\item Let \(O(n)\) be the set of \(n \times n\) orthogonal matrices, i.e.\ matrices \(A\) such that \(AA^T = I\). \(O(n)\) can be seen as a subset of \(\matrixring_n(\R) = \R^{n^2}\). Claim that \(O(n)\) is a manifold of dimension \(\frac{n(n - 1)}{2}\).

  Let \(S(n) \subseteq \matrixring_n(\R)\) be the subset of symmetric matrices. This is a subspace of \(\R^{n^2}\) so clearly a submanifold of dimension \(\frac{n(n + 1)}{2}\). Consider that map
  \begin{align*}
    f: \matrixring_n &\to S(n) \\
    A &\mapsto AA^T
  \end{align*}
  which is a smooth map. As \(f^{-1}(I) = O(n)\), if \(I\) is a regular value of \(f\) then \(O(n)\) is a manifold of dimension \(\frac{n(n - 1)}{2}\).
  \begin{align*}
    df_A(H) &= \lim_{s \to 0} \frac{(A + sH)(A + sH)^T - AA^T}{s} \\
            &= AH^T + HA^T
  \end{align*}
  Given \(C \in S(n)\), \(H = \frac{CA}{2}\) satisfies \(df_A(H) = C\) so \(df_A\) is surjective.
  \end{enumerate}
\end{eg}

\begin{remark}
  In addition to being a manifold, \(O(n)\) is a group under matrix multiplication. The group operations are smooth. We call it a \emph{Lie group}. In this case, objects that preserve geometric objects (i.e.\ transformations) are themselves geometric objects.

  It is also worthing pointing out that \(O(n)\) is not connected, as seen from the continuous map \(\det: O(n) \to \{\pm 1\}\). The connected component of identity is called \emph{special orthogonal group}
  \[
    SO(n) = \{A \in O(n): \det A = 1\}.
  \]
\end{remark}

\begin{definition}
  A set \(S \subseteq \R^n\) is said to be \emph{measure \(0\)} if for all \(\varepsilon > 0\) there exists a countable collection of cubes \(C_m\) such that \(S \subseteq \bigcup C_m\) and \(\sum \operatorname{volume} C_m < \varepsilon\).
\end{definition}

\begin{definition}
  Let \(X\) be an \(n\)-dim manfiold and \(S \subseteq X\). \(S\) has \emph{measure \(0\)} if for all local parameterisations \(\varphi: U \to V \subseteq X\), \(\varphi^{-1}(V \cap S) \subseteq \R^n\) is of measure \(0\).
\end{definition}

This is well-defined since we can show any manifold can be parameterised by a chart of countably many parameterisations and the countable union of measure \(0\) subsets in \(\R^n\) has measure \(0\).

\begin{remark}\leavevmode
  \begin{enumerate}
  \item An open subset \(V \subseteq X\) is not of measure \(0\).
  \item Suppose \(Y \subseteq X\) is a submanifold of \(X\) and \(\dim Y < \dim Y\). Then \(Y\) is of measure \(0\) in \(X.\)
  \end{enumerate}
\end{remark}

\begin{theorem}[Sard's Theorem]
  Let \(f: X \to Y\) be a smooth map between manifolds. Then the set of critical values of \(f\) is of measure \(0\) in \(Y\).
\end{theorem}

\begin{corollary}
  Regular values are dense.
\end{corollary}

\subsection{Transversality}

Suppose \(f: X \to Y\) is a smooth map between manifolds and \(Z \subseteq Y\) is a submanifold.

\begin{definition}[Transversality]\index{transversality}
  \(f\) is \emph{transversal to \(Z\)}, denoted \(f \transversal Z\), if for all \(x \in f^{-1}(Z)\),
  \[
    \im df_x + T_{f(x)}Z = T_{f(x)}Y.
  \]
\end{definition}

\begin{remark}\leavevmode
  \begin{enumerate}
  \item If \(f(X) \cap Z = \emptyset\) then \(f\) is vacuously transversal to \(Z\).
  \item This is a generalisation of regular value: if \(Z = \{y\}\) then \(f \transversal Z\) if and only if \(y\) is a regular value of \(f\) since the tangent space of a dimension \(0\) manifold is \(0\).
  \end{enumerate}
\end{remark}

\begin{eg}
  Let \(X = \R, Y = \R^2\), \(f(t) = (0, t)\) and \(Z \subseteq Y\) is the \(x\)-axis. Then \(df_0 = \binom{0}{1}\) and \(T_{(w, 0)}Z = \operatorname{span}\{\binom{w}{0}\}\) so \(\Im df_0 + T_{(w, 0)}Z = \R^2\) so \(f \transversal Z\).
\end{eg}

\begin{eg}
  Let \(f(t) = (t, t^2)\). Then \(df_0 = \binom{1}{0}\) so
  \[
    \Im df_0 + T_{(w, 0)}Z \neq \R^2
  \]
  so \(f\) is not transversal to \(Z\).
\end{eg}

The corresponding generalisation of pre-image theorem is

\begin{theorem}[Transversality theorem]\index{transversality theorem}
  Suppose \(f: X \to Y\) is a smooth map between manifolds and \(Z \subseteq Y\) is a submanifold. If \(f \transversal Z\) then \(f^{-1}(Z) \subseteq X\) is a submanifold of \(X\) with codimension equal to \(\codim_Y(Z)\).
\end{theorem}

\begin{proof}
  It is enough to show that for all \(x \in Z\), there exists \(V \subseteq Z\) open such that \(f^{-1}(V)\) is a submanifold of \(X\), i.e.\ we can work locally to deduce global result. This is because if \(\{V_i\}_{i \in I}\) is a cover of \(Z\) then
  \[
    f^{-1}(Z) = \bigcup_{i \in I} f^{-1}(V_i).
  \]

  Given \(x \in Z\), let \(\codim_Y Z = k\). Claim that there exist \(V \subseteq Z\) open containing \(x\): there exists \(\tilde V \subseteq Y\) open such that
  \[
    V = Z \cap \tilde V =
    \begin{cases}
      h_1 = 0 \\
      h_2 = 0 \\
      \vdots \\
      h_k = 0
    \end{cases}
  \]
  where \(k_i: \tilde V \to \R\) and moreover \(Z \cap \tilde V = h^{-1}(0)\) and \(0\) is a regular value of \(h\). This is basically saying that a submanifold of codimension \(k\) can be expressed as the pre-image of \(0\) where \(0\) is a regular value. This is left as an exercise.

  Now we prove the theorem. wlog assume \(Z = Z \cap \tilde V\). Consider \(h \compose f: X \to \R^k\). Its differential is
  \[
    d(h \compose f)_x = dh_{f(x)} \compose df_x.
  \]
  Claim that \(0\) is a regular value of \(h \compose f\): \(x \in (h \compose f)^{-1}(0)\) if and only if \(x \in f^{-1}(Z)\). Recall that \(T_yZ = \ker dh_y\) (see example sheet). From transversality we know
  \[
    \Im df_x + T_{f(x)}Z = T_{f(x)}Y
  \]
  so
  \[
    \Im df_x + \ker dh_{f(x)} = T_{f(x)}Y.
  \]
  Apply \(dh_{f(x)}\) to both sides, we get
  \[
    \Im dh_{f(x)} \compose df_x = \Im dh_{f(x)} = \R^k
  \]
  where the last equality is by surjectivity.

  Therefore \((h \compose f)^{-1}(0) \subseteq X\) is a submanifold of codimension \(k\). So does \(f^{-1}(Z)\).
\end{proof}

Consider the special case where \(X\) and \(Z\) are both submanifolds of \(Y\), i.e.\ \(f = \iota: X \to Y\). Then \(\iota \transversal Z\) if and only if
\[
  \Im d\iota_x + T_{\iota(x)}Z = T_{\iota(x)}Y
\]
which can be rewritten as
\[
  T_xX + T_xZ = T_xY.
\]
In this case write \(X \transversal Z\) (which is clearly a symmetric relation).

\begin{eg}
  If \(f \transversal Z\), it is not necessarily the case that \(f(X) \transversal Z\). Consider \(X = \R, Y = \R^2\), \(f(t) = (0, t^2)\) and \(Z \subseteq Y\) is the \(x\)-axis. Then \(f^{-1}(Z) \cap X\) is a submanifold of \(X\) of dimension \(0\).
\end{eg}

\begin{corollary}
  If \(\iota^{-1}(Z) \subseteq X\), then \(X \cap Z\) is a submanifold of \(Y\) with codimension
  \[
    \codim_Y(X \cap Z) = \codim_Y X + \codim_Y Z.
  \]
\end{corollary}

\subsection{Manifolds with boundary}

\begin{notation}
  \(\H^n = \{(x_1, \dots, x_n): x_i \geq 0\} \subseteq \R^n\).
\end{notation}

\begin{definition}[Manifold with boundary]\index{manifold!manifold with boundary}
  \(X \subseteq \R^n\) is an \(n\)-dimensional \emph{manifold with boundary} if for all \(x \in X\) there exists \(V \subseteq X\) open, \(U \subseteq \H^n\) open and \(\varphi: U \to V\) a diffeomorphism.
\end{definition}

It is possible that \(U\) is also open in \(\R^n\).

\begin{definition}[Boundary, interior]\index{boundary}\index{interior}
  The \emph{boundary} of \(X\), denoted \(\p X\), is defined to be
  \[
    \p X = \{x: \exists \varphi \text{ as above such that } x = \varphi(x_1, \dots, x_{n - 1}, 0)\}.
  \]
  Equivalently, \(x \in \varphi(U \cap \p \H^n)\) where \(\p \H^n = \{(x_1, \dots, x_{n - 1}, 0)\}\).

  The \emph{interior} of \(X\), denoted \(\Int X\), is defined to be \(X \setminus \p X\).
\end{definition}



\begin{remark}\leavevmode
  \begin{enumerate}
  \item The above definition is consistent, i.e.\ independent of parameterisation: if \(x \in \varphi(U \cap \p H^n)\) for some \(\varphi\) then it holds for all \(\varphi\) such that \(x \in \Im \varphi\).
  \item \(\H^n\) is an \(n\)-dimensional manifold with boundary.
  \item Boundary of manifolds is \emph{not} the same as topological boundary, defined to be the set difference between the closure and the interior.
  \item For any manifold with boundary \(X\), \(\p X\) and \(\Int X\) are manifolds without boundary of dimension \(\dim X - 1\) and \(\dim X\) respectively. This can be seen by restricting image of \(\varphi\) to \(\p \H\) and \(\Int \H\) respectively.
  \end{enumerate}
\end{remark}

\begin{eg}\leavevmode
  \begin{enumerate}
  \item The closed unit ball \(B_n \subseteq \R^n\) is a manifold with bounday, whose boundary is \(\p B_n = S^{n - 1}\).
  \item Any manifold is a manifold with boundary (without boundary).
  \end{enumerate}
\end{eg}

\begin{ex}[Classification of \(1\)-dim connected manifolds with boundary]
  Show that any \(1\)-dim connected manifolds is up to diffeomorphism one of
  \begin{table}[h!]
    \centering
    \begin{tabular}{|c|c|c|}
      \hline
      \(X\) & \(\p X\) & Compactness \\ \hline
      \(S^1\) & \(\emptyset\) & Yes \\ \hline
      \((0, 1)\) & \(\emptyset\) & No \\ \hline
      \([0, 1)\) & \(\{0\}\) & No \\ \hline
      \([0, 1]\) & \(\{0, 1\}\) & Yes \\ \hline
    \end{tabular}
    \caption{\(1\)-dim manifolds}
  \end{table}
\end{ex}

\begin{lemma}
  Let \(X\) be an \(n\)-dim manifold (without boundary) and \(f: X \to \R\) smooth. Suppose \(0\) is a regular value of \(f\). Then \(\{x \in X: f(x) \geq 0\} \subseteq X\) is a submanifold with boundary whose boundary is \(\{x \in X: f(x) = 0\}\).
\end{lemma}

\begin{proof}
  Immediate from pre-image theorem. Suppose \(x \in X\) such that \(f(x) > 0\) then \(x \in f^{-1}((0, \infty)\) which is open in \(X\). If \(f(x) = 0\), just proceed as in pre-image theorem and consider \((F|_U)^{-1}: I \times T(U) \to U\) restricted to \(I \cap \{t \geq 0\}\).
\end{proof}

\begin{eg}
  Let
  \begin{align*}
    f: \R^n &\to \R \\
    (x_1, \dots, x_n) &\mapsto 1 - \sum_{i = 1}^n x_i^2
  \end{align*}
  Then \(B^n = f^{-1}((0, \infty))\) is a manifold with boundary.
\end{eg}

Note that with the definition of tangent space as before, for all \(x \in X\) (even those on the bounday), we have
\[
  \dim dT_xX = \dim X.
\]

Also for \(x \in \p X\),
\[
  T_x\p X \leq T_xX.
\]

Next up is an extension of pre-image theorem

\begin{theorem}
  Let \(X\) be a manifold with boundary and \(f: X \to Y\) be a smooth map betwween manifolds. Suppose \(y \in Y\) is a regular value of \(f\) and \(f|_{\p X}\), then \(f^{-1}(y)\) is a submanifold with boundary of \(X\), with boundary \(\p f^{-1}(y) = f^{-1}(y) \cap \p X\).
\end{theorem}
In this case \(\dim X > \dim Y\).

\begin{proof}
  Note that if \(x \in f^{-1} \cap \Int X\) we can apply pre-image theorem. Thus assume \(x \in f^{-1}(y) \cap \p X\). As an open set in \(\H^n\) with non-empty intersection with its boundary is diffeomorphic to \(\H^n\), assume wlog \(X = \H^n\) and similarly \(Y = \R^m\), where \(\dim X = n, \dim Y = m\).

  Given such an \(x\), there exists \(U\) open in \(\R^n\) containing \(x\) such that there exists \(F: U \to \R^m\) smooth such that \(F|_{U \cap \H^n} = f|_U\). \(y\) is a regular value of \(f\) so \(df_x\) is surjective. For \(U\) small enough, \(df_z\) surjective for all \(z \in U \cap \H^n\). Then \(dF_z\) is surjective for all \(z \in U\). Then \(y\) is a regular value of \(U\).

  We want \(f^{-1}(y) \cap U\) to be a manifold with boundary. Claim that the set equal to \(F^{-1} \cap \H^n\). \(F^{-1}(y)\) is a manifold by pre-image theorem. Introduce projection map
  \begin{align*}
    \pi: \R^n &\to \R \\
    (x_1, \dots, x_n) &\mapsto x_n
  \end{align*}
  Then \(F^{-1}(y) \cap \H^n = F^{-1}(y) \{x: \pi(x) \geq 0\} = \{x: (\pi|_{f^{-1}(y)})^{-1}(x) \geq 0\}\). All we need to do is to show \(0\) is a regular value of the map \((\pi|_{f^{-1}(y)})^{-1}\). But from example sheet
  \[
    d\pi|_{f^{-1}(y)} = \pi|_{T_xF^{-1}(x)} = \pi|_{\ker dF_x} = \pi|_{\ker df_x}.
  \]
  Thus just need to show \(\ker df_x \neq \R^{n - 1} = T_x\p X\). Done by assumption on \(f\).
\end{proof}

If \(f \transversal Z\) and \(f|_{\p X} \transversal Z\)
\blindtext

\subsection{Degree mod \(2\)}

Let \(X\) be a manifold and \(Y\) be a manifold with boundary.

\begin{definition}[Smooth homotopy]\index{smooth homotopy}
  Two smooth maps \(f, g: X \to Y\) are said to be \emph{smoothly homotopic} if there exists a smooth map \(F: [0, 1] \times X \to Y\) such that \(F|_{\{0\} \times X} = f\) and \(F|_{\{1\} \times X} = g\).
\end{definition}

\begin{notation}
  For \(t \in [0, 1]\), \(f_t = F|_{\{t\} \times X}\). It is a restriction of a smooth map so is smooth.
\end{notation}

\begin{remark}
  \(f\) smoothly homeomorphic to \(g\) is an equivalence relation. See example sheet. This is slightly less trivial than its topological counterpart since arbitrary reparameterisation is not necessarily smooth.
\end{remark}

\begin{definition}[Smooth isotopy]\index{smooth isotopy}
  If \(f\) and \(g\) above are smoothly homotopic diffeomorphisms and \(f_t\) is a diffeomrophism for all \(t \in [0, 1]\) then we say that \(F\) is a \emph{smooth isotopy} and \(f\) and \(g\) are \emph{smoothly isotopic}.
\end{definition}

\begin{remark}
  This is also an equivalence relation.
\end{remark}

\begin{lemma}[Homogeneity lemma]\index{homogeneity lemma}
  Let \(X\) be a connected manifold, \(y, z \in X\). Then there exists a map \(\varphi: X \to X\) smoothly isotopic to the identity such that \(\varphi(y) = z\).
\end{lemma}

This says that in differential topology, there is no ``particular'' point in a connected component of a manifold.

\begin{proof}[Sketch of proof]
  Say \(y\) and \(z\) isotopic if there exists such a \(\varphi\). This defines an equivalence relation \(\sim\) on \(X\). It then suffices, by connectivity, to show that the equivalence classes are open. Given \(y \in X\), must find an open neighbourhood \(V\) of \(y\) such that for all \(z \in V\), \(z \sim y\).

  Suffices to show given \(z \in \cl B(0, 1) \subseteq \R^n\), there exists \(F: \R^n \times [0, 1] \to \R^n\) smooth isotopy of diffeomorphisms such that  \(f_0 = \id_{\R^n}\) and \(f_1(0) = z\) with \(f_t|_{\R^n\setminus \cl B(0, 2)}\) (by considering a chart covering ???, we can pushforward the map and extend it by identity outside the ball to ensure smoothness).

  One way to do this is to use cut-off function and ODE theory.
\end{proof}

\begin{lemma}
  Let \(X\) be a compact manifold, \(Y\) a manifold and \(\dim X = \dim Y\). Let \(f, g: X \to Y\) be smoothly homotopic. Let \(y\) be a regular value of both \(f\) and \(g\). Then
  \[
    \# f^{-1}(y) = \# g^{-1}(y) \pmod 2.
  \]
\end{lemma}

\begin{proof}
  Let \(F: X \times [0, 1] \to Y\) be the smooth homotopy between \(f\) and \(g\). Assume first \(y\) is a regular value of \(F\). Note that
  \[
    \p(X \times [0, 1]) = \{0\} \times X \cup \{1\} \times X
  \]
  so denote
  \[
    F|_{\p(X \times [0, 1])} = f \cup g.
  \]
  Then by generalised pre-image theorem, \(F^{-1}(y)\) is a manifold with boundary, whose boundary is
  \[
    \p F^{-1}(y) = F^{-1} \cap \p(X \times [0, 1]) = \left( \bigcup_{x \in f^{-1}(y)}\{0\} \times \{x\}\right) \cup \left( \bigcup_{x \in g^{-1}(y)}\{1\} \times \{x\}\right).
  \]
  By assumption \(F^{-1}(y)\) is a compact \(1\)-dim manifold, which by classification above is a union of disjoint circles or closed intervales. But both of these have even number of boundary points so
  \[
    \# \p F^{-1}(y) = \# f^{-1}(y) + \# g^{-1}(y) = 0 \pmod 2.
  \]

  Suppose instead \(y\) is not a regular value of \(F\). Consider an open neighbourhood \(V \subseteq Y\) of \(y\). By Sard's Theorem the sets of critical values of \(f, g\) and \(F\) are all of measure \(0\), so is the union of them. Then there exist \(\tilde y \in V\) that is a regular value of \(f, g, F\). But by stack of records theorem there exists \(V\) containing \(y\) such that \(f^{-1}(V) = \bigcup_{i = 1}^n U_i\) such that \(f|_{U_i}: U_i \to V\) is a diffeomorphism. 
\end{proof}

\begin{theorem}[Degree mod \(2\)]
  If \(y\) and \(z\) are regular values of \(f\) then
  \[
    \# f^{-1}(y) = f^{-1}(z) \pmod (2).
  \]
\end{theorem}

\begin{proof}
  \blindtext
\end{proof}

Putting everyting together, under the assumptions above, we have \(f^{-1}(y)\) does not depend on the choice of regular value \(y\). Moreover if \(g\) is smoothly homotopic to \(f\) then \(g^{-1}(z) = f^{-1}(y) \pmod 2\) for any regular value \(z\) of \(g\).

\begin{definition}[Degree mod \(2\)]\index{degree mod \(2\)}
  Suppose \(X\) is a compact manifold, \(Y\) a connected manifold and \(\dim X = \dim Y\). Let \(f: X \to Y\) be a smooth map. Then the \emph{degree mod \(2\)} of \(f\), denoted \(\deg_2(f)\), is defiend to be \(\# f^{-1}(y) \pmod 2\) for some regular value \(y\) of \(f\).
\end{definition}

By Sard's theorem, there exists a regular value and by theorems above, this is well-defined. Moreover, this only depends on the homotopy class of \(f\).

\begin{eg}\leavevmode
  \begin{enumerate}
  \item Let \(X, Y = S^n\), \(f: X \to Y\) be a constant map. Then \(\deg_2 f = 0\) since we can find \(y \in Y \setminus \Im f\) and \(\# f^{-1}(y) = 0\).
  \item Let \(X, Y = S^n\), \(\id: X \to Y\) be the identity. Then \(\deg_2 \id = 1\). Thus \(f\) and \(\id\) are not smoothly homotopic.
  \end{enumerate}
\end{eg}

\begin{corollary}
  There does not exist a smooth retraction \(B^n \to \p B^n= S^{n - 1}\), i.e.\ a map \(f: B^n \to \p B^n\) such that \(f|_{\p B^n}: \p B^n \to \p B^n\) is the identity.
\end{corollary}

\begin{proof}
  Suppose not. Define
  \begin{align*}
    F: S^{n - 1} \times [0, 1] &\to S^{n - 1} \\
    (x, t) &\mapsto f(tx)
  \end{align*}
  whose smoothness is left as an exercise. Then \(F\) is an homotopy of \(f\) and \(\id\). Absurd.
\end{proof}

\begin{corollary}[Brouwer's fixed point theorem]\index{Brouwer's fixed point theorem}
  Let \(f: B^n \to B^n\) be smooth. Then there exists \(x_0 \in B^n\) such that \(f(x_0) = x_0\).
\end{corollary}

\begin{proof}
  Suppose \(f\) is smooth without fixed point. Then the ray from \(f(x)\) to \(x\) intersects \(\p B^n\) at precisely one point, say \(g(x)\). Claim that \(g\) is smooth (left as an exercise). If \(x \in \p B^n\) then \(g(x) = x\). Then \(g\) is a smooth retraction. Absurd.
\end{proof}

\begin{corollary}[Brouwer's fixed point theorem]\index{Brouwer's fixed point theorem}
  Same true with continuous assumption.
\end{corollary}

\begin{proof}
  Weierstrass approximation theorem.
\end{proof}

\subsubsection{Intersection number mod 2}

Let \(X\) be a compact manifold, \(Y\) be a manifold, \(Z \subseteq Y\) a closed subset. Suppose \(f: X \to Y\) is smooth and \(f \transversal Z\) and \(\dim X + \dim Z = \dim Y\).

\begin{definition}
  The \emph{intersection number mod 2} is
  \[
    I_2(f, Z) = \# f^{-1}(Z) \pmod 2.
  \]
\end{definition}

Claim that if \(f\) and \(g\) are smoothly homotopic and \(g \transversal Z\) then \(I_2(f, Z) = I_2(g, Z)\).

In the speical case \(X \subseteq Y\), we have \(X \transversal Z\) and denote it \(I_2(X, Z)\).

\begin{eg}
  Let \(Y\) be a torus. Let \(X\) and \(Z\) be the two ``perpendicular circles'' on \(Y\). Then \(I_2(X, Z) = 1\).
\end{eg}

It turns out we can still define intersection number \emph{without} the assumption \(X \transversal Z\).

\begin{eg}
  Suppose \(\dim Y\) is even and \(\dim X\) is \(\frac{\dim Y}{2}\). Note that \(X\) is not transversal to \(X\). Let \(Y\) be the Möbius band and \(X\) a central circle, \(I_2(X, X) = 1\).
\end{eg}

\subsection{Abstract manifolds}

\begin{definition}[Smooth manifold]\index{manifold!abstract smooth manifold}
  A \emph{smooth \(n\)-dimensional manifold} is a Hausdorff, second-countable topological space \(X\) together with a collection of maps \(\varphi: U_\alpha \to V_\alpha\) where \(U_\alpha \subseteq \R^n, V_\alpha \subseteq X\), called \emph{charts}, with the following properties:
  \begin{enumerate}
  \item \(\varphi_\alpha: U_\alpha \to V_\alpha\) are homeomorphisms,
  \item \(\bigcup_\alpha V_\alpha = X\),
  \item \(\varphi_\beta^{-1} \compose \varphi_\alpha|_{\varphi_\alpha^{-1}(U_\alpha \cap V_\beta)}\) are diffeomorphisms for all \(\alpha, \beta\),
  \item lastly, it is conventional to require the collection of charts to be maximal, i.e.\ if \(\{\varphi_\alpha\}_\alpha \cup \{\tilde \varphi\}\) satisfies 1, 2, 3 then \(\tilde \varphi = \varphi_\alpha\) for some \(\alpha\).
  \end{enumerate}
\end{definition}

Note that Hausdorffness and second-countability are hereditary properties.

We can now define smooth maps \(X \to \R^m\) and smooth maps between manifolds, similar as before. Also compatibility condition requires us only to check smoothness on an open cover of \(X\).

What about differentials and tangent spaces? Since our manifold \(X\) is no longer sitting inside an Euclidean space, the differential of a chart is not very meaningful. This is point where the abstract theory diverges from what we have learned. In III Differential Geometry you will learn to do it properly using tangent bundles.

\begin{definition}[Immersion, submersion, embedding]\index{immersion}\index{submersion}\index{embedding}
  A smooth map \(f: X \to Y\) between manifolds is an \emph{immersion} if \(df_x\) is injective for all \(x \in X\), a \emph{submersion} if \(df_x\) is surjective for all \(x \in X\), and an \emph{embedding} if it is an immersion and a homeomorphism onto its image.
\end{definition}

\begin{theorem}[Whitney embedding theorem]\index{Whitney embedding theorem}
  Let \(X\) be a smooth manifold. Then there exists an embedding \(f: X \to \R^{2n + 1}\).
\end{theorem}

Thus the abstract formulation of manifold is not any more general than the one we begin with.

In fact, Whitney proved the better upper bound \(\R^{2n}\), which is in fact a sharp bound.

\section{Geometry}

\begin{definition}[Regular curve]
  Let \(X\) be a manifold. A \emph{regular curve} in \(X\) is a smooth immersion \(\gamma: I \to X\) where \(I \subseteq \R\) is an interval, i.e.\ \(d\gamma_t(1) \neq 0\).
\end{definition}

\begin{notation}
  We write \(\dot \gamma(t) = d\gamma_t\).
\end{notation}

\begin{ex}
  Suppose \(X = \R^3\) and \(\gamma(t) = (x(t), y(t), z(t))\), show that \(\dot \gamma(t) = (\dot x(t), \dot y(t), \dot z(t))\).
\end{ex}

\begin{definition}[Arc-length]\index{arc-length}
  Let \(\gamma: I \to \R^3\) is a regular curve and \(t_0 \in I\). The \emph{arc-length} of \(\gamma\) (measured from \(t_0\)) is
  \[
    s(t) = \int_{t_0}^t |\dot \gamma(t')| dt'
  \]
  where \(|\cdot|\) is the Euclidean norm.
\end{definition}

Arc-length is invariant to reparameterisation: a reparameterisation of \(\gamma\) is a map \(\hat \gamma(\hat t)\) such that \(t = t(\hat t)\) is a diffeomorphism, i.e.\ the following diagram commutes:
\[
  \begin{tikzcd}
    I \ar[r, "\gamma"] & \R^3 \\
    \hat I \ar[u] \ar[ur, "\hat \gamma"']
  \end{tikzcd}
\]
Wlog suppose \(\frac{dt}{d\hat t} > 0\). Let \(t_0 = t(\hat t_0)\), then
  \[
    s(t)
    = \int_{t_0}^t |\hat \gamma(t')| dt'
    = \int_{t_0}^t \left| \frac{d\gamma}{dt'} \right| dt'
    = \int_{\hat t_0}^{\hat t} \left| \frac{d\gamma}{d\hat t'} \right| dt'
    = s(\hat t)
  \]

Note that \(s'(t) = |\hat \gamma(t)| \neq 0\) by Fundmental Theorem of Calculus. Thus by Inverse Function Theorem and connectivity \(s(t)\) is a diffeomorphism, i.e.\ can be inverted to \(t(s)\). Thus we can reparameterise any regular curve by \(s\). We will write \(\gamma(s)\) for \(\gamma(t(s))\).

Missed a lecture

Frenet equations

\[
  \begin{pmatrix}
    \dot t \\
    \dot n \\
    \dot b
  \end{pmatrix}
  =
  \begin{pmatrix}
    0 & k & 0 \\
    -k & 0 & -\tau \\
    0 & \tau & 0 
  \end{pmatrix}
  \begin{pmatrix}
    t \\
    n \\
    b
  \end{pmatrix}
\]

The curvature \(k\) and torsion \(\tau\) is invariant under isometries. 

\begin{theorem}[Fundamental theorem of curves in \(\R^3\)]\index{Fundamental theorem of curves in \(\R^3\)}
  Given smooth function \(k > 0\) and \(\tau(s)\), \(s \in I\), there exists a regular curve \(\alpha: I \to \R^3\) such that \(s\) is the arclength, \(k(s)\) is the curvature and \(\tau(s)\) is the torision of \(\alpha\).

  Moreover, given \(k = \tilde k, \tau = \tilde \tau\), then there exists an isometry of \(\R^3\) such that
  \[
    \tilde \gamma = R \compose \gamma + b.
  \]
\end{theorem}

\begin{proof}
  Let \(b = \tilde \gamma(0) - \gamma(0)\) and we can assume \(\tilde \gamma(0) = \gamma(0) = 0\). Given two orthonormal basis \(\{t(0), s(0), n(0)\}\) and \(\{\tilde t(0), \tilde s(0), \tilde n(0)\}\) in the same orientation, it is an exercise in Euclidean geometry to show that there exists an element \(R \in \SO(3)\) sending the first triplet to the second.

  To avoid excessive notation, replace \(\gamma\) with \(R \compose \gamma\). Then \(\gamma(s)\) and \(\tilde \gamma(s)\) satisfy the Frenet equations, which is a linear ODE, with the same initial condition. Thus by uniqueness of solutions of ODE \(\gamma = \tilde \gamma\).
\end{proof}

Geometric intuition: curvature measures the turning of the tangent, \(k\) the magnitude and \(n\) measures the direction. Torsion measures the non-planiness of a curve.

\begin{proposition}
  Given a regular curve \(\gamma(s)\), \(\tau = 0\) if and only if \(\gamma\) lies in a plane.
\end{proposition}

\begin{proof}
  Suppose \(\tau = 0\). Then \(\dot b = 0\) so \((t \wedge n)' = 0\). Let \(\Pi\) be the plane spanned by \(t\) and \(n\) and so \(\Pi(s) = \Pi_0\) is constant.

  Consider the plane \(\gamma(0) + \Pi\), claim \(\gamma(I) \subseteq \gamma(0) + \Pi_0\): this is equivalent to claiming that \(\gamma(s) - \gamma(0) \in \Pi\) for all \(s\). But
  \[
    \gamma(s) - \gamma(0) = \int_0^s \gamma'(u)du = \int_0^s t(u)du \in \Pi.
  \]

  The other direction is left as an exercise.
\end{proof}

\begin{definition}[Plane curve]\index{plane curve}
  A regular curve is said to be a \emph{plane curve} if its image lies in a plane \(\Pi \subseteq \R^3\).
\end{definition}

Wlog we will consider all plane curves as maps \(\gamma: I \to \R^2\).

For regular plane curves, there is an alternative definition of the curvature:

\begin{definition}[Signed curvature]\index{signed curvature}
  Let \(\gamma(s)\) be a plane curve in \(\R^2\) parameterised by arc-length \(s\). \(t = \dot \gamma(s)\) is the tangent. Pick \(n(s)\) so that \(t, n\) has the usual orientation (exercise: exhibit such \(n\)). Then \(\dot t(s) \cdot t(s) = 0\) so define \emph{signed curvature} \(k(s)\) to be such that
  \[
    \dot t(s) = k(s)n(s)
  \]
  where \(k(s) \in \R\), i.e.\ not necessarily non-vanishing.
\end{definition}

Note the subtleties in the order of terms being defined: in general regular curves we defined \(k(s)\) and \(n(s)\) in terms of \(\dot t(s)\) but for plane curves, we fixed \(n(s)\) first (using orientation of orthonormal basis) and then define \(k(s)\) to be the signed magitude of \(\dot t(s)\).

\begin{ex}
  Define the Frenet formulae for plane curves:
  \begin{align*}
    \dot t(s) &= k(s)n(s) \\
    \dot n(s) &= -k(s)t(s)
  \end{align*}
\end{ex}

\subsection{Isoperimetric inequality in the plane}

The is the first global result we meet in differential geometry. Let \(\gamma\) be a smooth simple closed curve in \(\R^2\), i.e.\ \(\gamma: S^1 \to \R^2\) is a smooth embedding. Thanks to

\begin{theorem}[Jordan curve theorem]
  \(\gamma(S^1)\) is the boundary of a bounded, simply-connected open set in \(\R^2\).
\end{theorem}
we let \(\Omega\) be the bouned open set (alternatively assume \(\cl \Omega\) is a \(2\)-manifold with boundary in \(\R^2\) with boundary diffeomorphic to \(S^1\)).

\begin{note}
  We want to define \(\alpha(\R) = \alpha(S^1)\).
  
  Let \(\alpha: \R \to \R^2\) be a regular curve parameterised by arc-length. Let \(L\) be the first \(s > 0\) such that \(\alpha(L) = \alpha(0)\). Then \(\alpha\) be periodic with period \(L\). \(L\) is the arc-length of \(\gamma\).
\end{note}

\begin{theorem}[Isoperimetric inequality]\index{isoperimetric inequality}
  \[
    L^2 \geq 4\pi \operatorname{area}(\Omega)
  \]
  with equality if and only if \(\gamma\) is a circle.
\end{theorem}

\begin{remark}
  It suffices to assume \(\alpha\) is \(C^1\).
\end{remark}

Write \(\alpha(s) = (x(s), y(s))\) where \(x(s)\) and \(y(s)\) are \(C^1\) functions. Let \(\V V: \R^2 \to \R^2\) be the identity and think of it as a vector field. By divergence theorem applied to \(\V V\) in \(\Omega\),
\begin{align*}
  2 \operatorname{area}(\Omega)
  &= \int_\Omega \divergence \V V \\
  &= \int_0^L \V V \cdot (-n) ds \\
  &\leq \int_0^L \norm{\V V(s)} ds \\
  &\leq \left( \int_0^L \norm{\V V(s)}^2 ds \int_0^L 1^2 ds \right)^{1/2}
\end{align*}

Missed a lecture

\subsection{Riemannian metric}

something something Riemannian metric. See printed notes.

\begin{definition}[Isometry]\index{isometry}
  Let \(X\) and \(Y\) be (Riemannian) manifolds with Riemannian metrics \(g_p\) and \(\tilde g_q\) respectively. We say \(X\) and \(Y\) are \emph{isometric} if there exists \(f: X \to Y\) smooth such that
  \[
    df_x: T_xX \to T_{f(x)}Y
  \]
  is an isometry of inner product spaces, i.e.\ for all \(v, w \in T_xX\),
  \[
    g_x(v, w) = \tilde g_{f(x)}(df_x(v), df_x(w)).
  \]
  Furthermore \(df_x\) is invertible.
\end{definition}

Note that conventions vary. If \(df_x\) is not required to be invertible then \(f\) is usually called an \emph{isometric embedding}.

\begin{definition}[Local isometry]\index{isometry}
  Two Riemannian manifolds \(X\) and \(Y\) are \emph{locally isometric} around \(p \in X\) if there exist \(V \ni p\) open in \(X\) and \(W\) open in \(Y\) such that \(f: V \to W\) is an isometry.
\end{definition}

\begin{question}\leavevmode
  \begin{enumerate}
  \item When are two surfaces \(X\) and \(Y\) (say embedded in \(\R^3\)) locally isometric? Globally isometric?
  \item When is \(X\) locally isometric to the plane? (intrinsic geometry)
  \item When do \(X\) and \(Y\) differ by a Euclidean isometry, i.e.\ \(Y = R(X) + b\) where \(R \in \SO(3)\)? (extrinsic geometry)
  \end{enumerate}
\end{question}

Let \(X\) be a Riemannian \(2\)-manifold. Let \(\varphi: U \to V\) be a local parameterisation. Define
\begin{align*}
  E(u, v) &= g(\varphi_u, \varphi_u) > 0 \\
  F(u, v) &= g(\varphi_u, \varphi_v) \\
  G(u, v) &= g(\varphi_v, \varphi_v) > 0
\end{align*}
with \(EG - F^2 > 0\) and \(E, F, G\) smooth. Claim that \(\begin{psmallmatrix} E & F \\ F & G \end{psmallmatrix}\) defines a Riemannian metric.

A converse question is, given such \(E, F, G\), does there exist a local parameterisation \(\varphi: U \to V\) such that \(E, F, G\) are coefficients of \(I_p\)? The answer is positive for analytic functions, but in more general settings this is still an open problem. Thus it is possible that local Riemannian geometry is richer than Euclidean geometry.


First fundamental form

\begin{eg}[Embedded torus]
  
\end{eg}

\begin{ex}
  Do for general surfaces of revolution.
\end{ex}

\begin{definition}
  Let \(X\) be a surface in \(\R^3\). Let \(\varphi: U \to V\) be a local parameterisation. Let \(\Omega \subseteq V\) be an open set with smooth boundary. Then the area of \(\Omega\) is
  \[
    \int_{\varphi^{-1}(\Omega)} |\varphi_u \times \varphi_v| dudv.
  \]
\end{definition}

Claim that the definition is independent of parameterisation. Use chain rule for change of variable.

\begin{ex}
  \[
    \operatorname{area} \Omega = \int_{\varphi^{-1}(\Omega)} \sqrt{EG - F^2} dudv
  \]
  so the area depends only on the first fundamental form. Thus an isometry preserves area. In the other words, area is an intrinsic geometric property.
\end{ex}

\begin{remark}
  Using partition of unity, we can drop the restriction that \(\Omega\) lies in the range of a local parameterisation. In fact there exists a Borel measure on \(X\), denoted \(dA\), such that
  \[
    \operatorname{area} (\Omega) = \int_\Omega dA.
  \]
  Subsequently, we can define the integral of any smooth (actually measurable) map with respect to the measure.

  Note that \(E \subseteq X\) has measure zero as defined before if and only if \(E\) has measure \(0\) with respect to \(dA\).
\end{remark}

\begin{eg}
  The area of the embedded torus \(X\) define before is
  \[
    \operatorname{area}(X)
    = \int_0^{2\pi}\int_0^{2\pi} r(a + r\cos u) dudv
    = 4\pi^2 a r.
  \]
\end{eg}

\subsection{Gauss map}

Let \(X\) be a surface in \(\R^3\). Let \(\varphi: U \to V\) be a local parameterisation. Given \(p \in X\), \(\varphi_u, \varphi_v\) form a basis for \(T_pX\). Define a map
\begin{align*}
  f: V &\to S^2 \\
  p &\mapsto \frac{(\varphi_u \times \varphi_v)(\varphi^{-1}(p))}{|(\varphi_u \times \varphi_v)(\varphi^{-1}(p))|}
\end{align*}
which is a smooth map such that \(f(p) \perp T_pX\).

Note that the map with such property is not unique, for example taking it negative value.

\begin{definition}[Gauss map]\index{Gauss map}
  Let \(X\) be a surface in \(\R^3\). A \emph{Gauss map} of \(X\) is a smooth map \(N: X \to S^2\) such that \(N(p) \perp T_pX\) for all \(p \in X\).
\end{definition}

\begin{note}
  Let \(X\) be connected. If there exists a Gauss map then there exists precisely 2 Gauss maps \(N, -N\). Moreover
  \[
    N = \pm \frac{\varphi_u \times \varphi_v}{|\varphi_u \times \varphi_v|}
  \]
  in a local chart.
\end{note}

\begin{definition}[Orientability]\index{orientability}
  If there exists a Gauss map, we say \(X\) is \emph{orientable}. \((X, N)\) is called an \emph{oriented surface}.
\end{definition}

\begin{remark}\leavevmode
  \begin{enumerate}
  \item There exists non-orientable surface, such as the Möbius band.
  \item The above definition is an extrinsic one. It is possible to characterise orientability intrinsically using transition maps.
  \end{enumerate}
\end{remark}

\begin{eg}
  The sphere is orientable, with \(N = \pm \id\).
\end{eg}

Let \(X\) be an orientable surface and \(N: X \to S^2\) a Gauss map. Consider the differential \(dN_p: T_pX \to T_{N(p)}S^2\). Since \(N(p) \perp T_pX\) by definition, and \(N(p) \perp T_{N(p)} S^2\) by the observation in the example above, we can identify \(T_pX = T_{N(p)}S^2\). Thus we can treat \(dN_p: T_pX \to T_pX\) as an endormorphism of an inner product space. Next time we will show that it is self-adjoint with respect to the first fundamental form.


\blindtext

\subsection{Theorema Egregium}

Since \(\{\varphi_u, \varphi_v, N\}\) form an orthonormal basis, we can express \(\varphi_{uu}, \varphi_{uv}\) etc as unique linear combinations of them. But what are the indices? Recall that
\[
  e = - \inner{\varphi_u, N_u} = \inner{\varphi_{uu}, N}
\]
and \(N\) is orthogonal to \(\varphi_U, \varphi_v\) so we have
\[
  \varphi_{uu}(u, v) = \Gamma_{11}^1(u, v) \varphi_u + \Gamma_{11}^2 \varphi_v + eN
\]
and etc.

\begin{definition}
  \(\Gamma_{ij}^k\) are called the \emph{Christoffel symbols}.
\end{definition}

\begin{remark}\leavevmode
  \begin{enumerate}
  \item The Christoffel symbols are symmetric with repsect to lower indices, i.e.\ \(\Gamma_{ij}^k = \Gamma_{ji}^k\).
  \item \(\Gamma_{ij}^k = A(E, F, G, \p E, \dots, \p G)\) where \(A\) is a universal function. Take the inner product of the above relations with \(\varphi_u\) and recoginising \(\inner{\varphi_u, \varphi_u} = E\), we get
    \begin{align*}
      \Gamma_{11}^1 E + \Gamma_{1}^2 F &= \inner{\varphi_{uu}, \varphi_u} = \frac{1}{2} \p_u \inner{\varphi_u, \varphi_u} = \frac{1}{2} \p_u E \\
      \Gamma_{11}^1 F + \Gamma_{1}^2 G &= \inner{\varphi_{uu}, \varphi_v} = \p_u \inner{\varphi_u, \varphi_v} - \inner{\varphi_u, \varphi_{vu}} = \p_u F - \frac{1}{2} \p_v E
    \end{align*}
    Write it in matrix form,
    \[
      \begin{pmatrix}
        E & F \\
        F & G
      \end{pmatrix}
      \begin{pmatrix}
        \Gamma_{11}^1 \\
        \Gamma_{11}^2
      \end{pmatrix}
      =
      \begin{pmatrix}
        \frac{1}{2} \p_u E \\
        \p_u F - \frac{1}{2} \p_v E
      \end{pmatrix}
    \]
    We can repeat for the other \(\Gamma_{ij}^k\)'s.
  \end{enumerate}
\end{remark}

Let's go on and continue differentiating. Differentiate * with respect to \(v\) and ** with respect to \(u\), and by symmetry of partial derivatives,
\begin{align*}
  &\p_v \Gamma_{11}^1 \varphi_u + \p_v \Gamma_{11}^2 \varphi_v + (\p_v e) N + \Gamma_{11}^1 \varphi_{uv} + \Gamma_{11}^2 \varphi_{vv} + e N_v \\
  =& \p_u \Gamma_{12}^1 \varphi_u + \p_u \Gamma_{12}^2 \varphi_v + (\p_u f) N + \Gamma_{12}^1 \varphi_{uu} + \Gamma_{12}^2 \varphi_{vu} + f N_u
\end{align*}
Expand this in our favourite basis \(\{\varphi_u, \varphi_v, N\}\), we get three equations, each for a basis element. The first three components on each side are already in this form. Look up in * and **, we get
\begin{align*}
  \p_v \Gamma_{11}^2
\end{align*}

\blindtext

And after some algebra we find that \(K\) can be expressed purely in terms of the first fundamental form and its derivative.

\section{Critical points of length and curve}

Suppose \(X \subseteq \R^3\) is a surface and \(p, q \in X\). Let \(\Omega(p, q)\) be the set of all regular paths from \(p\) to \(q\). Then the length could be seen as a functional on \(\Omega(p, q)\). Since length depends only on the first fundamental form, i.e.\ an intrinsic geometric quantity, it turns out that the curves minimising the length functional are also special and intrinsic to the surface. They are called \emph{geodesic}.

Similarly, area is also a functional on the space of all surfaces. The area-miniming surfaces are called \emph{minimal surfaces} and eventually we can derive an equation depending on the mean curvature.



\printindex

\iffalse
Classical differential geometry concerning geometries of curves and surfaces, from a modern point of view

Contents:
I: notions of smmoth \(k\)-dim manifolds. We study differential topology.

Geometry is concerned with the study of rigid motions

invariants of curves: \(k\) curvature, \(\tau\) torsion
invariant of surfaces: \(K\) mean curvature, \(K\) Gaussian curvature
\fi

\end{document}
