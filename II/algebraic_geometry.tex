\documentclass[a4paper]{article}

\def\npart{II}

\def\ntitle{Algebraic Geometry}
\def\nlecturer{I.\ Grojnowski}

\def\nterm{Lent}
\def\nyear{2019}

\ifx \nauthor\undefined
  \def\nauthor{Qiangru Kuang}
\else
\fi

\ifx \ntitle\undefined
  \def\ntitle{Template}
\else
\fi

\ifx \nauthoremail\undefined
  \def\nauthoremail{qk206@cam.ac.uk}
\else
\fi

\ifx \ndate\undefined
  \def\ndate{\today}
\else
\fi

\title{\ntitle}
\author{\nauthor}
\date{\ndate}

%\usepackage{microtype}
\usepackage{mathtools}
\usepackage{amsthm}
\usepackage{stmaryrd}%symbols used so far: \mapsfrom
\usepackage{empheq}
\usepackage{amssymb}
\let\mathbbalt\mathbb
\let\pitchforkold\pitchfork
\usepackage{unicode-math}
\let\mathbb\mathbbalt%reset to original \mathbb
\let\pitchfork\pitchforkold

\usepackage{imakeidx}
\makeindex[intoc]

%to address the problem that Latin modern doesn't have unicode support for setminus
%https://tex.stackexchange.com/a/55205/26707
\AtBeginDocument{\renewcommand*{\setminus}{\mathbin{\backslash}}}
\AtBeginDocument{\renewcommand*{\models}{\vDash}}%for \vDash is same size as \vdash but orginal \models is larger
\AtBeginDocument{\let\Re\relax}
\AtBeginDocument{\let\Im\relax}
\AtBeginDocument{\DeclareMathOperator{\Re}{Re}}
\AtBeginDocument{\DeclareMathOperator{\Im}{Im}}
\AtBeginDocument{\let\div\relax}
\AtBeginDocument{\DeclareMathOperator{\div}{div}}

\usepackage{tikz}
\usetikzlibrary{automata,positioning}
\usepackage{pgfplots}
%some preset styles
\pgfplotsset{compat=1.15}
\pgfplotsset{centre/.append style={axis x line=middle, axis y line=middle, xlabel={$x$}, ylabel={$y$}, axis equal}}
\usepackage{tikz-cd}
\usepackage{graphicx}
\usepackage{newunicodechar}

\usepackage{fancyhdr}

\fancypagestyle{mypagestyle}{
    \fancyhf{}
    \lhead{\emph{\nouppercase{\leftmark}}}
    \rhead{}
    \cfoot{\thepage}
}
\pagestyle{mypagestyle}

\usepackage{titlesec}
\newcommand{\sectionbreak}{\clearpage} % clear page after each section
\usepackage[perpage]{footmisc}
\usepackage{blindtext}

%\reallywidehat
%https://tex.stackexchange.com/a/101136/26707
\usepackage{scalerel,stackengine}
\stackMath
\newcommand\reallywidehat[1]{%
\savestack{\tmpbox}{\stretchto{%
  \scaleto{%
    \scalerel*[\widthof{\ensuremath{#1}}]{\kern-.6pt\bigwedge\kern-.6pt}%
    {\rule[-\textheight/2]{1ex}{\textheight}}%WIDTH-LIMITED BIG WEDGE
  }{\textheight}% 
}{0.5ex}}%
\stackon[1pt]{#1}{\tmpbox}%
}

%\usepackage{braket}
\usepackage{thmtools}%restate theorem
\usepackage{hyperref}

% https://en.wikibooks.org/wiki/LaTeX/Hyperlinks
\hypersetup{
    %bookmarks=true,
    unicode=true,
    pdftitle={\ntitle},
    pdfauthor={\nauthor},
    pdfsubject={Mathematics},
    pdfcreator={\nauthor},
    pdfproducer={\nauthor},
    pdfkeywords={math maths \ntitle},
    colorlinks=true,
    linkcolor={red!50!black},
    citecolor={blue!50!black},
    urlcolor={blue!80!black}
}

\usepackage{cleveref}



% TODO: mdframed often gives bad breaks that cause empty lines. Would like to switch to tcolorbox.
% The current workaround is to set innerbottommargin=0pt.

%\usepackage[theorems]{tcolorbox}





\usepackage[framemethod=tikz]{mdframed}
\mdfdefinestyle{leftbar}{
  %nobreak=true, %dirty hack
  linewidth=1.5pt,
  linecolor=gray,
  hidealllines=true,
  leftline=true,
  leftmargin=0pt,
  innerleftmargin=5pt,
  innerrightmargin=10pt,
  innertopmargin=-5pt,
  % innerbottommargin=5pt, % original
  innerbottommargin=0pt, % temporary hack 
}
%\newmdtheoremenv[style=leftbar]{theorem}{Theorem}[section]
%\newmdtheoremenv[style=leftbar]{proposition}[theorem]{proposition}
%\newmdtheoremenv[style=leftbar]{lemma}[theorem]{Lemma}
%\newmdtheoremenv[style=leftbar]{corollary}[theorem]{corollary}

\newtheorem{theorem}{Theorem}[section]
\newtheorem{proposition}[theorem]{Proposition}
\newtheorem{lemma}[theorem]{Lemma}
\newtheorem{corollary}[theorem]{Corollary}
\newtheorem{axiom}[theorem]{Axiom}
\newtheorem*{axiom*}{Axiom}

\surroundwithmdframed[style=leftbar]{theorem}
\surroundwithmdframed[style=leftbar]{proposition}
\surroundwithmdframed[style=leftbar]{lemma}
\surroundwithmdframed[style=leftbar]{corollary}
\surroundwithmdframed[style=leftbar]{axiom}
\surroundwithmdframed[style=leftbar]{axiom*}

\theoremstyle{definition}

\newtheorem*{definition}{Definition}
\surroundwithmdframed[style=leftbar]{definition}

\newtheorem*{slogan}{Slogan}
\newtheorem*{eg}{Example}
\newtheorem*{ex}{Exercise}
\newtheorem*{remark}{Remark}
\newtheorem*{notation}{Notation}
\newtheorem*{convention}{Convention}
\newtheorem*{assumption}{Assumption}
\newtheorem*{question}{Question}
\newtheorem*{answer}{Answer}
\newtheorem*{note}{Note}
\newtheorem*{application}{Application}

%operator macros

%basic
\DeclareMathOperator{\lcm}{lcm}

%matrix
\DeclareMathOperator{\tr}{tr}
\DeclareMathOperator{\Tr}{Tr}
\DeclareMathOperator{\adj}{adj}

%algebra
\DeclareMathOperator{\Hom}{Hom}
\DeclareMathOperator{\End}{End}
\DeclareMathOperator{\id}{id}
\DeclareMathOperator{\im}{im}
\DeclareMathOperator{\coker}{coker}
\DeclarePairedDelimiter{\generation}{\langle}{\rangle}

%groups
\DeclareMathOperator{\sym}{Sym}
\DeclareMathOperator{\sgn}{sgn}
\DeclareMathOperator{\inn}{Inn}
\DeclareMathOperator{\aut}{Aut}
\DeclareMathOperator{\GL}{GL}
\DeclareMathOperator{\SL}{SL}
\DeclareMathOperator{\PGL}{PGL}
\DeclareMathOperator{\PSL}{PSL}
\DeclareMathOperator{\SU}{SU}
\DeclareMathOperator{\UU}{U}
\DeclareMathOperator{\SO}{SO}
\DeclareMathOperator{\OO}{O}
\DeclareMathOperator{\PSU}{PSU}
\DeclareMathOperator{\Sp}{Sp}


%hyperbolic
\DeclareMathOperator{\sech}{sech}

%field, galois heory
\DeclareMathOperator{\ch}{ch}
\DeclareMathOperator{\gal}{Gal}
\DeclareMathOperator{\emb}{Emb}



%ceiling and floor
%https://tex.stackexchange.com/a/118217/26707
\DeclarePairedDelimiter\ceil{\lceil}{\rceil}
\DeclarePairedDelimiter\floor{\lfloor}{\rfloor}


\DeclarePairedDelimiter{\innerproduct}{\langle}{\rangle}

%\DeclarePairedDelimiterX{\norm}[1]{\lVert}{\rVert}{#1}
\DeclarePairedDelimiter{\norm}{\lVert}{\rVert}



%Dirac notation
%TODO: rewrite for variable number of arguments
\DeclarePairedDelimiterX{\braket}[2]{\langle}{\rangle}{#1 \delimsize\vert #2}
\DeclarePairedDelimiterX{\braketthree}[3]{\langle}{\rangle}{#1 \delimsize\vert #2 \delimsize\vert #3}

\DeclarePairedDelimiter{\bra}{\langle}{\rvert}
\DeclarePairedDelimiter{\ket}{\lvert}{\rangle}




%macros

%general

%divide, not divide
\newcommand*{\divides}{\mid}
\newcommand*{\ndivides}{\nmid}
%vector, i.e. mathbf
%https://tex.stackexchange.com/a/45746/26707
\newcommand*{\V}[1]{{\ensuremath{\symbf{#1}}}}
%closure
\newcommand*{\cl}[1]{\overline{#1}}
%conjugate
\newcommand*{\conj}[1]{\overline{#1}}
%set complement
\newcommand*{\stcomp}[1]{\overline{#1}}
\newcommand*{\compose}{\circ}
\newcommand*{\nto}{\nrightarrow}
\newcommand*{\p}{\partial}
%embed
\newcommand*{\embed}{\hookrightarrow}
%surjection
\newcommand*{\surj}{\twoheadrightarrow}
%power set
\newcommand*{\powerset}{\mathcal{P}}

%matrix
\newcommand*{\matrixring}{\mathcal{M}}

%groups
\newcommand*{\normal}{\trianglelefteq}
%rings
\newcommand*{\ideal}{\trianglelefteq}

%fields
\renewcommand*{\C}{{\mathbb{C}}}
\newcommand*{\R}{{\mathbb{R}}}
\newcommand*{\Q}{{\mathbb{Q}}}
\newcommand*{\Z}{{\mathbb{Z}}}
\newcommand*{\N}{{\mathbb{N}}}
\newcommand*{\F}{{\mathbb{F}}}
%not really but I think this belongs here
\newcommand*{\A}{{\mathbb{A}}}

%asymptotic
\newcommand*{\bigO}{O}
\newcommand*{\smallo}{o}

%probability
\newcommand*{\prob}{\mathbb{P}}
\newcommand*{\E}{\mathbb{E}}

%vector calculus
\newcommand*{\gradient}{\V \nabla}
\newcommand*{\divergence}{\gradient \cdot}
\newcommand*{\curl}{\gradient \cdot}

%logic
\newcommand*{\yields}{\vdash}
\newcommand*{\nyields}{\nvdash}

%differential geometry
\renewcommand*{\H}{\mathbb{H}}
\newcommand*{\transversal}{\pitchfork}
\renewcommand{\d}{\mathrm{d}} % exterior derivative

%number theory
\newcommand*{\legendre}[2]{\genfrac{(}{)}{}{}{#1}{#2}}%Legendre symbol

%algebraic geometry
\DeclareMathOperator{\Spec}{Spec}
\DeclareMathOperator{\Proj}{Proj}

\renewcommand{\A}{\mathbb{A}}
\DeclareMathOperator{\ev}{ev}


\DeclareMathOperator{\Cl}{Cl}
\renewcommand*{\P}{\mathbb{P}}
\DeclareMathOperator{\Der}{Der} % derivation

\let\div\relax
%\AtBeginDocument{DeclareMathOperator{\div}{div}}

\begin{document}

\begin{titlepage}
  \begin{center}
    \includegraphics[width=0.6\textwidth]{logo.jpg}\par
    \vspace{1cm}
    {\scshape\huge Mathamatics Tripos \par}
    \vspace{2cm}
    {\huge Part \npart \par}
    \vspace{0.6cm}
    {\Huge \bfseries \ntitle \par}
    \vspace{1.2cm}
    {\Large\nterm, \nyear \par}
    \vspace{2cm}
    
    {\large \emph{Lectures by } \par}
    \vspace{0.2cm}
    {\Large \scshape \nlecturer}
    
    \vspace{0.5cm}
    {\large \emph{Notes by }\par}
    \vspace{0.2cm}
    {\Large \scshape \href{mailto:\nauthoremail}{\nauthor}}
 \end{center}
\end{titlepage}

\tableofcontents

\setcounter{section}{-1}

\section{Introduction}

Algebraic geometry is the study of polynomial equations.

\begin{eg}
  \(E = \{(x, y) \in \C^2: y^2 = x^3 - x\}\). Sketch this.

  Consider \(p: E \to \C, (x, y) \mapsto x\). For each \(x \notin \{0, \pm 1\}\), there are 2 points in \(p^{-1}(x)\). So this is a double cover ramified at \(0, \pm 1\), the precise meaning of these phrases will be defined later. How does this help us sketch? For \(x\) away from the three points, the preimage of a disk under \(p\) are two copies of the disk. If \(x\) is near \(0\), we have \(x^3 - x \approx -x\) so locally it looks like \(y^2 = -x\). If we project \((x, y)\) to \(x\) we get a disk winding around twice. But if we project to \(y\) we get a bijection.

  Still, how do we visualise \(E\)? First let's sketch it over \(\R\). If \((x, y) \in \R^2\) then \(y^2 \geq 0\) so \(x(x^2 - 1) \cdot 0\). Thus \(x \geq 1\) or \(-1 \leq x \leq 0\). Just like in high school, we can differentiate. (graph) The infinite bit should be visualised as a circle minus a point.

  Now let \((x, y) \in \C^2\). Let
  \[
    \Gamma = \{(x, y) \in E: y \in \R, x \in [-1, 0] \cup [1, \infty)\} = p^{-1}\{[-1, 1] \cup [1, \infty)\}.
  \]
  Claim \(E \setminus \Gamma\) is disconnected and it consists of two pieces, each isomorphic via \(p\) to \(\C \subseteq ([-1, 0] \cup [1, \infty))\). This is equivalent to the claim that if \(x \in \C \subseteq ([-1, 0] \cup [1, \infty))\) then can choose a square root of \(x^3 - x\), and then as you wander around, this remains a single-valued functions. The proof is left as an exercise.

  Granting this, we have two copies of \(\C \subseteq ([-1, 0] \cup [1, \infty))\). Turn one of them around and glue (graph).

  More surprisingly, solutions of equaitons have a topology!
\end{eg}

\section{The dictionary between algebra and geometry}

\subsection{Basic notions}

\begin{definition}[\(k\)-algebra]\index{\(k\)-algebra}
  Let \(k\) be a field. A \emph{(commutative) \(k\)-algebra} is a unital commutative ring countaining \(k\) as a subring.
\end{definition}

\begin{eg}
  \(k[x_1, \dots, x_n]\), the polynomial ring in \(n\)-variables.
\end{eg}

\begin{notation}
  If \(k\) is a fields, write \(\A^n = \A^n(k) = k^n\), the \emph{affine \(n\)-space}\index{affine space}.
\end{notation}

Observe that every \(f \in k[x_1, \dots, x_n]\) defines a function
\begin{align*}
  \A^n(k) &\to \A^1(k) \\
  (p_1, \dots, p_n) &\mapsto f(p_1, \dots, p_n) = \ev_p(f)
\end{align*}

This defines a map from \(k[x_1, \dots, x_n]\) to the space of all functions \(\A^n \to \A^1\). If \(k\) is finite then it is surjective but not injective, and if \(k\) is infinite then it is not surjective.

More generally, if \(L \supseteq k\) is an algebraic extension then one can define a function \(\A^n(L) \to \A^1(L)\) by evaluating \(f\) at a point in \(L^n\). Therefore \(f\) defines a function \(\A^n(\cl k) \to \A^1(\cl k)\) where \(\cl k\) is the algebraic closure of \(k\). So now the map \(k[x_1, \dots, x_n] \to \{\A^n(\cl k) \to \A^1(\cl k)\}\) is injective for all \(k\) but never surjective.

The conclusion is that we should think of \(k[x_1, \dots, x_n]\) as very special functions \(\cl k^n \to \cl k\), namely ``polynomial with \(k\)-coefficients''. As a concrete example, let \(k = \F_q\). Then \(x^q - x\) defines a function \(\cl k \to \cl k\) that is \emph{not} zero.

\begin{definition}[algebraic set]
  Let \(S \subseteq k[x_1, \dots, x_n]\). Define
  \[
    Z(S) = \{p \in \A^p: f(p) = 0 \text{ for all } f \in S\} \subseteq \A^n
  \]
  which are the simultaneous zeros of equations in \(S\). Such a subset is known as \emph{algebraic set}, \emph{Zariski closed subset of \(\A^n\)}.
\end{definition}

\begin{eg}\leavevmode
  \begin{enumerate}
  \item \(\A^n = Z(0)\).
  \item \(Z(x) = \{0\}\). Similarly \(Z(x - 7) = \{7\}\).
  \item If \(f(x) = (x - \lambda_1) \dots (x - \lambda_n)\) then \(Z(f) = \{\lambda_1, \dots, \lambda_n\}\).
  \item if \(k = \cl k\) then algebraic subsets of \(\A^1\) are \(\emptyset, \A^1\) or finite set of points of \(k\).
  \item In \(\A^2\), \(Z(y^2 - x^3 + x) = E\) which we sketched in introduction.
  \item In \(\A^2\), \(Z(x, y) = \{(0, 0)\}\), \(Z(xy)\) is the union of two axes. \(Z(y)\) is the \(x\)-axis and \(Z(y(y - 1), x(y - 1))\) is the union of a point and a line
  \end{enumerate}
\end{eg}

If \(J\) is the ideal generated by \(S\), i.e.
\[
  J = \left\{\sum a_i f_i: a_i \in k[x_1, \dots, x_n], f_i \in S\right\}
\]
then \(Z(J) = Z(S)\).

Recall from IB Groups, Rings and Modules

\begin{theorem}[Hilbert basis theorem]
  If \(k\) is Noetherian then so is \(k[x]\).
\end{theorem}

So every ideal in \(k[x_1, \dots, x_n]\) is finitely generated. Therfore there exist \(f_1, \dots, f_r \in k[x_1, \dots, x_n]\) such that
\[
  Z(S) = Z(f_1, \dots, f_r).
\]
Thus algebraic sets are solutions of finitely many polynomial equations.

\begin{lemma}\leavevmode
  \begin{enumerate}
  \item If \(I \subseteq J\) then \(Z(J) \subseteq Z(I)\).
  \item \(Z(0) = \A^n\) and \(Z(k[x_1, \dots, x_n]) = \emptyset\).
  \item \(Z(\bigcup J_i) = Z(\sum J_i) = \bigcap Z(J_i)\) for any (possibly infinite) family of ideals \(\{J_i\}\).
  \item \(Z(I \cap J) = Z(I) \cup Z(J)\) for ideals \(I, J\).
  \end{enumerate}
\end{lemma}

\begin{proof}
  1, 2, 3 are clear. For 4, \(\subseteq\) follows from 1. For \(\supseteq\), if \(x \notin Z(I)\) then exists \(f_1 \in I\) with \(f_1(x) \neq 0\) and if \(x \notin Z(J)\) then exists \(f_2 \in J\) with \(f_2(x) \neq 0\). Thus \(f_1f_2(x) = f_1(x)f_2(x) \neq 0\) so \(x \notin Z(f_1f_2)\). But \(f_1f_2 \in I \cap J\) as \(I\) and \(J\) are ideals. Thus \(x \notin Z(I \cap J)\).
\end{proof}

We can define a map goes the other direction. If \(Z \subseteq \A^n(\cl k)\) is a subset, define
\[
  I(Z) = \{f \in k[x_1, \dots, x_n]: f(p) = 0 \text{ for all } p \in Z\}.
\]

If \(f \in I(Z), g \in k[x_1, \dots, x_n]\), \(fg(p) = f(p)g(p) = 0\) if \(p \in Z\) so \(I(Z)\) is an ideal.

\begin{lemma}\leavevmode
  \begin{enumerate}
  \item If \(Z \subseteq Z'\) then \(I(Z') \subseteq I(Z)\).
  \item For any \(Y \subseteq \A^n\), \(Y \subseteq Z(I(Y))\).
  \item If \(V = Z(J)\) is an algebraic subset then \(V = Z(I(V))\).
  \item If \(J \subseteq k[x_1, \dots, x_n]\) is an ideal then \(J \subseteq I(Z(J))\).
  \end{enumerate}
\end{lemma}

\begin{proof}
  1, 2 and 4 are immediate. For 3, \(\supseteq\) follows from \(I(V) = I(Z(J) \supseteq J\) by 4 so \(Z(I(V)) \subseteq Z(J) = V\) by 1. \(\subseteq\) follows from 2.
\end{proof}

The first lemma says that algebraic subsets of \(\A^n\) form the closed sets of a topology on \(\A^n\). This is called the \emph{Zariski topology}\index{Zariski topology}.

\begin{eg}
  If \(X = \A^1(k)\) where \(k = \cl k\), the closed subsets are finite subsets of points of \(\A^1\).

  Note that if \(k = \C\), if \(Z \subseteq \A^n(k)\) is Zariski closed then it is closed in the usual sense.
\end{eg}

The second lemma says that \(Z(I(Y))\) is the smallest algebraic subset of \(\A^n\) containing \(Y\), i.e.\ the closure of \(Y\) in the Zariski topology.

\begin{eg}
  If \(k = \C\) and \(\Z \subseteq \C\) then \(Z(I(\Z)) = \C\) as if a polynomial vanishes at \(\Z\) then it must be zero.
\end{eg}

We have a correspondence
\[
  \begin{tikzcd}
    \{\text{algebraic subsets of } \A^n\} \ar[r, "I", shift left] & \{\text{ideals in } k[x_1, \dots, x_n]\} \ar[l, "Z", shift left]
  \end{tikzcd}
\]
Note that this is not quite a bijection. For example in \(k[x]\),
\[
  Z(x) = Z(x^2) = Z(x^3) = \cdots
\]
and more generally
\[
  Z(f_1^{q_1} \cdots f_r^{g_r}) = Z(f_1 \cdots f_r)
\]
where \(q_i > 0\). We will fix this shortly.

\subsection{Hilbert's Nullstellensatz}

Let \(Y \subseteq \A^n\) be an algebraic subset so \(Y = Z(I(Y))\). Recall that we have a map \(k[x_1, \dots, x_n] \to \{\cl k^n \to \cl k\}\). Hence by restriction we have a map \(k[x_1, \dots, x_n] \to \{Y \to \cl k\}\) as \(Y \subseteq \cl k^n\). By definition \(I(Y) \mapsto 0\). This motivates us to make the following definition:

\begin{definition}
  Let \(Y \subseteq \A^n\) be an algebraic set. Then
  \[
    k[Y] = k[x_1, \dots, x_n]/I(Y).
  \]
\end{definition}

We've just seen \(k[Y] \embed \{Y \to \cl k\}\) so \(k[Y]\) is a special class of functions on \(Y\), namely ``polynomial functions on \(Y\) with \(k\)-coefficients''.

\begin{eg}
  \(k[\A^n] = k[x_1, \dots, x_n]\).

  \(k[E] = k[x, y]/(y^2 - x^3 + x)\).
\end{eg}

Clearly \(k[Y]\) is a \(k\)-algebra. Our aim is to recover \(Y\) completely from this \(k\)-algebra.

Observe that if \(p \in Y \subseteq \A^n(k)\) then the map
\begin{align*}
  k[Y] &\to k \\
  f &\mapsto f(p)
\end{align*}
is an algebra homomorphism. It is surjective and its kernel, denoted
\[
  \mathfrak m_p = \{f \in k[Y]: f(p) = 0\},
\]
is a maximal ideal, as \(k[Y]/\mathfrak m_p\) is a field. So
\[
  \{\text{points in } Y\} \embed \{\text{algebra homomorphism } k[Y] \to \cl k\} \embed \{\text{max ideals } \mathfrak m \subseteq k[Y]\}.
\]
It is remarkable that if \(k = \cl k\) then all of these coincides (it is particularly so for the first inclusion, as it gives a translation between geometry and algebra. By contrast, the second inclusion is more or less a corollary of a standard result in algebra).

What are the maximal ideals of \(k[x_1, \dots, x_n]\)? We've observed if \(p \in k^n\) then \(\{f \in k[x_1, \dots, x_n]: f(p) = 0\}\) is a maximal ideal.

Not all maximal ideals are of the form \(\mathfrak m_p\), however. For example if \(k = \R\) then \((x^2 + 1) \subseteq \R[x]\) is a maximal ideal as \(\R[x]/(x^2 + 1) \cong \C\). Nevertheless, notice that \(\R \subseteq \C\) and this is an extension of \(\R\).

\begin{theorem}[Nullstellensatz]\index{Nullstellensatz}\leavevmode
  \label{thm:Nullstellensatz}
  If \(\mathfrak m \subseteq k[x_1, \dots, x_n]\) is a maximal ideal then \(k[x_1, \dots, x_n]/\mathfrak m = L\) is an algebraic field extension of \(k\), and finite-dimensional over \(k\).
\end{theorem}

Note that in this setting \(L\) is finite-dimensional over \(k\) if and only if every \(\alpha \in L\) is algebraic over \(k\). For the nontrivial direction, images of \(x_1, \dots, x_n\) in \(L\) generate \(L\) and each satisfies a polynomial equation of degree \(d_i\) so \(\dim_k L \leq d_1 \cdots d_n\).

\begin{corollary}
  If \(k = \cl k\) then the field embedding \(k \to L\) is an isomorphism, that is every maximal ideal of \(k[x_1, \dots, x_n]\) is of the form
  \[
    \mathfrak m_p = (x_1 - p_1, \dots, x_n - p_n)
  \]
  for \(p \in k^n\).
\end{corollary}

\begin{proof}
  \(L \supseteq k\) is an algebraic extension of fields so \(L = k\) as \(k = \cl k\) and \(p_i\) is the image of \(x_i\) under the map \(k[x_1, \dots, x_n] \to k[x_1, \dots, x_n]/\mathfrak m_p = L\).
\end{proof}

\begin{corollary}
  If \(Y \subseteq \A^n\) is an algebraic set. Suppose \(k = \cl k\). then we have bijection
  \[
    \begin{tikzcd}
      \{\text{points in } Y\} \ar[r] & \{\text{algebra homomorphisms } k[Y] \to k\} \ar[r] & \{\text{maximal ideals of } k[Y]\} \\
      p \ar[r] & \ev_p: f \mapsto f(p) \\
      & \varphi \ar[r] & \ker \varphi \\
      \varphi(p) & \varphi \ar[l] \\
      & k[Y] \to k[Y]/\mathfrak m = k & \mathfrak m \ar[l]
    \end{tikzcd}
  \]
\end{corollary}

\begin{proof}
  When \(Y = \A^n\) this is Nullstellensatz. In general, an algebra homomorphism \(\varphi: k[x_1, \dots, x_n]/I \to k\) is the same thing as an algebra homomorphism \(k[x_1, \dots, x_n] \to k\) with \(I\) its kernel.
\end{proof}

We will give better proof later when we are more adept at playing with polynomail equations, but for now we'll prove a special case.

\begin{proof}[Proof of \nameref{thm:Nullstellensatz} when \(k\) is uncountable]
  Suppose \(L\) is not algebraic. Then there exists \(t \in L\) which is not algebraic over \(k\) so \(k(t) \subseteq L\). But observe the following:
  \begin{enumerate}
  \item \(L\) has countable dimension as a vector space over \(k\).
  \item The set
    \[
      \left\{ \frac{1}{t - \lambda}: \lambda \in k\right\} \subseteq L
    \]
    is linearly independent: suppose not then exist \(\lambda_1, \dots, \lambda_r, a_1, \dots, a_r \in k\) with
    \[
      \sum_{i = 1}^r \frac{a_i}{t - \lambda_i} = 0.
    \]
    Clear the denoimators, we get an algebraic identity that \(t\) satisfies, contradicting \(t\) transcendental.
  \end{enumerate}
  This implies that \(\{\lambda: \lambda \in k\}\) is countable. Absurd.
\end{proof}

% aside on logic and model theory

\begin{corollary}[Nullstellensatz]
  Let \(k = \cl k\), \(I \subseteq k[x_1, \dots, x_n]\) an ideal. Then \(Z(I) \neq \emptyset\) if \(I \neq k[x_1, \dots, x_n]\).

  More generally, let \(k = \cl k\), \(I \subseteq k[Y]\) has \(Z(I) \neq \emptyset\) if \(I \neq k[Y]\).
\end{corollary}

\begin{proof}
  If \(I \neq k[x_1, \dots, x_n]\) then \(I \subseteq \mathfrak m\) for some maximal ideal \(\mathfrak m\). But Nullstellensatz says that \(Z(\mathfrak m) = \{p\}\) for some \(p \in k^n\) as \(\mathfrak m = \mathfrak m_p\) for some \(p\). Thus
  \[
    Z(I) \supseteq Z(\mathfrak m) = \{p\} \neq 0.
  \]
\end{proof}

This motivates us to give an abstract, algebraic definition of algebraic set.

\begin{definition}[radical]\index{radical}
  Let \(R\) be a ring and \(J \subseteq R\) an ideal. The \emph{radical} of \(J\) is
  \[
    \sqrt J = \{f \in R: f^n \in J \text{ for some } n \geq 1\}.
  \]
\end{definition}

\begin{lemma}
  Given an ideal \(J \subseteq R\), \(\sqrt J\) is an ideal.
\end{lemma}

\begin{proof}
  If \(f, g \in \sqrt J\) then \(f^n \in J, g^m \in J\) for some \(n, m\). Then
  \[
    (f + g)^{n + m} = \sum_{i = 0}^{n + m} \binom{n + m}{c} f^i g^{n + m - i} \in J
  \]
  so \(fg \in \sqrt J\).

  If \(r \in R, f \in \sqrt J\) then \(rf \in \sqrt J\).
\end{proof}

\begin{eg}\leavevmode
  \begin{enumerate}
  \item \(\sqrt{(x^n)} = (x)\) in \(k[x]\).
  \item If \(J\) is a prime ideal then \(\sqrt J = J\).
  \item If \(f \in k[x_1, \dots, x_n]\) is irreducible then \((f)\) is a prime ideal. As \(k[x_1, \dots, x_n]\) is a UFD, \(\sqrt{(f)} = (f)\).
  \end{enumerate}
\end{eg}

Note that \(Z(J) = Z(\sqrt J)\).

\begin{theorem}[Nullstellensatz]
  If \(k = \cl k\), \(J \subseteq k[x_1, \dots, x_n]\) then \(I(Z(J)) = \sqrt J\).
\end{theorem}

\begin{proof}
  Let \(f \in I(Z(J))\) so \(f(p) = 0\) for all \(p \in Z(J)\). We must show that \(f^n \in J\) for some \(n > 0\). Consider \(k[x_1, \dots, x_n, t]/(tf - 1) = k[x_1, \dots, x_n, \frac{1}{f}]\). Let \(I\) be the ideal in this ring generated by the image of \(J\). Claim that \(Z(I) = \emptyset\): if not, let \(p \in Z(I)\). As \(J \subseteq I\), \(p \in Z(J)\) so \(f(p) = 0\). But \(p = (p_1, \dots, p_n, p_t)\) with \(p_t f(p_1, \dots, p_n) = 1\), i.e.\ \(f(p) \neq 0\). Absurd.

  Then the corollary to the Nullstellensatz implies that \(I = k[x_1, \dots, x_n, \frac{1}{f}]\) (we used the fact \(k = \cl k\)). As \(1 \in I = (J)\),
  \[
    \sum_{i = 1}^N \frac{\gamma_i}{f^i} = 1
  \]
  for some \(\gamma_i \in J\) for some \(N \geq 1\). Multiply by \(f^N\), get
  \[
    f^N = \sum_{i = 1}^N \gamma_i f^{N -i} \in J.
  \]
\end{proof}

\begin{remark}
  Let's try to deconstruct this mysterious proof. What are the points of \(k[x_1, \dots, x_n, t]/(tf - 1) = k[Y]\)? Here
  \[
    Y = \{(p_1, \dots, p_n, p_t) \in \A^{n + 1}: p_t f(p_1, \dots, p_n) = 1\}
  \]
  which is isomorphic as a set to
  \[
    \{(p_1, \dots, p_n) \in \A^n: f(p) \neq 0\} = \A^n \setminus Z(f).
  \]
  So \(Y\) is a Zariski closed subset of \(\A^{n + 1}\) which is isomorphic as a set to \(\A^n \setminus Z(f)\), and our proof was asking in what sense does
  \[
    Z(f) \cap (\A^n \setminus Z(f)) = \emptyset
  \]
  mean in terms of the ideal \(J\).
\end{remark}

\begin{corollary}
  Suppose \(k = \cl k, I, J \subseteq k[x_1, \dots, x_n]\). Then \(Z(I) = Z(J)\) if and only if \(I(Z(I)) = I(Z(J))\) if and only if \(\sqrt I = \sqrt J\). That is we have a bijection between
  \[
    \begin{tikzcd}
      \{\text{Zariski closed subsets of } \A^n\} \ar[r, "I", shift left] & \{\text{radical ideal } I \subseteq k[x_1, \dots, x_n]\} \ar[l, "Z", shift left] \\
      p & \mathfrak m_p
    \end{tikzcd}
  \]
\end{corollary}

This is a hint that we may have an intrinsic characterisation of rings \(k[Y]\). We'll do this shortly.

\begin{definition}[(ir)reducible, disconnected]\index{reducible}\index{irreducible}\index{disconnected}
  An algebraic subset \(Y\) is \emph{reducible} if there exist algebraic subsets \(Y_1, Y_2 \neq Y\) such that \(Y_1 \cup Y_2\). It is \emph{irreducible} if is not reducible. It is \emph{disconnected if \(Y_1 \cap Y_2 = \emptyset\)}.
\end{definition}

\begin{eg}\leavevmode
  \begin{enumerate}
  \item \(Z(xy) = Z(x) \cup Z(y)\) is reducible.
  \item \(Z(y(y - 1), x(y - 1)) = Z(x, y) \cup Z(y - 1)\) is reducible and disconnected.
  \end{enumerate}
\end{eg}

In other words, \(Y\) is reducible/disconnected in Zariski topology. In usual topology, such as the usual one \(\R\), almost every set is reducible. However, in Zariski topology there is so little closed sets that this is actually a useful definition. In fact, they have a very nice algebraic characterisation:

\begin{lemma}
  \(Y\) is irreducible if and only if \(I(Y)\) is a prime ideal in \(k[x_1, \dots, x_n]\).
\end{lemma}

\begin{proof}
  If \(Y = Y_1 \cup Y_2\) is reducible then exists \(p \in Y_1 \setminus Y_2\) so exists \(f \in I(Y_2)\) with \(f(p) \neq 0\). Similarly exists \(q \in Y_2 \setminus Y_1\) so exists \(g \in I(Y_1)\) with \(g(q) \neq 0\). So
  \[
    fg \in I(Y_1) \cap I(Y_1) = I(Y).
  \]
  But \(f, g \notin I(Y)\) so \(I(Y)\) is not prime.

  Conversely if \(I(Y)\) is not prime then exists \(f_1, \dots, f_2 \in k[x_1, \dots, x_n]\) with \(f_1, f_2 \notin I(Y)\) but \(f_1f_2 \in I(Y)\). Set \(Y_i = Y \cap Z(f_i)\). Then \(Y_1 \cup Y_1 = Y\) as for \(p \in Y\), \(f_1f_2(p) = 0\) so \(f_1(p) = 0\) or \(f_2(p) = 0\) and \(Y_i \neq Y\) as \(f_2 \notin I(Y)\).
\end{proof}

\begin{eg}
  \(I = (x_1, \dots, x_m) \subseteq k[x_1, \dots, x_m]\) is prime, as
  \[
    k[x_1, \dots, x_n]/(x_1, \dots, x_m) = k[x_{m + 1}, \dots, x_n]
  \]
  is an integral domain.
\end{eg}

Recall that if \(R\) is a UFD, \(f \in R\) nonzero then if \(f\) is irreducible then \((f)\) is a prime ideal. Furthermore as \(k[x_1, \dots, x_n]\) is a UFD, it is an exercise to check that \(Z(y -x^2), Z(y^2 - x^3 + x)\) are irreducible.

Zariski topology is very different from usual topology: if \(X\) is an irreducible Zariski closed subset and \(U \subseteq X\) is a nonempty Zarisk open subset in \(X\) then \(\cl U = X\), i.e.\ nonempty Zariski open subsets are dense.

\begin{proof}
  Let \(Y = X \setminus U\) which is closed. Then \(\cl U \cup Y = X\) and \(U \neq \emptyset\) so \(Y \neq X\). But \(X\) is irreducible so \(\cl U = X\).
\end{proof}

\begin{application}[Cayley-Hamilton]
  Let \(A \in \text{Mat}_n(k)\), an \(n \times n\) matrix. Define its characteristic polynomial to be
  \[
    \chi_A(x) = \det(xI - A) \in k[x]
  \]
  This defines a map
  \begin{align*}
    \text{Mat}_n(k) &\to \text{Mat}_n(k) \\
    B &\mapsto \chi_A(B)
  \end{align*}
  Then for all \(A\), \(\chi_A(A) = 0\).

\begin{proof}
  Strategy:
  \begin{enumerate}
  \item The set of matrices for which Cayley-Hamilton holds is a Zariski closed subset of \(\A^{n^2}\).
  \item It holds for diagonalisable matrices, which is a Zariski open subset of \(\A^{n^2}\).
  \item Hence as \(\A^{n^2}\) is an irreducible algebraic set, it holds for all matrices.
  \end{enumerate}

  Let \(X = \text{Mat}_n(k) = k^{n^2} = \A^{n^2}\) be the space of matrix coefficients. It is an affine space so irreducible closed. Consider
  \[
    C = \{A \in \text{Mat}_n(k): \chi_A(A) = 0\}.
  \]
  Claim that this is a Zariski closed subset, cut out by \(n^2\) equations of the form \(\chi_A(A)_{ij} = 0\). We must check these equations are polynomial equations in the matrix coefficients of \(A\). Note that
  \[
    \chi_A(x) \in k[X \times \A^1] = k[\A^{n^2 + 1}],
  \]
  i.e.\ \(\det (xI - A)\) is a polynomial equation in \(n^2 + 1\) variables: matrix coefficients of \(A\) and \(x\). Now substitute \(x = A\). Note that matrix coefficients of \(A^r\), \((A^r)_{ij}\), are polynomials in the matrix coefficients of \(A\) (of degree \(r\)). Hence \(\chi_A(A)_{ij}\) are polynomial equations in coefficients of \(A\).

  As \(\text{Mat}_n(k) \subseteq \text{Mat}_n(\cl k)\), suffices to prove the case \(k = \cl k\). Note that
  \begin{align*}
    \chi_A(x) &= \chi_{gAg^{-1}}(x) \\
    \chi_A(gBg^{-1}) &= g \chi_A(B) g^{-1}
  \end{align*}
  for all \(g \in \GL_n(k)\). so \(\chi_A(A) = 0\) if and only if \(\chi_{gAg^{-1}} (gAg^{-1}) = 0\), so \(A\) satisfies its only characteristic polynomial if and only if \(gAg^{-1}\) does for all \(g \in \GL_n(k)\).

  Now let \(U\) be the set of all matrices with distinct eigenvalues. As \(k = \cl k\), \(A \in U\) implies that there exists \(g \in \GL_n(k)\) such that \(gAg^{-1}\) is
  \[
    \begin{pmatrix}
      \lambda_1 \\
      & \lambda_2 \\
      & & \ddots \\
      & & & \lambda_n
    \end{pmatrix}
  \]
  which clearly satisfies its own characteristic polynomial. Moreover \(U \neq \emptyset\) since we can always find distinct elements \(\lambda_1, \dots, \lambda_n\) of \(k\) as \(k = \cl k\).

  Left to show \(U\) is Zariski open. \(A \in U\) if and only \(\chi_A(x) \in k[x]\) has distinct roots. But a polynomial \(f\) has distinct roots if and only if \(f\) and \(f'\) have no common root, if and only if \(\Delta(f) \neq 0\), where the discriminant \(\Delta(f)\) is a polynomial in the coefficients of \(f\). Hence \(A \in U\) if and only if \(\Delta(\chi_A(x)) \neq 0\), so \(U\) is Zariski open.
\end{proof}
\end{application}

Now back to the abstract characterisation of algebraic varieties. We need some preliminary definitions:

\begin{definition}[nilpotent]\index{nilpotent}
  Let \(R\) be a ring. \(y \in R\) is \emph{nilpotent} if exists \(n > 0\) such that \(y^n = 0\).
\end{definition}

\begin{eg}\leavevmode
  \begin{enumerate}
  \item If \(R = k[x]\) then \(0\) is the only nilpotent.
  \item If \(R = k[x]/(x^7)\) then \(x\) is nilpotent as \(x^7 = 0\).
  \end{enumerate}
\end{eg}

\begin{ex}
  Let \(J \subseteq k[x_1, \dots, x_n]\) be an ideal and \(R = k[x_1, \dots, x_n]/J\). Then \(J = \sqrt J\) if and only if \(R\) has no nonzero nilpotents.
\end{ex}

\begin{corollary}
  Let \(k = \cl k\). If \(Y \subseteq \A^n\) is a Zariski closed subset then \(k[Y]\) is a finitely generated \(k\)-algebra with no nonzero nilpotents.
\end{corollary}

Conversely, given a finitely generated reduced \(k\)-algebra \(A\), there exists a surjection \(k[t_1, \dots, t_n] \to A\). As \(A\) is reduced, the kernel is radical. This is precisely the definition of a coordinate ring.

What do we gain from this? We need not to choose a generator of the \(k\)-algebra, which is the same as an embedding \(Y \embed \A^n\). In this abstract formulation, the ``points'' in the affine space corresponds to maximal ideals of the \(k\)-algebra.

\begin{definition}[affine algebraic variety]\index{affine algebraic variety}\index{point}
  An \emph{affine algebraic variety} over \(k\), where \(k\) is a field, is a finitely generated \(k\)-algebra \(R\) with no nonzero nilpotent elements.

  If \(k = \cl k\), define a \emph{point} of \(R\) to be a \(k\)-algebra homomorphism \(R \to k\). More generally if \(L \supseteq k\) is a field extension then an \emph{\(L\)-point of \(R\)} is a \(k\)-algebra homomorphism \(R \to L\).
\end{definition}

\begin{eg}
  Let \(J = \sqrt J \subseteq k[x_1, \dots, x_n]\) be a radical ideal and \(R = k[x_1, \dots, x_n]/J\) be an affine algebraic variety.
  
  Coversely, if \(R\) is such an algebra, choose generators \(\overline x_1, \dots, \overline x_n\) of \(R\) as a \(k\)-algebra so get a surjective map \(k[x_1, \dots, x_n] \to R\) where \(x_i \mapsto \overline x_i\). Let \(J\) be the kernel and \(J = \sqrt J\) per the exercise above.

  By Nullstellensatz, points of \(R\) is \(Z(J) \subseteq k^n\) given by
  \begin{align*}
    Z(J) &\to \{R \to k\} \\
    p = (p_1, \dots, p_n) &\mapsto (\ev_p: \overline x_i \mapsto p_i)
  \end{align*}
  In general, choice of generators \(\overline x_1, \dots, \overline x_n\) of \(R\) is the \emph{choice} of an embedding of points of \(R\) to \(\A^n\).
\end{eg}

\begin{eg}
  \(\R[x]/(x^2 + 1)\) has no \(\R\)-point, but it has two \(\C\)-points, given by \(x \mapsto \pm i\).
\end{eg}

We indulge in imprecision and often write ``let \(Y\) be an affine algebraic variety and \(R = k[Y]\) be its ring of functions''. What we really mean, when spelt out, is: let \(R\) be an affine algebraic variety with \(\cl k\)-points \(Y\).

\begin{definition}[morphism]\index{morphism}\index{isomorphism}
  A \emph{morphism} \(\gamma: X \to Y\) of affine algebraic varieties is a \(k\)-algebra homomorphism \(\gamma^*: k[Y] \to k[X]\).

  An \emph{isomorphism} \(\alpha: X \to Y\) is a morphism such that there exists an inverse morphism \(\beta: Y \to X\) such that \(\alpha\beta = 1_Y, \beta\alpha = 1_X\).
\end{definition}

Let's unpack the definition. Suppose \(X\) and \(Y\) are the points of \(R\) and \(S\) respectively. If \(\gamma^*: S \to R\) is a \(k\)-algebra homomorphism and \(p \in X\) is a point of \(X\), that is, if \(\ev_p: R \to \cl k\) is a \(k\)-algebra homomorphism, then \(\ev_p \compose \gamma^*: S \to \cl k\) is a \(k\)-algebra homomorphism, so a point in \(Y\). Thus \(\gamma^*\) defines a map \(X \to Y\), which we denote by \(\gamma\).

So this definition is a clever way of saying the map \(\gamma\) is defined by polynomial equations.

\begin{eg}\leavevmode
  \begin{enumerate}
  \item Let \(X = \A^1, Y = \{(x, y) \in \A^2: x^2 = y^3\} = Z(x^2 - y^3)\). Let \(R = k[t]\). Claim \(t \mapsto (t^3, t^2)\) is a morphism \(X \to Y\). Unpack the definition, we have \(k[Y] = k[x, y]/(x^2 - y^3)\) and a \(k\)-algebra homomorphism
    \begin{align*}
      \gamma^*: k[x, y]/(x^2 - y^3) &\to k[t] \\
      x &\mapsto t^3 \\
      y &\mapsto t^2
    \end{align*}
    Check that \(x^2 - y^3 \mapsto 0\) so it is well-defined.
  \end{enumerate}
\end{eg}

Unravel the definition of a morphism in general, let \(k[X] = k[x_1, \dots, x_n]/(s_1, \dots, s_\ell)\), \(k[Y] = k[y_1, \dots, y_m]/(r_1, \dots, r_k)\) (remember choice of generators \(x_1, \dots, x_n\) is choice of embeddings \(X \embed \A^n\)). Let \(\overline y_1, \dots, \overline y_m\) denote the image of \(y_1, \dots, y_m\) in \(k[Y]\). An algebra homomorphism \(\gamma^*: k[Y] \to k[X]\) is uniquely determined by where \(\overline y_1, \dots, \overline y_m\) go, i.e.\ by
\[
  \overline \Phi_i = \gamma^*(\overline y_i) \in k[X].
\]
Choose a polynomial \(\Phi_i \in k[x_1, \dots, x_n]\) whose reduction is \(\overline \Phi_i\). Such a choice determines an algebra homomorphism
\begin{align*}
  k[y_1, \dots, y_m] &\to k[x_1, \dots, x_n] \\
  y_i &\mapsto \Phi_i
\end{align*}
i.e.\ a morphism \(\A^n \to \A^m\), and the conditions on the polynomials \(\Phi_i\) ensure the image is in \(Y\) are the condition that the ideal \((r_1, \dots, r_k)\) is sent to \(0\) in \(k[X]\), i.e.\ \(r_i(\Phi_1, \dots, \Phi_m) \in (s_1, \dots, s_\ell) = 0 \in k[X]\).

\begin{question}
  Is the morphism in the above example an isomorphism?
\end{question}

\begin{eg}\leavevmode
  \begin{enumerate}
  \item A morphsim \(\A^1 \to \A^n\) is a \(k\)-algebra homomorphism \(k[x_1, \dots, x_n] \to k[t]\), which is the same as an \(n\)-tuple of polynomials \((\Phi_1(t), \dots, \Phi_n(t))\).
  \item A morphism \(X \to \A^1\) is an \(k\)-algebra homomorphism \(k[t] \to k[X]\), which is an element of \(k[X]\) (i.e.\ where \(t\) is sent to). This says that \(k[X]\) is precisely the functions \(X \to \A^1\), which is something we knew before!
  \item Suppose \(\ch k \neq 2\). Is there a morphism \(\A^1 \to E = \{(x, y): y^2 = x^3 - x\}\)? Suppose \(k = \C\), this is aksing if there is a polynomial map from the punctured sphere to the punctured torus. From analytic point of view this is impossible (there is not even an analytic functions does this). Algebraically, this is asking if there exist polynomials \(a(t), b(t) \in k[t]\) such that \(b^2 = a^3 - a\). See example sheet 1.
  \item Let \(X\) be an affine algebraic variety and let \(f \in k[X]\). Consider
    \[
      k[X] \to k[X][t]/(tf - 1) = k[Y]
    \]
    which defines a morphism \(Y \to X\). What is \(Y\) and what is the morphism? By definition a point of \(Y\) is a \(k\)-algebra homomorphism \(\gamma: k[X][t]/(tf - 1) \to k\). Suppose \(\gamma(t) = a\) then \(\gamma|_{k[X]} = \ev_p\) where \(p \in X\) such that \(a f(p) = 1\), i.e.\ \(f(p) = \frac{1}{a} \neq 0\). Conversely, if \(f(p) \neq 0\), set \(a = \frac{1}{f(p)}\), we get a \(k\)-algebra homomorphism. So
    \[
      Y = \{x \in X: f(x) \neq 0\} = X \setminus Z(f)
    \]
    which is Zariski open, and \(\gamma: Y \embed X\) is the inclusion. In general, Zariski open sets of the form \(X \setminus Z(f)\) are affine varieties in their own right, and the inclusion map is a morphism of affine varieties.
  \end{enumerate}
\end{eg}

By the same argument the complement of the subvariety cut out by a single polynomial is a variety. We call them

\begin{definition}[hypersurface]\index{hypersurface}
  If \(f \in k[x_1, \dots, x_n]\) then \(Z(f) \subseteq \A^n\) is called a \emph{hypersurface}.
\end{definition}

We may ask: is every Zariski open set also an affine variety, i.e.\ the image of an affine variety inside some bigger affine space under an injection?. No! \(\{(x, y) \in \A^2: (x, y) \neq (0, 0)\}\) is not an affine variety.

\section{Smooth points, dimension \& Noether normalisation}

Let \(X \subseteq \A^n\) be an affine variety and \(p \in X\). Let \(X = Z(I)\), \(I = (f_1, \dots, f_r)\). Tentatively we define
\begin{align*}
  T_pX
  &= \{v \in \A^n: \sum v_i \frac{\partial f}{\partial x_i}(p) = 0 \text{ for all } f \in I\} \\
  &= \{v \in \A^n: \sum v_i \frac{\partial f_j}{\partial x_i}(p) = 0, j = 1, \dots, r\}
\end{align*}
Translate \(T_pX\) from the origin to \(p \in \A^n\) so the equations are
\[
  \{v \in \A^n: \sum (v_i - p_i) \frac{\partial f}{\partial x_i}(p) = 0 \text{ for all } f \in I\}.
\]
This is the best linear approximation to \(X\) at the point \(p\), as
\[
  f(x) = f(p) + \sum (x_i - p_i) \frac{\partial f}{\partial x_i}(p) + \dots
\]
If \(X\) is complex analytic then this is indeed the analytic definition of tangent space. However it's not always the case.

\begin{eg}
  If \(I = (x^2 - y^3)\) then
  \[
    T_{(a, b)}(X) = \{(v_1, v_2): v_1(2a) + v_2(-3b^2) = 0\}.
  \]
  If \((a, b) \neq (0, 0)\) this is a line and if \((a, b) = (0, 0)\) then this is \(\A^2\). (graph)
\end{eg}

\begin{lemma}
  \(\{p \in X: \dim T_pX \geq t\}\) is a Zariski closed subset of \(X\) for all \(t \geq 0\).
\end{lemma}

\begin{proof}
  Write \(T_p X = \ker (A: k^n \to k^r)\) where \(A\) is the matrix
  \[
    \begin{pmatrix}
      \frac{\partial f_1}{\partial x_1}(p) & \cdots & \frac{\partial f_1}{\partial x_n}(p) \\
      & \ddots \\
      \frac{\partial f_r}{\partial x_1}(p) & \cdots & \frac{\partial f_r}{\partial x_n}(p)
    \end{pmatrix}
  \]
  By rank-nullity, \(\dim \ker A \geq t\) if and only if \(\text{rank} A \leq n - t\). But rank of a matrix \(A\) is greater than or equal to \(s\) if and only if there exists an \(s \times s\) subminor \(B\) with \(\det B \neq 0\), which is a polynomial equation in matrix coefficients. Thus \(\text{rank} A \leq n - t\) if and only if all \((n + 1 - t) \times (n + 1 - t)\) subminors have zero determinant. Hence
  \[
    \{p \in X: \dim T_pX \geq t\} = Z(f_1, \dots, f_r, \text{  determinants of subminors}).
  \]
\end{proof}

\begin{definition}[dimension]\index{dimension}
  Let \(X\) be an irreducible affine variety. Then
  \[
    \dim X = \min \{\dim T_pX: p \in X\}.
  \]
  If \(k \neq \cl k\) then \(p\) is taken to be \(\cl k\)-points.
\end{definition}

In a moment we'll show \(T_pX\) is independent of embedding \(X \embed \A^n\).

We require \(X\) to be irreducible as if not then each component can have different dimensions and \(\dim X\) is not a good notion but we may as well define it anyway: we let
\[
  \dim X = \max \{\dim X_i: X_i \text{ irreducible component of } X\}.
\]

\begin{lemma}
  Suppose \(k = \cl k\). Let \(f \in k[x_1, \dots, x_n]\) be a nonconstant irreducible polynomial. Then \(Z(f)\) has dimension \(n - 1\).
\end{lemma}

\begin{proof}
  \(\dim T_p Z(f)\) is either \(n\) or \(n - 1\) as there is only one equation. If \(\dim T_pZ(f) = n\) then \(\frac{\partial f}{\partial x_i}(p) = 0\) for all \(i\) so if \(\dim Z(f) = n\) then
  \[
    \frac{\partial f}{\partial x_i} \in I(Z(f)) = \sqrt{(f)} = (f)
  \]
  as \((f)\) is prime. Write \(\frac{\partial f}{\partial x_i} = fg\) for some \(g \in k[x_1, \dots, x_n]\). But \(\deg_{x_i} \frac{\partial f}{\partial x_i} < \deg_{x_i} f\) so \(g = 0, \frac{\partial f}{\partial x_i} = 0\). Thus we have shown \(\dim Z(f) = n\) implies that \(\frac{\partial f}{\partial x_i} = 0\) for all \(i\).

  If \(\ch k = 0\) then \(f\) is a constant, \(Z(f) = \emptyset\), contradiction. If \(\ch k = p\) this implies \(f \in k[x_1^p, \dots, x_n^p]\). Claim that there exists \(h \in k[x_1, \dots, x_n]\) such that \(f = h^p\), contradicting \(f\) being prime: write \(f = \sum a_\lambda x^{p \lambda}\) for \(a_\lambda \in k\). As \(k = \cl k\), \(a_\lambda^{1/p}\) exists. Set \(h(x) = \sum a_\lambda^{1/p} x^\lambda\). As \(\ch k = p\), \(h^p = f\).
\end{proof}

\begin{eg}\leavevmode
  \begin{enumerate}
  \item \(\dim \A^n = n\).
  \item Any plane curve \(f(x, y)\) has dimension \(1\).
  \end{enumerate}
\end{eg}

\begin{definition}[smooth, singular point]\index{smooth}\index{singular}
  Suppose \(k = \cl k\). Let \(X\) be an irreducible algebraic variety and \(p \in X\). \(p\) is \emph{smooth} if \(\dim T_pX = \dim X\). \(p\) is \emph{singular} otherwise.
\end{definition}

Thus the above lemma says that singular points form a Zariski closed subvariety and smooth points form a Zariski open subset, which is non-empty.

\begin{proposition}[nonexaminable]
  If \(k = \C\) and \(\dim X = d\), then \(p \in X\) is smooth if and only if there exists an isomorphism from a small ball around \(0 \in \C^d\) to a small neighbourhood of \(p \in X\) in the usual topology.
\end{proposition}

This is obviously false in Zariski topology.

\begin{proof}
  This is a consequence of implicit function theorem.
\end{proof}

\begin{definition}[derivation]\index{derivation}
  Let \(A\) be a \(k\)-algebra and \(\varphi: A \to k\) a \(k\)-algebra homomorphism. A \emph{derivation centred at \(\varphi\)} is a \(k\)-linear map \(D: A \to k\) such that
  \[
    D(fg) = \varphi(f) D(g) + D(f) \varphi(g)
  \]
  for all \(f, g \in A\). Write \(\Der(A, \varphi)\) for derivations centred at \(\varphi\).
\end{definition}

\begin{eg}
  \(f \mapsto \frac{\partial f}{\partial x}(p)\) is a derivation centred at \(p\).
\end{eg}

\begin{lemma}
  If \(X \subseteq \A^n\) then for all \(p \in X\),
  \[
    T_pX = \Der(k[X], \ev_p).
  \]
\end{lemma}

\begin{proof}
  If \(X = \A^n\), \(k[X] = k[x_1, \dots, x_n]\). Let \(D \in \Der(k[X], \ev_p)\). Let \(v_i = D(x_i)\). This gives a map
  \begin{align*}
    \Der(k[X], \ev_p) &\to \A^n \\
    D &\mapsto (D(x_i) = v_i)
  \end{align*}
  Conversely, given \(v \in \A^n\), define a derivation \(D\) by
  \[
    D(f) = \sum v_i \frac{\partial f}{\partial x_i}(p).
  \]

  In general, \(k[X] = k[x_1, \dots, x_n]/(f_1, \dots, f_r)\). Let \(p \in X = Z(f_1, \dots, f_r)\). Then
  \begin{align*}
    \Der(k[X], \ev_p)
    &= \{D \in \Der(k[x_1, \dots, x_n], \ev_p): D|_{(f_1, \dots, f_n)} = 0\} \\
    &= \{D \in \Der(k[x_1, \dots, x_n], \ev_p): \sum v_j \frac{\partial f_i}{\partial x_j}(p) = 0 \text{ for all } i\}
  \end{align*}
  where \(v_j = D(x_j)\).
\end{proof}

Observe that if \(\alpha: X \to Y\) is a morphism of varieties, i.e.\ \(\alpha^*: k[Y] \to k[X]\) is a \(k\)-algebra homomorphism, \(D \in \Der(k[X], \ev_p)\) then \(D \compose \alpha^* \in \Der(k[Y], \ev_{\alpha(p)})\). Thus we get a linear map \(T_pX \to T_{\alpha(p)}Y\).

\begin{ex}
  Let \(f \in k[X]\). Consider \(k[X] \to k[U] = k[X][t]/(tf - 1) = k[U]\). We get a morphism \(U = X \setminus Z(f) \to X\). Let \(p \in U\). Show this defines an isomorphism \(T_pU \to T_pX\).
\end{ex}

We have two more definitions of dimension of varieties, which turns out to be the same as the old one. To prove so we need some algebraic tools.

\begin{definition}[Krull dimension]\index{Krull dimension}
  Let \(X\) be an irreducible affine variety. The \emph{Krull dimension} of \(X\) is
  \begin{align*}
    \dim_{\text{Kr}} X
    &= \max \{r: Z_0 \subsetneq Z_1 \subsetneq \dots \subsetneq Z_r = X: Z_i \text{ irreducible Zariski closed}\} \\
    &= \max \{r: 0 = Z_r \lneq I_{r - 1} \lneq \dots \lneq I_0 = k[x]: I_r \text{ prime}\}
  \end{align*}
\end{definition}

\begin{eg}\leavevmode
  \begin{enumerate}
  \item If \(X = \A^1\) then \(\{\text{point}\} \subsetneq \A^1\) is the only such chain so \(X\) has Krull dimension \(1\).
  \item If \(X\) is a plane curve then it has Krull dimension \(1\), shown in example sheet 1.
  \end{enumerate}
\end{eg}

\begin{definition}[function field]\index{function field}
  Let \(X\) be an irrducible affine variety. Define the \emph{function field} of \(X\)
  \[
    k(X)
    = \operatorname{Frac} k[X]
    = \bigcup_{g \in k[X]} k[X][\frac{1}{g}]
    = \bigcup_{g \in k[X]} k[X \setminus Z(g)]
  \]
  which is non-zero as \(k[X]\) is an integral domain.

  We define the \emph{transcendence dimension} of \(X\) to be the transcendence degree of \(k(X)\) over \(k\)
  \[
    \dim_{\text{tr}} X = \operatorname{trdeg}_k k(X).
  \]
\end{definition}

\begin{eg}\leavevmode
  \begin{enumerate}
  \item \(k(\A^n) = k(x_1, \dots, x_n)\).
  \item \(E = \{(x, y): y^2 = x^3 - x\}\). Then \(k(E) = k(x)[y]/(y^2 - x^3 + x)\) which is an algebraic extension of \(k(x)\), so has trancedence dimension \(1\).
  \end{enumerate}
\end{eg}

\begin{theorem}
  Let \(X\) be an irreducible affine variety. Then
  \[
    \dim X = \dim_{\text{Kr}} X = \dim_{\text{tr}} X.
  \]
\end{theorem}

\begin{proof}
  Strategy of proof: show
  \[
    \dim \A^n = \dim_{\text{Kr}} \A^n = \dim_{\text{tr}} \A^n = n
  \]
  then reduce to this.
\end{proof}

We want to describe very special maps \(X \to Y\) with the property that \(\dim X = \dim Y, \dim_{\text{tr}} X = \dim_{\text{tr}} Y\), and then show these maps exist from \(X \to \A^n\) if \(\dim X = n\).

Suppose we have \(X, Y\) affine varieties such that
\begin{enumerate}
\item \(X\) and \(Y\) are irreducible,
\item there exists \(f \in k[Y][t]\) such that \(k[X] = k[Y][t]/(f(t))\) so
  \[
    f(t) = a_0(y) + a_1(y)t + \dots + a_N(y)t^N = f(y, t),
  \]
  with \(a_i(y) \in k[Y]\), \(a_N \neq 0\). This defines a morphism \(\varphi: X \to Y\).
\item \(f\) is a separable polynomial when regarded as an element of \(k(Y)[t]\), i.e.\ let
\[
  F(t) = \frac{1}{a_N(y)} f(t) = t^N + \frac{a_{N - 1}}{a_N} t^{N - 1} + \dots + \frac{a_0}{a_N},
\]
then \(F(t), F'(t)\) no common roots. In other words, \(k(Y) \subseteq k(X)\) is a separable algebraic extension.
\end{enumerate}

\paragraph{Claim 1}

\(\varphi(X)\) contains an open, hence dense subset of \(Y\).

\begin{proof}
  By definition
  \[
    X = \{(y_0, t_0) \in Y \times \A^1: f(y_0, t_0) = 0\}
  \]
  so if \(y_0 \in Y \setminus Z(a_N)\), that is \(a_N(y_0) \neq 0\), then \(f(y_0, t)\) is a polynomial in \(t\) of degree \(N\), so has exactly \(N\) roots (counting with multiplicity) over \(\cl k\), i.e.\ there exists\footnote{Lecturer suddenly declares \(k = \cl k\).}\((y_0, t_0) \in X\) and \(\varphi(y_0, t_0) = y_0\), in particular non-empty.
\end{proof}

\paragraph{Claim 2}

There exists a non-empty Zariski open subset of \(Y\) such that the natural map \(T_{(y_0, t_0)}X \to T_{y_0}Y\) is an isomorphism.

\begin{remark}
  Consider \(Y = \A^1, X = \{(y, t): y= t^p\}\) with \(\ch k = p\). Then
  \[
    T_{(a, b)}X = \{(v_y, v_t): v_y - (pt^{p - 1}|_{(a, b)}) v_t = 0\}
    = \{(0, v_t): v_t \in \A^1\}
  \]
  as \(p = 0\). So \(T_{(a, b)}X \to T_a Y\) is the zero map. Thus separability assumption is important.
\end{remark}

\begin{proof}
  Choose generators for \(k[Y]\), i.e.\ \(Y \subseteq \A^n\). Then
  \begin{align*}
    T_{y_0}Y &= \{v \in \A^n: \sum v_i \frac{\partial h}{\partial x_i} (y_0) = 0 \text{ for all } h \in I(Y)\} \\
    T_{(y_0, t_0)} X &= \{(v, \gamma) \in \A^n \times \A^1: \sum v_i \frac{\partial h}{\partial x_i}(y_0) = 0 \text{ for all } h \in I(Y), \\
    &\qquad \sum v_i \frac{\partial f}{\partial x_i}(y_0, t_0) + \gamma \frac{\partial f}{\partial t}(y_0, t_0) = 0\}
  \end{align*}
  as \(I(X) = I(Y, f)\). But then
  \[
    T_{(y_0, t_0)} = \{(v, \gamma) \in T_{y_0} Y \times \A^1: \sum v_i \frac{\partial f}{\partial x_i}(y_0, t_0) + \gamma \frac{\partial f}{\partial t}(y_0, t_0) = 0\}
  \]
  Claim this is equivalent to: there exists Zariski open subset of \(Y\) included in the above such that \(\frac{\partial f}{\partial t}(y_0, t_0) \neq 0\) for all \(y_0 \in U\), and this is immediate if \(\frac{\partial f}{\partial t}\) is not the zero polynomial in \(k[Y][t]\), but our assumption about separability implies it is not.
\end{proof}

\[
  \begin{tikzcd}
    \varphi^{-1}(U) \ar[r] \ar[d] & X \ar[d, "\varphi"] \\
    U \ar[r] & Y
  \end{tikzcd}
\]
where \(U\) has finite fibre and \(\varphi\) restricted to \(\varphi^{-1}(U)\) induces isomorphism of tangent space.

\begin{corollary}
  \[
    \dim X = \dim Y, \dim_{\text{tr}} X = \dim_{\text{tr}} Y.
  \]
\end{corollary}

\begin{proof}
  \(\dim_{\text{tr}} X = \dim_{\text{tr}} Y\) is an immediate algebraic fact.

  Let \(Y^{\text{sm}}\) be the smooth points of \(Y\). As \(Y\) is irreducible, this is an open dense set and hence \(U \cap Y^{\text{sm}}\) is non-empty so \(\dim T_y Y = \dim Y\) if \(y \in Y^{\text{sm}} \cap U\) and
  \[
    \dim T_{(y, t)}X = \dim T_yY = \dim Y
  \]
  for all \((y, t) \in \varphi^{-1}(y)\). But \(\varphi^{-1}(U \cap Y^{\text{sm}}\) is an open set and \(X\) is irreducible, so
  \[
    \dim X = \dim T_xX = \dim Y
  \]
  for all \(x \in \varphi^{-1}(U \cap Y^{\text{sm}})\).
\end{proof}

\begin{theorem}[Noether normalisation theorem]\index{Noether normalisation theorem}
  Let \(X\) be an irreducible affine variety over \(k\) with \(\dim X = d\). Then there exists a surjective map \(p: X \to \A^d\) which is a composite of the above form (and in particular, \(\varphi^{-1}(y)\) is a finite set for all \(y \in \A^d\)).
\end{theorem}

\begin{corollary}
  \[
    \dim X = \dim \A^d = d = \dim_{\text{tr}} \A^d = \dim_{\text{tr}} X.
  \]
\end{corollary}

\begin{eg}
  Let \(X = \C^* = \{(x, y) \in \C^2: xy = 1\}\). Then Noether normalisation asserts that there is a surjection \(\C^* \to \C\), i.e.
  \begin{align*}
    \C^* &\to \C \\
    t &\mapsto t + t^{-1} = z
  \end{align*}
  \(k[t, t^{-1}] = k[z][t]/(t^2 - zt + 1)\).
\end{eg}

\begin{ex}
  Find a surjective map \(\A^1 \setminus \{\lambda_1, \dots, \lambda_N\} \to \A^1\).
\end{ex}

It is clear that \(\varphi: X \to Y\) such that \(k[X] = k[Y][t]/(f(t))\) with \(f\) monic is particularly nice. \(\varphi\) is surjective, the fibres are finite. Such a \(\varphi\) is an example of a \emph{finite flat morphism}.

Note that \(k[Y] \subseteq k[X]\) is an \emph{integral extension} of rings.

\begin{definition}
  \(B \subseteq A\) is an integral extension of rings if for all \(a \in A\), there exists a monic polynomial \(f(t) \in B[t]\) such that \(f(a) = 0\).
\end{definition}

\begin{lemma}\leavevmode
  \begin{enumerate}
  \item If \(f\) is a monic polynomial, \(B[t]/(f(t))\) is an integral extension of \(B\).
  \item If \(C \subseteq B, B \subseteq A\) are integral extensions then so is \(C \subseteq A\).
  \end{enumerate}
\end{lemma}

\begin{theorem}[Noether normalisation]
  Let \(A\) be a finitely generated \(k\)-algebra where \(k\) is a field and suppose \(A\) is an integral domain. Then there exists \(z_1, \dots, z_n \in A\) which generate \(A\) as a \(k\)-algebra such that
  \begin{enumerate}
  \item there exists \(d\) such that \(z_1, \dots, z_d\) are algebraically independent over \(k\),
  \item for all \(i > d\), \(z_i\) is algebraic over \(k[z_1, \dots, z_{i - 1}]\) with monic minimal polynomial \(F_i\).
  \end{enumerate}
  In particular, \(A\) is integral over \(k[z_1, \dots, z_d]\).

  Moreover if \(\operatorname{Frac} A\) is a separable field extension of \(k\) then we can also ensure \(F_i\)'s are separable polynomials, and we can always do this if \(k = \cl k\).
\end{theorem}

\begin{corollary}[Nullstellensatz]
  If \(A\) is a finitely generated \(k\)-algebra that is also a field then \(A \supseteq k\) is algebraic.
\end{corollary}

\begin{lemma}
  If \(B \subseteq A\) is an integral ring extension then
  \[
    B^\times = A^\times \cap B.
  \]
\end{lemma}

\begin{proof}
  Let \(b \in A^\times \cap B\). Then exists \(a \in A\) such that \(ab = 1\). As \(A \supseteq B\) is integral, exists \(c_i \in B\) such that
  \[
    a^n + c_{n - 1} a^{n - 1} + \dots + c_0 = 0.
  \]
  Multiply by \(b^{n - 1}\) to get
  \[
    a = -c_{n - 1} - c_{n - 2} b - \dots - c_0b^{n - 1} \in B.
  \]
\end{proof}

\begin{proof}
  Let \(z_1, \dots, z_n\) be as in Noether, so \(A\) is generated by \(z_1, \dots, z_n\) and \(z_1, \dots, z_d\) are transcendental over \(k\) and \(z_{i + 1}\) is integral over \(k[z_1, \dots, z_i]\) for \(i \geq d\). Claim that if \(d > 0\) then \(A\) is \emph{not} a field: if \(d > 0\) then the unit in \(k[z_1, \dots, z_d]\) are just \(k^\times\). So \(z_1\) is not invertible in \(k[z_1, \dots, z_d]\), so not invertible in \(A\) by the lemma.
\end{proof}

\begin{proof}
  As \(A\) is finitely generated, there exist generators \(z_1, \dots, z_n\). wlog \(z_1, \dots, z_d\) are algebraiclly independent and \(A\) is algebraic over \(k[z_1, \dots, z_d]\). If \(d = n\) then done. Otherwise assume the theorem holds for all \(k\)-algebras with \(\leq n - 1\) generators. Let \(A' = k[z_1, \dots, z_{n - 1}]\). There exists nonzero \(f \in k[z_1, \dots, x_n]\) such that
  \[
    f(z_2, \dots, z_{n - 1}, z_n) = 0.
  \]
  Write \(f = \sum_{i \leq N} F_i\) where \(F_N, F_i \in k[x_1, \dots, x_n]\) of degree \(i\) as a polynomial in \(x_n\).

  Suppose \(k\) is infinite, then there exist \(\lambda_1, \dots, \lambda_n \in k\) such that
  \[
    F_N(\lambda_1, \dots, \lambda_n) \neq 0.
  \]
  Set \(x_i' = x_i - \lambda_i x_n\) for \(i < n\) and \(x_n' = x_n\). Note that
  \[
    x_1^{e_1} \cdots x_n^{e_n}
    = (x_1' + \lambda_1 x_n)^{e_1} \cdots (x_{n - 1}' + \lambda_{n - 1} x_n)^{e_{n - 1}} x_n^{e_n}
    = \lambda_1^{e_1} \lambda_{n - 1}^{e_1 + \dots + e_n} + \text{ polynomials in \(x_1', \dots, x_n'\) whose \(x_n'\) degree is smaller}
  \]
  Hence
  \[
    f(x_1', \dots, x_n') = F_N(\lambda_1, \dots, \lambda_n) \cdot x_n^{e_1, \dots, e_n} + \text{ lower \(x_n\)-degree poly}
  \]
  But this implies that \(z_n' = z_n\) is integral over \(k[z_1', \dots, z_{n - 1}'] = A''\). But \(A''\) is generated by \(n - 1\) elements, so inductive hypothesis gives the result.

  Separability requires further argument.

  If \(k\) is finite then we use an argument of Nagata: let \(x_i = x_i - x_n^{\gamma_i}\) for \(\gamma_i\) sufficiently large.
\end{proof}

Fri 8 Feb

Exercise

Let \(k = \cl k\)  and \(X, Y\) irreducible varieties over \(k\) with \(\varphi: X \to Y\) a morphism. Show that
\begin{enumerate}
\item \(k[Y] \embed k[X]\) if and oif \(\cl{\varphi(X) = Y}\) 
\item If \(\cl{\varphi(X)} = Y\) then \(\dim X \geq \dim Y\). In fact, for all \(y \in \varphi(X)\),
  \[
    \dim \varphi^{-1}(y) \geq \dim X - \dim Y
  \]
  and equality holds on a dense open subset. (hard! Require Noether normalisation)
\end{enumerate}

\section{Projective space}

We will first define projective space as a set. Let \(V\) be a vector space over \(k\) with \(\dim V = n + 1\), \(n \geq 0\). Define
\[
  \P V = \P^n = \{\text{lines through origin in } V\} = V \setminus \{0\}/k^\times.
\]
Suppose \(v \in V, v \neq 0\), \(kv = \{\lambda v: \lambda \in k\}\) is a line. Conversely, \(\ell \in \P V\) is a line if and only if \(\ell = kv\) for any \(v \in \ell \setminus \{0\}\).

\begin{note}\leavevmode
  \begin{enumerate}
  \item \(V \setminus \{0\}\) is not an affine algebraic variety if \(\dim V > 1\).
  \item Quotienting a vaiety by the action of group like \(k^\times\) is subtle, even if the variety is affine. This is the subject of geometric invariant theory.
  \end{enumerate}
\end{note}

Choose a basis \(e_0, \dots, e_n\) of \(V\), i.e.\ an isomorphism \(V \cong k^{n + 1}\), write \([x_0 : \dots : x_n] \in \P^n\) for the line through \(\sum x_i e_i\) so
\[
  [x_0: \dots : x_n] = [\lambda x_0 : \cdots : \lambda x_n]
\]
for all \(\lambda \in k^\times\).

Claim \(\P^n = \A^n \amalg \P^{n - 1}\):

\begin{proof}
  Consider \(p = [x_0 : \dots : x_n]\). If \(x_n = 0\), \(p = [x_0 : \dots : x_{n - 1} : 0]\) determines a unique point in \(\P^{n - 1}\), and conversely if \(x_n \neq 0\) then
  \[
    [x_0 : \dots : x_n] = [\frac{x_0}{x_n} : \dots \frac{x_{n - 1}}{x_n} : 1].
  \]
  This gives a bijection as required.
\end{proof}

\begin{corollary}
  \[
    \P^n = \A^n \amalg \A^{n - 1} \amalg \dots \amalg \A^0.
  \]
\end{corollary}

We can't make \(\P^n\) an algebraic variety by gluing together a closed and open subset (more data is needed).

\begin{ex}
  Consider \(Z(x^2 - y^3) \subseteq \A^2\), which can be written as \(k^\times \amalg \{\text{pt}\}\). On the other hand, \(\A^1 = k^\times \amalg \{\text{pt}\}\).
\end{ex}

We want to rephrase \(\P^n = \A^n \amalg \P^{n - 1}\). Let \(H \leq V\) be a hyperplane, let \(w_0 \in V \setminus H\). For example \(H = \{x: x_n = 0\}, w_0 = (0, \dots, 0, 1)\). Then
\begin{align*}
  \P H &\embed \P V \\
  kv &\mapsto kv
\end{align*}
We also have
\begin{align*}
  H &\embed \P V \\
  h &\mapsto k(h + w_0)
\end{align*}
It is an exercise to show that
\[
  \P V \setminus \P H \cong H = \A^n
\]
and the isomorphism depends on the choice of \(w_0\).

Set \(U_i = \{x = [x_0: \dots : x_n] \in \P^n: x_i \neq 0\}, H_i = \{(x_0, \dots, x_n): x_i = 0\} \cong \A^n\) so \(\P V \setminus \P H_i = U_i\). It is clear that
\[
  U_0 \cup U_1 \cup \dots \cup U_n = \P^n
\]
as if \(x = [x_0 : \dots : x_n] \in \P V\), some \(x_i \neq 0\) and then \(x \in U_i\).

\begin{eg}\leavevmode
  \begin{enumerate}
  \item For \(n = 1\), \(U_0 = \{[1: x_1]\}, U_1 = \{[x_0 : 1]\}\). The inclusion is
    \begin{align*}
      U_0 &\to \P^1 \\
      [x_0 : x_1] &\to \frac{x_1}{x_0} \in \A^1 \cup \{\infty\}
    \end{align*}
  \item \(n = 2\): \(\P^2 = U_0 \cup U_1 \cup U_2\). \(\P^2 = U_i \amalg \P^1\).

    (graph) three lines at infinity in \(\P^2\). Exercise: the pattern of \(\P^{n - 1}\)'s at \(\infty\) in \(\P^n\) is given by the boundary of the \(n\)-simplex.
  \end{enumerate}
\end{eg}

Consider such a map \(j: U \embed \P^n\) where \(U = U_i = \A^n\) for some \(i\). This is an open embedding of topological spaces. It is an exercise to check this is an open embedding of topological spaces.

As each \(U_i \cong \A^n\) is an affine variety, and
\begin{align*}
  U_i \cap U_j &\to U_j \\
  k^\times \times \A^{n - 1} &\to \A^n
\end{align*}
is a morphism of affine variety, the \(\P^n\) is a well-defined algebraic variety, and \(U \to \P^n\) is a morphism of algebraic varieties.

11/Feb/19

Lots of maps \(\A^n \embed \P^n\) (choose a hyperplane and a point off the hyperplane. \((x_0, \dots, x_{n -1}) \mapsto (x_0, \dots, x_{n - 1}, 1)\). Call the map \(i: \A^2 \to \P^2\).

Let \(E^0 = \{(x, y) \in \A^2: y^2 = x^3 - x\}\). What is \(\cl{i(E^0)}\) in \(\P^2\)? Let's work it out. As \([x: y: 1] = [X: Y: Z]\) for \(z \neq 0\), have \(x = \frac{X}{Z}, y = \frac{Y}{Z}\) so \(y^2 = x^3 - x\) gives
\[
  Y^2Z = X^3 - XZ^2
\]
so
\[
  i(E^0) = \{[X: Y: Z] \in \P^2: Y^2Z = X^3 - XZ^2, Z \neq 0\}.
\]
From now on write \(E^0\) for \(i(E^0)\). Then the closure will be the same equation but allow \(Z = 0\). This is done as follow. There are three charts: \(X \neq 0, Y \neq 0, Z \neq 0\). In chart \(Z \neq 0\), \(y^2 = x^3 - x\). In chart \(Y \neq 0\), put \(z = \frac{Z}{Y}, x = \frac{X}{Y}\) so the equation for \(E^0\) is \(z = x^3 - xz^2\) and \(z \neq 0\). On chart \(X \neq 0\), put \(y = \frac{Y}{X}, z = \frac{Z}{X}\), equation is \(y^2 = 1 - z^2\) and now take closure of \(E^0\) in each chart, gievs closure of \(E^0\) in \(\P^2\).

If \([X: Y: Z]\) is in the chart \(Y \neq 0\) but not in chart \(Z \neq 0\), that says \(z = 0\). The equation says \(x^3 = 0\), which has a unique solution \(x = 0\), so we get an extra point \([0: 1: 0]\).

If \([X: Y: Z]\) is  in the chart \(X \neq 0\) and not in the chart \(Z \neq 0\) then \(z = 0\) and have \(0 = 1\) which has no solution, so no extra points.

Thus the projective curve \(E\), defined as the closure of \(E^0\) in \(\P^2\), is \(E^0 \cup \{[0: 1: 0]\}\), which is what we wanted in the first lecture.

Given \(X = Z(I) \subseteq \A^n\) where \(I \subseteq k[x_1, \dots, x_n]\), what is the closure of \(X\) in \(\P^n\) under the map \((x_1, \dots, x_n) \mapsto [1: x_1: \cdots : x_n]\)? Given \(f \in k[x_1, \dots, x_n]\), make it homogeneous : if \(\deg f = d\), define
\[
  \tilde f(X_0, \dots, X_n) = X_0^d f(\frac{X_1}{X_0}, \dots, \frac{X_n}{X_0})
\]
so
\[
  \tilde f(1, x_1, \dots, x_n) = f(x_1, \dots, x_n).
\]

\begin{eg}
  If \(f = y^2 - x^3 + x\) then
  \[
    \tilde f = Z^3((Y/Z)^2 - (X/Z)^3 + (X/Z)) = ZY^2 - X^3 + XZ^2
  \]
  and define \(\tilde 0 = 0\).
\end{eg}

For an ideal \(I \subseteq k[x_1, \dots, x_n]\), define
\[
  \tilde I = (\tilde f: f \in I),
\]

\begin{ex}
  \(\tilde I\) is an ideal and is homogeneous. \(\tilde I|_{X_0 = 1} = I\).
\end{ex}

\begin{definition}
  Given \(f \in k[x_1, \dots, x_n]\), \(f\) is \emph{homogeneous} of degree \(d\) if
  \[
    f = \sum_{c_1 + \dots + c_n = d} a_{i_1 \cdots i_n} x_1^{i_1} \cdots x_n^{i_n},
  \]
  if and only if
  \[
    f(\lambda x_1, \dots, \lambda x_n) = \lambda^d f(x_1, \dots, x_n)
  \]
  for all \(\lambda \in k^\times\), if \(k\) is infinite.
\end{definition}

Any \(f \in k[x_1, \dots, x_n]\) can be written as \(f = \sum_{r = 0}^n f_{(r)}\) where \(f_{(r)}\) is homogeneous of degree \(r\).

\begin{definition}
  An ideal \(I \subseteq k[x_1, \dots, x_n]\) is homogeneous if for all \(f \in I\), \(f = \sum_r f_{(r)}\) then \(f_{(r)} \in I\) for all \(I\).
\end{definition}

\begin{eg}
  \((xy + y^2, y^3, x^2)\) is homogeneous but \((xy + y^3)\) is not.
\end{eg}

\begin{lemma}\leavevmode
  \begin{enumerate}
  \item \(I \subseteq k[x_1, \dots, x_n]\) is homogeneous if and only if \(I\) is generated by a finite set of homogeneous polynomials.
  \item Suppose \(k\) is infinite. \(\tilde I \subseteq k[x_0, \dots, x_n]\) is a homogeneous ideal if and only if \(\tilde X = Z(\tilde I) \subseteq \A^{n + 1}\) is invariant under the \(k^\times\)-action \((p_0,, \dots, p_n) \mapsto (\lambda p_0, \dots \lambda p_n)\).
  \end{enumerate}
\end{lemma}

\begin{proof}
  Exercise.
\end{proof}

This shows that Zariski closed subsets of \(\P^n\) are defined by homogeneous ideals in \(k[x_0, \dots, x_n]\), corresponding to \(k^\times\)-invariant closed subsets of \(\A^{n + 1}\).

Warning: if \(I = (f_1, \dots, f_r) \subseteq k[x_1, \dots, x_n]\), it need not be the case that \(\tilde I = (\tilde f_1, \dots, \tilde f_r)\). For example if \(I = (x - y^2, y) = (x, y) = I(\{0\})\), but \((xz - y^2, y) = (xz, y) \neq (x, y) = \tilde I\).

\begin{ex}
  Can you find an ideal such that \(\tilde I \neq (\tilde f_1, \dots, \tilde f_r)\) for any minimal generator \(f_1, \dots, f_r\) of \(I\)?
\end{ex}

\begin{definition}[quasi-projective/affine variety]\index{quasi-projective variety}\index{quasi-affine variety}
  A \emph{quasi-projective variety} is an open subvariety of a projective variety.

  A \emph{quasi-affine variety} is an open subvariety of a projective variety.
\end{definition}

\begin{eg}
  \(\C^2 \setminus \{(0, 0)\} \subseteq \C^2\) is a quasi-affine variety.
\end{eg}

\begin{remark}
  If \(X\) is an affine variety, \(f \in k[X]\) and \(X\) is irreducible, \(k[X \setminus Z(f)] = k[X][\frac{1}{f}]\) so
  \[
    k(X \setminus Z(f)) = \operatorname{Frac} k[X \setminus Z(f)] = \operatorname{Frac} k[X] = k(X).
  \]
  Hence if \(X\) is an affine algebraic variety, we can define \(k(X)\) to be \(k(U)\) for \(U\) any open affine subvariety of \(X\), for example for \(U\) an open set in a chart defining \(X\).

  For example in \(\P^n\),
  \[
    k(U_0)
    = k(\frac{x_1}{x_0}, \dots, \frac{x_n}{x_0})
    = k(U_n)
    = k(\frac{x_0}{x_n}, \dots, \frac{x_{n - 1}}{x_n}).
  \]
\end{remark}

Projective space is compact

Let \(k = \C\). Claim
\[
  \P^n = (\C^{n + 1} \setminus \{0\}) / \C^\times = S^{2n + 1}/S^1.
\]

\begin{proof}
  Define
  \[
    S^{2n + 1} = \{x \in \C^{n + 1}: \norm x = 1\}
  \]
  where \(\norm x = (\sum |x_i|^2)^{1/2}\). Consider the map
  \begin{align*}
    \C^{n + 1} \setminus \{0\} &\to S^{2n + 1} \\
    x = (x_0, \dots, x_n) &\mapsto \frac{1}{\norm x} (x_0, \dots, x_n)
  \end{align*}
  \(|\lambda| = 1\), i.e.\ \(\lambda \in \C^*\) if and only if \(\norm{\lambda x} = \norm x\) so this descends to a map
  \[
    (\C^{n + 1} \setminus 0)/ \C^\times \to S^{2n + 1}/S^1.
  \]
  \(S^{2n + 1}\) is compact in the usual topology and so is its quotient. Thus \(\P^n\) is compact in the usual topology.
\end{proof}

Surprisingly, this has an algebraic version in the Zariski topology.

\begin{definition}[proper]\index{proper}
  \(X\) is \emph{proper} if for every continuous map \(\varphi: X \to Y\), the image of a closed subset under \(\varphi\) is closed.
\end{definition}

\begin{theorem}[fundamental theorem of elimination theory]
  For any field \(k\), \(\P^n\) is proper.
\end{theorem}

\begin{corollary}
  If \(X \subseteq \A^n\) is an affine variety and \(X\) is proper then \(X\) is a finite set of points.
\end{corollary}

\begin{proof}
  Suppose \(X\) is not a finite set of points. Then as \(X\) is affine there exists a non-constant element \(\varphi \in k[X]\), that is a morphism \(\varphi: X \to \A^1\) which is not constant. But \(X\) is proper so \(\im \varphi\) is closed and by assumption, \(\varphi(X)\) is not a finite set of points. Hence \(\varphi(X) = \A^1\). Define \(\tilde \varphi: X \to \P^1\) to be the obvious composition. The image of \(\tilde \varphi\) is \(\A^1\) which is not closed in \(\P^1\) so \(X\) is not proper. Contradiction.
\end{proof}

\section{Curves}

From now on suppose \(k = \cl k\).

\begin{definition}[curve]\index{curve}
  A \emph{curve} is a quasi-projective algebraic variety \(X\) such that \(\dim X = 1\).
\end{definition}

\begin{eg}
  If \(F \in k[X_0, X_1, X_2]\) is an irreducible homogeneous polynomial then \(Z(F) \subseteq \P^2\) is an irreducible plane projective curve.
\end{eg}

Warning: not all curves can be embedded in \(\P^2\).

\begin{ex}
  \(\dim X = 1\) means that for all \(p \in X \setminus \{\text{finite set}\}\), \(\dim T_pX = 1\), if and only if \(\dim_{\text{tr}} k(X) = 1\), if and only if any Zariski closed subvariety of \(X\) is \(X\) or a finite set of points.
\end{ex}

\begin{definition}
  Let \(X\) be an irreducible algebraic variety and \(p \in X\). Define the \emph{local ring} at \(p\) to be
  \[
    \mathcal O_{X, p} = \{\frac{f}{g} \in k(X): g(p) \neq 0\},
  \]
  rational functions defined on some neighbourhood of \(p\). Define
  \[
    \mathfrak m_{X, p} = \{\gamma \in \mathcal O_{X, p}: \gamma(p) = 0\},
  \]
  the maximal ideal of \(\mathcal O_{X, p}\).
\end{definition}

\begin{ex}\leavevmode
  \begin{enumerate}
  \item If \(\gamma \in \mathcal O_{X, p} \setminus \mathfrak m_{X, p}\) then \(\gamma^{-1} \in \mathcal O_{X, p}\).
  \item Show \(\mathfrak m_{X, p}\) is the unique maximal ideal of \(\mathcal O_{X, p}\).
  \end{enumerate}
\end{ex}

Suppose \(k = \C\). Let \(X\) be a curve, \(p \in X\) a smooth curve. Then a small open neighbourhood of \(p\) in the usual topology is diffeomorphic to a small open neighbourhood of \(0\) in \(\C\) by implicit funciton theorem. The corresponding notion is power series convergent on some neighbourhood of \(p\). It is completely analogous that here is an algebraic replacement for it.

\begin{theorem}
  Let \(X\) be a curve, \(p \in X\) a smooth point. Write \(\mathfrak m = \mathfrak m_{X, p}\).
  \begin{enumerate}
  \item \(\mathfrak m\) is a principal ideal in \(\mathcal O_{X, p}\).
  \item \(\bigcap_{n \geq 1} \mathfrak m^n = \{0\}\).
  \end{enumerate}
\end{theorem}

\begin{eg}
  Intuition: Consider \(\{x^2 + y^2 = 2\} \subseteq \A^1\). If \(p \neq (0, \pm 1)\) then \(y - y_0\) is a ``local coordinate'' at \(p\). If \(k = \C\), \(p \neq (0, \pm 1)\), then we can write \(x\) in terms of \(y\) as a convergence power series for \(|y - y_0| < \varepsilon\). For example at \((1, 0)\),
  \[
    x
    = (1 - y^2)^{1/2}
    = \sum_{n \geq 0} \binom{1/2}{n} (-1)^n y^{2n}
  \]
  so
  \[
    x - 1 = -\frac{1}{2}y^2 + \text{ high order terms}
  \]
  so \(x - 1\) vanishes to order \(2\) at the point. In the theorem,
  \[
    \mathfrak m_{X, p} = (y - y_0)
  \]
  if \(\frac{\partial f}{\partial x}(p) \neq 0\). Alternatively,
  \[
    x - 1
    = \frac{x^2 - 1}{x + 1}
    = -\frac{y^2}{x + 1}
  \]
  and \(\frac{1}{x + 1} \in \mathcal O_{X, p} \setminus \mathfrak m_{X, p}\).
\end{eg}

\begin{proof}
  By definition of \(X\) there exists an affine open neighbourhood \(X_0\) of \(p\), i.e.\ an open subset \(X_0 \subseteq X\) which is an affine variety. Write \(k[X_0] = k[x_1, \dots, x_n]/I\). wlog \(p \in X_0\) corresponds to the point \((0, \dots, 0)\). Let us write \(\overline x_i\) for the image of \(x_i\) in \(k[X_0]\). Then
  \begin{align*}
    \mathcal O_{X, p} &= \{\frac{f}{g}: f, g \in k[X_0], g \notin (\overline x_1, \dots, \overline x_n)\} \\
    \mathfrak m_{X, p} &= \{\frac{f}{g}: f \in (\overline x_1, \dots, \overline x_n), g \notin (\overline x_1, \dots, \overline x_n)\}
  \end{align*}
  \(X\) is a curve smooth at \(p\) so \(\dim T_pX_0 = 1\). Thus \(T_pX_0 \subseteq \A^n\) is a line, and by changing coordinates we can assume it is the line \(x_2 = x_3 = \dots = x_n = 0\). In other words, if \(\tilde f_2, \tilde f_3, \dots\) generate the ideal \(I\) then write
  \[
    \tilde f_i = \sum a_{ij} x_j + \text{ quadratic and higher term}.
  \]
  Note that the higher terms do not contribute to the tangent space at \(0\). Thus \(\dim T_0 X = 1\) implies that \(\dim \ker (a_{ij}) = 1\), so by row reduction can assume that
  \[
    \tilde f_i = \lambda_i x_i + \text{ high order terms}
  \]
  for \(i = 2, \dots, n\) and
  \[
    \tilde f_i =  \text{ quadratic and higher terms}
  \]
  for \(i > n\). So there exist \(\tilde f_2, \dots, \tilde f_n \in I\), \(\tilde f_i = x_i + h_i\) where \(h_i\) is at least quadratic.

  Thus in \(k[X_0]\), \(\overline x_j = -h_j\) and
  \[
    \overline x_j \in (\overline x_1^2, \overline x_1 \overline x_2, \dots) = \mathfrak m^2
  \]
  for \(j \geq 2\). Thus
  \[
    \mathfrak m
    = (\overline x_1, \dots, \overline x_n)
    = \overline x_1 \mathcal O_{X, p} + \dots + \overline x_n \mathcal O_{X, p}
    = \overline x_1 \mathcal O_{X, p} + \mathfrak m^2
  \]
  We want to conclude that \(\mathfrak m = (\overline x_1)\). Invoke Nakayama's lemma

  \begin{proposition}
    Let \(R\) be a ring, \(M\) a finitely generated \(R\)-module, \(J \subseteq R\) an ideal. Then
    \begin{enumerate}
    \item if \(JM = M\) then exists \(r \in J\) such that \((1 + r) M = 0\).
    \item if \(N \subseteq M\) is a submodule such that \(JM + N = M\) then there exists \(r \in J\) such that \((1 + r) M = N\).
    \end{enumerate}
  \end{proposition}

  Apply Nakayama to \(R = \mathcal O_{X, p}, J = \mathfrak m_{X, p}\) and note that \(1 + r \in \mathcal O_{X, p}^*\) if \(r \in \mathfrak m_{X, p}\), so
  \[
    (1 + r) M = M.
  \]
  We would like to apply Nakayama to \(M = \mathfrak m_{X, p}, N = (x_1)\), so need to show \(M\) is finitely generated. But every ideal \(J \subseteq \mathcal O_{X, p}\) is finitely generated,
  \[
    J = \{\frac{f}{g}: f \in J \cap k[X_0], g \in k[X_0], g(p) \neq 0\}
  \]
  so \(g \cdot \frac{f}{g} = f \in J \subseteq k[X_0]\), but \(J \cap k[X_0]\) is an ideal in \(k[X_0]\), hence finitely generated by Hilbert basis theorem. As
  \[
    \mathfrak m = (x_1) + \mathfrak m^2,
  \]
  Nakayama 2 says that \(\mathfrak m \subseteq (x_1)\). But \((x_1) \subseteq \mathfrak m\) so equality. In particular \(\mathfrak m\) is the principal ideal generated by \(x_1\).

  Now let \(M = \bigcap_{n \geq 1} \mathfrak m^n\), \(J = \mathfrak m \subseteq \mathcal O_{X, p}\) so a finitely generated ideal. But \(\mathfrak m M = M\) so by Nakayama 1 \(M = 1\).
\end{proof}

\begin{ex}
  Apply this to the circle.
\end{ex}

Let \(X = Z(f) \subseteq \A^2\) be a plane curve, \(p = (x_0, y_0) \in X\) a smooth point. Then \(x- x_0\) generate \(\mathfrak m_{X, p}\) if and only if \(\frac{\partial f}{\partial y}(x_0, y_0) \neq 0\), and a similar statement holds for \(y\). Thus if
\[
  \frac{\partial f}{\partial x}(p) = \frac{\partial f}{\partial y}(p) = 0
\]
then \(p\) is not a smooth point. Thus we can either write \(y\) in terms of \(x\) locally or vice versa near a smooth point.

Exercise: check this is immediate from the theorem and its proof.

\begin{definition}
  A function \(t \in \mathfrak m_{X, p}\) such that \(\mathfrak m_{X, p} = (t)\) is called a \emph{local parameter} or \emph{local coordinate} at \(p\).
\end{definition}

Such is not unique but if \(t\) is a local parameter so is \(ut\) if \(u \in \mathcal O_{X, p}^*\) and all other local parameters are of this form.

\begin{corollary}[order of vanishing/pole]\index{order of vanishing/pole}
  Every \(f \in k(X)^*\) can be written uniquely as
  \[
    f = t^n \cdot u
  \]
  where \(n \in \Z, u \in \mathcal O_{X, p}^*\). We call \(n = \nu_p(f)\) the \emph{order of vanishing/pole} of \(f\) at \(p\).

  \begin{align*}
    \mathcal O_{X, p} &= \{f \in k(X): \nu_p(f) \geq 0\} \cup \{0\} \\
    \mathfrak m_{X, p} &= \{f \in k(X): \nu_p(f) \geq 1\} \cup \{0\}
  \end{align*}
\end{corollary}
This is independent of the choice of \(t\).


























\iffalse

\section{Algebraic curves}

\begin{eg}
  \(E^0 = \{y^2 = (x - \lambda_1)(x - \lambda_2)(x - \lambda_3)\}, \lambda_1\lambda_2\lambda_3 \neq 0, \lambda_i \neq \lambda_j\) for \(i \neq j\). Let \(E = \cl{E^0} \subseteq \P^2\). The closure is
  \[
    y^2t = (x - \lambda_1z)(x - \lambda_2z)(x - \lambda_3z).
  \]
  Each point on the other chart when \(z = 0\). Unique extra point \(x = z = 0, y \neq 0\), \(P_\infty = [0:1:0]\). We will compute \(L(nP_\infty)\). To begin, let's compute \(\div(x), div(y)\).
  \begin{align*}
    x &= 0 \text{ when } y = \pm \sqrt{-\lambda_1\lambda_2\lambda_3} = \pm c \\
    x &= \infty \text{ at } P_\infty
  \end{align*}
  At \(y = \pm c, \frac{\p f}{\p y} (0, \pm c) \neq 0\) so \(x\) is a local parameter so
  \[
    \div(x) = [0: c: 1] + [0: -c: 1] + n_\infty P_\infty.
  \]
  Let \(\deg \div(x) = 0\), so \(n_\infty = -2\). Similarly,
  \[
    \div(y) = \sum_{i = 1}^3 [\lambda_i: 0: -1] - 3 P_\infty.
  \]
  Hence \(x \in L(2 P_\infty), y \in L(3 P_\infty)\).

  Example \(xy \in L(5P_\infty)\), Claim \(L(P_\infty) = k\). Granting claim, get \(\dim L(nP_\infty) \leq n\) for \(n \geq 1\) by proposition v. But
  \begin{align*}
    1, x &\in L(2P_\infty) \\
    1, x, y &\in L(3P_\infty) \\
    1, x, y, x^2 &\in L(4P_\infty) \\
    1, x, y, x^2, xy &\in L(5P_\infty)
  \end{align*}
  Claim that these are linearly independent, so \(\dim L(nP_\infty) = n\) for \(n \leq 5\): this is obvious in \(k(x) \supseteq L(nP_\infty)\).
  \[
    1, x, y, x^2, xy, x^3, y^2 \in L(6P_\infty)
  \]
  but these are not linearly independent by our general result, or more importantly, \(y^2 = (x - \lambda_1)(x - \lambda_2)(x - \lambda_3)\) is the linear relation between them.
  \begin{ex}\leavevmode
    \begin{enumerate}
    \item Show that \(xx^i, x^iy\) are linearly independent and hence
    \item \(\dim L(nP_\infty) = n\) for all \(n \geq 1\).
    \end{enumerate}
  \end{ex}

  Observe how this compares with \(\P^1\), when \(\dim L(nP) = n + 1\) for \(n \geq 0\).

  Proof of claim that \(L(P_\infty) = k\): if not, \(L(P_\infty) = k + kt\) for some \(t \in k(E)\), then
  \[
    t^n \in L(nP_\infty) \setminus L((n - 1)P_\infty)
  \]
  so \(1, t, \dots, t^n \in L(nP_\infty)\) and are linearly independent so must be a basis (as \(\dim L(nP_\infty) \leq n + 1\)). But \(x \in L(2P_\infty), y \in L(3P_\infty)\), so \(x = a't^2 + b't + c = (at + b)^2 + d\) by completing the square if \(\ch k \neq 2\). Replace \(t\) with \(at + b\), so \(x = t^2 + d\) and \(y = g_3(t)\) for some cubic polynomial in \(t\). But that
  \[
    g_3(t)^2 = \prod_{i = 1}^3 (t^2 - (\lambda_i - d)).
  \]
  Now \(\lambda_i\)'s are distinct, so \(\lambda_i - d\) are distinct, so RHS is not a square in \(k[t]\), absurd.
\end{eg}

Given \(D \in \div X\) where \(X\) is a smooth projective curve, \(\ell(D) = \dim L(D) \geq 1\), put \(m = \ell (D) - 1\). Choose a basis \(f_0, \dots, f_m\) of \(L(D)\). Get a rational map \(\alpha_D = [f_0: \dots: f_n], X \to \P^m = \P(L(D)^*)\). As \(X\) is a smooth projective curve, \(\alpha_D\) extends to a morphism \(\alpha_D: X \to \P^m\). Notice that if \(D' = D + \div(g)\), \(f_0g, \dots, f_mg\) is a basis of \(L(D')\) but
\[
  [f_0g: \dots: f_mg] = [f_0: \dots: f_m]
\]
so \(\alpha_D: X \to \P^m = \P(L(D)^*)\) equals to \(\alpha_{D'}: X \to \P^m = \P(L(D')^*)\), i.e.\ the map only depends on the class of \(D\) in \(\Cl(X)\).

\begin{eg}
  \(X = \P^1, D = n(\alpha), \ell(D) = n + 1\). Choose a basis \(1, t, \dots, t^n\) of \(L(n(\infty))\), then
  \[
    \alpha_D(t) = [1:t: \dots: t^n]: \P^1 \to \P^n.
  \]
  i.e.
  \[
    \alpha_D[x_0: x_1] = [1: \frac{x_1}{x_0}: \dots: \frac{x_1^n}{x_0^n}] = [x_0^n: x_0^{n - 1}x_1: \dots: x_1^n].
  \]
\end{eg}

\begin{definition}[Embedding]\index{embedding}
  An \emph{embedding} is a morphism \(\alpha: X \to Y\) with the property that \(X\) is isomorphic to \(\alpha(X)\) via \(\alpha\).
\end{definition}

\begin{ex}\leavevmode
  \begin{enumerate}
  \item Show that the above \(\alpha_{n(\infty)}: \P^1 \to \P^n\) is an embedding.
  \item Show that the morphism
    \begin{align*}
      alpha: \P^1 &\to \P^2 \\
      t &\mapsto [1: t^2: t^3]
    \end{align*}
    is injective but \emph{not} an embedding.
  \end{enumerate}
\end{ex}

\begin{ex}
  Let \(X = E\) be elliptical curve, \(D = nP_\infty\). Show
  \begin{align*}
    \alpha_{P_\infty}: E &\to \P^0 \\
    \alpha_{2P_\infty}: E &\to \P^1 \\
    (x, y) &\mapsto x \\
    \alpha_{3P_\infty}: E &\to P^2 \\
    (x, y) &\mapsto (x, y)
  \end{align*}

  basis of \(L(2P_\infty)\) \(1, x\), basis of \(L(3P_\infty)\) \(1, x, y\).
  
  Notice that \(\alpha_{3P_\infty}\)  is an embedding \(E \subseteq \R^2\) but \(\alpha_{2P_\infty}, \alpha_{P_\infty}\) are not.
\end{ex}

\begin{theorem}[Embedding criteria]
  \(\alpha_D: X \to \P^m\) is an embedding if and only if for all \(p, q \in X\), \(\ell(D - p - q) = \ell(D) - 2\). When this happens, \(X\) is a curve in \(\P^m\) of degree \(\deg D\).
\end{theorem}

We have not yet defined the degree of a curve \(X \subseteq \P^m\). Suppose \(X \subseteq \P V, \dim V = m + 1\), \(X\) smooth. Let \(H \subseteq \P V\) be a hyperplane and suppose \(X \nsubseteq H\). Then \(X \cap H\) is a closed subvariety of \(X\), so a finite set of points, and we define
\[
  [X \cap H] \in \div X
\]
to be these points ``counted with multiplicity''. Define \(\deg X\) ot be \(\deg [X \cap H]\) which will depend on \(X \subseteq \P^m\) but no other choices.

\begin{eg}
  Let \(H = \{x_0 = 0\}\) for some linear function \(x_0 \in V^*\). If \(p \in X \cap H\), want to define \(v_p(x_0)\). But \(x_0 \in k(X)\). Pick \(x_1 \in V^*\) wit \(x_1(p) \neq 0\). Now \(\frac{x_0}{x_1} \in k(X)\). Then \(v_p(\frac{x_0}{x_1}\) makes sense, and
  \[
    v_p(\frac{x_0}{x_1'}) = v_p(\frac{x_0}{x_1}) + v_p(\frac{x_1}{x_1'}).
  \]
  If \(x_1'\) is another function in \(V^*\) with \(x_1'(p) \neq 0\) but then the last term is \(0\). So define
  \[
    n_p = v_p(\frac{x_0}{x_1})
  \]
  which is independent of choice of \(x_1\).
\end{eg}

\begin{definition}
  Let \([H \cap X] = \sum_{p \in H \cap X} n_pP \in \div (X), [H \cap X] \geq 0\). Define
  \[
    \deg X = \deg [H \cap X].
  \]
\end{definition}

If we pick \(H' = \{x_0' = 0\}, X \nsubseteq H'\).
\[
  v_p(\frac{x_0}{x_1'}) = v_p(\frac{x_0}{x_1}) + v_p(\frac{x_1}{x_1'})
\]
so \([H \cap X] = [H' \cap X] + \div (\frac{x_0}{x_0'})\), i.e.\ \([H \cap X] = [H' \cap X]\) are equal in \(\Cl(X)\).

\begin{lemma}
  Let \(X = Z(F) \subseteq \P^2\) where \(F\) is a homogeneous polynomial of degree \(d\), then
  \[
    \deg Z(F) = \deg F.
  \]
\end{lemma}
\fi


\printindex
\end{document}