\documentclass[a4paper]{article}

\def\npart{II}

\def\ntitle{Number Fields}
\def\nlecturer{J.\ Thorne}

\def\nterm{Lent}
\def\nyear{2018}

\ifx \nauthor\undefined
  \def\nauthor{Qiangru Kuang}
\else
\fi

\ifx \ntitle\undefined
  \def\ntitle{Template}
\else
\fi

\ifx \nauthoremail\undefined
  \def\nauthoremail{qk206@cam.ac.uk}
\else
\fi

\ifx \ndate\undefined
  \def\ndate{\today}
\else
\fi

\title{\ntitle}
\author{\nauthor}
\date{\ndate}

%\usepackage{microtype}
\usepackage{mathtools}
\usepackage{amsthm}
\usepackage{stmaryrd}%symbols used so far: \mapsfrom
\usepackage{empheq}
\usepackage{amssymb}
\let\mathbbalt\mathbb
\let\pitchforkold\pitchfork
\usepackage{unicode-math}
\let\mathbb\mathbbalt%reset to original \mathbb
\let\pitchfork\pitchforkold

\usepackage{imakeidx}
\makeindex[intoc]

%to address the problem that Latin modern doesn't have unicode support for setminus
%https://tex.stackexchange.com/a/55205/26707
\AtBeginDocument{\renewcommand*{\setminus}{\mathbin{\backslash}}}
\AtBeginDocument{\renewcommand*{\models}{\vDash}}%for \vDash is same size as \vdash but orginal \models is larger
\AtBeginDocument{\let\Re\relax}
\AtBeginDocument{\let\Im\relax}
\AtBeginDocument{\DeclareMathOperator{\Re}{Re}}
\AtBeginDocument{\DeclareMathOperator{\Im}{Im}}
\AtBeginDocument{\let\div\relax}
\AtBeginDocument{\DeclareMathOperator{\div}{div}}

\usepackage{tikz}
\usetikzlibrary{automata,positioning}
\usepackage{pgfplots}
%some preset styles
\pgfplotsset{compat=1.15}
\pgfplotsset{centre/.append style={axis x line=middle, axis y line=middle, xlabel={$x$}, ylabel={$y$}, axis equal}}
\usepackage{tikz-cd}
\usepackage{graphicx}
\usepackage{newunicodechar}

\usepackage{fancyhdr}

\fancypagestyle{mypagestyle}{
    \fancyhf{}
    \lhead{\emph{\nouppercase{\leftmark}}}
    \rhead{}
    \cfoot{\thepage}
}
\pagestyle{mypagestyle}

\usepackage{titlesec}
\newcommand{\sectionbreak}{\clearpage} % clear page after each section
\usepackage[perpage]{footmisc}
\usepackage{blindtext}

%\reallywidehat
%https://tex.stackexchange.com/a/101136/26707
\usepackage{scalerel,stackengine}
\stackMath
\newcommand\reallywidehat[1]{%
\savestack{\tmpbox}{\stretchto{%
  \scaleto{%
    \scalerel*[\widthof{\ensuremath{#1}}]{\kern-.6pt\bigwedge\kern-.6pt}%
    {\rule[-\textheight/2]{1ex}{\textheight}}%WIDTH-LIMITED BIG WEDGE
  }{\textheight}% 
}{0.5ex}}%
\stackon[1pt]{#1}{\tmpbox}%
}

%\usepackage{braket}
\usepackage{thmtools}%restate theorem
\usepackage{hyperref}

% https://en.wikibooks.org/wiki/LaTeX/Hyperlinks
\hypersetup{
    %bookmarks=true,
    unicode=true,
    pdftitle={\ntitle},
    pdfauthor={\nauthor},
    pdfsubject={Mathematics},
    pdfcreator={\nauthor},
    pdfproducer={\nauthor},
    pdfkeywords={math maths \ntitle},
    colorlinks=true,
    linkcolor={red!50!black},
    citecolor={blue!50!black},
    urlcolor={blue!80!black}
}

\usepackage{cleveref}



% TODO: mdframed often gives bad breaks that cause empty lines. Would like to switch to tcolorbox.
% The current workaround is to set innerbottommargin=0pt.

%\usepackage[theorems]{tcolorbox}





\usepackage[framemethod=tikz]{mdframed}
\mdfdefinestyle{leftbar}{
  %nobreak=true, %dirty hack
  linewidth=1.5pt,
  linecolor=gray,
  hidealllines=true,
  leftline=true,
  leftmargin=0pt,
  innerleftmargin=5pt,
  innerrightmargin=10pt,
  innertopmargin=-5pt,
  % innerbottommargin=5pt, % original
  innerbottommargin=0pt, % temporary hack 
}
%\newmdtheoremenv[style=leftbar]{theorem}{Theorem}[section]
%\newmdtheoremenv[style=leftbar]{proposition}[theorem]{proposition}
%\newmdtheoremenv[style=leftbar]{lemma}[theorem]{Lemma}
%\newmdtheoremenv[style=leftbar]{corollary}[theorem]{corollary}

\newtheorem{theorem}{Theorem}[section]
\newtheorem{proposition}[theorem]{Proposition}
\newtheorem{lemma}[theorem]{Lemma}
\newtheorem{corollary}[theorem]{Corollary}
\newtheorem{axiom}[theorem]{Axiom}
\newtheorem*{axiom*}{Axiom}

\surroundwithmdframed[style=leftbar]{theorem}
\surroundwithmdframed[style=leftbar]{proposition}
\surroundwithmdframed[style=leftbar]{lemma}
\surroundwithmdframed[style=leftbar]{corollary}
\surroundwithmdframed[style=leftbar]{axiom}
\surroundwithmdframed[style=leftbar]{axiom*}

\theoremstyle{definition}

\newtheorem*{definition}{Definition}
\surroundwithmdframed[style=leftbar]{definition}

\newtheorem*{slogan}{Slogan}
\newtheorem*{eg}{Example}
\newtheorem*{ex}{Exercise}
\newtheorem*{remark}{Remark}
\newtheorem*{notation}{Notation}
\newtheorem*{convention}{Convention}
\newtheorem*{assumption}{Assumption}
\newtheorem*{question}{Question}
\newtheorem*{answer}{Answer}
\newtheorem*{note}{Note}
\newtheorem*{application}{Application}

%operator macros

%basic
\DeclareMathOperator{\lcm}{lcm}

%matrix
\DeclareMathOperator{\tr}{tr}
\DeclareMathOperator{\Tr}{Tr}
\DeclareMathOperator{\adj}{adj}

%algebra
\DeclareMathOperator{\Hom}{Hom}
\DeclareMathOperator{\End}{End}
\DeclareMathOperator{\id}{id}
\DeclareMathOperator{\im}{im}
\DeclareMathOperator{\coker}{coker}
\DeclarePairedDelimiter{\generation}{\langle}{\rangle}

%groups
\DeclareMathOperator{\sym}{Sym}
\DeclareMathOperator{\sgn}{sgn}
\DeclareMathOperator{\inn}{Inn}
\DeclareMathOperator{\aut}{Aut}
\DeclareMathOperator{\GL}{GL}
\DeclareMathOperator{\SL}{SL}
\DeclareMathOperator{\PGL}{PGL}
\DeclareMathOperator{\PSL}{PSL}
\DeclareMathOperator{\SU}{SU}
\DeclareMathOperator{\UU}{U}
\DeclareMathOperator{\SO}{SO}
\DeclareMathOperator{\OO}{O}
\DeclareMathOperator{\PSU}{PSU}
\DeclareMathOperator{\Sp}{Sp}


%hyperbolic
\DeclareMathOperator{\sech}{sech}

%field, galois heory
\DeclareMathOperator{\ch}{ch}
\DeclareMathOperator{\gal}{Gal}
\DeclareMathOperator{\emb}{Emb}



%ceiling and floor
%https://tex.stackexchange.com/a/118217/26707
\DeclarePairedDelimiter\ceil{\lceil}{\rceil}
\DeclarePairedDelimiter\floor{\lfloor}{\rfloor}


\DeclarePairedDelimiter{\innerproduct}{\langle}{\rangle}

%\DeclarePairedDelimiterX{\norm}[1]{\lVert}{\rVert}{#1}
\DeclarePairedDelimiter{\norm}{\lVert}{\rVert}



%Dirac notation
%TODO: rewrite for variable number of arguments
\DeclarePairedDelimiterX{\braket}[2]{\langle}{\rangle}{#1 \delimsize\vert #2}
\DeclarePairedDelimiterX{\braketthree}[3]{\langle}{\rangle}{#1 \delimsize\vert #2 \delimsize\vert #3}

\DeclarePairedDelimiter{\bra}{\langle}{\rvert}
\DeclarePairedDelimiter{\ket}{\lvert}{\rangle}




%macros

%general

%divide, not divide
\newcommand*{\divides}{\mid}
\newcommand*{\ndivides}{\nmid}
%vector, i.e. mathbf
%https://tex.stackexchange.com/a/45746/26707
\newcommand*{\V}[1]{{\ensuremath{\symbf{#1}}}}
%closure
\newcommand*{\cl}[1]{\overline{#1}}
%conjugate
\newcommand*{\conj}[1]{\overline{#1}}
%set complement
\newcommand*{\stcomp}[1]{\overline{#1}}
\newcommand*{\compose}{\circ}
\newcommand*{\nto}{\nrightarrow}
\newcommand*{\p}{\partial}
%embed
\newcommand*{\embed}{\hookrightarrow}
%surjection
\newcommand*{\surj}{\twoheadrightarrow}
%power set
\newcommand*{\powerset}{\mathcal{P}}

%matrix
\newcommand*{\matrixring}{\mathcal{M}}

%groups
\newcommand*{\normal}{\trianglelefteq}
%rings
\newcommand*{\ideal}{\trianglelefteq}

%fields
\renewcommand*{\C}{{\mathbb{C}}}
\newcommand*{\R}{{\mathbb{R}}}
\newcommand*{\Q}{{\mathbb{Q}}}
\newcommand*{\Z}{{\mathbb{Z}}}
\newcommand*{\N}{{\mathbb{N}}}
\newcommand*{\F}{{\mathbb{F}}}
%not really but I think this belongs here
\newcommand*{\A}{{\mathbb{A}}}

%asymptotic
\newcommand*{\bigO}{O}
\newcommand*{\smallo}{o}

%probability
\newcommand*{\prob}{\mathbb{P}}
\newcommand*{\E}{\mathbb{E}}

%vector calculus
\newcommand*{\gradient}{\V \nabla}
\newcommand*{\divergence}{\gradient \cdot}
\newcommand*{\curl}{\gradient \cdot}

%logic
\newcommand*{\yields}{\vdash}
\newcommand*{\nyields}{\nvdash}

%differential geometry
\renewcommand*{\H}{\mathbb{H}}
\newcommand*{\transversal}{\pitchfork}
\renewcommand{\d}{\mathrm{d}} % exterior derivative

%number theory
\newcommand*{\legendre}[2]{\genfrac{(}{)}{}{}{#1}{#2}}%Legendre symbol

%algebraic geometry
\DeclareMathOperator{\Spec}{Spec}
\DeclareMathOperator{\Proj}{Proj}

\makeindex

\begin{document}

\begin{titlepage}
  \begin{center}
    \includegraphics[width=0.6\textwidth]{logo.jpg}\par
    \vspace{1cm}
    {\scshape\huge Mathamatics Tripos \par}
    \vspace{2cm}
    {\huge Part \npart \par}
    \vspace{0.6cm}
    {\Huge \bfseries \ntitle \par}
    \vspace{1.2cm}
    {\Large\nterm, \nyear \par}
    \vspace{2cm}
    
    {\large \emph{Lectures by } \par}
    \vspace{0.2cm}
    {\Large \scshape \nlecturer}
    
    \vspace{0.5cm}
    {\large \emph{Notes by }\par}
    \vspace{0.2cm}
    {\Large \scshape \href{mailto:\nauthoremail}{\nauthor}}
 \end{center}
\end{titlepage}

\tableofcontents

\setcounter{section}{-1}

\section{Motivation}

Recall the following example from IB Groups, Rings and Modules:

\begin{theorem}
  Let \(p\) be an odd prime, then \(p = a^2 + b^2\) if and only if \(p = 1 \mod 4\).
\end{theorem}

\begin{proof}
  If \(p = a^2 + b^2\) then \(p = 0, 1 \text{ or } 2 \mod 4\) so this condition is necessary.

  Suppose instead \(p = 1 \mod 4\), then \(\binom{-1}{p} = 1\) so there exists \(a \in \Z\) such that \(a^2 = 1 \mod p\), or \(p \divides a^2 + 1\). We can factor \(a^2 + 1 = (a + i)(a - i) \in \Z[i]\). We know from IB Groups, Rings and Modules that \(\Z[i]\) is a UFD. As \(p \divides (a + i)(a - i)\), if \(p\) is irreducible in \(\Z[i]\)  then \(p \divides a + i\) or \(p \divides a - i\). Thus \(p \in \Z[i]\) is reducible so \(p = z_1z_2\) with \(z_1z_2 \in \Z[i]\). If \(z_1 = A + Bi\) where \(A, B \in \Z\) then \(A^2 + B^2 = p\).
\end{proof}

\begin{notation}
  If \(R \subseteq S\) are rings and \(\alpha \in S\) then
  \[
    R[\alpha] = \left\{ \sum{i = 0}^n a_i\alpha^i \in s: a_i \in R \right\}
  \]
  which is the smallest subring of \(S\) containing both \(R\) and \(\alpha\).
\end{notation}

Another example is given \(p\) an odd prime, does the equation
\[
  x^p + y^p = z^p
\]
have solutions such that \(x, y, z \in \Z, xyz \neq 0\)?

\begin{theorem}[Kummer, 1850]
  If \(\Z[e^{2\pi i/p}]\) is a UFD then there are no solutions.
\end{theorem}

The strategy is to factor
\[
  x^p + y^p = \prod_{j = 0}^{p - 1} (x + e^{2\pi ij/p}j) \in \Z[e^{2\pi i/p}].
\]
We now know that \(\Z[e^{2\pi i/p}]\) is a UFD if and only if \(p \leq 19\), so unfortunately this does not lead us very far. Instead, we have the more power theorem

\begin{theorem}[Kummer, 1850]
  If \(p\) is a \emph{regular} prime then there are no solutions.
\end{theorem}

We will define regular prime later in this course. This theorem is more powerful that the previous one. To give an idea, if \(p < 100\) then \(p\) is regular if and only if \(p \neq 37, 59, 67\).

Given a number field, this course studies the ring of integers. In the end of the course we will come back to Kummer's theorem.

\section{Rings of integers}

Recall that a field extension \(L/K\) is an inclusion \(K \subseteq L\) of fields. The degree of \(L/K\) is
\[
  [L:K] = \dim_K L.
\]
We say \(L/K\) is finite if \([L:K] < \infty\).

\begin{definition}[Number field]\index{number field}
  A \emph{number field} is a finite extension \(L/\Q\).
\end{definition}

Here are two ways to construct number fields:
\begin{enumerate}
\item Let \(\alpha \in \C\) be an algebraic number. Then \(L = \Q(\alpha)\) is a number field.
\item Let \(K\) be a number field \(K\) and \(f(x) \in K[x]\) be irreducible. Then \(L = K[x]/(f(x))\) is a number field. Recall Tower Law from IID Galois Theory:
  \[
    [L:\Q] = [L:K][K:\Q] < \infty.
  \]
\end{enumerate}
Note that the first one comes with an embedding in \(\C\), but the second one doesn't (and in general there are more than one).

\begin{definition}[Algebraic integer]\index{algebraic integer}\leavevmode
  \begin{enumerate}
  \item Let \(L/K\) be a field extension. We say \(\alpha \in L\) is \emph{algebraic} over \(K\) if there exists a monic polynomial \(f(x) \in K[x]\) such that \(f(\alpha) = 0\).
  \item Let \(L/\Q\) be a field extension. We say \(\alpha \in L\) is an \emph{algebraic integer} if there exists a monic polynomial \(f(x) \in \Z[x]\) such that \(f(\alpha) = 0\).
  \end{enumerate}
\end{definition}

\begin{definition}[Minimal polynomial]\index{minimal polynomial}
  Let \(L/K\) be a field extension and let \(\alpha \in L\) be an algebraic over \(K\). We call the \emph{minimal polynomial} of \(\alpha\) over \(K\) the monic polynomial \(f_\alpha(x) \in K[x]\) of the least degree such that \(f_\alpha(\alpha) = 0\).
\end{definition}

Note that \(f_\alpha(x)\) is well-defined: firstly there exists some monic \(f(x) \in K[x]\) such that \(f(\alpha) = 0\) since \(\alpha\) is algebraic. If \(f_\alpha(x), f_\alpha'(x) \in K[x]\) both satisfy the definition of minimal plynomial then we apply the polynomial division algorithm to write
\[
  f_\alpha(x) = q(x)f_\alpha'(x) + r(x)
\]
where \(p(x), r(x) \in K[x]\) and \(\deg r < \deg f_\alpha'\). Evaluate at \(\alpha\), we get
\[
  0 = f_\alpha(\alpha) = p(\alpha)f_\alpha'(\alpha) + r(\alpha) = r(\alpha)
\]
so by minimality of \(\deg f_\alpha'\), \(r = 0\). Then \(\deg f_\alpha = \deg f_\alpha'\) and they are both monic so \(p = 1\). \(f_\alpha = f_\alpha'\).

\begin{lemma}
  Let \(L/\Q\) be a field extension and let \(\alpha \in L\) to be an algebraic integer. Then
  \begin{enumerate}
  \item the minimal polynomial \(f_\alpha(x)\) of \(\alpha\) over \(\Q\) lies in \(\Z[x]\);
  \item if \(g(x) \in \Z[x]\) satisfies \(g(\alpha) = 0\) then there exists \(q(x) \in \Z[x]\) such that \(g(x) = f_\alpha(x)q(x)\);
  \item the kernel of the ring homomorphism
    \begin{align*}
      \Z[x] &\to L \\
      f(x) &\mapsto f(\alpha)
    \end{align*}
    equals to \((f_\alpha(x))\).
  \end{enumerate}
\end{lemma}

\begin{proof}\leavevmode
  \begin{enumerate}
  \item Recall from IB Groups, Rings and Modules that given \(f(x) = \sum_{i = 0}^n a_ix^i \in \Z[x]\), we define the content to be
    \[
      c(f) = \gcd(a_n, \dots, a_0).
    \]
    Gauss' Lemma says that if \(f(x), g(x) \in \Z[x]\) then \(c(fg) = c(f)c(g)\).

    Since \(\alpha \in L\) is an algebraic integer, there exists a monic \(f(x) \in \Z[x]\) such that \(f(\alpha) = 0\). Thus \(c(f) = 1\). Apply polynomial division in \(\Q[x]\) to get
    \[
      f(x) = p(x)f_\alpha(x) + r(x)
    \]
    where \(p(x), r(x) \in \Q[x]\). Same as before, we must have \(r(x) = 0\) so \(f(x) = p(x)f_\alpha(x)\). Now choose integers \(n, m \geq 1\) such that \(np(x) \in \Z[x], c(np) = 1\) and \(mf_\alpha(x) \in \Z[x], c(mf_\alpha) = 1\). Then
    \[
      nmf(x) = np(x) \cdot mf_\alpha(x).
    \]
    Take contents,
    \[
      nm = c(nmf(x)) = c(np \cdot mf_\alpha) = c(np)c(mf_\alpha) = 1.
    \]
    Thus \(n = m = 1\) so \(f_\alpha(x) \in \Z[x]\).
  \item This is similar to the previous one. Let \(g(x) \in \Z[x]\) be such that \(g(\alpha) = 0\). wlog \(g(x) \neq 0\) and \(c(g) = 1\). We deduce \(g(x) = q(x)f_\alpha(x)\) where \(q(x) \in \Q[x]\). Choose \(k \geq 1\) such that \(kq(x) \in \Z[x]\) and \(c(kq) = 1\). Then
    \[
      k = c(kg) = c(kq \cdot f_\alpha) = c(kq) c(f_\alpha) = 1
    \]
    so \(q(x) \in \Z[x]\).
  \item Reformulation of (2).
  \end{enumerate}
\end{proof}









\printindex


\iffalse
Reading

Marcus, Number Fields
secret notes: https://www.dpmms.cam.ac.uk/~jat58/nfl2018
\fi
\end{document}
