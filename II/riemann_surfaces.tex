\documentclass[a4paper]{article}

\def\npart{II}

\def\ntitle{Riemann Surfaces}
\def\nlecturer{H.\ Krieger}

\def\nterm{Michaelmas}
\def\nyear{2018}

\ifx \nauthor\undefined
  \def\nauthor{Qiangru Kuang}
\else
\fi

\ifx \ntitle\undefined
  \def\ntitle{Template}
\else
\fi

\ifx \nauthoremail\undefined
  \def\nauthoremail{qk206@cam.ac.uk}
\else
\fi

\ifx \ndate\undefined
  \def\ndate{\today}
\else
\fi

\title{\ntitle}
\author{\nauthor}
\date{\ndate}

%\usepackage{microtype}
\usepackage{mathtools}
\usepackage{amsthm}
\usepackage{stmaryrd}%symbols used so far: \mapsfrom
\usepackage{empheq}
\usepackage{amssymb}
\let\mathbbalt\mathbb
\let\pitchforkold\pitchfork
\usepackage{unicode-math}
\let\mathbb\mathbbalt%reset to original \mathbb
\let\pitchfork\pitchforkold

\usepackage{imakeidx}
\makeindex[intoc]

%to address the problem that Latin modern doesn't have unicode support for setminus
%https://tex.stackexchange.com/a/55205/26707
\AtBeginDocument{\renewcommand*{\setminus}{\mathbin{\backslash}}}
\AtBeginDocument{\renewcommand*{\models}{\vDash}}%for \vDash is same size as \vdash but orginal \models is larger
\AtBeginDocument{\let\Re\relax}
\AtBeginDocument{\let\Im\relax}
\AtBeginDocument{\DeclareMathOperator{\Re}{Re}}
\AtBeginDocument{\DeclareMathOperator{\Im}{Im}}
\AtBeginDocument{\let\div\relax}
\AtBeginDocument{\DeclareMathOperator{\div}{div}}

\usepackage{tikz}
\usetikzlibrary{automata,positioning}
\usepackage{pgfplots}
%some preset styles
\pgfplotsset{compat=1.15}
\pgfplotsset{centre/.append style={axis x line=middle, axis y line=middle, xlabel={$x$}, ylabel={$y$}, axis equal}}
\usepackage{tikz-cd}
\usepackage{graphicx}
\usepackage{newunicodechar}

\usepackage{fancyhdr}

\fancypagestyle{mypagestyle}{
    \fancyhf{}
    \lhead{\emph{\nouppercase{\leftmark}}}
    \rhead{}
    \cfoot{\thepage}
}
\pagestyle{mypagestyle}

\usepackage{titlesec}
\newcommand{\sectionbreak}{\clearpage} % clear page after each section
\usepackage[perpage]{footmisc}
\usepackage{blindtext}

%\reallywidehat
%https://tex.stackexchange.com/a/101136/26707
\usepackage{scalerel,stackengine}
\stackMath
\newcommand\reallywidehat[1]{%
\savestack{\tmpbox}{\stretchto{%
  \scaleto{%
    \scalerel*[\widthof{\ensuremath{#1}}]{\kern-.6pt\bigwedge\kern-.6pt}%
    {\rule[-\textheight/2]{1ex}{\textheight}}%WIDTH-LIMITED BIG WEDGE
  }{\textheight}% 
}{0.5ex}}%
\stackon[1pt]{#1}{\tmpbox}%
}

%\usepackage{braket}
\usepackage{thmtools}%restate theorem
\usepackage{hyperref}

% https://en.wikibooks.org/wiki/LaTeX/Hyperlinks
\hypersetup{
    %bookmarks=true,
    unicode=true,
    pdftitle={\ntitle},
    pdfauthor={\nauthor},
    pdfsubject={Mathematics},
    pdfcreator={\nauthor},
    pdfproducer={\nauthor},
    pdfkeywords={math maths \ntitle},
    colorlinks=true,
    linkcolor={red!50!black},
    citecolor={blue!50!black},
    urlcolor={blue!80!black}
}

\usepackage{cleveref}



% TODO: mdframed often gives bad breaks that cause empty lines. Would like to switch to tcolorbox.
% The current workaround is to set innerbottommargin=0pt.

%\usepackage[theorems]{tcolorbox}





\usepackage[framemethod=tikz]{mdframed}
\mdfdefinestyle{leftbar}{
  %nobreak=true, %dirty hack
  linewidth=1.5pt,
  linecolor=gray,
  hidealllines=true,
  leftline=true,
  leftmargin=0pt,
  innerleftmargin=5pt,
  innerrightmargin=10pt,
  innertopmargin=-5pt,
  % innerbottommargin=5pt, % original
  innerbottommargin=0pt, % temporary hack 
}
%\newmdtheoremenv[style=leftbar]{theorem}{Theorem}[section]
%\newmdtheoremenv[style=leftbar]{proposition}[theorem]{proposition}
%\newmdtheoremenv[style=leftbar]{lemma}[theorem]{Lemma}
%\newmdtheoremenv[style=leftbar]{corollary}[theorem]{corollary}

\newtheorem{theorem}{Theorem}[section]
\newtheorem{proposition}[theorem]{Proposition}
\newtheorem{lemma}[theorem]{Lemma}
\newtheorem{corollary}[theorem]{Corollary}
\newtheorem{axiom}[theorem]{Axiom}
\newtheorem*{axiom*}{Axiom}

\surroundwithmdframed[style=leftbar]{theorem}
\surroundwithmdframed[style=leftbar]{proposition}
\surroundwithmdframed[style=leftbar]{lemma}
\surroundwithmdframed[style=leftbar]{corollary}
\surroundwithmdframed[style=leftbar]{axiom}
\surroundwithmdframed[style=leftbar]{axiom*}

\theoremstyle{definition}

\newtheorem*{definition}{Definition}
\surroundwithmdframed[style=leftbar]{definition}

\newtheorem*{slogan}{Slogan}
\newtheorem*{eg}{Example}
\newtheorem*{ex}{Exercise}
\newtheorem*{remark}{Remark}
\newtheorem*{notation}{Notation}
\newtheorem*{convention}{Convention}
\newtheorem*{assumption}{Assumption}
\newtheorem*{question}{Question}
\newtheorem*{answer}{Answer}
\newtheorem*{note}{Note}
\newtheorem*{application}{Application}

%operator macros

%basic
\DeclareMathOperator{\lcm}{lcm}

%matrix
\DeclareMathOperator{\tr}{tr}
\DeclareMathOperator{\Tr}{Tr}
\DeclareMathOperator{\adj}{adj}

%algebra
\DeclareMathOperator{\Hom}{Hom}
\DeclareMathOperator{\End}{End}
\DeclareMathOperator{\id}{id}
\DeclareMathOperator{\im}{im}
\DeclareMathOperator{\coker}{coker}
\DeclarePairedDelimiter{\generation}{\langle}{\rangle}

%groups
\DeclareMathOperator{\sym}{Sym}
\DeclareMathOperator{\sgn}{sgn}
\DeclareMathOperator{\inn}{Inn}
\DeclareMathOperator{\aut}{Aut}
\DeclareMathOperator{\GL}{GL}
\DeclareMathOperator{\SL}{SL}
\DeclareMathOperator{\PGL}{PGL}
\DeclareMathOperator{\PSL}{PSL}
\DeclareMathOperator{\SU}{SU}
\DeclareMathOperator{\UU}{U}
\DeclareMathOperator{\SO}{SO}
\DeclareMathOperator{\OO}{O}
\DeclareMathOperator{\PSU}{PSU}
\DeclareMathOperator{\Sp}{Sp}


%hyperbolic
\DeclareMathOperator{\sech}{sech}

%field, galois heory
\DeclareMathOperator{\ch}{ch}
\DeclareMathOperator{\gal}{Gal}
\DeclareMathOperator{\emb}{Emb}



%ceiling and floor
%https://tex.stackexchange.com/a/118217/26707
\DeclarePairedDelimiter\ceil{\lceil}{\rceil}
\DeclarePairedDelimiter\floor{\lfloor}{\rfloor}


\DeclarePairedDelimiter{\innerproduct}{\langle}{\rangle}

%\DeclarePairedDelimiterX{\norm}[1]{\lVert}{\rVert}{#1}
\DeclarePairedDelimiter{\norm}{\lVert}{\rVert}



%Dirac notation
%TODO: rewrite for variable number of arguments
\DeclarePairedDelimiterX{\braket}[2]{\langle}{\rangle}{#1 \delimsize\vert #2}
\DeclarePairedDelimiterX{\braketthree}[3]{\langle}{\rangle}{#1 \delimsize\vert #2 \delimsize\vert #3}

\DeclarePairedDelimiter{\bra}{\langle}{\rvert}
\DeclarePairedDelimiter{\ket}{\lvert}{\rangle}




%macros

%general

%divide, not divide
\newcommand*{\divides}{\mid}
\newcommand*{\ndivides}{\nmid}
%vector, i.e. mathbf
%https://tex.stackexchange.com/a/45746/26707
\newcommand*{\V}[1]{{\ensuremath{\symbf{#1}}}}
%closure
\newcommand*{\cl}[1]{\overline{#1}}
%conjugate
\newcommand*{\conj}[1]{\overline{#1}}
%set complement
\newcommand*{\stcomp}[1]{\overline{#1}}
\newcommand*{\compose}{\circ}
\newcommand*{\nto}{\nrightarrow}
\newcommand*{\p}{\partial}
%embed
\newcommand*{\embed}{\hookrightarrow}
%surjection
\newcommand*{\surj}{\twoheadrightarrow}
%power set
\newcommand*{\powerset}{\mathcal{P}}

%matrix
\newcommand*{\matrixring}{\mathcal{M}}

%groups
\newcommand*{\normal}{\trianglelefteq}
%rings
\newcommand*{\ideal}{\trianglelefteq}

%fields
\renewcommand*{\C}{{\mathbb{C}}}
\newcommand*{\R}{{\mathbb{R}}}
\newcommand*{\Q}{{\mathbb{Q}}}
\newcommand*{\Z}{{\mathbb{Z}}}
\newcommand*{\N}{{\mathbb{N}}}
\newcommand*{\F}{{\mathbb{F}}}
%not really but I think this belongs here
\newcommand*{\A}{{\mathbb{A}}}

%asymptotic
\newcommand*{\bigO}{O}
\newcommand*{\smallo}{o}

%probability
\newcommand*{\prob}{\mathbb{P}}
\newcommand*{\E}{\mathbb{E}}

%vector calculus
\newcommand*{\gradient}{\V \nabla}
\newcommand*{\divergence}{\gradient \cdot}
\newcommand*{\curl}{\gradient \cdot}

%logic
\newcommand*{\yields}{\vdash}
\newcommand*{\nyields}{\nvdash}

%differential geometry
\renewcommand*{\H}{\mathbb{H}}
\newcommand*{\transversal}{\pitchfork}
\renewcommand{\d}{\mathrm{d}} % exterior derivative

%number theory
\newcommand*{\legendre}[2]{\genfrac{(}{)}{}{}{#1}{#2}}%Legendre symbol

%algebraic geometry
\DeclareMathOperator{\Spec}{Spec}
\DeclareMathOperator{\Proj}{Proj}

\begin{document}

\begin{titlepage}
  \begin{center}
    \includegraphics[width=0.6\textwidth]{logo.jpg}\par
    \vspace{1cm}
    {\scshape\huge Mathamatics Tripos \par}
    \vspace{2cm}
    {\huge Part \npart \par}
    \vspace{0.6cm}
    {\Huge \bfseries \ntitle \par}
    \vspace{1.2cm}
    {\Large\nterm, \nyear \par}
    \vspace{2cm}
    
    {\large \emph{Lectures by } \par}
    \vspace{0.2cm}
    {\Large \scshape \nlecturer}
    
    \vspace{0.5cm}
    {\large \emph{Notes by }\par}
    \vspace{0.2cm}
    {\Large \scshape \href{mailto:\nauthoremail}{\nauthor}}
 \end{center}
\end{titlepage}

\tableofcontents

\section{Complex analysis \& Branching/Multivalued functions}

\subsection{Holomorphicity}

\begin{definition}
  A smooth function \(f: U \to \C\) from a domain (i.e.\ an open connected subset of \(\C\)) is \emph{holomorphic} or \emph{analytic} if either of the following holds:
  \begin{enumerate}
  \item \(f\) is differentiable in the sense of limits (which is equivalent to satisfying the Cauchy-Riemann equations),
  \item for each \(a \in U\), \(f\) has a power series expansion
    \[
      f(z) = \sum_{n \geq 0} a_n (z - a)^n,
    \]
    valid on some disk \(D(a, r)\) with positive radius \(r > 0\).
  \end{enumerate}
\end{definition}

\begin{remark}
  1 implies 2 since \(f\) being differentiable allows us to construct \(a_n\) using Cauchy Integral Formula. 2 implies 1 since \(f\) having power series allows term-by-term differentiation.
\end{remark}

By 2, if \(a \in U\) and \(f\) is not identically \(0\) near \(a\), then there exists some minimal \(m \geq 0\) such that \(a_m \neq 0\). It follows that \(f(z) = a_m (z - a)^m (1 + g(z - a))\) where \(\lim_{z \to a} g(z - a) = 0\). Therefore for \(z\) sufficiently close to \(a\), \(f\) is nonzero. This is known as

\begin{theorem}[Principle of isolated zeros]
  An analytic function on a domain \(U\) which is not identically zero has isolated zeros, i.e.\ around each \(a \in U\), there exists a disk \(\Delta_a\) on which \(f(z) \neq 0\) unless possibly at \(z = a\).
\end{theorem}

If \(f\) is identically \(0\) near \(a\), then there exists a disk \(\Delta_a\) on which \(f(z) = 0\) for all \(z \in \Delta_a\). Consider \(V := \bigcup_{a: f|_{\Delta_a} = 0} \Delta_a\) and \(W := \bigcup_{a: f \neq 0 \text{ near } a} \Delta_a\). \(V\) and \(W\) are open and disjoint so by connectivity of \(U\), one of them is empty so \(f = 0\) on \(U\) or has isolated zeros. Thus having isolated zero is a property of a domain, not a local property.

\begin{corollary}
  If \(f\) and \(g\) are analytic on \(U\) then either \(f = g\) on \(U\) or \(f(z) = g(z)\) on a discrete set.
\end{corollary}

\begin{definition}
  If \(f\) is analytic on the punctured disk \(D(a, r)^* := D(a, r) \setminus \{a\}\) for some \(r > 0\), then \(f\) has an isolated singularity at \(a\).
\end{definition}

In this case, we obtain the analogue of power series, \emph{Laurent series} at \(a\)
\[
  f(z) = \sum_{n = -\infty}^{\infty} c_n (z - a)^n.
\]

There are three possibilities:
\begin{enumerate}
\item removable singularity: \(c_n = 0\) for all \(n < 0\).
\item pole: there exists \(N < 0\) such that \(c_N \neq 0\) and \(c_n = 0\) for all \(n < N\). We say \(f\) has a pole of order \(-N\) and can write \(f(z) = (z - a)^N g(z)\) where \(g\) is analytic and nonzero at \(a\).
\item essential singularity: \(c_n \neq 0\) for infinitely many \(n < 0\).
\end{enumerate}

However, characterisation in terms of Laurent series is coordinate-dependent. Intrinsically, recall that

\begin{theorem}
  \(f\) has a removable singularity at \(a\) if and only if \(f\) is bounded on \(D(a, r)^*\).
\end{theorem}

\begin{theorem}[Casorati-Weierstrass]
  \(f\) has an essential singularity at \(a\) if and only if for every punctured disk \(D(a, r)^*\) in the domain of \(f\), the image \(f(D(a, r)^*)\) is dense in \(\C\).
\end{theorem}

For completeness sake, we state that \(f\) has a pole at \(a\) if and only if neither of the above happens (so \(\lim_{z \to a} |f(z)| = \infty\)).

This allows us, for example, to extend the definitions to infinity. Consider the Riemann sphere \(C_\infty\), on which a neighbourhood of infinity is the complement of a closed set not including \(\infty\). Mapping it to the complex plane, we define a puncutre disk around \(\infty\) to be the complement of a closed disk in \(\C\). Then we can talk conveniently about singularity at \(\infty\).

\begin{eg}
  \(f(z) = \frac{1}{e^z - 1}\) is meromorphic on \(\C\) with poles at \(z = 2\pi n i\) where \(n \in \Z\). By considering \(g(z) = \frac{z}{e^z - 1}\) which has removable singularity at \(0\), we know \(f\) has ple of order \(1\) at \(0\), and therefore at all poles by periodicity.

  At \(\infty\), we have an essential singularity : along the imaginary axis, \(|f(z)|\) can be arbitrarily big so it cannot be a removable singularity. Along the positive real axis, \(|f(z)| \to 0\) so it cannot be a pole.
\end{eg}

\begin{definition}[meromorphic function]\index{meromorphic function}
  \(f\) is \emph{meromorphic} on a domain \(U \subseteq \C_\infty\)  if it has only isolated singularies, none of which are essential.
\end{definition}

\subsection{Complex logarithm}

Given nonzero \(z = r e^{i \theta}\), if \(e^w = z\), we know that \(w = \log r + (2\pi n + \theta) i\) for some \(n \in \Z\). We can make a continuous choice of \(\log z\) on, for example, \(U = \C \setminus \R_{\geq 0}\), by choosing \(0 < \theta < 2\pi\) and fixing some \(n \in \Z\). This makes \(f_n(z) := \log r + (2\pi n + \theta)i\) a well-defined continuous analytic function on \(U\).

\begin{note}\leavevmode
  \begin{enumerate}
  \item If \(g: U \to V\) is an analytic bijection, then any inverse \(h: V \to U\) is analytic.
  \item If \(g: U \to V\) is analytic, then any \emph{continuous} inverse \(h: V \to U\) is analytic.
  \end{enumerate}
\end{note}

More naturally,

\begin{proposition}
  Fix \(n \in \Z\) and define \(h(z) := \int_{-1}^z \frac{dw}{w} + (2n + 1)\pi i\) for \(z \in U\), where the integral is taken over the straight line from \(-1\) to \(z\), then \(h\) is analytic on \(U\) and inverse to \(z \mapsto e^z\).
\end{proposition}

\begin{proof}
  First show \(h\) is analytic with \(f'(z) = \frac{1}{z}\).
  \[
    \frac{h(z + \tau) - h(z)}{\tau}
    = \frac{1}{\tau} \int_z^{z + \tau} \frac{dw}{w}
  \]
  for \(\tau\) sufficiently small (such that the triangle formed by \(-1\), \(z\) and \(z + \tau\) lies in \(U\)) by Cauchy's Theorem. Then
  \[
    \left| \frac{1}{\tau} \int_z^{z + \tau} \frac{dw}{w} - \frac{1}{z} \right|
    = \left| \frac{1}{\tau} \int_z^{z + \tau} \frac{z - w}{zw} dw \right|
    \to 0
  \]
  as \(\tau \to 0\).

  Now define \(g(z) = \frac{e^{h(z)}}{z}\) so \(g'(z) = \frac{z e^{h(z)} h'(z) - e^{h(z)}}{z}\) and so \(g'(z) = 0\) identically. \(g(-1) = 1\) so \(e^{h(z)} = z\) for all \(z \in U\).

\end{proof}

\begin{definition}[direct analytic continuation]\index{analytic continuation!direct}
  A \emph{function element} in a domain \(U\) is a pair \((f, D)\) where \(D\) is a subdomain of \(U\) and \(f\) is an analytic function on \(D\). Two function elements \((f, D)\) and \((g, E)\) are equivalent, write \((f, D) \sim (g, E)\) if \(D \cap E \neq \emptyset\) and \(f = g\) on \(D \cap E\).

  We say \((g, E)\) is a \emph{direct analytic continuation} of \((f, D)\).
\end{definition}

Why do we make such a definition? We know the power series
\[
  \sum_{r \geq 0} z^k = \frac{1}{1 - z}
\]
is defined on \(D(0, 1)\) and cannot be extended to any larger domain due to natural boundary. However, \(\frac{1}{1 - z}\) is homomorphic  on \(\C \setminus \{1\}\) so sometimes the domain forced by definition of a function is not the maximal possible. In other words, sometimes we are looking at the ``correct'' function with a ``wrong'' domain.

\begin{definition}[analytic continuation along path]\index{analytic continuation!along path}
  We say \((g, E)\) is an \emph{analytic continuation of \((f, D)\) along \(\gamma\)} if \(\gamma: [0, 1] \to U\) and there exist function elements \((f_i, D_i)\), \(i \in \{0, \dots, n\}\) and \(0 = t_0 < t_2 < \dots < t_n = 1\) such that
  \[
    (f, D) = (f_0, D_0) \sim (f_1, D_1) \sim \dots \sim (f_{n - 1}, D_{n - 1}) \sim (f_n, D_n) = (g, E)
  \]
  and \(\gamma([t_j, t_{j + 1}]) \subseteq D_j\) for \(j \in \{0, \dots, n - 1\}\).

  Write \((f, D) \approx_\gamma (g, E)\).
\end{definition}

\begin{remark}
  As \(\C\) has a path-connected basis for the topology, domains are path-connected.
\end{remark}

\begin{definition}[analytic continuation]\index{analytic continuation}
  We say \((g, E)\) is an \emph{analytic continuation} of \((f, D)\) if there exists a path \(\gamma\) such that \((f, D) \approx_\gamma (g, E)\). In this case we write \((f, D) \approx (g, E)\).
\end{definition}

\begin{remark}\leavevmode
  \begin{enumerate}
  \item If \((f, D) \approx_\gamma (g, E)\) and \((f, D) \approx_\gamma (h, E)\) then \(g = h\) by repeated application of the identity principle. In other words, \(g\) is completely determined by \(f\) and \(\gamma\).
  \item Analytic continuation is an equivalence relation (exercise), but direct analytic continuation is \emph{not} transitive, even if the pairwise intersections of the domains are nonempty. If fact, that is the whole point of analytic continuation along path.
  \end{enumerate}
\end{remark}

\begin{definition}[complete analytic function]\index{complete analytic function}
  An equivalence class of function elements under \(\approx\) is a \emph{complete analytic function}.
\end{definition}

\begin{eg}[complex logarithm]
  Let \(U = \C\) be the ambient space. Given \(\alpha < \beta\) in \(\R\), define
  \[
    E_{(\alpha, \beta)} := \{z = r^{i \theta}: r > 0, \alpha < \theta < \beta\}.
  \]
  Note \(\C \setminus \R_{\geq 0} = E_{(0, 2\pi)}\). If \(\beta - \alpha \leq 2\pi\), define
  \[
    f_{(\alpha, \beta)}(z) = \log r + i\theta
  \]
  where \(z = re^{i\theta}, \alpha < \theta < \beta\). Then \((f_{(\alpha, \beta)}, E_{(\alpha, \beta)})\) is a function element for any such \(\alpha, \beta\).

  Let
  \begin{align*}
    A &= (-\frac{\pi}{2}, \frac{\pi}{2}) \\
    B &= (\frac{\pi}{6}, \frac{7\pi}{6}) \\
    C &= (\frac{5\pi}{6}, \frac{11\pi}{6})
  \end{align*}
  and \(\gamma: [0, 1] \to U, t \mapsto e^{2\pi i t}\) and choose
  \[
    0 = t_0 < t_1 = \frac{1}{6} < t_2 = \frac{1}{2} < t_3 = \frac{5}{6} < t_4 = 1
  \]
  and \((f_A, E_A), (f_B, E_B), (f_C, E_C)\) the corresponding function elements.

  When the \emph{intervals} overlap, the function elements agree so
  \[
    (f_A, E_A) \sim (f_B, E_B) \sim (f_C, E_C),
  \]
  but
  \[
    f_C(z) = f_A(z) + 2\pi i, z \in E_A \cap E_C
  \]
  which shows nontransitivity of \(\sim\). In fact, \(f_A + 2\pi i \sim f_C\). However we see \((f_A, E_A) \approx_\gamma (f_C, E_C)\) and so \((f_A, E_A) \approx (f_C, E_C)\). By repeating the process with intervals moving to infinity to \(\R\), we see that all the \(\log r + (2\pi n + \theta) i\) are in the same class for \(\approx\). On the other hand, if \((f, D) \approx_\gamma (f_{A'}, E_{A'})\) for some interval \(A'\) then applying identity principle along the path to \(e^{f_i}\) shows that \(f\) is one of the branches of \(\log\).

  Now we can define a space that contains all branches of logarithm. On \(U = \C \setminus \R_{\geq 0}\), define
  \[
    f_n(z) = \log n + (2\pi n + \theta)i
  \]
  where \(0 < \theta < 2\pi\). Then \((f_n, U)\) are function elements in the complete analytic function of \(\log\), and ``almost'' all of them. Take \(\Z\) copies of \(U\) and we can glue them along \(\R_{\geq 0}\). More precisely, for any \(n \in \Z\) and \(\alpha > 0\), there exists a neighbourhood \(V\) of \(\alpha\) and a function element \((g, V)\) such that
  \[
    (f_{n + 1}, E_{(0, \varepsilon)}) \sim (g, V) \sim (f_n, E_{(2\pi - \varepsilon, 2\pi)})
  \]
  for some \(\varepsilon > 0\).

  This object is the ``gluing construction'' of the Riemann surface associated to \(\log\). Since these \((g, V)\) exist, the resulting surface \(R\) will admit a \emph{continuous} function \(f\) such that the following diagram commutes:
  \[
    \begin{tikzcd}
      R \ar[r, "f"] \ar[d, "\pi"] & \C \ar[dl, "\exp"] \\
      \C^*
    \end{tikzcd}
  \]
\end{eg}


% https://www.dpmms.cam.ac.uk/~hk439/teaching.html

\printindex
\end{document}
