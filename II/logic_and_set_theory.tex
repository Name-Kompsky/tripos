\documentclass[a4paper]{article}

\def\npart{II}

\def\ntitle{Logic and Set Theory}
\def\nlecturer{I.\ B.\ Leader}

\def\nterm{Lent}
\def\nyear{2018}

\ifx \nauthor\undefined
  \def\nauthor{Qiangru Kuang}
\else
\fi

\ifx \ntitle\undefined
  \def\ntitle{Template}
\else
\fi

\ifx \nauthoremail\undefined
  \def\nauthoremail{qk206@cam.ac.uk}
\else
\fi

\ifx \ndate\undefined
  \def\ndate{\today}
\else
\fi

\title{\ntitle}
\author{\nauthor}
\date{\ndate}

%\usepackage{microtype}
\usepackage{mathtools}
\usepackage{amsthm}
\usepackage{stmaryrd}%symbols used so far: \mapsfrom
\usepackage{empheq}
\usepackage{amssymb}
\let\mathbbalt\mathbb
\let\pitchforkold\pitchfork
\usepackage{unicode-math}
\let\mathbb\mathbbalt%reset to original \mathbb
\let\pitchfork\pitchforkold

\usepackage{imakeidx}
\makeindex[intoc]

%to address the problem that Latin modern doesn't have unicode support for setminus
%https://tex.stackexchange.com/a/55205/26707
\AtBeginDocument{\renewcommand*{\setminus}{\mathbin{\backslash}}}
\AtBeginDocument{\renewcommand*{\models}{\vDash}}%for \vDash is same size as \vdash but orginal \models is larger
\AtBeginDocument{\let\Re\relax}
\AtBeginDocument{\let\Im\relax}
\AtBeginDocument{\DeclareMathOperator{\Re}{Re}}
\AtBeginDocument{\DeclareMathOperator{\Im}{Im}}
\AtBeginDocument{\let\div\relax}
\AtBeginDocument{\DeclareMathOperator{\div}{div}}

\usepackage{tikz}
\usetikzlibrary{automata,positioning}
\usepackage{pgfplots}
%some preset styles
\pgfplotsset{compat=1.15}
\pgfplotsset{centre/.append style={axis x line=middle, axis y line=middle, xlabel={$x$}, ylabel={$y$}, axis equal}}
\usepackage{tikz-cd}
\usepackage{graphicx}
\usepackage{newunicodechar}

\usepackage{fancyhdr}

\fancypagestyle{mypagestyle}{
    \fancyhf{}
    \lhead{\emph{\nouppercase{\leftmark}}}
    \rhead{}
    \cfoot{\thepage}
}
\pagestyle{mypagestyle}

\usepackage{titlesec}
\newcommand{\sectionbreak}{\clearpage} % clear page after each section
\usepackage[perpage]{footmisc}
\usepackage{blindtext}

%\reallywidehat
%https://tex.stackexchange.com/a/101136/26707
\usepackage{scalerel,stackengine}
\stackMath
\newcommand\reallywidehat[1]{%
\savestack{\tmpbox}{\stretchto{%
  \scaleto{%
    \scalerel*[\widthof{\ensuremath{#1}}]{\kern-.6pt\bigwedge\kern-.6pt}%
    {\rule[-\textheight/2]{1ex}{\textheight}}%WIDTH-LIMITED BIG WEDGE
  }{\textheight}% 
}{0.5ex}}%
\stackon[1pt]{#1}{\tmpbox}%
}

%\usepackage{braket}
\usepackage{thmtools}%restate theorem
\usepackage{hyperref}

% https://en.wikibooks.org/wiki/LaTeX/Hyperlinks
\hypersetup{
    %bookmarks=true,
    unicode=true,
    pdftitle={\ntitle},
    pdfauthor={\nauthor},
    pdfsubject={Mathematics},
    pdfcreator={\nauthor},
    pdfproducer={\nauthor},
    pdfkeywords={math maths \ntitle},
    colorlinks=true,
    linkcolor={red!50!black},
    citecolor={blue!50!black},
    urlcolor={blue!80!black}
}

\usepackage{cleveref}



% TODO: mdframed often gives bad breaks that cause empty lines. Would like to switch to tcolorbox.
% The current workaround is to set innerbottommargin=0pt.

%\usepackage[theorems]{tcolorbox}





\usepackage[framemethod=tikz]{mdframed}
\mdfdefinestyle{leftbar}{
  %nobreak=true, %dirty hack
  linewidth=1.5pt,
  linecolor=gray,
  hidealllines=true,
  leftline=true,
  leftmargin=0pt,
  innerleftmargin=5pt,
  innerrightmargin=10pt,
  innertopmargin=-5pt,
  % innerbottommargin=5pt, % original
  innerbottommargin=0pt, % temporary hack 
}
%\newmdtheoremenv[style=leftbar]{theorem}{Theorem}[section]
%\newmdtheoremenv[style=leftbar]{proposition}[theorem]{proposition}
%\newmdtheoremenv[style=leftbar]{lemma}[theorem]{Lemma}
%\newmdtheoremenv[style=leftbar]{corollary}[theorem]{corollary}

\newtheorem{theorem}{Theorem}[section]
\newtheorem{proposition}[theorem]{Proposition}
\newtheorem{lemma}[theorem]{Lemma}
\newtheorem{corollary}[theorem]{Corollary}
\newtheorem{axiom}[theorem]{Axiom}
\newtheorem*{axiom*}{Axiom}

\surroundwithmdframed[style=leftbar]{theorem}
\surroundwithmdframed[style=leftbar]{proposition}
\surroundwithmdframed[style=leftbar]{lemma}
\surroundwithmdframed[style=leftbar]{corollary}
\surroundwithmdframed[style=leftbar]{axiom}
\surroundwithmdframed[style=leftbar]{axiom*}

\theoremstyle{definition}

\newtheorem*{definition}{Definition}
\surroundwithmdframed[style=leftbar]{definition}

\newtheorem*{slogan}{Slogan}
\newtheorem*{eg}{Example}
\newtheorem*{ex}{Exercise}
\newtheorem*{remark}{Remark}
\newtheorem*{notation}{Notation}
\newtheorem*{convention}{Convention}
\newtheorem*{assumption}{Assumption}
\newtheorem*{question}{Question}
\newtheorem*{answer}{Answer}
\newtheorem*{note}{Note}
\newtheorem*{application}{Application}

%operator macros

%basic
\DeclareMathOperator{\lcm}{lcm}

%matrix
\DeclareMathOperator{\tr}{tr}
\DeclareMathOperator{\Tr}{Tr}
\DeclareMathOperator{\adj}{adj}

%algebra
\DeclareMathOperator{\Hom}{Hom}
\DeclareMathOperator{\End}{End}
\DeclareMathOperator{\id}{id}
\DeclareMathOperator{\im}{im}
\DeclareMathOperator{\coker}{coker}
\DeclarePairedDelimiter{\generation}{\langle}{\rangle}

%groups
\DeclareMathOperator{\sym}{Sym}
\DeclareMathOperator{\sgn}{sgn}
\DeclareMathOperator{\inn}{Inn}
\DeclareMathOperator{\aut}{Aut}
\DeclareMathOperator{\GL}{GL}
\DeclareMathOperator{\SL}{SL}
\DeclareMathOperator{\PGL}{PGL}
\DeclareMathOperator{\PSL}{PSL}
\DeclareMathOperator{\SU}{SU}
\DeclareMathOperator{\UU}{U}
\DeclareMathOperator{\SO}{SO}
\DeclareMathOperator{\OO}{O}
\DeclareMathOperator{\PSU}{PSU}
\DeclareMathOperator{\Sp}{Sp}


%hyperbolic
\DeclareMathOperator{\sech}{sech}

%field, galois heory
\DeclareMathOperator{\ch}{ch}
\DeclareMathOperator{\gal}{Gal}
\DeclareMathOperator{\emb}{Emb}



%ceiling and floor
%https://tex.stackexchange.com/a/118217/26707
\DeclarePairedDelimiter\ceil{\lceil}{\rceil}
\DeclarePairedDelimiter\floor{\lfloor}{\rfloor}


\DeclarePairedDelimiter{\innerproduct}{\langle}{\rangle}

%\DeclarePairedDelimiterX{\norm}[1]{\lVert}{\rVert}{#1}
\DeclarePairedDelimiter{\norm}{\lVert}{\rVert}



%Dirac notation
%TODO: rewrite for variable number of arguments
\DeclarePairedDelimiterX{\braket}[2]{\langle}{\rangle}{#1 \delimsize\vert #2}
\DeclarePairedDelimiterX{\braketthree}[3]{\langle}{\rangle}{#1 \delimsize\vert #2 \delimsize\vert #3}

\DeclarePairedDelimiter{\bra}{\langle}{\rvert}
\DeclarePairedDelimiter{\ket}{\lvert}{\rangle}




%macros

%general

%divide, not divide
\newcommand*{\divides}{\mid}
\newcommand*{\ndivides}{\nmid}
%vector, i.e. mathbf
%https://tex.stackexchange.com/a/45746/26707
\newcommand*{\V}[1]{{\ensuremath{\symbf{#1}}}}
%closure
\newcommand*{\cl}[1]{\overline{#1}}
%conjugate
\newcommand*{\conj}[1]{\overline{#1}}
%set complement
\newcommand*{\stcomp}[1]{\overline{#1}}
\newcommand*{\compose}{\circ}
\newcommand*{\nto}{\nrightarrow}
\newcommand*{\p}{\partial}
%embed
\newcommand*{\embed}{\hookrightarrow}
%surjection
\newcommand*{\surj}{\twoheadrightarrow}
%power set
\newcommand*{\powerset}{\mathcal{P}}

%matrix
\newcommand*{\matrixring}{\mathcal{M}}

%groups
\newcommand*{\normal}{\trianglelefteq}
%rings
\newcommand*{\ideal}{\trianglelefteq}

%fields
\renewcommand*{\C}{{\mathbb{C}}}
\newcommand*{\R}{{\mathbb{R}}}
\newcommand*{\Q}{{\mathbb{Q}}}
\newcommand*{\Z}{{\mathbb{Z}}}
\newcommand*{\N}{{\mathbb{N}}}
\newcommand*{\F}{{\mathbb{F}}}
%not really but I think this belongs here
\newcommand*{\A}{{\mathbb{A}}}

%asymptotic
\newcommand*{\bigO}{O}
\newcommand*{\smallo}{o}

%probability
\newcommand*{\prob}{\mathbb{P}}
\newcommand*{\E}{\mathbb{E}}

%vector calculus
\newcommand*{\gradient}{\V \nabla}
\newcommand*{\divergence}{\gradient \cdot}
\newcommand*{\curl}{\gradient \cdot}

%logic
\newcommand*{\yields}{\vdash}
\newcommand*{\nyields}{\nvdash}

%differential geometry
\renewcommand*{\H}{\mathbb{H}}
\newcommand*{\transversal}{\pitchfork}
\renewcommand{\d}{\mathrm{d}} % exterior derivative

%number theory
\newcommand*{\legendre}[2]{\genfrac{(}{)}{}{}{#1}{#2}}%Legendre symbol

%algebraic geometry
\DeclareMathOperator{\Spec}{Spec}
\DeclareMathOperator{\Proj}{Proj}

\makeindex

\begin{document}

\begin{titlepage}
  \begin{center}
    \includegraphics[width=0.6\textwidth]{logo.jpg}\par
    \vspace{1cm}
    {\scshape\huge Mathamatics Tripos \par}
    \vspace{2cm}
    {\huge Part \npart \par}
    \vspace{0.6cm}
    {\Huge \bfseries \ntitle \par}
    \vspace{1.2cm}
    {\Large\nterm, \nyear \par}
    \vspace{2cm}
    
    {\large \emph{Lectures by } \par}
    \vspace{0.2cm}
    {\Large \scshape \nlecturer}
    
    \vspace{0.5cm}
    {\large \emph{Notes by }\par}
    \vspace{0.2cm}
    {\Large \scshape \href{mailto:\nauthoremail}{\nauthor}}
 \end{center}
\end{titlepage}

\tableofcontents

\section{Propositional Logic}

Let \(P\) be a set of \emph{primitive propositions}. Unless otherwise stated,
\[
  P = \{p_1, p_2, \dots \}.
\]
The \emph{language} or \emph{set of propositions} \(L = L(P)\) is defined inductively by
\begin{enumerate}
\item \( p \in P, p \in L\),
\item \(\bot \in L\) (reads ``false''),
\item if \(p, q \in L\) then \((p \implies q) \in L\).
\end{enumerate}

\begin{eg}
  \((p_1 \implies \bot)\), \(((p_1 \implies p_2) \implies (p_1 \implies p_3))\), \(((p_1 \implies \bot) \implies \bot)\) are elements of \(L\).
\end{eg}

\begin{note}\leavevmode
  \begin{enumerate}
  \item Each proposition is a finite union string of symbols from the alphabet \((, ), \implies, p_1, p_2, \dots\).
  \item ``Inductively defined'' means more precisely that we set
    \begin{align*}
      L_1 &= P \cup \{\bot\} \\
      L_{n + 1} &= L_n \cup \{(p \implies q): p, q \in L_n\}
    \end{align*}
    and then set \(L = L_1 \cup L_2 \cup \dots\). \(L_n\) can be seen as ``things born by time \(n\)''.
  \item Each proposition is built up \emph{uniquely} from (1), (2) and (3). For example, \(((p_1 \implies p_2) \implies (p_1 \implies p_3))\) came from \((p_1 \implies p_2)\) and \((p_1 \implies p_3)\).
  \end{enumerate}
\end{note}

Note that we often omit outer brackets or use different brackets for clarity.

We can now define for example, \(\neg p\) (reads ``not \(p\)'') as an abbreviation for \(p \implies \bot\), \(p \lor q\) (reads ``\(p\) or \(q\)'') for \((\neg p) \implies q\), \(p \land q\) (reads \(p\) and \(q\)'') for \(\neg (p \implies (\neg q))\).

\subsection{Semantic Entailment}

\begin{definition}[Valuation]\index{valuation}
  A \emph{valuation} is a function \(v: L \to \{0, 1\}\) such that
  \begin{enumerate}
  \item \(v(\bot) = 0\),
  \item \(v(p \implies q) = \begin{cases} 0 & \text{if } v(p) = 1, v(q) = 0 \\ 1 & \text{otherwise} \end{cases}\) for all \(p, q \in L\).
  \end{enumerate}
\end{definition}

\begin{remark}
  On \(\{0, 1\}\), we could define a constant \(\bot\) by \(\bot = 0\) and an operation \(\implies\) by
  \[
    (a \implies b) =
    \begin{cases}
      0 & \text{if } a = 1, b = 0 \\
      1 & \text{otherwise}
    \end{cases}
  \]
  Then a valuation is a function \(L \to \{0, 1\}\) that preserves the structure (\(\bot\) and \(\implies\)), i.e.\ it is a homomorphism.
\end{remark}

\begin{proposition}\leavevmode
  \begin{enumerate}
  \item If \(v\) and \(v'\) are valuations with \(v(p) = v'(p)\) for all \( \in P\), then \(v = v'\).
  \item For any \(w: P \to \{0, 1\}\), there exists a valuation \(v\) with \(v(p) = w(p)\) for all \(p \in P\).
  \end{enumerate}
\end{proposition}

In other words, a valuation is determined by its values on \(P\) and any values will do.

\begin{proof}\leavevmode
  \begin{enumerate}
  \item We have for all \(p \in L_1\), \(v(p) = v'(p)\). But if \(v(p) = v'(p)\) and \(v(q) = v'(q)\) then \(v(p \implies q) = v'(p \implies q)\) so \(v = v'\) on \(L_2\). Continue inductively, we have \(v = v'\) on \(L_n\) for all \(n\).
  \item Set \(v(p) = w(p)\) for all \(p \in P\) and \(v(\bot) = 0\). This defines \(v\) on \(L_1\). Having defined \(v\) on \(L_2\), use \(v(p \implies q) = \begin{cases} 0 & \text{if } v(p) = 1, v(q) = 0 \\ 1 & \text{otherwise} \end{cases}\) to define \(v\) on \(L_{n + 1}\).
  \end{enumerate}
\end{proof}

\begin{eg}
  In a valuation given by
  \begin{align*}
    v(p_1) &= 1 \\
    v(p_2) &= 1 \\
    v(p_n) &= 0 \text{ for all } n \geq 3
  \end{align*}
  we have \(v(\underbrace{(p_1 \implies p_2)}_{1} \implies \underbrace{p_3}_{0}) = 0\).
\end{eg}

\begin{definition}[Tautology]\index{Tautology}
  \(p\) is a \emph{tautology}, written \(\models p\) if \(v(p) = 1\) for all valuations \(p\).
\end{definition}

\begin{eg}\leavevmode
  \begin{enumerate}
  \item \(p \implies (q \implies p)\). ``A true statement is implied by anything''. Alternatively, we could make a truth table
    \begin{table}[h]
      \centering
      \begin{tabular}{c|c|c|c}
        \(v(p)\) & \(v(q)\) & \(v(q \implies p)\) & \(v(p \implies (q \implies p))\) \\\hline
        1 & 1 & 1 & 1 \\
        1 & 0 & 1 & 1 \\
        0 & 1 & 0 & 1 \\
        0 & 0 & 1 & 1
      \end{tabular}
      \caption{Truth table}
    \end{table}
  \item \((\neg \neg p) \implies p\), i.e.\ \(((p \implies \bot) \implies \bot) \implies p\). ``Law of excluded middle''.
  \item \((p \implies (q \implies r)) \implies ((p \implies q) \implies (p \implies r))\). This is an example where writing down a truth table is not so desirable. Instead, this is not a tautology only if we have \(v\) with
    \begin{align*}
      v(p \implies (q\implies r)) &= 1 \\
      v((p \implies q) \implies (q \implies r)) &= 0
    \end{align*}
    so \(v(p \implies q) = 1, v(p \implies r) = 0\) whence \(v(p) = 1, v(r) = 0\), so also \(v(q) = 1\). But then \(v(q \implies r) = 0\) so \(v(p \implies (q \implies r)) = 0\). Absurd.
  \end{enumerate}
\end{eg}

\begin{definition}[Entailment]\index{entailment}
  For \(S \subseteq L, t \in L\), we say \(s\) \emph{entails} or \emph{semantically implies} \(t\), written \(S \models t\), if \(v(s) = 1\) for all \(s \in S\) then \(v(t) = 1\) for each valuation \(v\).
\end{definition}

This says whenever all of \(S\) is true, \(t\) is true as well.

\begin{eg}
  \(\{p \implies q, q \implies r\} \models (p \implies r)\). Indeed, suppose not. So have \(v\) with \(v(p \implies q) = v(q \implies r) = 1, v(p \implies r) = 0\). Then \(v(p) = 1, v(r) = 0\), whence \(v(q) = 0\) (from \(v(q \implies r) = 1\)), so \(v(p \implies q) = 0\). Absurd.
\end{eg}

\begin{definition}[Model]\index{model}
  If \(v(t) = 1\), we say \(t\) is \emph{true in \(v\)} or that \(v\) is a \emph{model} of \(t\).

  For \(S \subseteq L\), \(v\) is a \emph{model} of \(S\) if \(v(s) = 1\) for all \(s \in S\).
\end{definition}

Using this terminology, \(S \implies t\) says that every model of \(S\) is a model of \(t\).

\begin{note}
  \(\models t\) is equivalent to \(\emptyset \models t\).
\end{note}

\subsection{Syntactic Implication}

For a notion of ``proof'', we'll need axioms and deduction rules. As axioms, we'll take
\begin{enumerate}
\item \(p \implies (q \implies p)\) for all \(p, q \in L\).
\item \((p \implies (q \implies r)) \implies ((p \implies q) \implies (p \implies r))\) for all \(p, q, r \in L\).
\item \((\neg \neg p) \implies p\) for all \(p \in L\).
\end{enumerate}

\begin{note}
  We have already checked that these are all tautologies. Sometimes we say 3 axiom schemes to mean 3 infinite sets of axioms.
\end{note}

As deduction rules, we'll take just \emph{modus ponens}: from \(p\) and \((p \implies q)\) we can deduce \(q\).

\begin{definition}[Proof]\index{proof}
  For \(S \subseteq L\) and \(t \in L\), a \emph{proof} of \(t\) from \(S\) consists of a finite sequence \(t_1, \dots, t_n\) of propositions, with \(t_n = t\) such that for every \(i\), the proposition \(t_i\) is an axiom, or a member of \(S\), or there exists \(j, k < i\) with \(t_j = (t_k \implies t_i)\).

  We say \(S\) is the \emph{hypotheses} or \emph{premises} and \(t\) is the \emph{conclusion}.
\end{definition}

\begin{definition}
  If there is a proof of \(t\) from \(S\), say \(S\) \emph{proves} or \emph{syntactically implies} \(t\), written \(S \yields t\). If \(\emptyset \yields t\), say \(t\) is a \emph{theorem}, written \(\yields t\).
\end{definition}

Missed a lecture

\begin{theorem}[Model Existence Lemma]
  Let \(S \subseteq L\) be consistent, then \(S\) has a model.
\end{theorem}

The idea is to would like to define a valuation \(v\) by \(v(p) = 1\) if and only if \(p \in S\). As \(1\) is preserved under \(\implies\), a more sensible aim is \(v(p) = 1\) if and only if \(S \yields p\).

But maybe neither \(S \yields p\) nor \(S \yields \neg p\). So we want to ``grow'' \(S\) to contain one of \(p\) or \(\neg p\) for each \(p \in L\) (while remaining consistent).

\begin{proof}
  Claim that for any consistent \(S \subseteq L\), \(S \cup \{p\}\) or \(S \cup \{\neg p\}\) is consistent:

  \begin{proof}
    If not, then \(S \cup \{p\} \yields \bot\) and \(S \cup \{\neg p\} \yields \bot\). But then \(S \yields (p \implies \bot)\) by deduction theorem, i.e.\ \(S \yields \neg p\). Then \(S \yields \bot\). Absurd.
  \end{proof}

  Now as \(L\) is countable, we can list \(L\) as \(t_1, t_2, \dots \). Put \(S_0 = S\). Set \(S_1 = S_0 \cup \{t_1\}\) or \(S_0 \cup \{\neg t_1\}\) such that \(S_1\) is consistent. Then Let \(S_2 = S_1 \cup \{t_2\}\) or \(S_1 \cup \{\neg t_2\}\) such that \(S_2\) is consistent and continue inductively. Let \(\cl S = S_0 \cup S_1 \cup \dots\). Then \(\cl S \supseteq S\) and \(\cl S\) is consistent (as each \(S_n\) is consistent and proofs are finite). For all either \(p \in L\) we have \(p \in \cl S\) or \(\neg p \in \cl S\). Also \(\cl S\) is \emph{deductively closed}, meaning that if \(\cl S \yields p\) then \(p \in \cl S\). Indeed if \(p \notin \cl S\) then \(\neg p \in \cl S\), so \(\cl S \yields p, \cl S \yields (\neg p)\), whence \(\cl S \yields \bot\). Absurd.

  Define a valuation
  \begin{align*}
    v: L &\to \{0, 1\} \\
    p &\mapsto
        \begin{cases}
          1 & p \in \cl S \\
          0 & \text{otherwise}
        \end{cases}
  \end{align*}
  Indeed, \(v(\bot) = 0\) as \(\bot \notin \cl S\). For \(v(p \implies q)\):
  \begin{itemize}
  \item if \(v(p) = 1, v(q) = 0\), we have \(p \in \cl S, q \notin \cl S\), and we want \(v(p \implies q) = 0\), i.e.\ \((p \implies q) \notin \cl S\). But if \((p \implies q) \in \cl S\) then \(\cl S \yields q\), \(q \in \cl S\). Absurd.
  \item if \(v(q) = 1\), we have \(q \in \cl S\), and we want \(v(p \implies q) \in \cl S\), i.e.\ \((p \implies q) \in \cl S\). But \(\yields q \implies (p \implies q)\) so \(\cl S \yields (p \implies q)\).
  \item if \(v(p) = 0\), we have \(p \implies \cl S\). Then \((\neg p) \in \cl S\). We want \((p \implies q) \in \cl S\). Thus we need \((p \implies \bot) \yields (p \implies q)\), which by Deduction theorem is equivalent to \(\{p \implies \bot, p\} \yields q\). Thus suffices to show that \(\bot \yields q\). But we have \(\yields (\neg \neg q) \implies q\), and \(\yields (\bot \implies (\neg \neg q))\). Thus \(\yields (\bot \implies q)\), i.e.\ \(\bot \implies q\). Done.
  \end{itemize}
\end{proof}

\begin{remark}\leavevmode
  \begin{enumerate}
  \item Sometimes this is called Completeness Theorem.
  \item What would happen if \(P\) is uncountable? In fact, the result still holds if \(P\) is uncountable. See Chapter 3.
  \end{enumerate}
\end{remark}

By remark before the above theorem, we now have

\begin{corollary}[Adequacy]
  Let \(S \subseteq L, t \in L\). Then if \(S \models t\) then \(S \yields t\).
\end{corollary}

\begin{theorem}[Completeness Theorem]\index{Completeness Theorem}
  Let \(S \subseteq L, t \in L\). Then \(S \yields t\) if and only if \(S \models t\).
\end{theorem}

\begin{proof}
  By soundness and adequacy.
\end{proof}

Some consequences:

\begin{corollary}[Compactness Theorem]
  Let \(S \subseteq L, t \in L\) with \(S \models t\). Then there exists a finite \(S' \subseteq S\) with \(S' \models t\).
\end{corollary}

\begin{proof}
  Trivial if we replace \(\models\) with \(\yields\) as proofs are finite.
\end{proof}

Specialising to \(t = \bot\), this theorem says that if \(S\) has no model then some finite \(S' \subseteq S\) has no model. Equivalently,

\begin{corollary}[Compactness Theorem, equivalent form]
  Let \(S \subseteq L\). If every finite finite subset of \(S\) has a model then \(S\) has a model.
\end{corollary}

\begin{proof}
  This is equivalent to the previous corollary because \(S \models t\) if and only if \(S \cup \{\neg t\}\) has no model and \(S' \models t\) if and only if \(S' \cup \{\neg p\}\) has no model.
\end{proof}

\begin{corollary}[Decidability Theorem]
  There is an algorithm to determine (in finite time) whether or not, for a given \(S \subseteq L, t\in L\), we have \(S \yields t\).
\end{corollary}

\begin{remark}
  Highly non-obvious.
\end{remark}

\begin{proof}
  Trivial to decide if \(S \models t\), just by drawing a truth table.
\end{proof}

\section{Well-orderings and Ordinals}

\begin{definition}[Total order]\index{total order}
  A \emph{total order} or \emph{linear order} on a set \(X\) is a relation \(<\) on \(X\) that is
  \begin{enumerate}
  \item irreflexive: for all \(x\), not \(x < x\),
  \item transitive: for all \(x, y, z\), \(x < y, y < z\) implies \(x < z\),
  \item trichotomous: for all  \(x, y\), \(x < y, x = y\) or \(y < x\).
  \end{enumerate}
\end{definition}

\begin{note}
  Two of 3 cannot hold: if \(x < y, y < x\) then \(x < x\), absurd.
\end{note}

\begin{notation}
  We write \(x \leq y\) if \(x < y\) or \(x = y\). Write \(y > x\) if \(x < y\) etc.

  In terms of \(\leq\), a total order is
  \begin{enumerate}
  \item reflexive: for all \(x\), \(x \leq x\),
  \item transitive: for all \(x, y, z\), \(x \leq y, y \leq z\) implies \(x \leq z\),
  \item antisymmetric: for all \(x, y\), \(x \leq y, y \leq\) implies \(x = y\),
  \item trichotomous: for all \(x, y\), \(x \leq y\) or \(y \leq x\).
  \end{enumerate}
\end{notation}

\begin{eg}\leavevmode
  \begin{enumerate}
  \item \(\N\) with usual order\footnote{In this course \(0 \in \N\). Write \(\N^+\) for \(\N \setminus \{0\}\).}.
  \item \(\Q\) and \(\R\) with usual order.
  \item \(\N^+\) with divisibility is \emph{not a total order} as for example, \(2\) and \(3\) are not related.
  \item Given a set \(S\), the power set \(\mathcal P(S)\) with \(x \leq y\) if \(x \subseteq y\) is \emph{not} a total order for \(|S| > 1\).
  \end{enumerate}
\end{eg}

\begin{definition}[Well-ordering]\index{well-ordering}
  A total order is a \emph{well-ordering} if every non-empty subset has a least element: for all \(S \subseteq X\), if \(X \neq \emptyset\) then exists \(x \in S\) such that \(x \leq y\) for all \(y \in S\).
\end{definition}

\begin{eg}\leavevmode
  \begin{enumerate}
  \item \(\N\) with usual order.
  \item \(\Z\) with usual order is \emph{not} a well-ordering. Similar for \(\Q\) and \(\R\).
  \item \(\{x \in \Q: x \geq 0\}\) is \emph{not} a well-ordering. For example, \(\{x \in \Q: x > 0\}\) does not have a least element.
  \item \(\{1 - 1/n: n = 2, 3, \dots\}\) is a well-ordering. This can be thought of \(\N\) squashed into \([0, 1]\).
  \item \(\{1 - 1/n: n = 2, 3, \dots\} \cup \{1\}\) is a well-ordering.
  \item In fact, we can take the union of 5 with any real larger than \(1\) and still have a well-ordering.
  \item \(\{1 - 1/n: n = 2, 3, \dots\} \cup \{2 - 1/n: n = 2, 3, \dots\}\), i.e.\ two copies of 5, is still a well-ordering.
  \end{enumerate}
\end{eg}

\begin{remark}
  \(X\) is well-ordered if and only if there is no \(x_1 \geq x_2 \geq x_3 > \dots\) in \(X\). Indeed, if there is such a sequence then \(S = \{x_1, x_2, \dots\}\) has no least element.
\end{remark}

\begin{corollary}
  If \(S \subseteq X\) has no least element, then for each \(x \in S\) there exists \(x' \in S\) with \(x' < x\). Thus we have \(x > x' > x'' > \dots\).
\end{corollary}

\begin{definition}[Order isomorphism]\index{order isomorphism}
  Total orders \(X\) and \(Y\) are \emph{isomorphic} if there exists a bijection \(f: X \to Y\) that is order-preserving, i.e.\ for all \(x < x'\), \(f(x) < f(x')\).
\end{definition}

\begin{eg}
  1 and 6 above are isomorphic. 7 and 8 are isomorphic. 6 and 7 are not isomorphic: for example, one has a greatest element and the other one doesn't.
\end{eg}

\begin{proposition}[Proof by induction]\index{proof by induction}
  Let \(X\) be a well-ordering and \(S \subseteq X\) be such that if \(y \in S\) for all \(y < x\) then \(x \in S\) for each \(x \in X\), then \(S = X\).

  Equivalently, if \(p(x)\) is a property such that for all \(x\), if \(p(y)\) for all \(y < x\) then \(p(x)\), then \(p(x)\) for all \(x \in X\).
\end{proposition}

\begin{proof}
  If \(S \neq X\) then let \(x\) be the least element in \(X \setminus S\). Then \(x \notin X\). But \(y \in S\) for all \(y < x\). Absurd.
\end{proof}

An application:

\begin{proposition}
  Let \(X\) and \(Y\) be isomorphic well-orderings. Then there is a \emph{unique} isomorphism from \(X\) to \(Y\).
\end{proposition}

\begin{remark}
  This is false for total orders in general. For example, From \(\Z\) to \(\Z\) we could take identity or \(x \mapsto x - 5\).
\end{remark}

\begin{proof}
  Let \(f, g\) be isomorphisms. We will show \(f(x) = g(x)\) for all \(x \in X\) by induction. Thus we may assume \(f(y) = g(y)\) for all \(y < x\) and want \(f(x) = g(x)\).

  Let \(a\) be the least element of \(Y \setminus \{f(y): y \in X\}\), which is non-empty. Then we must have \(f(x) = a\): if \(f(x) > a\) then some \(x' > x\) has \(f(x') = a < f(x)\), contradicting \(f\) being order-preserving. Same holds for \(g\). Thus \(f(x) = g(x)\).
\end{proof}

\begin{definition}[initial segment]\index{initial segment}
  In a total order \(X\), an \emph{initial segment} \(I\) is a subset of \(X\) such that \(x \in I, y < x\) implies \(y \in I\).
\end{definition}

\begin{eg}\leavevmode
  \begin{enumerate}
  \item For any \(x \in X\), set \(I_x = \{y \in X: y < x\}\).
  \item Not every initial segment is of this form. For example, in \(\R\) take \(\{x: x \leq 3\}\), or in \(\Q\), take \(\{x: x^2 < 2 \text{ or } x < 0\}\).
  \end{enumerate}
\end{eg}

\begin{note}
  In a well-ordering, every proper initial segment \(I\) is of the form \(I_x\) for some \(x\). Indeed, let \(x\) be the least element of \(X \setminus I\). Then \(y < x\) implies \(y \in I\) (by definition of \(y\)). Also if \(y \in I\) then must have \(y < x\): if \(y = x\) or \(y > x\) then \(x \in I\) which is a contradiction.
\end{note}

The aim to show every subset of a well-ordered \(X\) is isomorphic to an initial segment.

\begin{note}
  This is false for total orders. For example, \(\{1, 5, 9\} \subseteq \Z\), or \(\Q \subseteq \R\).
\end{note}

Given \(Y \subseteq X\), intuitively we want to map the smallest element of \(Y\) to the smallest element of \(X\) and continue this way. But how do we show every element of \(Y\) is mapped to somewhere? Instead, we should work backwards: given  \(y \in Y\), map \(y\) to the smallest element in \(X\) that is not mapped to.

\begin{theorem}[Definition by recursion]\index{definition of recursion}
  Let \(X\) be well-ordered, \(Y\) any set and \(G: \mathcal P(X \times Y) \to Y\). Then there exists \(f: X \to Y\) such that \(f(x) = G(f|_{I_x})\) for all \(x \in X\). Moreover \(f\) is unique.
\end{theorem}

\begin{note}\leavevmode
  \begin{enumerate}
  \item For \(f: A \to B\) and \(C \subseteq A\), the \emph{restriction} of \(f\) to \(C\) is
    \[
      f|_C = \{(x, f(x)): x \in C\}.
    \]
  \item Slogan: to define \(f(x)\), make use of \(f|_{I_x}\), i.e.\ the values of \(f(y)\) for \(y < x\).
  \end{enumerate}
\end{note}

\begin{proof}
  First we show existence. Define ``\(h\) is an attempt'' to mean \(h: I \to Y\) where \(I \subseteq X\) is some initial segment, and for all \(x \in I\) we have \(h(x) = G(h|_{I_x})\). Note that if \(h\) and \(h'\) are both attempts defined at \(x\) then \(h(x) = h'(x)\): by induction on \(x\), since if \(h(y) = h'(y)\) for all \(y < x\), then \(h(x) = h'(x)\).

  Also for all \(x \in X\) there exists an attempt defined at \(X\), by induction on \(x\). Indeed we want an attempt defined at \(x\), given that for all \(y < x\) there exists an attempt defined at \(y\). So for each \(y < x\) we have a unique attempt \(h_y\) defined on \(\{z: z\leq y\}\) (unique by what we just showed). Let
  \[
    h = \bigcup_{y < x} h(y),
  \]
  an attempt defined on \(I_x\) (which is single-valued by uniqueness) so
  \[
    h' = h \cup \{(x, G(h))\}
  \]
  is an attempt defined at \(x\). Now set \(f(x) = y\) if there exists an attempt \(h\) defined at \(x\) with \(h(x) = y\) (also single-valued).

  For uniqueness, if \(f\) and \(f'\) are both suitable then \(f(x) = f'(x)\) for all \(x \in X\) (by induction on \(X\)) --- since if \(f(y) = f'(y)\) for all \(y < x\) then \(f(x) = f'(x)\).
\end{proof}

A typical application:

\begin{proposition}[Subset collapse]
  Let \(X\) be well-ordered, \(Y \subseteq X\). Then \(Y\) is isomorphic to an initial segment of \(X\). Moreover, the initial segment is unique.
\end{proposition}

\begin{proof}
  To have an isomorphism \(f: Y \to I \subseteq X\), we need precisely that for all \(x \in Y\), \(f(x) = \min X \setminus \{f(y): y < x\}\). So done (existence and uniqueness) by the previous theorem. Note that \(X \setminus \{f(y): y < x\} \neq \emptyset\), because \(f(y) \leq y\) for all \(y\) by induction so \(x \notin \{f(y): y < x\}\).
\end{proof}

A note to the pedantic: in proving the set \(X \setminus \{f(y): y < x\}\) is non-empty, we seem to use a circular argument by assuming \(f\) exists. But this is just a shorthand for the longer version: define
\[
  f(x) =
  \begin{cases}
    \min X \setminus \{f(y): y < x\} & \text{if } X \setminus \{f(y): y < x\} \neq \emptyset \\
    \text{cabbage} & \text{otherwise}
  \end{cases}
\]
and then proceed to show \(f(x) \neq \text{cabbage}\) for all \(x \in X\).

In particular, a well-ordered \(X\) cannot be isomorphic to a proper initial segment of \(X\), by uniqueness in subset collapse.

So far we have proved that if two well-orderings are isomorphic there is a unique isomorphism, and that a subset of a well-ordering is isomorphic to a (unique) initial segment. The question now is, how do different general well-orderings relate to each other?

\begin{definition}
  Say \(X \leq Y\) if \(X\) is isomorphic to an initial segment of \(Y\).
\end{definition}

\begin{eg}
  Let \(X = \N\), \(Y = \{1 - 1/n: n = 2, 3, \dots\} \cup \{1\}\), then \(X \leq Y\).
\end{eg}

What we would hope, is that there is a total order on the set of all well-orderings. And indeed

\begin{theorem}
  Let \(X, Y\) be well-orderings, then \(X \leq Y\) or \(Y \leq X\).
\end{theorem}

\begin{proof}
  Suppose \(Y \nleq X\), to obtain \(f: X \to Y\) that is an isomorphic with an initial segment of \(Y\), need for all \(x \in X\),
  \[
    f(x) = \min Y \setminus \{f(y): y < x\}.
  \]
  (we can't have \(Y = \{f(y): y < x\}\) as then \(Y\) is isomorphic to \(I_x\)). Done by the theorem.
\end{proof}

\begin{proposition}
  Let \(X, Y\) be well-orderings with \(X \leq Y\) and \(Y \leq X\) then \(X\) and \(Y\) are isomorphic.
\end{proposition}

\begin{proof}
  Let \(f\) be an isomorphism from \(X\) to an initial segment of \(Y\) and \(g\) from \(Y\) to \(X\). Then \(g \compose f: X \to X\) is an initial segment of \(X\) (as an initial segment of an initial segment is an initial segment). so \(g \compose f = \id\) by uniqueness in subset collapse. Similarly \(f \compose g = \id_Y\) . Thus \(X\) is isomorphic to \(Y\).
\end{proof}





\printindex

\iffalse
logic is the interplay of syntax and semantics
set: stuff with sets, universe of sets

Contents

1: Propositional logic
2: Well-ordering and ordinals
3: Posets and Zorn's Lemma
4: Predicate logic
5: Set theory
6: Cardinals

Reading:
Johnstone, Notes on logic and set theory
van Dalen, Logic and structure
Hainal & Hamburger, Set theory
Forster, Logic, induction and sets
\fi

\end{document}
