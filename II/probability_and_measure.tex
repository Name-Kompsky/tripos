\documentclass[a4paper]{article}

\def\npart{II}

\def\ntitle{Probability and Measure}
\def\nlecturer{E.\ Breuillard}

\def\nterm{Michaelmas}
\def\nyear{2018}

\input{header}

\begin{document}

\input{titlepage}

\tableofcontents

\section{Lebesgue measure}

\subsection{Boolean algebra}

\begin{definition}[Boolean algebra]\index{Boolean algebra}
  Let \(X\) be a set. A \emph{Boolean algebra} on \(X\) is a family of subsets of \(X\) which
  \begin{enumerate}
  \item contains \(\emptyset\),
  \item is stable under finite unions and complementation.
  \end{enumerate}
\end{definition}

\begin{eg}\leavevmode
  \begin{itemize}
  \item The \emph{trivial Boolean algebra} \(\mathcal B = \{\emptyset, X\}\).
  \item The \emph{discrete Boolean algebra} \(\mathcal B = 2^X\), the family of all subsets of \(X\).
  \item Less trivially, if \(X\) is a topological space, the family of \emph{constructible sets} forms a Boolean algebra, where a constructible set is the finite union of locally closed set, i.e.\ a set \(E = U \cap F\) where \(U\) is open and \(F\) is closed.
  \end{itemize}
\end{eg}

\begin{definition}[finitely additive measure]\index{measure!finitely additive}
  Let \(X\) be a set and \(\mathcal B\) a Boolean algebra on \(X\). A \emph{finitely additive measure} on \((X, \mathcal B)\) is a function \(m: \mathcal B \to [0, +\infty]\) such that
  \begin{enumerate}
  \item \(m(\emptyset) = 0\),
  \item \(m(E \cup F) = m(E) + m(F)\) where \(E \cap F = \emptyset\).
  \end{enumerate}
\end{definition}

\begin{eg}\leavevmode
  \begin{enumerate}
  \item Counting measure: \(m(E) = \#E\), the cardinality of \(E\) where \(\mathcal B\) is the discrete Boolean algebra of \(X\).
  \item More generally, given \(f: X \to [0, +\infty]\), define for \(E \subseteq X\),
    \[
      m(E) = \sum_{e \in E} f(e).
    \]
  \item Suppose \(X = \coprod_{i = 1}^N X_i\), then define \(\mathcal B(X)\) to be the unions of \(X_i\)'s. Assign a weight \(a_i \geq 0\) to each \(X_i\) and define \(m(E) = \sum_{i: X_i \subseteq E} a_i\) for \(E \in \mathcal B\).
  \end{enumerate}
\end{eg}

\subsection{Jordan measure}

This section is a historic review and provides intuition for Lebesgue measure theory. We'll gloss over details of proofs in this section.

\begin{definition}
  A subset of \(\R^d\) is called \emph{elementary} if it is a finite union of \emph{boxes}, where a box is a set \(B = I_1 \times \dots I_d\) where each \(I_i\) is a finite interval of \(\R\).
\end{definition}

\begin{proposition}
  Let \(B \subseteq \R^d\) be a box. Let \(\mathcal E(B)\) be the family of elementary subsets of \(B\). Then
  \begin{enumerate}
  \item \(\mathcal E(B)\) is a Boolean algebra on \(B\),
  \item every \(E \in \mathcal E(B)\) is a disjoint finite union of boxes,
  \item if \(E \in \mathcal E(B)\) can be written as disjoint finite union in two ways, \(E = \bigcup_{i = 1}^n B_i = \bigcup_{j = 1}^m B_j'\), then \(\sum_{i = 1}^n |B_i| = \sum_{j = 1}^m |B_j'|\) where \(|B| = \prod_{i = 1}^d |b_i - a_i|\) if \(B = I_1 \times \dots \times I_d\) and \(I_i\) has endpoints \(a_i, b_i\).
  \end{enumerate}
\end{proposition}

Following this, we can define a finitely additive measure correponding to our intuition of length, area, volume etc:

\begin{proposition}
  Define \(m(E) = \sum_{i = 1}^n |B_i|\) if \(E\) is any elementary set and is the disjoint union of boxes \(B_i \subseteq \R^d\). Then \(m\) is a finitely additive measure on \(\mathcal E(B)\) for any box \(B\).
\end{proposition}

\begin{definition}
  A subset \(E \subseteq \R^d\) is \emph{Jordan measurable} if for any \(\varepsilon > 0\) there are elementary sets \(A, B\), \(A \subseteq E \subseteq B\) and \(m(B \setminus A) < \varepsilon\).
\end{definition}

\begin{remark}
  Jordan measurable sets are bounded.
\end{remark}

\begin{proposition}
  If a set \(E \subseteq \R^d\) is Jordan measurable, then
  \[
    \sup_{A \subseteq E \text{ elementary}} \{m(A)\} = \inf_{B \supseteq E \text{ elementary}} \{m(B)\}.
  \]
  In which case we define the \emph{Jordan measure} of \(E\) as
  \[
    m(E) = \sup_{A \subseteq E} \{m(A)\}.
  \]
\end{proposition}

\begin{proof}
  Take \(A_n \subseteq E\) such that \(m(A_n) \nearrow \sup\) and \(B_n \supseteq E\) such that \(m(B_n) \searrow \inf\). Note that
  \[
    \inf \leq m(B_n) = m(A_n) + m(B_n \setminus A_n) \leq \sup + m(B_n \setminus A_n) \leq \sup + \varepsilon
  \]
  for arbitrary \(\varepsilon > 0\) so they are equal.
\end{proof}

\begin{ex}\leavevmode
  \begin{enumerate}
  \item If \(B\) is a box, the family \(\mathcal J(B)\) of Jordan measurable subsets of \(B\) is a Boolean algebra.
  \item A subset \(E \subseteq [0, 1]\) is Jordan measurable if and only if \(\mathbf 1_E\), the indicator funciton on \(E\), is Riemann integrable.
  \end{enumerate}
\end{ex}

\subsection{Lebesgue measure}

Although Jordan measure corresponds to the intuition of length, area and volume, it suffer from a few severe problems and issues:
\begin{enumerate}
\item unbounded sets in \(\R^d\) are not Jordan measurable.
\item \(\mathbf 1_{\Q \cap [0, 1]}\) is not Riemann integrable, and therefore \(\Q \cap [0, 1]\) is not Jordan measurable.
\item pointwise limits of Riemann integrable functions \(f_n := \mathbf 1_{\frac{1}{n!} \Z \cap [0, 1]} \to \mathbf 1_{\Q \cap [0, 1]}\) is not Riemann integrable.
\end{enumerate}

The idea of Lebesgue is to use countable covers by boxes.

\begin{definition}
  A subset \(E \subseteq \R^d\) is \emph{Lebesgue measurable} if for all \(\varepsilon > 0\), there exists a countable union of boxes \(C\) with \(E \subseteq C\) and \(m^*(C \setminus E) < \varepsilon\), where \(m^*\), the \emph{Lebesgue outer measure}, is defined as
  \[
    m^*(E) = \inf \{\sum_{i \geq 1} |B_i|: E \subseteq \bigcup_{i \geq 1} B_i, B_i \text{ boxes}\}
  \]
  for \emph{every} subset \(E \subseteq \R^d\).
\end{definition}

\begin{remark}
  wlog in these definitions we may assume that boxes are open.
\end{remark}

% We are going to show that the family of Lebesgue measurable subsets is not only a Boolean algebra, but also stable under countable union. Next we are going to define the Lebesgue measure on the family, with the additive property (which is not possessed by Lebesgue outer measure). In fact, we can show that we cannot define a measure for \emph{all} subsets of a set.

\begin{proposition}
  \label{prop:Lebesgue measurable subset is Boolean algebra}
  The family \(\mathcal L\) of Lebesgue measurable subsets of \(\R^d\) is a Boolean algebra stable under countable unions.
\end{proposition}

\begin{lemma}\leavevmode
  \begin{enumerate}
  \item \(m^*\) is monotone: if \(E \subseteq F\) then \(m^*(E) \subseteq m^*(F)\).
  \item \(m^*\) is countably subadditive: if \(E = \bigcup_{n \geq 1} E_n\) where \(E_n \subseteq \R^d\) then
    \[
      m^*(E) \leq \sum_{n \geq 1} m^*(E_n).
    \]
  \end{enumerate}
\end{lemma}

\begin{proof}
  Monotonicity is obvious. For countable subadditivity, pick \(\varepsilon > 0\) and let \(C_n = \bigcup_{i \geq 1} C_{n, i}\) where \(C_{n, i}\) are boxes such that \(E_n \subseteq C_n\) and
  \[
    \sum_{i \geq 1} |C_{n, i}| \leq m^*(E_n) + \frac{\varepsilon}{2^n}.
  \]
  Then
  \[
    \sum_{n \geq 1} \sum_{i \geq 1} |C_{n, i}|
    \leq \sum_{n \geq 1} (m^*(E_n) + \frac{\varepsilon}{2^n})
    = \varepsilon + \sum_{n \geq 1} m^*(E_n)
  \]
  and \(E \subseteq \bigcup_{n \geq 1} C_n = \bigcup_{n \geq 1} \bigcup_{i \geq 1} C_{n, i}\) so
  \[
    m^*(E) \leq \varepsilon + \sum_{n \geq 1} m^*(E_n)
  \]
  for all \(\varepsilon > 0\).
\end{proof}

\begin{remark}
  Note that \(m^*\) is \emph{not} additive on the family of all subsets of \(\R^d\). However, it will be on \(\mathcal L\), as we will show later.
  % eg?
\end{remark}

\begin{lemma}
  If \(A, B\) are disjoint compact subsets of \(\R^d\) then
  \[
    m^*(A \cup B) = m^*(A) + m^*(B).
  \]
\end{lemma}

\begin{proof}
  \(\leq\) by the previous lemma so need to show \(\geq\). Pick \(\varepsilon > 0\). Let \(A \cup B \subseteq \bigcup_{n \geq 1} B_n\) where \(B_n\) are open boxes such that
  \[
    \sum_{n \geq 1} |B_n| \leq m^*(A \cup B) + \varepsilon.
  \]
  wlog we may assume that the side lengths of each \(B_n\) are \(< \frac{\alpha}{2}\), where
  \[
    \alpha = \inf \{\norm{x - y}_1: x \in A, y \in B\} > 0.
  \]
  where the inequality comes from the fact that \(A\) and \(B\) are compact and thus closed.

  wlog we may discard the \(B_n\)'s that do not interesect \(A \cup B\). Then by construction
  \[
    \sum_{n \geq 1} |B_n| = \sum_{n \geq 1, B_n \cap A = \emptyset} |B_n| + \sum_{n \geq 1, B_n \cap B = \emptyset} |B_n|
    \geq m^*(A) + m^*(B)
  \]
  so
  \[
    \varepsilon + m^*(A \cup B) \geq m^*(A) + m^*(B)
  \]
  for all \(\varepsilon\).
\end{proof}

\begin{lemma}
  If \(E \subseteq \R^d\) has \(m^*(E) = 0\) then \(E \in \mathcal L\).
\end{lemma}

\begin{definition}[null set]\index{null set}
  A set \(E \subseteq \R^d\) such that \(m^*(E) = 0\) is called a \emph{null set}.
\end{definition}

\begin{proof}
  For all \(\varepsilon > 0\), there exist \(C = \bigcup_{n \geq 1} B_n\) where \(B_n\) are boxes such that \(E \subseteq C\) and \(\sum_{n \geq 1} |B_n| \leq \varepsilon\). But
  \[
    m^*(C \setminus E) \leq m^*(C) \leq \varepsilon.
  \]
\end{proof}

\begin{lemma}
  \label{lem:open and closed sets are Lebesgue measurable}
  Every open subset of \(\R^d\) and every closed subset of \(\R^d\) is in \(\mathcal L\).
\end{lemma}

We will prove the lemma using the fact that the family of Lebesgue measurable subsets is stable under countable union, which itself \emph{does not} use this lemma. This lemma, however, will be used to show the stability under complementation. Since the proof is quite technical (it has more to do with general topology than measure theory), for brevity and fluency of ideas we present the proof the main proposition first.

\begin{proof}[Proof of \Cref{prop:Lebesgue measurable subset is Boolean algebra}]
  It is obvious that \(\emptyset \in \mathcal L\). To show it is stable under countable unions, start with \(E_n \in \mathcal L\) for \(n \geq 1\). Need to show \(E := \bigcup_{n \geq 1} E_n \in \mathcal L\).

  Pick \(\varepsilon > 0\). By assumption there exist \(C_n = \bigcup_{i \geq 1} B_{n, i}\) where \(B_{n, i}\) are boxes such that \(E_n \subseteq C_n\) and
  \[
    m^*(C_n \setminus E_n) < \frac{\varepsilon}{2^n}.
  \]
  Now
  \[
    E = \bigcup_{n \geq 1} E_n \subseteq \bigcup_{n \geq 1} C_n =: C
  \]
  so \(C\) is again a countable union of boxes and \(C \setminus E \subseteq \bigcup_{n \geq 1} C_n \setminus E_n\).
  so
  \[
    m^*(C \setminus E) \leq \sum_{n \geq 1} m^*(C_n \setminus E_n) \leq \sum_{n \geq 1} \frac{\varepsilon}{2^n} = \varepsilon
  \]
  by countable subadditivity so \(E \in \mathcal L\).

  To show it is stable under complementation, suppose \(E \in \mathcal L\). By assumption there exist \(C_n\) a countable union of boxes with \(E \subseteq C_n\) and \(m^*(C_n \setminus E) \leq \frac{1}{n}\). wlog we may assume the boxes are open so \(C_n\) is open, \(C_n^c\) is closed so \(C_n^c \in \mathcal L\). Thus \(\bigcup_{n \geq 1} C_n^c \in \mathcal L\) by first part of the proof.

  But
  \[
    m^*(E^c \setminus \bigcup_{n \geq 1} C_n^c)
    \leq m^*(E^c \setminus C_n^c)
    = m^*(C_n \setminus E)
    \leq \frac{1}{n}
  \]
  so \(m^*(E^c \setminus \bigcup_{n \geq 1} C_n^c) = 0\) so \(E^c \setminus \bigcup_{n \geq 1} C_n^c \in \mathcal L\) since it is a null set. But
  \[
    E^c = (E^c \setminus \bigcup_{n \geq 1} C_n^c) \cup \bigcup_{n \geq 1} C_n^c,
  \]
  both of which are in \(\mathcal L\) so \(E^c \in \mathcal L\).
\end{proof}

\begin{proof}[Proof of \Cref{lem:open and closed sets are Lebesgue measurable}]
  Every open set in \(\R^d\) is a countable union of boxes so is in \(\mathcal L\). It is more subtle for closed sets. %(in fact it is the key difference between this and Jordan measure).
  The key observation is that every closed set is the countable union of compact subsets so we are left to show compact sets of \(\R^d\) are in \(\mathcal L\).

  Let \(F \subseteq \R^d\) be compact. For all \(k \geq 1\), there exist \(O_k\) a countable union of open sets such that \(F \subseteq O_k := \bigcup_{i \geq 1} O_{k, i}\) where \(O_{k, i}\) are open boxes such that
  \[
    \sum_{i \geq 1} |O_{k, i}| \leq m^*(F) + \frac{1}{2^k}.
  \]
  By compactness there exist a finite subcover so we can assume \(O_k\) is a finite union of open boxes. Moreover, wlog assume that
  \begin{enumerate}
  \item the side lengths of \(O_{k, i}\) are \(\leq \frac{1}{2^k}\).
  \item for each \(i\), \(O_{k, i}\) intersects \(F\).
  \item \(O_{k + 1} \subseteq O_k\) (by replacing \(O_{k + 1}\) with \(O_{k + 1} \cap O_k\) iteratively).
  \end{enumerate}
  Then \(F = \bigcap_{k \geq 1} O_k\) and we are left to show \(m^*(O_k \setminus F) \to 0\). By additivity on disjoint compact sets,
  \[
    m^*(F) + m^*(\cl O_i \setminus O_{i + 1}) = m^*(F \cup (\cl O_i \setminus O_{i + 1}))
  \]
  so
  \[
    m^*(F) + m^*(\cl O_i \setminus O_{i + 1}) \leq m^*(\cl O_i)
    \leq \sum_{j \geq 1} |O_{i, j}|
    \leq m^*(F) + \frac{1}{2^i}
  \]
  so \(m^*(\cl O_i \setminus O_{i + 1}) \leq \frac{1}{2^i}\). Finally,
  \[
    m^*(O_k \setminus F)
    = m^*(\bigcup_{i \geq k} (O_i \setminus O_{i + 1}))
    \leq \sum_{i \geq k} m^*(O_i \setminus O_{i + 1})
    \leq \sum_{i \geq k} \frac{1}{2^i}
    = \frac{1}{2^{k - 1}}.
  \]
\end{proof}

The result we're working towards is

\begin{proposition}
  \label{prop:additivity of Lebesgue measure}
  \(m^*\) is countably additive on \(\mathcal L\), i.e.\ if \((E_n)_{n \geq 1}\) where \(E_n \in \mathcal L\) are pairwise disjoint then
  \[
    m^*(\bigcup_{n \geq 1} E_n) = \sum_{n \geq 1} m^*(E_n).
  \]
\end{proposition}

\begin{lemma}
  If \(E \in \mathcal L\) then for all \(\varepsilon > 0\) there exists \(U\) open, \(F\) closed, \(F \subseteq E \subseteq U\) such that \(m^*(U \setminus E) < \varepsilon\) and \(m^*(E \setminus F) < \varepsilon\).
\end{lemma}

\begin{proof}
  By definition of \(\mathcal L\), there exists a countable union of open boxes \(E \subseteq \bigcup_{n \geq 1} B_n\) such that \(m^*(\bigcup_{n \geq 1} B_n \setminus E) < \varepsilon\). Just take \(U = \bigcup_{n \geq 1} B_n\) which is open.

  For \(F\) do the same with \(E^c = \R^d \setminus E\) in place of \(E\).
\end{proof}

\begin{proof}[Proof of \Cref{prop:additivity of Lebesgue measure}]
  First we assume each \(E_n\) is compact. By a previous lemma \(m^*\) is additive on compact sets so for all \(N \in \N\),
  \[
    m^*(\bigcup_{n = 1}^N E_n) = \sum_{n = 1}^N m^*(E_n).
  \]
  In particular
  \[
    \sum_{n = 1}^N m^*(E_n) \leq m^*(\bigcup_{n \geq 1} E_n)
  \]
  since \(m^*\) is monotone. Take \(N \to \infty\) to get one inequality. The other direction holds by countable subadditivity of \(m^*\).

  Now assume that each \(E_n\) is a bounded subset in \(\mathcal L\). By the lemma there exists \(K_n \subseteq E_n\) closed, so compact, such that \(m^*(E_n \setminus K_n) \leq \frac{\varepsilon}{2^n}\). Since \(K_n\)'s are disjoint, by the previous case
  \[
    m^*(\bigcup_{n \geq 1} K_n) = \sum_{n \geq 1} m^*(K_n)
  \]
  then
  \begin{align*}
    &\sum_{n \geq 1} m^*(E_n) \\
    \leq& \sum_{n \geq 1} m^*(K_n) + m^*(E_n \setminus K_n) \\
    \leq& m^*(\bigcup_{n \geq 1} K_n) + \sum_{n \geq 1} \frac{\varepsilon}{2^n} \\
    \leq& m^*(\bigcup_{n \geq 1} E_n) + \varepsilon
  \end{align*}
  so one direction of inequality. Similarly the other direction holds by countable subadditivity of \(m^^*\).

  For the general case, note that \(\R^d = \bigcup_{n \in \Z^d} A_n\) where \(A_n\) is bounded and in \(\mathcal L\), for example by taking \(A_n\) to be product of half open intervals of unit length. Write \(E_n\) as \(\bigcup_{m \in Z^d} E_n \cap A_m\) so just apply the previous results to \((E_n \cap A_m)_{n\geq 1, m \in \Z^d}\).
\end{proof}

\begin{definition}[Lebesgue measure]\index{Lebesgue measure}
  \(m^*\) when restricted to \(\mathcal L\) is called the \emph{Lebesgue measure} and is simply denoted by \(m\).
\end{definition}

\begin{eg}[Vitali counterexample]
  Althought \(\mathcal L\) is pretty big (it includes all open and closed sets, countable unions and intersections of them, and has cardinality at least \(2^{2^{\aleph_0}}\) by considering an uncountable collection of uncountable compact null set, and each subset thereof), it does not include every subset of \(\R^d\).
  
  Consider \((\Q, +)\), the additive subgroup of \((\R, +)\). Pick a set of representative \(E\) of the cosets of \((\Q, +)\). Choose it inside \([0, 1]\). For each \(x \in \R\), there exists a unique \(e \in E\) such that \(x - e \in \Q\) (here we require axiom of choice). Claim that \(E \notin \mathcal L\) and \(m^*\) is not additive on the family of all subsets of \(\R^d\).

  \begin{proof}
    Pick distinct rationals \(p_1, \dots, p_N\) in \([0, 1]\). The sets \(p_i + E\) are pairwise disjoint so if \(m^*\) were additive then we would have
    \[
      m^*(\bigcup_{i = 1}^N p_i + E)
      = \sum_{i = 1}^N m^*(p_i + E)
      = N \sum_{i = 1}^N m^*(E)
    \]
    by translation invariance of \(m^*\). But then
    \[
      \bigcup_{i = 1}^N p_i + E \subseteq [0, 2]
    \]
    since \(E \subseteq [0, 1]\) so by monotonicity of \(m^*\) have
    \[
      m^*(\bigcup_{i = 1}^N p_i + E) \leq 2
    \]
    so for all \(N m^*(E) \leq 2\) so \(m^*(E) = 0\). But
    \[
      [0, 1] \subseteq \bigcup_{q \in \Q} E + q = \R,
    \]
    by countable subadditivity of \(m^*\),
    \[
      1 = m^*([0, 1]) \leq \sum_{q \in \Q} m^*(E + q) = 0.
    \]
    Absurd.

    In particular \(E \notin \mathcal L\) as \(m^*\) is additive on \(\mathcal L\).
  \end{proof}
\end{eg}

\section{Abstract measure theory}

In this chapter we extend measure theory to arbitrary set. Most part of the theory is developed by Fréchet and Carathéodory.

\begin{definition}[\(\sigma\)-algebra]\index{\(\sigma\)-algebra}
  A \emph{\(\sigma\)-algebra} on a set \(X\) is a Boolean algebra stable under countable unions.
\end{definition}

\begin{definition}[measurable space]\index{measurable space}
  A \emph{measurable space} is a couple \((X, \mathcal A)\) where \(X\) is a set and \(\mathcal A\) is a \(\sigma\)-algebra on \(X\).
\end{definition}

\begin{definition}[measure]\index{measure}
  A \emph{measure} on \((X, \mathcal A)\) is a map \(\mu: \mathcal A \to [0, \infty]\) such that
  \begin{enumerate}
  \item \(\mu(\emptyset) = 0\),
  \item \(\mu\) is countably additive (also known as \(\sigma\)-additive), i.e.\ for every family \((A_n)_{n \geq 1}\) of disjoint subsets in \(\mathcal A\), have
    \[
      \mu (\bigcup_{n \geq 1} A_n) = \sum_{n \geq 1} \mu(A_n).
    \]
  \end{enumerate}

  The triple \((X, \mathcal A, \mu)\) is called a measure space.
\end{definition}

\begin{eg}\leavevmode
  \begin{enumerate}
  \item \((\R^d, \mathcal L, m)\) is a measure space.
  \item \((X, 2^X, \#)\) where \(\#\) is the counting measure.
  \end{enumerate}
\end{eg}

\begin{proposition}
  Let \((X, \mathcal A, \mu)\) be a measure space. Then
  \begin{enumerate}
  \item \(\mu\) is monotone: \(A \subseteq B\) implies \(\mu(A) \subseteq \mu(B)\),
  \item \(\mu\) is countably subadditive: \(\mu (\bigcup_{n \geq 1} A_n) \leq \sum_{n \geq 1} \mu(A_n)\),
  \item upward monotone convergence: if
    \[
      E_1 \subseteq E_2 \subseteq \dots \subseteq E_n \subseteq \dots
    \]
    then
    \[
      \mu (\bigcup_{n \geq 1} E_n) = \lim_{n \to \infty} \mu(E_n) = \sup_{n \geq 1} \mu(E_n).
    \]
  \item downard monotone convergence: if
    \[
      E_1 \supseteq E_2 \supseteq \dots \supseteq E_n \supseteq \dots
    \]
    and \(\mu(E_1) < \infty\) then
    \[
      \mu (\bigcap_{n \geq 1} E_n) = \lim_{n \to \infty} \mu(E_n) = \inf_{n \geq 1} \mu(E_n).
    \]
  \end{enumerate}
\end{proposition}

\begin{proof}\leavevmode
  \begin{enumerate}
  \item
    \[
      \mu(B) = \mu(A) + \underbrace{\mu(B \setminus A)}_{\geq 0}
    \]
    by additivity of \(\mu\).
  \item See example sheet. The idea is that every countable union \(\bigcup_{n \geq 1} A_n\) is a disjoint countable union \(\bigcup_{n \geq 1} B_n\) where for each \(n\), \(B_n \subseteq A_n\). It then follows by \(\sigma\)-additivity.
  \item Let \(E_0 = \emptyset\) so
    \[
      \bigcup_{n \geq 1} E_n = \bigcup_{n \geq 1} (E_n \setminus E_{n - 1}),
    \]
    a disjoint union. By \(\sigma\)-additivity,
    \[
      \mu(\bigcup_{n \geq 1} E_n) = \sum_{n \geq 1} \mu(E_n \setminus E_{n - 1})
    \]
    but for all \(N\), by additivity of \(\mu\),
    \[
      \sum_{n = 1}^N \mu(E_n \setminus E_{n - 1}) = \mu(E_N)
    \]
    so take limit. The supremum part is obvious.
  \item Apply the previous result to \(E_1 \setminus E_n\).
  \end{enumerate}
\end{proof}

\begin{remark}
  Note the \(\mu(E_1) < \infty\) in the last part. Counterexample: \(E_n = [n, \infty) \subseteq \R\).
\end{remark}



















\printindex
\end{document}

% http://www.statslab.cam.ac.uk/~james/Lectures/pm.pdf

% Books: books listed on course handbook, as well as
% Intorduction to measure theory, T. Tao
% Real and Complex analysis, W. Rudin

% related courses: II Linear Analysis

% schedule
% wk 1: Lebesgue measure on \R^d
% wk 2: abstract measure theory
% wk 3: integration
% wk 4: measure theoretical foundations of probability theory
% wk 5: modes of convergence of random variables
% wk 6: Hilbert space techniques, L^p spaces
% wk 7: Fourier analysis, Gaussian random variables, Central Limit Theorem, Law of Large Number Theory
% wk8: Ergodic Theorem
