\documentclass[a4paper]{article}

\def\npart{II}

\def\ntitle{Probability and Measure}
\def\nlecturer{E.\ Breuillard}

\def\nterm{Michaelmas}
\def\nyear{2018}

\ifx \nauthor\undefined
  \def\nauthor{Qiangru Kuang}
\else
\fi

\ifx \ntitle\undefined
  \def\ntitle{Template}
\else
\fi

\ifx \nauthoremail\undefined
  \def\nauthoremail{qk206@cam.ac.uk}
\else
\fi

\ifx \ndate\undefined
  \def\ndate{\today}
\else
\fi

\title{\ntitle}
\author{\nauthor}
\date{\ndate}

%\usepackage{microtype}
\usepackage{mathtools}
\usepackage{amsthm}
\usepackage{stmaryrd}%symbols used so far: \mapsfrom
\usepackage{empheq}
\usepackage{amssymb}
\let\mathbbalt\mathbb
\let\pitchforkold\pitchfork
\usepackage{unicode-math}
\let\mathbb\mathbbalt%reset to original \mathbb
\let\pitchfork\pitchforkold

\usepackage{imakeidx}
\makeindex[intoc]

%to address the problem that Latin modern doesn't have unicode support for setminus
%https://tex.stackexchange.com/a/55205/26707
\AtBeginDocument{\renewcommand*{\setminus}{\mathbin{\backslash}}}
\AtBeginDocument{\renewcommand*{\models}{\vDash}}%for \vDash is same size as \vdash but orginal \models is larger
\AtBeginDocument{\let\Re\relax}
\AtBeginDocument{\let\Im\relax}
\AtBeginDocument{\DeclareMathOperator{\Re}{Re}}
\AtBeginDocument{\DeclareMathOperator{\Im}{Im}}
\AtBeginDocument{\let\div\relax}
\AtBeginDocument{\DeclareMathOperator{\div}{div}}

\usepackage{tikz}
\usetikzlibrary{automata,positioning}
\usepackage{pgfplots}
%some preset styles
\pgfplotsset{compat=1.15}
\pgfplotsset{centre/.append style={axis x line=middle, axis y line=middle, xlabel={$x$}, ylabel={$y$}, axis equal}}
\usepackage{tikz-cd}
\usepackage{graphicx}
\usepackage{newunicodechar}

\usepackage{fancyhdr}

\fancypagestyle{mypagestyle}{
    \fancyhf{}
    \lhead{\emph{\nouppercase{\leftmark}}}
    \rhead{}
    \cfoot{\thepage}
}
\pagestyle{mypagestyle}

\usepackage{titlesec}
\newcommand{\sectionbreak}{\clearpage} % clear page after each section
\usepackage[perpage]{footmisc}
\usepackage{blindtext}

%\reallywidehat
%https://tex.stackexchange.com/a/101136/26707
\usepackage{scalerel,stackengine}
\stackMath
\newcommand\reallywidehat[1]{%
\savestack{\tmpbox}{\stretchto{%
  \scaleto{%
    \scalerel*[\widthof{\ensuremath{#1}}]{\kern-.6pt\bigwedge\kern-.6pt}%
    {\rule[-\textheight/2]{1ex}{\textheight}}%WIDTH-LIMITED BIG WEDGE
  }{\textheight}% 
}{0.5ex}}%
\stackon[1pt]{#1}{\tmpbox}%
}

%\usepackage{braket}
\usepackage{thmtools}%restate theorem
\usepackage{hyperref}

% https://en.wikibooks.org/wiki/LaTeX/Hyperlinks
\hypersetup{
    %bookmarks=true,
    unicode=true,
    pdftitle={\ntitle},
    pdfauthor={\nauthor},
    pdfsubject={Mathematics},
    pdfcreator={\nauthor},
    pdfproducer={\nauthor},
    pdfkeywords={math maths \ntitle},
    colorlinks=true,
    linkcolor={red!50!black},
    citecolor={blue!50!black},
    urlcolor={blue!80!black}
}

\usepackage{cleveref}



% TODO: mdframed often gives bad breaks that cause empty lines. Would like to switch to tcolorbox.
% The current workaround is to set innerbottommargin=0pt.

%\usepackage[theorems]{tcolorbox}





\usepackage[framemethod=tikz]{mdframed}
\mdfdefinestyle{leftbar}{
  %nobreak=true, %dirty hack
  linewidth=1.5pt,
  linecolor=gray,
  hidealllines=true,
  leftline=true,
  leftmargin=0pt,
  innerleftmargin=5pt,
  innerrightmargin=10pt,
  innertopmargin=-5pt,
  % innerbottommargin=5pt, % original
  innerbottommargin=0pt, % temporary hack 
}
%\newmdtheoremenv[style=leftbar]{theorem}{Theorem}[section]
%\newmdtheoremenv[style=leftbar]{proposition}[theorem]{proposition}
%\newmdtheoremenv[style=leftbar]{lemma}[theorem]{Lemma}
%\newmdtheoremenv[style=leftbar]{corollary}[theorem]{corollary}

\newtheorem{theorem}{Theorem}[section]
\newtheorem{proposition}[theorem]{Proposition}
\newtheorem{lemma}[theorem]{Lemma}
\newtheorem{corollary}[theorem]{Corollary}
\newtheorem{axiom}[theorem]{Axiom}
\newtheorem*{axiom*}{Axiom}

\surroundwithmdframed[style=leftbar]{theorem}
\surroundwithmdframed[style=leftbar]{proposition}
\surroundwithmdframed[style=leftbar]{lemma}
\surroundwithmdframed[style=leftbar]{corollary}
\surroundwithmdframed[style=leftbar]{axiom}
\surroundwithmdframed[style=leftbar]{axiom*}

\theoremstyle{definition}

\newtheorem*{definition}{Definition}
\surroundwithmdframed[style=leftbar]{definition}

\newtheorem*{slogan}{Slogan}
\newtheorem*{eg}{Example}
\newtheorem*{ex}{Exercise}
\newtheorem*{remark}{Remark}
\newtheorem*{notation}{Notation}
\newtheorem*{convention}{Convention}
\newtheorem*{assumption}{Assumption}
\newtheorem*{question}{Question}
\newtheorem*{answer}{Answer}
\newtheorem*{note}{Note}
\newtheorem*{application}{Application}

%operator macros

%basic
\DeclareMathOperator{\lcm}{lcm}

%matrix
\DeclareMathOperator{\tr}{tr}
\DeclareMathOperator{\Tr}{Tr}
\DeclareMathOperator{\adj}{adj}

%algebra
\DeclareMathOperator{\Hom}{Hom}
\DeclareMathOperator{\End}{End}
\DeclareMathOperator{\id}{id}
\DeclareMathOperator{\im}{im}
\DeclareMathOperator{\coker}{coker}
\DeclarePairedDelimiter{\generation}{\langle}{\rangle}

%groups
\DeclareMathOperator{\sym}{Sym}
\DeclareMathOperator{\sgn}{sgn}
\DeclareMathOperator{\inn}{Inn}
\DeclareMathOperator{\aut}{Aut}
\DeclareMathOperator{\GL}{GL}
\DeclareMathOperator{\SL}{SL}
\DeclareMathOperator{\PGL}{PGL}
\DeclareMathOperator{\PSL}{PSL}
\DeclareMathOperator{\SU}{SU}
\DeclareMathOperator{\UU}{U}
\DeclareMathOperator{\SO}{SO}
\DeclareMathOperator{\OO}{O}
\DeclareMathOperator{\PSU}{PSU}
\DeclareMathOperator{\Sp}{Sp}


%hyperbolic
\DeclareMathOperator{\sech}{sech}

%field, galois heory
\DeclareMathOperator{\ch}{ch}
\DeclareMathOperator{\gal}{Gal}
\DeclareMathOperator{\emb}{Emb}



%ceiling and floor
%https://tex.stackexchange.com/a/118217/26707
\DeclarePairedDelimiter\ceil{\lceil}{\rceil}
\DeclarePairedDelimiter\floor{\lfloor}{\rfloor}


\DeclarePairedDelimiter{\innerproduct}{\langle}{\rangle}

%\DeclarePairedDelimiterX{\norm}[1]{\lVert}{\rVert}{#1}
\DeclarePairedDelimiter{\norm}{\lVert}{\rVert}



%Dirac notation
%TODO: rewrite for variable number of arguments
\DeclarePairedDelimiterX{\braket}[2]{\langle}{\rangle}{#1 \delimsize\vert #2}
\DeclarePairedDelimiterX{\braketthree}[3]{\langle}{\rangle}{#1 \delimsize\vert #2 \delimsize\vert #3}

\DeclarePairedDelimiter{\bra}{\langle}{\rvert}
\DeclarePairedDelimiter{\ket}{\lvert}{\rangle}




%macros

%general

%divide, not divide
\newcommand*{\divides}{\mid}
\newcommand*{\ndivides}{\nmid}
%vector, i.e. mathbf
%https://tex.stackexchange.com/a/45746/26707
\newcommand*{\V}[1]{{\ensuremath{\symbf{#1}}}}
%closure
\newcommand*{\cl}[1]{\overline{#1}}
%conjugate
\newcommand*{\conj}[1]{\overline{#1}}
%set complement
\newcommand*{\stcomp}[1]{\overline{#1}}
\newcommand*{\compose}{\circ}
\newcommand*{\nto}{\nrightarrow}
\newcommand*{\p}{\partial}
%embed
\newcommand*{\embed}{\hookrightarrow}
%surjection
\newcommand*{\surj}{\twoheadrightarrow}
%power set
\newcommand*{\powerset}{\mathcal{P}}

%matrix
\newcommand*{\matrixring}{\mathcal{M}}

%groups
\newcommand*{\normal}{\trianglelefteq}
%rings
\newcommand*{\ideal}{\trianglelefteq}

%fields
\renewcommand*{\C}{{\mathbb{C}}}
\newcommand*{\R}{{\mathbb{R}}}
\newcommand*{\Q}{{\mathbb{Q}}}
\newcommand*{\Z}{{\mathbb{Z}}}
\newcommand*{\N}{{\mathbb{N}}}
\newcommand*{\F}{{\mathbb{F}}}
%not really but I think this belongs here
\newcommand*{\A}{{\mathbb{A}}}

%asymptotic
\newcommand*{\bigO}{O}
\newcommand*{\smallo}{o}

%probability
\newcommand*{\prob}{\mathbb{P}}
\newcommand*{\E}{\mathbb{E}}

%vector calculus
\newcommand*{\gradient}{\V \nabla}
\newcommand*{\divergence}{\gradient \cdot}
\newcommand*{\curl}{\gradient \cdot}

%logic
\newcommand*{\yields}{\vdash}
\newcommand*{\nyields}{\nvdash}

%differential geometry
\renewcommand*{\H}{\mathbb{H}}
\newcommand*{\transversal}{\pitchfork}
\renewcommand{\d}{\mathrm{d}} % exterior derivative

%number theory
\newcommand*{\legendre}[2]{\genfrac{(}{)}{}{}{#1}{#2}}%Legendre symbol

%algebraic geometry
\DeclareMathOperator{\Spec}{Spec}
\DeclareMathOperator{\Proj}{Proj}

\begin{document}

\begin{titlepage}
  \begin{center}
    \includegraphics[width=0.6\textwidth]{logo.jpg}\par
    \vspace{1cm}
    {\scshape\huge Mathamatics Tripos \par}
    \vspace{2cm}
    {\huge Part \npart \par}
    \vspace{0.6cm}
    {\Huge \bfseries \ntitle \par}
    \vspace{1.2cm}
    {\Large\nterm, \nyear \par}
    \vspace{2cm}
    
    {\large \emph{Lectures by } \par}
    \vspace{0.2cm}
    {\Large \scshape \nlecturer}
    
    \vspace{0.5cm}
    {\large \emph{Notes by }\par}
    \vspace{0.2cm}
    {\Large \scshape \href{mailto:\nauthoremail}{\nauthor}}
 \end{center}
\end{titlepage}

\tableofcontents

\section{Lebesgue measure}

\subsection{Boolean algebra}

\begin{definition}[Boolean algebra]\index{Boolean algebra}
  Let \(X\) be a set. A \emph{Boolean algebra} on \(X\) is a family of subsets of \(X\) which
  \begin{enumerate}
  \item contains \(\emptyset\),
  \item is stable under finite unions and complementation.
  \end{enumerate}
\end{definition}

\begin{eg}\leavevmode
  \begin{itemize}
  \item The \emph{trivial Boolean algebra} \(\mathcal B = \{\emptyset, X\}\).
  \item The \emph{discrete Boolean algebra} \(\mathcal B = 2^X\), the family of all subsets of \(X\).
  \item Less trivially, if \(X\) is a topological space, the family of \emph{constructible sets} forms a Boolean algebra, where a constructible set is the finite union of locally closed set, i.e.\ a set \(E = U \cap F\) where \(U\) is open and \(F\) is closed.
  \end{itemize}
\end{eg}

\begin{definition}[finitely additive measure]\index{finitely additive measure}
  Let \(X\) be a set and \(\mathcal B\) a Boolean algebra on \(X\). A \emph{finitely additive measure} on \((X, \mathcal B)\) is a function \(m: \mathcal B \to [0, +\infty]\) such that
  \begin{enumerate}
  \item \(m(\emptyset) = 0\),
  \item \(m(E \cup F) = m(E) + m(F)\) where \(E \cap F = \emptyset\).
  \end{enumerate}
\end{definition}

\begin{eg}\leavevmode
  \begin{enumerate}
  \item Counting measure: \(m(E) = \#E\), the cardinality of \(E\) where \(\mathcal B\) is the discrete Boolean algebra of \(X\).
  \item More generally, given \(f: X \to [0, +\infty]\), define for \(E \subseteq X\),
    \[
      m(E) = \sum_{e \in E} f(e).
    \]
  \item Suppose \(X = \coprod_{i = 1}^N X_i\), then define \(\mathcal B(X)\) to be the unions of \(X_i\)'s. Assign a weight \(a_i \geq 0\) to each \(X_i\) and define \(m(E) = \sum_{i: X_i \subseteq E} a_i\) for \(E \in \mathcal B\).
  \end{enumerate}
\end{eg}

\subsection{Jordan measure}

This section is a historic review and provides intuition for Lebesgue measure theory. We'll gloss over details of proofs in this section.

\begin{definition}
  A subset of \(\R^d\) is called \emph{elementary} if it is a finite union of \emph{boxes}, where a box is a set \(B = I_1 \times \dots I_d\) where each \(I_i\) is a finite interval of \(\R\).
\end{definition}

\begin{proposition}
  Let \(B \subseteq \R^d\) be a box. Let \(\mathcal E(B)\) be the family of elementary subsets of \(B\). Then
  \begin{enumerate}
  \item \(\mathcal E(B)\) is a Boolean algebra on \(B\),
  \item every \(E \in \mathcal E(B)\) is a disjoint finite union of boxes,
  \item if \(E \in \mathcal E(B)\) can be written as disjoint finite union in two ways, \(E = \bigcup_{i = 1}^n B_i = \bigcup_{j = 1}^m B_j'\), then \(\sum_{i = 1}^n |B_i| = \sum_{j = 1}^m |B_j'|\) where \(|B| = \prod_{i = 1}^d |b_i - a_i|\) if \(B = I_1 \times \dots \times I_d\) and \(I_i\) has endpoints \(a_i, b_i\).
  \end{enumerate}
\end{proposition}

Following this, we can define a finitely additive measure correponding to our intuition of length, area, volume etc:

\begin{proposition}
  Define \(m(E) = \sum_{i = 1}^n |B_i|\) if \(E\) is any elementary set and is the disjoint union of boxes \(B_i \subseteq \R^d\). Then \(m\) is a finitely additive measure on \(\mathcal E(B)\) for any box \(B\).
\end{proposition}

\begin{definition}
  A subset \(E \subseteq \R^d\) is \emph{Jordan measurable} if for any \(\varepsilon > 0\) there are elementary sets \(A, B\), \(A \subseteq E \subseteq B\) and \(m(B \setminus A) < \varepsilon\).
\end{definition}

\begin{remark}
  Jordan measurable sets are bounded.
\end{remark}

\begin{proposition}
  If a set \(E \subseteq \R^d\) is Jordan measurable, then
  \[
    \sup_{A \subseteq E \text{ elementary}} \{m(A)\} = \inf_{B \supseteq E \text{ elementary}} \{m(B)\}.
  \]
  In which case we define the \emph{Jordan measure} of \(E\) as
  \[
    m(E) = \sup_{A \subseteq E} \{m(A)\}.
  \]
\end{proposition}

\begin{proof}
  Take \(A_n \subseteq E\) such that \(m(A_n) \nearrow \sup\) and \(B_n \supseteq E\) such that \(m(B_n) \searrow \inf\). Note that
  \[
    \inf \leq m(B_n) = m(A_n) + m(B_n \setminus A_n) \leq \sup + m(B_n \setminus A_n) \leq \sup + \varepsilon
  \]
  for arbitrary \(\varepsilon > 0\) so they are equal.
\end{proof}

\begin{ex}\leavevmode
  \begin{enumerate}
  \item If \(B\) is a box, the family \(\mathcal J(B)\) of Jordan measurable subsets of \(B\) is a Boolean algebra.
  \item A subset \(E \subseteq [0, 1]\) is Jordan measurable if and only if \(\mathbf 1_E\), the indicator funciton on \(E\), is Riemann integrable.
  \end{enumerate}
\end{ex}

\subsection{Lebesgue measure}

Although Jordan measure corresponds to the intuition of length, area and volume, it suffer from a few severe problems and issues:
\begin{enumerate}
\item unbounded sets in \(\R^d\) are not Jordan measurable.
\item \(\mathbf 1_{\Q \cap [0, 1]}\) is not Riemann integrable, and therefore \(\Q \cap [0, 1]\) is not Jordan measurable.
\item pointwise limits of Riemann integrable functions \(f_n := \mathbf 1_{\frac{1}{n!} \Z \cap [0, 1]} \to \mathbf 1_{\Q \cap [0, 1]}\) is not Riemann integrable.
\end{enumerate}

The idea of Lebesgue is to use countable covers by boxes.

\begin{definition}
  A subset \(E \subseteq \R^d\) is \emph{Lebesgue measurable} if for all \(\varepsilon > 0\), there exists a countable union of boxes \(C\) with \(E \subseteq C\) and \(m^*(C \setminus E) < \varepsilon\), where \(m^*\), the \emph{Lebesgue outer measure}, is defined as
  \[
    m^*(E) = \inf \{\sum_{i \geq 1} |B_i|: E \subseteq \bigcup_{i \geq 1} B_i, B_i \text{ boxes}\}
  \]
  for \emph{every} subset \(E \subseteq \R^d\).
\end{definition}

\begin{remark}
  wlog in these definitions we may assume that boxes are open.
\end{remark}

% We are going to show that the family of Lebesgue measurable subsets is not only a Boolean algebra, but also stable under countable union. Next we are going to define the Lebesgue measure on the family, with the additive property (which is not possessed by Lebesgue outer measure). In fact, we can show that we cannot define a measure for \emph{all} subsets of a set.

\begin{proposition}
  \label{prop:Lebesgue measurable subset is Boolean algebra}
  The family \(\mathcal L\) of Lebesgue measurable subsets of \(\R^d\) is a Boolean algebra stable under countable unions.
\end{proposition}

\begin{lemma}\leavevmode
  \begin{enumerate}
  \item \(m^*\) is monotone: if \(E \subseteq F\) then \(m^*(E) \subseteq m^*(F)\).
  \item \(m^*\) is countably subadditive: if \(E = \bigcup_{n \geq 1} E_n\) where \(E_n \subseteq \R^d\) then
    \[
      m^*(E) \leq \sum_{n \geq 1} m^*(E_n).
    \]
  \end{enumerate}
\end{lemma}

\begin{proof}
  Monotonicity is obvious. For countable subadditivity, pick \(\varepsilon > 0\) and let \(C_n = \bigcup_{i \geq 1} C_{n, i}\) where \(C_{n, i}\) are boxes such that \(E_n \subseteq C_n\) and
  \[
    \sum_{i \geq 1} |C_{n, i}| \leq m^*(E_n) + \frac{\varepsilon}{2^n}.
  \]
  Then
  \[
    \sum_{n \geq 1} \sum_{i \geq 1} |C_{n, i}|
    \leq \sum_{n \geq 1} (m^*(E_n) + \frac{\varepsilon}{2^n})
    = \varepsilon + \sum_{n \geq 1} m^*(E_n)
  \]
  and \(E \subseteq \bigcup_{n \geq 1} C_n = \bigcup_{n \geq 1} \bigcup_{i \geq 1} C_{n, i}\) so
  \[
    m^*(E) \leq \varepsilon + \sum_{n \geq 1} m^*(E_n)
  \]
  for all \(\varepsilon > 0\).
\end{proof}

\begin{remark}
  Note that \(m^*\) is \emph{not} additive on the family of all subsets of \(\R^d\). However, it will be on \(\mathcal L\), as we will show later.
  % eg?
\end{remark}

\begin{lemma}
  If \(A, B\) are disjoint compact subsets of \(\R^d\) then
  \[
    m^*(A \cup B) = m^*(A) + m^*(B).
  \]
\end{lemma}

\begin{proof}
  \(\leq\) by the previous lemma so need to show \(\geq\). Pick \(\varepsilon > 0\). Let \(A \cup B \subseteq \bigcup_{n \geq 1} B_n\) where \(B_n\) are open boxes such that
  \[
    \sum_{n \geq 1} |B_n| \leq m^*(A \cup B) + \varepsilon.
  \]
  wlog we may assume that the side lengths of each \(B_n\) are \(< \frac{\alpha}{2}\), where
  \[
    \alpha = \inf \{\norm{x - y}_1: x \in A, y \in B\} > 0.
  \]
  where the inequality comes from the fact that \(A\) and \(B\) are compact and thus closed.

  wlog we may discard the \(B_n\)'s that do not interesect \(A \cup B\). Then by construction
  \[
    \sum_{n \geq 1} |B_n| = \sum_{n \geq 1, B_n \cap A = \emptyset} |B_n| + \sum_{n \geq 1, B_n \cap B = \emptyset} |B_n|
    \geq m^*(A) + m^*(B)
  \]
  so
  \[
    \varepsilon + m^*(A \cup B) \geq m^*(A) + m^*(B)
  \]
  for all \(\varepsilon\).
\end{proof}

\begin{lemma}
  If \(E \subseteq \R^d\) has \(m^*(E) = 0\) then \(E \in \mathcal L\).
\end{lemma}

\begin{definition}[null set]\index{null set}
  A set \(E \subseteq \R^d\) such that \(m^*(E) = 0\) is called a \emph{null set}.
\end{definition}

\begin{proof}
  For all \(\varepsilon > 0\), there exist \(C = \bigcup_{n \geq 1} B_n\) where \(B_n\) are boxes such that \(E \subseteq C\) and \(\sum_{n \geq 1} |B_n| \leq \varepsilon\). But
  \[
    m^*(C \setminus E) \leq m^*(C) \leq \varepsilon.
  \]
\end{proof}

\begin{lemma}
  \label{lem:open and closed sets are Lebesgue measurable}
  Every open subset of \(\R^d\) and every closed subset of \(\R^d\) is in \(\mathcal L\).
\end{lemma}

We will prove the lemma using the fact that the family of Lebesgue measurable subsets is stable under countable union, which itself \emph{does not} use this lemma. This lemma, however, will be used to show the stability under complementation. Since the proof is quite technical (it has more to do with general topology than measure theory), for brevity and fluency of ideas we present the proof the main proposition first.

\begin{proof}[Proof of \Cref{prop:Lebesgue measurable subset is Boolean algebra}]
  It is obvious that \(\emptyset \in \mathcal L\). To show it is stable under countable unions, start with \(E_n \in \mathcal L\) for \(n \geq 1\). Need to show \(E := \bigcup_{n \geq 1} E_n \in \mathcal L\).

  Pick \(\varepsilon > 0\). By assumption there exist \(C_n = \bigcup_{i \geq 1} B_{n, i}\) where \(B_{n, i}\) are boxes such that \(E_n \subseteq C_n\) and
  \[
    m^*(C_n \setminus E_n) < \frac{\varepsilon}{2^n}.
  \]
  Now
  \[
    E = \bigcup_{n \geq 1} E_n \subseteq \bigcup_{n \geq 1} C_n =: C
  \]
  so \(C\) is again a countable union of boxes and \(C \setminus E \subseteq \bigcup_{n \geq 1} C_n \setminus E_n\).
  so
  \[
    m^*(C \setminus E) \leq \sum_{n \geq 1} m^*(C_n \setminus E_n) \leq \sum_{n \geq 1} \frac{\varepsilon}{2^n} = \varepsilon
  \]
  by countable subadditivity so \(E \in \mathcal L\).

  To show it is stable under complementation, suppose \(E \in \mathcal L\). By assumption there exist \(C_n\) a countable union of boxes with \(E \subseteq C_n\) and \(m^*(C_n \setminus E) \leq \frac{1}{n}\). wlog we may assume the boxes are open so \(C_n\) is open, \(C_n^c\) is closed so \(C_n^c \in \mathcal L\). Thus \(\bigcup_{n \geq 1} C_n^c \in \mathcal L\) by first part of the proof.

  But
  \[
    m^*(E^c \setminus \bigcup_{n \geq 1} C_n^c)
    \leq m^*(E^c \setminus C_n^c)
    = m^*(C_n \setminus E)
    \leq \frac{1}{n}
  \]
  so \(m^*(E^c \setminus \bigcup_{n \geq 1} C_n^c) = 0\) so \(E^c \setminus \bigcup_{n \geq 1} C_n^c \in \mathcal L\) since it is a null set. But
  \[
    E^c = (E^c \setminus \bigcup_{n \geq 1} C_n^c) \cup \bigcup_{n \geq 1} C_n^c,
  \]
  both of which are in \(\mathcal L\) so \(E^c \in \mathcal L\).
\end{proof}

\begin{proof}[Proof of \Cref{lem:open and closed sets are Lebesgue measurable}]
  Every open set in \(\R^d\) is a countable union of boxes so is in \(\mathcal L\). It is more subtle for closed sets. %(in fact it is the key difference between this and Jordan measure).
  The key observation is that every closed set is the countable union of compact subsets so we are left to show compact sets of \(\R^d\) are in \(\mathcal L\).

  Let \(F \subseteq \R^d\) be compact. For all \(k \geq 1\), there exist \(O_k\) a countable union of open sets such that \(F \subseteq O_k := \bigcup_{i \geq 1} O_{k, i}\) where \(O_{k, i}\) are open boxes such that
  \[
    \sum_{i \geq 1} |O_{k, i}| \leq m^*(F) + \frac{1}{2^k}.
  \]
  By compactness there exist a finite subcover so we can assume \(O_k\) is a finite union of open boxes. Moreover, wlog assume that
  \begin{enumerate}
  \item the side lengths of \(O_{k, i}\) are \(\leq \frac{1}{2^k}\).
  \item for each \(i\), \(O_{k, i}\) intersects \(F\).
  \item \(O_{k + 1} \subseteq O_k\) (by replacing \(O_{k + 1}\) with \(O_{k + 1} \cap O_k\) iteratively).
  \end{enumerate}
  Then \(F = \bigcap_{k \geq 1} O_k\) and we are left to show \(m^*(O_k \setminus F) \to 0\). By additivity on disjoint compact sets,
  \[
    m^*(F) + m^*(\cl O_i \setminus O_{i + 1}) = m^*(F \cup (\cl O_i \setminus O_{i + 1}))
  \]
  so
  \[
    m^*(F) + m^*(\cl O_i \setminus O_{i + 1}) \leq m^*(\cl O_i)
    \leq \sum_{j \geq 1} |O_{i, j}|
    % = |O_i|
    \leq m^*(F) + \frac{1}{2^i}
  \]
  so \(m^*(\cl O_i \setminus O_{i + 1}) \leq \frac{1}{2^i}\). Finally,
  \[
    m^*(O_k \setminus F)
    = m^*(\bigcup_{i \geq k} (O_i \setminus O_{i + 1}))
    \leq \sum_{i \geq k} m^*(O_i \setminus O_{i + 1})
    \leq \sum_{i \geq k} \frac{1}{2^i}
    = \frac{1}{2^{k - 1}}.
  \]
\end{proof}





















\printindex
\end{document}

% http://www.statslab.cam.ac.uk/~james/Lectures/pm.pdf

% Books: books listed on course handbook, as well as
% Intorduction to meaure theory, T. Tao
% Real and Complex analysis, W. Rudin

% related courses: II Linear Analysis

% schedule
% wk 1: Lebesgue measure on \R^d
% wk 2: abstract measure theory
% wk 3: integration
% wk 4: measure theoretical foundations of probability theory
% wk 5: modes of convergence of random variables
% wk 6: Hilbert space techniques, L^p spaces
% wk 7: Fourier analysis, Gaussian random variables, Central Limit Theorem, Law of Large Number Theory
% wk8: Ergodic Theorem
