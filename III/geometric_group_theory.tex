\documentclass[a4paper]{article}

\def\npart{III}

\def\ntitle{Geometric Group Theory}
\def\nlecturer{A.\ Khukhro}

\def\nterm{Lent}
\def\nyear{2020}

\ifx \nauthor\undefined
  \def\nauthor{Qiangru Kuang}
\else
\fi

\ifx \ntitle\undefined
  \def\ntitle{Template}
\else
\fi

\ifx \nauthoremail\undefined
  \def\nauthoremail{qk206@cam.ac.uk}
\else
\fi

\ifx \ndate\undefined
  \def\ndate{\today}
\else
\fi

\title{\ntitle}
\author{\nauthor}
\date{\ndate}

%\usepackage{microtype}
\usepackage{mathtools}
\usepackage{amsthm}
\usepackage{stmaryrd}%symbols used so far: \mapsfrom
\usepackage{empheq}
\usepackage{amssymb}
\let\mathbbalt\mathbb
\let\pitchforkold\pitchfork
\usepackage{unicode-math}
\let\mathbb\mathbbalt%reset to original \mathbb
\let\pitchfork\pitchforkold

\usepackage{imakeidx}
\makeindex[intoc]

%to address the problem that Latin modern doesn't have unicode support for setminus
%https://tex.stackexchange.com/a/55205/26707
\AtBeginDocument{\renewcommand*{\setminus}{\mathbin{\backslash}}}
\AtBeginDocument{\renewcommand*{\models}{\vDash}}%for \vDash is same size as \vdash but orginal \models is larger
\AtBeginDocument{\let\Re\relax}
\AtBeginDocument{\let\Im\relax}
\AtBeginDocument{\DeclareMathOperator{\Re}{Re}}
\AtBeginDocument{\DeclareMathOperator{\Im}{Im}}
\AtBeginDocument{\let\div\relax}
\AtBeginDocument{\DeclareMathOperator{\div}{div}}

\usepackage{tikz}
\usetikzlibrary{automata,positioning}
\usepackage{pgfplots}
%some preset styles
\pgfplotsset{compat=1.15}
\pgfplotsset{centre/.append style={axis x line=middle, axis y line=middle, xlabel={$x$}, ylabel={$y$}, axis equal}}
\usepackage{tikz-cd}
\usepackage{graphicx}
\usepackage{newunicodechar}

\usepackage{fancyhdr}

\fancypagestyle{mypagestyle}{
    \fancyhf{}
    \lhead{\emph{\nouppercase{\leftmark}}}
    \rhead{}
    \cfoot{\thepage}
}
\pagestyle{mypagestyle}

\usepackage{titlesec}
\newcommand{\sectionbreak}{\clearpage} % clear page after each section
\usepackage[perpage]{footmisc}
\usepackage{blindtext}

%\reallywidehat
%https://tex.stackexchange.com/a/101136/26707
\usepackage{scalerel,stackengine}
\stackMath
\newcommand\reallywidehat[1]{%
\savestack{\tmpbox}{\stretchto{%
  \scaleto{%
    \scalerel*[\widthof{\ensuremath{#1}}]{\kern-.6pt\bigwedge\kern-.6pt}%
    {\rule[-\textheight/2]{1ex}{\textheight}}%WIDTH-LIMITED BIG WEDGE
  }{\textheight}% 
}{0.5ex}}%
\stackon[1pt]{#1}{\tmpbox}%
}

%\usepackage{braket}
\usepackage{thmtools}%restate theorem
\usepackage{hyperref}

% https://en.wikibooks.org/wiki/LaTeX/Hyperlinks
\hypersetup{
    %bookmarks=true,
    unicode=true,
    pdftitle={\ntitle},
    pdfauthor={\nauthor},
    pdfsubject={Mathematics},
    pdfcreator={\nauthor},
    pdfproducer={\nauthor},
    pdfkeywords={math maths \ntitle},
    colorlinks=true,
    linkcolor={red!50!black},
    citecolor={blue!50!black},
    urlcolor={blue!80!black}
}

\usepackage{cleveref}



% TODO: mdframed often gives bad breaks that cause empty lines. Would like to switch to tcolorbox.
% The current workaround is to set innerbottommargin=0pt.

%\usepackage[theorems]{tcolorbox}





\usepackage[framemethod=tikz]{mdframed}
\mdfdefinestyle{leftbar}{
  %nobreak=true, %dirty hack
  linewidth=1.5pt,
  linecolor=gray,
  hidealllines=true,
  leftline=true,
  leftmargin=0pt,
  innerleftmargin=5pt,
  innerrightmargin=10pt,
  innertopmargin=-5pt,
  % innerbottommargin=5pt, % original
  innerbottommargin=0pt, % temporary hack 
}
%\newmdtheoremenv[style=leftbar]{theorem}{Theorem}[section]
%\newmdtheoremenv[style=leftbar]{proposition}[theorem]{proposition}
%\newmdtheoremenv[style=leftbar]{lemma}[theorem]{Lemma}
%\newmdtheoremenv[style=leftbar]{corollary}[theorem]{corollary}

\newtheorem{theorem}{Theorem}[section]
\newtheorem{proposition}[theorem]{Proposition}
\newtheorem{lemma}[theorem]{Lemma}
\newtheorem{corollary}[theorem]{Corollary}
\newtheorem{axiom}[theorem]{Axiom}
\newtheorem*{axiom*}{Axiom}

\surroundwithmdframed[style=leftbar]{theorem}
\surroundwithmdframed[style=leftbar]{proposition}
\surroundwithmdframed[style=leftbar]{lemma}
\surroundwithmdframed[style=leftbar]{corollary}
\surroundwithmdframed[style=leftbar]{axiom}
\surroundwithmdframed[style=leftbar]{axiom*}

\theoremstyle{definition}

\newtheorem*{definition}{Definition}
\surroundwithmdframed[style=leftbar]{definition}

\newtheorem*{slogan}{Slogan}
\newtheorem*{eg}{Example}
\newtheorem*{ex}{Exercise}
\newtheorem*{remark}{Remark}
\newtheorem*{notation}{Notation}
\newtheorem*{convention}{Convention}
\newtheorem*{assumption}{Assumption}
\newtheorem*{question}{Question}
\newtheorem*{answer}{Answer}
\newtheorem*{note}{Note}
\newtheorem*{application}{Application}

%operator macros

%basic
\DeclareMathOperator{\lcm}{lcm}

%matrix
\DeclareMathOperator{\tr}{tr}
\DeclareMathOperator{\Tr}{Tr}
\DeclareMathOperator{\adj}{adj}

%algebra
\DeclareMathOperator{\Hom}{Hom}
\DeclareMathOperator{\End}{End}
\DeclareMathOperator{\id}{id}
\DeclareMathOperator{\im}{im}
\DeclareMathOperator{\coker}{coker}
\DeclarePairedDelimiter{\generation}{\langle}{\rangle}

%groups
\DeclareMathOperator{\sym}{Sym}
\DeclareMathOperator{\sgn}{sgn}
\DeclareMathOperator{\inn}{Inn}
\DeclareMathOperator{\aut}{Aut}
\DeclareMathOperator{\GL}{GL}
\DeclareMathOperator{\SL}{SL}
\DeclareMathOperator{\PGL}{PGL}
\DeclareMathOperator{\PSL}{PSL}
\DeclareMathOperator{\SU}{SU}
\DeclareMathOperator{\UU}{U}
\DeclareMathOperator{\SO}{SO}
\DeclareMathOperator{\OO}{O}
\DeclareMathOperator{\PSU}{PSU}
\DeclareMathOperator{\Sp}{Sp}


%hyperbolic
\DeclareMathOperator{\sech}{sech}

%field, galois heory
\DeclareMathOperator{\ch}{ch}
\DeclareMathOperator{\gal}{Gal}
\DeclareMathOperator{\emb}{Emb}



%ceiling and floor
%https://tex.stackexchange.com/a/118217/26707
\DeclarePairedDelimiter\ceil{\lceil}{\rceil}
\DeclarePairedDelimiter\floor{\lfloor}{\rfloor}


\DeclarePairedDelimiter{\innerproduct}{\langle}{\rangle}

%\DeclarePairedDelimiterX{\norm}[1]{\lVert}{\rVert}{#1}
\DeclarePairedDelimiter{\norm}{\lVert}{\rVert}



%Dirac notation
%TODO: rewrite for variable number of arguments
\DeclarePairedDelimiterX{\braket}[2]{\langle}{\rangle}{#1 \delimsize\vert #2}
\DeclarePairedDelimiterX{\braketthree}[3]{\langle}{\rangle}{#1 \delimsize\vert #2 \delimsize\vert #3}

\DeclarePairedDelimiter{\bra}{\langle}{\rvert}
\DeclarePairedDelimiter{\ket}{\lvert}{\rangle}




%macros

%general

%divide, not divide
\newcommand*{\divides}{\mid}
\newcommand*{\ndivides}{\nmid}
%vector, i.e. mathbf
%https://tex.stackexchange.com/a/45746/26707
\newcommand*{\V}[1]{{\ensuremath{\symbf{#1}}}}
%closure
\newcommand*{\cl}[1]{\overline{#1}}
%conjugate
\newcommand*{\conj}[1]{\overline{#1}}
%set complement
\newcommand*{\stcomp}[1]{\overline{#1}}
\newcommand*{\compose}{\circ}
\newcommand*{\nto}{\nrightarrow}
\newcommand*{\p}{\partial}
%embed
\newcommand*{\embed}{\hookrightarrow}
%surjection
\newcommand*{\surj}{\twoheadrightarrow}
%power set
\newcommand*{\powerset}{\mathcal{P}}

%matrix
\newcommand*{\matrixring}{\mathcal{M}}

%groups
\newcommand*{\normal}{\trianglelefteq}
%rings
\newcommand*{\ideal}{\trianglelefteq}

%fields
\renewcommand*{\C}{{\mathbb{C}}}
\newcommand*{\R}{{\mathbb{R}}}
\newcommand*{\Q}{{\mathbb{Q}}}
\newcommand*{\Z}{{\mathbb{Z}}}
\newcommand*{\N}{{\mathbb{N}}}
\newcommand*{\F}{{\mathbb{F}}}
%not really but I think this belongs here
\newcommand*{\A}{{\mathbb{A}}}

%asymptotic
\newcommand*{\bigO}{O}
\newcommand*{\smallo}{o}

%probability
\newcommand*{\prob}{\mathbb{P}}
\newcommand*{\E}{\mathbb{E}}

%vector calculus
\newcommand*{\gradient}{\V \nabla}
\newcommand*{\divergence}{\gradient \cdot}
\newcommand*{\curl}{\gradient \cdot}

%logic
\newcommand*{\yields}{\vdash}
\newcommand*{\nyields}{\nvdash}

%differential geometry
\renewcommand*{\H}{\mathbb{H}}
\newcommand*{\transversal}{\pitchfork}
\renewcommand{\d}{\mathrm{d}} % exterior derivative

%number theory
\newcommand*{\legendre}[2]{\genfrac{(}{)}{}{}{#1}{#2}}%Legendre symbol

%algebraic geometry
\DeclareMathOperator{\Spec}{Spec}
\DeclareMathOperator{\Proj}{Proj}

\DeclareMathOperator{\rk}{rk} % rank

\begin{document}

\begin{titlepage}
  \begin{center}
    \includegraphics[width=0.6\textwidth]{logo.jpg}\par
    \vspace{1cm}
    {\scshape\huge Mathamatics Tripos \par}
    \vspace{2cm}
    {\huge Part \npart \par}
    \vspace{0.6cm}
    {\Huge \bfseries \ntitle \par}
    \vspace{1.2cm}
    {\Large\nterm, \nyear \par}
    \vspace{2cm}
    
    {\large \emph{Lectures by } \par}
    \vspace{0.2cm}
    {\Large \scshape \nlecturer}
    
    \vspace{0.5cm}
    {\large \emph{Notes by }\par}
    \vspace{0.2cm}
    {\Large \scshape \href{mailto:\nauthoremail}{\nauthor}}
 \end{center}
\end{titlepage}

\tableofcontents

\setcounter{section}{-1}

\section{Introduction}

Contents:
\begin{enumerate}
\item free groups: ``universal property'', study subgroups using topology,
\item group presentations and constructions, ways of making new groups from old,
\item Cayley graphs, viewing groups geometrically (e.g.\ \(\Z\)), connections to group actions,
\item geometric properties of groups, growth, other geometric invariants, ``dictionary'' between algebra and geometry,
\item amenable groups.
\end{enumerate}

\section{Free groups}

Let \(S\) be a set, called an \emph{alphabet}\index{alphabet}, and let \(S^{-1}\) be the set of formal inverses of elements in \(S\), i.e.\ \(S^{-1} = \{s^{-1}: s \in S\}\). A \emph{word}\index{word} in the alphabet \(S\) is a finite sequence of elements in \(S \cup S^{-1}\) and the empty word. A word is \emph{reduced}\index{word!reduced} if it does not contain occurrences of \(ss^{-1}, s^{-1}s\). Given a word, we can reduce it by removing any such subwords. For example if \(S = \{a, b, c\}\), \(aa^{-1}bcb^{-1}bc^{-1}\) is a word and we can reduce it to \(bcc^{-1}\), and further to \(b\). This induces an equivalence relation such that there is a unique reduced word in each class. We also write \(s^2\) for \(ss\).

\begin{definition}[free group]\index{free group}
  The \emph{free group} on the set \(S\), denoted \(F(S)\), is the set of reduced words in \(S\), with the operation of concatenation (followed by reduction if necessary).
\end{definition}

Free groups satisfies the universal property

\begin{theorem}
  Given a free group \(F(S)\) with an inclusion \(\iota: S \to F(S)\), whenever \(G\) is a group with a function \(\varphi: S \to G\), there is a unique group homomorphism \(\overline \varphi: F(S) \to G\) such that the following diagram commutes
  \[
    \begin{tikzcd}
      S \ar[r, "\iota"] \ar[dr, "\varphi"] & F(S) \ar[d, "\overline \varphi"] \\
      & G
    \end{tikzcd}
  \]
\end{theorem}

\begin{proof}
  Given \(\varphi: S \to G\), define \(\overline \varphi: F(S) \to G\) by \(\overline \varphi(s_{i_1}^{\alpha_1} \cdots s_{i_n}^{\alpha_n}) = \varphi(s_{i_1})^{\alpha_1} \cdots \varphi(s_{i_n})^{\alpha_n}\). Check this is a homomorphism.
\end{proof}

\begin{definition}[rank]\index{rank}
  The cardinality of \(S\) is the \emph{rank} of \(F(S)\), denoted by \(\operatorname{rk}(F(S))\).
\end{definition}

\begin{corollary}
  If \(|S| = |T|\) the \(F(S) \cong F(T)\).
\end{corollary}

\begin{proof}
  If \(|S| = |T|\) then there exists a bijection \(\phi: S \to T\). Consider
  \[
    \begin{tikzcd}
      S \ar[r] \ar[dr, "\theta"] & F(S) \ar[d, dotted, "\overline \theta"] \\
      & F(S)
\end{tikzcd}
  \]
  where \(\overline \theta\) is a homomorphism by the universal property. Similarly we have \(\overline{\theta^{-1}}: F(T) \to F(S)\) and \(\overline{\theta^{-1}} \compose \overline \theta: F(S) \to F(S)\) extends the identity map \(S \to F(S)\) so must be the identity on \(F(S)\). Same for the other way so \(\overline \theta\) is an isomorphism.
\end{proof}

\begin{notation}
  Write \(F_n\) for the isomorphism class of \(F(S)\) with \(|S| = n\).
\end{notation}

\begin{ex}
  If \(F_n \cong F_m\) then \(n = m\).
\end{ex}

\begin{corollary}
  Every group is a quotient of a free group.
\end{corollary}

\begin{proof}
  Given \(G\), consider \(F(G)\). By the universal property exists a homomorphism \(\pi: F(G) \to G\) extending the identity, so must be surjective.
\end{proof}

\begin{definition}
  Let \(G\) be a group, \(A \subseteq G\) a subset. Define \(\langle A\rangle\) to be the intersection of all subgroups containing \(A\), i.e.\ the unique smallest subgroup containing \(A\).We also call it the subgroup generated by \(A\).
\end{definition}

\begin{definition}
  \(G\) is \emph{generated} by \(A \subseteq G\) if \(\langle A \rangle = G\). Then \(A\) is a \emph{generating set} of \(G\). \(G\) is \emph{finitely generated} if exists a finite generating set of \(G\).
\end{definition}

\begin{notation}
  Write \(\langle a_1, \dots, a_n \rangle\) to mean \(\langle \{a_1, \dots, a_n\} \rangle\).
\end{notation}

\begin{eg}\leavevmode
  \begin{enumerate}
  \item \(\Z_n, \Z\) can be generated by one element.
  \item \(\Z^n\) can be generated by \(\geq n\) elements.
  \item \(F_2 = \langle a, b \rangle = \langle a, ab \rangle\) so generating sets are not unique.
  \end{enumerate}
\end{eg}

\begin{definition}
  A group \(F\) is \emph{freely generated} by \(S \subseteq F\) if for any group \(G\) and any map \(\varphi: S \to G\), exists a unique homomorphism \(\tilde \varphi: F \to G\) extending \(\varphi\).
\end{definition}

\begin{lemma}
  If \(F\) is freely generated by \(S\) then \(F\) is generated by \(S\).
\end{lemma}

\subsection{Subgroups of free groups}

Let's see some examples of subgroups of free groups.

\begin{itemize}
\item Given any \(e \ne w \in F_n\), \(\langle w \rangle \cong \Z\).
\item Given \(T \subseteq S\), \(\langle T \rangle\) is a free subgroup of \(F(S)\) of rank \(|T|\).
\item If \(S = \{a, b\}\), the set \(\{a^{-n} ba^n: n \in \N\}\) freely generates a subgroup of \(F_2\), so isomorphic to \(F_\infty\) (exercise).
\end{itemize}

\begin{remark}
  Subgroups of finitely generated groups are not necessarily finitely generated.
\end{remark}

Revision of fundamental groups. See IID Algebraic Topology. Particularly relevant to this course is \(\pi_1(\bigvee_{i = 1}^n S^1) = F_n\), and a connected loop-free graph is contractible so has trivial \(\pi_1\).

It is the fact that if \(X\) is sufficiently nice and \(Y \subseteq X\) is closed simply connected, then collapsing \(Y\) to a piont does not alter \(\pi_1(X)\). In particular, for graphs we can collapsing \(T\), a maximal spanning tree, to get a bouquet of circles. Since maximal spanning tree always exists (use axiom of choice if the graph is infinite), \(\pi_1\) of a graph is a free group of rank equal to the number edges not in the maximal spanning tree.

Recall the Galois correspondence between subgroups of \(\pi_1(X)\) and covering spaces: we have a bijection between covering maps \(p: (\tilde X, \tilde x_0) \to (X, x_0)\) and subgroups of \(\pi_1(X, x_0)\).

Thus let \(X = \bigvee_{i = 1}^n S^1\). For any \(H \leq \pi_1(X) \cong F_n\), there is a covering space \(\overline X\) with \(\pi_1(\overline X) \cong H\). Since \(\overline X\), being a cover of a graph, is a graph, we have \(H\) is free. This shows that every subgroup of a free group is free.

We work out the rank of \(H\) given its index in \(F_n\). The index of \(H\) in \(F_n\) is exactly the degree of the covering map \(\overline X \to X\), i.e.\ the number of vertices of \(\overline X\). Each vertex of \(\overline X\) has degree \(2n\) so the number of edges in \(\overline X\) is \([F_n: H] \cdot 2n \cdot \frac{1}{2} = [F_n : H] \cdot n\). To work out the number of edges not in a maximal spanning tree, use the graph theoretic fact that a tree on \(n\) vertices has exactly \(n - 1\) edges (exercise), so the number of edges not in a maximal spanning tree is
\[
  [F_n: H] \cdot n - [F_n : H] - 1 = (n - 1) [F_n : H] + 1.
\]

\begin{theorem}[Nielsen-Schreier]\index{Nielsen-Schreier formula}
  Every subgroup of a free group is free and if the subgroup has finite index then
  \[
    \rk(H) = [F_n: H] (\rk(F_n) - 1) + 1.
  \]
\end{theorem}
Mnemonic:
\[
  \rk(H) - 1 = (\rk(F_n) - 1) [F_n : H].
\]

\begin{eg}
  A degree \(2\) cover of \(S^1 \vee S^1\) realises \(F_3\) as a subgroup of index \(2\) in \(F_2\).
\end{eg}

The group of \emph{covering transformation}, or \emph{deck transformation} of a cover is the group of isomorphisms \(\overline X \to \overline X\).

A cover is \emph{normal}\index{normal cover} if for any two lifts of the basepoint \(x_0 \in X\), there is a covering transformation of \(\overline X\) sending one to the other.

Normal covering spaces correspond to normal subgroups of \(\pi_1(X)\). If the cover is normal then the group of covering transformations is isomorphic to \(\pi_1(X)\) quotiented by the corresponding subgroup.

\begin{eg}
  In the previous example we have \(F_3 \normal F_2\). We can have a nonnormal index 3, and a normal one.
\end{eg}

\section{Group presentations and constructions}

\begin{definition}[normal closure]\index{normal closure}
  The \emph{normal closure} of a subset \(A \subseteq G\), denoted \(\langle\langle A\rangle\rangle\), is the unique smallest normal subgroup of \(G\) containing \(A\).
\end{definition}

Given a free group \(F(S)\) and \(R \subseteq F(S)\), we write \(\langle S|R \rangle\) for the group \(F(S)/\langle\langle R\rangle\rangle\). \(R\) and \(S\) are called \emph{generators} and \emph{relators} respectively.

\begin{definition}[group presentation]\index{group presentation}
  A \emph{presentation} of a group \(G\) is an isomorphism of \(G\) with a group of the form \(\langle S|R \rangle\).

  \(G\) is \emph{finitely presented} if it admits a presentation \(\langle S|R \rangle\) with \(S, R\) finite.
\end{definition}

\begin{eg}\leavevmode
  \begin{enumerate}
  \item If \(R = \emptyset\) then \(\langle S|R \rangle \cong F(S)\).
  \item \(\langle a|a^n \rangle \cong \Z_n\).
  \item \(\langle a, b|aba^{-1}b^{-1}\rangle\) is a presentation of \(\Z^2\): let \(\Z^2 = \{(c^n, d^m): n, m \in \Z\}\). We have a homomorphism
    \begin{align*}
      \varphi: F(a, b) &\to \Z^2 \\
      a &\mapsto c \\
      b &\mapsto d
    \end{align*}
    Need to show \(\ker \varphi = \langle\langle aba^{-1}b^{-1} \rangle\rangle\). \(\supseteq\) is clear since \(\Z^2\) is abelian so have \(F(a, b)/\langle\langle aba^{-1}b^{-1} \rangle\rangle \surj F(a, b)/\ker \varphi\). The domain is a 2-generated abelian group. But the only 2-generated abelian group that surjects onto \(\Z^2\) is itself.
  \item More generally a finitely generated abelian group is always finitely presented.
  \item The same is true for nilpotent groups. Recall that \(G\) is \emph{nilpotent}\index{nilpotent} if the \emph{lower central series}\index{lower central series} of \(G\) terminates in a finite number of steps. The lower central series of \(G\) is
    \[
      G_0 = G, G_{i + 1} = [G_i, G].
    \]
  \item \(\langle a, b| aba^{-1}b^{-2}, a^{-2}b^{-1}ab \rangle = \{1\}\).
  \end{enumerate}
\end{eg}

\begin{remark}\leavevmode
  \begin{enumerate}
  \item It is difficult to tell which group is given by a particular presentation. Indeed there does not exist an algorithm that, upon input of a presentation, can determine whether the corresponding group is trivial. The is the \emph{word problem}, introduced by Dehn in early 20th century. The classes of groups for which it does have a solution are often geometry.
  \item There are uncountably many isomorphism classes of finitely generated groups (even 2-generated). For reference, see de la Harpe \emph{Geometric Group Theory} IIIB. But there are only countably many isomorphism classes of finitely presented groups.
  \end{enumerate}
\end{remark}

The notion of finite presentation makes sense without fixing a pecific surjection of a free group.

\begin{theorem}
  Given a not necessarily finite presentation \(\langle (s_j)_{j \in J} | (r_i)|_{r \in I} \rangle\) of a finitely presented group \(G\), there exists a finite subset \(J_0 \subseteq J\) and a finite set \((\tilde r_i)_{i \in I_0}\) of elements of the free group \(F((s_j)_{j \in J_0})\) such that \(\langle (s_j)_{j \in J_0} | (\tilde r_i)_{i \in I_0} \rangle\) is a finite presentation of \(G\).
\end{theorem}

\begin{proof}[``Proof'']
  de la Harpe has a proof but it seems to be wrong.
\end{proof}

Our aim is to prove that finite index subgroups of finitely generated (resp finitely presented) groups are finitely generated (resp finitely presented).

\begin{definition}[Schreier transversal]\index{Schreier transversal}
  Let \(F(S)\) be a free group and \(H \leq F(S)\) a subgroup. A (right) \emph{Schreier transversal} for \(H\) in \(F(S)\) is a set \(J\) of reduced words such that each right coset of \(H\) in \(G\) contains exactly one word of \(J\), called a \emph{representative} of this class, and all initial segments of these words are also in \(J\).

  For \(g \in F(S)\), denote by \(\overline g\) the element of \(J\) such that \(Hg = H \overline g\).
\end{definition}

\begin{theorem}
  For any \(H \leq F(S)\), there a Schreier transversal \(J\). Moreover \(H\) is freely generated by the set
  \[
    \{ts(\overline{ts})^{-1}: t \in J, s \in S \text{ and } ts(\overline{ts})^{-1} \ne 1\}.
  \]
\end{theorem}

\begin{proof}
  Take \(X = \bigvee_S S^1\) so \(\pi_1X = F(S)\). Take \(\overline X\) to be the cover corresponding to \(H \leq F(S)\). The vertices of \(\overline X\) correspond to cosets of \(H\) in \(F(S)\) and choosing a path from a fixed basepoint to a vertex gives us a coset representative for that coset. Pick a maximal spanning tree \(T \subseteq \overline X\). Choosing the unique path to each vertex in \(T\) gives us coset representatives with initial segments that are also such paths. Since \(H \cong \pi_1\overline X\) and it is freely generated by the set of loops with exactly one edge not in \(T\), this generating set is of the required form.
\end{proof}

\begin{remark}
  The argument also shows that the set of Schreier transversals for \(H\) in \(F(S)\) is in bijection with the set of maximal spanning trees in \(\overline X\).
\end{remark}

Write \(\gamma(t, s) = ts(\overline{ts})^{-1}\). Explicitly, given \(h \in H\) written as \(s_1s_2 \cdots s_n\) where \(s_i \in S \cup S^{-1}\), we can write
\[
  h = \gamma(1, s_1) \gamma(\overline s_1, s_2) \cdots \gamma(\overline{s_1 \cdots s_{i - 1}}, s_i) \cdots \gamma(\overline{s_1 \cdots s_{n - 1}}, s_n)
\]
(use \(\gamma(t, s^{-1}) = \gamma(\overline{ts^{-1}}, s)^{-1}\)).
This is the \emph{Reidemeister-Schreier rewriting process}\index{Reidemeister-Schreier rewriting process}.

\begin{theorem}
  Let \(G\) be a group with presentation \(\langle S| R\rangle\) and let \(\varphi: F(S) \to G\) correspond to this presentation. Let \(G_1 \leq G\) and let \(H\) be the subgroup of \(F(S)\) containing \(\ker \varphi\) such that \(\varphi(H) = G_1\). Then \(G_1\) has presentation
  \[
    \langle \gamma(t, s): t \in J, s \in S, \gamma(t, s) \ne 1| trt^{-1}: t \in J, r \in R \rangle
  \]
  where \(J\) is a Schreier transversal for \(H\) in \(F(S)\).
\end{theorem}

\begin{proof}
  Have \(G_1 = H/\langle\langle R\rangle\rangle^{F(S)}\) and would like to find some possibly larger set of words \(R'\) in \(H\) such that \(G_1 = H/\langle\langle R' \rangle\rangle^H\). Let \(H\) be generated freely by \(\gamma(t, s)\)'s. The subgroup \(\langle\langle R\rangle\rangle^{F(S)}\) is generated by \(\{grg^{-1}: g \in F(S), r \in R\}\), and writing each \(g\) as \(g = h_g \overline g\) where \(h_g \in H, \overline g \in J\), we have
  \[
    grg^{-1}
    = (h_g \overline g) r (h_g \overline g)^{-1}
    = h_g (\overline g r \overline g^{-1}) h_g^{-1}
  \]
  and so can take \(R' = \{trt^{-1}: t \in J, r \in R\}\). Thus \(G_1\) has the required presentation.
\end{proof}

\begin{corollary}
  Any subgroup of finite index in a finitely generated (resp. finitely presented) group is itself finitely generated (resp. finitely generated).
\end{corollary}

\begin{proof}
  If \([G: G_1] < \infty\) then \([F(S): H] < \infty\) so \(J\) is finite.
\end{proof}

\subsection{Free product}

One way to create new finitely generated/presented groups from old one is via free products. Given two groups \(A, B\), a \emph{normal form}\index{normal form} is an expression of the form \(g_1g_2 \cdots g_n\) where \(n \geq 0\) such that if \(n = 0\), take the identity element, \(g_i \in (A \setminus \{1\}) \amalg (B \setminus \{1\})\) and consecutive elements \(g_i, g_{i + 1}\) do not lie in the same group. \(n\) is the \emph{length of normal form}. We define multiplication of normal forms inductively by
\begin{itemize}
\item \((g_1 \cdots g_n) \cdot 1 = 1 \cdot (g_1 \cdots g_n) = g_1 \cdots g_n\).
\item For \(n, m \geq 1\), set
  \[
    (g_1 \cdots g_n)(h_1 \cdots h_m) =
    \begin{cases}
      g_1 \cdots g_nh_1 \cdots h_m & \text{if \(g_n, h_1\) in different groups} \\
      g_1 \cdots g_{n - 1}kh_2 \cdots h_m & \text{if \(g_n, h_1\) in same group, \(g_nh_1 = k \ne 1\)} \\
      (g_1 \cdots g_{n - 1})(h_2 \cdots h_m) & \text{if \(g_n, h_1\) in same group, \(g_nh_1 = 1\)}
    \end{cases}
  \]
\end{itemize}

\begin{definition}[free product]\index{free product}
  The set of normal forms with this multiplication forms a group \(A * B\), called the \emph{free product} of \(A\) and \(B\).
\end{definition}

\begin{remark}\leavevmode
  \begin{enumerate}
  \item The groups \(A, B\) embed naturally into \(A * B\).
  \item If \(A, B \leq G\) such that any \(g \ne 1\) in \(G\) can be represented in a unique way as a product \(g = g_1 \cdots g_n\) with \(g_i \in A \cup B \setminus \{1\}\) and consecutive \(g_i, g_{i + 1}\) not in the same group, then \(G = A * B\).
  \end{enumerate}
\end{remark}

\begin{theorem}
  If \(A = \langle S_A|R_A \rangle, B = \langle S_B|R_B \rangle\) and \(S_A \cap S_B = \emptyset\) then
  \[
    A * B = \langle S_A \cup S_B| R_A \cup R_B \rangle.
  \]
\end{theorem}

\begin{proof}
  Let \(\varphi: F(S_A) \to A, \psi: F(S_B) \to B\) be the homomorphisms with \(\ker \varphi = \langle\langle R_A \rangle\rangle^{F(S_A)}, \ker \psi = \langle\langle R_B \rangle\rangle^{F(S_B)}\). Let \(\theta: F(S_A \cup S_B) \to A * B\) be the homomorphism coinciding with \(\varphi\) on \(S_A\) and \(\psi\) on \(S_B\). Need to show \(\ker \theta = \langle\langle R_A \cup R_B \rangle\rangle^{F(S_A \cup S_B)}\). \(\supseteq\) is trivial. For \(\subseteq\), consider \(g = g_1 \cdots g_n \in \ker \theta\) in normal form (using \(F(S_A \cup S_B) = F(S_A) * F(S_B)\)). Then
  \[
    \theta(g) = \theta(g_1) \cdots \theta(g_n) = 1
  \]
  in \(A * B\). Thus exists \(i\) such that \(\theta(g_i) = 1\), so \(g_i \in \ker \varphi\) or \(g_i \in \ker \psi\). Proceed by induction.
\end{proof}

\begin{eg}
  \(D_\infty = \langle a, b| a^2 = 1, a^{-1}ba = b^{-1} \rangle\). It is the automorphism group of the graph \(C_\infty\), where \(a\) is a reflection (say about the origin) and \(b\) is a translation. Then \(D_\infty\) is generated by \(a\) and \(c = ba\), both have order \(2\). Can check (by acting on \(C_\infty\)'s vertices and edges) that \((ca)^n, (ca)^n c, a(ca)^n c, a (ca)^n\) give different elements of \(D_\infty\). By remark above
  \[
    D_\infty
    = \langle a, c| a^2, c^2\rangle
    = \langle a|a^2 \rangle * \langle c|c^2 \rangle
    = \Z_2 * \Z_2
  \]
\end{eg}

\begin{remark}
  \(\Z_2 * \Z_2\) is the only free product of non-trivial groups that does not contain a non-abelian free group. For example \(\Z_2 * \Z_3 \supseteq [\Z_2, \Z_3] \cong F_2\).
\end{remark}





\printindex
\end{document}