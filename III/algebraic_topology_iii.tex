\documentclass[a4paper]{article}

\def\npart{III}

\def\ntitle{Algebraic Topology}
\def\nlecturer{J.\ Rassmussen}

\def\nterm{Michaelmas}
\def\nyear{2019}

\ifx \nauthor\undefined
  \def\nauthor{Qiangru Kuang}
\else
\fi

\ifx \ntitle\undefined
  \def\ntitle{Template}
\else
\fi

\ifx \nauthoremail\undefined
  \def\nauthoremail{qk206@cam.ac.uk}
\else
\fi

\ifx \ndate\undefined
  \def\ndate{\today}
\else
\fi

\title{\ntitle}
\author{\nauthor}
\date{\ndate}

%\usepackage{microtype}
\usepackage{mathtools}
\usepackage{amsthm}
\usepackage{stmaryrd}%symbols used so far: \mapsfrom
\usepackage{empheq}
\usepackage{amssymb}
\let\mathbbalt\mathbb
\let\pitchforkold\pitchfork
\usepackage{unicode-math}
\let\mathbb\mathbbalt%reset to original \mathbb
\let\pitchfork\pitchforkold

\usepackage{imakeidx}
\makeindex[intoc]

%to address the problem that Latin modern doesn't have unicode support for setminus
%https://tex.stackexchange.com/a/55205/26707
\AtBeginDocument{\renewcommand*{\setminus}{\mathbin{\backslash}}}
\AtBeginDocument{\renewcommand*{\models}{\vDash}}%for \vDash is same size as \vdash but orginal \models is larger
\AtBeginDocument{\let\Re\relax}
\AtBeginDocument{\let\Im\relax}
\AtBeginDocument{\DeclareMathOperator{\Re}{Re}}
\AtBeginDocument{\DeclareMathOperator{\Im}{Im}}
\AtBeginDocument{\let\div\relax}
\AtBeginDocument{\DeclareMathOperator{\div}{div}}

\usepackage{tikz}
\usetikzlibrary{automata,positioning}
\usepackage{pgfplots}
%some preset styles
\pgfplotsset{compat=1.15}
\pgfplotsset{centre/.append style={axis x line=middle, axis y line=middle, xlabel={$x$}, ylabel={$y$}, axis equal}}
\usepackage{tikz-cd}
\usepackage{graphicx}
\usepackage{newunicodechar}

\usepackage{fancyhdr}

\fancypagestyle{mypagestyle}{
    \fancyhf{}
    \lhead{\emph{\nouppercase{\leftmark}}}
    \rhead{}
    \cfoot{\thepage}
}
\pagestyle{mypagestyle}

\usepackage{titlesec}
\newcommand{\sectionbreak}{\clearpage} % clear page after each section
\usepackage[perpage]{footmisc}
\usepackage{blindtext}

%\reallywidehat
%https://tex.stackexchange.com/a/101136/26707
\usepackage{scalerel,stackengine}
\stackMath
\newcommand\reallywidehat[1]{%
\savestack{\tmpbox}{\stretchto{%
  \scaleto{%
    \scalerel*[\widthof{\ensuremath{#1}}]{\kern-.6pt\bigwedge\kern-.6pt}%
    {\rule[-\textheight/2]{1ex}{\textheight}}%WIDTH-LIMITED BIG WEDGE
  }{\textheight}% 
}{0.5ex}}%
\stackon[1pt]{#1}{\tmpbox}%
}

%\usepackage{braket}
\usepackage{thmtools}%restate theorem
\usepackage{hyperref}

% https://en.wikibooks.org/wiki/LaTeX/Hyperlinks
\hypersetup{
    %bookmarks=true,
    unicode=true,
    pdftitle={\ntitle},
    pdfauthor={\nauthor},
    pdfsubject={Mathematics},
    pdfcreator={\nauthor},
    pdfproducer={\nauthor},
    pdfkeywords={math maths \ntitle},
    colorlinks=true,
    linkcolor={red!50!black},
    citecolor={blue!50!black},
    urlcolor={blue!80!black}
}

\usepackage{cleveref}



% TODO: mdframed often gives bad breaks that cause empty lines. Would like to switch to tcolorbox.
% The current workaround is to set innerbottommargin=0pt.

%\usepackage[theorems]{tcolorbox}





\usepackage[framemethod=tikz]{mdframed}
\mdfdefinestyle{leftbar}{
  %nobreak=true, %dirty hack
  linewidth=1.5pt,
  linecolor=gray,
  hidealllines=true,
  leftline=true,
  leftmargin=0pt,
  innerleftmargin=5pt,
  innerrightmargin=10pt,
  innertopmargin=-5pt,
  % innerbottommargin=5pt, % original
  innerbottommargin=0pt, % temporary hack 
}
%\newmdtheoremenv[style=leftbar]{theorem}{Theorem}[section]
%\newmdtheoremenv[style=leftbar]{proposition}[theorem]{proposition}
%\newmdtheoremenv[style=leftbar]{lemma}[theorem]{Lemma}
%\newmdtheoremenv[style=leftbar]{corollary}[theorem]{corollary}

\newtheorem{theorem}{Theorem}[section]
\newtheorem{proposition}[theorem]{Proposition}
\newtheorem{lemma}[theorem]{Lemma}
\newtheorem{corollary}[theorem]{Corollary}
\newtheorem{axiom}[theorem]{Axiom}
\newtheorem*{axiom*}{Axiom}

\surroundwithmdframed[style=leftbar]{theorem}
\surroundwithmdframed[style=leftbar]{proposition}
\surroundwithmdframed[style=leftbar]{lemma}
\surroundwithmdframed[style=leftbar]{corollary}
\surroundwithmdframed[style=leftbar]{axiom}
\surroundwithmdframed[style=leftbar]{axiom*}

\theoremstyle{definition}

\newtheorem*{definition}{Definition}
\surroundwithmdframed[style=leftbar]{definition}

\newtheorem*{slogan}{Slogan}
\newtheorem*{eg}{Example}
\newtheorem*{ex}{Exercise}
\newtheorem*{remark}{Remark}
\newtheorem*{notation}{Notation}
\newtheorem*{convention}{Convention}
\newtheorem*{assumption}{Assumption}
\newtheorem*{question}{Question}
\newtheorem*{answer}{Answer}
\newtheorem*{note}{Note}
\newtheorem*{application}{Application}

%operator macros

%basic
\DeclareMathOperator{\lcm}{lcm}

%matrix
\DeclareMathOperator{\tr}{tr}
\DeclareMathOperator{\Tr}{Tr}
\DeclareMathOperator{\adj}{adj}

%algebra
\DeclareMathOperator{\Hom}{Hom}
\DeclareMathOperator{\End}{End}
\DeclareMathOperator{\id}{id}
\DeclareMathOperator{\im}{im}
\DeclareMathOperator{\coker}{coker}
\DeclarePairedDelimiter{\generation}{\langle}{\rangle}

%groups
\DeclareMathOperator{\sym}{Sym}
\DeclareMathOperator{\sgn}{sgn}
\DeclareMathOperator{\inn}{Inn}
\DeclareMathOperator{\aut}{Aut}
\DeclareMathOperator{\GL}{GL}
\DeclareMathOperator{\SL}{SL}
\DeclareMathOperator{\PGL}{PGL}
\DeclareMathOperator{\PSL}{PSL}
\DeclareMathOperator{\SU}{SU}
\DeclareMathOperator{\UU}{U}
\DeclareMathOperator{\SO}{SO}
\DeclareMathOperator{\OO}{O}
\DeclareMathOperator{\PSU}{PSU}
\DeclareMathOperator{\Sp}{Sp}


%hyperbolic
\DeclareMathOperator{\sech}{sech}

%field, galois heory
\DeclareMathOperator{\ch}{ch}
\DeclareMathOperator{\gal}{Gal}
\DeclareMathOperator{\emb}{Emb}



%ceiling and floor
%https://tex.stackexchange.com/a/118217/26707
\DeclarePairedDelimiter\ceil{\lceil}{\rceil}
\DeclarePairedDelimiter\floor{\lfloor}{\rfloor}


\DeclarePairedDelimiter{\innerproduct}{\langle}{\rangle}

%\DeclarePairedDelimiterX{\norm}[1]{\lVert}{\rVert}{#1}
\DeclarePairedDelimiter{\norm}{\lVert}{\rVert}



%Dirac notation
%TODO: rewrite for variable number of arguments
\DeclarePairedDelimiterX{\braket}[2]{\langle}{\rangle}{#1 \delimsize\vert #2}
\DeclarePairedDelimiterX{\braketthree}[3]{\langle}{\rangle}{#1 \delimsize\vert #2 \delimsize\vert #3}

\DeclarePairedDelimiter{\bra}{\langle}{\rvert}
\DeclarePairedDelimiter{\ket}{\lvert}{\rangle}




%macros

%general

%divide, not divide
\newcommand*{\divides}{\mid}
\newcommand*{\ndivides}{\nmid}
%vector, i.e. mathbf
%https://tex.stackexchange.com/a/45746/26707
\newcommand*{\V}[1]{{\ensuremath{\symbf{#1}}}}
%closure
\newcommand*{\cl}[1]{\overline{#1}}
%conjugate
\newcommand*{\conj}[1]{\overline{#1}}
%set complement
\newcommand*{\stcomp}[1]{\overline{#1}}
\newcommand*{\compose}{\circ}
\newcommand*{\nto}{\nrightarrow}
\newcommand*{\p}{\partial}
%embed
\newcommand*{\embed}{\hookrightarrow}
%surjection
\newcommand*{\surj}{\twoheadrightarrow}
%power set
\newcommand*{\powerset}{\mathcal{P}}

%matrix
\newcommand*{\matrixring}{\mathcal{M}}

%groups
\newcommand*{\normal}{\trianglelefteq}
%rings
\newcommand*{\ideal}{\trianglelefteq}

%fields
\renewcommand*{\C}{{\mathbb{C}}}
\newcommand*{\R}{{\mathbb{R}}}
\newcommand*{\Q}{{\mathbb{Q}}}
\newcommand*{\Z}{{\mathbb{Z}}}
\newcommand*{\N}{{\mathbb{N}}}
\newcommand*{\F}{{\mathbb{F}}}
%not really but I think this belongs here
\newcommand*{\A}{{\mathbb{A}}}

%asymptotic
\newcommand*{\bigO}{O}
\newcommand*{\smallo}{o}

%probability
\newcommand*{\prob}{\mathbb{P}}
\newcommand*{\E}{\mathbb{E}}

%vector calculus
\newcommand*{\gradient}{\V \nabla}
\newcommand*{\divergence}{\gradient \cdot}
\newcommand*{\curl}{\gradient \cdot}

%logic
\newcommand*{\yields}{\vdash}
\newcommand*{\nyields}{\nvdash}

%differential geometry
\renewcommand*{\H}{\mathbb{H}}
\newcommand*{\transversal}{\pitchfork}
\renewcommand{\d}{\mathrm{d}} % exterior derivative

%number theory
\newcommand*{\legendre}[2]{\genfrac{(}{)}{}{}{#1}{#2}}%Legendre symbol

%algebraic geometry
\DeclareMathOperator{\Spec}{Spec}
\DeclareMathOperator{\Proj}{Proj}

\DeclareMathOperator{\Map}{Map} % space of continuous maps between spaces
\renewcommand{\b}{\p}

\begin{document}

\begin{titlepage}
  \begin{center}
    \includegraphics[width=0.6\textwidth]{logo.jpg}\par
    \vspace{1cm}
    {\scshape\huge Mathamatics Tripos \par}
    \vspace{2cm}
    {\huge Part \npart \par}
    \vspace{0.6cm}
    {\Huge \bfseries \ntitle \par}
    \vspace{1.2cm}
    {\Large\nterm, \nyear \par}
    \vspace{2cm}
    
    {\large \emph{Lectures by } \par}
    \vspace{0.2cm}
    {\Large \scshape \nlecturer}
    
    \vspace{0.5cm}
    {\large \emph{Notes by }\par}
    \vspace{0.2cm}
    {\Large \scshape \href{mailto:\nauthoremail}{\nauthor}}
 \end{center}
\end{titlepage}

\tableofcontents

\setcounter{section}{-1}

\section{Homotopy}

\begin{definition}[homotopy]\index{homotopy}
  Suppose \(X, Y\) are topological spaces, \(f_0, f_1: X \to Y\) continuous. We say \(f_0\) is \emph{homotopic} to \(f_1\) is there is a continuous \(F: X \times I \to Y\) with \(F(x, 0) = f_0(x), F(x, 1) = f_1(x)\). We write \(f_0 \sim f_1\).
\end{definition}

Let \(f_t(x) = F(x, t)\). Then \(f_t\) is a path from \(f_0\) to \(f_1\) in \(\Map(X, Y) = \{f: X \to Y \text{ continuous}\}\).

\begin{convention}
  All spaces are topological spaces and all maps are continous.
\end{convention}

\begin{eg}\leavevmode
  \begin{enumerate}
  \item Let \(f_0, f_1: \R^n \to \R^n\), \(f_0(x) = 0, f_1(x) = x\) then \(f_0 \sim f_1\) via \(f_t(x) = tx\).
  \item Let \(S^1 = \{z \in \C: |z| = 1\}\). Take \(f_0, f_1: S^1 \to S^1\), \(f_0(z) = z, f_1(z) = -z\). Then \(f_0 \sim f_1\) via \(f_t(z) = e^{i\pi t}z\).
  \item Let \(S^n = \{v \in \R^{n + 1}: \norm v = 1\}\). Take \(f_0, f_1: S^n \to S^n\), \(f_0(v) = v, f_1(v) = -v\) the \emph{antipodal map}. We already knew \(f_0 \sim f_1\) and we'll soon see \(f_0 \nsim f_1\).
  \item Let \(f_0, f_1: S^1 \to S^2\), \(f_0(x, y) = (0, 0, 1), f_1(x, y) = (x, y, 0)\). Then \(f_0 \sim f_1\) via \(f_t(x, y) = (tx, ty, \sqrt{1 - t^2})\).
  \item Let \(D^n = \{v \in \R^n: \norm v \leq 1\}\). Say \(f: S^{n - 1} \to Y\) extends to \(D^n\) if there exists \(F: D^n \to Y\) with \(F|_{S^{n - 1}} = f\). Then \(f\) extends to \(D^n\) if and only if \(f\) is homotopic to a constant map as we can define \(f_t(v) = F(tv)\).
  \end{enumerate}
\end{eg}

We state here some lemmas that will be assumed and whose proofs are omitted.

\begin{lemma}
  Homotopy is an equivalence relation on \(\Map(X, Y)\).
\end{lemma}

\begin{definition}
  We let \([X, Y]\) to be \(\Map(X, Y)/\sim\), i.e.\ the set of homotopy classes of maps \(X \to Y\). It is also the set of path components of \(\Map(X, Y)\). We write \([f]\) for the class of \(f\) in \([X, Y]\).
\end{definition}

\begin{lemma}
  Suppose \(f_0, f_1: X \to Y, g_0, g_1 : Y \to Z\). If \(f_0 \sim f_1, g_0 \sim g_1\) then \(g_0 \compose f_0 \sim g_1 \compose f_1\).
\end{lemma}

\begin{notation}
  If \(c \in Y\), we denote by \(c_X: X \to Y\) the constant map with image \(c\).
\end{notation}

\begin{corollary}
  Any \(f: X \to \R^n\) is homotopic to \(0_X\).
\end{corollary}

In other words, \([X, \R^n]\) has one element.

\begin{proof}
  We know \(\id_{\R^n} \sim 0_{\R^n}\) so
  \[
    f = \id_{\R^n} \compose f \sim 0_{\R^n} \compose f = 0_X.
  \]
\end{proof}

\begin{definition}[contractible]\index{contractible}
  \(X\) is \emph{contractible} if \(\id_X \sim c_X\) for some \(c \in X\).
\end{definition}

\begin{proposition}
  \(Y\) is contractible if and only if \([X, Y]\) has one element for all \(Y\).
\end{proposition}

\begin{proof}
  Only if is the same as the proof of the corollary. For the other direction, \([Y, Y]\) has one element so \(\id_Y \sim c_Y\) for any \(c \in Y\).
\end{proof}

\begin{definition}[homotopy equivalence]\index{homotopy equivalence}
  Spaces \(X\) and \(Y\) are \emph{homotopy equivalence} if there are maps \(f: X \to Y, g: Y \to X\) such that \(f \compose g \sim \id_Y, g \compose f \sim \id_X\). We write \(X \sim Y\).
\end{definition}

\begin{eg}
  \(X \sim \{p\}\) if and only if \(X\) is contractible.
\end{eg}

\begin{proof}
  The only map \(f: X \to \{p\}\) is \(f(x) = p\). Let \(g: \{p\} \to X, g(p) = c\). Then \(f \compose g = \id_{\{p\}}\) and \(g \compose f = c_X\). Then \(g \compose f \sim \id_X\) if and only if \(c_X \sim \id_X\) if and only if \(X\) is contractible.
\end{proof}

\begin{lemma}
  If \(X_1 \sim X_2, Y_1 \sim Y_2\) then there is a bijection between \([X_1, Y_1]\) and \([X_2, Y_2]\).
\end{lemma}

The basic questions that algebraic topology tries to answer is the follow: given spaces \(X\) and \(Y\), is \(X \sim Y\)? What is \([X, Y]\)?

One of the tools used is homotopy groups, which we mention briefly here.

\begin{definition}[map of pairs]\index{map of pairs}
  A map \(f: (X, A) \to (Y, B)\) means that
  \begin{itemize}
  \item \(A \subseteq X, B \subseteq Y\),
  \item \(f: X \to Y\),
  \item \(f(A) \subseteq B\).
  \end{itemize}
\end{definition}

If \(f_0, f_1: (X, A) \to (Y, B)\), we say \(f_0 \sim f_1\) is there exist \(F: (X \times I, A \times I) \to (Y, B)\) with \(F(x, 0) = f_0(x), F(x, 1) = f_1(x)\).

\begin{notation}
  We denote by \(*\) the point \((-1, 0, \dots, 0) \in S^n\).
\end{notation}

\begin{definition}[homotopy group]\index{homotopy group}
  If \(p \in X\), we define the \emph{\(n\)th homotopy group} of \((X, p)\) to be
  \[
    \pi_n(X, p) = [(S^n, *), (X, p)] = [(D^n, S^{n - 1}), (X, p)] = [(I^n, \b I^n), (X, p)]
  \]
  where the last equality is a homeomorphism and the second equality is induced by
  \begin{align*}
    \pi: D^n &\to D^n/S^{n -1} = S^n \\
    v &\mapsto (1 - 2 \norm v, v \sqrt{1 - (1 - 2\norm v)^2})
  \end{align*}
\end{definition}

For \(n > 0\), \(\pi_n(X, p)\) is a group. The identity is \(p_{S^n}\). For \(n > 1\), \(\pi_n(X, p)\) is abelian.

A pointed map between pointed spaces \(f: (X, p) \to (Y, q)\) induces
\begin{align*}
  f_*: \pi_n(X, p) &\to \pi_n(Y, q) \\
  [\gamma] &\mapsto [f \compose \gamma]
\end{align*}
which is well-defined by lemma 2.

This defines a functor between the cateogry of pointed spaces with pointed maps to the category of groups with homomorphisms: it sends a space \((X, p)\) to \(\pi_n(X, p)\) and a map \(f: (X, p) \to (Y, q)\) to the homomorphism \(f_*: \pi_n(X, p) \to \pi_n(Y, q)\), satisfying
\begin{enumerate}
\item \((\id_{(X, p)})_* = \id_{\pi_n(X, p)}\),
\item \((f \compose g)_* = f_* \compose g_*\).
\end{enumerate}





\printindex
\end{document}
