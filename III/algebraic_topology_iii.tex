\documentclass[a4paper]{article}

\def\npart{III}

\def\ntitle{Algebraic Topology}
\def\nlecturer{J.\ Rasmussen}

\def\nterm{Michaelmas}
\def\nyear{2019}

\ifx \nauthor\undefined
  \def\nauthor{Qiangru Kuang}
\else
\fi

\ifx \ntitle\undefined
  \def\ntitle{Template}
\else
\fi

\ifx \nauthoremail\undefined
  \def\nauthoremail{qk206@cam.ac.uk}
\else
\fi

\ifx \ndate\undefined
  \def\ndate{\today}
\else
\fi

\title{\ntitle}
\author{\nauthor}
\date{\ndate}

%\usepackage{microtype}
\usepackage{mathtools}
\usepackage{amsthm}
\usepackage{stmaryrd}%symbols used so far: \mapsfrom
\usepackage{empheq}
\usepackage{amssymb}
\let\mathbbalt\mathbb
\let\pitchforkold\pitchfork
\usepackage{unicode-math}
\let\mathbb\mathbbalt%reset to original \mathbb
\let\pitchfork\pitchforkold

\usepackage{imakeidx}
\makeindex[intoc]

%to address the problem that Latin modern doesn't have unicode support for setminus
%https://tex.stackexchange.com/a/55205/26707
\AtBeginDocument{\renewcommand*{\setminus}{\mathbin{\backslash}}}
\AtBeginDocument{\renewcommand*{\models}{\vDash}}%for \vDash is same size as \vdash but orginal \models is larger
\AtBeginDocument{\let\Re\relax}
\AtBeginDocument{\let\Im\relax}
\AtBeginDocument{\DeclareMathOperator{\Re}{Re}}
\AtBeginDocument{\DeclareMathOperator{\Im}{Im}}
\AtBeginDocument{\let\div\relax}
\AtBeginDocument{\DeclareMathOperator{\div}{div}}

\usepackage{tikz}
\usetikzlibrary{automata,positioning}
\usepackage{pgfplots}
%some preset styles
\pgfplotsset{compat=1.15}
\pgfplotsset{centre/.append style={axis x line=middle, axis y line=middle, xlabel={$x$}, ylabel={$y$}, axis equal}}
\usepackage{tikz-cd}
\usepackage{graphicx}
\usepackage{newunicodechar}

\usepackage{fancyhdr}

\fancypagestyle{mypagestyle}{
    \fancyhf{}
    \lhead{\emph{\nouppercase{\leftmark}}}
    \rhead{}
    \cfoot{\thepage}
}
\pagestyle{mypagestyle}

\usepackage{titlesec}
\newcommand{\sectionbreak}{\clearpage} % clear page after each section
\usepackage[perpage]{footmisc}
\usepackage{blindtext}

%\reallywidehat
%https://tex.stackexchange.com/a/101136/26707
\usepackage{scalerel,stackengine}
\stackMath
\newcommand\reallywidehat[1]{%
\savestack{\tmpbox}{\stretchto{%
  \scaleto{%
    \scalerel*[\widthof{\ensuremath{#1}}]{\kern-.6pt\bigwedge\kern-.6pt}%
    {\rule[-\textheight/2]{1ex}{\textheight}}%WIDTH-LIMITED BIG WEDGE
  }{\textheight}% 
}{0.5ex}}%
\stackon[1pt]{#1}{\tmpbox}%
}

%\usepackage{braket}
\usepackage{thmtools}%restate theorem
\usepackage{hyperref}

% https://en.wikibooks.org/wiki/LaTeX/Hyperlinks
\hypersetup{
    %bookmarks=true,
    unicode=true,
    pdftitle={\ntitle},
    pdfauthor={\nauthor},
    pdfsubject={Mathematics},
    pdfcreator={\nauthor},
    pdfproducer={\nauthor},
    pdfkeywords={math maths \ntitle},
    colorlinks=true,
    linkcolor={red!50!black},
    citecolor={blue!50!black},
    urlcolor={blue!80!black}
}

\usepackage{cleveref}



% TODO: mdframed often gives bad breaks that cause empty lines. Would like to switch to tcolorbox.
% The current workaround is to set innerbottommargin=0pt.

%\usepackage[theorems]{tcolorbox}





\usepackage[framemethod=tikz]{mdframed}
\mdfdefinestyle{leftbar}{
  %nobreak=true, %dirty hack
  linewidth=1.5pt,
  linecolor=gray,
  hidealllines=true,
  leftline=true,
  leftmargin=0pt,
  innerleftmargin=5pt,
  innerrightmargin=10pt,
  innertopmargin=-5pt,
  % innerbottommargin=5pt, % original
  innerbottommargin=0pt, % temporary hack 
}
%\newmdtheoremenv[style=leftbar]{theorem}{Theorem}[section]
%\newmdtheoremenv[style=leftbar]{proposition}[theorem]{proposition}
%\newmdtheoremenv[style=leftbar]{lemma}[theorem]{Lemma}
%\newmdtheoremenv[style=leftbar]{corollary}[theorem]{corollary}

\newtheorem{theorem}{Theorem}[section]
\newtheorem{proposition}[theorem]{Proposition}
\newtheorem{lemma}[theorem]{Lemma}
\newtheorem{corollary}[theorem]{Corollary}
\newtheorem{axiom}[theorem]{Axiom}
\newtheorem*{axiom*}{Axiom}

\surroundwithmdframed[style=leftbar]{theorem}
\surroundwithmdframed[style=leftbar]{proposition}
\surroundwithmdframed[style=leftbar]{lemma}
\surroundwithmdframed[style=leftbar]{corollary}
\surroundwithmdframed[style=leftbar]{axiom}
\surroundwithmdframed[style=leftbar]{axiom*}

\theoremstyle{definition}

\newtheorem*{definition}{Definition}
\surroundwithmdframed[style=leftbar]{definition}

\newtheorem*{slogan}{Slogan}
\newtheorem*{eg}{Example}
\newtheorem*{ex}{Exercise}
\newtheorem*{remark}{Remark}
\newtheorem*{notation}{Notation}
\newtheorem*{convention}{Convention}
\newtheorem*{assumption}{Assumption}
\newtheorem*{question}{Question}
\newtheorem*{answer}{Answer}
\newtheorem*{note}{Note}
\newtheorem*{application}{Application}

%operator macros

%basic
\DeclareMathOperator{\lcm}{lcm}

%matrix
\DeclareMathOperator{\tr}{tr}
\DeclareMathOperator{\Tr}{Tr}
\DeclareMathOperator{\adj}{adj}

%algebra
\DeclareMathOperator{\Hom}{Hom}
\DeclareMathOperator{\End}{End}
\DeclareMathOperator{\id}{id}
\DeclareMathOperator{\im}{im}
\DeclareMathOperator{\coker}{coker}
\DeclarePairedDelimiter{\generation}{\langle}{\rangle}

%groups
\DeclareMathOperator{\sym}{Sym}
\DeclareMathOperator{\sgn}{sgn}
\DeclareMathOperator{\inn}{Inn}
\DeclareMathOperator{\aut}{Aut}
\DeclareMathOperator{\GL}{GL}
\DeclareMathOperator{\SL}{SL}
\DeclareMathOperator{\PGL}{PGL}
\DeclareMathOperator{\PSL}{PSL}
\DeclareMathOperator{\SU}{SU}
\DeclareMathOperator{\UU}{U}
\DeclareMathOperator{\SO}{SO}
\DeclareMathOperator{\OO}{O}
\DeclareMathOperator{\PSU}{PSU}
\DeclareMathOperator{\Sp}{Sp}


%hyperbolic
\DeclareMathOperator{\sech}{sech}

%field, galois heory
\DeclareMathOperator{\ch}{ch}
\DeclareMathOperator{\gal}{Gal}
\DeclareMathOperator{\emb}{Emb}



%ceiling and floor
%https://tex.stackexchange.com/a/118217/26707
\DeclarePairedDelimiter\ceil{\lceil}{\rceil}
\DeclarePairedDelimiter\floor{\lfloor}{\rfloor}


\DeclarePairedDelimiter{\innerproduct}{\langle}{\rangle}

%\DeclarePairedDelimiterX{\norm}[1]{\lVert}{\rVert}{#1}
\DeclarePairedDelimiter{\norm}{\lVert}{\rVert}



%Dirac notation
%TODO: rewrite for variable number of arguments
\DeclarePairedDelimiterX{\braket}[2]{\langle}{\rangle}{#1 \delimsize\vert #2}
\DeclarePairedDelimiterX{\braketthree}[3]{\langle}{\rangle}{#1 \delimsize\vert #2 \delimsize\vert #3}

\DeclarePairedDelimiter{\bra}{\langle}{\rvert}
\DeclarePairedDelimiter{\ket}{\lvert}{\rangle}




%macros

%general

%divide, not divide
\newcommand*{\divides}{\mid}
\newcommand*{\ndivides}{\nmid}
%vector, i.e. mathbf
%https://tex.stackexchange.com/a/45746/26707
\newcommand*{\V}[1]{{\ensuremath{\symbf{#1}}}}
%closure
\newcommand*{\cl}[1]{\overline{#1}}
%conjugate
\newcommand*{\conj}[1]{\overline{#1}}
%set complement
\newcommand*{\stcomp}[1]{\overline{#1}}
\newcommand*{\compose}{\circ}
\newcommand*{\nto}{\nrightarrow}
\newcommand*{\p}{\partial}
%embed
\newcommand*{\embed}{\hookrightarrow}
%surjection
\newcommand*{\surj}{\twoheadrightarrow}
%power set
\newcommand*{\powerset}{\mathcal{P}}

%matrix
\newcommand*{\matrixring}{\mathcal{M}}

%groups
\newcommand*{\normal}{\trianglelefteq}
%rings
\newcommand*{\ideal}{\trianglelefteq}

%fields
\renewcommand*{\C}{{\mathbb{C}}}
\newcommand*{\R}{{\mathbb{R}}}
\newcommand*{\Q}{{\mathbb{Q}}}
\newcommand*{\Z}{{\mathbb{Z}}}
\newcommand*{\N}{{\mathbb{N}}}
\newcommand*{\F}{{\mathbb{F}}}
%not really but I think this belongs here
\newcommand*{\A}{{\mathbb{A}}}

%asymptotic
\newcommand*{\bigO}{O}
\newcommand*{\smallo}{o}

%probability
\newcommand*{\prob}{\mathbb{P}}
\newcommand*{\E}{\mathbb{E}}

%vector calculus
\newcommand*{\gradient}{\V \nabla}
\newcommand*{\divergence}{\gradient \cdot}
\newcommand*{\curl}{\gradient \cdot}

%logic
\newcommand*{\yields}{\vdash}
\newcommand*{\nyields}{\nvdash}

%differential geometry
\renewcommand*{\H}{\mathbb{H}}
\newcommand*{\transversal}{\pitchfork}
\renewcommand{\d}{\mathrm{d}} % exterior derivative

%number theory
\newcommand*{\legendre}[2]{\genfrac{(}{)}{}{}{#1}{#2}}%Legendre symbol

%algebraic geometry
\DeclareMathOperator{\Spec}{Spec}
\DeclareMathOperator{\Proj}{Proj}

\graphicspath{{figures/}}

\DeclareMathOperator{\Map}{Map} % space of continuous maps between spaces
\renewcommand{\b}{\p}

\begin{document}

\begin{titlepage}
  \begin{center}
    \includegraphics[width=0.6\textwidth]{logo.jpg}\par
    \vspace{1cm}
    {\scshape\huge Mathamatics Tripos \par}
    \vspace{2cm}
    {\huge Part \npart \par}
    \vspace{0.6cm}
    {\Huge \bfseries \ntitle \par}
    \vspace{1.2cm}
    {\Large\nterm, \nyear \par}
    \vspace{2cm}
    
    {\large \emph{Lectures by } \par}
    \vspace{0.2cm}
    {\Large \scshape \nlecturer}
    
    \vspace{0.5cm}
    {\large \emph{Notes by }\par}
    \vspace{0.2cm}
    {\Large \scshape \href{mailto:\nauthoremail}{\nauthor}}
 \end{center}
\end{titlepage}

\tableofcontents

\setcounter{section}{-1}

\section{Homotopy}

\begin{definition}[homotopy]\index{homotopy}
  Suppose \(X, Y\) are topological spaces, \(f_0, f_1: X \to Y\) continuous. We say \(f_0\) is \emph{homotopic} to \(f_1\) is there is a continuous \(F: X \times I \to Y\) with \(F(x, 0) = f_0(x), F(x, 1) = f_1(x)\). We write \(f_0 \sim f_1\).
\end{definition}

Let \(f_t(x) = F(x, t)\). Then \(f_t\) is a path from \(f_0\) to \(f_1\) in \(\Map(X, Y) = \{f: X \to Y \text{ continuous}\}\).

\begin{convention}
  All spaces are topological spaces and all maps are continous.
\end{convention}

\begin{eg}\leavevmode
  \begin{enumerate}
  \item Let \(f_0, f_1: \R^n \to \R^n\), \(f_0(x) = 0, f_1(x) = x\) then \(f_0 \sim f_1\) via \(f_t(x) = tx\).
  \item Let \(S^1 = \{z \in \C: |z| = 1\}\). Take \(f_0, f_1: S^1 \to S^1\), \(f_0(z) = z, f_1(z) = -z\). Then \(f_0 \sim f_1\) via \(f_t(z) = e^{i\pi t}z\).
  \item Let \(S^n = \{v \in \R^{n + 1}: \norm v = 1\}\). Take \(f_0, f_1: S^n \to S^n\), \(f_0(v) = v, f_1(v) = -v\) the \emph{antipodal map}. We already knew \(f_0 \sim f_1\) and we'll soon see \(f_0 \nsim f_1\).
  \item Let \(f_0, f_1: S^1 \to S^2\), \(f_0(x, y) = (0, 0, 1), f_1(x, y) = (x, y, 0)\). Then \(f_0 \sim f_1\) via \(f_t(x, y) = (tx, ty, \sqrt{1 - t^2})\).
  \item Let \(D^n = \{v \in \R^n: \norm v \leq 1\}\). Say \(f: S^{n - 1} \to Y\) extends to \(D^n\) if there exists \(F: D^n \to Y\) with \(F|_{S^{n - 1}} = f\). Then \(f\) extends to \(D^n\) if and only if \(f\) is homotopic to a constant map as we can define \(f_t(v) = F(tv)\).
  \end{enumerate}
\end{eg}

We state here some lemmas that will be assumed and whose proofs are omitted.

\begin{lemma}
  Homotopy is an equivalence relation on \(\Map(X, Y)\).
\end{lemma}

\begin{definition}
  We let \([X, Y]\) to be \(\Map(X, Y)/\sim\), i.e.\ the set of homotopy classes of maps \(X \to Y\). It is also the set of path components of \(\Map(X, Y)\). We write \([f]\) for the class of \(f\) in \([X, Y]\).
\end{definition}

\begin{lemma}
  Suppose \(f_0, f_1: X \to Y, g_0, g_1 : Y \to Z\). If \(f_0 \sim f_1, g_0 \sim g_1\) then \(g_0 \compose f_0 \sim g_1 \compose f_1\).
\end{lemma}

\begin{notation}
  If \(c \in Y\), we denote by \(c_X: X \to Y\) the constant map with image \(c\).
\end{notation}

\begin{corollary}
  Any \(f: X \to \R^n\) is homotopic to \(0_X\).
\end{corollary}

In other words, \([X, \R^n]\) has one element.

\begin{proof}
  We know \(\id_{\R^n} \sim 0_{\R^n}\) so
  \[
    f = \id_{\R^n} \compose f \sim 0_{\R^n} \compose f = 0_X.
  \]
\end{proof}

\begin{definition}[contractible]\index{contractible}
  \(X\) is \emph{contractible} if \(\id_X \sim c_X\) for some \(c \in X\).
\end{definition}

\begin{proposition}
  \(Y\) is contractible if and only if \([X, Y]\) has one element for all \(Y\).
\end{proposition}

\begin{proof}
  Only if is the same as the proof of the corollary. For the other direction, \([Y, Y]\) has one element so \(\id_Y \sim c_Y\) for any \(c \in Y\).
\end{proof}

\begin{definition}[homotopy equivalence]\index{homotopy equivalence}
  Spaces \(X\) and \(Y\) are \emph{homotopy equivalence} if there are maps \(f: X \to Y, g: Y \to X\) such that \(f \compose g \sim \id_Y, g \compose f \sim \id_X\). We write \(X \sim Y\).
\end{definition}

\begin{eg}
  \(X \sim \{p\}\) if and only if \(X\) is contractible.
\end{eg}

\begin{proof}
  The only map \(f: X \to \{p\}\) is \(f(x) = p\). Let \(g: \{p\} \to X, g(p) = c\). Then \(f \compose g = \id_{\{p\}}\) and \(g \compose f = c_X\). Then \(g \compose f \sim \id_X\) if and only if \(c_X \sim \id_X\) if and only if \(X\) is contractible.
\end{proof}

\begin{lemma}
  If \(X_1 \sim X_2, Y_1 \sim Y_2\) then there is a bijection between \([X_1, Y_1]\) and \([X_2, Y_2]\).
\end{lemma}

The basic questions that algebraic topology tries to answer is the follow: given spaces \(X\) and \(Y\), is \(X \sim Y\)? What is \([X, Y]\)?

One of the tools used is homotopy groups, which we mention briefly here.

\begin{definition}[map of pairs]\index{map of pairs}
  A map \(f: (X, A) \to (Y, B)\) means that
  \begin{itemize}
  \item \(A \subseteq X, B \subseteq Y\),
  \item \(f: X \to Y\),
  \item \(f(A) \subseteq B\).
  \end{itemize}
\end{definition}

If \(f_0, f_1: (X, A) \to (Y, B)\), we say \(f_0 \sim f_1\) is there exist \(F: (X \times I, A \times I) \to (Y, B)\) with \(F(x, 0) = f_0(x), F(x, 1) = f_1(x)\).

\begin{notation}
  We denote by \(*\) the point \((-1, 0, \dots, 0) \in S^n\).
\end{notation}

\begin{definition}[homotopy group]\index{homotopy group}
  If \(p \in X\), we define the \emph{\(n\)th homotopy group} of \((X, p)\) to be
  \[
    \pi_n(X, p) = [(S^n, *), (X, p)] = [(D^n, S^{n - 1}), (X, p)] = [(I^n, \b I^n), (X, p)]
  \]
  where the last equality is a homeomorphism and the second equality is induced by
  \begin{align*}
    \pi: D^n &\to D^n/S^{n -1} = S^n \\
    v &\mapsto (1 - 2 \norm v, v \sqrt{1 - (1 - 2\norm v)^2})
  \end{align*}
\end{definition}

For \(n > 0\), \(\pi_n(X, p)\) is a group. The identity is \(p_{S^n}\). For \(n > 1\), \(\pi_n(X, p)\) is abelian.

A pointed map between pointed spaces \(f: (X, p) \to (Y, q)\) induces
\begin{align*}
  f_*: \pi_n(X, p) &\to \pi_n(Y, q) \\
  [\gamma] &\mapsto [f \compose \gamma]
\end{align*}
which is well-defined by lemma 2.

This defines a functor between the cateogry of pointed spaces with pointed maps to the category of groups with homomorphisms: it sends a space \((X, p)\) to \(\pi_n(X, p)\) and a map \(f: (X, p) \to (Y, q)\) to the homomorphism \(f_*: \pi_n(X, p) \to \pi_n(Y, q)\), satisfying
\begin{enumerate}
\item \((\id_{(X, p)})_* = \id_{\pi_n(X, p)}\),
\item \((f \compose g)_* = f_* \compose g_*\).
\end{enumerate}
Furthermore \(f_*\) is homotopy invariant: if \(f \sim g\) then \(f_* = g_*\) since
\[
  f_*([\gamma]) = [f \compose \gamma] = [g \compose \gamma] = g_*([\gamma]).
\]

\begin{eg}
  \begin{table}[ht]
    \centering
    \begin{tabular}{c|c|c|c|c|c|c|c|c|c}
      & 1 & 2 & 3 & 4 & 5 \\ \hline
      \(\pi_n(S^1)\) & \(\Z\) &&&&&& \\ \hline
      \(\pi_n(S^2)\) & \(0\) & \(\Z\) & \(\Z\) & \(\Z/2\) & \(\Z/2\) & \(\Z/12\) &\(\Z/2\)
    \end{tabular}
    \caption{Homotopy groups of \(S^n\)}
  \end{table}
\end{eg}

\section{Homology}

The goal is to define functors \(H_n\) from the category of spaces with continuous maps to the category of abelian groups with homomorphisms, satifying
\begin{enumerate}
\item if \(f \sim g\) then \(f_* = g_8\),
\item dimension axiom: informally \(H_n(X) = 0\) if \(n > \dim X\).
\end{enumerate}

\subsection{Chain complexes}

Let \(R\) be a commutative ring (e.g.\ \(\Z, \Q, \Z/p\)).

\begin{definition}[chain complex]\index{chain complex}
  A \emph{chain complex} \((C_*, d)\) over \(R\) is
  \begin{enumerate}
  \item \(R\)-modules \(C_i\) for \(i \in \Z\), and
  \item homomorphisms \(d_i: C_i \to C_{i - 1}\) such that
  \item \(d_i \compose d_{i + 1} = 0\) for all \(i\).
  \end{enumerate}
  We usually write
  \[
    \begin{tikzcd}
      \cdots \ar[r] & C_{i + 1} \ar[r, "d_{i + 1}"] & C_i \ar[r, "d_i"] & C_{i - 1} \ar[r] & \cdots
    \end{tikzcd}
  \]
\end{definition}

\begin{notation}
  Note that \(C_*\) can mean two different things: it can either mean
  \[
    C_* =
    \begin{cases}
      \Z & * = 0 \\
      0 & \text{otherwise}
    \end{cases}
  \]
  or \(C_* = \bigoplus_{i \in \Z} C_i, d = \sum d_i: C_* \to C_{* - 1}\).
\end{notation}

\subsubsection{Chain complex of a simplex}

\begin{definition}[simplex]\index{simplex}
  The \emph{\(n\)-dimensional simplex} is
  \[
    \Delta^n = \{(v_0, \dots, v_n) \in \R^{n + 1}: v_i \geq 0, \sum_{i = 0}^n v_i = 1\}.
  \]
  For \(n < 0\) we set \(\Delta^n = \emptyset\).
\end{definition}

\begin{definition}[face]
  If \(I = \{i_0 < i_1 < \dots < i_k\} \subseteq \{0, 1, \dots n\}\) then
  \[
    f_I = \{v \in \Delta^n: v_i = 0 \text{ if } i \notin I\}
  \]
  is a \(k\)-dimensional \emph{face} of \(\Delta^n\). The \emph{face map} is
  \begin{align*}
    F_I: \Delta^k &\to f_I \\
    w &\mapsto v
  \end{align*}
  where
  \[
    v_i =
    \begin{cases}
      0 & i \notin I \\
      w_j & i = \varphi(j)
    \end{cases}
  \]
  where
  \begin{align*}
    \varphi: \{0, \dots, k\} &\to I \\
    j &\mapsto i_j
  \end{align*}
\end{definition}

\begin{definition}
  The \emph{reduced chain complex} of the simplex \(\Delta^n\) \(\widetilde S_*(\Delta^n)\) is the chain complex over \(\Z\) defined by
  \[
    \widetilde S_k(\Delta^n) = \langle f_I: |I| = k + 1 \rangle,
  \]
  the free abelian group with basis \(f_I\) for \(I\) a \(k\)-dimensional face, and
  \begin{align*}
    d_k: \widetilde S_k(\Delta^n) &\to \widetilde S_{k - 1}(\Delta^n) \\
    f_I &\mapsto \sum_{j = 0}^k (-1)^j f_{I \setminus \{i_j\}}
  \end{align*}
\end{definition}

\begin{eg}
  Take \(n = 2\). Then
  \begin{align*}
    C_2 &= \langle f_{012} \rangle \\
    C_1 &= \langle f_{01}, f_{02}, f_{12} \rangle \\
    C_0 &= \langle f_0, f_1, f_2 \rangle \\
    C_{-1} &= \langle f_\emptyset \rangle
  \end{align*}
  and for example we have
  \begin{align*}
    d(f_{012}) &= f_{12} - f_{02} + f_{01} \\
    d(f_{12}) &= f_2 - f_1 \\
    d(f_{02}) &= f_2 - f_0 \\
    d(f_{01}) &= f_1 - f_0
  \end{align*}
  so
  \[
    d^2(f_{012}) = 0.
  \]
\end{eg}

\begin{proposition}
  We have
  \[
    d^2 = 0
  \]
  so it is indeed a chain complex.
\end{proposition}

\begin{proof}
  Enough to check \(d^2(f_I) = 0\). \(d^2(f_I)\) is a sum of terms of the form \(f_{I \setminus \{i_j, i_{j'}\}}\) where \(i_j < i_{j'}\). The coefficient of \(f_{I \setminus \{i_j, i_{j'}\}}\) is
  \[
    (-1)^j (-1)^{j' - 1} + (-1)^{j'} (-1)^{j}
  \]
  where the first term is by omitting \(i_j\) first and then \(i_{j'}\), and the second by omitting \(i_{j'}\) first and then \(i_j\). Then have opposite signs.
\end{proof}

Note that if we have a chain complex then \(d^2 = 0\) so \(\im d_{i + 1} \subseteq \ker d_i\).

\begin{definition}[homology group]\index{homology group}
  If \((C_*, d)\) is a chain complex, its \emph{\(i\)th homology group} is
  \[
    H_i(C_*) = \frac{\ker d_i}{\im d_{i + 1}}.
  \]
  We let
  \[
    H_*(C_*) = \bigoplus_{i \in \Z} H_i = \frac{\ker d}{\im d}.
  \]
\end{definition}

\begin{eg}
  \(H_*(\widetilde S_*(\Delta^2)) = 0\).
\end{eg}

\begin{eg}[unreduced complex of a simplex]
  Define the unreduced complex to be
  \[
    S_*(\Delta^n) =
    \begin{cases}
      \widetilde S_k(\Delta^n) & k \geq 0 \\
      0 & k < 0
    \end{cases}
  \]
  Check that
  \[
    H_*(S(\Delta^2)) =
    \begin{cases}
      \Z & k = 0 \\
      0 & k \neq 0
    \end{cases}
  \]
\end{eg}

\begin{definition}[chain maps]\index{chain maps}
  If \((C, d)\) and \((C', d')\) are chain complexes over \(R\), a \emph{chain map} \(f: (C, d) \to (C', d')\) is homomorphisms \(f_i: C_i \to C_i'\) such that
  \[
    \begin{tikzcd}
      \cdots \ar[r] & \ar[r] C_{i + 1} \ar[r, "d_{i + 1}"] \ar[d, "f_{i + 1}"] & C_i \ar[r, "d_i"] \ar[d, "f_i"] & C_{i - 1} \ar[r] \ar[d, "f_{i - 1}"] & \cdots \\
      \cdots \ar[r] & \ar[r] C'_{i + 1} \ar[r, "d'_{i + 1}"] & C'_i \ar[r, "d'_i"] & C'_{i - 1} \ar[r] & \cdots
    \end{tikzcd}
  \]
  commutes. That is to say let \(f = \sum f_i: C_* \to C'_*\) then we have
  \[
    d'f = fd.
  \]
\end{definition}

\begin{eg}
  If \(f_I\) is a \(k\)-dim face of \(\Delta^n\) then there is a chain map
  \begin{align*}
    \varphi_I: \widetilde S_*(\Delta^k) &\to \widetilde S_*(\Delta^n) \\
    f_J &\mapsto f_{\varphi(J)}
  \end{align*}
  where \(\varphi(j) = i_j\) as before.
\end{eg}

If \(f: (C, d) \to (C', d')\) is a chain map then it follows that \(f(\ker d) \subseteq \ker d', f(\im d) \subseteq \im d'\), so there is a well-defined map
\begin{align*}
  f_*: H_*(C) &\to H_*(C') \\
  [z] &\mapsto [f(z)]
\end{align*}

\begin{lemma}\leavevmode
  \begin{enumerate}
  \item \(\id_C\) is a chain map and \((\id_C)_* = \id_{H_*(C)}\).
  \item If \(f: C \to C', g: C' \to C''\) are chain maps then so is \(g \compose f\) and \((g \compose f)_* = g_* \compose f_*\).
  \end{enumerate}
  In other words, there is a functor \(H_*\) from the category of chain complexes over \(R\) with chain maps to the category of \(R\)-modules.
\end{lemma}

\subsection{Singular chain complex}

Let \(X\) be a topological space. A \emph{singular \(k\)-simplex} in \(X\) is a map \(\sigma: \Delta^k \to X\). Thus a singular \(0\)-simplex is a point in \(X\) and a \(1\)-simplex is a curve in \(X\).

\begin{definition}[singular chain complex]\index{singular chain complex}
  A \emph{singular chain complex} \(C_*(X)\) is given by
  \[
    C_k(X) = \langle \sigma: \Delta^k \to X \text{ continuous} \rangle,
  \]
  the free abelian group generated by \(\sigma\)'s and for \(\sigma: \Delta^k \to X\),
  \[
    d(\sigma) = \sum_{j = 0}^k (-1)^j \sigma \compose F_{\{0, \dots, k\} \setminus \{j\}}.
  \]
\end{definition}

Elements of the chain groups are finite sums \(\sum_{i = 1}^N a_i \sigma_i\) where \(a_i \in \Z\).

\begin{lemma}
  \(d^2 = 0\) so this is a chain complex.
\end{lemma}

\begin{proof}
  If \(\sigma: \Delta^k \to X\), consider the homomorphism
  \begin{align*}
    \varphi_\sigma: S_*(\Delta^k) &\to C_*(X) \\
    f_I &\mapsto \sigma \compose F_I
  \end{align*}
  \(d\) was chosen so \(d \varphi_\sigma = \varphi_\sigma d\). Then
  \[
    d^2(\sigma) = d^2(\sigma \compose \id_{\Delta^k}) = d^2(\varphi_\sigma(f_{\{0, \dots, k\}}))
    = \varphi_\sigma(d^2(f_{\{0, \dots, k\}})) = \varphi_\sigma(0) = 0
  \]
  since \(d^2 = 0\) in \(S_*(\Delta^k)\).
\end{proof}

We have a variant called \emph{reduced singular chain complex} of \(X\) which is defined by
\[
  \widetilde C_k(X) = \langle \sigma: \Delta^k \to X \rangle
\] 
for \(k \geq -1\) and \(\widetilde C_k(X) = 0\) for \(k < -1\). We have
\[
  \widetilde C_k(X) =
  \begin{cases}
    C_k(X) & k \geq 0 \\
    \langle \sigma_\emptyset \rangle \cong \Z & k = -1
  \end{cases}
\]
and if \(\sigma: \Delta^0 \to X\) then \(d\sigma = \sigma_\emptyset\).

\begin{definition}[singular homology]\index{singular homology}
  \(H_n(X) = H_n(C_*(X))\) and \(\widetilde H_n(X) = H_n(\widetilde C_*(X))\) are the \(n\)th \emph{(reduced) singular homology groups} of \(X\).
\end{definition}

If \(f: X \to Y\) is a map, define
\begin{align*}
  f_\#: C_*(X) &\to C_*(Y) \\
  \sigma &\mapsto f \compose \sigma
\end{align*}
Then
\begin{align*}
  d(f_\#(\sigma)) &= \sum_{j = 0}^n (-1)^j (f \compose \sigma) \compose F_{\{0, \dots, k\} \setminus \{j\}} \\
  &= \sum_{j = 0}^k (-1)^j f \compose (\sigma \compose F_{\{0, \dots, k\} \setminus \{j\}}) \\
  &= f_\# (d \sigma)
\end{align*}
so \(f_\#\) is a chain map.

\begin{lemma}\leavevmode
  \begin{enumerate}
  \item \((\id_X)_\# = \id_{C_*(X)}\).
  \item \((f \compose g)_\# = f_\# \compose g_\#\).
  \end{enumerate}
  In other words, there is a functor from the category of topological spaces to the cateogory of chain complexes over \(\Z\).
\end{lemma}

\begin{notation}
  If \(f: X \to Y\), write \(f_*: H_*(X) \to H_*(Y)\) instead of \((f_\#)_*\).
\end{notation}

\begin{corollary}
  There is a functor from the category of topological spaces to the category of \(\Z\)-modules.
\end{corollary}

\begin{proof}
  Composition of functors is a functor.
\end{proof}

\begin{eg}
  Let \(X = S^1\) and \(\sigma \in C_1(S^1)\) be the loop starting at \(p\) and loops around \(S^1\) once. Then \(d \sigma = \sigma_p - \sigma_p = 0\). Let \(\sigma_1, \sigma_2\) be paths from \(p\) to \(q\) and from \(q\) to \(p\). It is an exercise to find \(\tau \in C_2(X)\) with \(d\tau = \sigma - (\sigma_1 + \sigma_2)\), so \([\sigma] = [\sigma_1 + \sigma_2]\).
  \begin{figure}[ht]
  \centering
  \begin{minipage}{0.5\textwidth}
    \centering
    \def\svgwidth{.8\columnwidth}
    \input{figures/singular1_on_S1.pdf_tex}
  \end{minipage}%
  \begin{minipage}{0.5\textwidth}
    \centering
    \def\svgwidth{.8\columnwidth}
    \input{figures/singular2_on_S1.pdf_tex}
  \end{minipage}
\end{figure}
\end{eg}

\begin{proposition}\leavevmode
  \begin{enumerate}
  \item If \(X\) is path-connected then \(H_0(X) \cong \Z\).
  \item If \(X\) is the singleton \(\{p\}\) then
    \[
      H_*(X) =
      \begin{cases}
        \Z & * = 0 \\
        0 & \text{otherwise}
      \end{cases}
    \]
    and \(\widetilde H_*(X) = 0\).
  \item Let \(\pi_0(X)\) be the set of path-components of \(X\). Then
    \[
      H_*(X) = \bigoplus_{P \in \pi_0(X)} H_*(P).
    \]
  \end{enumerate}
\end{proposition}

\begin{proof}\leavevmode
  \begin{enumerate}
  \item We have
    \begin{align*}
      \ker d_0 &= C_0(X) = \langle \sigma_p: p \in X \rangle \\
      \im d_1 &= \text{span} \{\sigma_p - \sigma_q: p, q \in X\}
    \end{align*}
    since \(X\) is path-connected. Thus
    \begin{align*}
      \ker d_0 &\to \Z \\
      \sum a_i \sigma_{p_i} &\mapsto \sum a_i
    \end{align*}
    is a surjective homomorphism with kernel \(\im d_1\).
  \item There is a unique map \(\sigma_n: \Delta^n \to X\) and
    \[
      d \sigma_n = \sum_{j = 0}^n (-1)^j \sigma_{n - 1} =
      \begin{cases}
        \sigma_{n - 1} & n \text{ even and } n > 0 \\
        0 & \text{otherwise}
      \end{cases}
    \]
    so
    \begin{align*}
      \ker d &= \langle \sigma_0, \sigma_1, \sigma_3, \dots \rangle \\
      \im d &= \langle \sigma_1, \sigma_3, \sigma_5, \dots \rangle
    \end{align*}
    so the result follows. The reduced homology is left as an exercise.
  \item Let \(\iota_P: P \embed X\) be the inclusion. Then we have
    \[
      j = \sum (C_p)_\#: \bigoplus_{P \in \pi_0(X)} C_*(P) \to C_*(X)
    \]
    an injective map. \(\Delta^k\) is path-connected so if \(\sigma: \Delta^k \to X\) then \(\im \sigma \subseteq P\) for some \(P \in \pi_0(X)\) so \(j\) is also surjective.

    In general, if \(\{(C^\alpha, d^\alpha)\}_{\alpha \in A}\) is a family of chain complexes then so is
    \[
      (C^{\text{tot}}, D) = (\bigoplus_{\alpha \in A} C^\alpha, \sum_{\alpha \in A} d^\alpha)
    \]
    and
    \begin{align*}
      \ker D &= \bigoplus_{\alpha \in A} \ker d^\alpha \\
      \im D &= \bigoplus_{\alpha \in A} \im d^\alpha
    \end{align*}
    so
    \[
      H_*(C^{\text{tot}}) \cong \bigoplus_{\alpha \in A} H_*(C^\alpha).
    \]
    Now apply this to \(j\).
  \end{enumerate}
\end{proof}

\subsection{Homotopy invariance}

If \(g_0, g_1: X \to Y\) are homotopic then we want to show \(g_{0*} = g_{1*}: H_*(X) \to H_*(Y)\).

\begin{definition}[chain homotopy]\index{chain homotopy}
  Two chain maps \(g_0, g_1: (C, d) \to (C', d')\) are \emph{chain homotopic}, written \(g_0 \sim g_1\), if there is a homomorphism \(h: C_* \to C_{* + 1}'\) such that
  \[
    d'h + hd = g_1 - g_0.
  \]
\end{definition}

\begin{lemma}
  Chain homotopy is an equivalence relation.
\end{lemma}

\begin{proposition}
  If \(g_0\) and \(g_1\) are chain homotopic then \(g_{0*} = g_{1*}: H_*(C) \to H_*(C')\).
\end{proposition}

\begin{proof}
  Suppose \([x] \in H_*(C)\). Then
  \begin{align*}
    g_{1*}[x] - g_{0*}[x]
    &= [g_1(x) - g_0(x)] \\
    &= [d'h(x) + hd(x)] \\
    &= [d'h(x)] \\
    &= 0
  \end{align*}
  since \(d'h(x) \in \im d'\).
\end{proof}

\begin{definition}[chain homotopy equivalent]\index{chain homotopy equivalent}
  Chain complexes \((C, d)\) and \((C', d')\) are \emph{chain homotopy equivalent}, written \(C \sim C'\) if there exist chain maps \(f: C \to C', g: C' \to C\) such that \(fg \sim \id_{C'}, gf \sim \id_C\).
\end{definition}

\begin{ex}
  If \(C \sim C'\) then \(H_*(C) \cong H_*(C')\).
\end{ex}

\subsubsection{Universal chain homotopy}

Let \(c_n, c_n': \Delta^n \to \Delta^n \times [0, 1]\), \(c_n(v) = (v, 0), c_n'(v) = (v, 1)\) and consider the chain maps \(\varphi_{c_n}, \varphi_{c_n'}: S_*(\Delta^n) \to C_*(\Delta^n \times [0, 1])\), \(\varphi_{c_n}(f_I) = c_n \compose F_I\).

\begin{notation}
  \(\Delta^n \times [0, 1]\) is a convex subset of \(\R^{n + 1} \times [0, 1]\). If \(p_0, \dots, p_k \in \Delta^n \times [0, 1]\), define a map
  \begin{align*}
    [p_0 \cdots p_k]: \Delta^k &\to \Delta^n \times [0, 1] \\
    v &\mapsto \sum_{i = 0}^k v_i p_i
  \end{align*}
  Then
  \[
    d[p_0 \cdots p_k] = \sum_{j = 0}^k (-1)^j [p_0 \cdots \hat p_j \cdots p_k]
  \]
  where the hat above \(\hat p_j\) means that \(p_j\) is omitted.

  Furthermore we call \(f_i \times 0 = i\) and \(f_i \times 1 = i'\).
\end{notation}

\begin{figure}[ht]
  \centering
  \def\svgwidth{.8\columnwidth}
  \input{figures/chain_homotopy.pdf_tex}
  \caption{\(\Delta^1 \times [0, 1]\)}
  \label{fig:chain homotopy}
\end{figure}

The intuition for chain homotopy is illustrated by \Cref{fig:chain homotopy}. Suppose we set \(h(f_0), h(f_1)\) to be the segments \(0 \times [0, 1]\) and \(1 \times [0, 1]\), and \(h(f_{01})\) the square \(\Delta^1 \times [0, 1]\). Then
\begin{align*}
  dh(f_0) &= \varphi_{c'}(f_0) - \varphi_c(f_0) \\
  hd(f_0) &= h(0) = 0
\end{align*}
so \(dh(f_0) + hd(f_0) = \varphi_{c'}(f_0) - \varphi_c(f_0)\) and
\begin{align*}
  dh(f_{01}) &= (\text{top} + \text{bottom}) + (\text{sides}) \\
  hd(f_{01}) &= - (\text{sides})
\end{align*}
so again
\[
  dh(f_{01}) + hd(f_{01}) = \text{top} + \text{bottom} = \varphi_{c'}(f_{01}) - \varphi_c(f_{01}).
\]
Thus \(h\) would be a chain homotopy if it didn't map \(f_{01}\) to the square, which is not a simplex. To overcome this problem we cut the square into triangles \(00'1'\) and \(011'\). It is worthwhile to pause for a second to think what a chain homotopy for \(\Delta^2\) looks like.

\begin{proposition}
  \(\varphi_{c_n} \sim \varphi_{c_n'}\).
\end{proposition}

\begin{proof}
  Define
  \begin{align*}
    U_n: S_*(\Delta^n) &\to C_{* + 1}(\Delta^n \times [0, 1]) \\
    f_I &\mapsto \sum_{j = 0}^k (-1)^j [i_0 \dots i_j i_j' \cdots i_k']
  \end{align*}
  for \(I = \{i_0 < i_1 < \dots < i_k\}\). Then
  \begin{align*}
    U_n d(f_I) &= \sum_{a < b} (-1)^{a + b - 1} [i_0 \cdots \hat i_a i_b i_b' \cdots i_k'] \\
               &+ \sum_{a > b} (-1)^{a + b} [i_0 \cdots i_b i_b' \cdots \hat i_a' \cdots i_k'] \\
    dU_n(f_I) &= \sum_{a < b} (-1)^{b + a} [i_0 \cdots \hat i_a \cdots i_b i_b' \cdots i_k'] \\
               &+ \sum_{a > b} (-1)^{b + a + 1} [i_0 \cdots i_b i_b' \cdots \hat i_a \cdots i_k'] \\
               &+ \sum_{b = 0}^k (-1)^{b + b} [i_0 \cdots i_{b - 1} i_b' \cdots i_k'] \\
               &+ \sum_{b = 1}^{k + 1} (-1)^{b - 1 + b} [i_0 \cdots i_{b - 1} i_b' \cdots i_k']
  \end{align*}
  so almost everything cancels out and we have
  \[
    (dU_n + U_nd) (f_I) = [i_0' \cdots i_k'] - [i_0 \cdots i_k] = \varphi_{c_n'} (f_I) - \varphi_{c_n}(f_I).
  \]
\end{proof}

\begin{notation}
  Let \(\overline F_I = F_I \times \id_{[0, 1]}: \Delta^k \times [0, 1] \to \Delta^n \times [0, 1]\).
\end{notation}

\begin{lemma}
  The following diagram commutes:
  \[
    \begin{tikzcd}
      S_*(\Delta^k) \ar[r, "\varphi_I"] \ar[d, "U_k"] & S_*(\Delta^n) \ar[d, "U_n"] \\
      C_{* + 1}(\Delta^k \times [0, 1]) \ar[r, "\overline F_{I \#}"] & C_{* + 1} (\Delta^n \times [0, 1])
    \end{tikzcd}
  \]
\end{lemma}

\begin{proof}
  Checking definitions.
\end{proof}

\begin{theorem}
  Suppose \(g_0, g_1: X \to Y\) are homotopic then \(g_{0\#} \sim g_{1\#}\).
\end{theorem}

\begin{proof}
  Let \(G: X \times [0, 1] \to Y\) be the homotopy. Define
  \begin{align*}
    G_\sigma: \Delta^n \times [0, 1] &\to Y \\
    (v, t) &\mapsto G(\sigma(v), t)
  \end{align*}
  Then \(G_{\sigma \compose F_I} = G_\sigma \compose \overline F_I\). Define
  \begin{align*}
    h: C_*(X) &\to C_{* + 1}(Y) \\
    \sigma &\mapsto G_{\sigma \#} (U_n (f_{0 \cdots n}))
  \end{align*}
  then
  \begin{align*}
    dh(\sigma) &= d G_{\sigma \#} (U_n (f_{0 \cdots n})) = G_{\sigma \#} (d U_n(f_{0 \cdots n})) \\
    hd(\sigma) &= h (\sum (-1)^j \sigma \compose F_{\hat j}) \\
               &= \sum (-1)^j G_{\sigma \compose F_{\hat j} \#} (U_{n - 1} (f_{0 \cdots n - 1})) \\
               &= \sum (-1)^j G_{\sigma \#} \overline F_{\hat j\#} (U_{n - 1}(f_{0 \cdots n - 1})) \\
               &= \sum (-1)^j G_{\sigma \#} (U_n(\varphi_{\hat j} (f_{0 \cdots n - 1}))) \quad \text{by lemma} \\
               &= G_{\sigma \#} (U_n (\sum (-1)^j \varphi_{\hat j} (f_{0 \cdots n - 1}))) \\
               &= G_{\sigma \#} (U_n d(f_{0 \cdots n}))
  \end{align*}
  so
  \begin{align*}
    (dh + hd)(\sigma) &= G_{\sigma \#} (U_nd + dU_n) (f_{0 \cdots n}) \\
                      &= G_{\sigma \#} ((\varphi_{i_n'} - \varphi_{i_n}) (f_{0 \cdots n})) \\
                      &= G_{\sigma \#} (i_n' - i_n) \\
                      &= g_1 \compose \sigma - g_0 \compose \sigma \\
                      &= g_{1\#}(\sigma) - g_{0\#}(\sigma)
  \end{align*}
\end{proof}

\begin{corollary}
  If \(g_0, g_1: X \to Y\) are homotopic then \(g_{0*} = g_{1*}: H_*(X) \to H_*(Y)\).
\end{corollary}

\begin{corollary}
  If \(X\) are \(Y\) are homotopy equivalent then \(H_*(X) \cong H_*(Y)\).
\end{corollary}

\begin{proof}
  \(X \sim Y\) so we have \(f: X \to Y, g: Y \to X\) with \(f \compose g \sim \id_Y, g \compose f \sim \id_X\). Then
  \[
    f_* \compose g_* = (f \compose g)_* = (\id_Y)_* = \id_{H_*(Y)}
  \]
  and similarly \(g_* \compose f_* = \id_{H_*(X)}\) so \(g_*\) and \(f_*\) are inverses to each other.
\end{proof}

\begin{corollary}
  If \(X\) is contractible then
  \[
    H_*(X) =
    \begin{cases}
      \Z & * = 0 \\
      0 & * > 0
    \end{cases}
  \]
\end{corollary}

\subsection{Homology of a pair}

\subsubsection{Exact sequence}

Suppose we have a sequence
\[
  \begin{tikzcd}
    \cdots \ar[r] & A_{i + 1} \ar[r, "f_{i + 1}"] & A_i \ar[r, "f_i"] & A_{i - 1} \ar[r, "f_{i - 1}"] & \cdots
  \end{tikzcd}
\]
where \(A_i\)'s are \(R\)-modules and \(f_i\)'s are homomorphisms.

\begin{definition}[exact sequence]\index{exact sequence}
  We say the sequence is \emph{exact} at \(A_i\) if \(\ker f_i = \im f_{i + 1}\). We say the sequence is \emph{exact} if it is exact at all \(A_i\).
\end{definition}

In other words, the sequence is exact is the same as saying \((A_*, f)\) is a chain complex with \(H_*(A) = 0\).

\begin{eg}\leavevmode
  \begin{enumerate}
  \item \(
    \begin{tikzcd}
      0 \ar[r] & A \ar[r, "\iota"] & B 
    \end{tikzcd}
  \) is exact at \(A\) if and only if \(\iota\) is injective.

\item \(
    \begin{tikzcd}
      B \ar[r, "\pi"] & C \ar[r] & 0
    \end{tikzcd}
  \) is exact at \(C\) if and only if \(\pi\) is surjective.
\item \(
  \begin{tikzcd}
    0 \ar[r] & A \ar[r] & 0
  \end{tikzcd}
  \) is exact if and only if \(A = 0\).
\item \(
  \begin{tikzcd}
    0 \ar[r] & A \ar[r, "f"] & B \ar[r] & 0
  \end{tikzcd}
  \) is exact if and only if \(f: A \to B\) is an isomorphism.
\item \(
  \begin{tikzcd}
    0 \ar[r] & A \ar[r, "\iota"] & B \ar[r, "\pi"] & C \ar[r] & 0
  \end{tikzcd}
  \) is exact if and only if \(\iota: A \embed B \) and \(\pi: B \to C\) is a surjection with kernel \(\im A\). This is called a \emph{short exact sequence} (SES). In particular, a long exact sequence gives a bunch of short exact sequences
  \[
    \begin{tikzcd}
      0 \ar[r] & \coker f_{i + 2} \ar[r, "f_{i + 1}"] & A_i \ar[r, "f_i"] & \ker f_{i - 1} \ar[r] & 0
    \end{tikzcd}
  \]
  \end{enumerate}
\end{eg}

\begin{definition}
  A sequence
  \[
    \begin{tikzcd}
      0 \ar[r] & A_* \ar[r, "\iota"] & B_* \ar[r, "\pi"] & C_* \ar[r] & 0
    \end{tikzcd}
  \]
  is a \emph{SES of chain complexes} if
  \begin{enumerate}
  \item \(A_*, B_*, C_*\) are chain complexes arnd \(\iota, \pi\) are chain maps.
  \item \(
    \begin{tikzcd}
      0 \ar[r] & A_i \ar[r, "\iota"] & B_i \ar[r, "\pi"] & C_i \ar[r] & 0
    \end{tikzcd}
    \) is a exact for all \(i\).
  \end{enumerate}
\end{definition}

\begin{proposition}[snake lemma]\index{snake lemma}
  If
  \[
    \begin{tikzcd}
      0 \ar[r] & A_* \ar[r, "\iota"] & B_* \ar[r, "\pi"] & C_* \ar[r] & 0
    \end{tikzcd}
  \]
  is a SES of chain complexes then there is a long exact sequence on homology
  \[
    \begin{tikzcd}
      H_*(A) \ar[r, "\iota_*"] & H_*(B) \ar[r, "\pi_*"] & H_*(C) \ar[dll, "\b", out=0, in=180] \\
      H_{* - 1}(A) \ar[r, "\iota_*"] & H_{* - 1}(B) \ar[r, "\pi_*"] & H_{* - 1}(C) \ar[dll, "\b", out=0, in=180] \\
      H_{* - 2} \ar[r] & \cdots
    \end{tikzcd}
  \]
  where the map \(\b\) is called the \emph{boundary map}.
\end{proposition}

\begin{proof}
  The map \(\b\) is defined as follow: suppose given \([c] \in H_n(C)\) so \(dc = 0\).
  \begin{enumerate}
  \item \(\pi\) is surjective so exists \(b \in B_n\) with \(\pi(b) = c\).
  \item \(\pi db = d\pi b = dc = 0\).
  \item The sequence is exact at \(B_{n - 1}\) so exists \(a \in A_{n - 1}\) with \(\iota a = db\).
  \item \(\iota d a = d \iota a = ddb = d^2 b = 0\) so by injectivity of \(\iota\), \(da = 0\).
  \item Finally define \(\b([c]) = [a]\).
  \end{enumerate}
  \[
    \begin{tikzcd}
      0 \ar[r] & A_n \ar[r, "\iota"] \ar[d, "d"] & B_n \ar[r, "\pi"] \ar[d, "d"] & C_n \ar[r] \ar[d, "d"] & 0 \\
      0 \ar[r] & A_{n - 1} \ar[r] & B_{n - 1} \ar[r] & C_{n - 1} \ar[r] & 0
    \end{tikzcd}
  \]
  We have to check this is well-defined and the resulting sequence is exact. We check exactness at \(H_{n - 1}(A)\):
  \begin{align*}
    [a] \in \ker \iota_*
    &\iff \iota a = db \text{ for some } b \in B_n \\
    &\iff [a] = \b [\pi b] \\
    &\iff [a] \in \im \b
  \end{align*}
  The rest are left as exercises.
\end{proof}

\begin{eg}
  Recall that if \(X \neq \emptyset\), we can express unreduced homology as
  \[
    H_*(X) =
    \begin{cases}
      \widetilde H_*(X) & * > 0 \\
      \widetilde H_*(X) \oplus \Z & * = 0
    \end{cases}
  \]
  We can show this using the snake lemma. Let \(K_* = \langle \sigma_\emptyset \rangle\) if \(* = -1\) and \(0\) otherwise, then
  \[
    H_*(K) =
    \begin{cases}
      \langle \sigma_\emptyset \rangle & * = -1 \\
      0 & * \neq -1
    \end{cases}
  \]
  so we have a SES
  \[
    \begin{tikzcd}
      0 \ar[r] & K_* \ar[r] & \widetilde C_*(X) \ar[r] & C_*(X) \ar[r] & 0
    \end{tikzcd}
  \]
  so we have a long exact sequence that looks like
  \[
    \begin{tikzcd}
      H_*(K) \ar[r] & \widetilde H_*(X) \ar[r] & H_*(X) \ar[r] & H_{* - 1}(K)
    \end{tikzcd}
  \]
  so for \(* > 0\), \(\tilde H_*(X) \cong H_*(X)\). The only interesting thing happens at \(* = 0\) which gives
  \[
    \begin{tikzcd}
      0 \ar[r] & \widetilde H_0(X) \ar[r] & H_0(X) \ar[r, "\b"] & \Z \ar[r] & \widetilde H_{-1}(X) \ar[r] & 0
    \end{tikzcd}
  \]
  Let \(p \in X\) be any point and let \(\sigma_p: \Delta^0 \to X\) be an element of \(\widetilde H_0\). As \(d \sigma_p = \sigma_\emptyset\), we have \(\b[\sigma_p] = \sigma_\emptyset\) so \(\b\) is surjective. Thus \(\widetilde H_{-1}(X) = 0\) so \(H_0(X) \cong \widetilde H_0(X) \oplus \Z\) by example sheet 1 question 3.
\end{eg}

\subsubsection{Subcomplexes and quotient complexes}

\begin{definition}[subcomplex, quotient complex]\index{chain complex!subcomplex}\index{chain complex!quotient complex}
  Suppose \((C_*, d)\) is a chain complex. We say \(A_*\) is a \emph{subcomplex} of \(C_*\) if
  \begin{enumerate}
  \item \(A_* = \bigoplus_{i \in \Z} A_i\) where \(A_i \subseteq C_i\) is a submodule.
  \item \(d(A_i) \subseteq A_{i - 1}\).
  \end{enumerate}
  If so then \((A_*, d)\) is a chain complex.

  Let \(Q_i = C_i/A_i\) then \(d: C_i \to C_{i - 1}\) induces \(d_Q: Q_i \to Q_{i - 1}\) with \(d_Q^2 = d^2 = 0\). Call \((Q_*, d_Q)\) the \emph{quotient complex}.

  In other words, there is a SES
  \[
    \begin{tikzcd}
      0 \ar[r] & A_* \ar[r] & C_* \ar[r] & Q_* \ar[r] & 0
    \end{tikzcd}
  \]
\end{definition}

Suppose \(A \subseteq X\). If \(\sigma: \Delta^k \to X\) has \(\im \sigma \subseteq A\) then \(\sigma \compose F_{\{0, \dots, k\} \setminus \{j\}}: \Delta^{k - 1} \to X\) has image in \(A\) as well, so \(d \sigma \in C_*(A)\). Therefore \(C_*(A)\) is a subcomplex of \(C_*(X)\). We then define

\begin{definition}[homology of a pair]\index{homology group!of a pair}
  If \(A \subseteq X\), we define
  \[
    C_*(X, A) = C_*(X)/C_*(A)
  \]
  and \(H_*(X, A) = H_*(C_*(X, A))\) is the \emph{homology of the pair} \((X, A)\).
\end{definition}

We have the SES
\[
  \begin{tikzcd}
    0 \ar[r] & C_*(A) \ar[r] & C_*(X) \ar[r] & C_*(X, A) \ar[r] & 0
  \end{tikzcd}
\]
whose corresponding long exact sequence is the \emph{long exact sequence of the pair} \((X, A)\)
\[
  \begin{tikzcd}
    \cdots \ar[r] & H_*(A) \ar[r, "\iota_*"] & H_*(X) \ar[r] & H_*(X, A) \ar[r, "\b"] & H_{* - 1}(A) \ar[r] & \cdots
  \end{tikzcd}
\]
where \(\iota: A \embed X\) is the inclusion.

\begin{eg}
  Let \((X, A) = (D^1, S^0)\). We have
  \[
    H_*(S^0) =
    \begin{cases}
      \Z \oplus \Z & * = 0 \\
      0 & * \neq 0
    \end{cases}
    \quad
    H_*(D^1) =
    \begin{cases}
      \Z & * = 0 \\
      0 & * \neq 0
    \end{cases}
  \]
  so the long exact sequence of the pair \((D^1, S^0)\) gives
  \[
    \begin{tikzcd}
      H_1(D^1) \ar[r] \ar[d, equal] & H_1(X, A) \ar[r] & H_0(S^0) \ar[r] \ar[d, equal] & H_0(D^1) \ar[r] \ar[d, equal] & H_0(X, A) \ar[r] & 0\\
      0 & & \Z \oplus \Z & \Z &
    \end{tikzcd}
  \]
  It is an exercise to check the map \(H_0(S^0) \to H_0(D^1)\) is surjective and thus \(H_1(X, A) \cong \Z\).
\end{eg}




\printindex
\end{document}
