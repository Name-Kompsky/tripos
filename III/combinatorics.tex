\documentclass[a4paper]{article}

\def\npart{III}

\def\ntitle{Combinatorics}
\def\nlecturer{I.\ B.\ Leader}

\def\nterm{Michaelmas}
\def\nyear{2018}

\ifx \nauthor\undefined
  \def\nauthor{Qiangru Kuang}
\else
\fi

\ifx \ntitle\undefined
  \def\ntitle{Template}
\else
\fi

\ifx \nauthoremail\undefined
  \def\nauthoremail{qk206@cam.ac.uk}
\else
\fi

\ifx \ndate\undefined
  \def\ndate{\today}
\else
\fi

\title{\ntitle}
\author{\nauthor}
\date{\ndate}

%\usepackage{microtype}
\usepackage{mathtools}
\usepackage{amsthm}
\usepackage{stmaryrd}%symbols used so far: \mapsfrom
\usepackage{empheq}
\usepackage{amssymb}
\let\mathbbalt\mathbb
\let\pitchforkold\pitchfork
\usepackage{unicode-math}
\let\mathbb\mathbbalt%reset to original \mathbb
\let\pitchfork\pitchforkold

\usepackage{imakeidx}
\makeindex[intoc]

%to address the problem that Latin modern doesn't have unicode support for setminus
%https://tex.stackexchange.com/a/55205/26707
\AtBeginDocument{\renewcommand*{\setminus}{\mathbin{\backslash}}}
\AtBeginDocument{\renewcommand*{\models}{\vDash}}%for \vDash is same size as \vdash but orginal \models is larger
\AtBeginDocument{\let\Re\relax}
\AtBeginDocument{\let\Im\relax}
\AtBeginDocument{\DeclareMathOperator{\Re}{Re}}
\AtBeginDocument{\DeclareMathOperator{\Im}{Im}}
\AtBeginDocument{\let\div\relax}
\AtBeginDocument{\DeclareMathOperator{\div}{div}}

\usepackage{tikz}
\usetikzlibrary{automata,positioning}
\usepackage{pgfplots}
%some preset styles
\pgfplotsset{compat=1.15}
\pgfplotsset{centre/.append style={axis x line=middle, axis y line=middle, xlabel={$x$}, ylabel={$y$}, axis equal}}
\usepackage{tikz-cd}
\usepackage{graphicx}
\usepackage{newunicodechar}

\usepackage{fancyhdr}

\fancypagestyle{mypagestyle}{
    \fancyhf{}
    \lhead{\emph{\nouppercase{\leftmark}}}
    \rhead{}
    \cfoot{\thepage}
}
\pagestyle{mypagestyle}

\usepackage{titlesec}
\newcommand{\sectionbreak}{\clearpage} % clear page after each section
\usepackage[perpage]{footmisc}
\usepackage{blindtext}

%\reallywidehat
%https://tex.stackexchange.com/a/101136/26707
\usepackage{scalerel,stackengine}
\stackMath
\newcommand\reallywidehat[1]{%
\savestack{\tmpbox}{\stretchto{%
  \scaleto{%
    \scalerel*[\widthof{\ensuremath{#1}}]{\kern-.6pt\bigwedge\kern-.6pt}%
    {\rule[-\textheight/2]{1ex}{\textheight}}%WIDTH-LIMITED BIG WEDGE
  }{\textheight}% 
}{0.5ex}}%
\stackon[1pt]{#1}{\tmpbox}%
}

%\usepackage{braket}
\usepackage{thmtools}%restate theorem
\usepackage{hyperref}

% https://en.wikibooks.org/wiki/LaTeX/Hyperlinks
\hypersetup{
    %bookmarks=true,
    unicode=true,
    pdftitle={\ntitle},
    pdfauthor={\nauthor},
    pdfsubject={Mathematics},
    pdfcreator={\nauthor},
    pdfproducer={\nauthor},
    pdfkeywords={math maths \ntitle},
    colorlinks=true,
    linkcolor={red!50!black},
    citecolor={blue!50!black},
    urlcolor={blue!80!black}
}

\usepackage{cleveref}



% TODO: mdframed often gives bad breaks that cause empty lines. Would like to switch to tcolorbox.
% The current workaround is to set innerbottommargin=0pt.

%\usepackage[theorems]{tcolorbox}





\usepackage[framemethod=tikz]{mdframed}
\mdfdefinestyle{leftbar}{
  %nobreak=true, %dirty hack
  linewidth=1.5pt,
  linecolor=gray,
  hidealllines=true,
  leftline=true,
  leftmargin=0pt,
  innerleftmargin=5pt,
  innerrightmargin=10pt,
  innertopmargin=-5pt,
  % innerbottommargin=5pt, % original
  innerbottommargin=0pt, % temporary hack 
}
%\newmdtheoremenv[style=leftbar]{theorem}{Theorem}[section]
%\newmdtheoremenv[style=leftbar]{proposition}[theorem]{proposition}
%\newmdtheoremenv[style=leftbar]{lemma}[theorem]{Lemma}
%\newmdtheoremenv[style=leftbar]{corollary}[theorem]{corollary}

\newtheorem{theorem}{Theorem}[section]
\newtheorem{proposition}[theorem]{Proposition}
\newtheorem{lemma}[theorem]{Lemma}
\newtheorem{corollary}[theorem]{Corollary}
\newtheorem{axiom}[theorem]{Axiom}
\newtheorem*{axiom*}{Axiom}

\surroundwithmdframed[style=leftbar]{theorem}
\surroundwithmdframed[style=leftbar]{proposition}
\surroundwithmdframed[style=leftbar]{lemma}
\surroundwithmdframed[style=leftbar]{corollary}
\surroundwithmdframed[style=leftbar]{axiom}
\surroundwithmdframed[style=leftbar]{axiom*}

\theoremstyle{definition}

\newtheorem*{definition}{Definition}
\surroundwithmdframed[style=leftbar]{definition}

\newtheorem*{slogan}{Slogan}
\newtheorem*{eg}{Example}
\newtheorem*{ex}{Exercise}
\newtheorem*{remark}{Remark}
\newtheorem*{notation}{Notation}
\newtheorem*{convention}{Convention}
\newtheorem*{assumption}{Assumption}
\newtheorem*{question}{Question}
\newtheorem*{answer}{Answer}
\newtheorem*{note}{Note}
\newtheorem*{application}{Application}

%operator macros

%basic
\DeclareMathOperator{\lcm}{lcm}

%matrix
\DeclareMathOperator{\tr}{tr}
\DeclareMathOperator{\Tr}{Tr}
\DeclareMathOperator{\adj}{adj}

%algebra
\DeclareMathOperator{\Hom}{Hom}
\DeclareMathOperator{\End}{End}
\DeclareMathOperator{\id}{id}
\DeclareMathOperator{\im}{im}
\DeclareMathOperator{\coker}{coker}
\DeclarePairedDelimiter{\generation}{\langle}{\rangle}

%groups
\DeclareMathOperator{\sym}{Sym}
\DeclareMathOperator{\sgn}{sgn}
\DeclareMathOperator{\inn}{Inn}
\DeclareMathOperator{\aut}{Aut}
\DeclareMathOperator{\GL}{GL}
\DeclareMathOperator{\SL}{SL}
\DeclareMathOperator{\PGL}{PGL}
\DeclareMathOperator{\PSL}{PSL}
\DeclareMathOperator{\SU}{SU}
\DeclareMathOperator{\UU}{U}
\DeclareMathOperator{\SO}{SO}
\DeclareMathOperator{\OO}{O}
\DeclareMathOperator{\PSU}{PSU}
\DeclareMathOperator{\Sp}{Sp}


%hyperbolic
\DeclareMathOperator{\sech}{sech}

%field, galois heory
\DeclareMathOperator{\ch}{ch}
\DeclareMathOperator{\gal}{Gal}
\DeclareMathOperator{\emb}{Emb}



%ceiling and floor
%https://tex.stackexchange.com/a/118217/26707
\DeclarePairedDelimiter\ceil{\lceil}{\rceil}
\DeclarePairedDelimiter\floor{\lfloor}{\rfloor}


\DeclarePairedDelimiter{\innerproduct}{\langle}{\rangle}

%\DeclarePairedDelimiterX{\norm}[1]{\lVert}{\rVert}{#1}
\DeclarePairedDelimiter{\norm}{\lVert}{\rVert}



%Dirac notation
%TODO: rewrite for variable number of arguments
\DeclarePairedDelimiterX{\braket}[2]{\langle}{\rangle}{#1 \delimsize\vert #2}
\DeclarePairedDelimiterX{\braketthree}[3]{\langle}{\rangle}{#1 \delimsize\vert #2 \delimsize\vert #3}

\DeclarePairedDelimiter{\bra}{\langle}{\rvert}
\DeclarePairedDelimiter{\ket}{\lvert}{\rangle}




%macros

%general

%divide, not divide
\newcommand*{\divides}{\mid}
\newcommand*{\ndivides}{\nmid}
%vector, i.e. mathbf
%https://tex.stackexchange.com/a/45746/26707
\newcommand*{\V}[1]{{\ensuremath{\symbf{#1}}}}
%closure
\newcommand*{\cl}[1]{\overline{#1}}
%conjugate
\newcommand*{\conj}[1]{\overline{#1}}
%set complement
\newcommand*{\stcomp}[1]{\overline{#1}}
\newcommand*{\compose}{\circ}
\newcommand*{\nto}{\nrightarrow}
\newcommand*{\p}{\partial}
%embed
\newcommand*{\embed}{\hookrightarrow}
%surjection
\newcommand*{\surj}{\twoheadrightarrow}
%power set
\newcommand*{\powerset}{\mathcal{P}}

%matrix
\newcommand*{\matrixring}{\mathcal{M}}

%groups
\newcommand*{\normal}{\trianglelefteq}
%rings
\newcommand*{\ideal}{\trianglelefteq}

%fields
\renewcommand*{\C}{{\mathbb{C}}}
\newcommand*{\R}{{\mathbb{R}}}
\newcommand*{\Q}{{\mathbb{Q}}}
\newcommand*{\Z}{{\mathbb{Z}}}
\newcommand*{\N}{{\mathbb{N}}}
\newcommand*{\F}{{\mathbb{F}}}
%not really but I think this belongs here
\newcommand*{\A}{{\mathbb{A}}}

%asymptotic
\newcommand*{\bigO}{O}
\newcommand*{\smallo}{o}

%probability
\newcommand*{\prob}{\mathbb{P}}
\newcommand*{\E}{\mathbb{E}}

%vector calculus
\newcommand*{\gradient}{\V \nabla}
\newcommand*{\divergence}{\gradient \cdot}
\newcommand*{\curl}{\gradient \cdot}

%logic
\newcommand*{\yields}{\vdash}
\newcommand*{\nyields}{\nvdash}

%differential geometry
\renewcommand*{\H}{\mathbb{H}}
\newcommand*{\transversal}{\pitchfork}
\renewcommand{\d}{\mathrm{d}} % exterior derivative

%number theory
\newcommand*{\legendre}[2]{\genfrac{(}{)}{}{}{#1}{#2}}%Legendre symbol

%algebraic geometry
\DeclareMathOperator{\Spec}{Spec}
\DeclareMathOperator{\Proj}{Proj}

\newcommand{\shadow}{\partial}
\renewcommand{\P}{\mathbb P}

\begin{document}

\begin{titlepage}
  \begin{center}
    \includegraphics[width=0.6\textwidth]{logo.jpg}\par
    \vspace{1cm}
    {\scshape\huge Mathamatics Tripos \par}
    \vspace{2cm}
    {\huge Part \npart \par}
    \vspace{0.6cm}
    {\Huge \bfseries \ntitle \par}
    \vspace{1.2cm}
    {\Large\nterm, \nyear \par}
    \vspace{2cm}
    
    {\large \emph{Lectures by } \par}
    \vspace{0.2cm}
    {\Large \scshape \nlecturer}
    
    \vspace{0.5cm}
    {\large \emph{Notes by }\par}
    \vspace{0.2cm}
    {\Large \scshape \href{mailto:\nauthoremail}{\nauthor}}
 \end{center}
\end{titlepage}

\tableofcontents

\section{Set systems}

\begin{definition}[set system]\index{set system}
  Let \(X\) be a set. A \emph{set system} on \(X\) (or \emph{family of subsets}) of \(X\) is a family \(\mathcal A \subseteq \powerset (X)\).
\end{definition}

\begin{eg}
  We write \(X^{(r)} = \{A \subseteq X: |A| = r\}\)
\end{eg}

In this course we almost deal exclusively with finite setss so unless otherwise stated, in this course we assume all sets are finite and \(X = [n] = \{1, 2, \dots, n\}\).

For example, \(|X^{(r)}| = \binom{n}{r}\). More concretely, for example,
\[
  [4]^{[2]} = \{12, 13, 14, 23, 24, 34\}
\]
where \(12\) denotes \(\{1, 2\}\) to avoid heavy notation. Therefore \(|[4]^{[2]}| = 6\).

What is the mental picture for power set? Often we make \(\powerset (X)\) into a graph, called \(Q_n\), by joining \(A\) to \(B\) if \(|A \Delta B| = 1\) where \(\Delta\) is the symmetric difference, i.e.\ if \(A = B \cup \{i\}\) for some \(i \notin B\) (or vice verse).

\begin{eg}
  \(Q_3\)
\end{eg}

\begin{eg}
  General picture for \(Q_n\) where \(n\) is even:

  and when \(n\) is odd:
\end{eg}

If we identify a set \(A \subseteq X\) with a \(\{0, 1\}\) sequence of length \(n\) (e.g.\ \(134 \leftrightarrow 1011000 \cdots 0\)), via \(A \leftrightarrow \mathbf{1}_A \text{ or } \chi_A\). In this way, we can represent \(Q_n\) as a \(n\)-dimensional cube:

For this reason, \(Q_n\) is often called the \emph{hypercube} or \emph{discrete cube} or \emph{\(n\)-cube}. In this way, the study of a set system become the study of a graph.

\subsection{Chains \& Antichains}

\begin{definition}[chain]\index{chain}
  A family \(\mathcal A \subseteq \powerset (X)\) is a \emph{chain} if for all \(A, B \in \mathcal A, A \subseteq B \text{ or } B \subseteq A\).
\end{definition}

\begin{eg}
  \(\{12, 125, 123589\}\).
\end{eg}

On the other hand, we have

\begin{definition}[antichain]\index{antichain}
  A family \(\mathcal A \subseteq \powerset (X)\) is an \emph{antichain} if for all \(A, B \in \mathcal A\) with \(A \neq B\), \(A \nsubseteq B\).
\end{definition}

\begin{eg}
  \(\{1, 467, 2456\}\).
\end{eg}

A natural question is: how large can a chain be? Obviously we can achieve \(|\mathcal A| = n + 1\). We also cannot exceed \(n + 1\) since a chain must meet each ``level'' \(X^{(r)}\) for \(0 \leq r \leq n\) in at most one place.

A less trivial question is how large an antichain can be. We could achieve \(|\mathcal A| = n\) by enumerating all the singletons. It is maximal since adjoining any nonempty set to the family will result in an inclusion. However, it is not the largest antichain. Indeed, we could take \(\mathcal A = X^{(r)}\) for any \(r\). Thus we can achieve \(|\mathcal A| = \binom{n}{\floor{n/2}}\). Can we beat it?

Pause for a moment and consider the longest chain problem. Why can a chain meet each level at only one place? One way to see this is that each level is an antichain: as we can decompose \(Q_n\) into \(n + 1\) antichains, we cannot have a chain longer than that. Inspired by this, we try to decompose \(Q_n\) into chains.

\begin{theorem}[Sperner's lemma]\index{Sperner's lemma}
  Let \(\mathcal A \subseteq \powerset(X)\) where \(|X| = n\) be an antichain. Then
  \[
    |\mathcal A| \leq \binom{n}{\floor{n/2}}.
  \]
\end{theorem}

\begin{proof}
  Sufficient to partition \(\powerset(X)\) into \(\binom{n}{\floor{n/2}}\) chains. For this sufficient to show:
  \begin{enumerate}
  \item for all \(r < \frac{n}{2}\), there exist matchings from \(X^{(r)}\) to \(X^{(r + 1)}\) where a matching is just a set of non-adjacent edges,
  \item for all \(r > \frac{n}{2}\), there exist matchings from \(X^{(r)}\) to \(X^{(r - 1)}\).
  \end{enumerate}
  Then put these matchings together to form chains. Each passes through \(X^{(n/2)}\) so there are \(\binom{n}{\floor{n/2}}\) of them.

  By taking complements, sufficient to prove 1. Consider the subgraph of \(Q_n\) spanned by \(X^{(r)} \cup X^{(r + 1)}\) which is bipartite. For any \(\mathcal B \subseteq X^{(r)}\), let \(\Gamma(\mathcal B)\) be the neighbourhood (in \(X^{(r + 1)}\)) of \(\mathcal B\). Then we have

  \[
    \# (\mathcal B - \Gamma(\mathcal B) \text{ edges}) = |\mathcal B| (n - r)
  \]
  as each point in \(X^{(r)}\) has degree \(n - r\). Meanwhile
  \[
    \# (\mathcal B - \Gamma(\mathcal B) \text{ edges}) \leq |\Gamma(\mathcal B)| (r + 1)
  \]
  as each point in \(X^{(r + 1)}\) has degree \(r + 1\). Thus
  \[
    |\Gamma(\mathcal B)| \geq |\mathcal B| \frac{n - r}{r + 1} \geq |\mathcal B|
  \]
  as \(r < \frac{n}{2}\). Hence by Hall's theorem there exist matchings.
\end{proof}

\begin{remark}\leavevmode
  \begin{enumerate}
  \item \(\binom{n}{\floor{n/2}}\) is achieveable, e.g.\ \(\mathcal A = X^{(\floor{n/2})}\).
  \item Note that the theorem says nothing about extremal cases --- which antinchain have this size?
  \end{enumerate}
\end{remark}

The aim is to show that for \(\mathcal A\) an antichain,
\[
  \sum_{r = 0}^n \frac{\mathcal A \cap X^{(r)}}{\binom{n}{r}} \leq 1.
\]
Note that this trivially implies Sperner's lemma. We will use the same setup but bound the numbers more carefully. To justify the following definition, we will write \(X^{(r)}\) above \(X^{(r - 1)}\) in \(Q_n\).

\begin{definition}
  Let \(\mathcal A \subseteq X^{(r)}\) for some \(1 \leq r \leq n\). The \emph{shadow} or \emph{lower shadow} of \(\mathcal A\) is
  \[
    \shadow A = \shadow^- A = \{A - \{i\}: A \in \mathcal A, i \in A\}.
  \]
  so \(\shadow A \subseteq X^{(r - 1)}\).
\end{definition}

\begin{eg}
  Let \(\mathcal A = \{123, 124, 134, 135\} \subseteq X^{(3)}\). Then
  \[
    \shadow A = \{12, 13, 23, 14, 24, 34, 15, 35\} \subseteq X^{(2)}.
  \]
\end{eg}

\begin{lemma}[Local LYM]\index{local LYM}
  Let \(\mathcal A \subseteq X^{(r)}\) where \(1 \leq r \leq n\). Then
  \[
    \frac{|\shadow \mathcal A|}{\binom{n}{r - 1}} \geq \frac{|\mathcal A|}{\binom{n}{r}}.
  \]
\end{lemma}

Informally, the fraction of the later occupied increases when we take the shadow.

\begin{proof}
  \[
    \#(\mathcal A - \shadow \mathcal A \text{ edges in } Q_n) = r|\mathcal A|
  \]
  by counting from above and
  \[
    \#(\mathcal A - \shadow \mathcal A \text{ edges in } Q_n) \leq (n - r + 1) |\shadow \mathcal A|
  \]
  counting from below so
  \[
    \frac{|\shadow \mathcal A|}{|\mathcal A|} \geq \frac{r}{n - r + 1}
  \]
  but
  \[
    \frac{\binom{r}{r - 1}}{\binom{n}{r}} = \frac{r}{n - r + 1}.
  \]
\end{proof}

When does equality hold in local LYM? We'll need
\[
  (A - \{i\}) \cup \{j\} \in \mathcal A
\]
for all \(a \in \mathcal A, i \in A, j \notin A\). Hence \(\mathcal A = X^{(r)}\) or \(\emptyset\).

\begin{theorem}[LYM inequality]\index{LYM inequality}
  Let \(\mathcal A \subseteq \powerset(X)\) be an antichain. Then
  \[
    \sum_{r = 0}^n \frac{\mathcal A \cap X^{(r)}}{\binom{n}{r}} \leq 1.
  \]
\end{theorem}

\begin{proof}
  The whole idea of the proof can be summarised by ``bubble down with local LYM''. Let \(\mathcal A_r = \mathcal A \cap X^{(r)}\). Obviously
  \[
    \frac{|\mathcal A_n|}{\binom{n}{n}} \leq 1.
  \]
  Also \(\shadow \mathcal A_n\) and \(\mathcal A_{n - 1}\) are distinct as \(\mathcal A\) is an antichain. Thus
  \[
    \frac{|\shadow \mathcal A_n|}{\binom{n}{n - 1}} + \frac{|\mathcal A_{n - 1}|}{\binom{n}{n - 1}}
    = \frac{|\shadow \mathcal A_n \cup \mathcal A_{n - 1}|}{\binom{n}{n - 1}} \leq 1
  \]
  so
  \[
    \frac{|\mathcal A_n|}{\binom{n}{n}} + \frac{|\mathcal A_{n - 1}|}{\binom{n}{n - 1}} \leq 1
  \]
  by local LYM.

  Also \(\shadow (\shadow \mathcal A_n \cup \mathcal A_{n - 1})\) is distint from \(\mathcal A_{n - 2}\), again since \(\mathcal A\) is an antichain. Thus
  \[
    \frac{|\shadow (\shadow \mathcal A_n \cup \mathcal A_{n -1})|}{\binom{n}{n - 2}} +\frac{|\mathcal A_{n - 2}|}{\binom{n}{n - 2}} \leq 1,
  \]
  so
  \[
    \frac{|\shadow \mathcal A_n \cup \mathcal A_{n - 1}|}{\binom{n}{n - 1}} + \frac{|\mathcal A_{n - 1}|}{\binom{n}{n - 2}} \leq 1,
  \]
  so
  \[
    \frac{|\mathcal A_n|}{\binom{n}{n}} + \frac{|\mathcal A_{n - 1}|}{\binom{n}{n - 1}} + \frac{|\mathcal A_{n - 2}|}{\binom{n}{n - 2}} \leq 1
  \]
  Keep going we have the result desired.
\end{proof}

Again we can ask when we have equality in LYM inequality. This happens if and only if we have equality in each use of local LYM, so the ``first'' (greatest) \(r\) with \(\mathcal A_r \neq \emptyset\) must have \(\mathcal A_r = X^{(r)}\) so \(\mathcal A = X^{(r)}\). Thus we know equality in Sperner's lemma (\(\mathcal A = X^{(n/2)}\) for \(n\) even, similar for \(n\) odd)

\begin{proof}[Proof 2]
  Choose uniformly at random a maximal chain \(\mathcal C\) (i.e.\ \(C_0 \subseteq C_1 \subseteq \dots \subseteq C_n\) with \(|C_i| = i\)). For a given \(r\)-set \(A\), \(\P(A \in \mathcal C) = \binom{n}{r}^{-1}\) as all \(r\)-sets are equally likely. Therefore
  \[
    \P(\mathcal A_r \text{ meets } \mathcal C) = \frac{|\mathcal A_r|}{\binom{n}{r}}
  \]
  since the events are disjoint. Furthermore for different \(r\) the events that \(\mathcal A\) meets \(X^r\) are also disjoint so
  \[
    \P(\mathcal A \text{ meets } \mathcal C) = \sum_{r = 0}^n \frac{|\mathcal A_r|}{\binom{n}{r}}
  \]
  and of course it is less than \(1\).
\end{proof}

%how do we think of the first line (from which the rest follows)? inspired by the first proof of LYM %maybe no?

\begin{remark}
  Equivalently, the number of maximal chains is \(n!\) and the number of them containing a given \(r\)-set is \(r! (n - r)!\), so
  \[
    \sum_{r = 0}^n |\mathcal A_r| r! (n - r)! \leq n!
  \]
  so this is probability in disguise
\end{remark}

\section{Shadows}

For \(\mathcal A \subseteq X^{(r)}\), we know \(|\shadow A| \geq |A| \frac{r}{n - r + 1}\) --- but equality is rare (only for \(\mathcal A = \emptyset\) or \(\mathcal A = X^{(r)}\)). The natural question is: given \(|\mathcal A|\), how should we choose \(\mathcal A \subseteq X^{(r)}\) to minimise \(|\shadow A|\)? Informally, this asks how ``tightly'' can we pack some \(r\)-sets.

If \(|A| = \binom{k}{r}\), it is believable that we would take \(A = [k]^{(r)}\), which gives \([k]^{(r - 1)}\). What if \(\binom{k}{r} < |\mathcal A| <\binom{k + 1}{r}\), it is believable that we'd take \([k]^{(r)}\) and some other \(r\)-sets from \([k + 1]^{(r)}\). For example if \(\mathcal A \subseteq X^{(3)}\) with \(|\mathcal A| = \binom{7}{3} + \binom{4}{2}\), we would take
\[
  \mathcal A = [7]^{(3)} \cup \{A \cup \{8\}: A \in [4]^{(2)}\},
\]
i.e.\ take those in \([7]\) of size \(3\) so that they are as tightly packed as possible and then choose some other stuff.

If we increment the size of \(\mathcal A\) by \(1\), it is believable that we should take the above \(\mathcal A\) and adjoin another element from \([4]\). Thus it seems that there is a total order on subsets of \(X\) of a given size, and we just take the first \(|\mathcal A|\).

\subsection{Two total orderings on \(X^{(r)}\)}

\begin{definition}[lexicographic order]\index{lexicographic order}
  Given \(A, B \in X^{(r)}\), say \(A = a_1, \dots, a_r, B = b_1, \dots, b_r\), where we use the notation to mean \(a_1 < \dots < a_r\), say \(A < B\) in the \emph{lexicographic} or \emph{lex} order if for some \(i\) have \(a_i < b_i\) and \(a_j = b_j\) for all \(j < i\).
\end{definition}

Equivalently, \(a_i < b_i\), where \(i = \min\{j: a_j \neq b_i\}\). The slogan is ``use small number''.

\begin{eg}
  Lex on \([4]^{(2)}\):
  \[
    12, 13, 14, 23, 24, 34.
  \]

  Lex on \([6]^{(3)}\):
  \begin{align*}
    &123, 124, 125, 126, 134, 135, 136, 145, 146, 156, \\
    &234, 235, 236, 245, 246, 256, 345, 346, 356, 456.
  \end{align*}
\end{eg}

However, in the shadow minimisation problem we want to avoid large numbers as much as possible, i.e.\ keep the largest number as small as possible.

\begin{definition}[colexicographic order]\index{colexicographic order}
  Say \(A < B\) isn the \emph{colexicographic} or \emph{colex} order if for some \(i\) have \(a_i < b_i\) and \(a_j = b_j\) for all \(j > i\).
\end{definition}

Equivalently, \(a_i < b_i\) where \(i = \max\{j: a_j \neq b_j\}\). The slogan is ``avoid large number''. Equivalently, \(A < B\) if \(\sum_{i \in A} 2^i < \sum_{i \in B} 2^i\).

\begin{eg}
  Colex on \([4]^{(2)}\):
  \[
    12, 13, 23, 14, 24, 34.
  \]

  Colex on \([6]^{(2)}\):
  \begin{align*}
    &123, 124, 134, 234, 125, 135, 235, 145, 245, 345,\\
    &126, 136, 236, 146, 246, 346, 156, 256, 356, 456.
  \end{align*}
\end{eg}

\begin{note}
  In colex, \([k]^{(r)}\) is an initial segment of \([k + 1]^{(r)}\), meaning that it is the first \(t\) elements for some \(t\). Therefore we could view colex as an enumeration of \(\N^{(r)}\). Try this with lex and see what happens!
\end{note}

Following our heuristics just now, the aim is to show initial segments of cloex minimise \(\shadow\), i.e.\ if \(\mathcal A \subseteq X^{(r)}\) and \(\mathcal C \subseteq X^{(r)}\) is the first \(|\mathcal A|\) \(r\)-sets in colex then \(|\shadow \mathcal A| \geq |\shadow \mathcal C|\), This is Kruskal-Katona theorem, the first theorem in combinatorics.
In particular, \(|\mathcal A| = \binom{k}{r}\) implies \(|\shadow \mathcal A| \geq \binom{k}{r - 1}\). However, unless \(\mathcal A\) is written in very nice form, it is very difficult to estimate \(\shadow \mathcal A\).

\subsection{Compressions}

The idea is to ``replace'' \(\mathcal A \subseteq X^{(r)}\) with some \(\mathcal A' \subseteq X^{(r)}\) such that
\begin{enumerate}
\item \(|\mathcal A'| = |\mathcal A|\),
\item \(|\shadow \mathcal A'| \leq |\shadow \mathcal A|\),
\item \(\mathcal A'\) ``looks more like \(\mathcal C\)'' than \(\mathcal A\) did.
\end{enumerate}

Ideally, we would compress
\[
  \mathcal A \to \mathcal A' \to \mathcal A'' \to \dots \to \mathcal B
\]
where either \(\mathcal B = \mathcal C\), or \(\mathcal B\) is so similar to \(\mathcal C\) that we can see directly that \(|\shadow \mathcal B| \geq |\shadow \mathcal C|\).

\paragraph{Colex prefers \(1\) to \(2\)}

\begin{definition}[\(ij\)-compression]\index{\(ij\)-compression}
  For \(1 \leq i < j \leq n\), the \emph{\(ij\)-compression} \(C_{ij}\) is defined by for \(A \subseteq X\),
  \[
    C_{ij} (A) =
    \begin{cases}
      A - j + i & \text{if } j \in A, i \notin A \\
      A & \text{otherwise}
    \end{cases}
  \]
  and for \(\mathcal A \subseteq \powerset(X)\),
  \[
    C_{ij} (\mathcal A) = \{C_{ij}(A): A \in \mathcal A\} \cup \{A \in \mathcal A: C_{ij}(A) \in \mathcal A\}.
  \]
\end{definition}

\begin{eg}
  If \(\mathcal A = \{123, 134, 234, 235, 247\}\), then
  \[
    C_{12}(\mathcal A) = \{123, 134, 234, 135, 147\}.
  \]
\end{eg}

\(|C_{ij}(\mathcal A)| = |\mathcal A|\) and after the compression it looks ``more'' like colex than lex. Say \(\mathcal A\) is \emph{\(ij\)-compressed} if \(C_{ij}(\mathcal A) = \mathcal A\). It is also intuitively obvious that the operation is indeed a compression in the sense that it decreases the shadow.

\begin{proposition}
  \label{prop:ij-compression}
  Let \(\mathcal A \subseteq X^{(r)}\), \(1 \leq i < j \leq n\) then
  \[
    |\shadow C_{ij}(\mathcal A)| \leq |\shadow \mathcal A|.
  \]
\end{proposition}

\begin{proof}
  Write \(\mathcal A'\) for \(C_{ij} (\mathcal A)\). We'll show that if \(B \in \shadow \mathcal A' - \shadow \mathcal A\) then \(i \in B, j \neq B\) and \(B \cup j - i \in \shadow \mathcal A - \shadow \mathcal A'\) (so \(B\) has a preimage under \(C_{ij}\)). Then done.

  We have \(B \cup x \in \mathcal A'\) for some \(x \notin B\) and \(B \cup x \notin \mathcal A\). Hence \(i \in B \cup x, j \notin B \cup x\) and \((B \cup x) \cup j - i \in \mathcal A\) (so \(B \cup x\) is the set that has been compressed). Note that \(x \neq i\) else \(B \cup j \in \mathcal A\), contradicting \(B \notin \shadow \mathcal A\).

  Certainly \(B \cup j - i \in \shadow \mathcal A\). Claim that \(B \cup j - i \notin \shadow \mathcal A'\), thereby completing the proof: suppose \((B \cup j - i) \cup y \in \mathcal A'\). We cannot have \(y = i\), else \(B \cup j \in \mathcal A'\), whence \(B \cup j \in \mathcal A\), contradiction. Thus \(j \in (B \cup j - i) \cup y\) and \(i \notin (B \cup j - i) \cup y\), so \((B \cup j - i) \cup y \in \mathcal A\) and \(B \cup y \in \mathcal A\) by definition of \(C_{ij}\). Contradiction.
\end{proof}

\begin{remark}
  We have actually showed that
  \[
    \shadow C_{ij}(\mathcal A) \subseteq C_{ij} (\shadow \mathcal A),
  \]
  ``shadow of compression lives inside compression of shadow''.
\end{remark}

\begin{definition}[left-compressed]\index{left-compressed}
  Say \(\mathcal A \subseteq X^{(r)}\) is \emph{left-compressed} if \(C_{ij}(\mathcal A) = \mathcal A\) for all \(i < j\).
\end{definition}

\begin{proposition}
  \label{prop:left compression decreases shadow}
  Let \(\mathcal A \subseteq X^{(r)}\). Then there exists left-compressed \(\mathcal B \subseteq X^{(r)}\) with \(|\mathcal B| = |\mathcal A|\) and \(|\shadow \mathcal B| \leq |\shadow \mathcal A|\).
\end{proposition}

\begin{proof}
  Morally we only have to show that any sequence of compression terminate. Among all \(\mathcal B \subseteq X^{(r)}\) with \(|\mathcal B| = |\mathcal A|\) and \(|\shadow \mathcal B| \leq |\shadow \mathcal A|\), choose one with
  \[
    \sum_{A \in \mathcal B} \sum_{x \in A} x
  \]
  minimal. Then \(\mathcal B\) is left compressed, as if \(C_{ij}(\mathcal B) \neq \mathcal B\) then we contradict minimality.
\end{proof}

\begin{note}\leavevmode
  \begin{enumerate}
  \item Alternatively, we may apply one \(C_{ij}\) then another and so on --- it must terminate.
  \item In fact we can apply each \(C_{ij}\) at most once if we choose a sensible order.
  \end{enumerate}
\end{note}

Certainly initial segments of colex are left-compressed. The converse is blatantly false, e.g.\ \(A = \{123, 124, 125, 126, 127\}\). We need to do more.

\paragraph{colex prefers \(23\) to \(14\)}

We can compress not only singletons but also sets of larger sizes:

\begin{definition}[\(UV\)-compression]\index{\(UV\)-compression}
  For \(U, V \subseteq X\) with \(|U| = |V|\) and \(U \cap V = \emptyset\), the \emph{\(UV\)-compression} \(C_{UV}\) is defined by for \(A \subseteq X\),
  \[
    C_{UV} (A) =
    \begin{cases}
      A \cup U - V & \text{if } V \subseteq A, U \cap A = \emptyset \\
      A & \text{otherwise}
    \end{cases}
  \]
  and for \(\mathcal A \subseteq X^{(r)}\),
  \[
    C_{UV} (\mathcal A) = \{C_{UV}(A): A \in \mathcal A\} \cup \{A \in \mathcal A: C_{UV}(A) \in \mathcal A\}.
  \]
\end{definition}

\begin{eg}
  If \(\mathcal A = \{123, 134, 235, 145, 146, 157\}\), then
  \[
    C_{23, 14}(\mathcal A) = \{123, 134, 235, 145, 236, 157\}.
  \]
\end{eg}

Similar observation: \(|C_{UV}(\mathcal A)| = |\mathcal A|\). Say \(\mathcal A\) is \emph{\(UV\)-compressed if \(C_{UV}(\mathcal A) =\mathcal A\)}. Sadly, \(C_{UV}\) need \emph{not} decrease shadow --- e.g.\ \(\mathcal A = \{146, 467\}\) then \(C_{23, 14}(\mathcal A) = \{236, 467\}\) so \(|\shadow \mathcal A| = 5, |\shadow C_{23, 14}(\mathcal A)| = 6\). Intuitively, removing one elements and adding another sends a set to a ``close neighbour'' which shares largely the same shadow, but once we start doing \(UV\)-compression, some things are ``moved a long way''.

However, note that \(\mathcal A\) in the above example is not left-compressed. It turns out once you have done the ``smaller'' compressions, doing a larger compression always decrease the shadow. Formally,

\begin{proposition}
  \label{prop:UV-compression}
  Let \(\mathcal A \subseteq X^{(r)}\) and \(U,V \subseteq X\) with \(|U| = |V|\) and \(U \cap V = \emptyset\). Suppose that for all \(x \in U\) exists \(y \in V\) such that \(\mathcal A\) is \((U - x, V - y)\)-compressed then
  \[
    |\shadow C_{UV}(\mathcal A)| \leq |\shadow \mathcal A|.
  \]
\end{proposition}

\begin{proof}
  Write \(\mathcal A'\) for \(C_{UV}(\mathcal A)\). Given \(B \in \shadow \mathcal A' - \shadow \mathcal A\), we'll show that \(U \subseteq B, V \cap B = \emptyset\) and \(B \cup V - U \in \shadow \mathcal A - \shadow \mathcal A'\).

  We have \(B \cup x \in \mathcal A'\) for some \(x \notin B\), with \(B \cup x \notin \mathcal A\). So \(U \subseteq B \cup x, V \cap (B \cup x) = \emptyset\) and \((B \cup x) \cup V - U \in \mathcal A\). Thus certainly \(V \cap B = \emptyset\).

  If \(x \in U\), have \(\mathcal A\) is \((U - x, V - y)\)-compressed for some \(y \in V\) so from \((B \cup x) \cup V - U \in \mathcal A\) we obtain \(B \cup y \in \mathcal A\), contradicting \(B \notin \shadow \mathcal A\). Hence \(x \notin U\) and so \(U \subseteq B\). Also \(B \cup V - U \in \shadow \mathcal A\) (as \((B \cup x) \cup V - U \in \mathcal A\)).

  Suppose \(B \cup V - U \in \shadow \mathcal A'\) then \((B \cup V - U) \cup w \in \mathcal A'\) for some \(w\).
  \begin{enumerate}
  \item if \(w \notin U\), then \(V \subseteq (B \cup V - U) \cup w\) and \(U \cap ((B \cup V - U) \cup w) = \emptyset\) so from \((B \cup V - U) \cup w \in \mathcal A'\) we conclude that both \((B \cup V - U) \cup w \in \mathcal A\) and \(B \cup w \in \mathcal A\), contradicting \(B \notin \shadow \mathcal A\).
  \item if \(w \in U\), we have \(\mathcal A\) is \((U - w, V - z)\)-compressed for some \(z \in V\). From \((B \cup V - U) \cup w \in \mathcal A\) (as it is in \(\mathcal A'\) and contains \(V\), so could not have moved), we deduce \(B \cup z \in \mathcal A\), contradicting \(B \notin \shadow \mathcal A\).
  \end{enumerate}
\end{proof}

\begin{remark}
  We have actually showed that
  \[
    \shadow C_{UV}(\mathcal A) \subseteq C_{UV}(\shadow \mathcal A).
  \]
\end{remark}

\begin{theorem}[Krustal-Katona]\index{Krustal-Katona theorem}
  \label{thm:Krustal-Katona}
  Let \(\mathcal A \subseteq X^{(r)}\) where \(1 \leq r \leq n\) and let \(\mathcal C\) be the initial segment of colex on \(X^{(r)}\) with \(|\mathcal C| = |\mathcal A|\), then
  \[
    |\shadow \mathcal A| \geq |\shadow \mathcal C|.
  \]
  In particular if \(|\mathcal A| = \binom{k}{r}\) then
  \[
    |\shadow \mathcal A| \geq \binom{k}{r - 1}.
  \]
\end{theorem}

\begin{proof}
  Let
  \[
    \Gamma = \{(U, V): U, V \subseteq X, |U| = |V| > 0, U \cap V = \emptyset, \max U < \max V\}
  \]
  which are the ordered pairs \((U, V)\) with \(U < V\) in colex, which are exactly the sensible pairs to do \(UV\) compression on in order to decrease shadow. Define a sequence of set systems \(\mathcal A_0, \mathcal A_1, \dots\) in \(X^{(r)}\) as follow:
  \begin{enumerate}
  \item \(\mathcal A_0 = \mathcal A\).
  \item If \(\mathcal A_k\) is \((U, V)\)-compressed for all \((U, V) \in \Gamma\), then stop the sequence with \(\mathcal A_k\).
  \item If not, choose \((U, V) \in \Gamma\) such that \(\mathcal A_k\) is not \((U, V)\)-compressed with \(U\) \emph{minimal}. Set \(\mathcal A_{k + 1} = C_{UV}(\mathcal A_K)\). Note that for all \(x \in U\), we have \((U - x, V - y) \in \Gamma \cup \{(\emptyset, \emptyset)\}\) for \(y = \min V\) so by \Cref{prop:UV-compression} have \(|\shadow \mathcal A_{k + 1}| \leq |\shadow \mathcal A_k|\).
  \end{enumerate}

  The sequence must terminate, for example because as \(\sum_{A \in \mathcal A_k}\sum_{i \in A} 2^i\) is decreasing in \(k\).

  The final system \(\mathcal B = \mathcal A_k\) satisfies \(|\mathcal B| = |\mathcal A|\) and \(|\shadow B| \leq |\shadow \mathcal A|\). Moreover, \(\mathcal B\) is \((U, V)\)-compressed for all \((U, V) \in \Gamma\). Claim that \(\mathcal B = \mathcal C\):
  \begin{proof}
    Suppose \(\mathcal B\) is not an initial segment of colex. Then there exists \(A < B\) in colex with \(A \notin \mathcal B\) and \(B \in \mathcal B\). But then \(U = A - B\) and \(V = B - A\) have \((U, V) \in \Gamma\) and \(C_{UV}(B) = A\). Absurd.
  \end{proof}
\end{proof}

\begin{remark}\leavevmode
  \begin{enumerate}
  \item Equivalently, we may state the theorem in numerical form by translating initial segments of colex into the size of its constituent ``lex'' parts: if \(\mathcal A \subseteq X^{(r)}\) with
    \[
      |\mathcal A| = \binom{k_r}{r} + \binom{k_{r - 1}}{r - 1} + \dots + \binom{k_s}{s}
    \]
    where \(k_r > k_{r - 1} > \dots > k_s\) and \(s > 0\), then
    \[
      |\shadow \mathcal A| \geq \binom{k_r}{k - 1} + \binom{k_{r - 1}}{r - 2} + \dots + \binom{k_s}{s - 1}.
    \]
  \item In proof of Krustal-Katona we used only \Cref{prop:UV-compression}, but neither \Cref{prop:ij-compression} nor \Cref{prop:left compression decreases shadow}. However, deriving simpler results for \(ij\)-compression provides motivation and intuition for the further full-fledged proof.
  \item Can we ask for uniqueness? We can check that if \(|\shadow \mathcal A| = |\shadow \mathcal C|\) and \(|\mathcal A| = \binom{k}{r}\) then \(\mathcal A = Y^{(r)}\) for some \(k\)-set \(Y\). Thus this is unique up to isomorphism. But in general it is not true that \(|\shadow \mathcal A| = |\shadow \mathcal C|\) implies \(\mathcal A\) is isomorphic to \(\mathcal C\) (\(\mathcal A \subseteq \powerset(X), \mathcal B \subseteq \powerset(Y)\) are isomorphic if there exists a bijection between \(X\) and \(Y\) sending \(\mathcal A\) to \(\mathcal B\)).
  \end{enumerate}
\end{remark}

What about upper shadow?

\begin{definition}[upper shadow]\index{upper shadow}
  For \(\mathcal A \subseteq X^{(r)}\) where \(0 \leq r \leq n - 1\), the \emph{upper shadow} of \(\mathcal A\) is
  \[
    \shadow^+ \mathcal A = \{A \cup x: A \in \mathcal A, x \notin A\}.
  \]
\end{definition}

Note that \(A < B\) in colex if and only if \(A^c < B^c\) in lex with ground-set order reversed. You can mess around with complement and other set operations, but think about it until it becomes clear!

\begin{corollary}
  Let \(\mathcal A \subseteq X^{(r)}\) wher \(0 \leq r \leq n - 1\) and let \(\mathcal C\) be the initial segment of lex with \(|\shadow C| = |\mathcal A|\). Then
  \[
    |\shadow^+ \mathcal A| \geq |\shadow^+ \mathcal C|.
  \]
\end{corollary}

\begin{proof}
  Take complements.
\end{proof}

Also the shadow of an initial segment of colex is again an initial segment of colex. Indeed, if
\[
  \mathcal C = \{A \subseteq X^{(r)}: A \leq a_1,a_2,\dots,a_r\}
\]
then
\[
  \shadow \mathcal C = \{B \subseteq X^{(r - 1)}: B \leq a_2,\dots,a_r\}.
\]

Krustal-Katona also proves a generalised version of itself:

\begin{corollary}
  Let \(\mathcal A \subseteq X^{(r)}\) and let \(\mathcal C \subseteq X^{(r)}\) be the initial segment of colex with \(|\mathcal C| = |\mathcal A|\). Then
  \[
    |\shadow^t \mathcal A| \geq |\shadow^t \mathcal C|
  \]
  for all \(1 \leq t \leq r\). In particular if \(|\mathcal A| = \binom{k}{r}\) then
  \[
    |\shadow^t \mathcal A| \geq \binom{k}{r - t}.
  \]
\end{corollary}

\begin{proof}
  If \(|\shadow^t \mathcal A| \geq |\shadow^t \mathcal C|\) then \(|\shadow^{t + 1}\mathcal A| \geq |\shadow^{t + 1} \mathcal C|\) by \nameref{thm:Krustal-Katona}.
\end{proof}








\printindex
\end{document}

% Three parts:
% Chapter 1: set systems
% Chapter 2: isoperimetric inequalities
% Chapter 3: projections
% Books: Combinatorics, Bollobas, CUP 1986, excellent for Chapter 1 and Chapter 2 (and gentle!) and for future development of the course;
% Combinatorics of finite sets, Anderson, OUP 1987, simple and clear, good for Chapter 1