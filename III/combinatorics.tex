\documentclass[a4paper]{article}

\def\npart{III}

\def\ntitle{Combinatorics}
\def\nlecturer{I.\ B.\ Leader}

\def\nterm{Michaelmas}
\def\nyear{2018}

\ifx \nauthor\undefined
  \def\nauthor{Qiangru Kuang}
\else
\fi

\ifx \ntitle\undefined
  \def\ntitle{Template}
\else
\fi

\ifx \nauthoremail\undefined
  \def\nauthoremail{qk206@cam.ac.uk}
\else
\fi

\ifx \ndate\undefined
  \def\ndate{\today}
\else
\fi

\title{\ntitle}
\author{\nauthor}
\date{\ndate}

%\usepackage{microtype}
\usepackage{mathtools}
\usepackage{amsthm}
\usepackage{stmaryrd}%symbols used so far: \mapsfrom
\usepackage{empheq}
\usepackage{amssymb}
\let\mathbbalt\mathbb
\let\pitchforkold\pitchfork
\usepackage{unicode-math}
\let\mathbb\mathbbalt%reset to original \mathbb
\let\pitchfork\pitchforkold

\usepackage{imakeidx}
\makeindex[intoc]

%to address the problem that Latin modern doesn't have unicode support for setminus
%https://tex.stackexchange.com/a/55205/26707
\AtBeginDocument{\renewcommand*{\setminus}{\mathbin{\backslash}}}
\AtBeginDocument{\renewcommand*{\models}{\vDash}}%for \vDash is same size as \vdash but orginal \models is larger
\AtBeginDocument{\let\Re\relax}
\AtBeginDocument{\let\Im\relax}
\AtBeginDocument{\DeclareMathOperator{\Re}{Re}}
\AtBeginDocument{\DeclareMathOperator{\Im}{Im}}
\AtBeginDocument{\let\div\relax}
\AtBeginDocument{\DeclareMathOperator{\div}{div}}

\usepackage{tikz}
\usetikzlibrary{automata,positioning}
\usepackage{pgfplots}
%some preset styles
\pgfplotsset{compat=1.15}
\pgfplotsset{centre/.append style={axis x line=middle, axis y line=middle, xlabel={$x$}, ylabel={$y$}, axis equal}}
\usepackage{tikz-cd}
\usepackage{graphicx}
\usepackage{newunicodechar}

\usepackage{fancyhdr}

\fancypagestyle{mypagestyle}{
    \fancyhf{}
    \lhead{\emph{\nouppercase{\leftmark}}}
    \rhead{}
    \cfoot{\thepage}
}
\pagestyle{mypagestyle}

\usepackage{titlesec}
\newcommand{\sectionbreak}{\clearpage} % clear page after each section
\usepackage[perpage]{footmisc}
\usepackage{blindtext}

%\reallywidehat
%https://tex.stackexchange.com/a/101136/26707
\usepackage{scalerel,stackengine}
\stackMath
\newcommand\reallywidehat[1]{%
\savestack{\tmpbox}{\stretchto{%
  \scaleto{%
    \scalerel*[\widthof{\ensuremath{#1}}]{\kern-.6pt\bigwedge\kern-.6pt}%
    {\rule[-\textheight/2]{1ex}{\textheight}}%WIDTH-LIMITED BIG WEDGE
  }{\textheight}% 
}{0.5ex}}%
\stackon[1pt]{#1}{\tmpbox}%
}

%\usepackage{braket}
\usepackage{thmtools}%restate theorem
\usepackage{hyperref}

% https://en.wikibooks.org/wiki/LaTeX/Hyperlinks
\hypersetup{
    %bookmarks=true,
    unicode=true,
    pdftitle={\ntitle},
    pdfauthor={\nauthor},
    pdfsubject={Mathematics},
    pdfcreator={\nauthor},
    pdfproducer={\nauthor},
    pdfkeywords={math maths \ntitle},
    colorlinks=true,
    linkcolor={red!50!black},
    citecolor={blue!50!black},
    urlcolor={blue!80!black}
}

\usepackage{cleveref}



% TODO: mdframed often gives bad breaks that cause empty lines. Would like to switch to tcolorbox.
% The current workaround is to set innerbottommargin=0pt.

%\usepackage[theorems]{tcolorbox}





\usepackage[framemethod=tikz]{mdframed}
\mdfdefinestyle{leftbar}{
  %nobreak=true, %dirty hack
  linewidth=1.5pt,
  linecolor=gray,
  hidealllines=true,
  leftline=true,
  leftmargin=0pt,
  innerleftmargin=5pt,
  innerrightmargin=10pt,
  innertopmargin=-5pt,
  % innerbottommargin=5pt, % original
  innerbottommargin=0pt, % temporary hack 
}
%\newmdtheoremenv[style=leftbar]{theorem}{Theorem}[section]
%\newmdtheoremenv[style=leftbar]{proposition}[theorem]{proposition}
%\newmdtheoremenv[style=leftbar]{lemma}[theorem]{Lemma}
%\newmdtheoremenv[style=leftbar]{corollary}[theorem]{corollary}

\newtheorem{theorem}{Theorem}[section]
\newtheorem{proposition}[theorem]{Proposition}
\newtheorem{lemma}[theorem]{Lemma}
\newtheorem{corollary}[theorem]{Corollary}
\newtheorem{axiom}[theorem]{Axiom}
\newtheorem*{axiom*}{Axiom}

\surroundwithmdframed[style=leftbar]{theorem}
\surroundwithmdframed[style=leftbar]{proposition}
\surroundwithmdframed[style=leftbar]{lemma}
\surroundwithmdframed[style=leftbar]{corollary}
\surroundwithmdframed[style=leftbar]{axiom}
\surroundwithmdframed[style=leftbar]{axiom*}

\theoremstyle{definition}

\newtheorem*{definition}{Definition}
\surroundwithmdframed[style=leftbar]{definition}

\newtheorem*{slogan}{Slogan}
\newtheorem*{eg}{Example}
\newtheorem*{ex}{Exercise}
\newtheorem*{remark}{Remark}
\newtheorem*{notation}{Notation}
\newtheorem*{convention}{Convention}
\newtheorem*{assumption}{Assumption}
\newtheorem*{question}{Question}
\newtheorem*{answer}{Answer}
\newtheorem*{note}{Note}
\newtheorem*{application}{Application}

%operator macros

%basic
\DeclareMathOperator{\lcm}{lcm}

%matrix
\DeclareMathOperator{\tr}{tr}
\DeclareMathOperator{\Tr}{Tr}
\DeclareMathOperator{\adj}{adj}

%algebra
\DeclareMathOperator{\Hom}{Hom}
\DeclareMathOperator{\End}{End}
\DeclareMathOperator{\id}{id}
\DeclareMathOperator{\im}{im}
\DeclareMathOperator{\coker}{coker}
\DeclarePairedDelimiter{\generation}{\langle}{\rangle}

%groups
\DeclareMathOperator{\sym}{Sym}
\DeclareMathOperator{\sgn}{sgn}
\DeclareMathOperator{\inn}{Inn}
\DeclareMathOperator{\aut}{Aut}
\DeclareMathOperator{\GL}{GL}
\DeclareMathOperator{\SL}{SL}
\DeclareMathOperator{\PGL}{PGL}
\DeclareMathOperator{\PSL}{PSL}
\DeclareMathOperator{\SU}{SU}
\DeclareMathOperator{\UU}{U}
\DeclareMathOperator{\SO}{SO}
\DeclareMathOperator{\OO}{O}
\DeclareMathOperator{\PSU}{PSU}
\DeclareMathOperator{\Sp}{Sp}


%hyperbolic
\DeclareMathOperator{\sech}{sech}

%field, galois heory
\DeclareMathOperator{\ch}{ch}
\DeclareMathOperator{\gal}{Gal}
\DeclareMathOperator{\emb}{Emb}



%ceiling and floor
%https://tex.stackexchange.com/a/118217/26707
\DeclarePairedDelimiter\ceil{\lceil}{\rceil}
\DeclarePairedDelimiter\floor{\lfloor}{\rfloor}


\DeclarePairedDelimiter{\innerproduct}{\langle}{\rangle}

%\DeclarePairedDelimiterX{\norm}[1]{\lVert}{\rVert}{#1}
\DeclarePairedDelimiter{\norm}{\lVert}{\rVert}



%Dirac notation
%TODO: rewrite for variable number of arguments
\DeclarePairedDelimiterX{\braket}[2]{\langle}{\rangle}{#1 \delimsize\vert #2}
\DeclarePairedDelimiterX{\braketthree}[3]{\langle}{\rangle}{#1 \delimsize\vert #2 \delimsize\vert #3}

\DeclarePairedDelimiter{\bra}{\langle}{\rvert}
\DeclarePairedDelimiter{\ket}{\lvert}{\rangle}




%macros

%general

%divide, not divide
\newcommand*{\divides}{\mid}
\newcommand*{\ndivides}{\nmid}
%vector, i.e. mathbf
%https://tex.stackexchange.com/a/45746/26707
\newcommand*{\V}[1]{{\ensuremath{\symbf{#1}}}}
%closure
\newcommand*{\cl}[1]{\overline{#1}}
%conjugate
\newcommand*{\conj}[1]{\overline{#1}}
%set complement
\newcommand*{\stcomp}[1]{\overline{#1}}
\newcommand*{\compose}{\circ}
\newcommand*{\nto}{\nrightarrow}
\newcommand*{\p}{\partial}
%embed
\newcommand*{\embed}{\hookrightarrow}
%surjection
\newcommand*{\surj}{\twoheadrightarrow}
%power set
\newcommand*{\powerset}{\mathcal{P}}

%matrix
\newcommand*{\matrixring}{\mathcal{M}}

%groups
\newcommand*{\normal}{\trianglelefteq}
%rings
\newcommand*{\ideal}{\trianglelefteq}

%fields
\renewcommand*{\C}{{\mathbb{C}}}
\newcommand*{\R}{{\mathbb{R}}}
\newcommand*{\Q}{{\mathbb{Q}}}
\newcommand*{\Z}{{\mathbb{Z}}}
\newcommand*{\N}{{\mathbb{N}}}
\newcommand*{\F}{{\mathbb{F}}}
%not really but I think this belongs here
\newcommand*{\A}{{\mathbb{A}}}

%asymptotic
\newcommand*{\bigO}{O}
\newcommand*{\smallo}{o}

%probability
\newcommand*{\prob}{\mathbb{P}}
\newcommand*{\E}{\mathbb{E}}

%vector calculus
\newcommand*{\gradient}{\V \nabla}
\newcommand*{\divergence}{\gradient \cdot}
\newcommand*{\curl}{\gradient \cdot}

%logic
\newcommand*{\yields}{\vdash}
\newcommand*{\nyields}{\nvdash}

%differential geometry
\renewcommand*{\H}{\mathbb{H}}
\newcommand*{\transversal}{\pitchfork}
\renewcommand{\d}{\mathrm{d}} % exterior derivative

%number theory
\newcommand*{\legendre}[2]{\genfrac{(}{)}{}{}{#1}{#2}}%Legendre symbol

%algebraic geometry
\DeclareMathOperator{\Spec}{Spec}
\DeclareMathOperator{\Proj}{Proj}

\begin{document}

\begin{titlepage}
  \begin{center}
    \includegraphics[width=0.6\textwidth]{logo.jpg}\par
    \vspace{1cm}
    {\scshape\huge Mathamatics Tripos \par}
    \vspace{2cm}
    {\huge Part \npart \par}
    \vspace{0.6cm}
    {\Huge \bfseries \ntitle \par}
    \vspace{1.2cm}
    {\Large\nterm, \nyear \par}
    \vspace{2cm}
    
    {\large \emph{Lectures by } \par}
    \vspace{0.2cm}
    {\Large \scshape \nlecturer}
    
    \vspace{0.5cm}
    {\large \emph{Notes by }\par}
    \vspace{0.2cm}
    {\Large \scshape \href{mailto:\nauthoremail}{\nauthor}}
 \end{center}
\end{titlepage}

\section{Set Systems}

\begin{definition}[set system]\index{set system}
  Let \(X\) be a set. A \emph{set system} on \(X\) (or \emph{family of subsets}) of \(X\) is a family \(\mathcal A \subseteq \powerset (X)\).
\end{definition}

\begin{eg}
  We write \(X^{(r)} = \{A \subseteq X: |A| = r\}\)
\end{eg}

In this course we almost deal exclusively with finite setss so unless otherwise stated, in this course we assume all sets are finite and \(X = [n] = \{1, 2, \dots, n\}\).

For example, \(|X^{(r)}| = \binom{n}{r}\). More concretely, for example,
\[
  [4]^{[2]} = \{12, 13, 14, 23, 24, 34\}
\]
where \(12\) denotes \(\{1, 2\}\) to avoid heavy notation. Therefore \(|[4]^{[2]}| = 6\).

What is the mental picture for power set? Often we make \(\powerset (X)\) into a graph, called \(Q_n\), by joining \(A\) to \(B\) if \(|A \Delta B| = 1\) where \(\Delta\) is the symmetric difference, i.e.\ if \(A = B \cup \{i\}\) for some \(i \notin B\) (or vice verse).

\begin{eg}
  \(Q_3\)
\end{eg}

\begin{eg}
  General picture for \(Q_n\) where \(n\) is even:

  and when \(n\) is odd:
\end{eg}

If we identify a set \(A \subseteq X\) with a \(\{0, 1\}\) sequence of length \(n\) (e.g.\ \(134 \leftrightarrow 1011000 \cdots 0\)), via \(A \leftrightarrow \mathbf{1}_A \text{ or } \chi_A\). In this way, we can represent \(Q_n\) as a \(n\)-dimensional cube:

For this reason, \(Q_n\) is often called the \emph{hypercube} or \emph{discrete cube} or \emph{\(n\)-cube}. In this way, the study of a set system become the study of a graph.

\subsection{Chains \& Antichains}

\begin{definition}[chain]\index{chain}
  A family \(\mathcal A \subseteq \powerset (X)\) is a \emph{chain} if for all \(A, B \in \mathcal A, A \subseteq B \text{ or } B \subseteq A\).
\end{definition}

\begin{eg}
  \(\{12, 125, 123589\}\).
\end{eg}

On the other hand, we have

\begin{definition}[antichain]\index{antichain}
  A family \(\mathcal A \subseteq \powerset (X)\) is an \emph{antichain} if for all \(A, B \in \mathcal A\) with \(A \neq B\), \(A \nsubseteq B\).
\end{definition}

\begin{eg}
  \(\{1, 467, 2456\}\).
\end{eg}

A natural question is: how large can a chain be? Obviously we can achieve \(|\mathcal A| = n + 1\). We also cannot exceed \(n + 1\) since a chain must meet each ``level'' \(X^{(r)}\) for \(0 \leq r \leq n\) in at most one place.

A less trivial question is how large an antichain can be. We could achieve \(|\mathcal A| = n\) by enumerating all the singletons. It is maximal since adjoining any nonempty set to the family will result in an inclusion. However, it is not the largest antichain. Indeed, we could take \(\mathcal A = X^{(r)}\) for any \(r\). Thus we can achieve \(|\mathcal A| = \binom{n}{\floor{n/2}}\). Can we beat it?

\printindex
\end{document}

% Three parts:
% Chapter 1: set systems
% Chapter 2: isoperimetric inequalities
% Chapter 3: projections
% Books: Combinatorics, Bollobas, CUP 1986, excellent for Chapter 1 and Chapter 2 (and gentle!) and for future development of the course;
% Combinatorics of finite sets, Anderson, OUP 1987, simple and clear, good for Chapter 1

\iffalse

% TODO: clearup simple galois correspondence diagram

% roots of a polynomial
\DeclareMathOperator{\Root}{Root}

% norm
\DeclareMathOperator{\n}{N}

% transcendence degree
\DeclareMathOperator{\trdeg}{trdeg}

\newcommand*{\red}[1]{\overline{#1}}


\makeindex

\begin{document}

\begin{titlepage}
  \begin{center}
    \includegraphics[width=0.6\textwidth]{logo.jpg}\par
    \vspace{1cm}
    {\scshape\huge Mathamatics Tripos \par}
    \vspace{2cm}
    {\huge Part \npart \par}
    \vspace{0.6cm}
    {\Huge \bfseries \ntitle \par}
    \vspace{1.2cm}
    {\Large\nterm, \nyear \par}
    \vspace{2cm}
    
    {\large \emph{Lectures by } \par}
    \vspace{0.2cm}
    {\Large \scshape \nlecturer}
    
    \vspace{0.5cm}
    {\large \emph{Notes by }\par}
    \vspace{0.2cm}
    {\Large \scshape \href{mailto:\nauthoremail}{\nauthor}}
 \end{center}
\end{titlepage}

\tableofcontents

\setcounter{section}{-1}

\section{History}

The primary motivation of this course is to study polynomial equations in one variable and to consider whether there is a formula involving roots, i.e.\ solution by \emph{radicals}.

Quadratics have been well understood since long long time ago and we have already studied them at school. For cubics and quartics, it took long time before people discovered how to solve them by radicals. In 1770 Lagrange studied why it worked. However, In 1799 Ruffini claimed that there were some quintics that \emph{can't} be solved by radicals, i.e.\ there is no general formula, although his proof had gaps.

In 1824, Abel (1802 -- 1829) first accepted proof of insolubility using existing ideas about permutations of roots. In 1831, Galois (1811 -- 1832) first explained why some polynomials are soluble by radicals and others are not. He made use of a \emph{group of permutations} of the roots and he realised in particular the importance of \emph{normal subgroups}.

Galois' work was not known in his lifetime --- it was only published by Liouville in 1846 who realised that it fit in well with the work of Cauchy on permutations.

Galois had submitted his work for various competitions and for entry into the École Polytechnique. He died in a duel, leaving a \(6\frac{1}{2}\) page letter indicating his thoughts about future development.

Most of this course is Galois Theory but it is presented in a slightly more modern way in terms of field extensions.

Recall from IB Groups, Rings and Modules that if \(f(t)\) is an irreducible polynomial in \(K[t]\) for some field \(K\) then \(K[t]/(f(t))\) is a field where \(f(t)\) is the ideal generated by \(f(t)\). This is the starting point of this course.

This course requires quite a lot of IB Groups, Rings and Modules but no content about module is required except in one place where it is useful to know the Structural Theorem of Finitely Generated Abelian Groups.

\section{Field Extensions}

\subsection{Field Extensions}

\begin{definition}[Field extension]\index{field extension}
  A \emph{field extension} \(K \leq L\) is the inclusion of a field \(K\) into another field \(L\), with the same \(0\) and \(1\) and the restriction of \(+\) and \(\cdot\) in \(L\) to \(K\) gives \(+\) and \(\cdot\) in \(K\).
\end{definition}

\begin{eg}
  \(\Q \leq \R, \R \leq \C, \Q \leq \Q(\sqrt 2) = \{\lambda + \mu \sqrt 2: \lambda, \mu \in \Q\}\) and \(\Q(i) = \{\lambda + \mu i: \lambda, \mu \in \Q\} \leq \C\) are all field extensions.
\end{eg}

Suppose \(K \leq L\) is a field extension. Then \(L\) is a \(K\)-vector space with addition given by the field and scalar multiplication given by the multiplication in the field \(L\).

\begin{definition}[Degree of extension]
  The \emph{degree} of \(L\) over \(K\) is \(\dim_kL\), the dimension of the \(K\)-vector space \(L\). It is denoted by \(|L:K|\). It may or may not be finite.
\end{definition}

\begin{definition}[Finite extension]\index{field extension!finite}
  If \(|L:K| < \infty\) the extension is \emph{finite}. Otherwise it is \emph{infinite}.
\end{definition}

\begin{eg}\leavevmode
  \begin{enumerate}
  \item \(|\C:\R| = 2\) since \(\{1, i\}\) is a basis.
  \item Similarly \(|\Q(i):\Q| = 2\).
  \item \(\Q \leq \R\) is an infinite extension.
  \end{enumerate}
\end{eg}

\begin{theorem}[Tower Law]
  \label{thm:tower}
  Suppose \(K \leq L \leq M\) are field extensions. Then
  \[
    |M:K| = |M:L||L:K|.
  \]
\end{theorem}

\begin{proof}
  Assume \(|M:L| < \infty, |L:K| < \infty\). Then we take an \(L\)-basis \(\{f_i\}_{i = 1}^b\) and a \(K\)-basis \(\{e_j\}_{j = 1}^a\).

  Now take \(m \in M\). \(m = \sum_{i = 1}^{b} \mu_if_i\) for some \(\mu_i \in L\). For each \(\mu_i\), we can write \(\mu_i = \sum_{j = 1}^{\infty} \lambda_{ij}e_j\) for some \(\lambda_{ij} \in K\). Thus
  \[
    m = \sum_{i = 1}^{b} \sum_{j = 1}^{a} \lambda_{ij} f_ie_j
  \]
  so \(\{f_ie_j\}\) span \(M\).

  To show linear independence, it suffices to show that if \(m = 0\) then each of the \(\lambda_{ij}\) is zero. When \(m = 0\), the linear independence of \(f_i\) forces each \(\mu_i\) to be zero. Then the linear independence of \(e_i\) forces \(\lambda_{ij}\) to be zero as required.

  The proof for infinite extensions is omitted. Observe (not very rigorously) that if \(M\) is an infinite extension of \(L\) then it is an infinite extension of \(K\), and if \(L\) is an infinite extension of \(K\) then the larger field \(M\) must also be an infinite extension of \(K\).
\end{proof}

\begin{eg}
  \label{eg:Q(sqrt 2, i)}
  Consider
  \[
    \Q \leq \Q(\sqrt 2) \leq \Q(\sqrt 2, i).
  \]
  \(\Q(\sqrt 2)\) has \(\{1, \sqrt 2\}\) as a \(\Q\)-basis. \(\Q(\sqrt 2, i)\) has \(\{1, i\}\) as a \(\Q(\sqrt 2)\)-basis. Now \(\Q(\sqrt 2, i)\) has basis \(\{1, \sqrt 2, i, i \sqrt 2\}\) over \(\Q\). Thus
  \[
    |\Q(\sqrt 2, i):\Q| = 4 = 2 \cdot 2 = |\Q(\sqrt 2, i):\Q(\sqrt 2)||\Q(\sqrt 2):\Q|.
  \]
\end{eg}

What are the intermediate fields between \(\Q\) and \(\Q(i, \sqrt 2)\)? Obviously we have \(\Q(\sqrt 2), \Q(i)\) and \(\Q(i\sqrt 2)\). Are there any more?

To answer, this, the Galois correspondence arising in the \nameref{thm:fundamental} gives an \emph{order-reversing} bijection between the lattice of intermediate subfields and the subgroups of a group of ring automorphisms of the extension field (\(\Q(i, \sqrt 2)\) here) that fix the smaller field element-wise.

Ring automorphisms of \(\Q(i, \sqrt 2)\) that fixes \(\Q\) include the identity \(e: \sqrt 2 \mapsto \sqrt 2, i \mapsto i\) and complex conjugation \(g: \sqrt 2 \mapsto \sqrt 2, i \mapsto -i\). Notice that \(i\) and \(-i\) play the same role in the field \(\Q(i, \sqrt 2)\) --- they are both roots of \(t^2 + 1 = 0\). There is another automorphism
\begin{align*}
  h: \sqrt 2 &\mapsto -\sqrt 2 \\
  i &\mapsto i
\end{align*}
which switches the roots of \(t^2 - 2 = 0\). Notice that the composition
\begin{align*}
  gh: \sqrt 2 &\mapsto -\sqrt 2 \\
  i &\mapsto -i
\end{align*}
These form a group of order \(4\), which equals to \(|\Q(i, \sqrt 2): \Q|\).

\[
  \begin{tikzcd}
    & \Q(i, \sqrt 2) \ar[dl, dash, "2"] \ar[d, dash, "2"] \ar[dr, dash, "2"] \\
    \Q(\sqrt 2) \ar[dr, dash, "2"] & \Q(i) \ar[d, dash, "2"] & \Q(i\sqrt 2) \ar[dl, dash, "2"] \\
    & \Q
  \end{tikzcd}
  \quad
  \begin{tikzcd}
    & \{e\} \ar[dl, dash, "2"] \ar[d, dash, "2"] \ar[dr, dash, "2"] \\
    \{e, g\} \ar[dr, dash, "2"] & \{e, h\} \ar[d, dash, "2"] & \{e, gh\} \ar[dl, dash, "2"] \\
    & \{e, g, h, gh\}
  \end{tikzcd}
\]

The recipe for producing an intermediate subfield from a subgroup is to take the elements of \(\Q(i, \sqrt 2)\) which are fixed by all elements of the subgroup. For example, \(\Q(i\sqrt 2)\) is the field of elements fixed by both \(e\) and \(gh\). This correspondence doesn't always work for all finite field extensions: it only works for \emph{Galois extensions}.

In the correspondence, normal extensions correspond to normal subgroups. In this example all subgroups are normal and hence the extensions are normal.

We will also prove \nameref{thm:primitive}, which in the context of finite extensions of \(\Q\) tells us that they are necessarily of the form \(\Q(\alpha)\) for some \(\alpha\). For example, \(\Q(i, \sqrt 2) = \Q(i + \sqrt 2)\).

We now do a review of IB Groups, Rings and Modules. Suppose \(K \leq L\) is a field extension. Take \(\alpha \in L\). Define
\[
  I_\alpha = \{f \in K[t]: f(\alpha) = 0\}.
\]

\begin{definition}[Algebraic, transcendental]
  \(\alpha\) is \emph{algebraic} over \(K\) if \(I_\alpha \neq \{0\}\). Otherwise \(\alpha\) is \emph{transcendental} over \(K\).
\end{definition}

\begin{definition}[Algebraic extension]\index{extension!algebraic}
  \(L\) is \emph{algebraic over} \(K\) if for all \(\alpha \in L\), \(\alpha\) is algebraic over \(K\).
\end{definition}

\begin{remark}
  \(I_\alpha\) is the kernel of the ring homomorphism
  \begin{align*}
    K[t] &\to L \\
    f &\mapsto f(\alpha)
  \end{align*}
  evaluation at \(\alpha\). It follows that \(I_\alpha\) is an ideal of \(K[t]\).
\end{remark}

\begin{eg}\leavevmode
  \begin{enumerate}
  \item \(\sqrt 2\) is algebraic over \(\Q\) since it is a root of \(t^2 - 2\).
  \item \(\pi\) is transendental over \(\Q\).
  \end{enumerate}
\end{eg}

\begin{lemma}
  Let \(K \leq L\) be a finite field extension. Then \(L\) is algebraic over \(K\).
\end{lemma}

\begin{proof}
  Let \(|L:K| = n\). Take \(\alpha \in L\). Consider \(1, \alpha, \alpha^2, \dots, \alpha^n\). These must be linearly dependent in the \(n\)-dimensional \(K\)-vector space \(L\) so
  \[
    \sum_{i = 0}^n \lambda_i\alpha^i = 0
  \]
  for some \(\lambda_i \in K\) not all zero. Then \(\alpha\) is a root of \(f(t) = \sum_{i = 0}^n \lambda_it^i\). Thus \(\alpha\) is algebraic over \(K.\)
\end{proof}

The non-zero ideal \(I_\alpha\) (where \(\alpha\) is algebraic over \(K\)) is principal since \(K[t]\) is a PID. Then

\begin{definition}[Minimal polynomial]\index{minimal polynomial}
  \(I_\alpha = (f_\alpha(t))\) and \(f_\alpha(t)\) can be assumed to be monic. Such a monic \(f_\alpha(t)\) is the \emph{minimal polynomial} of \(\alpha\) over \(K\).
\end{definition}

\begin{remark}
  Multiplication by \(\alpha\) within the field \(L\) gives a \(K\)-linear map \(L \to L\), an automorphism of \(L\) if \(\alpha \neq 0\).

  In IB Groups, Rings and Modules, we proved that the minimal polynomial of a linear map is unique.
\end{remark}

\begin{eg}\leavevmode
  \begin{enumerate}
  \item The minimal polynomial of \(\sqrt 2\) over \(\Q\) is \(t^2 - 2\).
  \item The minimal polynomial of \(\sqrt 2\) over \(\R\) is \(t - \sqrt 2\).
  \end{enumerate}
\end{eg}

\begin{lemma}
  Suppose \(K \leq L\) is a field extension, \(\alpha \in L\) and \(\alpha\) is algebraic over \(K\). Then the minimal polynomial \(f_\alpha(t)\) of \(\alpha\) over \(K\) is irreducible in \(K[t]\) and so \(I_\alpha\) is a prime ideal.
\end{lemma}

\begin{proof}
  Suppose not and \(f_\alpha(t) = p(t)q(t)\). We must show that either \(p(t)\) or \(q(t)\) is a unit in \(K[t]\). Note that \(0 = f(\alpha) = p(\alpha)q(\alpha)\) and so \(p(\alpha) = 0\) or \(q(\alpha) = 0\). Assume wlog \(p(\alpha) = 0\). Thus \(p(t) \in I_\alpha\) and so \(p(t) = f_\alpha(t)r(t)\) since \(I_\alpha = (f_\alpha(t))\). Thus \(f_\alpha(t) = f_\alpha(t)r(t)q(t)\) and so \(r(t)q(t) = 1\) in \(K[t]\). Thus \(q(t)\) is a unit as required.

  Recall that irreducible elements in an integral domain are prime and generate prime ideals of \(K[t]\). Thus \(I_\alpha\) is a prime ideal.
\end{proof}

\begin{definition}[Generated field]
  Suppose \(K \leq L\) is a field extension and \(\alpha \in L\). \(K(\alpha)\) is defined to be the smallest subfield of \(L\) containing \(K\) and \(\alpha\). It is called the \emph{field generated} by \(K\) and \(\alpha\).
\end{definition}

\begin{definition}[Simple extension]\index{extension!simple}
  \(L\) is a \emph{simple extension} if \(L = K(\beta)\) for some \(\beta \in L\).
\end{definition}

Given \(\alpha_1, \dots, \alpha_n \in L\), \(K \leq L\), \(K(\alpha_1, \dots, \alpha_n)\) is the smallest intermediate field containing \(\alpha_1, \dots, \alpha_n\). It is the field generated by \(K\) and \(\alpha_1, \dots, \alpha_n\). On the other hand, \(K[\alpha] = \im (K[t] \to L, f \mapsto f(\alpha))\) is the ring generated by \(K\) and \(\alpha\).

\begin{theorem}
  Suppose \(K \leq L\) is a field extension and \(\alpha \in L\) is algebraic. Then
  \begin{enumerate}
  \item \(K(\alpha) = K[\alpha] = \im (f \mapsto f(\alpha))\).
  \item \(|K(\alpha):K| = \deg f_\alpha(f)\) where \(f_\alpha(t)\) is the minimal polynomial of \(\alpha\) over \(K\).
  \end{enumerate}
\end{theorem}

\begin{proof}\leavevmode
  \begin{enumerate}
  \item Clearly \(K[\alpha] \leq K(\alpha)\). We need to show that if \(0 \neq \beta \in K[\alpha]\) then it is a unit in \(K[\alpha]\) and so \(K[\alpha]\) is a field. Let \(\beta = g(\alpha)\) for some \(g(t) \in K[t]\). Since \(\beta = g(\alpha) \neq 0\), \(g(t) \notin I_\alpha = (f_\alpha(t))\). Therefore \(f_\alpha(t) \ndivides g(t)\). Since \(f_\alpha(t)\) is irreducible and \(K[t]\) is a PID, there exists \(r(t), s(t) \in K[t]\) with
    \[
      r(t)f_\alpha(t) + s(t)g(t) = 1 \in K[t].
    \]
    Thus \(s(\alpha)g(\alpha) = 1 \in K[\alpha]\) and thus \(\beta = g(\alpha)\) is a unit as required.
  \item Let \(n = \deg f_\alpha(t)\). We will show that \(\{1, \alpha, \dots, \alpha^{n - 1}\}\) is a \(K\)-vector space basis of \(K[\alpha]\).
    \begin{itemize}
    \item spanning: suppose
      \[
        f_\alpha(t) = t^n + a_{n - 1}t^{n - 1} + \dots + a_0
      \]
      with \(a_i \in K\). Then
      \[
        \alpha^n = -a_{n - 1}\alpha^{n - 1} - \dots - a_0.
      \]
      This implies that \(\alpha^n\) is a linear combination of \(\{\alpha^i\}^{n - 1}_0\). An easy induction shows that \(a^m\) for \(m \geq n\) is likewise a linear combination of \(\{\alpha^i\}^{n - 1}_0\). Spanning thus follows.
    \item linear independence: suppose there is a relation \(\sum_{i = 0}^{n - 1} \lambda_{i}\alpha^i = 0\). Let \(g(t) = \sum_{i = 0}^{n - 1}\lambda_it^i\). Since \(g(\alpha) = 0\), we have \(g(t) \in I_\alpha = (f_\alpha(t))\). Thus \(g(t) = 0\) or \(f_\alpha(t) \divides g(t)\). The latter is not possible by degree consideration. Thus \(g(t) = 0 \in K[\alpha]\) and all the \(\lambda_i\)'s are zero.
    \end{itemize}
  \end{enumerate}
\end{proof}

\begin{corollary}
  If \(K \leq L\) is a field extension and \(\alpha \in L\) then \(\alpha\) is algebraic over \(K\) if and only if \(K \leq K(\alpha)\) is finite.
\end{corollary}

\begin{proof}
  Straightforward by combining previous results.
\end{proof}

\begin{corollary}
  \label{cor:divisibility of extension}
  Let \(K \leq L\) be a field extension with \(|L:K| = n\) and \(\alpha \in L\). Then \(\deg f_\alpha(t) \divides n\).
\end{corollary}

\begin{proof}
  By \nameref{thm:tower} on \(K \leq K(\alpha) \leq L\), we deduce that \(|K(\alpha):K|\) divides \(|L:K|\). The result follows from the corollary above.
\end{proof}

\subsection{Digression on (non-)constructability}

In schedules it mentions ``other classical problems'' and we are in a position to tackle some of these using \Cref{cor:divisibility of extension}.

A classical problem from Greek geometry concerns the existence or otherwise of constructions using ruler and compasses. Here ``ruler'' means single unmarked straightedge. If you're an expert you can divide a line between two points into arbitrarily many equal segments, bisect an angle and produce parallel lines. In addition, given a polygon, you can produce a square of the same area or double the area. However, you cannot
\begin{enumerate}
\item duplicate the cube (i.e.\ given a cube you cannot produce a cube of double the volume);
\item trisect the angle \(\pi/3\);
\item square a circle (i.e.\ given a circle you cannot construct a square of the same area).
\end{enumerate}

Assume we are given a set \(P_0\) of points in \(\R^2\),
\begin{itemize}
\item \emph{ruler operation} draws a straight line through any points in \(P_0\),
\item \emph{compass operation} draws a circle with centre being a point in \(P_0\) and radius being distance between a pair of points in \(P_0\).
\end{itemize}

\begin{definition}[Constructible number]\index{constructible number}
  The points of intersection of any two distinct lines or circles drawn using these operations are \emph{constructible} in one step from \(P_0\).

  A point \(r \in \R^2\) is \emph{constructible} from \(P_0\) if there is a finite sequence \(r_1, r_2, \dots, r_n = r\) such that \(r_i\) is constructible in one step from \(P_0 \cup \{r_1, \dots, r_{i - 1}\}\).
\end{definition}

\begin{ex}
  Construct the midpoint of a line between two points.
\end{ex}

Let \(K_0\) be the subfield of \(\R\) generated by \(\Q\) and the coordinates of points in \(P_0\). Let \(r_0 = (x_i, y_i)\). Set \(K_i = K_{i - 1}(x_i, y_i)\). Then
\[
  K_0 \leq K_1 \leq \dots \leq K_n \leq \R.
\]

\begin{lemma}
  \(x_i, y_I\) are both roots in \(K_i\) of quadratic polynomials in \(K_{i - 1}[t]\).
\end{lemma}

\begin{proof}
  There are three cases:
  \begin{itemize}
  \item line intersects line.
  \item circle intersects circle,
  \item circle intersects circle.
  \end{itemize}

  For the line intersecting circle case, the line passes through \(A = (p, q)\) and \(B = (r, s)\) aand the circle centres at \(C = (t, u)\) and has radius \(w\). Then the equation of line is
  \[
    \frac{x - p}{r - p} = \frac{y - q}{s - q}
  \]
  and the equation of circle is
  \[
    (x - t)^2 + (y - u)^2 = w^2.
  \]
  Solving gives coordinates of the intersection satisfying quadratic polynomials over \(K_{i - 1}\).

  The other cases are similar.
\end{proof}

\begin{theorem}
  If \(r = (x, y)\) is constuctible from a set \(P_0\) of points in \(\R^2\) and if \(K_0\) is the subfield of \(\R\) generated by \(\Q\) and the coordinates of the points in \(P_0\) then the degrees \(|K_0(x) : K_0|\) and \(|K_0(y) : K_0|\) are powers of \(2\).
\end{theorem}

\begin{proof}
  Use the previous notation \(K_i = K_{i - 1}(x_i, y_i)\). By \nameref{thm:tower},
  \[
    |K_i : K_{i - 1}| = |K_{i - 1}(x_i, y_i) : K_{i - 1}(x_i)| |K_{i - 1}(x_i) : K_{i - 1}|.
  \]
  But the previous lemma tells us that \(K_{i - 1}(x_i) : K_{i - 1}| = 1 \text{ or } 2\), depending on whether the quadratic polynomial satisfied by the point of intersection is irreducible or not. Similarly \(y_i\) satisfies a quadratic polynomial over \(K_{i - 1}\), and hence over \(K_{i - 1}(x_i)\) and so \(K_{i - 1}(x_i, y_i) : K_{i - 1}(x_i)| = 1 \text{ or } 2\). Therefore \(K_i : K_{i - 1} = 1, 2 \text{ or } 4\) (but in fact \(4\) does not happen). Therefore by \nameref{thm:tower},
  \[
    |K_n : K_0| = |K_n : K_{n - 1}| \cdots |K_1 : K_0|
  \]
  is a power of \(2\).

  If \(r = (x, y)\) is contrucible from \(P_0\) then \(x, y \in P_n\) for some \(n\) and therefore
  \begin{align*}
    K_0 &\leq K_0(x) \leq K_n \\
    K_0 &\leq K_0(y) \leq K_n
  \end{align*}
  and the \nameref{thm:tower} again gives that \(|K_0(x) : K_0|\) and \(|K_0(y) : K_0|\) are powers of \(2\).
\end{proof}

To use this for proofs about non-constructibility we need to be reasonably expert at working out minimal polynomials. Recall from IB Groups, Rings and Modules

\begin{theorem}[Guass' Lemma]
  Let \(f(t)\) be a primitive integral polynomial. Then \(f(t)\) is irreducible in \(\Q[t]\) if and only if it is irreducible in \(\Z[t]\).
\end{theorem}

\begin{theorem}[Eisenstein's criterion]
  Let \(f(t) = a_nt^n + a_{n - 1}t^{n - 1} + \dots + a_0 \in \Z[t]\). Suppose there is a prime \(p\) such that
  \begin{itemize}
  \item \(p \ndivides a_n\),
  \item \(p \divides a_{n - 1}, p \divides a_{n - 2}, \dots, p \divides a_0\),
  \item \(p^2 \ndivides a_0\),
  \end{itemize}
  then \(f(t)\) is irreducible in \(\Z[t]\).
\end{theorem}

\begin{eg}
  Suppose \(p\) is prime. Then
  \[
    f(t) = t^{p - 1} + t^{p - 2} + \dots + 1
  \]
  is irreducible over \(\Q\) by considering \(f(t + 1)\) and using \(p\) as the prime in Eisenstein.
\end{eg}

Another method to determine reducibility is to consider an integral polynomials \(f(t) \pmod p\). If \(f(t)\) is irreducible in \(\Z[t]\) then it is irreducile over \(\Z/p\Z\). So if we find a prime \(p\) such that \(f(t) \pmod p\) is irreducible then \(f(t)\) is irreducible in \(\Z[t]\).

\begin{eg}
  \(t^3 + t + 1\) is irreducible mod \(2\): if it were reducible it would have a linear factor and so the polynomial would have a root mod \(2\), but \(0, 1\) are not roots. So \(t^3 + t + 1\) is irreducible in \(\Z[t]\) and hence irreducible in \(\Q[t]\).
\end{eg}

\begin{remark}
  On example sheet we will meet an irreducible polynomials in \(Z[t]\) which is reducible mod \(p\) for all primes \(p\).
\end{remark}

Now back to construcibility.

\begin{theorem}
  The cube cannot be duplicated by rulers and compasses.
\end{theorem}

\begin{proof}
  The problem amounts to whether given a unit distance one can construct points distance \(\alpha\) apart where \(\alpha\) satisfies \(t^3 - 2 = 0\). Starting with points \(P_0 = \{(0, 0), (1, 0)\}\). Can we produce \((\alpha, 0)\)? No. If we could, \(|\Q(\alpha): \Q|\) would be a power of \(2\). We show \(|\Q(\alpha) : Q| = 3\): since \(\alpha\) satisfies \(t^3 - 2\), which by Eisenstein is irreducible over \(\Z\) and hence over \(\Q\). Thus it is the minimal polynomials of \(\alpha\) over \(\Q\).
\end{proof}

\begin{theorem}
  The circle cannot be squared using ruler and compasses.
\end{theorem}

\begin{proof}
  Starting with \((0, 0)\) and \((1, 0)\), can we construct \((\sqrt \pi, 0)\) so that we have a square of side length \(\sqrt \pi\), and hence the area of the square equals to the are of the unit circle?

  The answer is no since \(\pi\) and hence \(\sqrt \pi\) is transcendental over \(\Q\).
\end{proof}

Now return to further theory develpment.

\begin{lemma}
  Let \(K \leq L\) be a field extension. Then
  \begin{enumerate}
  \item \(\alpha_1, \dots, \alpha_n \in L\) are algebraic over \(K\) if and only if \(K \leq K(\alpha_1, \dots, \alpha_n)\) is a finite extension.
  \item If \(K \leq M \leq L\) is such that \(K \leq M\) is finite, then there exists \(\alpha_1, \dots, \alpha_n \in L\) such that \(K(\alpha_1, \dots, \alpha_n) = M\).
  \end{enumerate}
\end{lemma}

\begin{proof}\leavevmode
  \begin{enumerate}
  \item We know \(\alpha\) is algebraic over \(K\) if and only if \(K \leq K(a)\) is a finite extension. Suppose for each \(i\), \(\alpha_i\) is algebraic over \(K\), hence over \(K(\alpha_1, \dots, \alpha_{i - 1})\) and so \(|K(\alpha_1, \dots, \alpha_i) : K(\alpha_1, \dots, \alpha_{i - 1})| < \infty\). By \nameref{thm:tower} applied to
    \[
      K \leq K(\alpha_1) \leq \dots \leq K(\alpha_1, \dots \alpha_n),
    \]
    we get \(|K(\alpha_1, \dots, \alpha_n) : K| < \infty\).

    Conversely, consider
    \[
      K \leq K(\alpha_1) \leq K(\alpha_1, \dots, \alpha_n).
    \]
    Then the \nameref{thm:tower} says that if \(|K(\alpha_1, \dots, \alpha_n) : K| < \infty\) then \(|K(\alpha_i) : K| < \infty\) and so \(\alpha_i\) is algebraic over \(K\).

  \item If \(|M : K| = n\) then \(M\) is an \(n\)-dimensional \(K\)-vector space so there exists a \(K\)-basis \(\{\alpha_1, \dots, \alpha_n\}\) of \(M\). Then \(K(\alpha_1, \dots, \alpha_n) \leq M\). However, any elements of \(M\) is a \(K\)-linear combination of \(\alpha_1, \dots, \alpha_n\) and so lies in \(K(\alpha_1, \dots, \alpha_n)\) so equality.
  \end{enumerate}
\end{proof}

\begin{definition}[\(K\)-homomorphism]\index{\(K\)-homomorphism}
  Suppose \(K \leq L, K \leq L'\) are field extensions. A \emph{\(K\)-homomorphism} \(\phi: L \to L'\) is a ring homomorphism such that \(\phi|_K = \id\).
\end{definition}

\begin{notation}
  Let
  \[
    \Hom_K(L, L') = \{K\text{-homomorphisms } L \to L'\}.
  \]
  A \(K\)-homomorphism \(\phi: L \to L'\) is a \emph{\(K\)-isomorphism} if it is a ring isomorphism.
\end{notation}

\begin{notation}
  \(\aut_K(L) = \{K\text{-isomorphism} L \to L'\}\) which has a group structure.
\end{notation}

\begin{lemma}
  Suppose \(K \leq L, K \leq L'\) are field extensions. Then
  \begin{enumerate}
  \item any \(K\)-homomorphism \(\phi: L \to L'\) is injective and \(K \leq \phi(L)\) is a field extension;
  \item if \(|L : K| = |L' : K| < \infty\), then a \(K\)-homomorphism \(\phi: L \to L'\) is a \(K\)-isomorphism.
  \end{enumerate}
\end{lemma}

\begin{proof}\leavevmode
  \begin{enumerate}
  \item \(L\) is a field and \(\ker \phi\) is an ideal of \(L\). Note \(1 \mapsto 1\) and so \(\ker \phi \neq L\), so \(\ker \phi = 0\). Therefore \(\phi(L)\) is a field and \(K \leq \phi(L)\) is a field extension.

  \item \(\phi\) is an injective \(K\)-linear map and so \(|\phi(L) : K| = |L : K|\), which, by considering the dimensions of \(K\)-vector spaces, is smaller than \(|L' : K| = |L : K|\) by assumption. Thus \(\phi: L \to L'\) is a \(K\)-isomorphism. In particular if \(L = L'\) then \(\phi\) is a \(K\)-automorphism of \(L\).
  \end{enumerate}
\end{proof}

\begin{notation}
  If \(K \leq L\) is a field extension and if \(f(t) \in K[t]\), we denote the set of roots of \(f\) in \(L \) by \(\Root_f(L)\).
\end{notation}

\begin{definition}[Splitting polynomial, splitting field]\index{splitting polynomial}\index{splitting field}
  Let \(K \leq L\) be a field extension and \(f(t) \in K[t]\). We say \(f\) \emph{splits over \(L\)} if
  \[
    f(t) = a(t - \alpha_1) (t - \alpha_2) \cdots (t - \alpha_n)
  \]
  where \(a \in K, \alpha_1, \dots, \alpha_n \in L\).

  We say \(L\) is a splitting field for \(f\) over \(K\) if \(L = K(\alpha_1, \dots, \alpha_n)\).
\end{definition}

\begin{remark}
  This is equivalent to saying that \(L\) is a splitting field for \(f\) over \(K\) if and only if
  \begin{enumerate}
  \item \(f\) splits over \(L\),
  \item if \(K \leq M \leq L\) and \(f\) splits over \(M\) then \(M = L\).
  \end{enumerate}
\end{remark}

\begin{eg}
  
\end{eg}
\blindtext

\section{Separable, Normal and Galois Extensions}

\subsection{Separable Extension}

\begin{definition}[Separable polynomial]\index{separable}
  Let \(K\) be a field and \(f(t) \in K[t]\). Suppose \(f(t)\) is irreducible in \(K[t]\) and \(L\) is a splitting field of \(f(t)\) over \(K\). Then \(f(t)\) is \emph{separable} over \(K\) if \(f(t)\) has no repeated roots in \(L\). For general \(f(t)\), we say \(f(t)\) is \emph{separable} over \(K\) if every irreducible factor in \(K[t]\) is separable over \(K\).

  All constant polynomials are deemed to be separable.
\end{definition}

\begin{definition}[Formal differentiation]\index{differentiation}
  If \(K\) is a field, the \emph{formal differentiation} is the \(K\)-linear map 
  \begin{align*}
    D: K[t] &\to K[t] \\
    t^n &\mapsto n t^{n-1}
  \end{align*}
\end{definition}

\begin{notation}
  Denote \(D(f(t))\) by \(f'(t)\).
\end{notation}

\begin{lemma}
  \label{lem:formal derivative}
  Let \(K\) be a field, \(f(t), g(t) \in K[t]\). Then
  \[
    (f(t)g(t))' = f'(t)g(t) + f(t)g'(t)
  \]
  and if \(f(t) \neq 0\), \(f(t)\) has a repeated root in a splitting field if and only if \(f(t)\) and \(f'(t)\) have a common irreducible factor in \(K[t]\).
\end{lemma}

\begin{proof}
  \(D\) is a \(K\)-linear map so we only need to check for \(f(t) = t^n\). It is left as an exercise.

  Let \(\alpha\) be a repeated root in a splitting field \(L\). Then \(f(t) = (t - \alpha)^2 g(t) \in L[t]\) so \(f'(t) = (t - \alpha)^2g'(t) + 2(t - \alpha)g(t)\) and \(f'(\alpha) = 0\). Therefore the minimal polynomial \(f_\alpha(t)\) of \(\alpha\) in \(K[t]\) divides both \(f(t)\) and \(f'(t)\), so it is a common irreducible factor of \(f(t)\) and \(f'(t)\).

  Conversely, let \(h(t)\) be a common irreducible factor of \(f(t)\) and \(f'(t)\) in \(K[t]\). Pick a root \(\alpha \in L\) of \(h(t)\). Then \(f(\alpha) = f'(\alpha) = 0\). Then \(f(t) = (t - \alpha)g(t) \in L[t]\) and \(f'(t) = (t - \alpha)g'(t) + g(t)\). Since \(f'(\alpha) = 0\), we have \((t - \alpha) \divides f'(t)\) so \((t -\alpha) \divides g(t)\). Hence \((t - \alpha)^2 \divides f(t)\).
\end{proof}

\begin{corollary}
  If \(K\) is a field and \(f(t) \in K[t]\) is irreducible, then
  \begin{enumerate}
  \item if \(\ch K = 0\) then \(f(t)\) is separable over \(K\),
  \item if \(\ch K = p > 0\) then \(f(t)\) is not separable if and only if \(f(t) \in K[t^p]\).
  \end{enumerate}
\end{corollary}

\begin{proof}
  By \Cref{lem:formal derivative}, \(f(t)\) is not separable if and only if \(f(t)\) and \(f'(t)\) have a common irreducible factor. But since \(f(t)\) is irreducible, the only possible factor is \(f(t)\) itself, i.e.\ \(f(t) \divides f'(t)\). \(f'(t) = 0\) as it has a smaller degree. But if \(f(t) = \sum_{i = 0}^{n} a_it^i\) then \(f'(t) = \sum_{i = 1}^{n} ia_it^{i - 1}\) so \(f'(t) = 0\) if and only if \(ia_i = 0\) for all \(i \geq 1\). Thus
  \begin{enumerate}
  \item if \(\ch K = 0\), \(f'(t) \neq 0\) for non-constant \(f(t)\) so \(f(t)\) is separable over \(K\),
  \item if \(\ch K = p > 0\), then if \(f'(t) = 0\) we must have \(ia_i = 0\) for all \(i \geq 1\), i.e.\ \(f(t)\) is not separable if and only if \(f(t) \in K[t^p]\).
  \end{enumerate}
\end{proof}

\begin{definition}[Separable element]\index{separable}
  If \(K \leq L\) is a field extension, we say \(\alpha \in L\) is \emph{separable} over \(K\) if its minimal polynomial is separable over \(K\).
\end{definition}

\begin{definition}[Purely inseparable]
  If the minimal polynomial of \(\alpha\) is \(f_\alpha(t) = (t - \alpha)^n = t^n - \alpha^n\) where \(n\) is a power of \(\ch K\), \(\alpha\) is said to be \emph{purely inseparably} over \(K\).
\end{definition}

\begin{definition}[Separable extension]\index{field extension!separable}
  Given \(K \leq L\), \(L\) is \emph{separable} over \(K\) if all elements of \(L\) are separable over \(K\).
\end{definition}

\begin{eg}\leavevmode
  \begin{enumerate}
  \item Let \(\Q \leq L\) be an algebraic extension. Then \(L\) is separable over \(\Q\).
  \item Let \(L = \F_p(X)\), the field of rational functions in \(X\) over \(\F_p\). It has \(K = \F_p(X^p)\) as a subfield. Then \(K \leq L\) is not separable:
    \begin{proof}
      Observe that if \(f(t) = t^p - X^p \in K[t]\) then \(f'(t) = 0\). But \(t^p - X^p = (t - X)^p \in L[t]\). However, \(f(t) \in K[t]\) is irreducible: suppose \(f(t) = g(t)h(t) \in K[t] \subseteq L[t]\), we get \(g(t) = (t - X)^r\) for some \(0 < r < p\) if the factorisation is non-trivial. But this would mean \(X^r \in K\). However \(r\) and \(p\) are coprime so there exist \(a, b \in \Z\) such that \(ar + bp = 1\), so \(X = (X^r)^a (X^p)^b \in K\). Thus we would have \(X = u(X^p)/v(X^p)\), absurd.

      Thus \(f(t) = t^p - X^p\) is the minimal polynomial of \(X\) over \(K\). \(\alpha\) is purely inseparable over \(K\) and it follows that \(K \leq L\) is not separable.
    \end{proof}
  \item Let \(\F\) be a finite field with \(|\F| = m\), a power of \(\ch \F\), and \(f(t) = t^n - 1\) where \(n = m - 1\). We know this is separable over \(\F_p\) since we saw that \(f(t)\) has distinct linear factors in \(\F[t]\).
  \end{enumerate}
\end{eg}

\begin{remark}
  It is useful to have an alternative approach to separability of field extensions without having to check separability of minimal polynomials for all elements of the larger field. This is where we start thinking about \(K\)-homomorphisms.
\end{remark}

\begin{lemma}
  \label{lem:homomorphism of algebraic extension}
  Let \(M = K(\alpha)\) where \(\alpha\) is algebraic over \(K\). Let \(f_\alpha(t)\) be the minimal polynomial of \(\alpha\) over \(K\). For any field extension \(K \leq L\), the number of \(K\)-homomorphisms \(M \to L\) is equal to the number of distinct roots of \(f_\alpha(t)\) in \(L\). Thus
  \[
    |\Hom_K(M, L)| \leq \deg f_\alpha(t) = |K(\alpha):K| = |M:K|.
  \]
\end{lemma}

\begin{proof}
  We know that any \(K\)-homomorphism \(M \to L\) is injective. Since \(K(\alpha) \cong K[t]/(f_\alpha(t))\), for any root \(\beta\) of \(f_\alpha(t)\) in \(L\), we can define a \(K\)-homomorphism
  \begin{align*}
    K[t]/(f_\alpha(t)) &\to L \\
    t + (f_\alpha(t)) &\mapsto \beta
  \end{align*}

  Conversely, for any \(K\)-homomorphism \(\phi: M \to L\), the image \(\phi(\alpha)\) must satisfy \(f_\alpha(\phi(\alpha)) = 0\). These two maps are inverses to each other. Thus there is a one-to-one correspondence
  \[
    \{K\text{-homomorphism } M \to L\} \leftrightarrow \{\text{root of } f_\alpha(t) \text{ in } L\}.
  \]
\end{proof}

\begin{eg}
  Let \(K = \Q, L = \Q(\sqrt[3]{2})\). Then \(\alpha = \sqrt[3]{2}\) has minimal polynomial \(f_\alpha(t) = t^3 - 2\) over \(\Q\). There is only one \(K\)-homomorphism \(M = K(\sqrt[3]{2}) \to L\), i.e.\ the identity map.
\end{eg}

\begin{corollary}
  In the previous lemma, the number of \(K\)-homomorphisms \(K(\alpha) \to L\) equals to \(\deg f_\alpha(t)\) if and only if \(L\) is large enough (so that it contains a splitting field for \(f_\alpha(t)\)) and \(\alpha\) is separable over \(K\).
\end{corollary}

\begin{proof}
  Trivial from \Cref{lem:homomorphism of algebraic extension}.
\end{proof}

\begin{lemma}
  \label{lem:homomorphic extension of algebraic extension}
  Let \(K \leq M\) be a field extension and \(M_1 = M(\alpha)\), where \(\alpha\) is algebraic over \(M\). Let \(f(t)\) be the minimal polynomial of \(\alpha\) over \(M\) and let \(K \leq L\). Let \(\phi: M \to L\) be a \(K\)-homomorphism. Then there is a one-to-one correspondence
  \[
    \{\text{extension } \phi_1: M_1 \to L \text{ of } \phi\} \leftrightarrow \{\text{root of } \phi(f(t)) \text{ in } L\}.
  \]
\end{lemma}

\begin{remark}
  \Cref{lem:homomorphism of algebraic extension} is a special case where \(M = K\) and \(\phi = \id_K\).
\end{remark}

\begin{proof}
  \(f(t) \in M[t]\) is irreducible so \(\phi(f(t)) \in \phi(M)[t]\) is irreducible. Any extension \(\phi_1: M_1 \to L\) of \(\phi\) produces a root \(\phi_1(\alpha)\) of \(\phi(f(t))\).

  Conversely, given a root \(\gamma\) of \(\phi(f(t))\) in \(L\),
  \[
    M_1 = M(\alpha) \cong M[t]/(f(t)) \cong \phi(M)[t]/(\phi(f(t))) \cong \phi(M)(\gamma) \leq L
  \]
  so we get an extension \(\phi_1\) of \(\phi\) as required.
\end{proof}

\begin{corollary}
  Given \(L\) contains a splitting field of \(f(t)\), the number of \(\phi_1\) extending \(\phi\) equals to the number of distinct roots of \(f(t)\) in \(L\).

  This equals to \(|M_1:M|\) if and only if \(\alpha\) is separable over \(M\).
\end{corollary}

\begin{corollary}
  \label{cor:homomorphic extension of finite extension}
  Let \(K \leq M \leq N\) be finite field extensions and \(K \leq L\). Let \(\phi: M \to L\) be a \(K\)-homomorphism. Then the number of extensions of \(\phi\) to maps \(\theta: N \to L\) is less than or equal to \(|N:M|\).

  Moreover, such \(\theta\) exists if \(L\) is large enough.
\end{corollary}

\begin{proof}
  Pick \(\alpha_1, \dots \alpha_n\) so that \(N = M(\alpha_1, \dots, \alpha_r)\) and set \(M_i = M(\alpha_1, \dots, \alpha_i)\). Thus we have got
  \[
    M \leq M_1 \leq \cdots \leq M_r = N.
  \]
  Use \Cref{lem:homomorphic extension of algebraic extension} repeatedly, there are at most \(|M_1:M|\) extensions \(\phi_1: M_1 \to L\) of \(\phi\) and \(|M_{i + 1}:M|\) extensions \(\phi_{i + 1}: M_{i + 1} \to L\) of \(\phi_i\) for \(1 \leq i < r\) so by Tower Law the number of extensions \(\theta: N \to L\) of \(\phi\) is less than or equal to
  \[
    |M_r:M_{r - 1}||M_{r - 1}:M_{r - 2}|\cdots|M_1:M| = |N:M|.
  \]

  The last bit come from the proof of \Cref{lem:homomorphic extension of algebraic extension} since we need \(L\) to contain a root.
\end{proof}

\begin{remark}
  The proof shows that the number of extensions \(\theta\) of \(\phi\) is \(|N:M|\) if and only if \(L\) is large enough and \(\alpha_i\) is separable over \(M(\alpha_1, \dots, \alpha_{i - 1})\) for all \(i\).
\end{remark}

\begin{theorem}\label{thm:separability of finite extension}
  Let \(K \leq N\) be a field extension with \(|N:K| = n\) and \(N = K(\alpha_1, \dots, \alpha_n)\), say. Then TFAE:
  \begin{enumerate}
  \item \(K \leq N\) is separable.
  \item Each \(\alpha_i\) is separable over \(K(\alpha_1, \dots, \alpha_{i - 1})\).
  \item If \(L\) is large enough then there are exactly \(n\) distinct \(K\)-homomorphisms \(N \to L\).
  \end{enumerate}
\end{theorem}

\begin{proof}\leavevmode
  \begin{itemize}
  \item \(1 \Rightarrow 2\): If \(K \leq N\) is separable then each \(\alpha_i\) is separable over \(K\). As \(K \leq K(\alpha_1, \dots, \alpha_{i - 1})\), the minimal polynomial of \(\alpha_i\) over \(K(\alpha_1, \dots, \alpha_{i - 1})\) divides that over \(K\). So if the latter has distinct roots in a splitting field then the former does as well.
  \item \(2 \Rightarrow 3\): \Cref{cor:homomorphic extension of finite extension}.
  \item \(3 \Rightarrow 1\): Suppose 3 is true and 1 is false, we shall get a contradiction. Suppose there exists \(\beta \in N\) that is not separable over \(K\). Then there are strictly less than \(|K(\beta):K|\) \(K\)-homomorphisms \(\phi: K(\beta) \to L\). Each \(\phi\) extends to at most \(|N:K(\beta)|\) extensions \(\theta: N \to L\). Thus there are stricly less than \(|N:K(\beta)||K(\beta):K| = |N:K| = n\) \(K\)-homomorphisms \(N \to L\). Absurd.
  \end{itemize}
\end{proof}

\begin{corollary}
  A finite extension is separable if and only if it is separably generated.
\end{corollary}

\begin{proof}
  \(1 \Leftrightarrow 2\) above.
\end{proof}

\begin{lemma}\label{lem:transitivity of separability}
  If \(K \leq M \leq L\) are finite extensions then \(M \leq L\) and \(K \leq M\) are both separable if and only if \(K \leq L\) is separable.
\end{lemma}

\begin{proof}
  Example sheet.
\end{proof}

\begin{eg}
  Let \(\F\) be a finite field with \(|\F| = m\). Then the multiplicative group of order \(n = m - 1\) is cyclic. Take a generator \(\alpha\), then \(\F = \F(\alpha)\). Since \(\alpha^n = 1\), the minimal polynomial of \(\alpha\) divides \(t^n - 1\).

  Since \(t^n - 1\) has distinct roots (all the non-zero elements of \(\F\)), the minimal polynomial of \(\alpha\) is separable so \(\F = \F(\alpha)\) is separable over \(\F\).
\end{eg}

\begin{theorem}[Primitive Element Theorem]
  \label{thm:primitive}
  Any finite separable extension \(K \leq M\) is simple, i.e.\ \(M = K(\alpha)\) for some \(\alpha\), which is called the \emph{primitive element}.
\end{theorem}

\begin{proof}
  \label{proof:primitive}
  If \(K\) is a finite field then \(M\) is also finite. So we can take \(\alpha\) to be generator of the multiplicative group of \(M\), which is cyclic.

  Now assume \(K\) is an infinite field. Since \(K \leq M\) is a finite extension, \(M = K(\alpha_1, \dots, \alpha_n)\) for some \(\alpha_i\). It suffices to show that any field \(M = K(\alpha, \beta)\) with \(\beta\) separable over \(K\) is of the form \(K(\gamma)\).

  Take \(f(t)\) and \(g(t)\) to be the minimal polynomials of \(\alpha\) and \(\beta\) over \(K\) respectively. Let \(L\) be the splitting field for \(f(t)g(t)\) over \(K(\alpha, \beta)\). The distinct zeros of \(f(t)\) in \(L\) are \(\alpha_1 = \alpha, \alpha_2, \dots, \alpha_a\) and those of \(g(t)\) are \(\beta_1 = \beta, \beta_2, \dots, \beta_b\). By separability we know \(b = \deg g(t)\). Choose \(\lambda \in K\) such that all \(\alpha_i + \lambda \beta_j\) are distinct (this is possible since \(K\) is infinite). Now we set \(\gamma = \alpha + \lambda \beta\) (remember \(\alpha\) is \(\alpha_1\) and \(\beta\) is \(\beta_1\)). Let \(F(t) = f(\gamma - \lambda t) \in K(\gamma)[t]\). We have \(g(\beta) = 0\) and \(F(\beta) = f(\alpha) = 0\). Thus \(F(t)\) and \(g(t)\) have a common zero. Any other common zero would have to be \(\beta_j\) for some \(j > 1\). But then \(F(\beta_j) = f(\alpha + \lambda(\beta - \beta_j))\). By assumption \(\alpha + \gamma(\beta - \beta_j)\) is never an \(\alpha_i\) so \(f(\beta_j) \neq 0\).

  Now separability of \(g(t)\) says that the linear factors are all disinct. So \(t - \beta\) is a highest common factor of \(F(t)\) and \(g(t)\) in \(L[t]\). However, the minimal polynomial \(h(t)\) of \(\beta\) over \(K(\gamma)\) divides \(F(t)\) and \(g(t)\) in \(K(\gamma)[t]\), and hence in \(L[t]\). This imples \(h(t) = t - \beta\), and so \(\beta \in K(\gamma)\). Therefore \(\alpha = \gamma - \lambda \beta \in K(\gamma)\). So \(K(\alpha, \beta) \leq K(\gamma)\). The other direction is obvious.
\end{proof}

\begin{ex}
  In our example on page~\pageref{eg:Q(sqrt 2, i)} we have \(\Q \leq \Q(\sqrt 2, i)\). We had intermediate subfields \(\Q(\sqrt 2), \Q(i)\) and \(\Q(i\sqrt 2)\). If we follow the procedure of the \hyperref[proof:primitive]{proof} of \Cref{thm:primitive}, \(\alpha = \sqrt 2, \beta = i, f(t) = t^2 - 2\) and \(g(t) = t^2 + 1\). Consider some \(\sqrt 2 + \lambda i\) where \(\pm \sqrt 2 \pm \lambda i\) are all distinct, for example \(\lambda = 1\). The proof show that \(\Q(\sqrt 2, i) = \Q(\sqrt 2 + i)\).
\end{ex}

\subsection{Trace \& Norm}

These will also be used in IID Number Fields.

\begin{definition}[Trace \& Norm]\index{trace}\index{norm}
  Let \(K \leq M\) be a finite field extension and \(\alpha \in M\). Multiplication by \(a\) is a \(K\)-linear map \(\theta_\alpha: M \to M\). The \emph{trace} of \(\alpha\) over \(K\) is
  \[
    \Tr_{M/K}(\alpha) = \Tr \theta_\alpha \in K
  \]
  and the \emph{norm} of \(\alpha\) over \(K\) is
  \[
    \n_{M/K}(\alpha) = \det \theta_\alpha \in K.
  \]
\end{definition}

\begin{note}
  Trace and norm depend on the ground field.
\end{note}

\begin{theorem}
  \label{thm:trace and norm}
  Suppose \(f_\alpha(t) = t^s + a_{s - 1}t^{s - 1} + \dots + a_0\) is the minimal polynomial of \(\alpha\) over \(K\). Let \(r = |M: K(\alpha)|\). Then the characteristic polynomial of \(\theta_\alpha\) is \((f_\alpha(t))^r\) and
  \begin{align*}
    \Tr_{M/K}(\alpha) &= -r a_{s-1} \\
    \n_{M/K}(\alpha) &= ((-1)^s a_0)^r
  \end{align*}
\end{theorem}

\begin{proof}
  Regard \(M\) as a \(K(\alpha)\)-vector space with basis \(\beta_1 = 1, \beta_2, \dots, \beta_r\). Now take the \(K\)-vecotr space basis \(1, \alpha, \alpha^2, \dots, \alpha^{s - 1}\) of \(K(\alpha)\). Then \(\{\alpha^i \beta_j\}_{i < s, j \leq r}\) is a \(K\)-vector space basis for \(M\). Multiplication by \(\alpha\) in \(K(\alpha)\) is represented by the matrix
    \[
      A =
      \begin{pmatrix}
        0 & & & \cdots & -a_0 \\
        1 & 0 & & \cdots & -a_1 \\
        0 & 1 & 0 & \cdots & -a_2 \\
        \vdots & & & \ddots & \vdots \\
        0 & 0 & \cdots & 1 & -a_{s - 1}
      \end{pmatrix}
    \]
    so multiplication by \(\alpha\) in \(M\) is represented by the \(rs \times rs\) matrix
    \[
      \begin{pmatrix}
        A & 0 & \cdots & 0 \\
        0 & A & \cdots & 0 \\
        \vdots & & \ddots & \vdots \\
        0 & 0 & \cdots & A
      \end{pmatrix}
    \]
    whose characteristic polynomial is \((f_\alpha(t))^r\). By inspecting the coefficients of the characteristic polynomial we get the trace and norm.
\end{proof}

\begin{theorem}
  \label{thm:trace and norm of separable extension}
  Let \(K \leq M\) be a finite separable field extension and \(|M:K| = n\). Let \(\alpha \in M\) and \(K \leq L\) large enough so that there are \(n\) distinct \(K\)-homomorphisms \(\sigma_1, \dots, \sigma_n: M \to L\). Then the characteristic polynomial of \(\theta_\alpha: M \to M\) is
  \[
    \prod_{i = 1}^{n}(t - \sigma_i(\alpha)).
  \]
  Thus
  \begin{align*}
    \Tr_{M/K}(\alpha) &= \sum_{i = 1}^{n} \sigma_i(\alpha) \\
    \n_{M/K}(\alpha) &= \prod_{i = 1}^{n} \sigma_i(\alpha)
  \end{align*}
\end{theorem}

\begin{proof}
  Let
  \[
    f_\alpha(t) = \prod_{i = 1}^{s} (t - \alpha_i) = t^s + a_{s - 1}t^{s - 1} + \dots + a_0
  \]
  be the minimal polynomial of \(\alpha\) over \(K\) in \(L[t]\) where \(L\) is large enough that \(f_\alpha(t)\) splits in \(L\). There are \(s\) \(K\)-homomorphisms \(K(\alpha) \to L\), corresponding to maps sending \(\alpha\) to \(\alpha_i\). Each of these extends in \(r = |M:K(\alpha)|\) ways to give \(K(\alpha)\)-homomorphisms \(M \to L\). However, each of such extension mapping \(\alpha \mapsto \alpha_i\) still does so. So there are \(r\) maps sending \(\alpha \to \alpha_i\) for each \(i\). Thus if the \(n = rs\) distinct \(K\)-homomorphism \(M \to L\) are \(\sigma_1, \dots, \sigma_n\) then
  \[
    \sum_{i = 1}^{n} \sigma_i(\alpha) = r(\alpha_1 + \dots \alpha_s) = -ra_{s - 1} = \Tr_{M/K} (\alpha)
  \]
  since the sum of roots of \(f_\alpha(t)\) is \(-a_{s - 1}\), and
  \[
    \prod_{i = 1}^{n} \sigma_i(\alpha) = ((-1)^s a_0)^r = \n_{M/K}(\alpha).
  \]
\end{proof}

% TODO: check for repetition
%
%
%
%
%
Recall from \Cref{thm:trace and norm} that the characteristic polynomial of \(\theta_\alpha\) is \((f_\alpha(t))^r\) where \(f_\alpha(t)\) is the minimal polynomial of \(\alpha\) over \(K\) and \(r = |M:K(\alpha)|\). The characteristic polynomial is \(\prod_{n = 1}^{s} (t - \alpha_i)^r\) where \(\alpha_i\) are the roots of \(f_\alpha(t)\) in a splitting field. We also saw that those root are \(\sigma_1(\alpha), \dots, \sigma_n(\alpha)\) so the characteristic polynomial is
\[
  \prod_{i = 1}^{n} (t - \sigma_i(\alpha)).
\]

\begin{theorem}
  \label{thm:vandermonde}
  Let \(K \leq M\) be a finite separable extension. We define a \(K\)-bilinear form
  \begin{align*}
    T: M \times M &\to K \\
    (x, y) &\mapsto \Tr_{M/K}(xy)
  \end{align*}
  where \(xy\) is the product in \(M\). Then this is non-degenerate. In particular, the \(K\)-linear map \(\Tr_{M/K}: M \to K\) is non-zero. Thus it is surjective.
\end{theorem}

\begin{remark}
  If \(K \leq M\) is a finite extension which is not separable then \(\Tr_{M/K}\) is always zero. It follows that \(T\) is degnerate. See example sheet.
\end{remark}

\begin{proof}
  \label{proof:vandermonde}
  By \nameref{thm:primitive}, separability implies that \(M = K(\alpha)\) for some \(\alpha\). We have a \(K\)-basis \(\{\alpha^i\}_{i = 0}^{n - 1}\) of \(K(\alpha)\) where \(n = |M:K|\). The \(K\)-belinear form \(T\) is represented by the matrix
  \[
    A =
    \begin{pmatrix}
      \Tr_{M/K}(1) & \Tr_{M/K}(\alpha) & \cdots \\
      \Tr_{M/K}(\alpha) & \Tr_{M/K}(\alpha^2) & \cdots \\
      \cdots & \cdots & \cdots
    \end{pmatrix}
  \]
  Let \(L\) be a splitting field of the minimal polynomial \(f(t)\) of \(\alpha\) over \(K\).Then \(f_\alpha(t) = \prod_{i = 1}^n (t - \alpha_i)\) with \(\alpha_1, \dots, \alpha_n \in L\). The entries in \(A\) are of the form \(\Tr_{M/K}(\alpha^\ell)\) which is \(\alpha_1^\ell + \dots + \alpha_n^\ell\) by \Cref{thm:trace and norm of separable extension}.

  Now consider \(\Delta = \prod_{i < j} (\alpha_i - \alpha_j)\), the \emph{Vandermonde determinant}, is the determinant of the \emph{Vandermonde matrix}
  \[
    V =
    \begin{pmatrix}
      1 & 1 & \cdots & 1 \\
      \alpha_1 & \alpha_2 & \cdots & \alpha_n \\
      \alpha_1^2 & \alpha_2^2 & \cdots & \alpha_n^2 \\
      \vdots & & \ddots & \vdots \\
      \alpha_1^{n - 1} & \alpha_2^{n - 1} & \dots & \alpha_n^{n - 1}
    \end{pmatrix}
  \]
  Observe that \(A_{ij} = \sum_n \alpha_n^{i + j - 2}, V_{ij} = \alpha_j^{i - 1}\) so
  \begin{align*}
    (VV^T)_{ik} &= V_{ij}V_{kj} \\
                &= \alpha_j^{i - 1} \alpha_j^{k - 1} \\
                &= \sum_n \alpha_n^{i + k - 2} \\
                &= A_{ik}
  \end{align*}
  so \(VV^T = A\). Thus \(0 \neq D = \Delta^2 = |VV^T| = |A|\). Thus \(A\) is non-singular and therfore the bilinear form \(T\) is non-degenerate.
\end{proof}

\begin{remark}
  We will meet \(D\) again shortly. It is the \emph{discriminant} of the polynomial \(f_\alpha(t)\).
\end{remark}

\subsection{Normal Extensions}

We met this definition before:

\begin{definition}[Normal extension]\index{field extension!normal}
  An extension \(K \leq M\) is \emph{normal} if for every \(\alpha \in M\), the minimal polynomial \(f_\alpha(t)\) of \(\alpha\) over \(K\) splits over \(M\).
\end{definition}

\begin{theorem}
  Let \(K \leq M\) be a finite field extension. Then \(K \leq M\) is normal if and only if \(M\) is the splitting field for some \(f(t) \in K[t]\).
\end{theorem}

\begin{proof}\leavevmode
  \begin{itemize}
  \item \(\Rightarrow\): suppose \(K \leq M\) is normal. Pick \(\alpha_1, \dots, \alpha_r \in M\) such that \(M = K(\alpha_1, \dots, \alpha_k)\). Let \(f_{\alpha_i}(t)\) be the minimal polynomial of \(\alpha_i\) over \(K\). Let \(f(t) = \prod_{n = 1}^{k} f_{\alpha_i}(t)\).

    By normality, each \(f_{\alpha_i}(t)\) splits over \(M\) and so is \(f(t)\). \(M\) is the splitting field of \(f(t)\) over \(K\) since if \(\beta_1, \dots, \beta_m\) are the roots of \(f(t)\) then \(M = K(\beta_1, \dots, \beta_m)\).
  \item \(\Leftarrow\): suppose \(M\) is a splitting field for \(f(t)\) over \(K\). Then \(M = K(\beta_1, \dots, \beta_m)\) where \(\beta_j\) are the roots of \(f(t)\) over \(M\). Take \(\alpha \in M\). Let \(f_\alpha(t)\) be the minimal polynomial of \(\alpha\) over \(K\). Let \(M \leq L\) be large enough so that \(f_\alpha(t)\) splits over \(L\).

    Now consdier a \(K\)-homomorphism \(\phi: M \to L\): \(\phi(\beta_j)\) is also a root of \(f(t)\) and is therefore one of the \(\beta_j\). Injectivity of \(K\)-homomorphisms implies that \(\phi\) permutes \(\beta_j\). However, \(M = K(\beta_1, \dots, \beta_m)\) and so \(\phi\) is determined by the images of \(\beta_j\), thus \(\phi(M) = M\).

    However, if \(\alpha_i\) is a root of \(f_\alpha(t)\) in \(L\), then there is a \(K\)-homomorphism
    \begin{align*}
      K(\alpha) &\to K(\alpha_i) \\
      \alpha &\mapsto \alpha_i
    \end{align*}
    This extends by \ref{cor:homomorphic extension of finite extension} to a \(K\)-homomorphism \(\phi: M \to L\). But \(\phi(M) = M\) so \(\alpha_i \in M\). Thus \(M\) is normal over \(K\).
  \end{itemize}
\end{proof}

\begin{remark}
  As the same for separability, being normal is equivalent to being normally generated. We will show in example sheet that \(K \leq L\) is a normal and finite extension if and only if \(L = K(\alpha_1, \dots, \alpha_r)\) with the minimal polynomial of each adjoined element splitting over \(L\).
\end{remark}

\begin{definition}[\(K\)-automorphism group]
  Let \(K \leq M\) be a finite extension. Its \emph{\(K\)-automorphism group} is
  \[
    \aut_K(M) = \Hom_K(M, M).
  \]
\end{definition}

From before we know that such \(K\)-automorphisms are isomorphisms and thus have inverses (well, it name says so).

\begin{lemma}
  \[
    |\aut_K(M)| \leq |M:K|.
  \]
\end{lemma}

\begin{proof}
  \Cref{cor:homomorphic extension of finite extension}.
\end{proof}

\begin{theorem}
  \label{thm:galois criterion}
  Let \(K \leq M\) be a finite field extension, then
  \[
    |\aut_K(M)| = |M:K|
  \]
  if and only if the extension is both normal and separable.
\end{theorem}

\begin{definition}[Galois extension]\index{field extension!Galois}
  A finite field extension that is normal and separable is a \emph{Galois extension}.
\end{definition}

\begin{definition}[Galois group]\index{Galois group}
  Let \(K \leq M\) be a Galois extension. Then the \(K\)-automorphism group of \(M\) is the \emph{Galois group} of \(M\) over \(K\), denoted by
  \[
    \gal(M/K).
  \]
\end{definition}

\begin{remark}
  Some authors use the term ``Galois group'' as a synonym for automorphism group even when the extension is not Galois.
\end{remark}

\begin{proof}[Proof of \Cref{thm:galois criterion}]
  \label{proof:galois criterion}
  Suppose \(|\aut_K(M)| = |M:K| = n\). Let \(M \leq L\) be large enough. The \(n\) distinct \(K\)-automorphisms \(\phi: M \to M\) extend to \(n\) \(K\)-homomorphisms \(\phi: M \to L\) and \Cref{thm:separability of finite extension} says that \(M\) is separable over \(K\).

  For normality, pick \(\alpha \in M\) with minimal polynomial \(f_\alpha(t)\) over \(K\). Take \(M = K(\alpha_1, \dots, \alpha_m)\) as in the proof of \Cref{cor:homomorphic extension of finite extension} with \(\alpha = \alpha_1\) and \(L = M\). We only get \(|M:K|\) extensions of the inclusion \(K \embed M\) if each inequality in the proof is an equality. In particular, we need the number of \(K\)-homomorphisms \(K(\alpha_1) \to M\) to be \(|K(\alpha_1):K|\). But then root correspondence says we have \(|K(\alpha):K|\) distinct roots of \(f_\alpha(t)\) in \(M\). Thus \(f_\alpha(t)\) splits over \(M\).

  Conversely, suppose \(K \leq M\) is separable and normal. Then for \(K \leq M \leq L\) with \(L\) large enough, separability implies there are \(|M:K|\) \(K\)-homomorphisms \(\phi: M \to L\) by \Cref{thm:separability of finite extension}. However, \(K \leq M\) is normal implies that it is the splitting field for some polynomial \(f(t) \in K[t]\) and thus \(M = K(\alpha_1, \dots, \alpha_n)\) where \(f(t) = \prod_{i = 1}^n (t - \alpha_i)\). Note that \(\phi(\alpha_j)\) is also a root of \(\phi(f(t)) = f(t)\), and is therefore one of the \(\alpha_j\). Thus \(\phi(M) = M\). Thus \(|\aut_K(M)| = |M:K|\).
\end{proof}

\begin{remark}
  In the previous proof we have shown that if \(K \leq M \leq L\), \(\phi \in \Hom_K(M, L)\) and \(K \leq M\) is normal then \(\phi(M) = M\).
\end{remark}

\begin{eg}\leavevmode
  \begin{enumerate}
  \item Consider \(\Q \leq \Q(\sqrt 2, i)\), which is Galois:
    \[
      \gal(\Q(\sqrt 2, i)/\Q) = \generation{\sigma: \sqrt 2 \mapsto -\sqrt 2, \tau: i \mapsto -i} \cong C_2 \times C_2.
    \]
    All non-identity elements have order \(2\).
  \item Let \(f(t) = t^3 - 2\). The splitting field of \(f(t)\) over \(\Q\) is \(\Q(\sqrt[3]{2}, \omega)\) where \(\omega\) is a primitive cubic root of unity. Thus \(\Q \leq \Q(\sqrt[3]{2}, \omega)\) is Galois with \(|\gal(\Q(\sqrt[3]{2}, \omega)/\Q|) = |\Q(\sqrt[3]{2}, \omega):\Q| = 6\). The Galois group contains
    \begin{align*}
      \sigma: \sqrt[3]{2} &\mapsto \omega\sqrt[3]{2}, \, \omega \mapsto \omega \\
      \tau: \sqrt[3]{2} &\mapsto \sqrt[3]{2}, \, \omega \mapsto \omega^2 \text{ (complex conjugation)}
    \end{align*}
    and is generated by these two elements. It is an exercise to show that \(\gal(\Q(\sqrt[3]{2}, \omega)/\Q) \cong D_6 \cong S_3\).
  \end{enumerate}
\end{eg}

\section{Fundamental Theorem of Galois Theory}

\begin{definition}[Fixed field]\index{fixed field}
  Let \(K \leq L\) be a field extension and \(H \leq \aut_K(L)\). The \emph{fixed field} of \(H\) is
  \[
    L^H = \{\alpha \in L: \sigma(\alpha) = \alpha \text{ for all } \sigma \in H \}.
  \]
\end{definition}

The fixed field is a field and \(K \leq L^H \leq L\).

\begin{theorem}[Fundamental Theorem of Galois Theory]
  \label{thm:fundamental}
  Let \(K \leq L\) be a finite Galois extension. Then
  \begin{enumerate}
  \item There is a one-to-one correspondence
    \begin{align*}
      \left\{
        \begin{array}[h]{c}
          \text{intermediate subfield} \\
          K \leq M \leq L
        \end{array}
      \right\}
      &\longleftrightarrow
      \left\{
        \begin{array}[h]{c}
          \text{subgroup } \\
          H \leq \gal(L/K)
        \end{array}
      \right\} \\
      %
      M &\mapsto \aut_M(L) \\
      L^H &\mapsfrom H
    \end{align*}
  \item \(H\) is a normal subgroup of \(\gal(L/K)\) if and only if \(K \leq L^H\) is normal if and only if \(K \leq L^H\) is Galois.
  \item If \(H \normal \gal(L/K)\) then the map
    \[
      \theta: \gal(L/K) \to \gal(L^H/K)
    \]
    given by restriction to \(L^H\) is a surjective group homomorphism with kernel \(H\).
  \end{enumerate}
\end{theorem}

\begin{remark}\leavevmode
  \begin{enumerate}
  \item Observe that \(M \leq L\) is Galois and so we could have written \(\gal(L/M)\) instead of \(\aut_M(L)\). To see this,
    \begin{itemize}
    \item separability: follows from \Cref{lem:transitivity of separability}.
    \item normality: if \(\alpha \in L\) then the minimal polynomial of \(\alpha\) over \(M\) divides the minimal polynomial of \(\alpha\) over \(K\). But the latter splits over \(L\) (see example sheet).
    \end{itemize}
  \item If \(K \leq M\) is normal then the remark after \hyperref[proof:galois criterion]{proof} of \Cref{thm:galois criterion} says if \(\sigma: L \to L\) then \(\sigma(M) = M\) and so we can talk about the restriction of \(\alpha\) to \(M\) giving an automorphism of \(M\).
  \end{enumerate}
\end{remark}

\begin{eg}\leavevmode
  \begin{enumerate}
  \item \(\Q \leq \Q(\sqrt{2}, i)\). We saw in example on page~\pageref{eg:Q(sqrt 2, i)} the lattices of intermediate subfields and subgroups \(\gal(\Q(\sqrt 2, i)/\Q) \cong C_2 \times C_2\) which is abelian. Thus all subgroups are normal and all intermediate subfields are normal extensions of \(\Q\).
  \item \(\Q \leq \Q(\sqrt[3]{2}, \omega)\).
    \[
      \begin{tikzcd}
        & \Q(\sqrt[3]{2}, \omega) \ar[dl, dash, "3"] \ar[d, dash, "2"] \ar[dr, dash, "2"] \ar[drr, dash, "2"] \\
        \Q(\omega) \ar[dr, dash, "2"] & \Q(\sqrt[3]{2}) \ar[d, dash, "3"] & \Q(\omega\sqrt[3]{2}) \ar[dl, dash, "3"] & \Q(\omega^2\sqrt[3]{2}) \ar[dll, dash, "3"] \\
        & \Q
      \end{tikzcd}
    \]
    \[
      \begin{tikzcd}
        & 1 \ar[dl, dash, "3"] \ar[d, dash, "2"] \ar[dr, dash, "2"] \ar[drr, dash, "2"] \\
        \generation{\sigma} \ar[dr, dash, "2"] & \generation{\tau} \ar[d, dash, "3"] & \generation{\sigma^2\tau} \ar[dl, dash, "3"] & \generation{\sigma\tau} \ar[dll, dash, "3"] \\
        & \generation{\sigma, \tau} \cong D_6
      \end{tikzcd}
    \]
    The subgroup \(H\) of order \(3\) is normal but those of order \(2\) are not so the map \(\gal(\Q(\sqrt[3]{2}, \omega)/\Q) \to \gal(\Q(\omega)/\Q)\), i.e.\ \(D_6 \to C_2\), generated by conjugation has kernel \(\generation{\sigma}\).
    \end{enumerate}
\end{eg}

\begin{theorem}[Artin's]
  \label{thm:artin}
  Let \(K \leq L\) be a field extension and \(H \leq \aut_K(L)\) be a finite subgroup. Let \(M = L^H\). Then \(M \leq L\) is a finite Galois extension and \(H = \gal(L/M)\).
\end{theorem}

\begin{remark}
  This implies one way of Galois correspondence:
  \[
    H \mapsto L^H \mapsto \gal(L/L^H) = H.
  \]
\end{remark}

\begin{proof}
  Take \(\alpha \in L\). We first show that \(|M(\alpha):M| \leq |H|\). Let \(\{\alpha_1, \dots, \alpha_n\}\) be the distinct images of \(\alpha\) under \(H\), i.e.\ \(\{\phi(\alpha): \phi \in H\}\). Define \(g(t) = \prod_{i = 1}^{n} (t - \alpha_i) \). Each \(\phi\) induces an endomorphism on \(L[t]\) under which \(g(t)\) is invariant since \(\phi\) permutes the \(\alpha_i\). Thus the coefficients lie in \(L^H = M\) and so \(g(t) \in M[t]\). By definition \(g(\alpha) = 0\) since \(\alpha\) is one of the \(\alpha_i\). Hence the minimal polynomial \(f_\alpha(t)\) of \(\alpha\) over \(M\) divides \(g(t)\). Thus
  \[
    |M(\alpha):M| = \deg f_\alpha(t) \leq \deg g(t) \leq |H|.
  \]
  This step shows that \(\alpha\) is algebraic over \(M\) and \(f_\alpha(t)\) is separable since \(g(t)\) is. Thus \(M \leq L\) is a separable extension.

  Next we show that \(M \leq L\) is a simple extension. Pick \(\alpha \in L\) with \(|M(\alpha):M|\) maximal. We will show that \(L = M(\alpha)\) for this \(\alpha\). Suppose \(\beta \in L\). Then \(M \leq M(\alpha, \beta)\) is finite and separably generated and hence is a finite separable extension. By \nameref{thm:primitive} \(M(\alpha, \beta) = M(\gamma)\) for some \(\gamma\). But then
  \[
    M \leq M(\alpha) \leq M(\gamma)
  \]
  so by the maximality of \(|M(\alpha):M|\), \(M(\alpha) = M(\gamma)\). Thus \(\beta \in M(\gamma) = M(\alpha)\) so \(L = M(\alpha)\).

  It follows that \(|L:M| \leq |H|\).

  Finally, \(H \leq \aut_M(L)\) so
  \[
    |L:M| = |M(\alpha):M| \leq |H| \leq |\aut_M(L)| \leq |L:M|
  \]
  so we must have equality throughtout, i.e.\ \(|L:M| = |\aut_M(L)| = |H|\) so \(M \leq L\) is a finite Galois extension and \(H = \gal(L/M)\).
\end{proof}

\begin{theorem}
  \label{thm:fixed field criterion for galois extension}
  Let \(K \leq L\) be a finite field extension. TFAE:
  \begin{enumerate}
  \item \(K \leq L\) is Galois,
  \item \(L^H = K\) where \(H = \aut_K(L)\).
  \end{enumerate}
\end{theorem}

\begin{remark}
  The theorem allows some authors to give yet another definition of a Galois extension.
\end{remark}

\begin{proof}\leavevmode
  \begin{enumerate}
  \item \(1 \Rightarrow 2\): Let \(M = L^H\) where \(H = \aut_K(L)\). By \nameref{thm:artin} \(M \leq L\) is a Galois extension. Now we have two Galois extensions, giving the equalities
    \begin{align*}
      |H| &= |\gal(L/K)| = |L:K| \\
          &= |\gal(L/M)| = |L:M|
    \end{align*}
    As \(K \leq L^H = M\), equality.
  \item \(2 \Rightarrow 1\): \nameref{thm:artin}.
  \end{enumerate}
\end{proof}

\begin{proof}[Proof of \nameref{thm:fundamental}]\leavevmode
  \begin{enumerate}
  \item Composing the maps
    \begin{align*}
      H &\mapsto L^H \\
      M &\mapsto \gal(L/M)
    \end{align*}
    gives \(H \to H\) by \nameref{thm:artin}. Composition the other way \(M \mapsto \gal(L/M) \mapsto L^H\) where \(H = \gal(L/M)\) gives \(M\) by \Cref{thm:fixed field criterion for galois extension}.
    %since \(M \leq L^H\) and
    %\[
    %  |L:L^H| = |H| = |\gal(L/M)| = |L:M|.
    %\]
  \item Take \(H \leq \gal(L/K)\). Then for \(\phi \in \gal(L/K)\), \(L^{\phi H \phi^{-1}} = \phi(L^H)\) so by 1 \(H \normal \gal(L/K)\) if and only if \(\phi(L^H) = L^H\). Let \(M = L^H\). We will show that \(K \leq M\) is normal if and only if \(\phi(M) = M\) for every \(\phi \in \gal(L/K)\):
    \begin{itemize}
    \item \(\Rightarrow\): remark 2 after \Cref{thm:fundamental}.
    \item \(\Leftarrow\): suppose \(\phi(M) = M\) for all \(\phi \in \gal(L/K)\). Pick \(\alpha \in M\) and let \(f_\alpha(t)\) be its minimal polynomial over \(K\). We take \(\beta\) as a root for \(f_\alpha(t)\) in \(L\) (which is possible by normality). Then there is a \(K\)-homomorphism
      \begin{align*}
        K(\alpha) &\to K(\beta) \\
        \alpha &\mapsto \beta
      \end{align*}
      This extends to a \(K\)-homomorphism \(\phi: L \to L\). Since we assume \(\phi(M) = M\), \(\phi(\alpha) = \beta \in M\). Thus \(K \leq M\) is normal.
    \end{itemize}

    Note that \(K \leq M\) is separable since \(K \leq M \leq L\) and \(K \leq L\) is separable. Thus \(K \leq M\) is Galois.
  \item By remark 2 after \Cref{thm:fundamental}, the restriction map \(\theta: \gal(L/K) \to \gal(L^H/K)\) is well-defined. Surjectivity follows from being able to extend a \(K\)-homomorphism \(L^H \to L^H \leq L\) to a \(K\)-homomorphism \(L \to L\). Clearly \(H \leq \ker \theta\). However
    \begin{align*}
      \frac{|L:K|}{|\ker \theta|} &= \frac{|\gal(L/K)|}{|\ker \theta|} \\
                                  &= |\gal(L^H/K)| \text{ by surjectivity of \(\theta\)} \\
                                  &= |L^H:K| \text{ since \(K \leq L^H\) is Galois} \\
                                  &= \frac{|L:K|}{|L:L^H|} \text{ by Tower Law}
    \end{align*}
    Then
    \[
      |\ker \theta| = |L:L^H| = |\gal(L/L^H)| = |H|
    \]
    by \nameref{thm:artin}. Thus \(\ker \theta = H\).
  \end{enumerate}
\end{proof}

\subsection{Galois Group of Polynomials}

\begin{definition}[Galois group]\index{Galois group}
  Let \(f(t) \in K[t]\) be a separable polynomial and \(K \leq L\) with \(L\) a splitting field for \(f(t)\). Then the \emph{Galois group} of \(f(t)\) over \(K\) is
  \[
    \gal(f) = \gal(L/K).
  \]
\end{definition}

Since \(L\) is a splitting field for \(f(t)\), \(L = K(\alpha_1, \dots, \alpha_n)\) where \(\alpha_1, \dots, \alpha_n\) are the roots of \(f(t)\) in \(L\). Observe that if \(\phi \in \gal(L/K)\) it maps bijectively roots of \(f(t)\) to itself. Thus \(\phi\) permutes the \(\alpha_i\).

Moreover if \(\phi\) fixes each \(\alpha_i\) it also fixes all elements of \(L\) and so it is the identity map. Thus \(\gal(f)\) may be regarded as a permutation group of the \(n\) roots, so in particular admitting a permutation representation in \(S_n\).

\begin{lemma}
  \label{lem:galois orbits}
  Suppose separable \(f(t) = g_1(t) \cdots g_s(t)\) with \(g_i(t)\) irreducible in \(K[t]\) is a factorisation in \(K[t]\). Then the orbits of \(\gal(f)\) acting on the roots of \(f(t)\) correspond to the factors \(g_j(t)\): two roots are in the same orbit if and only if they are roots of the same \(g_i(t)\).

  In particular if \(f(t) \in K[t]\) is irreducible, there is only one orbit, i.e.\ \(\gal(f)\) acts transitively on the roots of \(f(t)\).
\end{lemma}

\begin{proof}
  Let \(\alpha_k\) and \(\alpha_\ell\) be in the same orbit under \(\gal(f)\). Thus there is \(\phi \in \gal(f)\) with \(\alpha_\ell = \phi(\alpha_k)\). But if \(\alpha_k\) is a root of \(g_j(t)\) then \(\phi(\alpha_k)\) is also a root of \(g_j(t)\).

  Conversely if \(\alpha_k\) and \(\alpha_\ell\) are roots of \(g_j(t)\), then
  \[
    K(\alpha_k) \cong K[t]/(g_j(t)) \cong K(\alpha_\ell) \leq L
  \]
  Denote by \(\phi_0\) the isomorphsim \(K(\alpha_k) \to K(\alpha_\ell)\). \(\phi_0\) extends to \(\phi \in \gal(L/K)\). Thus \(\alpha_k\) and \(\alpha_\ell\) are in the same orbit.
\end{proof}

\begin{lemma}
  \label{lem:transitive subgroups of Sn}
  The transitive subgroups of \(S_n\) for \(n \leq 5\) are
  \[
  \begin{array}[h]{c|l}
    n & \\ \hline
    2 & S_2 \cong C_2 \\ \hline
    3 & A_3 \cong C_3, S_3 \\ \hline
    4 & C_4, V_4, D_8, A_4, S_4 \\ \hline
    5 & C_5, D_{10}, H_{20}, A_5, S_5 \\ \hline
  \end{array}
  \]
  where \(H_{20}\) is generated by a \(5\)-cycle and a \(4\)-cycle.
\end{lemma}

\begin{theorem}
  \label{thm:galois group Sp}
  Let \(p\) be a prime and \(f(t) \in \Q[t]\) irreducible of degree \(p\). Suppose \(f(t)\) has exactly \(2\) non-real roots in \(\C\). Then
  \[
    \gal(f) \cong S_p.
  \]
\end{theorem}

\begin{proof}
  \(\gal(f)\) acts on the \(p\) distinct roots of \(f(t)\) in a splitting field \(L\) of \(f(t)\) (in \(\C\)). By \Cref{lem:galois orbits} the irreducibility of \(f(t)\) implies that \(\gal(f)\) acts transitively on \(p\) roots so by orbit-stabiliser \(p \divides |\gal(f)|\) but
  \[
    |\gal(f)| \leq |S_p| = p!
  \]
  and so \(\gal(f)\) has a Sylow \(p\)-subgroup of order \(p\), necessarily cyclic. Thus \(\gal(f)\) contains a \(p\)-cycle. Supposition that there are exactly \(2\) non-real roots gives that complex conjugation yields a transposition in \(\gal(f)\). The \(p\)-cycle and the transposition generate \(S_p\).
\end{proof}

\begin{eg}
  Let \(f(t) = t^5 - 6t + 3 \in \Q[t]\), then claim \(\gal(f) \cong S_5\):

  \begin{proof}
    \(f(t)\) is irreducible by Eisenstein with \(p = 3\). We want to show that \(f(t)\) has \(3\) real roots (so \(2\) non-real roots) and apply the previous theorem.

    Now we apply knowledge from analysis:
    \[
      f(-2) = -17, f(-1) = 8, f(1) = -2, f(2) = 23
    \]
    and \(f'(t) = 5t^4 - 6\) which has two real roots. Thus by Intermediate Value Theorem there are \(3\) real roots while by Rolle's Theorem there are at most \(3\) real roots.
  \end{proof}
\end{eg}

\begin{definition}[Discriminant]\index{discriminant}
  Let \(f(t) \in K[t]\) have distinct roots \(\alpha_1, \dots, \alpha_n\) (in a splitting field) (\(f(t)\) need not be irreducible). Set \(\Delta = \prod_{i < j}(\alpha_i - \alpha_j)\). Then the \emph{discriminant} of \(f\) is
  \[
    D(f) = \Delta^2 = \prod_{i < j}(\alpha_i - \alpha_j)^2 = (-1)^{\frac{n(n-1)}{2}} \prod_{i \neq j}(\alpha_i - \alpha_j).
  \]
\end{definition}

\begin{remark}
  We have already met this in the \hyperref[proof:vandermonde]{proof} of \Cref{thm:vandermonde}.
\end{remark}

\begin{lemma}
  Let \(f(t) \in K[t]\) be separable of degree \(n\) with \(\ch K \neq 2\). Then
  \[
    \gal(f) \leq A_n \Leftrightarrow D(f) \text{ is a square in } K.
  \]
\end{lemma}

\begin{proof}
  Let \(L\) be a splitting field of \(f(t)\) over \(K\). Then \(D(f) \neq 0\) and is fixed by all elements of \(G = \gal(L/K)\) as the latter permutes the roots. Thus \(D \in K\) since \(L^G = K\) by Galois correspondence.

  If \(\sigma \in G\) then \(\sigma(\Delta) = (\sgn \sigma) \Delta\) where we are regarding \(G\) as a subgroup of \(S_n\) and \(\sgn\) is the signature (this is where we need \(\ch K \neq 2\)). Thus if \(G \leq A_n\) we got that \(\Delta\) is fixed by all \(\sigma \in G\). Thus \(\Delta \in K = L^G\).

  On the other hand if \(G \nleq A_n\) we get \(\sigma(\Delta) = -\Delta\) if \(\sigma\) is odd and so \(\Delta \notin K = L^G\). Finally note that if \(D\) has square roots they must be \(\pm \Delta\).
\end{proof}

\begin{eg}\leavevmode
  \begin{enumerate}
  \item \(n = 2\): \(f(t) =t^2 + bt + c = (t - \alpha_1)(t - \alpha_2)\) has discriminant
    \[
      D(f) = (\alpha_1 - \alpha_2)^2 - (\alpha_1 + \alpha_2)^2 - 4\alpha_1\alpha_2 = b^2 - 4c
    \]
  \item \(n = 3\): \(f(t) = t^3 + ct + d\) has discriminant
    \[
      D(t) = -4c^3 - 27d^2
   \]
   \begin{remark}
     Any general monic cubic \(g(t)\) can be put into this form by a suitable substitution \(f(t) = g(t_1 + \lambda)\) for suitable \(\lambda\). Note \(D(f) = D(g)\).
   \end{remark}
 \end{enumerate}
\end{eg}

\begin{eg}\leavevmode
  \begin{itemize}
  \item \(f(t) = t^3 - t - 1 \in \Q[t]\) irreducible in \(\Z[t]\) since irreducible mod \(2\). \(D(f) = -23\) which is not a square in \(\Q\). Thus \(\gal(f) \cong S_3\).
  \item \(f(t) = t^3 - 3t - 1 \in \Q[t]\) irredicible since irreducible mod \(2\). \(D(f) = 81\) which is a square so \(\gal(f) \cong A_3\).
  \end{itemize}
\end{eg}

Now we move on to irreducible quartics. We saw that the possible Galois groups are
\[
  V_4, A_4, C_4, D_8, S_4
\]
with the first two being subgroups of \(A_4\). From looking at the discriminant one gets information as whether the group is one of \(V_4, A_4\) or one of \(C_4, D_8, S_4\). We need further methods to pin down which group we are dealing with.

\begin{theorem}[mod \(p\) reduction]
  \label{thm:mod p reduction}
  Let \(f(t) \in \Z[t]\) be monic of degree \(n\) with \(n\) distinct roots in a splitting field. Let \(p\) be a prime such that \(\red f(t)\), the reduction of \(f(t)\) mod \(p\), also has \(n\) distinct roots in a splitting field. Let \(\red f(t) = \red{g_1}(t) \dots \red{g_s}(t)\) be the factorisation into irreducible in \(\F_p[t]\) with \(n_j =  \deg \red{g_j}(t)\). Then
  \[
    \gal(\red f) \embed \gal(f)
  \]
  and has an element of cycle type \((n_1, n_2, \dots, n_s)\).
\end{theorem}

\begin{proof}
  See \hyperref[rmk:mod p reduction]{Remark} after discussion about Galois groups over finite fields. The fact that \(\gal(\bar f) \embed \gal(f)\) is from IID Number Fields. Look at Tony Scholl's teaching page on Galois Theory.
\end{proof}

\begin{eg}
  Given a quartic of the form \(f(t) = t^4 + dt + e\), its discriminant is \(D(t) = -27d^4 + 256e^3\). Consider \(f(t) = t^4 - t -1\), irreducible since irreducible mod \(2\) and \(D(f) = -283\) which is not a square.

  Consider mod \(7\),
  \[
    \red f(t) = t^4 - t - 1 = (t + 4)(t^3 + 3t^2 + 2t + 5)
  \]
  the second factor is irreducible over \(\F_7\) since it has no roots in \(\F_7\). By \Cref{thm:mod p reduction} \(\gal(f)\) contains an element of cycle type \((1, 3)\), i.e.\ a \(3\)-cycle. We deduce that \(\gal(f) \cong S_4\) as it is the only transitive subgroup of \(S_4\) that contains an odd permutation and a \(3\)-cycle.
\end{eg}

\subsection{Galois Theory of Finite Fields}

Recall what we already know from \Cref{sec:section 1}: a finite field \(\F\) is of characteristic \(p > 0\) where \(p\) is a prime and \(|\F| = p^r\) for some \(r\). The multiplicative group of \(\F\) is cyclic. It is a splitting field for \(t^n - 1\) over \(\F_p\) where \(n = p^r - 1\). By the uniqueness of splitting fields this is unique. Observe we could also describe \(\F\) as the splitting field of \(t^{p^r} - t\) over \(\F_p\).

What we haven't shown yet is that for any \(p^r\) there is a field with \(|\F| = p^r\).

\begin{definition}[Frobenius automorphism]\index{Frobenius automorphism}
  Let \(\F\) be a finite field of characteristic \(p\). The \emph{Frobenius automorphism} of \(\F\) is
  \begin{align*}
    \phi_p: \F &\to \F \\
    \alpha &\mapsto \alpha^p
  \end{align*}
\end{definition}

\begin{remark}
  \((\alpha + \beta)^p = \alpha^p + \beta^p\) since all terms in binomial expansion is divisible by \(p\). Also \(\F_p\) is fixed under this so this is an \(\F_p\)-automorphism.
\end{remark}

Since \(t^{p^r} - t\) splits as a product of linear factors \((t - \alpha)\) in \(\F\), we have that \(\F_p \leq \F\) is a Galois extension and so we consider \(G = \gal(\F/\F_p)\). It has order \(r\) since \(|\F:\F_p| = r\).

\begin{theorem}[Galois group of finite fields]
  Let \(\F\) be a finite field with \(|\F| = p^r\). Then \(\F_p \leq \F\) is a Galois extension with
  \[
    \gal(\F/\F_p) = \generation{\phi_p} \cong C_r.
  \]
\end{theorem}

\begin{proof}
  It remains to show that the order of Frobenius automorphism is \(r\). Suppose \(\phi_p^s = \id\). Then \(\alpha^{p^s} = \alpha\) for all \(\alpha \in \F\). But \(t^{p^s} - t\) has at most \(p^s\) roots in \(\F\). So we conclude \(s \geq r\). Observe that \(\phi_p^r = \id\) since \(\alpha^{p^r} = \alpha\) for all \(\alpha \in \F\).
\end{proof}

Now apply \nameref{thm:fundamental},
\[
  \{\text{intermediate fields } \F_p \leq M \leq \F\} \leftrightarrow \{\text{subgroups } H \leq G \}
\]
where \(G = \gal(\F/\F_p)\) is cyclic.

But we know all about subgroups of a cyclic group with generator \(\phi_p\) with order \(r\). There is exactly one group of order \(s\) for each \(s \divides r\) generated by \(\phi_p^{r/s}\). The corresponding intermediate subfields are the fixed fields \(\F^{\generation{\phi_p^{r/s}}}\) and \(|\F:\F^{\generation{\phi_p^{r/s}}}| = s\). By Tower Law \(|\F^{\generation{\phi_p^{r/s}}}:\F_p| = r/s\).

Observe that all subgroups of cyclic groups are normal and therefore all our intermediate fields are normal extensions of \(\F_p\). Thus \(\gal(\F^{\generation{\phi_p^{r/s}}}/\F_p) \cong \gal(\F/\F_p)/H\) where \(H = \generation{\phi_p^{r/s}}\).

\begin{corollary}
  \label{cor:galois group of finite fields}
  Let \(\F_p \leq M \leq \F\) be finite fields. Then \(\gal(\F/M)\) is cyclic, generated by \(\phi_p^n\) where \(\phi_p\) is the Frobenius map and \(|M| = p^n\) and \(M\) is the fixed field of \(\generation{\phi_p^n}\).
\end{corollary}

\begin{proof}
  Set \(n = r/s\).
\end{proof}

\begin{theorem}[Existence of finite fields]
  Let \(p\) be a prime and \(n \geq 1\). Then there is a field of order \(p^n\) unique up to isomorphism.
\end{theorem}

\begin{proof}
  Consider the splitting field \(L\) of \(f(t) = t^{p^n} - t\) over \(\F_p\). \(\F_p \leq L\) is a finite Galois extension. However the roots of \(f(t)\) form a field, the fixed field of \(\phi_p^n\). Set \(L = \F\) and \(|\F:\F_p| = n\).
\end{proof}

\begin{remark}[\nameref{thm:mod p reduction}]
  \label{rmk:mod p reduction}
  We will discuss in IID Number Fields that \(\gal(\red f) \embed \gal(f)\) if \(f(t) \in \Z[t]\). We factorised \(\red f(t) = \red{g_1}(t)\cdots \red{g_s}(t)\) as a product of irreducibles (actually \(\red{g_s}(t)\) lives in \(p\)-adic integers). We knew from \Cref{lem:galois orbits} that the orbits of \(\gal(\red f)\) correspond to the factorisation. We know \(\gal(\red f)\) is cyclic generated by the Frobenius map, which must have cyclic type \((n_1, \dots, n_s)\) where \(n_j = \deg \red{g_j}(t)\).
\end{remark}

\section{Cyclotomic and Kummer Extensions, Cubics and Quartics, Solution by Radicals}

\subsection{Cyclotomic Extensions}

\begin{definition}[Cyclotomic extension]\index{field extension!cyclotomic}
  Suppose \(\ch K = 0\) or \(p\) is prime where \(p \ndivides m\). The \emph{\(m\)th cyclotomic extension} of \(K\) is the splitting field \(L\) of \(t^m - 1\).
\end{definition}

\begin{remark}
  \(f(t) = t^m - 1\) and \(f'(t) = mt^{m - 1}\) have no common roots and so the roots of \(f(t)\) are distinct, which are the \(m\)th roots of unity. They form a finite subgroup \(\mu_m\) of \(L^\times\) and hence by (1.28) \Cref{tbf} a cyclic group \(\generation \zeta\). Thus \(L = K(\zeta)\) is a simple extension.
\end{remark}

\begin{definition}[Primitive root of unity]
  An element \(\zeta \in \mu_m\) is a \emph{primitive} \(m\)th root of unity if \(\mu_m = \generation \zeta\).
\end{definition}

Choosing a primitive \(m\)th root of unity determines an isomorphism
\[
  \mu_m \to \Z/m\Z.
\]
Note that \(\zeta^i\) is a generator of \(\mu_m\) if and only if \((i, m) = 1\) and so the primitive roots of unity correspond to elements of \((\Z/m\Z)^\times\), the unit group of \(\Z/m\Z\).

Now consider Galois groups of cyclotomic extensions. We will see that they must be abelian. Observe \(f(t) = t^m - 1\) is separable and so the extension \(K \leq L\) is Galois. Let \(G = \gal(L/K)\). An element \(\sigma \in G\) sends a primitive \(m\)th root of unity \(\zeta\) to another \(m\)th root of unity \(\zeta^i\) where \((i, m) = 1\). Then \(\zeta \mapsto \zeta^i\) determines a \(K\)-homomorphism
\[
  K(\zeta) \to K(\zeta)
\]
which is an injective map.

\begin{definition}
  Define
  \begin{align*}
    \theta: G &\to (\Z/m\Z)^\times \\
    \sigma &\mapsto i \text{ where } \sigma(\zeta) = \zeta^i
  \end{align*}
  It is a group homomorphism since if \(\sigma(\zeta) = \zeta^i, \phi(\zeta) = \zeta^j\) then \((\sigma\phi)(\zeta) = \sigma(\zeta^j) = \zeta^{ij}\).
\end{definition}

Thus \(G\) is abelian and we regard \(G\) as a subgroup of \((\Z/m\Z)^\times\).

\begin{definition}[Cyclotomic polynomial]\index{cyclotomic polynomial}
  The \emph{\(m\)th cyclotomic polynomial} is
  \[
    \Phi_m(t) = \prod_{i \in (\Z/m\Z)^\times} (t - \zeta^i),
  \]
  the product of the linear factors of \(t^m - 1\) corresponding to the primitive \(m\)th roots of unity.
\end{definition}

\begin{remark}
  \begin{align*}
    f(t) &= t^m - 1 \\
         &= \prod_{i \in \Z/m\Z} (t - \zeta^i) \\
         &= \prod_{d \divides m} \Phi_d(t)
  \end{align*}
\end{remark}

\begin{eg}
  Take \(K = \Q\), then
  \begin{align*}
    \Phi_1(t) &= t - 1 \\
    \Phi_2(t) &= t + 1 \\
    \Phi_3(t) &= t^2 + t + 1 \\
    \Phi_4(t) &= t^2 + 1 \\
    \intertext{and from which we can deduce that}
    \Phi_8(t) &= t^4 + 1
  \end{align*}
  since \(t^8 - 1 = (t - 1)(t + 1)(t^2 + 1)(t^4 + 1)\).
\end{eg}

\begin{lemma}\leavevmode
  \begin{itemize}
  \item \(\Phi_m(t) \in \Z[t]\) if \(\ch K = 0\) (i.e.\ \(\Q \embed K\)).
  \item \(\Phi_m(t) \in \F_p[t]\) if \(\ch K = p\) (i.e.\ \(\F_p \embed K\)).
  \end{itemize}
\end{lemma}

\begin{proof}
  By induction on \(m\). If \(m = 1\) then done.

  For \(m > 1\), consider
  \[
    f(t) = t^m - 1 =\Phi_m(t) \prod_{\substack{d \divides m \\ d \neq m}} \Phi_d(t)
  \]
  Note that \(\prod_{d \divides m, d \neq m} \Phi_d(t)\) is monic and is in \(\Z[t]\) or \(\F_p[t]\) by induction hypothesis. Now
  \begin{itemize}
  \item if \(\ch K = 0\) we deduce \(\Phi_m(t) \in \Q[t]\) by division of polynomials. By Gauss' Lemma it is in \(\Z[t]\).
  \item if \(\ch K = p > 0\) we deduce by division that \(\Phi_m(t) \in \F_p[t]\).
  \end{itemize}
\end{proof}

\begin{lemma}
  \label{lem:irreducibility of cyclotomic}
  The homomorphism \(\theta: G \to (\Z/m\Z)^\times\) defined above is an isomorphism if and only if \(\Phi_m(t)\) is irreducible.
\end{lemma}

\begin{proof}
  We know from \Cref{lem:galois orbits} that the orbits of \(G = \gal(L/K)\) correspond to the factorisation of \(f(t)\) in \(K[t]\). In particular the primitive \(m\)th roots of unity form one orbit if and only if \(\Phi_m(t)\) is irreducible. Then \(\theta\) is surjective if and only if \(\Phi_m(t)\) is irreducible.
\end{proof}

\begin{theorem}
  Let \(L\) be the \(m\)th cyclotomic extension of finite fields \(\F = \F_q\) where \(q = p^n\). Then the Galois group \(G = \gal(L/\F)\) is isomorphic to the cyclic subgroup of \((\Z/m\Z)^\times\) generated by \(q\).
\end{theorem}

\begin{proof}
  We know from \Cref{cor:galois group of finite fields} that \(G\) is generated by \(\alpha \mapsto \alpha^{p^n}\) so
  \[
    \theta(G) = \theta(\generation{\alpha \mapsto \alpha^{p^n}}) =  \generation{p^n} = \generation q \leq (\Z/m\Z)^\times.
  \]
\end{proof}

\begin{remark}
  If \((\Z/m\Z)^\times\) is not cyclic then \(\theta\) is not surjective for any finite field and \(\Phi_m(t)\) is reducible over \(\F\).
\end{remark}

\begin{eg}
  Let \(\F = \F_3\). Then
  \[
    \Phi_8(t) =  t^4 + 1 = (t^2 + t - 1)(t^2 - t - 1)
  \]
  so \(t^8 - 1\) factorises as a product of linear and quadratic polynomials mod \(3\). So \(L = \F_9\), the unique field of order \(9\), whose multiplicative group is isomorphic to \(C_8\). As \(|\gal(L/\F_3)| = 2\), it is isomorphic to \(C_2\). But
  \[
    (\Z/8\Z)^\times = \{1, 3, 5, 7\} \cong C_2 \times C_2
  \]
  so \(\theta\) is not surjective and \(\Phi_8(t)\) is reducible.
\end{eg}

So far we have worked exclusively with finite fields with non-zero characteristic. The following theorem is a result regarding \(\Q\):

\begin{theorem}
  \label{thm:cyclotomic extension of Q}
  For all \(m > 0\), \(\Phi_m(t)\) is irreducible in \(\Z[t]\) and hence in \(\Q[t]\). Thus \(\theta\) is an isomorphism and
  \[
    \gal(\Q(\zeta)/\Q) \cong (\Z/m\Z)^\times
  \]
  where \(\zeta\) is a primitive \(m\)th root of unity.
\end{theorem}

\begin{remark}
  We already knew this when \(m = p\) for \(p\) prime by substitution and Eisenstein.
\end{remark}

\begin{proof}
  Gauss' Lemma implies that irreducibility in \(\Z[t]\) gives irreducibility in \(\Q[t]\). \Cref{lem:irreducibility of cyclotomic} says that irreducibility corresponds to surjectivity of \(\theta\). Thus it is left to show that \(\Phi_m(t)\) is irreducible in \(\Z[t]\).

  Suppose not and \(\Phi_m(t) = g(t)h(t)\) in \(\Z[t]\), with \(g(t)\) irreducible and monic with \(\deg g(t) < \deg \Phi_m(t)\). Let \(\Q \leq L\) be the \(m\)th cyclotomic extension and \(\zeta\) be a primitive \(m\)th root and unity and also a root of \(g(t)\). Claim that if \(p \ndivides m\) and \(p\) is prime then \(\zeta^p\) is also a root of \(g(t)\) in \(L\):

  \begin{proof}
    Suppose not, then \(\zeta^p\) is also a primitive \(m\)th root of unity (since \(p \ndivides m\)) and a root of \(\Phi_m(t)\). The supposition implies that \(\zeta^p\) is a root of \(h(t)\). Define
    \[
      r(t) = h(t^p),
    \]
    then \(r(\zeta) = 0\). But \(g(t)\) is the minimal polynomial of \(\zeta\) over \(\Q\) and so \(g(t) \divides r(t)\) in \(\Q[t]\). By Gauss' Lemma \(r(t) = g(t)s(t)\) with \(s(t) \in \Z[t]\). Now reduce mod \(p\),
    \[
      \red r(t) = \red g(t) \red s(t).
    \]
    But \(\red r(t) - \red h(t^p) = (\red h(t))^p\). If \(\red a(t)\) is any irreducible factor of \(\red g(t)\) in \(\F_p[t]\) then \(\red a(t) \divides (\red h(t))^p\) and so \(\red a(t) \divides \red h(t)\). But then \((\red a(t))^2 \divides \red g(t) \red h(T) = \red \Phi_m(t)\). So \(\red \Phi_m(t)\) has a repeated root and thus \(t^m - 1\) has repeated roots mod \(p\). Absurd.
  \end{proof}
  
  So the claim is true. Now consider a root \(\gamma\) of \(h(t)\). It is also a primitive \(m\)th root of unity and \(\gamma = \zeta^i\) for some \(i\) coprime with \(m\). Factorise it as \(i = p_1 \cdots p_k\) with \(p_j\) prime and not necessarily distinct and \(p_j \ndivides m\). Apply the claim repeatedly, we get that \(\gamma\) is also a root of \(g(t)\) and so \(\Phi_m(t)\) has a repeated root. Absurd. Thus \(\Phi_m(t)\) is irreducible over \(\Q\).
\end{proof}

\begin{definition}[Cyclic extension]\index{field extension!cyclic}
  An extension \(K \leq L\) is \emph{cyclic} if the extension is Galois and \(\gal(L/K)\) is cyclic.
\end{definition}

\begin{definition}[Abelian extension]\index{field extension!abelian}
  An extension \(K \leq L\) is \emph{abelian} if the extension is Galois and \(\gal(L/K)\) is abelian.
\end{definition}

\begin{eg}\leavevmode
  \begin{enumerate}
  \item We saw in \Cref{cor:galois group of finite fields} that for finite fields \(\F \leq L\) is cyclic.
  \item Cyclotomic extensions are abelian.
  \item \Cref{thm:cyclotomic extension of Q} says \(\gal(\Q(\zeta)/\Q) \cong  (\Z/m\Z)^\times\) and so if \(m = 8\), \(\Q \leq \Q(\zeta)\) is a non-cyclic abelian extension.
  \end{enumerate}
\end{eg}

\subsection{Kummer Theory}

Rather than consider the \(m\)th root of unity, in this section we consider Galois extensions \(K \leq L\) where \(L\) is the splitting field of a polynomial of the form \(t^m - \lambda\) where \(\lambda \in K\).

\begin{theorem}
  \label{thm:splitting field of tm - lambda}
  Let \(f(t) = t^m - \lambda \in K[t]\) and \(\ch K \ndivides m\). Then the splitting field \(L\) of \(f(t)\) over \(K\) contains a primitive \(m\)th root of unity \(\zeta\) and \(\gal(L/K(\zeta))\) is cyclic of order dividing \(m\). Moreover \(f(t)\) is irreducible over \(K(\zeta)\) if and only if \(|L:K(\zeta)| = m\).
\end{theorem}

\begin{remark}
  We have the following Galois correspondence
  \[
    \left.
    \begin{array}[h]{c}
      \text{cyclic }\left\{
      \begin{array}[h]{c}
        L \\
        | \\
        K(\zeta)
      \end{array}
      \right. \\
      \hspace{37pt} | \\
      \hspace{37pt} K
    \end{array}
  \right\} \text{abelian}
  \qquad
    \begin{array}[h]{c}
      1 \\
      | \\
      \gal(L/K(\zeta)) \\
      | \\
      \gal(L/K)
    \end{array}
  \]
  with \(\gal(L/K(\zeta)) \normal \gal(L/K)\). By \nameref{thm:fundamental}
  \[
    \gal(L/K)/\gal(L/K(\zeta)) \cong \gal(K(\zeta)/K)
  \]
  which is abelian.
\end{remark}

\begin{proof}
  Since \(t^m - \lambda\) and \(mt^{m - 1}\) are coprime, we know that \(t^m - \lambda\) has distinct roots, say \(\alpha_1, \dots \alpha_m\) in the splitting field \(L\). Thus \(K \leq L\) is Galois. Since \((\alpha_i\alpha_j^{-1})^m = \lambda \lambda^{-1} = 1\), the elements
  \[
    \alpha_1\alpha_1^{-1} = 1, \alpha_2\alpha_1^{-1}, \dots, \alpha_m\alpha_1^{-1}
  \]
  are \(m\) distinct \(m\)th roots of unity in \(L\) and so
  \[
    t^m - \lambda = (t - \beta)(t - \zeta\beta)(t - \zeta^2\beta) \cdots (t - \zeta^{n - 1}\beta) \in L[t]
  \]
  where \(\beta = \alpha_1\) and \(\zeta\) is a primitive \(m\)th root of unity. Thus
  \[
    L = K(\zeta, \beta).
  \]

  Let \(\sigma \in \gal(L/K(\zeta))\). It is determined by its action on \(\beta\). Note that \(\sigma(\beta)\) is also a root of \(t^m - \lambda\) so \(\sigma(\beta) = \zeta^{j(\sigma)}\beta\) where \(0 \leq j(\sigma) < m\). Also if \(\sigma, \tau \in \gal(L/K(\zeta))\) then
  \[
    \tau\sigma(\beta) = \tau(\zeta^{j(\sigma)}\beta) = \zeta^{j(\sigma)}\tau(\beta) = \zeta^{j(\sigma)}\zeta^{j(\tau)}\beta
  \]
  where the second inequality comes from the fact that \(\zeta\) is fixed by \(\tau\). Thus there is a group homomorphism
  \begin{align*}
    \Theta: \gal(L/K(\zeta)) &\to \Z/m\Z \\
    \sigma &\mapsto j(\sigma)
  \end{align*}
  Note that \(j(\sigma) = 0\) only if \(\sigma\) is the identity and so \(\Theta\) is injective. Thus \(\gal(L/K(\zeta))\) is isomorphic to a subgroup of \(\Z/m\Z\), so a cyclic group of order dividing \(m\).

  Finally,
  \[
    |L:K(\zeta)| = |\gal(L/K(\zeta))| \leq m
  \]
  with equality precisely when the action of \(\gal(L/K(\zeta))\) is transitive on the roots and that is when \(t^m - \lambda\) is irreducible over \(K(\zeta)\) by \Cref{lem:galois orbits}.
\end{proof}

\begin{eg}\leavevmode
  \begin{enumerate}
  \item \(f(t) = t^6 + 3\) over \(\Q\). Let \(\zeta = - \omega\) be a primitive \(6\)th root of unity where \(\omega\) is a primitive cubic root of unity. Then
  \[
    \Q(\zeta) = \Q(\omega) = \Q(\frac{1}{2}(1 + \sqrt{-3})) = \Q(\sqrt{-3}).
  \]
  \(f(t)\) is irreducible over \(\Q\) by Eisenstein with \(3\). However over \(\Q(\zeta) = \Q(\sqrt{-3})\) \(f(t)\) factorises as
  \[
    f(t) = (t^3 + \sqrt{-3}) (t^3 - \sqrt{-3})
  \]
  so the Galois group of the splitting field \(L\) of \(f(t)\) over \(\Q(\zeta)\) is a proper subgroup of \(\Z/6\Z\). We have the following Galois correspondence:
  \[
    \begin{array}[h]{ccc}
      L & 1 & \\
      | & | & 3 \\
      \Q(\zeta) & \gal(L/\Q(\zeta)) & \\
      | & | & 2 \\
      \Q & \gal(L/\Q) &
    \end{array}
  \]
   Note that
   \begin{align*}
     \gal(L/\Q(\zeta)) &\cong \Z/3\Z \embed \Z/6\Z \\
     \gal(\Q(\zeta)/\Q) &= \generation{i \mapsto -i} \cong \Z/2\Z
   \end{align*}

   Let \(\beta\) be a root of \(f(t)\). Then the roots are
   \[
     \begin{array}[h]{cccccc}
       \beta & \zeta \beta & \zeta^2 \beta & \zeta^3 \beta & \zeta^4 \beta & \zeta^5 \beta \\ \hline \hline
       \beta & & \omega^2 \beta & & \omega\beta & \\ \hline
       & - \omega \beta & & - \beta & & - \omega^2 \beta
     \end{array}
   \]
   where a \(3\)-cycle is given by permuting \(\beta, \omega^2 \beta, \omega \beta\) and the orbits are shown as the last two rows above. Denote the \(3\)-cycle \(\sigma\) and complex conjugation \(\tau\), since
   \[
     \tau \sigma \tau^{-1} = \sigma^{-1},
   \]
   we get dihedral relation so \(\gal(L/\Q)\) is dihedral of order \(6\).
 \item \(f(t) = t^5 - 2\) over \(\Q\). It is irreducible by Eisenstein. Let \(L\) be the splitting field of \(f(t)\) over \(\Q\) and \(\zeta\) be a primitive \(5\)th root of unity. We have the following Galois correspondence:
   \[
     \begin{array}[h]{ccc}
       L & 1 & \\
       | & | & 5 \\
       \Q(\zeta) & \gal(L/\Q(\zeta)) & \\
       | & | & 4 \\
       \Q & \gal(L/\Q)
     \end{array}
   \]
   Let \(\beta\) be a root of \(f(t)\). Then \(\Q \leq \Q(\beta) \leq L\) so \(5 \divides |L:\Q|\) and thus \(5 \divides |\gal(L/\Q(\zeta))|\). Since we also have \(\gal(L/\Q(\zeta)) \embed \Z/5\Z\), it follows that \(\gal(L/\Q(\zeta)) \cong \Z/5\Z\).

   We can thus deduce that \(f(t)\) remains irreducible over \(\Q(\zeta)\) and \(|\gal(L/\Q)| = 20\). By irreducibility of \(f(t)\) over \(\Q\), \(\gal(L/\Q)\) is a transitive subgroup of \(S_5\). By \Cref{lem:transitive subgroups of Sn}, it is isomorphic to \(H_{20}\), the subgroup generated by a \(5\)-cycle and a \(4\)-cycle. By \nameref{thm:fundamental},
   \[
     \gal(L/\Q)/\gal(L/\Q(\zeta)) \cong \gal(\Q(\zeta)/\Q) \cong (\Z/5\Z)^\times
   \]
   which is cyclic and contains a \(4\)-cycle. Thus without a priori knowledge of \(H_{20}\) we know our subgroup of \(S_5\) contains a \(4\)-cycle.
 \end{enumerate}
\end{eg}

\Cref{thm:splitting field of tm - lambda} tells us the property of the splitting field of \(t^m - \lambda\). The following theorem gives its converse:

\begin{theorem}
  \label{thm:kummer theory}
  Suppose \(K \leq L\) is a cyclic extension with \(|L:K| = m\) where \(\ch K \ndivides m\) and \(K\) contains a primitive \(m\)th root of unity. Then there exists \(\lambda \in K\) such that \(t^m - \lambda\) is irreducible over \(K\), and \(L\) is the splitting field of \(t^m - \lambda\) over \(K\). If \(\beta\) is a root of \(t^m - \lambda\) in \(L\) then \(L = K(\beta)\).
\end{theorem}

Since we are going to restate the hypothesis in the theorem many times, we give it a name

\begin{definition}[Kummer extension]\index{field extension!Kummer}
  A cyclic extension \(K \leq L\) with \(|L:K| = m\) where \(\ch K \ndivides m\) and \(K\) contains a primitve \(m\)th root of unity is a \emph{Kummer extension}.
\end{definition}

We need the following to prove the theorem:

\begin{lemma}[Linear independence of group characters]
  \label{lem:linear independence of group characters}
  Let \(\phi_1, \dots, \phi_n\) be embeddings of a field \(K\) into a field \(L\). Then there do not exist \(\lambda_1, \dots, \lambda_n\) not all zero such that
  \[
    \lambda_1 \phi_1(x) + \dots + \lambda_n \phi_n(x) = 0
  \]
  for all \(x \in K\).
\end{lemma}

\begin{proof}
  Example sheet 2 Q10.
\end{proof}

\begin{proof}[Proof of \Cref{thm:kummer theory}]
  \label{proof:kummer theory}
  Let \(\gal(G/K) = \generation \sigma\) which has order \(m\). Observe that \(\{\sigma^i\}_{i = 0}^{m - 1}\) are distinct maps \(L \to L\) and we can apply \nameref{lem:linear independence of group characters}: there exists \(\alpha \in L\) such that
  \[
    \beta = \alpha + \zeta \sigma(\alpha) + \dots + \zeta^{m - 1} \sigma^{m - 1}(\alpha) \neq 0
  \]
  where \(\zeta\) is a primitive \(m\)th root of unity.

  Observe that \(\sigma(\beta) = \zeta^{-1} \beta = \beta\) and so \(\beta \in K\), the fixed field of \(\gal(L/K)\). Also \(\sigma(\beta^m) = \sigma(\beta)^m = \beta^m\). Let \(\lambda = \beta^m \in K\). Then
  \[
    t^m - \lambda = (t - \beta)(t - \zeta\beta)\cdots(t - \zeta^{m - 1}\beta) \in L[t]
  \]
  so \(K(\beta)\) is the splitting field of \(t^m - \lambda\) over \(K\) (recall \(\zeta \in K\)).

  Observe also that \(\{\sigma^i\}_{i = 0}^{m - 1}\) are distinct \(K\)-automorphisms and so \(|K(\beta):K| \geq m\). Therefore
  \[
    L = K(\beta) = K(\zeta\beta).
  \]
\end{proof}

\(t^m - \lambda\) is the minimal polynomial of \(\beta\) over \(K\) and hence is irreducible.

\begin{definition}[Radical extension]\index{field extension!radical}
  A field extension \(K \leq L\) is an \emph{extension by racidals} if there exists
  \[
    K = L_0 \leq L_ 1 \leq \cdots \leq L_n = L
  \]
  such that each \(L_i \leq L_{i + 1}\) is either cyclotomic or Kummer extension.
\end{definition}

\begin{definition}[Solubility by radicals]\index{solubility}
  A polynomial \(f(t) \in K[t]\) is \emph{soluble by radicals} if its splitting field lies in an extension by radicals.
\end{definition}

\subsection{Cubics}

In this section and the following one we assume \(\ch K \neq 2\) for discussion about discriminant and \(\ch K \neq 3\) for cubic Kummer extension.

We have already seen that if \(f(t)\) is a monic irreducible cubic in \(K[t]\) with \(L\) its splitting field over \(K\),
\[
  G = \gal(f) = \gal(L/K)
\]
is either \(A_3\) or \(S_3\) since irreducibility implies that the action on roots is transitive. Thus we have the following correspondence:

\[
  \begin{array}[h]{ccc}
    L & 1 & \\
    | & | & 3 \\
    K(\Delta) & G \cap A_3 \\
    | & | & 1 \text{ or } 2 \\
    K & G &
  \end{array}
\]
where \(\Delta^2 = D(f)\) is the discriminant of \(f\). But to see if we can solve \(f\) by radicals we want to make use of \Cref{thm:kummer theory} and so we need to adjoin the appropriate roots of unity. So we get a bigger picture:
\[
  \begin{tikzcd}
    & L(\omega) \ar[dl, dash] \ar[dr, dash, "1 \text{ or } 2"] & \\
    K(\Delta, \omega) \ar[dr, dash, "1 \text{ or } 2"] & & L \ar[dl, dash, "3"] \\
    & K(\Delta) \ar[d, dash, "1 \text{ or } 2"] & \\
    & K &
  \end{tikzcd}
\]
where \(\omega\) is a primitive cubic root of unity. From the Tower Law \(|L(\omega):K(\Delta, \omega)| = 3\). Hence
\[
  \gal(L(\omega)/K(\Delta, \omega)) \cong C_3.
\]
We can now apply \Cref{thm:kummer theory} to see that
\[
  L(\omega) = K(\Delta, \omega)(\beta)
\]
where \(\beta\) is a root of an irreducible polynomial \(t^3 - \lambda \in K(\Delta, \omega)[t]\).

In fact from the \hyperref[proof:kummer theory]{proof} of \Cref{thm:kummer theory} we see that
\[
  \beta = \alpha_1 + \omega \alpha_2 + \omega^2 \alpha_3
\]
where \(\alpha_1, \alpha_2, \alpha_3\) are roots of \(f(t)\). Now all the extensions
\[
  K \leq K(\Delta) \leq K(\Delta, \omega) \leq L(\omega)
\]
are either cyclotomic or Kummer. Thus \(f(t)\) is soluble by radicals.

Let's give a try to our theory. Given an irreducible cubic
\[
  f(t) = t^3 + at^2 + bt + c = (t - \alpha_1)(t - \alpha_2)(t - \alpha_3),
\]
we have \(\alpha_1 + \alpha_2 + \alpha_3 = -a\). But we don't actually need that many parameters: let \(\alpha' = \alpha_i + \frac{a}{3}\) so that \(\alpha_1' + \alpha_2' + \alpha_3' = 0\) and the \(\alpha'\)'s are roots of the polynomial \(g(t) = t^3 + pt + q\) and most importantly,
\[
  K(\alpha_1, \alpha_2, \alpha_3) = K(\alpha_1', \alpha_2', \alpha_3')
\]
so the splitting field for \(g(t)\) is the same as that for \(f(t)\). Thus we could work with \(g(t)\) instead. The discriminant of \(g(t)\) is \(D(g) = -4p^3 - 27q^2\).

Set
\begin{align*}
  \beta &= \alpha_1' + \omega \alpha_2' + \omega^2 \alpha_3' \\
  \gamma &= \alpha_1' + \omega^2 \alpha_2' + \omega \alpha_3'
\end{align*}
then
\begin{align*}
  \beta \gamma &= \alpha_1'^2 + \alpha_2'^2 + \alpha_3'^3 + (\omega + \omega^2)(\alpha_1'\alpha_2' + \alpha_1'\alpha_3' + \alpha_2'\alpha_3') \\
               &= (\alpha_1' + \alpha_2' + \alpha_3')^2 - 3(\alpha_1'\alpha_2' + \alpha_1'\alpha_3' + \alpha_2'\alpha_3') \\
               &= -3p
\end{align*}
and so \(\beta^3\gamma^3 = -27p^3\). Also,
\begin{align*}
  \beta^3 + \gamma^3 &= (\alpha_1' + \omega\alpha_2' + \omega^2\alpha_3')^3 \\
                     &+ (\alpha_1' + \omega^2\alpha_2' + \omega\alpha_3')^3 \\
                     &+ \underbrace{(\alpha_1' + \alpha_2' + \alpha_3')^3}_{= 0} \\
                     &= 3(\alpha_1'^3 + \alpha_2'^3 + \alpha_3'^3) + 18\alpha_1'\alpha_2'\alpha_3' \\
                     &= -27q
\end{align*}
since \(\alpha_1'^3 = -p\alpha_1' - q\) and so \(\alpha_1'^3 + \alpha_2'^3 + \alpha_3'^3 = -3q\).

Thus after some messy algebra, we find that \(\beta^3\) and \(\gamma^3\) are roots of a quadratic
\[
  t^2 + 27qt - 27p^3
\]
and so are
\[
  -\frac{27}{2}q \pm \frac{3\sqrt{-3}}{2}\sqrt{-27q^2 - 4p^3} = -\frac{27}{2}q \pm \frac{3\sqrt{-3}}{2}\sqrt D.
\]
We can solve for \(\beta^3\) and \(\gamma^3\) in \(K(\sqrt{-3}, \sqrt D) = K(\omega, \Delta)\). From here we can get \(\beta\) by adjoining a cubic root of \(\beta^3\) and set \(\gamma = -\frac{3p}{\beta}\). Finally we solve in \(L(\omega)\) for \(\alpha_1', \alpha_2', \alpha_3'\)
\begin{align*}
  \alpha_1' &= \frac{1}{3}(\beta + \gamma) \\
  \alpha_2' &= \frac{1}{3}(\omega^2\beta + \omega\gamma) \\
  \alpha_3' &= \frac{1}{3}(\omega\beta + \omega^2\gamma)
\end{align*}

\subsection{Quartics}

As with the cubics, by making a substitution of the form \(\alpha_i' = \alpha_i + \frac{a}{4}\) we may assume that the sum of the roots is zero and so the \(t^3\) term vanishes, leaving a general quartic polynomial of the form
\begin{align*}
  f(t) &= t^4 + bt^2 + ct + d \\
       &= (t - \alpha_1)(t - \alpha_2)(t - \alpha_3)(t - \alpha_4)
\end{align*}
which we assume to be monic and irreducible.

Let \(L = K(\alpha_1, \dots, \alpha_4)\) be the splitting field of \(f(t)\) over \(K\).
\[
  \begin{array}[h]{ccc}
    L & 1 & \\
    | & | & \\
    M & G \cap V_4 \normal G & \\
    | & | & \\
    K(\Delta) & G \cap A_4 & \\
    | & | & \\
    K & G &
  \end{array}
\]
where \(M\) is the fixed field of the normal subgroup \(G \cap V_4 \normal G\), which is an normal extension of \(K\). By \nameref{thm:fundamental},
\[
  \gal(M/K) \cong G/(G \cap V_4).
\]
To determine this group concretely, we use a little knowledge from group theory: there is a group isomorphism
\[
  S_4/V_4 \cong S_3.
\]
Let \(\theta: S_4 \to S_3\) denote the quotient map, which has kernel precisely \(V_4\). Therefore \(\theta|_G: G \to S_3\) induces an isomorphism
\[
  G/\ker \theta|_G = G/(G \cap V_4) \cong \im \theta|_G \leq S_3.
\]
We therefore seek a cubic for which \(M\) is the splitting field. We introduce \emph{resolvent cubic}: set
\begin{align*}
  x &= \alpha_1 + \alpha_2 \\
  y &= \alpha_1 + \alpha_3 \\
  z &= \alpha_1 + \alpha_4
\end{align*}
and so
\begin{align*}
  \alpha_1 &= \frac{1}{2}(x + y + z) \\
  \alpha_2 &= \frac{1}{2}(x - y - z) \\
  \alpha_3 &= \frac{1}{2}(-x + y - z) \\
  \alpha_4 &= \frac{1}{2}(-x - y + z) \\
\end{align*}
Thus
\[
  K(\alpha_1, \dots, \alpha_4) = K(x, y, z).
\]
Remebering that \(\alpha_1 + \alpha_2 + \alpha_3 + \alpha_4 = 0\),
\begin{align*}
  x^2 &= (\alpha_1 + \alpha_2)^2 = -(\alpha_1 + \alpha_2)(\alpha_3 + \alpha_4) \\
  y^2 &= (\alpha_1 + \alpha_3)^2 = -(\alpha_1 + \alpha_3)(\alpha_2 + \alpha_4) \\
  z^2 &= (\alpha_1 + \alpha_4)^2 = -(\alpha_1 + \alpha_4)(\alpha_3 + \alpha_3)
\end{align*}
These are all distinct since for example if \(y^2 = z^2\) then \(y = \pm z\) so \(\alpha_3 = \alpha_4\) or \(\alpha_1 = \alpha_2\).

Now consider the \emph{resolvent cubic}
\[
  g(t) = (t - x^2) (t - y^2)(t - z^2) \in K[t].
\]

\(x^2, y^2\) and \(z^2\) are permuted by \(G\) and are fixed by \(G \cap V_4\). Thus
\[
  K(x^2, y^2, z^2) \leq M = L^{G \cap V_4}.
\]
We claim that we actually have equality here:

\begin{proof}
 Check \(D(f) = D(g)\) --- this will be addressed on example sheet --- so \(K(\Delta) \leq K(x^2, y^2, z^2)\).

Now observe that
\[
  \gal(L/(K(x^2, y^2, z^2)) = \gal(K(x, y, z)/K(x^2, y^2, z^2))
\]
and in
\[
  K(x^2, y^2, z^2) \leq K(x, y^2, z^2) \leq K(x, y, z^2) \leq K(x, y, z),
\]
each extension is of degree either \(1\) or \(2\) so \(|K(x, y, z):K(x^2, y^2, z^2)|\) divides \(8\) so elements of \(\gal(L/K(x^2, y^2, z^2))\) have order dividing \(8\). But since
\[
  \gal(L/K(x^2, y^2, z^2)) \leq G \cap A_4,
\]
by checking the order of elements in \(A_4\) we see that
\[
  \gal(L/K(x^2, y^2, z^2)) \leq G \cap V_4.
\]
Therefore by \nameref{thm:fundamental}
\[
  M = K(x^2, y^2, z^2).
\]
\end{proof}

Consider the coefficients of \(g(t)\): its roots \(x^2, y^2, z^2\) are permuted by \(G\) and so it has coefficients in \(K\). We can actually write down these coefficients:
\begin{align*}
  x^2 + y^2 + z^2 &= -2b \\
  x^2y^2 + x^2z^2 + y^2z^2 &= b^2 - 4d \\
  xyz &= -c \\
  x^2y^2z^2 &= c^2
\end{align*}
so
\[
  g(t) = t^3 + 2bt^2 + (b^2 - 4d)t - c^2.
\]
We know how to solve cubics so we can solve for \(x^2, y^2, z^2\), from which we can solve for \(x, y, z\). Finally we use formula \(\alpha_1 = \frac{1}{2}(x + y + z)\) etc to recover the roots for the quartic. Whew, done!

\begin{remark}
  In our map
  \[
    \theta: S_4 \to S_3,
  \]
  we have restriction
  \[
    \theta|_{A_4}: A_4 \to A_3 \cong C_3
  \]
  and
  \[
    \theta|_{G \cap A_4}: G \cap A_4 \to A_3
  \]
  so
  \[
    G \cap A_4 /(G \cap V_4) \leq A_3
  \]
  which is either \(1\) or \(A_3\), corresponding to \(M/K(\Delta)\). If the resolvent cubic is irreducible then it is isomorphic to \(A_3\). Otherwise it is the trivial group.
\end{remark}

\begin{eg}
  Let \(f(t) = t^4 + 4t^2 + 2\). As an aside even if we didn't study Galois theory we could solve it. Then the resolvent cubic \(g(t) = t^3 + 8t^2 + 8t\). Note that \(c = 0\) in \(f(t)\) so \(g(t)\) has no constant term and thus reducible. It follows that \(M = K(\Delta)\).
\end{eg}

\subsection{Solubility by Radicals}

Now suppose we have a Galois extension \(K \leq L\) with
\[
  K = L_0 \leq L_1 \leq \cdots \leq L_m = L
\]
such that each \(L_i \leq L_{i + 1}\) is either cyclotomic or Kummer extension.

Let \(G = \gal(G/K)\). There is a corresponding chain of subgroups of \(G\)
\[
  G = G_0 \geq G_1 \geq \cdots \geq G_m = 1
\]
with \(G_i = \gal(L/K_i)\) and, by \nameref{thm:fundamental}, \(L_i = L^{G_i}\). However each extension \(L_i \leq L_{i + 1}\) is Galois and we know
\[
  G_{i + 1} = \gal(L/L_{i + 1}) \normal \gal(L/L_i) = G_i
\]
and
\[
  G_i/G_{i + 1} \cong \gal(L_{i + 1}/L_i)
\]
which is abelian if \(L_i \leq L_{i + 1}\) is cyclotomic and cyclic if \(L_i \leq L_{i + 1}\) is Kummer. It is the perfect time to introduce some group theory:

\begin{definition}[Soluble group]\index{soluble group}
  A group is \emph{soluble} if there is a chain of subgroups
  \begin{equation*}
    \label{eqn:abelian soluble}
    1 \normal G_m \normal G_{m - 1} \normal \cdots \normal G_1 \normal G_0 = G
    \tag{\(\ast\)}
  \end{equation*}
  with \(G_i/G_{i + 1}\) abelian.
\end{definition}

\begin{remark}
  Note that some authors use ``cyclic'' in the definition. While we will prove shortly that they are equivalent for a finite group \(G\), in general they are different.
\end{remark}

\begin{eg}\leavevmode
  \begin{enumerate}
  \item \(S_3\) is soluble since
    \[
      1 \normal \generation{(123)} \normal S_3.
    \]
  \item \(S_4\) is soluble since
    \[
      1 \normal V_4 \normal A_4 \normal S_4.
    \]
    and \(A_4/V_4 \cong C_3\), \(S_4/A_4 \cong C_2\).
  \item \(A_5\) is not soluble: we have proved in IA Groups and again in IB Groups, Rings and Modules that \(A_5\) is simple, and therefore any normal subgroup is \(1\) or \(A_5\). Then any chain of normal subgroups would have non-abelian quotients and thus \(A_5\) is not soluble.
  \end{enumerate}
\end{eg}

\begin{lemma}
  A finite group \(G\) is soluble if and only if
  \begin{equation*}
    \label{eqn:cyclic soluble}
    1 = G_m \normal G_{m - 1} \normal \cdots \normal G_1 \normal G_0 = G
    \tag{\(\dag\)}
  \end{equation*}
  with \(G_i/G_{i + 1}\) cyclic.
\end{lemma}
This says that we only have to consider cyclic extensions.

\begin{proof}
  The if part is easy. For the only if part, from Structural Theorem of Finitely Generated Abelian Groups, if \(A\) is abelian then there is a chain
  \[
    0 = A_r \normal A_{r - 1} \normal \cdots \normal A_0 = A
  \]
  with \(A_r/A_{r + 1}\) cyclic. Thus if we have a chain \eqref{eqn:abelian soluble} with abelian factors \(G_i/G_{i + 1}\), we can refine it to one of the form \eqref{eqn:cyclic soluble} by Third Isomorphism Theorem.
\end{proof}

\begin{definition}[Derived subgroup]
  The \emph{derived subgroup} \(G'\) of a group \(G\) is the subgroup generated by all the commutators
  \[
    g_1g_2g_1^{-1}g_2^{-1}
  \]
  for \(g_1, g_2 \in G\).
\end{definition}

Note that it is the subgroup \emph{generated} by all such elements since it is not obvious that they are closed.

\begin{lemma}
  Let \(K \normal G\). Then \(G/K\) is abelian if and only if \(G' \leq K\).
\end{lemma}

\begin{proof}
  \(G/K\) is abelian if and only if
  \[
    Kg_1Kg_2Kg_1^{-1}Kg_2^{-1} = K
  \]
  for all \(g_1, g_2 \in G\), if and only if
  \[
    g_1g_2g_1^{-1}g_2^{-1} \in K
  \]
  if and only if \(G' \leq K\).
\end{proof}

\begin{remark}
  Some interesting asides:
  \begin{enumerate}
  \item In IID Representation Theory we will prove Burnside's Theorem: if \(|G| = p^aq^b\) with \(p, q\) distinct primes then \(G\) is soluble.
  \item Feit-Thompson Theorem: if \(|G|\) is odd then \(G\) is soluble.
  \item There is an analogue of Sylow's Theorem due to Philip Hall: given a finite group \(G\), for every \(m\) and \(n\) coprime and \(|G| = mn\), there is a subgroup of order \(m\) if and only if \(G\) is soluble.
  \end{enumerate}
\end{remark}

\begin{definition}[Derived series]\index{derived series}
  The \emph{derived series} \(\{G^{(m)}\}\) of \(G\) is defined inductively as
  \begin{align*}
    G^{(0)} &= G \\
    G^{(i + 1)} &= (G^{(i)})'
  \end{align*}
  Thus
  \[
    \cdots \normal G^{(2)} \normal G^{(1)} \normal G^{(0)} = G
  \]
  with \(G^{(i)}/G^{(i + 1)}\) abelian.
\end{definition}

\begin{lemma}
  Given a finite group \(G\), \(G\) is soluble if and only if \(G^{(m)} = 1\) for some \(m\).
\end{lemma}

\begin{proof}
  If \(G^{(m)} = 1\) then the derived series gives a chain of the form \eqref{eqn:abelian soluble} as in the definition of solubility.

  Conversely, if there is a chain of the form \eqref{eqn:abelian soluble}
  \[
    1 \normal G_m \normal \cdots \normal G_1 \normal G
  \]
  with \(G_i/G_{i + 1}\) abelian then induction shows that \(G^{(j)} \leq G_j\) and so \(G^{(m)} = 1\).
\end{proof}

\begin{remark}
  The derived series is the fastest descending chain with abelian factors.
\end{remark}

\begin{lemma}\leavevmode
  \begin{enumerate}
  \item Let \(H \leq G\). If \(G\) is soluble then \(H\) is soluble.
  \item Let \(H \normal G\). Then \(G\) is soluble if and only if \(H\) and \(G/H\) are both soluble.
  \end{enumerate}
\end{lemma}

\begin{proof}\leavevmode
  \begin{enumerate}
  \item \(G\) is soluble so there is some \(m\) such that \(G^{(m)} = 1\). But \(H^{(m)} \leq G^{(m)}\) and so \(H\) is soluble.
  \item Let \(H \normal G\). Suppose \(G\) is soluble. Then by the previous line \(H\) is soluble. Let \(G^{(m)} = 1\), observed that
    \[
      (G/H)' = G'H/H \leq G/H
    \]
    and inductively
    \[
      (G/H)^{(j)} = G^{(j)}H/H \leq G/H.
    \]
    Thus \((G/H)^{(m)} = H/H = 1\) so \(G/H\) is soluble.

    Conversely, suppose both \(H\) and \(G/H\) are soluble, i.e.\ \(H^{(r)} = 1\) and \((G/H)^{(s)} = H/H\). But from above \((G/H)^{(s)} = G^{(s)}H/H\) so \(G^{(s)}H = H\) and thus \(G^{(s)} \leq H\). Therefore
    \[
      G^{(r + s)} \leq H^{(r)} = 1
    \]
    so \(G\) is soluble.
  \end{enumerate}
\end{proof}

\begin{eg}
  \(S_5\) is not soluble since its subgroup \(A_5\) is not soluble.
\end{eg}

\begin{theorem}
  \label{thm:solubility}
  Let \(K\) be a field with \(\ch K = 0\) and \(f(t) \in K[t]\). Then \(f(t)\) is soluble by radicals over \(K\) if and only if \(\gal(f)\) over \(K\) is soluble.
\end{theorem}

\begin{remark}
  We don't need to restrict to \(\ch K = 0\). What we need to do for a particular \(f(t)\) is to avoid a finite number of bad characteristics (to avoid characteristics smaller than \(\deg f(t)\)).
\end{remark}

\begin{corollary}
  If \(f(t)\) is a monic irreducible polynomial in \(K[t]\) with \(\ch K = 0\) and \(\gal(f) \cong A_5\) or \(S_5\) then \(f(t)\) is not soluble by radicals.
\end{corollary}

\begin{eg}
  In the example after \Cref{thm:galois group Sp} on page~\pageref{thm:galois group Sp}, \(f(t) = t^5 - 6t + 3 \in \Q[t]\) has Galois group \(S_5\) over \(\Q\): recall that it has three real roots and so complex conjugation gives a transposition. \(f(t)\) is irreducible and so \(5 \divides |\gal(f)|\) and so there is a \(5\)-cycle. Together they generate \(S_5\). Thus \(f(t)\) is not soluble by radicals.
\end{eg}

\begin{proof}[Proof of \Cref{thm:solubility}]
  Suppose \(f(t)\) is soluble by radicals. Then the splitting field of \(f(t)\) over \(K\), \(L\), lies in an extension of \(K\) by radicals:
  \[
    K = L_0 \leq L_1 \leq \dots \leq L_m
  \]
  with each \(L_i \leq L_{i + 1}\) either cyclotomic or Kummer.

  \begin{lemma}
    If \(K \leq N\) is an extension by radicals then there exists \(N \leq N'\) with \(K \leq N'\) an extension of radicals and a Galois extension.
  \end{lemma}

  \begin{proof}
    Suppose we have a sequence as above and want to extend this into a Galois extension of the same form (assume \(\ch K = 0\)). By \nameref{thm:primitive} \(L_m = K(\alpha_1)\) for some \(\alpha_1\). Let \(g(t)\) be the minimal polynomial of \(\alpha_1\) over \(K\) with splitting field \(M\). Thus \(M = K(\alpha_1, \alpha_2, \dots, \alpha_n)\) where \(\alpha_1, \dots, \alpha_n\) are roots of \(g(t)\). There are \(K\)-homomorphisms \(\phi_i: M \to M, \alpha_1 \mapsto \alpha_i\) extending \(K\)-homomorphisms \(K(\alpha_1) \to K(\alpha_i) \leq M\). The tower
    \[
      K \leq \phi_1(K) \leq \phi_2(L_1) \leq \dots \leq \phi_1(L_m) = K(\alpha_1)
    \]
    has cyclotomic or Kummer extensions at each step as before. Consider
    \[
      L_m = K(\alpha_1) \leq \phi_2(L_1)(\alpha_1) \leq \phi_2(L_2)(\alpha_1) \leq \dots \leq \phi_2(L_m)(\alpha_1) = K(\alpha_1, \alpha_2).
    \]
    Consider the extension \(\phi_2(L_j)(\alpha_1) \leq \phi_2(L_{j + 1})(\alpha_1)\) in this tower. There are two cases:
    \begin{itemize}
    \item if \(L_j \leq L_{j + 1}\) is cyclotomic then all the roots of unity adjoined are now in \(L_m = K(\alpha_1)\) and so \(\phi_2(L_j)(\alpha_i) = \phi_2(L_{j + 1})(\alpha_1)\).
    \item if \(L_j \leq L_{j + 1}\) is Kummer then we obtain \(L_{j + 1}\) by adjoining roots of an element of \(L_j\) and so we obtain \(\phi_2(L_{j + 1})\) by adjoining roots of an element in \(\phi_2(L_j)\). Hence we get from \(\phi_2(L_j)(\alpha_1)\) to \(\phi_2(L_{j + 1})(\alpha_1)\) by adjoining roots of an element of \(\phi_2(L_j)\). So this is a Kummer extension.\
    \end{itemize}

    Now continue to get a suitable chain \(K(\alpha_1,\alpha_2) \leq \dots \leq K(\alpha_1, \alpha_2, \alpha_3)\) etc. Thus we get a suitable chain from \(K\) to \(K(\alpha_1, \dots, \alpha_n) = M\). Observe that \(K \leq M\) is Galois.
  \end{proof}

  Assuming this lemma and so we may assume \(K \leq L_m\) is Galois. By \nameref{thm:fundamental} there is a corresponding chain of subgroups \(\gal(L_m/K)\). Previous discussion shows that \(\gal(L_m/K)\) is soluble. So to sum up we have \(K \leq L \leq L_m\) with \(L_m\) Galois so by \nameref{thm:fundamental}
  %
  %
  %
  % what about L_m/L? Why is it Galois?
  \[
    \gal(L/K) \cong \gal(L_m/K)/\gal(L_m/L)
  \]
  But quotients of soluble group are soluble so \(\gal(L/K)\) is soluble.

  Conversely, assume \(\ch K = 0\) and suppose \(G = \gal(f)\) over \(K\) is soluble. Let \(L\) be the splitting field of \(f(t)\) over \(K\) and so \(|G| = |L:K| = n\). Set \(m = n!\) and let \(\zeta\) be a primitive \(m\)th root of unity and consider \(L(\zeta)\).

  \[
    \begin{tikzcd}
      & L(\zeta) \ar[dl, dash, "\leq n"'] \ar[dr, dash] & \\
      K(\zeta) \ar[dr, dash] & & L \ar[dl, dash, "n"] \\
      & K &
    \end{tikzcd}
  \]

  Our proof is similar to that used for cubics. Observe that \(|L(\zeta):K(\zeta)| \leq n\): by \nameref{thm:primitive} \(L = K(\alpha)\) for some \(\alpha\) with minimal polynomial \(g(t)\) of degree \(n\). Then \(L(\zeta) = K(\zeta)(\alpha)\) and the minimal polynomial of \(\alpha\) over \(K(\zeta)\) divides \(g(t)\) and so has degree less than or equal to \(n\).

  Note that \(\gal(L(\zeta)/L)\) is abelian since the extension is cyclotomic. Then \(\gal(L(\zeta)/K)\) is soluble since the subgroup \(\gal(L(\zeta)/L\) is soluble and the quotient group \(\gal(L/K) \cong \gal(L(\zeta)/K)/\gal(L(\zeta)/L)\) is also soluble. Then the subgroup \(\gal(L(\zeta)/K(\zeta))\) is soluble. Thus there is a chain of subgroups
  \[
    1 = G_m \normal \cdots \normal G_1 \normal G_0 \normal \gal(L(\zeta)/K(\zeta))
  \]
  with \(G_i/G_{i + 1}\) cyclic (since for a finite group cyclic is equivalent to abelian in the definition of a soluble group). Now use \nameref{thm:fundamental} to get a corresponding chain of fields
  \[
    L(\zeta) = K_m \geq \dots \geq K_1\geq K(\zeta)
  \]
  with each \(K_i \leq K_{i + 1}\) Galois with cyclic Galois group.

  By \Cref{thm:kummer theory} all those extensions are Kummer (note all the extensions are of degree less than or equal to \(n\) and so we have the appropriate roots of unity, this is why we choose \(m = n!\)). Thus we have embedded \(L\) in an extension of \(K\) by radicals.
\end{proof}

\begin{eg}\leavevmode
  \begin{enumerate}
  \item Recall previously we had \(f(t) = t^4 + 4t^2 + 2\) which is irreducible by Eisenstein over \(\Q\). The resolvent cubic \(g(t) = t^3 + 8t^2 + 8t\) is reducible. It actually has roots \(0, -4 \pm 2\sqrt 2\). We thus have
    \[
      \begin{tikzcd}
        L = K \left(\sqrt{-4 + 2\sqrt 2}, \sqrt{-4 - 2\sqrt 2} \right) \ar[d, dash, "4"] \\
        K(-4 + 2\sqrt 2, -4 - 2\sqrt 2) = K(\sqrt 2) \ar[d, dash, "2"] \\
        K
      \end{tikzcd}
    \]
    so \(|L:K| = 8\). The Galois group is a transitive subgroup of \(S_4\) of order \(8\). It is the dihedral group.

    In particular the roots \(\pm \sqrt{-2 \pm \sqrt 2}\) are two conjugate pairs so complex conjugation gives double transposition.
  \item \(f(t) = t^4 + 2t + 2\) which is irreducible by Eisenstein over \(\Q\). A quick (hmm) calculation shows that the discriminant is \(101 \cdot 4^2\) which is not a square. Thus the Galois group contains an odd permutation.

    The resolvent cubic \(g(t) = t^3 - 8t - 4\) is irreducible mod \(5\). Therefore there is a \(3\)-cycle in the Galois group. We deduce that \(\gal(f) = S_4\).
  \item Finally a quintic: \(f(t) = t^5 - t - 1\) is irreducible over \(\Q\) since it is irreducible mod \(5\) and so the Galois group contains a \(5\)-cycle and is a transitive subgroup of \(S_5\). Reduction mod \(2\), \(f(t)\) factorises as a product of irreducible cubic and an irreducible quadractic
    \[
      \red f(t) = (t^3 + t^2 + 1)(t^2 + t + 1)
    \]
    so \(\gal(\red f)\) is generated
    % why generated
    by an element of cycle type \((3, 2)\). Thus \(\gal(f)\) also contains an element of cycle type \((3, 2)\). Then \(g^3\) is a transposition. A subgroup of \(S_5\) containing a \(5\)-cycle and a transposition must be \(S_5\). Therefore \(f(t)\) is not solvable by radicals.
  \end{enumerate}
\end{eg}

\section{Final Thoughts}

\subsection{Algebraic Closure}

\begin{definition}[Algebracially closed]\index{algebraically closed}
  A field \(L\) is \emph{algebraically closed} if any \(f(t) \in L[t]\) splits into a product of linear factors in \(L[t]\).
\end{definition}

\begin{remark}
  This is equivalent to saying that any \(f(t) \in L[t]\) has a root in \(L\), or that any algebraic extension of \(L\) is \(L\) itself.
\end{remark}

\begin{definition}[Algebraic closure]\index{algebraic closure}
  An extension \(K \leq L\) is an \emph{algebraic closure} of \(K\) if \(K \leq L\) is algebraic and \(L\) is algebraically closed.
\end{definition}

\begin{lemma}
  \label{lem:characterisation of algebraic closure}
  If \(K \leq L\) is algebraic and every polynomial in \(K[t]\) splits completely over \(L\) then \(L\) is an algebraic closure of \(K\).
\end{lemma}

\begin{proof}
  We need to show that \(L\) is algebraically closed. Suppose \(L \leq L(\alpha)\) is a finite extension and \(f_\alpha(t) = t^n + a_{n - 1}t^{n - 1} + \dots + a_0\) is the minimal polynomial of \(\alpha\) over \(L\). Let \(M = K(a_0, \dots, a_{n - 1})\). Then \(M \leq M(\alpha)\) is a finite extension. But each \(a_i\) is algebraic over \(K\) and so \(|M:K| < \infty\). Hence \(|M(\alpha):K| < \infty\) by Tower Law and so \(\alpha\) is algebraic over \(K\).

  The minimal polynomial of \(\alpha\) over \(K\) must split over \(L\) and so \(\alpha \in L\). Thus any algebraic extension of \(L\) is \(L\) itself.
\end{proof}

\begin{eg}[Algebraic number]
  Let \(\mathcal A = \{\alpha \in \C: \alpha \text{ algebraic over } \Q\}\). It is a subfield of \(\C\): if \(\alpha\) and \(\beta\) are algebraic over \(\Q\) then \(|\Q(\alpha, \beta): \Q| < \infty\). So if \(\gamma = \alpha + \beta, \alpha - \beta, \alpha\beta\) or \(\alpha/\beta\) for \(\beta \neq 0\) we get \(\Q(\gamma) \leq \Q(\alpha, \beta)\) and so \(|\Q(\gamma):\Q| < \infty\) so \(\gamma\) is algebraic over \(\Q\) and so \(\gamma \in \mathcal A\). Therefore \(\mathcal A = \cl \Q\) is an algebraic closure of \(\Q\).
\end{eg}

Like construction of many other objects in this course, we want to prove the existence and uniqueness of algebraic closures. In general we need to appeal to Zorn's Lemma. This will be covered in IID Logic and Set Theory, and it fact it is equivalent to both Axiom of Choice and Well-ordering Principle.

We will do a quick ad hoc statement of Zorn's Lemma here and relegate the main discussion to IID Logic and Set Theory.

\begin{definition}[Partial order]
  \((\zeta, \leq)\) is a \emph{partial order} on a set \(\zeta\) is
  \begin{enumerate}
  \item reflexive: \(x \leq x\) for all \(x \in \zeta\).
  \item transitive: \(x \leq y\) and \(y \leq z\) implies \(x \leq z\).
  \item antisymmetric: if \(x \leq y\) and \(y \leq x\) then \(x = y\).
  \end{enumerate}
\end{definition}

\begin{definition}[Total order]
  A partial order \((\zeta, \leq)\) is a \emph{total order} if it is total: for any \(x, y \in \zeta\), either \(x \leq y\) or \(y \leq x\).
\end{definition}

\begin{definition}[Chain]
  A \emph{chain} in a partially ordered set \((\zeta, \leq)\) is a totally ordered subset.
\end{definition}

\begin{theorem}[Zorn's Lemma]\index{Zorn's Lemma}
  Let \((\zeta, \leq)\) be a non-empty partially ordered set. Suppose that any chain has an upper bound. Then \(\zeta\) has a maximal element.
\end{theorem}

Recall a result derived from Zorn's Lemma in IB Groups, Rings and Modules:

\begin{lemma}
  Let \(R\) be a ring. Then \(R\) has a maximal ideal.
\end{lemma}

\begin{proof}
  Let \(\zeta\) be the set of proper ideals of \(R\). \(\{0\}\) is a proper ideal so \(\zeta\) is non-empty. Let \(\zeta\) be partially orderd by inclusion.

  Note that an ideal \(I \ideal R\) is proper if and only if \(1 \notin I\). Any chain of proper ideals has an upper bound in \(\zeta\), namely the union of the chain, so by Zorn's Lemma \(\zeta\) has a maximal element, i.e.\ a maximal ideal of \(R\).
\end{proof}

\begin{theorem}[Existence of algebraic closure]
  Any field has an algebraic closure.
\end{theorem}

\begin{proof}
  Let
  \[
    \zeta = \{(f(t), j): f(t) \text{ irreducible monic in } K[t], 1 \leq j \leq \deg f \}.
  \]
  For each pair \(s = (f(t), j) \in \zeta\), introduce an indeterminate \(X_s = X_{f, j}\). Consider the polynomial ring \(K[X_s: s \in \zeta]\) and set
  \[
    \tilde f(t) = f(t) - \prod_{j = 1}^{\deg f} (t - X_{f, j}) \in K[X_s: s \in \zeta][t].
  \]
  Let \(I \ideal K[X_s: s \in \zeta]\) be the ideal generated by coefficients of all the \(\tilde f(t)\)'s, which we denote by \(a_{f, \ell}\) for \(0 \leq \ell < \deg f\). Claim that \(I \neq K[X_s: s \in \zeta]\):
  \begin{proof}
    Suppose \(1 \in I\) for contradiction. We thus have a relation
    \begin{equation*}
      \label{eqn:ideal}
      b_1 a_{f_1, \ell_1} + \dots + b_N a_{f_N, \ell_N} = 1 \in K[X_s: s \in \zeta].
      \tag{\(\ast\)}
    \end{equation*}
    Let \(L\) be a splitting field for the product \(f_1(t)\cdots f_N(t)\).For each \(i\), \(f_i\) splits over \(L\) so write
    \[
      f_i(t) = \prod_{j = 1}^{\deg f_i} (t - \alpha_{ij}).
    \]
    Define a \(K\)-linear ring homomorphism which is identity on \(K\)
    \begin{align*}
      \theta: K[X_s: s \in \zeta] &\to L \\
      X_{f_i, j} &\mapsto \alpha_{ij} \\
      X_s &\mapsto 0 \text{ otherwise}
    \end{align*}
    This induces a map \(\theta: K[X_s: s \in \zeta][t] \to L[t]\). Then
    \begin{align*}
      \theta(\tilde f_i(t)) &= \theta(f_i(t)) - \prod_{j = 1}^{\deg f_i} \theta(t - X_{f_i, j}) \\
                            &= f_i(t) - \prod_{j = 1}^{\deg f_i} (t - \alpha_{ij}) \\
                            &= 0
    \end{align*}
    so \(\theta(a_{f_i, j}) = 0\) since \(a_{f_i, j}\) are the coefficients of \(\tilde f_i(t)\). But applying \(\theta\) to \eqref{eqn:ideal} shows \(0 = 1\). Absurd.
  \end{proof}

  We have thus shown that \(I \ideal K[X_s: s \in \zeta]\) is proper. By Zorn's Lemma, there is a maximal ideal \(P \ideal K[X_s: s \in \zeta]\) containing \(I\). Set
  \[
    L_1 = K[X_s: s \in \zeta]/P,
  \]
  which is a field. We thus have a field extension \(K \leq L_1\). Claim that \(L_1\) is an algebraic closure of \(K\):

  \begin{proof}
    We first show \(K \leq L_1\) is algebraic. \(L_1\) is generated by the images \(x_{f, j}\) of the \(X_{f, j}\). However \(\tilde f(t)\) has coefficients in \(I\) and so its image in \(L_1[t]\) is the zero polynomial. Thus in \(L_1[t]\),
    \begin{equation*}
      \label{eqn:f}
      f(t) = \prod_{j = 1}^{\deg f} (t - x_{f, j})
      \tag{\(\dag\)}
    \end{equation*}
    and so \(f(x_{f, j}) = 0\). Thus the \(x_{f, j}\)'s are algebraic. Any element of \(L_1\) involves only finitely many of the \(x_{f, j}\)'s and so is algebraic over \(K\).

    Moreover, from \eqref{eqn:f} any \(f(t) \in K[t]\) splits completely over \(L_1\). The claim thus follows from \Cref{lem:characterisation of algebraic closure}.
  \end{proof}
\end{proof}

\begin{theorem}
  Suppose \(\theta: K \to L\) is a ring homomorphism and \(L\) is algebraically closed. Suppose \(K \leq M\) is an algebraic extension, then \(\theta\) can be extended to a homomorphism \(\phi: M \to L\), i.e.\ \(\phi|_K = \theta\).
\end{theorem}

\begin{proof}
  Let
  \[
    \zeta = \{(N, \phi): K \leq N \leq M: \phi: N \to L \text{ homomorphism extending } \theta \}
  \]
  and define a partial order \(\leq\) on it by
  \[
    (N_1, \phi_1) \leq (N_2, \phi_2) \text{ if }
    \begin{cases}
      N_1 \leq N_2 & \\
      \phi_2|_{N_1} = \phi_1 & 
    \end{cases}
  \]
  \((K, \theta) \in \zeta\) so it is non-empty. If there is an ascending chain
  \[
    (N_1, \phi_1) \leq (N_2, \phi_2) \leq \cdots
  \]
  then set \(N = \bigcup_\lambda N_\lambda\), which is a subfield of \(M\) and we can define a homomorphism \(N \to L\) as follow: if \(\alpha \in N\) then \(\alpha \in N_\lambda\) for some \(\lambda\) and we send \(\alpha \mapsto \phi_\lambda(\alpha)\). This is well-defined. Thus this is an upper bound for the chain in \(\zeta\). Zorn's Lemma thus says that there is an maximal element of \(\zeta\), \((N, \phi)\).

  We now show that \(N = M\). Given \(\alpha \in M\), it is algebraic over \(K\) and hence over \(N\). Let \(f_\alpha(t)\) be its minimal polynomial over \(N\). \(\phi(f_\alpha(t)) \in L[t]\) and so splits completely since \(L\) is algebraically closed. Write
  \[
    \phi(f_\alpha(t)) = (t - \beta_1) \cdots (t - \beta_r).
  \]
  Since \(\phi(f_\alpha(\beta_j)) = 0\) there is a map
  \begin{align*}
    N(\alpha) \cong N[t]/(f_\alpha(t)) &\to L \\
    \alpha &\mapsto \beta_1
  \end{align*}
  extending \(\phi\). Maximality of \((N, \phi)\) implies that \(N(\alpha) = N\). Thus \(\alpha \in N\) and \(N = M\).
\end{proof}

With this theorem we can finally show

\begin{theorem}[Uniqueness of Algebraic Closure]
  If \(K \leq L_1, K \leq L_2\) are two algebraic closures of \(K\), there exists an isomorphism \(\phi: L_1 \to L_2\).
\end{theorem}

\begin{proof}
  By the previous theorem there is a homomorphism \(\phi: L_1 \to L_2\) extending the embedding of \(K\) in \(L_2\). Since \(K \leq L_2\) is algebraic, so is \(K \leq \phi(L_1)\). But \(L_1\) is algebraically closed and so \(\phi(L_1)\) is algebraically closed. Thus \(L_2 = \phi(L_1)\) and \(\phi\) is an isomorphism.
\end{proof}

\subsection{Symmetric Polynomials \& Invariant Theory}

In the build-up of \nameref{thm:fundamental} we met \nameref{thm:artin}, which says that if \(K \leq L\), \(H\) is a finite subgroup of \(\aut_K(L)\) and \(M = L^H\) then \(M \leq L\) is a Galois extension and \(H = \gal(L/M)\).

\begin{eg}
  Let \(L = K(X_1, \dots, X_n)\). Let \(S_n\) be the permutation group on the \(X_i\)'s. These permutations induce \(K\)-automorphisms of \(L\). By \nameref{thm:artin}, \(M = L^{S_n} \leq L\) is Galois and \(\gal(L/M) = S_n\). Thus \(S_n\) is a Galois group of \emph{some} field extension.

\end{eg}

Since we can regard any finite group \(G\) as a subgroup of some \(S_n\), any finite group is a Galois group of \emph{some} finite field extension
\[
  \Q \leq K.
\]
Finding the field extension is the \emph{inverse Galois problem}.\index{inverse Galois problem}

Using notation from the previous example, let
\[
  f(t) = \prod_{i = 1}^n (t - X_i) = t^n - s_1 t^{n - 1} + s_2 t^{n - 2} + \cdots + (-1)^n s_n \in M[t]
\]
where
\begin{align*}
  s_1 &= X_1 + \dots + X_n \\
  s_1 &= \sum_{i < j} X_iX_j \\
      &\vdots \\
  s_n &= X_1X_2\cdots X_n
\end{align*}

\begin{definition}[Elementary symmetric polynomial]\index{elementary symmetric polynomial}
  The \(s_i\)'s are the \emph{elementary symmetric polynomials}.
\end{definition}

\begin{definition}[Algebraic independence]\index{algebraic independence}
  \(\alpha_1, \dots, \alpha_n\) are \emph{algebraically independent} over \(K\) if the ring homomorphism
    \begin{align*}
      K[Y_1, \dots, Y_n] &\to K[\alpha_1, \dots, \alpha_n] \leq L \\
      Y_i &\mapsto \alpha_i
    \end{align*}
    is an isomorphism where \(K[Y_1, \dots, Y_n]\) is the polynomial ring in \(Y_1, \dots, Y_n\).
\end{definition}

\begin{theorem}
  The fixed field \(M = L^{S_n}\) equals to \(K(s_1, \dots, s_n)\) and the \(s_i\)'s are algebraically independent over \(K\) in \(L\).
\end{theorem}

\begin{proof}
  Certainly the \(s_i\)'s are fixed by \(S_n\) so \(M_1 = K(s_1, \dots, s_n) \leq M\). Observe that \(L\) is the splitting field for \(f(t) = \prod_{i = 1}^n (t - X_i)\) over \(M_1\). But \(f(t)\) has degree \(n\) and so by a proposition we proved in example sheet, the degree of the splitting field over \(M_1\) is at most \(n!\). \nameref{thm:artin} implies that \(|L:M| = |S_n| = n!\) so \(M_1 = M\).

  To show the algebraic independence of \(s_i\)'s, we make use of the idea of transcendence bases and transcendence degree: we may consider the algebraically independent subsets of \(L\) and partially order them by inclusion. Note that the union of a chain is algebraically independent and thus an upper bound of the chain. By Zorn's Lemma there is a maximal algebraically independent subset, which is a \emph{transcendence basis}.\index{transcendence basis} The \emph{transcendence degree} of \(L\) over \(K\) is the cardinality of such a set, which is well-defined by Steinitz Exchange Lemma for infinite sets, and is denoted \(\trdeg_K(L)\).

  Since \(L\) is algebraic over \(M\), we have \(\trdeg_K(L) = \trdeg_K(M)\). If \(s_i\)'s were algebraically dependent then \(\trdeg_K(M) < n\) as \(M\) would be algebraic over a subfield generated by fewer elements and \(\trdeg_K K(\alpha_1, \dots, \alpha_n) \leq n\) by induction. Absurd.
\end{proof}

Polynomial invariant theory studies \(K[X_1, \dots, X_n]^H\) for a finite subgroup \(H \leq S_n\) and as an aside, rather than confining ourselves to permutations of the variables, we may also consider \(H \leq \GL(V)\) where \(V\) is the \(K\)-vector space generated by \(X_1, \dots, X_n\).

\begin{question}\leavevmode
  \begin{enumerate}
  \item Is \(K[X_1, \dots, X_n]^H\) finitely generated?
  \item Is \(K[X_1, \dots, X_n]^H\) isomormphic to any polynomial algebra?
  \end{enumerate}
\end{question}

We have shown above that the fixed field of \(K(X_1, \dots, X_n)\) is \(K(s_1, \dots, s_n)\). A similar result holds for the ring:

\begin{theorem}
  \[
    K[X_1, \dots, X_n]^{S_n} = K[s_1, \dots, s_n].
  \]
\end{theorem}

\begin{definition}[Symmetric polynomial]\index{symmetric polynomial}
  The elements of \(K[s_1, \dots, s_n]\) are the \emph{symmetric polynomials}.
\end{definition}

\begin{proof}
  Let \(f(X_1, \dots, X_n) \in K[X_1, \dots, X_n]^{S_n}\). The proof is by induction on the total degree of \(f\). If the total degree is \(0\) then \(f\) is a constant polynomial and thus in \(K\), so in \(K[s_1, \dots, s_n]\).

  Suppose the total degree of \(f\) is positive. Let
  \begin{align*}
    \theta: K[X_1, \dots, X_n] &\to K[X_1, \dots, X_{n - 1}] \\
    g(X_1, \dots, X_n) &\mapsto g(X_n, \dots, X_{n - 1}, 0)
  \end{align*}
  i.e.\ the projection of the first \(n - 1\) coordinates. Then \(\ker \theta = (X_n)\). Since \(f(X_1, \dots, X_n)\) is fixed by \(S_n\), by slight abuse of notation \(\theta(f(X_1, \dots, X_n)) = f(X_1, \dots, X_{n - 1})\) is fixed under the subgroup \(S_{n - 1}\), i.e.\ the subgroup that fixes \(n\). Note that in particular
  \[
    \theta(s_j(X_1, \dots, X_n)) = s_j(X_1, \dots, X_{n - 1})
  \]
  for \(j \leq n - 1\) where \(s_j\) is the \(j\)th elementary symmetric polynomial. Apply induction,
  \[
    \theta(f(X_1, \dots, X_n)) = p(s_1(X_1, \dots, X_{n - 1}), \dots, s_{n - 1}(X_1, \dots, X_{n - 1}))
  \]
  where \(p\) is a polynomial. Rearrange,
  \[
    \theta(f(X_1, \dots, X_n) - p(s_1(X_1, \dots, X_n), \dots, s_{n - 1}(X_1, \dots, X_n))) = 0
  \]
  so \(X_n\) divides the polynomial
  \begin{equation*}
    \label{eqn:symmetric polynomial}
    f(X_1, \dots, X_n) - p(s_1(X_1, \dots, X_n), \dots, s_{n - 1}(X_1, \dots, X_n))
    \tag{\(\ast\)}
  \end{equation*}
  which is a symmetric polynomial since it is fixed under \(S_n\). It follows that \(X_i\) divides \eqref{eqn:symmetric polynomial} for all \(i\). However, \(K[X_1, \dots, X_n]\) is a UFD and the \(X_i\)'s are coprime so the product \(X_1\cdots X_n\) divides \eqref{eqn:symmetric polynomial}. Write
  \[
    f(X_1, \dots, X_n) = g(X_1, \dots, X_n) X_1\cdots X_n + p(s_1(X_1, \dots, X_n), \dots, s_{n - 1}(X_1, \dots, X_n)).
  \]
  Observe that the total degree of \(g(X_1, \dots, X_n)\) is smaller that that of \(f(X_1, \dots, X_n)\) and that \(g(X_1, \dots, X_n)\) is fixed under \(S_n\). Applying induction to \(g(X_1, \dots, X_n)\) and so it is a polynomial in \(s_i\)'s. Thus \(f(X_1, \dots, X_n) \in K[s_1, \dots, s_n]\).
\end{proof}

\begin{eg}
  \(K[X_1, \dots, X_n]^{A_n}\) is generated by \(s_i\)'s and
  \[
    \Delta(X_1, \dots, X_n) = \prod_{i < j}(X_i - X_j).
  \]
\end{eg}

Emmy Noether in 1920s consider other subgroups of \(S_n\) and showed that the invariant rings are Noetherian.

\begin{theorem}[Chevalley-Shephard-Todd]
  \(\C[X_1, \dots, X_n]^H\) where \(H \leq \GL(V)\) finite is isomorphic to a polynomial algebra if and only if \(H\) is generated by pesudoreflections, whose \(1\)-eigenspace has codimension \(1\).
\end{theorem}



\printindex
\end{document}

\fi
