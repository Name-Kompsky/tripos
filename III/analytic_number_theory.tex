\documentclass[a4paper]{article}

\def\npart{III}

\def\ntitle{Analytic Number Theory}
\def\nlecturer{T.\ Bloom}

\def\nterm{Lent}
\def\nyear{2019}

\ifx \nauthor\undefined
  \def\nauthor{Qiangru Kuang}
\else
\fi

\ifx \ntitle\undefined
  \def\ntitle{Template}
\else
\fi

\ifx \nauthoremail\undefined
  \def\nauthoremail{qk206@cam.ac.uk}
\else
\fi

\ifx \ndate\undefined
  \def\ndate{\today}
\else
\fi

\title{\ntitle}
\author{\nauthor}
\date{\ndate}

%\usepackage{microtype}
\usepackage{mathtools}
\usepackage{amsthm}
\usepackage{stmaryrd}%symbols used so far: \mapsfrom
\usepackage{empheq}
\usepackage{amssymb}
\let\mathbbalt\mathbb
\let\pitchforkold\pitchfork
\usepackage{unicode-math}
\let\mathbb\mathbbalt%reset to original \mathbb
\let\pitchfork\pitchforkold

\usepackage{imakeidx}
\makeindex[intoc]

%to address the problem that Latin modern doesn't have unicode support for setminus
%https://tex.stackexchange.com/a/55205/26707
\AtBeginDocument{\renewcommand*{\setminus}{\mathbin{\backslash}}}
\AtBeginDocument{\renewcommand*{\models}{\vDash}}%for \vDash is same size as \vdash but orginal \models is larger
\AtBeginDocument{\let\Re\relax}
\AtBeginDocument{\let\Im\relax}
\AtBeginDocument{\DeclareMathOperator{\Re}{Re}}
\AtBeginDocument{\DeclareMathOperator{\Im}{Im}}
\AtBeginDocument{\let\div\relax}
\AtBeginDocument{\DeclareMathOperator{\div}{div}}

\usepackage{tikz}
\usetikzlibrary{automata,positioning}
\usepackage{pgfplots}
%some preset styles
\pgfplotsset{compat=1.15}
\pgfplotsset{centre/.append style={axis x line=middle, axis y line=middle, xlabel={$x$}, ylabel={$y$}, axis equal}}
\usepackage{tikz-cd}
\usepackage{graphicx}
\usepackage{newunicodechar}

\usepackage{fancyhdr}

\fancypagestyle{mypagestyle}{
    \fancyhf{}
    \lhead{\emph{\nouppercase{\leftmark}}}
    \rhead{}
    \cfoot{\thepage}
}
\pagestyle{mypagestyle}

\usepackage{titlesec}
\newcommand{\sectionbreak}{\clearpage} % clear page after each section
\usepackage[perpage]{footmisc}
\usepackage{blindtext}

%\reallywidehat
%https://tex.stackexchange.com/a/101136/26707
\usepackage{scalerel,stackengine}
\stackMath
\newcommand\reallywidehat[1]{%
\savestack{\tmpbox}{\stretchto{%
  \scaleto{%
    \scalerel*[\widthof{\ensuremath{#1}}]{\kern-.6pt\bigwedge\kern-.6pt}%
    {\rule[-\textheight/2]{1ex}{\textheight}}%WIDTH-LIMITED BIG WEDGE
  }{\textheight}% 
}{0.5ex}}%
\stackon[1pt]{#1}{\tmpbox}%
}

%\usepackage{braket}
\usepackage{thmtools}%restate theorem
\usepackage{hyperref}

% https://en.wikibooks.org/wiki/LaTeX/Hyperlinks
\hypersetup{
    %bookmarks=true,
    unicode=true,
    pdftitle={\ntitle},
    pdfauthor={\nauthor},
    pdfsubject={Mathematics},
    pdfcreator={\nauthor},
    pdfproducer={\nauthor},
    pdfkeywords={math maths \ntitle},
    colorlinks=true,
    linkcolor={red!50!black},
    citecolor={blue!50!black},
    urlcolor={blue!80!black}
}

\usepackage{cleveref}



% TODO: mdframed often gives bad breaks that cause empty lines. Would like to switch to tcolorbox.
% The current workaround is to set innerbottommargin=0pt.

%\usepackage[theorems]{tcolorbox}





\usepackage[framemethod=tikz]{mdframed}
\mdfdefinestyle{leftbar}{
  %nobreak=true, %dirty hack
  linewidth=1.5pt,
  linecolor=gray,
  hidealllines=true,
  leftline=true,
  leftmargin=0pt,
  innerleftmargin=5pt,
  innerrightmargin=10pt,
  innertopmargin=-5pt,
  % innerbottommargin=5pt, % original
  innerbottommargin=0pt, % temporary hack 
}
%\newmdtheoremenv[style=leftbar]{theorem}{Theorem}[section]
%\newmdtheoremenv[style=leftbar]{proposition}[theorem]{proposition}
%\newmdtheoremenv[style=leftbar]{lemma}[theorem]{Lemma}
%\newmdtheoremenv[style=leftbar]{corollary}[theorem]{corollary}

\newtheorem{theorem}{Theorem}[section]
\newtheorem{proposition}[theorem]{Proposition}
\newtheorem{lemma}[theorem]{Lemma}
\newtheorem{corollary}[theorem]{Corollary}
\newtheorem{axiom}[theorem]{Axiom}
\newtheorem*{axiom*}{Axiom}

\surroundwithmdframed[style=leftbar]{theorem}
\surroundwithmdframed[style=leftbar]{proposition}
\surroundwithmdframed[style=leftbar]{lemma}
\surroundwithmdframed[style=leftbar]{corollary}
\surroundwithmdframed[style=leftbar]{axiom}
\surroundwithmdframed[style=leftbar]{axiom*}

\theoremstyle{definition}

\newtheorem*{definition}{Definition}
\surroundwithmdframed[style=leftbar]{definition}

\newtheorem*{slogan}{Slogan}
\newtheorem*{eg}{Example}
\newtheorem*{ex}{Exercise}
\newtheorem*{remark}{Remark}
\newtheorem*{notation}{Notation}
\newtheorem*{convention}{Convention}
\newtheorem*{assumption}{Assumption}
\newtheorem*{question}{Question}
\newtheorem*{answer}{Answer}
\newtheorem*{note}{Note}
\newtheorem*{application}{Application}

%operator macros

%basic
\DeclareMathOperator{\lcm}{lcm}

%matrix
\DeclareMathOperator{\tr}{tr}
\DeclareMathOperator{\Tr}{Tr}
\DeclareMathOperator{\adj}{adj}

%algebra
\DeclareMathOperator{\Hom}{Hom}
\DeclareMathOperator{\End}{End}
\DeclareMathOperator{\id}{id}
\DeclareMathOperator{\im}{im}
\DeclareMathOperator{\coker}{coker}
\DeclarePairedDelimiter{\generation}{\langle}{\rangle}

%groups
\DeclareMathOperator{\sym}{Sym}
\DeclareMathOperator{\sgn}{sgn}
\DeclareMathOperator{\inn}{Inn}
\DeclareMathOperator{\aut}{Aut}
\DeclareMathOperator{\GL}{GL}
\DeclareMathOperator{\SL}{SL}
\DeclareMathOperator{\PGL}{PGL}
\DeclareMathOperator{\PSL}{PSL}
\DeclareMathOperator{\SU}{SU}
\DeclareMathOperator{\UU}{U}
\DeclareMathOperator{\SO}{SO}
\DeclareMathOperator{\OO}{O}
\DeclareMathOperator{\PSU}{PSU}
\DeclareMathOperator{\Sp}{Sp}


%hyperbolic
\DeclareMathOperator{\sech}{sech}

%field, galois heory
\DeclareMathOperator{\ch}{ch}
\DeclareMathOperator{\gal}{Gal}
\DeclareMathOperator{\emb}{Emb}



%ceiling and floor
%https://tex.stackexchange.com/a/118217/26707
\DeclarePairedDelimiter\ceil{\lceil}{\rceil}
\DeclarePairedDelimiter\floor{\lfloor}{\rfloor}


\DeclarePairedDelimiter{\innerproduct}{\langle}{\rangle}

%\DeclarePairedDelimiterX{\norm}[1]{\lVert}{\rVert}{#1}
\DeclarePairedDelimiter{\norm}{\lVert}{\rVert}



%Dirac notation
%TODO: rewrite for variable number of arguments
\DeclarePairedDelimiterX{\braket}[2]{\langle}{\rangle}{#1 \delimsize\vert #2}
\DeclarePairedDelimiterX{\braketthree}[3]{\langle}{\rangle}{#1 \delimsize\vert #2 \delimsize\vert #3}

\DeclarePairedDelimiter{\bra}{\langle}{\rvert}
\DeclarePairedDelimiter{\ket}{\lvert}{\rangle}




%macros

%general

%divide, not divide
\newcommand*{\divides}{\mid}
\newcommand*{\ndivides}{\nmid}
%vector, i.e. mathbf
%https://tex.stackexchange.com/a/45746/26707
\newcommand*{\V}[1]{{\ensuremath{\symbf{#1}}}}
%closure
\newcommand*{\cl}[1]{\overline{#1}}
%conjugate
\newcommand*{\conj}[1]{\overline{#1}}
%set complement
\newcommand*{\stcomp}[1]{\overline{#1}}
\newcommand*{\compose}{\circ}
\newcommand*{\nto}{\nrightarrow}
\newcommand*{\p}{\partial}
%embed
\newcommand*{\embed}{\hookrightarrow}
%surjection
\newcommand*{\surj}{\twoheadrightarrow}
%power set
\newcommand*{\powerset}{\mathcal{P}}

%matrix
\newcommand*{\matrixring}{\mathcal{M}}

%groups
\newcommand*{\normal}{\trianglelefteq}
%rings
\newcommand*{\ideal}{\trianglelefteq}

%fields
\renewcommand*{\C}{{\mathbb{C}}}
\newcommand*{\R}{{\mathbb{R}}}
\newcommand*{\Q}{{\mathbb{Q}}}
\newcommand*{\Z}{{\mathbb{Z}}}
\newcommand*{\N}{{\mathbb{N}}}
\newcommand*{\F}{{\mathbb{F}}}
%not really but I think this belongs here
\newcommand*{\A}{{\mathbb{A}}}

%asymptotic
\newcommand*{\bigO}{O}
\newcommand*{\smallo}{o}

%probability
\newcommand*{\prob}{\mathbb{P}}
\newcommand*{\E}{\mathbb{E}}

%vector calculus
\newcommand*{\gradient}{\V \nabla}
\newcommand*{\divergence}{\gradient \cdot}
\newcommand*{\curl}{\gradient \cdot}

%logic
\newcommand*{\yields}{\vdash}
\newcommand*{\nyields}{\nvdash}

%differential geometry
\renewcommand*{\H}{\mathbb{H}}
\newcommand*{\transversal}{\pitchfork}
\renewcommand{\d}{\mathrm{d}} % exterior derivative

%number theory
\newcommand*{\legendre}[2]{\genfrac{(}{)}{}{}{#1}{#2}}%Legendre symbol

%algebraic geometry
\DeclareMathOperator{\Spec}{Spec}
\DeclareMathOperator{\Proj}{Proj}

\newtheorem*{fact}{Fact}

\begin{document}

\begin{titlepage}
  \begin{center}
    \includegraphics[width=0.6\textwidth]{logo.jpg}\par
    \vspace{1cm}
    {\scshape\huge Mathamatics Tripos \par}
    \vspace{2cm}
    {\huge Part \npart \par}
    \vspace{0.6cm}
    {\Huge \bfseries \ntitle \par}
    \vspace{1.2cm}
    {\Large\nterm, \nyear \par}
    \vspace{2cm}
    
    {\large \emph{Lectures by } \par}
    \vspace{0.2cm}
    {\Large \scshape \nlecturer}
    
    \vspace{0.5cm}
    {\large \emph{Notes by }\par}
    \vspace{0.2cm}
    {\Large \scshape \href{mailto:\nauthoremail}{\nauthor}}
 \end{center}
\end{titlepage}

\tableofcontents

\setcounter{section}{-1}

\section{Introduction}

Analytic number theory is the study of numbers using analysis. In particular it answers quantitative questions. ``Numbers'' means natural numbers in this course, which excludes \(0\).

\begin{eg}\leavevmode
  \begin{enumerate}
  \item How many primes are there? We know there are infinitely many but can we have a more precise answer? Let \(\pi(x)\) be the number of primes smaller than or equal to \(x\). Then by the famous prime number theorem, \(\pi(x) \sim \frac{x}{\log x}\).
  \item How may twin primes are there? It is not known whethere there are infinitely many. From 2014 Zhang, Maynard, Polymath, there are infinitely many primes at most \(246\) apart. It's been conjectured that the asymptotic bound is \(\sim \frac{x}{(\log x)^2}\).
  \item How many primes are there congruent to \(a\) mod \(q\) where \((a, q) = 1\)? There are infinitely many by Dirichlet's theorem. The guess is \(\frac{1}{\varphi(q)}\frac{x}{\log x}\). This is known for small \(q\).
  \end{enumerate}
\end{eg}

The course is divided into four parts:
\begin{enumerate}
\item elementary techniques (using real analysis),
\item sieve methods,
\item Riemann zeta function/Prime number theorem (using complex analysis),
\item primes in arithmetic progression. 
\end{enumerate}

\section{Elementary techniques}

Review of asymptotic notations:
\begin{itemize}
\item Landau notation: \(f(x) = O(g(x))\) if there is \(C > 0\) such that \(|f(x)| \leq C|g(x)|\) for all large enough \(x\).
\item Vinogradov notation: \(f \ll g\) is the same as \(f = O(g)\).
\item \(f \sim g\) if \(\lim_{x \to \infty} \frac{f(x)}{g(x)} = 1\), i.e.\ \(f = (1 + o(1)) g\).
\item \(f = o(g)\) if \(\lim_{x \to \infty} \frac{f(x)}{g(x)} = 0\).
\end{itemize}

\subsection{Arithmetic functions}

These are just functions \(f: \N \to \C\). An important operation for multiplicative number theory is \emph{multiplicative convolution}\index{multiplicative convolution}
\[
  f * g(n) = \sum_{ab = n} f(a)g(b)
\]

\begin{eg}\leavevmode
  \begin{enumerate}
  \item \(1(n) = 1\) for all \(n\). Caution: this is not identity on \(\N\).
  \item \emph{Möbius function}\index{Möbius function}
    \[
      \mu(n) =
      \begin{cases}
        (-1)^k & \text{ if } n = p_1 \dots p_k \\
        0 & \text{ if \(n\) is divisible by a square}
      \end{cases}
    \]
  \item \emph{Liouville function}
    \[
      \lambda(n) = (-1)^k
    \]
    if \(n = p_1 \dots p_k\) where \(p_i\)'s are not necessarily distinct.
  \item \emph{divisor function}
    \[
      \tau(n) = \#d \text{ such that } d \divides n = \sum_{ab = n} 1 = 1 * 1(n).
    \]
    This is sometimes also denoted by \(d(n)\).
  \end{enumerate}
\end{eg}

\begin{definition}[multiplicative function]\index{multiplicative function}
  An arithmetic function \(f\) is \emph{multiplicative} if
  \[
    f(nm) = f(n)f(m)
  \]
  whenever \((n, m) = 1\).
\end{definition}

In particular a multiplicative function is determined by its values on prime powers \(f(p^k)\).

\begin{fact}
  If \(f\) and \(g\) are multiplicative then so is \(f * g\).
\end{fact}

\begin{eg}
  \(1, \mu, \lambda, \tau\) are multiplicative. \(\log n\) is not multiplicative.
\end{eg}

\begin{fact}[Möbius inversion]\index{Möbius inversion}
  \(1 * f = g\) if and only if \(\mu * g = f\). That is,
  \[
    \sum_{d \divides n} f(d) = g(n)
  \]
  if and only if
  \[
    \sum_{d \divides n} g(d) \mu(\frac{n}{d}) = f(n).
  \]

  For example
  \[
    \sum_{d \divides n} \mu(d) =
    \begin{cases}
      1 & n = 1 \\
      0 & \text{otherwise}
    \end{cases}
    = 1 * \mu (n)
  \]
  is multiplicative so enough to check the identity for prime powers. If \(n = p^k\) then \(\{d: d \divides n\} = \{1, p, \dots, p^k\}\) so LHS equals to \(1 - 1 + 0 + \dots = 0\) unless \(k = 1\) when LHS equals to \(\mu(1) = 1\).
\end{fact}

Our goal is to study primes. Our first might be that we shall work with
\[
  1_p(n) =
  \begin{cases}
    1 & n \text{ prime} \\
    0 & \text{otherwise}
  \end{cases}
\]
as then \(\pi(x) = \sum_{1 \leq n \leq x} 1_p(n)\). But this is very awkward to work with, as to begin with, this is not multiplicative. Instead, we are going to work almost exclusively with \emph{von Mangoldt function}\index{von Mangoldt function}
\[
  \Lambda(n) =
  \begin{cases}
    \log p & n = p^k \\
    0 & \text{otherwise}
  \end{cases}
\]
``assign weight \(\log p\) to prime power \(n\)''

\begin{lemma}
  \[
    1 * \Lambda = \log.
  \]
  and
  \[
    \mu * \log = \Lambda.
  \]
\end{lemma}

\begin{proof}
  The second part follows from Möbius inversion. Thus if \(n = p_1^{k_1} \dots p_r^{k_r}\),
  \begin{align*}
    1 * \Lambda(n)
    &= \sum_{d \divides n} \Lambda(d)
    = \sum_{i = 1}^r \sum_{j = 1}^{k_i} \Lambda(p_i^j) \\
    &= \sum_{i = 1}^r \sum_{j = 1}^{k_i} \log(p_i)
    = \sum_{i = 1}^r k_i \log p_i \\
    &= \sum_{i = 1}^r \log (p_i^{k_i})
    = \log n
  \end{align*}
\end{proof}

Therefore
\begin{align*}
  \Lambda(n)
  &= \sum_{d \divides n} \mu(d) \log(\frac{n}{d}) \\
  &= \log n \sum_{d \divides n} \mu(d) - \sum_{d \divides n} \mu(d) \log d \\
  &= - \sum_{d \divides n} \mu(d) \log d
\end{align*}
For example
\[
  \sum_{1 \leq n \leq x} \Lambda(n)
  = - \sum_{1 \leq n \leq x} \sum_{d \divides n} \mu(d) \log d
  = - \sum_{d \leq x} \mu(d) \log d \left(\sum_{1 \leq n \leq x, d \divides n} 1\right)
\]
by reversing summation. But now the term in the inner summation is very easy to understand:
\[
  \sum_{1 \leq n \leq x, d \divides n} 1 = \floor*{\frac{x}{d}} = \frac{x}{d} + O(1).
\]
Thus
\[
  \sum_{1 \leq n \leq x} \Lambda(n)
  = -x \sum_{d \leq x} \mu(d) \frac{\log d}{d} + O\left(\sum_{d \leq x} \mu(d) \log d\right).
\]
We'll see more of these examples.

\subsection{Summation}

Given an arithmetic function \(f\), we can ask for estimates of \(\sum_{1 \leq n \leq x} f(n)\). We say that \(f\) has \emph{average order}\index{average order} \(g\) if
\[
  \sum_{1 \leq n \leq x} f(n) \sim x g(x).
\]
``average size of \(f\) is \(g\)''.

\begin{eg}\leavevmode
  \begin{enumerate}
  \item \(f = 1\) then
    \[
      \sum_{1 \leq n \leq x} f(x) = \floor{x} = x + O(1) \sim x
    \]
    so average order of \(1\) is \(1\).
  \item \(f(n) = n\):
    \[
      \sum_{1 \leq n \leq x} n \sim \frac{x^2}{2}
    \]
    so average of \(n\) is \(\frac{n}{2}\).
  \end{enumerate}
\end{eg}

\begin{lemma}[partial summation]\index{partial summation}
  If \((a_n)\) is a sequence of complex numbers and \(f\) is such that \(f'\) is continuous. Then
  \[
    \sum_{1 \leq n \leq x} a_n f(n) = A(x) f(x) - \int_1^x A(t)f'(t) dt
  \]
  where \(A(x) = \sum_{1 \leq n \leq x} a_n\).
\end{lemma}

This is the discrete analogus of integration by parts.

\begin{proof}
  Suppose \(x = N\) is an integer. Note that \(a_n = A(n) - A(n - 1)\), so
  \begin{align*}
    \sum_{1 \leq n \leq N} a_nf(n)
    &= \sum_{1 \leq n \leq N} f(n) (A(n) - A(n - 1)) \\
    &= A(N)f(N) - \sum_{n = 1}^{N - 1} A(n) (f(n + 1) - f(n))
  \end{align*}
  Now
  \[
    f(n + 1) - f(n) = \int_n^{n + 1} f'(t) dt
  \]
  so
  \begin{align*}
    \sum_{1 \leq n \leq N} a_n f(n)
    &= A(N)f(N) - \sum_{n = 1}^{N - 1} A(n) \int_n^{n + 1} f'(t) dt \\
    &= A(N)f(N) - \int_1^N A(t) f'(t) dt
  \end{align*}
  where the last step is because \(A(n) = A(t)\) for \(t \in [n, n + 1)\).

  If \(N = \floor x\) then
  \[
    A(x)f(x)
    = A(N)f(x)
    = A(N) f(N) + \int_N^x f'(t) dt.
  \]
\end{proof}

As a simple application

\begin{lemma}
  \[
    \sum_{1 \leq n \leq x} \frac{1}{n} = \log x + \gamma + O(\frac{1}{x}).
  \]
\end{lemma}

\begin{proof}
  Partial summation with \(f(x) = \frac{1}{x}\) and \(a_n = 1\), so \(A(x) = \floor x\). Therefore
  \begin{align*}
    \sum_{1 \leq n \leq x} \frac{1}{n}
    &= \frac{\floor x}{x} + \int_1^x \frac{\floor t}{t^2} dt
    \intertext{Write \(\floor t= t - \{t\}\),} \\
    &= 1 + O(\frac{1}{x}) + \int_1^x \frac{1}{t} dt - \int_1^x \frac{\{t\}}{t^2} dt \\
    &= 1 + O(\frac{1}{x}) + \log x - \int_1^\infty \frac{\{t\}}{t^2} dt + \underbrace{\int_x^\infty \frac{\{t\}}{t^2} dt}_{\leq \int_x^\infty \frac{1}{t^2} dt \leq \frac{1}{x}} \\
    &= \gamma + O(\frac {1}{x}) + \log x + O(\frac{1}{x}) \\
    &= \log x + \gamma + O(\frac{1}{x})
  \end{align*}
\end{proof}

This is an amazing result and the only thing we did is to replace the discrete summation by the continuous analogue to it. In essence this is the whole reason analytic number theory works.

\(\gamma\) can be seen as a measure of the difference between between \(\log\) and its discrete approximation. It is called \emph{Euler-Mascheroni constant}\index{Euler-Mascheroni constant}. 
Surprisingly little is known about \(\gamma\). It is approximately \(0.577\dots\). We don't even know if \(\gamma\) is rational or not.

\begin{lemma}
  \[
    \sum_{1 \leq n \leq x} \log n = x \log x - x + O(\log x).
  \]
\end{lemma}

\begin{proof}
  Partial summation with \(f(x) = \log x, a_n = A\) so \(A(x) = \floor{x}\). As a side note, in the previous example, most error comes from the integral term (the mass is evenly distributed). By constrast in this example most error comes from the ``sum'' term.
  \begin{align*}
    \sum_{1 \leq n \leq x} \log n
    &= \floor{x} \log x - \int_1^x \frac{\floor{t}}{t} dt \\
    &= x \log x + O(\log x) - \int_1^x dt + O(\int_1^x \frac{1}{t}dt) \\
    &= x \log x + O(\log x) - x + O(\log x)
  \end{align*}
\end{proof}

\subsection{Divisor function}

Recall that
\[
  \tau(n) = 1 * 1(n) = \sum_{ab = n} 1 = \sum_{d \divides n} 1.
\]

\begin{theorem}
  \[
    \sum_{1 \leq n \leq x} \tau(n) = x \log x + (2 \tau - 1) x + O(x^{1/2})
  \]
  so in particular average order of \(\tau\) is \(\log\).
\end{theorem}

\begin{proof}
  First attempt:
  \begin{align*}
    \sum_{1 \leq n \leq x} \tau(n)
    &= \sum_{1 \leq n \leq x} \sum_{d \divides n} 1
    = \sum_{1 \leq d \leq x} \sum_{1 \leq n \leq x, d \divides n} 1 \\
    &= \sum_{1 \leq d \leq x} \floor*{\frac{x}{d}} \\
    &= \sum_{1 \leq d \leq x} \frac{x}{d} + O(x)
    = x \sum_{1 \leq d \leq x} \frac{1}{d} + O(x) \\
    &= x\log x + \gamma x + O(x)
  \end{align*}
  This is not a very good bound (the error might be as large as one of the terms!) but shows that at least the first term is correct. The main drawback is we used the estimate
  \[
    \sum_{1 \leq d \leq x} O(1) = O(x).
  \]

  To reduce the error term, we use \emph{(Dirichlet's) hyperbola trick}\index{hyperbola trick}
  \[
    \sum_{1 \leq n \leq x} \tau(n)
    = \sum_{1 \leq n \leq x} \sum_{ab = n} 1
    = \sum_{ab \leq x} 1
    = \sum_{a \leq x} \sum_{b \leq x/a} 1
  \]
  The intuition is like this: \(\sum_{1 \leq n \leq x} \tau(n)\) counts the number of integral points below the hyperbola \(k_1k_2 = x\) in the first quadrant. The old methods amounts to an estimation by integral, while in the new method we count the number of points lying below the line \(k_2 = x^{1/2}\), add the number of points to the left of \(k_1 = x^{1/2}\), and finally subtract those points in the box \([0, x^{1/2}]^2\) which are double counted.

  Thus when summing over \(ab \leq x\), we can sum over \(a \leq x^{1/2}\) and \(b \leq x^{1/2}\) respectively, and then minus pairs \(a, b \leq \sqrt x\). Thus
  \begin{align*}
    \sum_{1 \leq n \leq x} \tau(n)
    &= \sum_{a \leq x^{1/2}} \sum_{b \leq x/a} 1 + \sum_{b \leq x^{1/2}} \sum_{a \leq x/b} 1 - \sum_{a, b \leq x^{1/2}} 1 \\
    &= 2 \sum_{a \leq x^{1/2}} \floor*{\frac{x}{a}} - \floor{x^{1/2}}^2 \\
    &= 2 \sum_{a \leq x^{1/2}} \frac{x}{a} + O(x^{1/2}) - x + O(x^{1/2}) \\
    &= 2x\log x^{1/2} + 2 \gamma x - x + O(x^{1/2}) \\
    &= x \log x + (2\gamma - 1) x + O(x^{1/2})
  \end{align*}
\end{proof}

\begin{remark}
  Improving this \(O(x^{1/2})\) error term is a famous and hard problem. Probably \(O(x^{1/4 + \varepsilon})\)? The best result so far is \(O(x^{0.3149})\).
\end{remark}

A note on average order: \(\tau\) has average order \(\log\) does not mean \(\tau(n) \ll \log n\), i.e.\ average order does not imply individual values.

\begin{theorem}
  For all \(n\)
  \[
    \tau(n) \leq n^{O(\frac{1}{\log \log n})}.
  \]
  In particular \(\tau(n) \ll_\varepsilon n^\varepsilon\) for all \(\varepsilon > 0\) where \(\ll_\varepsilon\) means that \(|\tau(n)| \leq C_\varepsilon |n^\varepsilon|\) eventually where \(C_\varepsilon\) is a constant depends on \(\varepsilon\).
\end{theorem}




\printindex
\end{document}

% www.thomasbloom.org/ant.html