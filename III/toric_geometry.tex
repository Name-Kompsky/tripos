\documentclass[a4paper]{article}

\def\npart{III}

\def\ntitle{Toric Geometry}
\def\nlecturer{D.\ Ranganathan}

\def\nterm{Michaelmas}
\def\nyear{2019}

\ifx \nauthor\undefined
  \def\nauthor{Qiangru Kuang}
\else
\fi

\ifx \ntitle\undefined
  \def\ntitle{Template}
\else
\fi

\ifx \nauthoremail\undefined
  \def\nauthoremail{qk206@cam.ac.uk}
\else
\fi

\ifx \ndate\undefined
  \def\ndate{\today}
\else
\fi

\title{\ntitle}
\author{\nauthor}
\date{\ndate}

%\usepackage{microtype}
\usepackage{mathtools}
\usepackage{amsthm}
\usepackage{stmaryrd}%symbols used so far: \mapsfrom
\usepackage{empheq}
\usepackage{amssymb}
\let\mathbbalt\mathbb
\let\pitchforkold\pitchfork
\usepackage{unicode-math}
\let\mathbb\mathbbalt%reset to original \mathbb
\let\pitchfork\pitchforkold

\usepackage{imakeidx}
\makeindex[intoc]

%to address the problem that Latin modern doesn't have unicode support for setminus
%https://tex.stackexchange.com/a/55205/26707
\AtBeginDocument{\renewcommand*{\setminus}{\mathbin{\backslash}}}
\AtBeginDocument{\renewcommand*{\models}{\vDash}}%for \vDash is same size as \vdash but orginal \models is larger
\AtBeginDocument{\let\Re\relax}
\AtBeginDocument{\let\Im\relax}
\AtBeginDocument{\DeclareMathOperator{\Re}{Re}}
\AtBeginDocument{\DeclareMathOperator{\Im}{Im}}
\AtBeginDocument{\let\div\relax}
\AtBeginDocument{\DeclareMathOperator{\div}{div}}

\usepackage{tikz}
\usetikzlibrary{automata,positioning}
\usepackage{pgfplots}
%some preset styles
\pgfplotsset{compat=1.15}
\pgfplotsset{centre/.append style={axis x line=middle, axis y line=middle, xlabel={$x$}, ylabel={$y$}, axis equal}}
\usepackage{tikz-cd}
\usepackage{graphicx}
\usepackage{newunicodechar}

\usepackage{fancyhdr}

\fancypagestyle{mypagestyle}{
    \fancyhf{}
    \lhead{\emph{\nouppercase{\leftmark}}}
    \rhead{}
    \cfoot{\thepage}
}
\pagestyle{mypagestyle}

\usepackage{titlesec}
\newcommand{\sectionbreak}{\clearpage} % clear page after each section
\usepackage[perpage]{footmisc}
\usepackage{blindtext}

%\reallywidehat
%https://tex.stackexchange.com/a/101136/26707
\usepackage{scalerel,stackengine}
\stackMath
\newcommand\reallywidehat[1]{%
\savestack{\tmpbox}{\stretchto{%
  \scaleto{%
    \scalerel*[\widthof{\ensuremath{#1}}]{\kern-.6pt\bigwedge\kern-.6pt}%
    {\rule[-\textheight/2]{1ex}{\textheight}}%WIDTH-LIMITED BIG WEDGE
  }{\textheight}% 
}{0.5ex}}%
\stackon[1pt]{#1}{\tmpbox}%
}

%\usepackage{braket}
\usepackage{thmtools}%restate theorem
\usepackage{hyperref}

% https://en.wikibooks.org/wiki/LaTeX/Hyperlinks
\hypersetup{
    %bookmarks=true,
    unicode=true,
    pdftitle={\ntitle},
    pdfauthor={\nauthor},
    pdfsubject={Mathematics},
    pdfcreator={\nauthor},
    pdfproducer={\nauthor},
    pdfkeywords={math maths \ntitle},
    colorlinks=true,
    linkcolor={red!50!black},
    citecolor={blue!50!black},
    urlcolor={blue!80!black}
}

\usepackage{cleveref}



% TODO: mdframed often gives bad breaks that cause empty lines. Would like to switch to tcolorbox.
% The current workaround is to set innerbottommargin=0pt.

%\usepackage[theorems]{tcolorbox}





\usepackage[framemethod=tikz]{mdframed}
\mdfdefinestyle{leftbar}{
  %nobreak=true, %dirty hack
  linewidth=1.5pt,
  linecolor=gray,
  hidealllines=true,
  leftline=true,
  leftmargin=0pt,
  innerleftmargin=5pt,
  innerrightmargin=10pt,
  innertopmargin=-5pt,
  % innerbottommargin=5pt, % original
  innerbottommargin=0pt, % temporary hack 
}
%\newmdtheoremenv[style=leftbar]{theorem}{Theorem}[section]
%\newmdtheoremenv[style=leftbar]{proposition}[theorem]{proposition}
%\newmdtheoremenv[style=leftbar]{lemma}[theorem]{Lemma}
%\newmdtheoremenv[style=leftbar]{corollary}[theorem]{corollary}

\newtheorem{theorem}{Theorem}[section]
\newtheorem{proposition}[theorem]{Proposition}
\newtheorem{lemma}[theorem]{Lemma}
\newtheorem{corollary}[theorem]{Corollary}
\newtheorem{axiom}[theorem]{Axiom}
\newtheorem*{axiom*}{Axiom}

\surroundwithmdframed[style=leftbar]{theorem}
\surroundwithmdframed[style=leftbar]{proposition}
\surroundwithmdframed[style=leftbar]{lemma}
\surroundwithmdframed[style=leftbar]{corollary}
\surroundwithmdframed[style=leftbar]{axiom}
\surroundwithmdframed[style=leftbar]{axiom*}

\theoremstyle{definition}

\newtheorem*{definition}{Definition}
\surroundwithmdframed[style=leftbar]{definition}

\newtheorem*{slogan}{Slogan}
\newtheorem*{eg}{Example}
\newtheorem*{ex}{Exercise}
\newtheorem*{remark}{Remark}
\newtheorem*{notation}{Notation}
\newtheorem*{convention}{Convention}
\newtheorem*{assumption}{Assumption}
\newtheorem*{question}{Question}
\newtheorem*{answer}{Answer}
\newtheorem*{note}{Note}
\newtheorem*{application}{Application}

%operator macros

%basic
\DeclareMathOperator{\lcm}{lcm}

%matrix
\DeclareMathOperator{\tr}{tr}
\DeclareMathOperator{\Tr}{Tr}
\DeclareMathOperator{\adj}{adj}

%algebra
\DeclareMathOperator{\Hom}{Hom}
\DeclareMathOperator{\End}{End}
\DeclareMathOperator{\id}{id}
\DeclareMathOperator{\im}{im}
\DeclareMathOperator{\coker}{coker}
\DeclarePairedDelimiter{\generation}{\langle}{\rangle}

%groups
\DeclareMathOperator{\sym}{Sym}
\DeclareMathOperator{\sgn}{sgn}
\DeclareMathOperator{\inn}{Inn}
\DeclareMathOperator{\aut}{Aut}
\DeclareMathOperator{\GL}{GL}
\DeclareMathOperator{\SL}{SL}
\DeclareMathOperator{\PGL}{PGL}
\DeclareMathOperator{\PSL}{PSL}
\DeclareMathOperator{\SU}{SU}
\DeclareMathOperator{\UU}{U}
\DeclareMathOperator{\SO}{SO}
\DeclareMathOperator{\OO}{O}
\DeclareMathOperator{\PSU}{PSU}
\DeclareMathOperator{\Sp}{Sp}


%hyperbolic
\DeclareMathOperator{\sech}{sech}

%field, galois heory
\DeclareMathOperator{\ch}{ch}
\DeclareMathOperator{\gal}{Gal}
\DeclareMathOperator{\emb}{Emb}



%ceiling and floor
%https://tex.stackexchange.com/a/118217/26707
\DeclarePairedDelimiter\ceil{\lceil}{\rceil}
\DeclarePairedDelimiter\floor{\lfloor}{\rfloor}


\DeclarePairedDelimiter{\innerproduct}{\langle}{\rangle}

%\DeclarePairedDelimiterX{\norm}[1]{\lVert}{\rVert}{#1}
\DeclarePairedDelimiter{\norm}{\lVert}{\rVert}



%Dirac notation
%TODO: rewrite for variable number of arguments
\DeclarePairedDelimiterX{\braket}[2]{\langle}{\rangle}{#1 \delimsize\vert #2}
\DeclarePairedDelimiterX{\braketthree}[3]{\langle}{\rangle}{#1 \delimsize\vert #2 \delimsize\vert #3}

\DeclarePairedDelimiter{\bra}{\langle}{\rvert}
\DeclarePairedDelimiter{\ket}{\lvert}{\rangle}




%macros

%general

%divide, not divide
\newcommand*{\divides}{\mid}
\newcommand*{\ndivides}{\nmid}
%vector, i.e. mathbf
%https://tex.stackexchange.com/a/45746/26707
\newcommand*{\V}[1]{{\ensuremath{\symbf{#1}}}}
%closure
\newcommand*{\cl}[1]{\overline{#1}}
%conjugate
\newcommand*{\conj}[1]{\overline{#1}}
%set complement
\newcommand*{\stcomp}[1]{\overline{#1}}
\newcommand*{\compose}{\circ}
\newcommand*{\nto}{\nrightarrow}
\newcommand*{\p}{\partial}
%embed
\newcommand*{\embed}{\hookrightarrow}
%surjection
\newcommand*{\surj}{\twoheadrightarrow}
%power set
\newcommand*{\powerset}{\mathcal{P}}

%matrix
\newcommand*{\matrixring}{\mathcal{M}}

%groups
\newcommand*{\normal}{\trianglelefteq}
%rings
\newcommand*{\ideal}{\trianglelefteq}

%fields
\renewcommand*{\C}{{\mathbb{C}}}
\newcommand*{\R}{{\mathbb{R}}}
\newcommand*{\Q}{{\mathbb{Q}}}
\newcommand*{\Z}{{\mathbb{Z}}}
\newcommand*{\N}{{\mathbb{N}}}
\newcommand*{\F}{{\mathbb{F}}}
%not really but I think this belongs here
\newcommand*{\A}{{\mathbb{A}}}

%asymptotic
\newcommand*{\bigO}{O}
\newcommand*{\smallo}{o}

%probability
\newcommand*{\prob}{\mathbb{P}}
\newcommand*{\E}{\mathbb{E}}

%vector calculus
\newcommand*{\gradient}{\V \nabla}
\newcommand*{\divergence}{\gradient \cdot}
\newcommand*{\curl}{\gradient \cdot}

%logic
\newcommand*{\yields}{\vdash}
\newcommand*{\nyields}{\nvdash}

%differential geometry
\renewcommand*{\H}{\mathbb{H}}
\newcommand*{\transversal}{\pitchfork}
\renewcommand{\d}{\mathrm{d}} % exterior derivative

%number theory
\newcommand*{\legendre}[2]{\genfrac{(}{)}{}{}{#1}{#2}}%Legendre symbol

%algebraic geometry
\DeclareMathOperator{\Spec}{Spec}
\DeclareMathOperator{\Proj}{Proj}

\begin{document}

\begin{titlepage}
  \begin{center}
    \includegraphics[width=0.6\textwidth]{logo.jpg}\par
    \vspace{1cm}
    {\scshape\huge Mathamatics Tripos \par}
    \vspace{2cm}
    {\huge Part \npart \par}
    \vspace{0.6cm}
    {\Huge \bfseries \ntitle \par}
    \vspace{1.2cm}
    {\Large\nterm, \nyear \par}
    \vspace{2cm}
    
    {\large \emph{Lectures by } \par}
    \vspace{0.2cm}
    {\Large \scshape \nlecturer}
    
    \vspace{0.5cm}
    {\large \emph{Notes by }\par}
    \vspace{0.2cm}
    {\Large \scshape \href{mailto:\nauthoremail}{\nauthor}}
 \end{center}
\end{titlepage}

\tableofcontents

\section{Toric variety}

Roughly speaking, algebraic geometry studies the duality between polynomial equations and varieties. Let \(S = \C[x_1, \dots, x_n]\). Let \(I \subseteq S\) be an ideal generated by \(f_1, \dots, f_s\). We have polynomials of different flavours:

\subparagraph{monomial ideal}

A monomial ideal is an ideal \(I \subseteq S\) generated by monomials. The corresponding vanishing loci \(V(I)\) is a union of coordinate subspaces of \(\C^n\).

\begin{eg}\leavevmode
  \begin{itemize}
  \item \(x_1 = 0\) in \(S\), \(x_1x_2 = 0\) in \(\C[x_1, x_2]\) which is the union of two axes in \(\C^2\) so reducible.
  \item \(V(x_1^2)\) is the axis with ``additional structure'' as \((x_1^2)\) is not reduced.
  \end{itemize}
\end{eg}

Monomial ideals are thus the simplest kind of varieties and can be studied in a completely combinatorial way.

\paragraph{trinomial ideals}

A trinomial is a polynomial with at most three nonzero distinct monomial terms. Let \(I \subseteq S\) be an ideal generated by trinomials.

\begin{lemma}
  Any affine variety is the vanishing locus of a trinomial ideal.
\end{lemma}

\begin{remark}
  To be more precise, \(S/I\) as a ring is isomorphic to \(\C[x_1, \dots, C_N]/\mathfrak a\) where \(\mathfrak a \subseteq \C[x_1, \dots, x_n]\) is generated by trinomials.
\end{remark}

In other words, there is no way to distinguish a trinomial ideal and a variety.

\begin{proof}[Sketch]
  There is a trick to prove the lemma. Let \(I \subseteq S\) be an ideal. Let \(a_1 x^{\alpha_1} + \dots + a_\ell x^{\alpha_\ell} = 0\) be an equation determined by \(I\). For each such equation, introduce \((\ell - 3)\) new variables \(\{z_i\}_{i = 1}^{\ell - 3}\). Replace the equation with
  \begin{align*}
    a_1 x^{\alpha_1} + a_2 x^{\alpha_2} + z_1
    &= -z_1 + z_2 + a_3 x^{\alpha_3} \\
    &= -z_2 + z_3 + a_4 x^{\alpha_4} \\
    &= \cdots \\
    &= -z_{\ell - 3} + a_{\ell - 1} x^{\alpha_{\ell - 1}} + a_\ell x^{\alpha_\ell}
  \end{align*}
  Repeat for other equations.
\end{proof}

\paragraph{binomial ideals}

Let \(I \subseteq \C[x_1, \dots, x_n] = S\) be an ideal generated by binomials. Assume that \(I\) is prime (so \(V(I)\) is irreducible and reduced, therefore integral). Let \(X = V(I)\), say \(\dim X = m\). Then we have
\begin{itemize}
\item \(X\) contains \((\C^*)^m\) as a Zariski dense open subset.
\item The multiplication map \((\C^*)^m \times (\C^*)^m \to (\C^*)^m\) extends to a map \(X \times (\C^*)^m \to X\).
\end{itemize}

\begin{eg}\leavevmode
  \begin{itemize}
  \item \(\C^n\) generated by \(I = (0)\). It contains \((\C^*)^n\) as a dense open subset and we have multiplication
    \begin{align*}
      \C^n \times (\C^*)^n &\to \C^n \\
      ((z_1, \dots, z_n), (t_1, \dots, t_n)) &\mapsto (z_1t_1, \dots, z_nt_n)
    \end{align*}
  \item \(V(x_1 - x_2) \subseteq \C^3\) is isomorphic to \(\C^2\). \(C^2\) contains a copy of \((\C^*)^2\), given by \(\{(t, t, t'): t, t' \in \C^*\}\).
  \item \(V(x_1^2 - x_2) \subseteq \C^3\) contains a copy of \((\C^*)^2\), given by \(\{(t, t^2, t'): t, t' \in \C\}\).
  \end{itemize}
\end{eg}

Note that the key fact here is that a monomial is a homomorphism \((\C^*)^n \to \C^*\) so the ideal \(I\) encoded the intersection of translates of kernels of homomorphisms.

\begin{definition}[algebraic torus]\index{algebraic torus}
  An \emph{algebraic torus} (over \(\C\)) of rank \(1\) is the group variety \(\mathbb G_m = \C^*\). A rank \(n\) algebraic torus is \(\mathbb G_m^n = (\C^*)^n\).
\end{definition}

\begin{remark}
  \(\mathbb G_m^n\) is a group variety as it is the loci \(V(x_1 \cdots x_{n + 1} - 1)\) and the group operations are all morphisms of varieties.
\end{remark}






\iffalse
Multi-index notation: if \(\alpha = (\alpha_1, \dots, \alpha_n) \in \N^n\) then
\[
  X^\alpha = x_1^{\alpha_1} \cdots x_n^{\alpha_n}.
\]
\fi

\printindex
\end{document}
