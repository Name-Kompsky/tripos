\documentclass[a4paper]{article}

\def\npart{III}

\def\ntitle{3 Manifolds}
\def\nlecturer{S.\ Rasmussen}

\def\nterm{Lent}
\def\nyear{2019}

\ifx \nauthor\undefined
  \def\nauthor{Qiangru Kuang}
\else
\fi

\ifx \ntitle\undefined
  \def\ntitle{Template}
\else
\fi

\ifx \nauthoremail\undefined
  \def\nauthoremail{qk206@cam.ac.uk}
\else
\fi

\ifx \ndate\undefined
  \def\ndate{\today}
\else
\fi

\title{\ntitle}
\author{\nauthor}
\date{\ndate}

%\usepackage{microtype}
\usepackage{mathtools}
\usepackage{amsthm}
\usepackage{stmaryrd}%symbols used so far: \mapsfrom
\usepackage{empheq}
\usepackage{amssymb}
\let\mathbbalt\mathbb
\let\pitchforkold\pitchfork
\usepackage{unicode-math}
\let\mathbb\mathbbalt%reset to original \mathbb
\let\pitchfork\pitchforkold

\usepackage{imakeidx}
\makeindex[intoc]

%to address the problem that Latin modern doesn't have unicode support for setminus
%https://tex.stackexchange.com/a/55205/26707
\AtBeginDocument{\renewcommand*{\setminus}{\mathbin{\backslash}}}
\AtBeginDocument{\renewcommand*{\models}{\vDash}}%for \vDash is same size as \vdash but orginal \models is larger
\AtBeginDocument{\let\Re\relax}
\AtBeginDocument{\let\Im\relax}
\AtBeginDocument{\DeclareMathOperator{\Re}{Re}}
\AtBeginDocument{\DeclareMathOperator{\Im}{Im}}
\AtBeginDocument{\let\div\relax}
\AtBeginDocument{\DeclareMathOperator{\div}{div}}

\usepackage{tikz}
\usetikzlibrary{automata,positioning}
\usepackage{pgfplots}
%some preset styles
\pgfplotsset{compat=1.15}
\pgfplotsset{centre/.append style={axis x line=middle, axis y line=middle, xlabel={$x$}, ylabel={$y$}, axis equal}}
\usepackage{tikz-cd}
\usepackage{graphicx}
\usepackage{newunicodechar}

\usepackage{fancyhdr}

\fancypagestyle{mypagestyle}{
    \fancyhf{}
    \lhead{\emph{\nouppercase{\leftmark}}}
    \rhead{}
    \cfoot{\thepage}
}
\pagestyle{mypagestyle}

\usepackage{titlesec}
\newcommand{\sectionbreak}{\clearpage} % clear page after each section
\usepackage[perpage]{footmisc}
\usepackage{blindtext}

%\reallywidehat
%https://tex.stackexchange.com/a/101136/26707
\usepackage{scalerel,stackengine}
\stackMath
\newcommand\reallywidehat[1]{%
\savestack{\tmpbox}{\stretchto{%
  \scaleto{%
    \scalerel*[\widthof{\ensuremath{#1}}]{\kern-.6pt\bigwedge\kern-.6pt}%
    {\rule[-\textheight/2]{1ex}{\textheight}}%WIDTH-LIMITED BIG WEDGE
  }{\textheight}% 
}{0.5ex}}%
\stackon[1pt]{#1}{\tmpbox}%
}

%\usepackage{braket}
\usepackage{thmtools}%restate theorem
\usepackage{hyperref}

% https://en.wikibooks.org/wiki/LaTeX/Hyperlinks
\hypersetup{
    %bookmarks=true,
    unicode=true,
    pdftitle={\ntitle},
    pdfauthor={\nauthor},
    pdfsubject={Mathematics},
    pdfcreator={\nauthor},
    pdfproducer={\nauthor},
    pdfkeywords={math maths \ntitle},
    colorlinks=true,
    linkcolor={red!50!black},
    citecolor={blue!50!black},
    urlcolor={blue!80!black}
}

\usepackage{cleveref}



% TODO: mdframed often gives bad breaks that cause empty lines. Would like to switch to tcolorbox.
% The current workaround is to set innerbottommargin=0pt.

%\usepackage[theorems]{tcolorbox}





\usepackage[framemethod=tikz]{mdframed}
\mdfdefinestyle{leftbar}{
  %nobreak=true, %dirty hack
  linewidth=1.5pt,
  linecolor=gray,
  hidealllines=true,
  leftline=true,
  leftmargin=0pt,
  innerleftmargin=5pt,
  innerrightmargin=10pt,
  innertopmargin=-5pt,
  % innerbottommargin=5pt, % original
  innerbottommargin=0pt, % temporary hack 
}
%\newmdtheoremenv[style=leftbar]{theorem}{Theorem}[section]
%\newmdtheoremenv[style=leftbar]{proposition}[theorem]{proposition}
%\newmdtheoremenv[style=leftbar]{lemma}[theorem]{Lemma}
%\newmdtheoremenv[style=leftbar]{corollary}[theorem]{corollary}

\newtheorem{theorem}{Theorem}[section]
\newtheorem{proposition}[theorem]{Proposition}
\newtheorem{lemma}[theorem]{Lemma}
\newtheorem{corollary}[theorem]{Corollary}
\newtheorem{axiom}[theorem]{Axiom}
\newtheorem*{axiom*}{Axiom}

\surroundwithmdframed[style=leftbar]{theorem}
\surroundwithmdframed[style=leftbar]{proposition}
\surroundwithmdframed[style=leftbar]{lemma}
\surroundwithmdframed[style=leftbar]{corollary}
\surroundwithmdframed[style=leftbar]{axiom}
\surroundwithmdframed[style=leftbar]{axiom*}

\theoremstyle{definition}

\newtheorem*{definition}{Definition}
\surroundwithmdframed[style=leftbar]{definition}

\newtheorem*{slogan}{Slogan}
\newtheorem*{eg}{Example}
\newtheorem*{ex}{Exercise}
\newtheorem*{remark}{Remark}
\newtheorem*{notation}{Notation}
\newtheorem*{convention}{Convention}
\newtheorem*{assumption}{Assumption}
\newtheorem*{question}{Question}
\newtheorem*{answer}{Answer}
\newtheorem*{note}{Note}
\newtheorem*{application}{Application}

%operator macros

%basic
\DeclareMathOperator{\lcm}{lcm}

%matrix
\DeclareMathOperator{\tr}{tr}
\DeclareMathOperator{\Tr}{Tr}
\DeclareMathOperator{\adj}{adj}

%algebra
\DeclareMathOperator{\Hom}{Hom}
\DeclareMathOperator{\End}{End}
\DeclareMathOperator{\id}{id}
\DeclareMathOperator{\im}{im}
\DeclareMathOperator{\coker}{coker}
\DeclarePairedDelimiter{\generation}{\langle}{\rangle}

%groups
\DeclareMathOperator{\sym}{Sym}
\DeclareMathOperator{\sgn}{sgn}
\DeclareMathOperator{\inn}{Inn}
\DeclareMathOperator{\aut}{Aut}
\DeclareMathOperator{\GL}{GL}
\DeclareMathOperator{\SL}{SL}
\DeclareMathOperator{\PGL}{PGL}
\DeclareMathOperator{\PSL}{PSL}
\DeclareMathOperator{\SU}{SU}
\DeclareMathOperator{\UU}{U}
\DeclareMathOperator{\SO}{SO}
\DeclareMathOperator{\OO}{O}
\DeclareMathOperator{\PSU}{PSU}
\DeclareMathOperator{\Sp}{Sp}


%hyperbolic
\DeclareMathOperator{\sech}{sech}

%field, galois heory
\DeclareMathOperator{\ch}{ch}
\DeclareMathOperator{\gal}{Gal}
\DeclareMathOperator{\emb}{Emb}



%ceiling and floor
%https://tex.stackexchange.com/a/118217/26707
\DeclarePairedDelimiter\ceil{\lceil}{\rceil}
\DeclarePairedDelimiter\floor{\lfloor}{\rfloor}


\DeclarePairedDelimiter{\innerproduct}{\langle}{\rangle}

%\DeclarePairedDelimiterX{\norm}[1]{\lVert}{\rVert}{#1}
\DeclarePairedDelimiter{\norm}{\lVert}{\rVert}



%Dirac notation
%TODO: rewrite for variable number of arguments
\DeclarePairedDelimiterX{\braket}[2]{\langle}{\rangle}{#1 \delimsize\vert #2}
\DeclarePairedDelimiterX{\braketthree}[3]{\langle}{\rangle}{#1 \delimsize\vert #2 \delimsize\vert #3}

\DeclarePairedDelimiter{\bra}{\langle}{\rvert}
\DeclarePairedDelimiter{\ket}{\lvert}{\rangle}




%macros

%general

%divide, not divide
\newcommand*{\divides}{\mid}
\newcommand*{\ndivides}{\nmid}
%vector, i.e. mathbf
%https://tex.stackexchange.com/a/45746/26707
\newcommand*{\V}[1]{{\ensuremath{\symbf{#1}}}}
%closure
\newcommand*{\cl}[1]{\overline{#1}}
%conjugate
\newcommand*{\conj}[1]{\overline{#1}}
%set complement
\newcommand*{\stcomp}[1]{\overline{#1}}
\newcommand*{\compose}{\circ}
\newcommand*{\nto}{\nrightarrow}
\newcommand*{\p}{\partial}
%embed
\newcommand*{\embed}{\hookrightarrow}
%surjection
\newcommand*{\surj}{\twoheadrightarrow}
%power set
\newcommand*{\powerset}{\mathcal{P}}

%matrix
\newcommand*{\matrixring}{\mathcal{M}}

%groups
\newcommand*{\normal}{\trianglelefteq}
%rings
\newcommand*{\ideal}{\trianglelefteq}

%fields
\renewcommand*{\C}{{\mathbb{C}}}
\newcommand*{\R}{{\mathbb{R}}}
\newcommand*{\Q}{{\mathbb{Q}}}
\newcommand*{\Z}{{\mathbb{Z}}}
\newcommand*{\N}{{\mathbb{N}}}
\newcommand*{\F}{{\mathbb{F}}}
%not really but I think this belongs here
\newcommand*{\A}{{\mathbb{A}}}

%asymptotic
\newcommand*{\bigO}{O}
\newcommand*{\smallo}{o}

%probability
\newcommand*{\prob}{\mathbb{P}}
\newcommand*{\E}{\mathbb{E}}

%vector calculus
\newcommand*{\gradient}{\V \nabla}
\newcommand*{\divergence}{\gradient \cdot}
\newcommand*{\curl}{\gradient \cdot}

%logic
\newcommand*{\yields}{\vdash}
\newcommand*{\nyields}{\nvdash}

%differential geometry
\renewcommand*{\H}{\mathbb{H}}
\newcommand*{\transversal}{\pitchfork}
\renewcommand{\d}{\mathrm{d}} % exterior derivative

%number theory
\newcommand*{\legendre}[2]{\genfrac{(}{)}{}{}{#1}{#2}}%Legendre symbol

%algebraic geometry
\DeclareMathOperator{\Spec}{Spec}
\DeclareMathOperator{\Proj}{Proj}

\renewcommand{\boundary}{\partial}
\newcommand{\interior}{\ocirc}
\renewcommand{\P}{{\mathbb P}}
\newcommand{\immerse}{\looparrowright}

\begin{document}

\begin{titlepage}
  \begin{center}
    \includegraphics[width=0.6\textwidth]{logo.jpg}\par
    \vspace{1cm}
    {\scshape\huge Mathamatics Tripos \par}
    \vspace{2cm}
    {\huge Part \npart \par}
    \vspace{0.6cm}
    {\Huge \bfseries \ntitle \par}
    \vspace{1.2cm}
    {\Large\nterm, \nyear \par}
    \vspace{2cm}
    
    {\large \emph{Lectures by } \par}
    \vspace{0.2cm}
    {\Large \scshape \nlecturer}
    
    \vspace{0.5cm}
    {\large \emph{Notes by }\par}
    \vspace{0.2cm}
    {\Large \scshape \href{mailto:\nauthoremail}{\nauthor}}
 \end{center}
\end{titlepage}

\tableofcontents

\setcounter{section}{-1}

\section{Why 3?}

\subsection{Motivation}

\paragraph{Poincare conjecture (1904)}

Question: how can we distinguish \(S^3\) fom other 3-manifolds? The strategy is to find an invariant that distinguishes \(S^3\). The frst guess is homology but

\begin{theorem}[Poincare]
  There exists a closed oriented 3-manifold \(P\) with \(H_*(P) \simeq H_*(S^3)\) but with \(P \ncong S^3\).
\end{theorem}

\begin{notation}
  We use \(\cong\) to denote homeomorphism and \(\simeq\) to denote isomorphism.
\end{notation}

This is proven in the following way: first invent the fundamental group \(\pi_1\), then construct \(P\), which is now known as (-1)-Dehn surgery on left-handed trefoil knot \(K_T \subseteq S^3\). Finally show that \(|\pi_1(P)| = 120, |\pi_1(S^3)| = 1\) and \(H_*(P) \simeq H_*(S^3)\).

\subsection{Homotopy}

\paragraph{Review of homotopy theory}

homotopy, fundamental groups and higher homotopy groups, homotopy equivalence, weak homotopy equivalence

\paragraph{Homotopy vs.\ homology}

Let \(X\) and \(Y\) be path-connected topological spaces.

\begin{theorem}[Hurewicz]\leavevmode
  \begin{enumerate}
  \item \(H_1(X, \Z) \simeq \pi_1(X)/[\pi_1(X), \pi_1(X)]\).
  \item If \(\pi_i(X) = 1\) for \(i = \{1, \dots, n\}\) then
    \begin{align*}
      H_i(X) &= 0 \text{ for } i \leq n, i \neq 0 \\
      H_{n + 1} &\simeq \pi_{n + 1}(X)
    \end{align*}
  \end{enumerate}
\end{theorem}

\begin{theorem}[Whitehead]
  If \(X, Y\) are CW complexes. Then a weak homotopy equivalence of \(X\) and \(Y\) is also a homotopy equivalence.
\end{theorem}

\begin{theorem}[Whitehead-homology variant]
  Suppose \(X, Y\) are simply-connected CW complexes. If the induced homomorphisms \(f_*: H_k(X; \Z) \to H_k(Y; \Z)\) are isomorphisms for all \(k \leq \dim X\) then \(f: X \to Y \) is a homotopy equivalence.
\end{theorem}

\begin{theorem}
  Any homotopy equivalence \(f: X \to Y\) induces isomorphisms on homology, cohomology, cohomology ring structure (for any coefficients).
\end{theorem}

\subsection{*Simplifications in higher dimension}

Let \(\mathcal C\) be the smooth category when \(n \geq 5\) and topological category \(n \geq 4\).

\begin{theorem}[Whitney trick]
  Suppose \(\dim X = n\) where \(n \geq 4\) and \(P, Q \subseteq X\) are \(\mathcal \C\)-embedded submanifolds and \(\dim P + \dim Q = \dim X\). Then \(P, Q\) can be locally \(\mathcal C\)-isotoped so that the geometric intersection number equal to the absolute value of algebraic intersection of \(P, Q\). Note that algebraic intersection number is signed while teh geometric counterpart is not.
\end{theorem}

\begin{convention}
  When we say topological embeddings we always mean locally flat embeddings, which will be defined later in the course.
\end{convention}

\begin{definition}[\(h\)-cobordism]
  Let \(W\) with \(\boundary W = X_1 \amalg X_2\) be a cobordism from \(X_1\) to \(X_2\). \(W\) is an \emph{\(h\)-cobordism} if the embeddings \(X_i \embed W\) are homotopy equivalences.
\end{definition}

\begin{convention}
  All manifolds are compact connected and oriented unless otherwise stated.
\end{convention}

\begin{theorem}[\(h\)-cobordism]
  Suppose \(\dim X_i = n, \dim W = n + 1\), \(W\) is a \(h\)-cobordism from \(X_1\) to \(X_2\). If \(\pi_1(X_i) = \pi_1(W) = 1\) and \(n \geq 4\) then \(W\) is \(\mathcal C\)-isomorphic to \(X_1 \times [0, 1]\).
\end{theorem}

\subsection{Generalised Poincare conjecture}

Poincare conjecture: if \(S\) is compact oriented \(3\)-manifold homotopy equivalent to \(S^n\), then does \(S \cong S^n\)?

Generalised Poincare conjecture: if \(S\) is compact oriented \(n\)-manifold homotopy equivalent to \(S^n\), then does \(S \cong S^n\)?

It turns out for \(n \geq 4\), the generalised Poincare conjecture is a corollary of \(h\)-cobordism theorem. Sketch of proof for \(n \geq 5\): suppose \(S\) is homotopy equivalent to \(S^n\), Then \(\pi_*(S) \simeq \pi_*(S^n), H_*(S) \simeq H_*(S^n)\). Delete two balls from \(S\) to obtain \(W \cong S \setminus \interior B_1^n \amalg \interior B_2^n\). Claim that \(W\) is a \(h\)-cobordism: apply Mayer-Vietoris with \(A = W, B = B_1^n \amalg B_2^n\). Then \(A \cap B = S^{n - 1} \amalg S^{n - 1} =_{\text{htp}} W \amalg \{0, 1\}, A \cup B = S, A \amalg B = W \).

\[
  \begin{tikzcd}
    H_n(S^{n - 1} \amalg S^{n - 1}) \ar[r] & H_n(W \amalg \{0, 1\}) \ar[r] & H_n(S) \ar[dll, out=0, in=180] \\
    H_{n - 1}(S^{n - 1} \amalg S^{n - 1}) \ar[r] & H_{n - 1}(W \amalg \{0, 1\}) \ar[r] & H_{n - 1}(S)
  \end{tikzcd}
\]

The first term vanishes because of dimension, the second term vanishes because \(W\) is not closed. By homotopy equivalence we get
\[
  \begin{tikzcd}
    0 \ar[r] & \Z \ar[r] & \Z \oplus \Z \ar[r] & H_{n - 1}(W \amalg \{0, 1\}) \ar[r] & 0
  \end{tikzcd}
\]
We can compute that \(H_{n - 1}(W \amalg \{0, 1\}) \simeq \Z\). It is an exercise to show that there is an induced isomorphism on homology \(H_k(S^n_i) \to H_k(W)\) for each \(k\). Moreover \(\pi_1(W) = 1\) so \(S^n_i \to W\) are homotopy equivalent.

Therefore \(W \cong S^{n - 1} \times [0, 1]\) So \(S \cong B_1^n \cup W \cup B_2^n\). By Alexander trick map on a \(S^{n - 1}\) can be extended \emph{topologically} to a map on \(B^n\) with \(\boundary B^n = S^n\). Extends this homeomorphism over the two balls.

Note that this only applies to topological category and smooth generalised Poincare conjecture is still open in \(n \geq 4\).

\subsection{Why not higher than 5?}

Moral: homotopy-theoretic techniques can be used to answer most/many questions about topology or smooth structures in dimension \(\geq 5\).

\section{Lecture 2: Why 3-manifolds? + Embeddings/Knots}

\subsection*{Active research areas}

\begin{enumerate}
\item An interaction with 4-dimensional manifolds (smooth/symplectic/complex structures)
\begin{enumerate}
\item Dimension reduction reduces 4-dimensional invariant to 3-dimensional ones (that are fancier ``categorified'') and maps induced by cobordisms.
\item symplectic form \(\omega\) on \(X^4\) \(\implies\) \emph{contact structure} \(\xi\) on \(Y = \b X\).
\item Stein structure (complex/symplectic structure) on \(X\) \(\implies\) Stein-fillable contact structure.
\item Normal complex structure sin \((X, 0)\) is a real cone over \(Y = \)Linkm(X, 0.
\end{enumerate}

\item Geometric group theory: fundamental groups, especially of 3-manifolds:

prime, atoroidal non lens space 3 manifolds \(\iff\) fundmental groups of such 3-manifolds.

\item 2-dimensional structure
  \begin{enumerate}
  \item contact stucture: \(\xi\) everywhere nonintegrable \(2\)-lane field. ``tight'' contact structure classification
  \item minimal genus representatives of embedded surfaces, or knot genus. This is better understood. Thurston norm. The 4-dimensional analogue is still open.
  \item Foliations. Taut folations classification. Seifert fibered
  \end{enumerate}

\item 1-dimensional structure: knots and links
  \begin{enumerate}
  \item embedddings \(\amalg_i S^1_i \embed S^3\). Every 3-manifold can be realised as \emph{Dehn surgery} on  a link \(L \embed S^3\). Thus the theory of knot theory is richer that of 3-manifold. We study 3-manifolds via knot invariants (WIlten-Reshetikhin-Turaev invariant).
  \item Relations to other areas
    \begin{enumerate}
    \item Chern-Simons knot invarints: \(K \subseteq S^3\) \(\iff\) Gromov-Witten invariants on \(O(-1) \underbrace{\oplus}_{\C\P^1} O(-1)\).
    \item Homfly homology of \(n\)str braids \(\iff\) DC sheaves on \(\operatorname{HIlb}^n(\C)\).
    \item Khovanov homology of links in \(S^3\) \(\iff\) DC sheaves on other spaces.
    \end{enumerate}
  \end{enumerate}
\end{enumerate}

\subsection{Course themes}

\begin{enumerate}
\item Decompositions/Constructions of 3-manifolds.
  \begin{enumerate}
  \item surface decompositions/constructures
    \begin{enumerate}
    \item prime decomposition --- cut along essential \(S^2\)
    \item JSJ decomposition --- cut along essential \(T\).
    \item Mapping tori \(\iff\) surface fibrations.
    \end{enumerate}
  \item quotient spaces
    \begin{enumerate}
    \item Hyperbolic quotients
    \item quotients of \(S^7\). Seifert fibration
    \item Morse theoretic
      \begin{enumerate}
      \item handle decomposition
      \item Heegaard splittings/diagrams
      \end{enumerate}
    \item Dehn surgery on links
    \end{enumerate}
  \end{enumerate}
\item Structure + Invariants for 3-manifolds
  \begin{enumerate}
  \item Knots \& links
    \begin{enumerate}
    \item complement \(S^3 \setminus K\)
    \item \(\pi_1(S^3 \setminus K)\)
    \item Alexander polynomials + Turaev torsion
    \end{enumerate}
  \item Essential/incompressible embedded surfaces, Thurston norm
  \item Foliations
  \end{enumerate}
\end{enumerate}

\section{Embeddings}

\begin{definition}[link]\index{link}
  A \emph{link} is an embedding \(L = \amalg_i S^1_i \embed S^3\) considered up to isotopy. This embedding is either smooth or topoogical and locally flat. These two notions are equivalent.
\end{definition}

Let \(X\) and \(Y\) be topological manifolds.

\begin{definition}[topological embedding]\index{topological embedding}
  A \emph{topological embedding} \(X \embed Y\) is a map \(X \embed Y\) which is a homeomorphism onto its image.
\end{definition}

\begin{definition}[immersion]\index{immersion}
  If \(X\) and \(Y\) are also smooth then a map \(f: X \to Y\) is an \emph{immersion} if \(d_xf: T_xX \to T_{f(x)}Y\) is injective for all \(x \in X\).
\end{definition}

As a consequence of inverse function theorem, any immersion is locally an embedding.

\begin{definition}[smooth embedding]\index{smooth embedding}
  A \emph{smooth embedding} is a topological embedding that is also an immersion.
\end{definition}

\begin{corollary}
  If \(X, Y\) are smooth compact then any bijective immersion is an embedding.
\end{corollary}

\begin{theorem}[Moise]\index{Moise theorem}
  There is a canonical correpondence between topological structures and smooth structures on 3-manifolds.
\end{theorem}

Thus 3-manifolds up to homeomorphism bijects to 3-manifolds up to diffeomorphism.

\begin{definition}[local flatness]\index{local flatness}
  A topologically embedded submanifold \(X \subseteq Y\) is \emph{locally flat} at \(x \in X\) if \(x\) has a neighbourhood \(x \in U \subseteq Y\) with homeomorphisms \((U \cap X, U) \cong (\R^{\dim X}, \R^{\dim Y})\).

  A \emph{locally flat embedding} is locally flat everywhere.
\end{definition}

\begin{convention}
  From now on any embedding is smooth or locally flat.
\end{convention}

\begin{definition}[regular neighbourhood]\index{regular manifolds}
  A \emph{regular neighbourhood} of an embedded submanifold \(X \subseteq Y\) is a tubular/collar neighbourhood if the embedding is smooth/topologically flat.
\end{definition}

In 3-dimensions normal bundles are trivial so a regular neighbourhood \(\nu(X)\) is just \(D^2 \times X \embed Y\) if \(\dim X = 1\) and \(D^1 \times X \embed Y\) if \(\dim Y = 2\).

In particular, neighbourhood of a not \(K \embed S^3\) is just a solid torus \(D^2 \times S^1 \embed S^3\).

\section{Lecture 3: Link diagrams \& Alexander Skein relations}

\begin{eg}
  Wild knot: not locally flat embedding
\end{eg}

\begin{definition}[isotopy]\index{isotopy}
  An \emph{isotopy} in category \(\mathcal C\) from \(f_1\) to \(f_2: X \to Y\) is a homotopy through maps of type \(\mathcal C\).
\end{definition}

The point is, all knots (including wild knot) are isotopic through non-locally flat embeddings to an unknot, and all knots are homotopic to an unknot so we want to exclude the ``bad'' homotopies where a knot can cross itself.

\subsection{Knot and link diagrams}

\begin{definition}[link]\index{link}
  A \emph{link} is an (oriented) embedding \(\iota: \coprod_i S_i^1 \embed S^3\) of (oriented circles), considered up to isotopy.
\end{definition}

\begin{definition}[link projection]
  A \emph{link projection} is an immersion \(L \immerse \Gamma \embed \R^2\), induced by
  \[
    \begin{tikzcd}
      L \ar[r, hook] \ar[d, "p|_L"] & S^3 \setminus \{x_0\} \ar[r, "\cong"] & \R^3 \ar[r, "\cong"] & \R^2 \times \R \ar[d, "p"] \\
      \Gamma \ar[rrr] & & & \R^2
    \end{tikzcd}
  \]
  such that \(x_0 \notin L\) and \(p|_L\) is an embedding except at double point singularities.
\end{definition}

This aweful looking definition is just a formalisation of a familiar concept that facilitates the study of knots:

\begin{definition}[link diagram]\index{link diagram}
  A \emph{link diagram} \(D = (\Gamma, \text{crossing} (D))\) of a link \(L \subseteq S^3\) is an embedded graph \(\Gamma \embed \R^2\) from a link projection of \(D\), together with decorations at double points to label crossings. We draw a gap in the lower strand.
\end{definition}

\begin{theorem}[Reidemeister moves]\index{Reidemeister moves}
  Let \(D_1\) and \(D_2\) be link diagrams for respective links \(L_1, L_2 \subseteq S^3\). Then \(L_1\) and \(L_2\) are isotopic if and only if \(D_1\) and \(D_2\) are related by some combination of the fuollowing moves:
\end{theorem}

It is more important to know that such moves exist than what they actually are.

\subsection{Alexander Skein relation}

To compute the alexander polynomial, you first choose an orientation for the link \(L \subseteq S^3\). However, the resulting polynomial is independent of choice of orientation for knots.

\begin{theorem}[Alexander]\index{Alexander polynomial}
  The \emph{Alexander polynomial}
  \[
    \Delta: \{\text{link diagram}\} \to \Z[t^{-1/2}, t^{1/2}]
  \]
  is specified by 2 conditions:
  \begin{enumerate}
  \item normalisation: \(\Delta(u) = 1\) where \(u\) is the unknot.
  \item Skein relation: \(\Delta(negative crossing) - \Delta(positive crossing) = \Delta(oriented resolution) (t^{-1/2} - t^{1/2})\) for all \(c \in \text{crossing}(D)\).
  \end{enumerate}
  \(\Delta(D_1) = \Delta(D_2)\) if \(D_1\) and \(D_2\) are diagrams for isotopic links.
\end{theorem}

\begin{theorem}[equivalence of Alexander polynomial]
  Later we will define an Alexander polynomial for 3-manifolds with \(b_1 > 0\). With respect to this definition,
  \[
    \Delta_{\text{link}}(L) = \Delta_{\text{3-manifold}}(S^3 \setminus L)
  \]
  for any link \(L \subseteq S^3\).
\end{theorem}













\printindex
\end{document}

% https://www.dpmms.cam.ac.uk/~sr727/2019_3manifolds