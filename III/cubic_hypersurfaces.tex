\documentclass[a4paper]{article}

\def\npart{III}

\def\ntitle{Cubic Hypersurfaces}
\def\nlecturer{D.\ Huybrechts}

\def\nterm{Lent}
\def\nyear{2020}

\ifx \nauthor\undefined
  \def\nauthor{Qiangru Kuang}
\else
\fi

\ifx \ntitle\undefined
  \def\ntitle{Template}
\else
\fi

\ifx \nauthoremail\undefined
  \def\nauthoremail{qk206@cam.ac.uk}
\else
\fi

\ifx \ndate\undefined
  \def\ndate{\today}
\else
\fi

\title{\ntitle}
\author{\nauthor}
\date{\ndate}

%\usepackage{microtype}
\usepackage{mathtools}
\usepackage{amsthm}
\usepackage{stmaryrd}%symbols used so far: \mapsfrom
\usepackage{empheq}
\usepackage{amssymb}
\let\mathbbalt\mathbb
\let\pitchforkold\pitchfork
\usepackage{unicode-math}
\let\mathbb\mathbbalt%reset to original \mathbb
\let\pitchfork\pitchforkold

\usepackage{imakeidx}
\makeindex[intoc]

%to address the problem that Latin modern doesn't have unicode support for setminus
%https://tex.stackexchange.com/a/55205/26707
\AtBeginDocument{\renewcommand*{\setminus}{\mathbin{\backslash}}}
\AtBeginDocument{\renewcommand*{\models}{\vDash}}%for \vDash is same size as \vdash but orginal \models is larger
\AtBeginDocument{\let\Re\relax}
\AtBeginDocument{\let\Im\relax}
\AtBeginDocument{\DeclareMathOperator{\Re}{Re}}
\AtBeginDocument{\DeclareMathOperator{\Im}{Im}}
\AtBeginDocument{\let\div\relax}
\AtBeginDocument{\DeclareMathOperator{\div}{div}}

\usepackage{tikz}
\usetikzlibrary{automata,positioning}
\usepackage{pgfplots}
%some preset styles
\pgfplotsset{compat=1.15}
\pgfplotsset{centre/.append style={axis x line=middle, axis y line=middle, xlabel={$x$}, ylabel={$y$}, axis equal}}
\usepackage{tikz-cd}
\usepackage{graphicx}
\usepackage{newunicodechar}

\usepackage{fancyhdr}

\fancypagestyle{mypagestyle}{
    \fancyhf{}
    \lhead{\emph{\nouppercase{\leftmark}}}
    \rhead{}
    \cfoot{\thepage}
}
\pagestyle{mypagestyle}

\usepackage{titlesec}
\newcommand{\sectionbreak}{\clearpage} % clear page after each section
\usepackage[perpage]{footmisc}
\usepackage{blindtext}

%\reallywidehat
%https://tex.stackexchange.com/a/101136/26707
\usepackage{scalerel,stackengine}
\stackMath
\newcommand\reallywidehat[1]{%
\savestack{\tmpbox}{\stretchto{%
  \scaleto{%
    \scalerel*[\widthof{\ensuremath{#1}}]{\kern-.6pt\bigwedge\kern-.6pt}%
    {\rule[-\textheight/2]{1ex}{\textheight}}%WIDTH-LIMITED BIG WEDGE
  }{\textheight}% 
}{0.5ex}}%
\stackon[1pt]{#1}{\tmpbox}%
}

%\usepackage{braket}
\usepackage{thmtools}%restate theorem
\usepackage{hyperref}

% https://en.wikibooks.org/wiki/LaTeX/Hyperlinks
\hypersetup{
    %bookmarks=true,
    unicode=true,
    pdftitle={\ntitle},
    pdfauthor={\nauthor},
    pdfsubject={Mathematics},
    pdfcreator={\nauthor},
    pdfproducer={\nauthor},
    pdfkeywords={math maths \ntitle},
    colorlinks=true,
    linkcolor={red!50!black},
    citecolor={blue!50!black},
    urlcolor={blue!80!black}
}

\usepackage{cleveref}



% TODO: mdframed often gives bad breaks that cause empty lines. Would like to switch to tcolorbox.
% The current workaround is to set innerbottommargin=0pt.

%\usepackage[theorems]{tcolorbox}





\usepackage[framemethod=tikz]{mdframed}
\mdfdefinestyle{leftbar}{
  %nobreak=true, %dirty hack
  linewidth=1.5pt,
  linecolor=gray,
  hidealllines=true,
  leftline=true,
  leftmargin=0pt,
  innerleftmargin=5pt,
  innerrightmargin=10pt,
  innertopmargin=-5pt,
  % innerbottommargin=5pt, % original
  innerbottommargin=0pt, % temporary hack 
}
%\newmdtheoremenv[style=leftbar]{theorem}{Theorem}[section]
%\newmdtheoremenv[style=leftbar]{proposition}[theorem]{proposition}
%\newmdtheoremenv[style=leftbar]{lemma}[theorem]{Lemma}
%\newmdtheoremenv[style=leftbar]{corollary}[theorem]{corollary}

\newtheorem{theorem}{Theorem}[section]
\newtheorem{proposition}[theorem]{Proposition}
\newtheorem{lemma}[theorem]{Lemma}
\newtheorem{corollary}[theorem]{Corollary}
\newtheorem{axiom}[theorem]{Axiom}
\newtheorem*{axiom*}{Axiom}

\surroundwithmdframed[style=leftbar]{theorem}
\surroundwithmdframed[style=leftbar]{proposition}
\surroundwithmdframed[style=leftbar]{lemma}
\surroundwithmdframed[style=leftbar]{corollary}
\surroundwithmdframed[style=leftbar]{axiom}
\surroundwithmdframed[style=leftbar]{axiom*}

\theoremstyle{definition}

\newtheorem*{definition}{Definition}
\surroundwithmdframed[style=leftbar]{definition}

\newtheorem*{slogan}{Slogan}
\newtheorem*{eg}{Example}
\newtheorem*{ex}{Exercise}
\newtheorem*{remark}{Remark}
\newtheorem*{notation}{Notation}
\newtheorem*{convention}{Convention}
\newtheorem*{assumption}{Assumption}
\newtheorem*{question}{Question}
\newtheorem*{answer}{Answer}
\newtheorem*{note}{Note}
\newtheorem*{application}{Application}

%operator macros

%basic
\DeclareMathOperator{\lcm}{lcm}

%matrix
\DeclareMathOperator{\tr}{tr}
\DeclareMathOperator{\Tr}{Tr}
\DeclareMathOperator{\adj}{adj}

%algebra
\DeclareMathOperator{\Hom}{Hom}
\DeclareMathOperator{\End}{End}
\DeclareMathOperator{\id}{id}
\DeclareMathOperator{\im}{im}
\DeclareMathOperator{\coker}{coker}
\DeclarePairedDelimiter{\generation}{\langle}{\rangle}

%groups
\DeclareMathOperator{\sym}{Sym}
\DeclareMathOperator{\sgn}{sgn}
\DeclareMathOperator{\inn}{Inn}
\DeclareMathOperator{\aut}{Aut}
\DeclareMathOperator{\GL}{GL}
\DeclareMathOperator{\SL}{SL}
\DeclareMathOperator{\PGL}{PGL}
\DeclareMathOperator{\PSL}{PSL}
\DeclareMathOperator{\SU}{SU}
\DeclareMathOperator{\UU}{U}
\DeclareMathOperator{\SO}{SO}
\DeclareMathOperator{\OO}{O}
\DeclareMathOperator{\PSU}{PSU}
\DeclareMathOperator{\Sp}{Sp}


%hyperbolic
\DeclareMathOperator{\sech}{sech}

%field, galois heory
\DeclareMathOperator{\ch}{ch}
\DeclareMathOperator{\gal}{Gal}
\DeclareMathOperator{\emb}{Emb}



%ceiling and floor
%https://tex.stackexchange.com/a/118217/26707
\DeclarePairedDelimiter\ceil{\lceil}{\rceil}
\DeclarePairedDelimiter\floor{\lfloor}{\rfloor}


\DeclarePairedDelimiter{\innerproduct}{\langle}{\rangle}

%\DeclarePairedDelimiterX{\norm}[1]{\lVert}{\rVert}{#1}
\DeclarePairedDelimiter{\norm}{\lVert}{\rVert}



%Dirac notation
%TODO: rewrite for variable number of arguments
\DeclarePairedDelimiterX{\braket}[2]{\langle}{\rangle}{#1 \delimsize\vert #2}
\DeclarePairedDelimiterX{\braketthree}[3]{\langle}{\rangle}{#1 \delimsize\vert #2 \delimsize\vert #3}

\DeclarePairedDelimiter{\bra}{\langle}{\rvert}
\DeclarePairedDelimiter{\ket}{\lvert}{\rangle}




%macros

%general

%divide, not divide
\newcommand*{\divides}{\mid}
\newcommand*{\ndivides}{\nmid}
%vector, i.e. mathbf
%https://tex.stackexchange.com/a/45746/26707
\newcommand*{\V}[1]{{\ensuremath{\symbf{#1}}}}
%closure
\newcommand*{\cl}[1]{\overline{#1}}
%conjugate
\newcommand*{\conj}[1]{\overline{#1}}
%set complement
\newcommand*{\stcomp}[1]{\overline{#1}}
\newcommand*{\compose}{\circ}
\newcommand*{\nto}{\nrightarrow}
\newcommand*{\p}{\partial}
%embed
\newcommand*{\embed}{\hookrightarrow}
%surjection
\newcommand*{\surj}{\twoheadrightarrow}
%power set
\newcommand*{\powerset}{\mathcal{P}}

%matrix
\newcommand*{\matrixring}{\mathcal{M}}

%groups
\newcommand*{\normal}{\trianglelefteq}
%rings
\newcommand*{\ideal}{\trianglelefteq}

%fields
\renewcommand*{\C}{{\mathbb{C}}}
\newcommand*{\R}{{\mathbb{R}}}
\newcommand*{\Q}{{\mathbb{Q}}}
\newcommand*{\Z}{{\mathbb{Z}}}
\newcommand*{\N}{{\mathbb{N}}}
\newcommand*{\F}{{\mathbb{F}}}
%not really but I think this belongs here
\newcommand*{\A}{{\mathbb{A}}}

%asymptotic
\newcommand*{\bigO}{O}
\newcommand*{\smallo}{o}

%probability
\newcommand*{\prob}{\mathbb{P}}
\newcommand*{\E}{\mathbb{E}}

%vector calculus
\newcommand*{\gradient}{\V \nabla}
\newcommand*{\divergence}{\gradient \cdot}
\newcommand*{\curl}{\gradient \cdot}

%logic
\newcommand*{\yields}{\vdash}
\newcommand*{\nyields}{\nvdash}

%differential geometry
\renewcommand*{\H}{\mathbb{H}}
\newcommand*{\transversal}{\pitchfork}
\renewcommand{\d}{\mathrm{d}} % exterior derivative

%number theory
\newcommand*{\legendre}[2]{\genfrac{(}{)}{}{}{#1}{#2}}%Legendre symbol

%algebraic geometry
\DeclareMathOperator{\Spec}{Spec}
\DeclareMathOperator{\Proj}{Proj}

\renewcommand*{\P}{\mathbb{P}}
\newcommand{\sh}[1]{\mathcal{#1}} % sheaf
\DeclareMathOperator{\Pic}{Pic} % Picard group
\newcommand{\rational}{\dashrightarrow} % rational map

\begin{document}

\begin{titlepage}
  \begin{center}
    \includegraphics[width=0.6\textwidth]{logo.jpg}\par
    \vspace{1cm}
    {\scshape\huge Mathamatics Tripos \par}
    \vspace{2cm}
    {\huge Part \npart \par}
    \vspace{0.6cm}
    {\Huge \bfseries \ntitle \par}
    \vspace{1.2cm}
    {\Large\nterm, \nyear \par}
    \vspace{2cm}
    
    {\large \emph{Lectures by } \par}
    \vspace{0.2cm}
    {\Large \scshape \nlecturer}
    
    \vspace{0.5cm}
    {\large \emph{Notes by }\par}
    \vspace{0.2cm}
    {\Large \scshape \href{mailto:\nauthoremail}{\nauthor}}
 \end{center}
\end{titlepage}

\tableofcontents

\setcounter{section}{-1}

\section{Introduction}

Algebraic geometry studies the locus \(X = V(F_j) \subseteq \P^N\) of homogeneous polynomials. For degree \(1, 2\) this is essentially linear algebra and algebraic geometry starts with degree \(3\) and one polynomial. \(X = V(F) \subseteq \P^{n + 1}\) is a \emph{cubic hypersurface}. For simplicity we assume for now \(X\) is smooth.

\begin{itemize}
\item \(n = 0\): \(X = \{x_1, x_2, x_3\} \subseteq \P^1\).
\item \(n = 1\): \(X\) is a cubic plane curve in \(\P^2\).
\item \(n = 2\): classical theory of cubic surfaces. e.g.\ 27 lines on a cubic hypersurface.
\item \(n = 3\): cubic threefold. Work of Griffiths et al, techniques from Hodge theory.
\item \(n = 4\): cubic fourfolds. Future classic.
\end{itemize}

Links to abelian varieties, K3 surfaces etc.

In this lecture we will introduce techniques, eg.\ Hodge theory, \(K(var)\), motifs, \(D^b(Coh)\) and apply them to cubics.

Introduce \(F(X)\), the Fano varieties of lines

\section{Lecture 1}

Plan
\begin{enumerate}
\item basic facts,
\item classical construction,
\item rationality,
\item existence of lines
\end{enumerate}

Let \(X = V(F) \subseteq \P^{n + 1} = \P = \{(V)\) where \(\dim V = n + 2\), \(F \in k[x_1, \dots, x_{n + 1}]_3 = S^3(V)\). Assume \(X\) is smooth. There are two exact sequences:
\begin{enumerate}
\item (co)normal bundle sequence
  \[
    \begin{tikzcd}
      0 \ar[r] & \sh I_X \ar[r] & \sh I_{\P/X} \ar[r] & N_{X/\P} \ar[r] & 0
    \end{tikzcd}
  \]
  and dually
  \[
    \begin{tikzcd}
      0 
    \end{tikzcd}
  \]
\item Euler sequence
  \[
    \begin{tikzcd}
      0 \ar[r] & \sh O \ar[r, "s"] & \sh O(1)^{\oplus n + 2} \ar[r] & \sh I_\P \ar[r] & 0
    \end{tikzcd}
  \]
  on \(\P^{n + 1}\)
\end{enumerate}

\begin{corollary}
  \(\omega_X = \Lambda^n \Omega_X = \det \Sigma_{\P/X} \otimes \sh O(3) = \sh O(-n - 2 + 3) = \sh O(1 - n)\). THis is because when we take the determinant of the dual sequence we get \(\det \Omega_\P = \sh N^\vee \otimes \det \Omega_X\)

  Thus \(\omega_X^*\) is (very) ample for \(n > 1\). This is called a \emph{Fano variety}.
\end{corollary}

\begin{corollary}
  The restriction map \(H^q(\P, \Omega_\P^p) \to H^q(X, \Omega_X^p)\) is an isomorphism for \(p + q < n\).

  \(H^q(X, \Omega_X^p \otimes \sh O(k)) = 0\) for \(k > 0, p + q > n\). This is a special case of Kodaira vanishing theorem: for \(\ch = 0\), \(H^q(X, \Omega^p \otimes \sh L) = 0\) for \(p + q > \dim X\) for \(\sh L\) ample. This is computed by taking the \(p\)th exterior power
  \[
    \begin{tikzcd}
      0 \ar[r] & \Omega_X^{p - 1} \otimes \sh O(-3) \ar[r] & \Omega_\P^p|_X \ar[r] & \Omega_X^p \ar[r] & 0
    \end{tikzcd}
  \]
  and
  \[
    \begin{tikzcd}
      0 \ar[r] & \Omega_\P^p \otimes \sh O(-3) \ar[r] & \Omega_\P^p \ar[r] & \Omega_\P^p|_X \ar[r] & 0
    \end{tikzcd}
  \]
\end{corollary}

So far everything works over arbitrary field. If \(k = \C\), we can look at the singular cohomology of the analytic space, which is a complex manifold if \(X\) is smooth, and \(H^*(X, \Z) = H^*(X^{\text{an}}, \Z)\), a finitely generated \(\Z\)-module.

Hodge decomposition: \(H^k(X, \Z) \otimes_\Z \C = H^k(X, \C) = \bigoplus_{p + q = k} H^{p, q}(X)\), \(H^{p, q}(X) = H^q(X, \Omega_X^p)\).

Lefschetz hyperplane theorem: \(H^k(\P; \Z) \to H^k(X; \Z)\) is an isomorphism for \(k < n\). For \(k \ne n\)
\[
  H^k(X; \Z) =
  \begin{cases}
    \Z & k = 0 \pmod 2 \\
    0 & k = 1 \pmod 2
  \end{cases}
\]

Exponential sequence on \(X^{\text{an}}\)
\[
  \begin{tikzcd}
    0 \ar[r] & \Z \ar[r] & \sh O \ar[r, "\exp"] & \sh O^* \ar[r] & 0
  \end{tikzcd}
\]
In the long exact sequence we have
\[
  H^1(X, \sh O^*) \cong H^2(X, \Z) \cong \Pic (X) = \Z
\]
for \(n > 2\).

\begin{corollary}
  \(\Pic (X) \cong \Z \cdot \sh O_X(1)\) for \(n > 2\).
\end{corollary}

The open question is the middle cohomology \(H^n(X, \Z)\).
\begin{enumerate}
\item \(b_n = rk H^n(X, \Z) = ?\)
\item the intersection for \(H^n(X, \Z) \times H^n(X, \Z) \to \Z\) is nondegenerate unimodular so for \(n = 1 \pmod 2\) it is given by an alternating form. For \(n = 0 \pmod 2\) it is more complicated.
\end{enumerate}

Topological Euler characteristic
\[
  e(X) = \sum_{i = 1}^{2n} (-1)^i b_i(X) =
  \begin{cases}
    n + b_n & n = 0 \pmod 2 \\
    n + 1 - b_n & n = 1 \pmod 2
  \end{cases}
\]

Computation using Chern class:
\[
  e(X) = \int_X c_n(\sh T_X) = \int_X [\frac{c(\sh T_{\P/X})}{c(\sh O(3))}]_{2n} = \int_X [\frac{(1 + h)^{n + 2}}{1 + 3h}]_{2n}
\]
where \(h - c_1(\sh O(1)), c(\sh O(1)) = 1 + h\). So
\[
  e(X) = \int_\P [3h \frac{(1 + h)^{n + 2}}{1 + 3h}]_{2n + 2} = \frac{1}{3} ((-2)^{n + 2} + 3n + 5).
\]

To put it in a generating function,
\[
  \sum_{n = 0}^\infty e(X_n) z^{n + 1} = \frac{3z}{(1 - z)^2 (1 + 2z)}
\]

\(n = 0, 1, 2, 3, 4\), \(e = 3, 0, 9, -6, 27\)

\begin{definition}
  A (smooth projective) variety \(X\) is \emph{unirational} (of degree \(d\)) if there exists a rational map \(\P^n \rational X\) generically of degree \(d\), where \(n = \dim X\). It is rational if it is unirational of degree \(1\), if and only if \(\P^n \supseteq U \subseteq X\) one dense.
\end{definition}

Question: are cubic hypersurfaces rational/unirational?

Let \(X \subseteq \P\) be a smooth cubic. For \(x \in X\), let \(T_xX\) be the tangent space. We get a rational map
\begin{align*}
  \P(T_xX) &\rational X \\
  v &\mapsto y
\end{align*}
given as follow: \(x\) and \(v\) gives a line \(L_v \subseteq \P\) through \(x\). Then \(y\) is defined as the third intersection of \(L_v\) with \(X\). This is not defined if \(L_v \subseteq X\). Note that if \(L_v \subseteq X\) for all \(v\) then \(\P^n = \bigcup L_v \subseteq X\), contradicting smoothness.

The map is generically injective: if \(xy = L_v\) then \(v\) is determined.

Assume \(L [ X\) is a line. Then \(\P(\sh I_{X/L}) \to L \cong \P^1\) so \(\P(\sh I_X/L)\) is rational. \(\P(I_{X/L}) \rational X\) generically two to one.

\begin{eg}
  
\end{eg}

\begin{proposition}
  Let \(X\) be a smooth cubic of dimension \(n \geq 3\) and \(x \in X\). Then exists \(x \in L \subseteq X\) a line.
\end{proposition}

\begin{corollary}
  Every smooth cubic sontains a line is unirational in degree \(2\).
\end{corollary}

\begin{proof}
  Let \(x = [a_0: \cdots a_{n + 1}], X = V(F), F \in k[x_0, \dots, x_{n + 1}]_3\). Let
  \[
    \mathbb T_xX = V(\sumx_i \p_i F(a)) \cong \P^n \subseteq \P^{n + 1}
  \]
  be the projective tangent space and
  \[
    P_xX = V(\sum a_i \p_i F) \subseteq \P^{n + 1}
  \]
  quadric be the polar of the hypersurface. Choose a generic hyperplane \(\P^n \subseteq \P^{n + 1}\) such that \(x \notin \P^n\). As \(n \geq 3\), the intersection \(\P^n \cap \mathbb T_x \cap P_xX \cap X \ne \emptyset\) so pick a point \(y\). Let \(y\) be the line through \(x\) and \(y\). Claim \(L \subseteq X\): wlog \(x = [1:0: \cdots :0], y = [0:1: 0: \cdots:0], x \notin \P^n = V(x_0), P_x = V(\p_0F)\). As \(y \in P_x\), have \(\p_0F(y) = 0\). As \(x \in \mathbb T_x\), have \(\p_0F(x) = 0\). Thus \(F|_L\) has zeros at \(x\) and \(y\) both with multiplicity \(2\). As \(F\) is cubic, \(F|_L = 0\) so \(L \subseteq X\).
\end{proof}

Later: \(F(X) = \{L \subseteq X ?\}\)

\paragraph{Projection from a linear subspace}

Let \(P = \P(W) \subseteq \P = \P(V)\) where \(\dim V = n + 2, \dim W = k\). Pick \(U \subseteq V\) such that \(U \oplus W = V\), so \(\dim U = n + 2 - k\). Choose \(x \notin P\). Then \(\overline \{xP\}\), the subspace generated by \(P\) and \(x\), is isomorphic to \(\P^k\) so intersects \(\P(U)\) at a single point \(y\). We thus get an induced map \(\P \setminus P \to \P(U)\), thus a rational map \(\varphi: \P \rational \P(U)\). \(\varphi\) is the map induced by the linear system \(|\sh I_P \otimes \sh O(1)| \subseteq |\sh O(1)|\). Consider the SES
\[
  \begin{tikzcd}
    0 \ar[r] & \sh I_P \ar[r] & \sh O_\P \ar[r] & \sh O_P \ar[r] & 0
  \end{tikzcd}
\]
and
\[
  \begin{tikzcd}
    0 \ar[r] & H^0(I_P \otimes O(1)) \ar[r] & H^0(\sh O_\{(1)) \ar[r] & H^0(\sh O_P(1)) \\
    & U^* & V^* & W^*
  \end{tikzcd}
\]
Consider the blow up and a morphism \(\varphi: \P \to \P(U)\)...

Claim \(N_{P/\P} \cong V/W \otimes \sh O(1)\).

\begin{ex}
  \(\sh F \cong \sh O(1) \oplus (W^* \otimes \sh O)\).
\end{ex}

Let \(X \subseteq \P\) be a smooth hypersurface and assume \(P \subseteq X\). Blowup
\[
  \begin{tikzcd}
    & E \ar[d] \\
    \P(N_{P/X}) \ar[r, "\cong"] & E_x \ar[r, "\pi"] \ar[d] & \tilde X = \Operatorname{Bl}_P(X) \ar[r, "\subseteq"] & \tilde P \\
    & P & X & \P
  \end{tikzcd}
\]
have SES
\[
  \begin{tikzcd}
    0 \ar[r] & N_{P/X} \ar[r] & N_{P/\P} \ar[r] & N_{X/\P}|_P \ar[r] & 0
  \end{tikzcd}
\]

\begin{eg}
  \(X \subseteq \P\) quartic, \(P\) is a point \(x \in X\). \(\varphi: BL_pX \to \P(U)\) is generically injective so \(X\) is rational. If \(X\) is cubic then \(\varphi\) is generically two-to-one.
\end{eg}


  



\printindex
\end{document}

% http://www.math.uni-bonn.de/people/huybrech/Notes.pdf
% same content but in a different order