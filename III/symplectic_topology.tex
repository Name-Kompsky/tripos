\documentclass[a4paper]{article}

\def\npart{III}

\def\ntitle{Symplectic Topology}
\def\nlecturer{A.\ Keating}

\def\nterm{Lent}
\def\nyear{2020}

\ifx \nauthor\undefined
  \def\nauthor{Qiangru Kuang}
\else
\fi

\ifx \ntitle\undefined
  \def\ntitle{Template}
\else
\fi

\ifx \nauthoremail\undefined
  \def\nauthoremail{qk206@cam.ac.uk}
\else
\fi

\ifx \ndate\undefined
  \def\ndate{\today}
\else
\fi

\title{\ntitle}
\author{\nauthor}
\date{\ndate}

%\usepackage{microtype}
\usepackage{mathtools}
\usepackage{amsthm}
\usepackage{stmaryrd}%symbols used so far: \mapsfrom
\usepackage{empheq}
\usepackage{amssymb}
\let\mathbbalt\mathbb
\let\pitchforkold\pitchfork
\usepackage{unicode-math}
\let\mathbb\mathbbalt%reset to original \mathbb
\let\pitchfork\pitchforkold

\usepackage{imakeidx}
\makeindex[intoc]

%to address the problem that Latin modern doesn't have unicode support for setminus
%https://tex.stackexchange.com/a/55205/26707
\AtBeginDocument{\renewcommand*{\setminus}{\mathbin{\backslash}}}
\AtBeginDocument{\renewcommand*{\models}{\vDash}}%for \vDash is same size as \vdash but orginal \models is larger
\AtBeginDocument{\let\Re\relax}
\AtBeginDocument{\let\Im\relax}
\AtBeginDocument{\DeclareMathOperator{\Re}{Re}}
\AtBeginDocument{\DeclareMathOperator{\Im}{Im}}
\AtBeginDocument{\let\div\relax}
\AtBeginDocument{\DeclareMathOperator{\div}{div}}

\usepackage{tikz}
\usetikzlibrary{automata,positioning}
\usepackage{pgfplots}
%some preset styles
\pgfplotsset{compat=1.15}
\pgfplotsset{centre/.append style={axis x line=middle, axis y line=middle, xlabel={$x$}, ylabel={$y$}, axis equal}}
\usepackage{tikz-cd}
\usepackage{graphicx}
\usepackage{newunicodechar}

\usepackage{fancyhdr}

\fancypagestyle{mypagestyle}{
    \fancyhf{}
    \lhead{\emph{\nouppercase{\leftmark}}}
    \rhead{}
    \cfoot{\thepage}
}
\pagestyle{mypagestyle}

\usepackage{titlesec}
\newcommand{\sectionbreak}{\clearpage} % clear page after each section
\usepackage[perpage]{footmisc}
\usepackage{blindtext}

%\reallywidehat
%https://tex.stackexchange.com/a/101136/26707
\usepackage{scalerel,stackengine}
\stackMath
\newcommand\reallywidehat[1]{%
\savestack{\tmpbox}{\stretchto{%
  \scaleto{%
    \scalerel*[\widthof{\ensuremath{#1}}]{\kern-.6pt\bigwedge\kern-.6pt}%
    {\rule[-\textheight/2]{1ex}{\textheight}}%WIDTH-LIMITED BIG WEDGE
  }{\textheight}% 
}{0.5ex}}%
\stackon[1pt]{#1}{\tmpbox}%
}

%\usepackage{braket}
\usepackage{thmtools}%restate theorem
\usepackage{hyperref}

% https://en.wikibooks.org/wiki/LaTeX/Hyperlinks
\hypersetup{
    %bookmarks=true,
    unicode=true,
    pdftitle={\ntitle},
    pdfauthor={\nauthor},
    pdfsubject={Mathematics},
    pdfcreator={\nauthor},
    pdfproducer={\nauthor},
    pdfkeywords={math maths \ntitle},
    colorlinks=true,
    linkcolor={red!50!black},
    citecolor={blue!50!black},
    urlcolor={blue!80!black}
}

\usepackage{cleveref}



% TODO: mdframed often gives bad breaks that cause empty lines. Would like to switch to tcolorbox.
% The current workaround is to set innerbottommargin=0pt.

%\usepackage[theorems]{tcolorbox}





\usepackage[framemethod=tikz]{mdframed}
\mdfdefinestyle{leftbar}{
  %nobreak=true, %dirty hack
  linewidth=1.5pt,
  linecolor=gray,
  hidealllines=true,
  leftline=true,
  leftmargin=0pt,
  innerleftmargin=5pt,
  innerrightmargin=10pt,
  innertopmargin=-5pt,
  % innerbottommargin=5pt, % original
  innerbottommargin=0pt, % temporary hack 
}
%\newmdtheoremenv[style=leftbar]{theorem}{Theorem}[section]
%\newmdtheoremenv[style=leftbar]{proposition}[theorem]{proposition}
%\newmdtheoremenv[style=leftbar]{lemma}[theorem]{Lemma}
%\newmdtheoremenv[style=leftbar]{corollary}[theorem]{corollary}

\newtheorem{theorem}{Theorem}[section]
\newtheorem{proposition}[theorem]{Proposition}
\newtheorem{lemma}[theorem]{Lemma}
\newtheorem{corollary}[theorem]{Corollary}
\newtheorem{axiom}[theorem]{Axiom}
\newtheorem*{axiom*}{Axiom}

\surroundwithmdframed[style=leftbar]{theorem}
\surroundwithmdframed[style=leftbar]{proposition}
\surroundwithmdframed[style=leftbar]{lemma}
\surroundwithmdframed[style=leftbar]{corollary}
\surroundwithmdframed[style=leftbar]{axiom}
\surroundwithmdframed[style=leftbar]{axiom*}

\theoremstyle{definition}

\newtheorem*{definition}{Definition}
\surroundwithmdframed[style=leftbar]{definition}

\newtheorem*{slogan}{Slogan}
\newtheorem*{eg}{Example}
\newtheorem*{ex}{Exercise}
\newtheorem*{remark}{Remark}
\newtheorem*{notation}{Notation}
\newtheorem*{convention}{Convention}
\newtheorem*{assumption}{Assumption}
\newtheorem*{question}{Question}
\newtheorem*{answer}{Answer}
\newtheorem*{note}{Note}
\newtheorem*{application}{Application}

%operator macros

%basic
\DeclareMathOperator{\lcm}{lcm}

%matrix
\DeclareMathOperator{\tr}{tr}
\DeclareMathOperator{\Tr}{Tr}
\DeclareMathOperator{\adj}{adj}

%algebra
\DeclareMathOperator{\Hom}{Hom}
\DeclareMathOperator{\End}{End}
\DeclareMathOperator{\id}{id}
\DeclareMathOperator{\im}{im}
\DeclareMathOperator{\coker}{coker}
\DeclarePairedDelimiter{\generation}{\langle}{\rangle}

%groups
\DeclareMathOperator{\sym}{Sym}
\DeclareMathOperator{\sgn}{sgn}
\DeclareMathOperator{\inn}{Inn}
\DeclareMathOperator{\aut}{Aut}
\DeclareMathOperator{\GL}{GL}
\DeclareMathOperator{\SL}{SL}
\DeclareMathOperator{\PGL}{PGL}
\DeclareMathOperator{\PSL}{PSL}
\DeclareMathOperator{\SU}{SU}
\DeclareMathOperator{\UU}{U}
\DeclareMathOperator{\SO}{SO}
\DeclareMathOperator{\OO}{O}
\DeclareMathOperator{\PSU}{PSU}
\DeclareMathOperator{\Sp}{Sp}


%hyperbolic
\DeclareMathOperator{\sech}{sech}

%field, galois heory
\DeclareMathOperator{\ch}{ch}
\DeclareMathOperator{\gal}{Gal}
\DeclareMathOperator{\emb}{Emb}



%ceiling and floor
%https://tex.stackexchange.com/a/118217/26707
\DeclarePairedDelimiter\ceil{\lceil}{\rceil}
\DeclarePairedDelimiter\floor{\lfloor}{\rfloor}


\DeclarePairedDelimiter{\innerproduct}{\langle}{\rangle}

%\DeclarePairedDelimiterX{\norm}[1]{\lVert}{\rVert}{#1}
\DeclarePairedDelimiter{\norm}{\lVert}{\rVert}



%Dirac notation
%TODO: rewrite for variable number of arguments
\DeclarePairedDelimiterX{\braket}[2]{\langle}{\rangle}{#1 \delimsize\vert #2}
\DeclarePairedDelimiterX{\braketthree}[3]{\langle}{\rangle}{#1 \delimsize\vert #2 \delimsize\vert #3}

\DeclarePairedDelimiter{\bra}{\langle}{\rvert}
\DeclarePairedDelimiter{\ket}{\lvert}{\rangle}




%macros

%general

%divide, not divide
\newcommand*{\divides}{\mid}
\newcommand*{\ndivides}{\nmid}
%vector, i.e. mathbf
%https://tex.stackexchange.com/a/45746/26707
\newcommand*{\V}[1]{{\ensuremath{\symbf{#1}}}}
%closure
\newcommand*{\cl}[1]{\overline{#1}}
%conjugate
\newcommand*{\conj}[1]{\overline{#1}}
%set complement
\newcommand*{\stcomp}[1]{\overline{#1}}
\newcommand*{\compose}{\circ}
\newcommand*{\nto}{\nrightarrow}
\newcommand*{\p}{\partial}
%embed
\newcommand*{\embed}{\hookrightarrow}
%surjection
\newcommand*{\surj}{\twoheadrightarrow}
%power set
\newcommand*{\powerset}{\mathcal{P}}

%matrix
\newcommand*{\matrixring}{\mathcal{M}}

%groups
\newcommand*{\normal}{\trianglelefteq}
%rings
\newcommand*{\ideal}{\trianglelefteq}

%fields
\renewcommand*{\C}{{\mathbb{C}}}
\newcommand*{\R}{{\mathbb{R}}}
\newcommand*{\Q}{{\mathbb{Q}}}
\newcommand*{\Z}{{\mathbb{Z}}}
\newcommand*{\N}{{\mathbb{N}}}
\newcommand*{\F}{{\mathbb{F}}}
%not really but I think this belongs here
\newcommand*{\A}{{\mathbb{A}}}

%asymptotic
\newcommand*{\bigO}{O}
\newcommand*{\smallo}{o}

%probability
\newcommand*{\prob}{\mathbb{P}}
\newcommand*{\E}{\mathbb{E}}

%vector calculus
\newcommand*{\gradient}{\V \nabla}
\newcommand*{\divergence}{\gradient \cdot}
\newcommand*{\curl}{\gradient \cdot}

%logic
\newcommand*{\yields}{\vdash}
\newcommand*{\nyields}{\nvdash}

%differential geometry
\renewcommand*{\H}{\mathbb{H}}
\newcommand*{\transversal}{\pitchfork}
\renewcommand{\d}{\mathrm{d}} % exterior derivative

%number theory
\newcommand*{\legendre}[2]{\genfrac{(}{)}{}{}{#1}{#2}}%Legendre symbol

%algebraic geometry
\DeclareMathOperator{\Spec}{Spec}
\DeclareMathOperator{\Proj}{Proj}

\renewcommand*{\P}{\mathbb{P}}
\newcommand{\w}{\wedge} % wedge product
\DeclareMathOperator{\Vect}{Vect} % vector field
\DeclareMathOperator{\Vol}{Vol} % volume form
\DeclareMathOperator{\Symp}{Symp}
\newcommand{\immersion}{\looparrowright}

\begin{document}

\begin{titlepage}
  \begin{center}
    \includegraphics[width=0.6\textwidth]{logo.jpg}\par
    \vspace{1cm}
    {\scshape\huge Mathamatics Tripos \par}
    \vspace{2cm}
    {\huge Part \npart \par}
    \vspace{0.6cm}
    {\Huge \bfseries \ntitle \par}
    \vspace{1.2cm}
    {\Large\nterm, \nyear \par}
    \vspace{2cm}
    
    {\large \emph{Lectures by } \par}
    \vspace{0.2cm}
    {\Large \scshape \nlecturer}
    
    \vspace{0.5cm}
    {\large \emph{Notes by }\par}
    \vspace{0.2cm}
    {\Large \scshape \href{mailto:\nauthoremail}{\nauthor}}
 \end{center}
\end{titlepage}

\tableofcontents

\setcounter{section}{-1}

\section{Introduction and Motivations}

Let \(M\) be a manifold. In Riemannian geometry, we put a nondegenerate symmetric bilinear form on \(T_xM\). In symplectic geometry we put a non-degenerate skew symmetric form instead. By basic linear algebra, by a change of basis, such a form is
\[
  \Omega =
  \begin{pmatrix}
    0 & 1 \\
    -1 & 0 \\
    & & \ddots \\
    & & & 0 & 1 \\
    & & & -1 & 0
  \end{pmatrix}
\]
We define the \emph{symplectic group} to be
\[
  \Sp_{2n}(\R) = \{A \in \GL_{2n}(\R): A^T\Omega A = \Omega\}.
\]

A symplectic manifold is a \(2n\)-manifold \(M^{2n}\) with an atlas of charts such that the derivatives of transition maps are in \(\Sp_{2n}(\R)\). We will prove that this is equivalent to \((M^{2n}, \omega)\), where \(\omega \in \Omega^2(M)\) closed (\(\d \omega = 0\)), everywhere nondegenerate (\(\omega^{\w n} \ne 0\)) at all points.

\begin{ex}
  \((\R^{2n}, \sum \d x_i \w \d y_i)\) gives \(\Omega\) with respect to \(\frac{\partial  }{\partial x_i}, \frac{\partial  }{\partial y_i}\). We call this sympletic form \(\omega_{\text{std}}\).
\end{ex}

In fact, this example is the ``local model'' for all symplectic manifolds. In other words, they have no local invariants.

Motivation 1: mechanices. Given a particle in \(\R^n\) and a potential \(U\), define \(H = U + \frac{\dot q^2}{2}\) to be the energy. Then we can work out the flow of Hamilton's equations
\[
  \frac{\partial H}{\partial p} = \dot q, \frac{\partial H}{\partial q} = - \dot p.
\]
It is a fact that the flow preserves the symplectic form \(\sum \d p_i \w \d q_i\).

Motivation 2: symmetry groups. We want to classify groups acting locally on \(\R^k\) such that
\begin{itemize}
\item act locally transitively (or reduce dim - orbit)
\item noo invariant foliations: not of the form \((x, y) \mapsto (f(x), g(x, y))\) (or reduce dimension).
\end{itemize}

\begin{theorem}[Lie]
  If such a group is finite dimensional, it is one of finitely many families (e.g.\ \(\SO(n), \SU(n), \SO(p, q)\) etc).
\end{theorem}

\begin{theorem}[Cartan]
  If such a group is infinite dimensional, it is one of
  \begin{itemize}
  \item \(\operatorname{Diff}(\R^k)\): all diffeomorphisms (preserving orientation),
  \item \(\operatorname{Vol}(\R^k)\): all diffeomorphisms preserving volume form,
  \item \(\operatorname{Symp}(\R^{2\ell})\): symplectomorphisms, i.e.\ diffeomorphisms preserving symplectic structure,
  \item \(\operatorname{Cont}(\R^{2\ell + 1})\): contactomorphism (odd dimensional analogue of symplectormorphism)
  \item and their conformal analogues.
  \end{itemize}
\end{theorem}

Motivation 3: difference with volume. WeAs \(A \in \Sp_{2n}(\R)\) implies \(\det A = 1\), we have inclusion \(\operatorname{Symp}(\R^{2n}) \subseteq \operatorname{Vol}(\R^{2n})\).

\begin{theorem}[Moser]\leavevmode
  \begin{enumerate}
  \item Two volume forms on a closed manifold are equivalent if and only if they have the same total volume.
  \item Suppose \(U, V\) are connected open in \(\R^k\). There is a volume form-preserving \(U \embed V\) if and only if \(\operatorname{vol}(U) \leq \operatorname{vol}(V)\).
  \end{enumerate}
\end{theorem}

By contrast

\begin{theorem}[Gromov non-squeezing]
  There is no symplectic embedding \(B^{2n}(R) \embed B^2(r) \times \R^{2n - 2}\) if \(R > r\).
\end{theorem}

Motivation 4: complex geometry. Any smooth affine variety has a natural symplectic form

Course outline:
\begin{itemize}
\item background: extra bit of differential geometry, almost complex structure, first Chern class,
\item basic symplecti geometry: distinguished submanifolds, local models, some constants on symplectic manifolds,
\item constructions: e.g.\ new sympletric manifolds from old,
\item holomorphic curves: invariants given by generalisation of Cauchy-Riemann equations, proof of non-squeezing theorem
\end{itemize}

\section{(More) Differential geometry}

\paragraph{Tensor algebra}

Let \(E\) be a vector space over \(\F\). We define the \emph{tensor algebra} of \(E\) to be
\[
  T(E) = \bigoplus_{i \geq 0} E^{\otimes i}
\]
where \(E^{\otimes 0} \cong \F\). Then we define the \emph{exterior algebra} to be
\[
  \Lambda^*E = T(E)/\langle v \otimes v \rangle_{\text{as algebra}}
\]
which has a natural grading \(\Lambda^*E = \bigoplus_{k \geq 0} \Lambda^kE\),
\[
  \Lambda^kE = E^{\otimes k}/\langle w_1 \otimes \cdots \otimes w_k: w_i = w_j \text{ for some } i \ne j \rangle_{\text{as vector space}}
\]
If \(\dim_\F E = n\) then \(\dim \Lambda^*E = 2^n, \dim \Lambda^kE = \binom{n}{k}\). The tensor product on \(T(E)\) induces wedge product on \(\Lambda^*(E)\) which is bilinear, associative and graded commutative.

If \(A: E \to F\) is a linear map then it induces a map \(\Lambda^k A: \Lambda^k E \to \Lambda^k F\). In \(\dim E = \dim F = n\) then we can identify \(\Lambda^nA: \Lambda^nE \to \Lambda^nF\) can be identified with \(\det A: \F \to \F\).

\paragraph{Vector fields and differential forms}

Suppose \(M^n\) is a manifold\footnote{In this course all manifolds are assumed to be smooth unless stated otherwise.}. Then we have the tangent and cotangent bundle \(TM, T^*M\). \emph{Vector fields} and \emph{\(k\)-forms} on \(M\) are defined to be
\begin{align*}
  \Vect(M) &= \Gamma(TM) = \Gamma(M, TM) \\
  \Omega^k(M) &= \Gamma(M, \Lambda^kT^*M)
\end{align*}
The \(0\)-forms are also the smooth functions on \(M\), \(C^\infty(M) = \Omega^0(M)\).

In local coordinates \(x_1, \dots, x_n\) on \(M\), \(X \in \Vect(M)\) can be written locally as
\[
  X_p = \sum_{i = 1}^n X^i \frac{\partial  }{\partial x_i}|_p
\]
where each \(X^i\) is a smooth function.

Given \(X \in \Vect(M), f \in C^\infty(M)\), we can differentiate \(f\) along \(X\) by \((Xf)_p = X_p(f)\). In local coordinates,
\[
  (Xf)_p = \sum X^i(p) \frac{\partial f}{\partial x_i}|_p.
\]
We can check this is well-defined and it is a derivation in the sense that \(X(fg) = fX(g) + gX(f)\).

\paragraph{Pullbacks}

Suppose \(f: M \to N\) is smooth. It induces \(f^*: C^\infty(M) \to C^\infty(N), g \mapsto g \compose f\) and also induces a function on \(1\)-forms by \((f^*\varphi)_x = (Df_*)^* (\varphi_{f(x)})\). It then induces \(f^*: \Omega^*(N) \to \Omega^*(M)\) with the properties that
\begin{enumerate}
\item \(f^*\) is linear and \(f^*(\varphi \w \theta) = f^*\varphi \w f^* \theta\).
\item \((f \compose g)^* \varphi = g^*f^* \varphi\)
\end{enumerate}

\paragraph{Differential on \(\Omega^*(M)\)}

Let \(U \subseteq M\) be a chart with coordinates \(x_i\). Then a local basis for \(\Lambda^kU\) is \(\{\d x_I = \d x_{i_1} \w \cdots \w \d x_{i_k}\}\) where \(I = \{i_1 < \dots < i_k\}\). If \(\varphi: U \to \R\) then we have
\begin{align*}
  \d: C^\infty(M) &\to \Omega^1(M) \\
  \varphi &\mapsto \d \varphi = \sum \frac{\partial \varphi}{\partial x_i} \d x_i
\end{align*}
Note that \((\d \varphi)X = X\varphi \in C^\infty(M)\). In general, if \(\varphi = \sum \varphi_i \d x_I \in \Omega^k(M)\) where \(\varphi_I\) smooth functions, then \(\d \varphi = \sum \d \varphi_I \w \d x_I\). Can check this is well-defined and satisfies
\begin{enumerate}
\item \(\d (\varphi_1 + \varphi_2) = \d \varphi_1 + \d \varphi_2\),
\item \(\d (\varphi_1 \w \varphi_2) = \d \varphi_1 \w \varphi_2 + (-1)^k \varphi_1 \w \d \varphi_2\) if \(\varphi_1 \in \Omega^k\),
\item \(\d^2 = 0\),
\item \(\d (f^*\varphi) = f^*(\d \varphi)\).
\end{enumerate}
Moreover we can show these properties uniquely determines \(\d\).

It follows that we have the \emph{de Rham complex}
\[
  \begin{tikzcd}
    0 \ar[r] & \Omega^0(M) \ar[r, "\d"] & \Omega^1(M) \ar[r, "\d"] & \Omega^2(M) \ar[r] & \cdots \ar[r, "\d"] & \Omega^n(M) \ar[r] & 0
  \end{tikzcd}
\]
which gives rise to de Rham cohomology \(H^*_{\text{dR}}(M)\). By de Rham theorem this is isomorphic to \(H^*(M; \R)\), singular cohomology with coefficients in \(\R\). \((\Omega^*M, \w, \d)\) is the de Rham algebra. Morgan 1978 shows that it characterises the rational homotopy type of algebraic varieties.

\paragraph{Isotopies and vector fields}

\begin{definition}[isotopy]\index{isotopy}
  A smooth map \(\rho: M \times \R \to M\) is an \emph{isotopy} if \(\rho_t = \rho(-, t): M \to M\) is a diffeomorphism for each \(t\) and \(\rho_0 = \id_M\).
\end{definition}

We could replace \(\R\) with open intervals containing \(0\).

Given an isotopy \(\rho\), we get a time-dependent vector field, say \(v_t\), as follows:
\[
  v_t|_p = \frac{d}{ds} \rho_s(q)|_{s = t}
\]
where \(q = \rho_t^{-1}(p)\), i.e.
\[
  \frac{d \rho_t}{dt} = v_t \compose \rho_t.
  \tag{\ast}
\]

Conversely, given a time-dependent vector field \(v_t\), if \(M\) is compact or if \(v_t\) is compactly supported, by Picard's theorem on existence of solutions to ODEs, there is an isotopy \(\rho\) such that \(\rho_0 = \id\) and the ODE (\(\ast\)) is satisfied. For compact \(M\) we have a one-to-one correspondence
\[
  \{\text{isotopies of } M\} \longleftrightarrow \{\text{time-dependent vector fields on } M\}.
\]
For non-compact \(M\), the flow still exists locally (i.e.\ at each point \(p\) for sufficiently small interval of time) by Picard(-Lindelöf).

\begin{definition}\index{exponential map}
  If \(v_t = v\) (independent of \(t\)), its flow is called the \emph{exponential map} of \(v\), denoted \(\exp(tv)\).
\end{definition}

Useful formula (III Differential Geometry Example sheet 2 question 3): for \(\theta \in \Omega^1(M), X, Y \in \Vect(M)\), have
\[
  \d \theta(X, Y) = X \theta(Y) - Y \theta(X) - \theta([X, Y]).
\]

\paragraph{Interior product}

Suppose \(\alpha \in \Omega^{p + 1}(M), X \in \Gamma(TM)\), the we define the \emph{interior product} \(X \lrcorner \alpha = \iota_X a \in \Omega^p(X)\) to be
\[
  \iota_X(\alpha)(u) = \alpha(X \w u)
\]
for \(u \in \Gamma(\Lambda^pTM)\).

\paragraph{Lie derivatives}

DG ES2 Q 11*

Let \(M\) be a manifold, \(X \in \Gamma(TM)\) a vector field. Then we have a local flow \(\varphi_t: M \to M\) for \(t \in (-\delta, \delta)\). Given \(\alpha \in \Omega^*(M)\), the \emph{Lie derivative} of \(\alpha\) with respect to \(X\) is
\[
  \mathcal L_X(\alpha) = \frac{d}{dt} (\varphi_t^*\alpha)|_{t = 0} \in \Omega^*(M)
\]
and has the same degree as \(\alpha\) if \(\alpha\) has pure degree. For \(V \in \Gamma(\Lambda^kTM)\), we similarly define
\[
  \mathcal L_XV = \frac{d }{d t}((\varphi_{-t})_*V)|_{t = 0}
\]
where \((\varphi_t)_* = \Lambda^k D\varphi_t\).

Properties:
\begin{enumerate}
\item \(\mathcal L_Xf = Xf\) for \(f \in C^\infty(M)\).
\item \(\mathcal L_X(Y) = [X, Y]\) for \(X, Y \in \Gamma(TM)\).
\item \(\mathcal L_X(\d \alpha) = \d \mathcal L_X \alpha\) for \(\alpha \in \Omega^*(M)\).
\item Cartan's formula: \(\mathcal L_X = \iota_X \compose \d + \d \compose \iota_X\).
\item For a time-dependent \(X_t\) with flow \(\varphi_t\), \(\frac{d}{dt}(\varphi_t^*\alpha) = \varphi_t^* \mathcal L_{X_t}\alpha\) for \(\alpha \in \Omega^*(M)\).
\end{enumerate}

\begin{proof}[Sketch proof]\leavevmode
  \begin{enumerate}
  \item \(\mathcal L_X f = \frac{d}{dt}|_{t = 0} (f \compose \varphi_t) = Xf\).
  \item Let \(\varphi_t\) be the flow of \(X\). Use the slightly unusual notation \(\varphi_t^*(Y)|_p = (D\varphi_t^{-1})(Y_{\varphi_t(p)})\). Check
    \[
      \varphi_t^*(Y)(f \compose \varphi_t) = Y(f) \compose \varphi_t
    \]
    so have
    \[
      \frac{\varphi_t^*(Y)(f \compose \varphi_t) - \varphi_t^*(Y)(f)}{t} + \frac{\varphi_t^*(Y)(f) - Y(f)}{t} = \frac{Y(f) \compose \varphi_t - Y(f)}{t}
    \]
    take limit as \(t \to 0\),
    \[
      YX(f) + \mathcal L_X(Y)(f) = XY(f).
    \]
  \item Omitted.
  \item General strategy: check the formula holds for \(0\)-forms, both sides commute with \(\d\), both sides are derivations for \((\Omega^*(M), \w)\), and use the fact that the equations are local and for a local coordinate patch \(U\), \(\Omega^*(U)\) is generated as an algebra by \(\Omega^0(U)\) and \(\d \Omega^0(U)\).
  \item Same as 4.
  \end{enumerate}
\end{proof}

\begin{lemma}
  For a smooth family \(\alpha_t \in \Omega^k(M)\),
  \[
    \frac{d}{dt}(\varphi_t^* \alpha_t) = \varphi_t^*(\mathcal L_{X_t} \alpha_t + \frac{d \alpha_t}{dt}).
  \]
\end{lemma}

\begin{proof}
  Treat LHS as the derivative of a function of two variables,
  \[
    \frac{d}{dt}(\varphi_t^* \alpha_t)
    = \frac{d}{dx}(\varphi_x^* \alpha_t)|_{x = t} + \frac{d}{dy}(\varphi_t^* \alpha_y)|_{y = t}
    = \varphi_t^*\mathcal L_{X_t}\alpha_t + \varphi_t^* \frac{d\alpha_t}{dt}.
  \]
\end{proof}

\paragraph{Oreintations}

Let \(E\) be an \(n\)-dim \(\R\) vector space. Then an orientation on \(E\) is an equivalence class of ordered basis \((e_1, \dots, e_n)\) under the equivalence relation \((e_1, \dots, e_n) \sim (f_1, \dots, f_n)\) if and only if the endomorphism \(A: e_i \mapsto f_i\) has \(\det A > 0\).

Let \(\pi: E \to B\) be a rank \(k\) real vector bundle. An orientation on \(E\) is a coherent choice of orientations on each fibre \(E_b\), where ``coherent'' means that for local trivialisation \(\pi^{-1}(U) \cong \R^k \times U\), the choice is constant.

Let \(M^n\) be a manifold. An orientation of \(M\) is an orientation of \(TM\) (if exists). We will denote by \(\overline M\) the manifold \(M\) with opposite orientation. If \(M^n\) is a manifold with boundary \(\p M\), an orientation of \(M\) induces an orientation on \(\p M\): a basis \((e_1, \dots, e_{n - 1})\) for \(T_x(\p M)\) is positively oriented if \((n_x, e_1, \dots, e_{n - 1})\) is for ``\(T_xM\)'', where \(n_x\) is the outward pointing normal vector.

Note if \(M\) is a compact oriented \(1\)-manifold with boundary the \(\sum_{p \in \p M} \operatorname{or}(p) = 0\).

\paragraph{Integration}

In vector calculus, we have if \(f: (U, x_i) \to (V, y_i)\) is a diffeomorphism of open subsets of \(\R^k\), then
\[
  \int_V a dy_1 \cdots dy_k = \int_U (a \compose f) |\det(Df)| dx_1 \cdots dx_k.
\]
In differential geometry we formulate integration in this way: for \(\varphi = a \d y_1 \w \cdots \w \d y_k\), if \(\varphi\) preserves orientation then
\[
  \int_V \varphi = \int_U f^* \varphi.
\]

\begin{lemma}
  If \(X\) is an oriented \(k\)-manifold, there is a well-defined integration map
  \[
    \int_X: \Omega^k_c \to \R
  \]
  where \(\Omega^k_c\) are \(k\)-forms with compact support.
\end{lemma}

A \emph{volume form} on \(M^k\) is a nowhere zero section \(\d \Vol \in \Omega^k(M)\), which is equivalent to a choice of trivialisation \(\Lambda^kT^*M \cong \R \times M\). \(M\) is orientable if and only if \(\Lambda^kT^*M\) is trivial. Note that \(\d (\d \Vol) = 0\).

\begin{theorem}[Stokes]
  \[
    \int_M \d \alpha = \int_{\p M} \alpha.
  \]
\end{theorem}

\begin{corollary}
  For \(X\) closed oriented, we have a surjection \(\int_X: H^k_{\text{dR}}(X) \to \R\).
\end{corollary}

\begin{proof}[Sketch proof]
  Let \(U \subseteq \R^k_+\) be an open chart. Use linearity and partition of unity, it suffices to work in \(U\). Then use standard results from multivariate calculus/Fubini's theorem. See example sheet 1.
\end{proof}

\section{Symplectic linear algebra}

Recall the standard symplectic form \(\omega_{\text{std}} = \sum \d x_i \w \d y_i\) on \((\R^{2n}, (x_i, y_i))\): with respect to \(\frac{\partial  }{\partial x_1}, \frac{\partial  }{\partial y_1}, \dots, \frac{\partial  }{\partial x_n}, \frac{\partial  }{\partial y_n}\), it has matrix
\[
  \Omega_0 =
  \begin{pmatrix}
    0 & 1 \\
    -1 & 0 \\
    & & \ddots \\
    & & & 0 & 1 \\
    & & & -1 & 0
  \end{pmatrix}
\]
Define
\begin{align*}
  \Sp_{2n}(\R)
  &= \{A \in \GL_{2n}(\R): A^*\omega_0 = \omega_0\} \\
  &= \{A \in \GL_{2n}(\R): A^T\Omega_0A = \Omega_0\}
\end{align*}
by identifying \(A\) with its matrix representation.

Recall from linear algebra

\begin{lemma}
  Suppose \((V, \Omega)\) is a vector space with a non-degenerate alternating (or skew-symmetric) bilinear form \(\Omega\) (i.e.\ \(V\) is a \emph{symplectic vector space}\index{symplectic vector space}), then there is a basis \(\mathcal B = (u_1, v_1, \dots, u_n, v_n)\) of \(V\) such that \([\Omega]_{\mathcal B} = \Omega_0\).
\end{lemma}

\begin{proof}[Sketch]
  By non-degeneracy exist \(u_1, v_1\) such that \(\Omega(u_1, v_1) = 1\). \(u_1, v_1\) are linearly independent since \(\Omega\) is alternating. Then \(V = \langle u_1, v_1 \rangle \oplus \{w: \Omega(u_1, w) = \Omega(v_1, w) = 0\}\). Proceed by induction.
\end{proof}

\begin{corollary}
  Symplectic vector spaces are even-dimensional and \(\Omega \in \Lambda^2V^*\) is non-degenerate if and only if \(\Omega^n \ne 0 \in \Lambda^{2n}V^*\).
\end{corollary}

\begin{definition}[symplectic complement]\index{symplectic complement}
  Suppose \(U \leq (V, \Omega)\). The \emph{symplectic complement} of \(U\) in \(V\) is
  \[
    U^{\Omega} = \{w \in V: \Omega(w, u) = 0 \text{ for all } u \in U\}.
  \]
\end{definition}

\begin{definition}[symplectic, (co)isotropic, Lagrangian subspace]\index{symplectic subspace}\index{isotropic subspace}\index{coisotropic subspace}\index{Lagrangian subspace}
  Let \((V, \Omega)\) be a symplectic vector space.
  \begin{itemize}
  \item \(U \leq V\) is a \emph{symplectic subspace} if \(U \cap U^\Omega = 0\), i.e.\ \(\Omega|_U\) is nondegenerate.
  \item \(U \leq V\) is an \emph{isotropic subspace} if \(U \leq U^\Omega\).
  \item \(U \leq V\) is a \emph{coisotropic subspace} if \(U^\Omega \leq U\).
  \item \(U \leq V\) is a \emph{Lagrangian subspace} if it is both isotropic and coisotropic.
  \end{itemize}
\end{definition}

\begin{proposition}
  An isotropic subspace has dimension at most \(\frac{1}{2} \dim V\) and a coisotropic subspace has dimension at least \(\frac{1}{2} \dim V\). If \(U\) is isotropic of dimension \(\frac{1}{2} \dim V\), it is also coisotropic (and vice versa), in which case it is Lagrangian.
\end{proposition}

\begin{proof}
  \(\Omega\) nondegenerate gives a surjection \(V \to U^*, v \mapsto \Omega(-, v)\). Thus \(\dim U^* + \dim U^0 = \dim V\), so \(\dim U + \dim U^\Omega = \dim V\).
\end{proof}

\section{Symplectic manifolds: first notions}

\(\varphi \in \Omega^2(M)\) gives
\begin{align*}
  \mu_\varphi: TM &\to T^*M \\
  u &\mapsto (v \mapsto \varphi(u, v))
\end{align*}
i.e.\ \(\mu_\varphi u = \iota_u \varphi\). \(\varphi\) is \emph{non-degenerate} if \(\mu_\varphi\) is an isomorphism, which happens if and only if \(\varphi^n\) is nonwhere zero, where \(\dim M = 2n\).

\begin{definition}[symplectic form]\index{symplectic form}
  A closed and non-degenerate form \(\omega \in \Omega^2(M)\) is called a \emph{symplectic form}.
\end{definition}
First condition a linear algebra condition, the second a local condition: to check some integral is zero only have to check locally.

\begin{definition}[symplectic manifold]\index{symplectic manifold}
  \((M^{2n}, \omega)\) is a symplectic manifold.
\end{definition}

\begin{definition}[symplectic structure]\index{symplectic structure}
  A \emph{symplectic structure} is a symplectic form up to pullback by diffeomorphism.
\end{definition}

\begin{proposition}
  A 2-fold is symplectic if and only if it is orientable.
\end{proposition}

\begin{proof}
  A non-degenerate form on \(M^2\) is a volume form. By dimension reason all \(2\)-forms are closed.
\end{proof}

\begin{proposition}
  Suppose a closed manifold \(M^{2n}\) is symplectic. Then \(H^{2i}_{\text{dR}}(M) \ne 0\) for \(0 \leq i \leq n\).
\end{proposition}

\begin{proof}
  \([\omega] \in H^2_{\text{dR}}(M)\) and \(\omega^n\) is a volume form, say \(\d \Vol\). Thus
  \[
    [\omega]^n = [\omega^{\w n}] = [\d \Vol] \ne 0 \in H^{2n}_{\text{dR}}(M).
  \]
\end{proof}

\begin{eg}
  \(S^4\) is not symplectic.
\end{eg}

\subsection{Hamiltonian flows}

Suppose \((M^{2n}, \omega)\) is symplectic and \(f \in C^\infty(M)\). We can construct a vector field as follow: \(\d f \in \Omega^1(M)\). Use the bundle isomorphism \(\mu_\omega: TM \to T^*M\) we obtain \(X_f = (\mu_\omega)^{-1}(\d f)\). It is the unique vector field such that \(\iota_{X_f} \omega = \d f\).

\begin{definition}[Hamiltonian flow]\index{Hamiltonian flow}
  The flow of \(X_f\) is called the \emph{Hamiltonian flow} of \(f\).
\end{definition}

\begin{proposition}
  Wherever defined, the Hamiltonian flow of a function acts by symplectomorphism.
\end{proposition}

\begin{proof}
  By Cartan's formula
  \[
    \mathcal L_{X_f} \omega = \iota_{X_f} \d \omega + \d \iota_{X_f} \omega
    = 0 + \d (\d f) = 0.
  \]
\end{proof}

\begin{remark}
  This means that symplectic manifolds have larger spaces of symmetries than Riemannian manifolds. For example compare isometries of \(S^2\) vs. symplectomorphisms.
\end{remark}

\begin{eg}
  Hamilton's equations
  \[
    \frac{\partial H}{\partial p} = \dot q, \frac{\partial H}{\partial q} = - \dot p
  \]
  where \(q\) is position and \(p\) is momentum. This is same as following the flow of \(X_H = (\frac{\partial H}{\partial p}, - \frac{\partial H}{\partial q})\), for \(\omega = \sum \d q_i \w \d p_i\). Note \(\iota_{X_H} \omega = \d H\). This shows that classical Hamiltonian flows are examples of Hamiltonian flows in the sense of symplectic geometry and are through symplectomorphisms of \(\R^{2n}\).
\end{eg}

\subsection{(Almost) complex manifolds}

\begin{definition}[complex manifolds]
  A \emph{complex manifold} is a manifold \(M^{2n}\) covered by charts \(u_\alpha \subseteq \C^n\) such that the transition maps are biholomorphisms. Equivalently, for two charts \((U_\alpha, \varphi_\alpha)\) and \((U_\beta, \varphi_\beta)\), we require \(D(\varphi_\alpha \compose \varphi_b^{-1}) \in \GL_n(\C) \subseteq \GL_{2n}(\R)\).
\end{definition}

The (co)tangent spaces of \(M\) are naturally complex vector spaces.

\begin{definition}[almost complex structure]\index{almost complex structure}
  An \emph{almost complex structure} (acs) on a smooth manifold \(M\) is an endomorphism \(J: TM \to TM\) such that \(J^2 = -\id\).
\end{definition}

\begin{definition}[integrable]\index{almost complex structure!integrable}
  If an almost complex structure \(J\) comes from a complex structure then we say \(J\) is \emph{integrable}.
\end{definition}

\begin{remark}\leavevmode
  \begin{enumerate}
  \item A complex manifold is almost complex: the complex structure is induced by multiplication by \(i\).
  \item \(J\) extends to a map \(TM \otimes_\R \C \to TM \otimes_\R \C\), which we also denote by \(J\). For complex manifolds we get
    \[
      J(\frac{\partial  }{\partial z_j}) = i \frac{\partial  }{\partial z_j}, J(\frac{\partial  }{\partial \overline z_j}) = -i \frac{\partial  }{\partial \overline z_j}
    \]
    where \(TM \otimes \C = \C \langle \frac{\partial  }{\partial x_j}, \frac{\partial  }{\partial y_j} \rangle\), and
    \[
      \frac{\partial  }{\partial z_j} = \frac{1}{2}(\frac{\partial  }{\partial x_j} - i \frac{\partial  }{\partial y_j}),
      \frac{\partial  }{\partial \overline z_j} = \frac{1}{2}(\frac{\partial  }{\partial x_j} + i \frac{\partial  }{\partial y_j}),
    \]
    and their dual basis is
    \[
      \d z_j = \d x_j + i \d y_j, \d \overline z_j = \d x_j - i \d y_j.
    \]
  \item The complexified cotangent bundle splits as \(T^*M \otimes \C = T^*M^{1, 0} \oplus T^*M^{0, 1}\) where
    \begin{align*}
      T^*M^{1, 0} &= \{\alpha: \alpha(Jv) = i \alpha(v)\} = \C \langle \d z_j \rangle \\
      T^*M^{0, 1} &= \{\alpha: \alpha(Jv) = -i \alpha(v)\} = \C \langle \d \overline z_j \rangle \\
    \end{align*}
    This then induces splitting on sections
    \[
      \Omega^1(M; \C) = \Gamma(T^*M \otimes \C) = \Omega^{1, 0} \oplus \Omega^{0, 1}.
    \]
    More generally, define
    \[
      \Omega^{p, q}(M) = \Gamma(\Lambda^pT^*M^{1, 0} \otimes \Lambda^qT^*M^{0, 1}) \leq \Gamma(\Lambda^{p + q}(T^*M \otimes \C)) = \Omega^{p + q}(M; \C).
    \]
    For \(M\) a complex manifold, a section for \(\Omega^{p, q}\) is given in locally coordintes by
    \[
      \sum \alpha_{PQ} \d z_P \w \d \overline Z_Q
    \]
    where \(\alpha_{PQ}\) are smooth functions.
  \item \(\d: \Omega^k(M) \to \Omega^{k + 1}(M)\) induces \(\d: \Omega^k(M; \C) \to \Omega^{k + 1}(M; \C)\). Suppose \(\d: \Omega^0(M; \C) \to \Omega^1(M; \C)\). By composition with projections to \((1, 0)\)-forms and \((0, 1)\)-forms, we get \(\d = \p + \overline \p\).

    For a \emph{complex manifold}, we can
    \begin{enumerate}
    \item talk about holomorphic functions (functions that are holomorphic on each chart), and \(f \in C^\infty(M)\) is holomorphic if and only if \(\overline \p f = 0\).
    \item \(\d(\Omega^{p, q}) \subseteq \Omega^{p + 1, q} \oplus \Omega^{p, q + 1}\) (this needn't be true for almost complex manifolds).
    \item \(\d^2 = 0\) implies \(\overline \p = 0\) so we can form the Dolbeault complex
      \[
        \begin{tikzcd}
          \cdots \ar[r] & \Omega^{\bullet, k} \ar[r, "\overline \p"] & \Omega^{\bullet, k + 1} \ar[r, "\overline \p"] & \cdots
        \end{tikzcd}
      \]
      The cohomology of this complex is Dolbeault cohomology \(H_{\overline \p}^{\bullet, k}(M)\).
    \end{enumerate}
  \end{enumerate}
\end{remark}

\subsection{Kähler manifold}

\begin{definition}[Kähler manifold]\index{Kähler manifold}
  A \emph{Kähler manifold} is a complex manifold with a closed positive-definite real \((1, 1)\)-form: \(\omega \in \Omega^{1, 1}(M)\) such that
  \begin{itemize}
  \item \(\p \omega = \overline \p \omega = 0\),
  \item \(\omega = \frac{i}{2} \sum_{j, k} h_{jk} \d z_j \w \d \overline z_k\) with \((h_{jk})\) a hermitian postive definite matrix.
  \end{itemize}
\end{definition}

\begin{ex}
  Expand the local expression in terms of \(\d x_j\) and \(\d y_j\) to show that the factor \(\frac{i}{2}\) is sensible.
\end{ex}

\begin{proposition}
  Kähler forms are symplectic forms.
\end{proposition}

\begin{proof}
  A Kähler form \(\omega\) is closed, and
  \[
    \omega^n = n! \left(\frac{i}{2} \right)^n \underbrace{\det(h_{jk})}_{> 0} \d z_1 \w \d \overline z_1 \w \cdots \w \d z_n \w \d \overline z_n
  \]
  and since \(\frac{i}{2} \d z_i \w \d \overline z_i = \d x_i \w \d y_i\), \(\omega^n\) is nowhere zero.
\end{proof}

\begin{definition}[plurisubharmonic]\index{plurisubharmonic}
  Let \(M\) be a complex manifold. A smooth function \(\rho: M \to \R\) is strictly \emph{plurisubharmonic} (plush) if
  \[
    \left( \frac{\partial^2 \rho}{\partial z_j \partial \overline z_k} \right)
  \]
  is positive definite everywhere.
\end{definition}

Check that for such a function, \(\frac{i}{2} \p \overline \p \rho\) defines a Kähler form on \(M\).

\begin{remark}
  Such an \(M\) can't be closed.
\end{remark}

\begin{eg}\leavevmode
  \begin{enumerate}
  \item \(\rho(z) = |z|^2\) on \(\C^n \cong \R^{2n}\) gives \(\frac{i}{2} \p \conj \p \rho = \omega_{\text{std}}\).
  \item \(\rho(z) = \log(1 + |z|^2)\) on \(\C^n \cong \R^{2n}\). To check this is plush, look at \((1, 0, \dots, 0)\) and use \(U(n)\)-invariance. Check also the induced volume form has finite total volume.
  \item A complex submanifold of a Kähler manifold is Kähler with the pullback form.
  \item \(\C \P^n\) is Kähler: \(\P^n\) can be covered by charts \(U_i = \{z_i \ne 0\}\). The transition functions are of the form
    \[
      \varphi: (u_1, \dots, u_n) \mapsto (\frac{1}{u_1}, \frac{u_2}{u_1}, \dots, \frac{u_n}{u_1})
    \]
    so
    \[
      \varphi^*(\frac{i}{2} \p \overline \p(\log(1 + |z|^2))
      = \frac{i}{2} \p \overline \p (\log(1 + |z|^2) + \log (\frac{1}{|z_1|^2}))).
    \]
    Thus the local Kähler forms patch to give a global one. For details see III Complex Manifolds.
  \end{enumerate}
\end{eg}

\begin{theorem}[Hodge]
  If \(X\) is compact Kähler then
  \begin{itemize}
  \item \(H^k_{\mathrm{dR}}(X) \otimes \C \cong \bigoplus_{i + j = k} H^{i, j}_{\conj \p}(X)\).
  \item \(H^{i, j}_{\conj \p}(X) \cong \conj{H^{j, i}_{\conj \p}(X)}\). In particular they have the same dimension.
  \end{itemize}
\end{theorem}

\begin{corollary}
  If \(X\) is compact Kähler then any Betti number of odd degree is even.
\end{corollary}

\begin{theorem}[Lefschetz]
  For \(X\) compact Kähler the wedge product
  \[
    - \w [\omega]^k: H^{n - k}_{\mathrm{dR}}(X) \to H^{n + k}_{\mathrm{dR}}(X)
  \]
  is an isomorphism, where \(n\) is the complex dimension of \(X\).
\end{theorem}

\begin{remark}\leavevmode
  \begin{enumerate}
  \item It is easy to write down a compact complex manifold that is not Kähler: consider \((\C^2 \setminus \{0\})/z \sim 2z\). It inherits a complex structure from \(\C^2\). It is homeomorphic to \(S^1 \times S^3\), so \(b_2 = 0\). Thus it can't be symplectic.
  \item Most examples of complex Kähler manifold we'll see are projective, but plenty are not (e.g.\ take deformations of complex surfaces in \(\P^3\). c.f.\ K3 surfaces).
  \end{enumerate}
\end{remark}

\subsection{Almost complex structure on symplectic manifolds}

\begin{definition}[compatible almost complex structure]\index{almost complex structure!compatible}
  An almost complex structure \(J\) on a symplectic manfiold \((M, \omega)\) is \emph{compatible} with \(\omega\) if
  \begin{enumerate}
  \item \(\omega(Ju, Jv) = \omega(u, v)\),
  \item \(\omega(v, J(v)) > 0\) unless \(v = 0\).
  \end{enumerate}
\end{definition}

\begin{note}
  It follows that \(\omega(\cdot, J \cdot)\) is a positive-definite symmetric bilinear form, so gives a Riemannian metric \(g(w, u) = \omega(u, Jv)\). Such a triple \((\omega, J, g)\) is sometimes called a compatible triple. We'll see any two of them determines the third.
\end{note}

\begin{proposition}
  Any symplectic manifold \((M, \omega)\) admits a compatible almost complex structure.
\end{proposition}

\begin{proof}
  Let \((V, \Omega)\) be a symplectic vector space. Fix any metric \(g\) on \(V\). As \(g\) and \(\Omega\) both determine isomorphisms \(V \to V^*\), there exists \(A \in \End V\) such that \(\omega(u, v) = g(Au, v)\). Note that
  \[
    \omega(u, v) = -\omega(v, u) = -g(Av, u) = -g(u, Av)
  \]
  so \(A^* = -A\) with respect to \(g\). As \(g(AA^*v, v) = g(A^*v, A^*v) > 0\) for all \(v \ne 0\) and \((AA^*)^* = AA^*\), \(AA^*\) is positive definite symmetric. Thus by choosing an orthnormal basis, \(AA^* = BDB^{-1}\), where \(B\) is the diagonal matrix with entries \(\lambda_1, \cdots \lambda_{2n}\). We can thus take the positive square root and define \(J = (\sqrt{AA^*})^{-1}A\).
  \begin{enumerate}
  \item \(J^2 = (\sqrt{-A^2})^{-1}A(\sqrt{-A^2})^{-1}A = -\id\).
  \item \(J^* = -J\) (implies \(JJ^* = \id\) so \(J\) is orthogonal).
  \item For compatibility,
    \[
      \omega(Ju, Jv) = g(AJu, Jv) = g(JAu, Jv) = g(Au, v) = \omega(u, v).
    \]
  \item \(\omega(u, Ju) = g(-JAu, u) = g(\sqrt{AA^*} u, u) > 0\).
  \end{enumerate}

  Since \(A\) is uniquely determined and we are taking the positive square root, we can use this construction on each \(T_xM\), for a choice of Riemannian metric on \(M\). Our procedure on \((V, \Omega)\) is canonical so this works globally.
\end{proof}

\begin{note}
  Given \((M, \Omega)\), there is a bijection
  \[
    \{\text{compatible complex structure } J\} \longleftrightarrow \{\text{Riemannian metric } M\}
  \]
  Note that RHS is convex so contractible, and it follows that LHS is also contractible. We will see that for many applications, there is ``essentially no choice'' of \(J\).
\end{note}

\begin{definition}[symplectic vector bundle]\index{symplectic vector bundle}
  A \emph{symplectic vector bundle} is vector bundle \(\pi: E \to B\) with a section \(\Omega \in \Gamma(\Lambda^2E^*)\) such that \(\Omega|_{E_b} = \omega_b\) is symplectic on each fibre and locally trivial, i.e.\ \((\pi^{-1}(U), \Omega) \cong (U \times \R^{2n}, \omega_0)\).
\end{definition}


\begin{corollary}
  Such an \(E\) admits the structure of a complex vector bundle, uniquely determined up to a contractible choice.
\end{corollary}

Note that this is weaker than a holomorphic bundle.

\subsection{First Chern class}

What is the (unique) compatible complex structure good for? We can define a topological invariant. The \emph{first Chern class} of a complex vector bundle \(E \to B\) is an element of \(c_1(E) \in H^2(B; \Z)\). We present three definitions here. They are equivalent, but we will not prove so.
\begin{enumerate}
\item Algebraic topology: let \(\det E = \Lambda^{\mathrm{rank} E}E\), which is a complex line bundle over \(B\), so can be obtained as a pullback of \(\mathcal L_{\mathrm{taut}} \to BU(1) = \C\P^\infty\). Note the tautological bundle can be described as
  \begin{align*}
    \mathcal L_{\mathrm{taut}}
    &= \{(w, z) \in \C\P^n \times \C^{n + 1}: z \in w\} \\
    &= \{(w, z): z_iw_j = z_jw_i \text{ for all } i, j\} \subseteq \C\P^n \times \C^{n + 1}
  \end{align*}
  The advantage of the second description is that it is a smooth projective variety so a Kähler manifold.

  Suppose \(\det E = \varphi^* \mathcal L_{\mathrm{taut}}\). \(H^*(\C\P^\infty) = \Z[c_1^{\mathrm{univ}}]\) where \(c_1\) has degree \(2\). We then define \(c_1(E) = \varphi^*(c_1^{\mathrm{univ}})\).
\item Algebraic topology, alternative: take \(s: B \to \det E\) a generic section. Then \(\dim_\R(s(B) \cap s_0(B)) = n - 2\) (recall \(E\) is complex). Then define \(c_1(E)\) to be the Poincaré dual of this class.
\item Differential geometry: suppose \(\d_A\) is a connection on \(E\), \(F_A = \d \cdot \d_A \in \Omega^2(\End E)\) the curvature. Then \(c_1(E) = [\frac{1}{2\pi i} \tr F_A]\).
\end{enumerate}

Properties of \(c_1(E)\):
\begin{enumerate}
\item from the second definition, \(c_1(E) = e(\det E) \in H^2(B; \Z)\).
\item \(c_1(E) \in H^2(B; \Z)\) is invariant if \(J_E\) change continuously, so we get an invariant of the symplectic bundle.
\item \(c_1(f^*E) = f^*c_1(E)\).
\item \(c_1(E^*) = -c_1(E)\).
\item For any short exact sequence of complex vector bundles \(0 \to E \to \mathcal E \to E' \to 0\), then \(c_1(\mathcal E) = c_1(E) + c_1(E')\).
\item \(c_1(E \otimes F) = \mathrm{rk} E c_1(F) + \mathrm{rk} F c_1(E)\).
\item For \(M\) compact, complex line bundles over \(M\) up to isomorphism is isomorphic to \(H^2(M; \Z)\) via \(c_1\).
\end{enumerate}

\begin{notation}
  For a manifold with amost complex structure, define \(c_1(M) = c_1(TM)\).
\end{notation}

\begin{ex}\leavevmode
  \begin{enumerate}
  \item \(c_1(\Sigma_g) = (2 - 2g) PD(\mathrm{pt})\). This is Guass-Bonnet: count the signed zeros of a generic vector field. As \(\Sigma_g\) has complex dimension \(1\), this is just the Euler class of \(T\Sigma_g\).
  \item \(c_1(\mathcal L_{\mathrm{taut}} \to \C\P^n) = -[H] = PD(\C\P^{n -1}) \in H^2(\C\P^n) \cong \Z \langle H\rangle\). \([H]\) is the hyperplane class. In algebraic geometry we have \(\mathcal L_{\mathrm{taut}} = \mathcal O(-1)\).
  \item \(c_1(T \C\P^n) = c_1(\C\P^n) = (n + 1) [H]\).
  \end{enumerate}
\end{ex}

Constraint for \(4\)-manifolds:

\paragraph{adjunction formula}

Suppose \(C \subseteq (X^4, J)\) is an almost complex curve in an almost complex surface (\(JTC = TC \subseteq TX\)). Then
\[
  2g(C) - 2 = -c_1(X) \cdot [C] + [C]^2
\]
where \([C]^2\) is the self-intersection number of \(C\). Thus the genus of \(C\) is determiend by its homology class.

\begin{proof}
  We have a short exact sequence
  \[
    \begin{tikzcd}
      0 \ar[r] & TC \ar[r] & TX|_C \ar[r] & \nu_{C/X} \ar[r] & 0
    \end{tikzcd}
  \]
  where \(\nu_{C/X}\) is the normal bundle. So \(c_1(TX|_C) = c_1(TC) + c_1(\nu_{C/X})\). Pair this with \([C] \in H_2(C; \Z)\),
  \[
    c_1(TX) \cdot [C] = \underbrace{c_1(TC) \cdot [C]}_{\chi(C) = 2 - 2g(C)} + c_1(\nu_{C/X}) \cdot [C].
  \]
  The last term is \([C]^2\), argued as follow: the self-intersection number is the signed number of intersection points of \(C\) and a small (smooth) pushoff of \(C\), say \(C'\). We can identify \(C\) with the zero section of the normal bundle and \(C'\) a small generic section of \(\nu_{C/X}\).
\end{proof}

Recall for oriented \(X^4\), \(H^2(X; \R)\) has symmetric non-degenerate cup product pairing, so gives rise to \emph{signature} \(\sigma(X) = b_+ - b_-\). We state without proof

\begin{theorem}[Hirzebruch signature theorem]
  Let \(X^4\) be an almost complex manifold. Then \(c_1(X)^2 = 2 \chi(X) + 3\sigma(X)\). In particular this is a topological invariant.
\end{theorem}

Fact: \(c_1^2 = \sigma \pmod 8\).

\begin{corollary}
  If \(X^4\) admits an almost complex structure then \(X \# X, X \# X \# X \# X\) etc do not admit almost complex structures.
\end{corollary}

\begin{proof}
  \(1 - b_1 + b_+\) is even if \(X\) admits an almost complex structure. Now \(b_\pm(X \# X) = 2 b_\pm(X)\) etc.
\end{proof}


\paragraph{Local forms for symplectic manifolds}

\begin{theorem}[Moser stability]\index{Moser stability}
  If we have a smooth family of symplectic forms \(\{\omega_t\}\) on a closed symplectic manifold \(M\) such that \([\omega_t] \in H^2_{\mathrm{dR}}(X)\) is constant, then there is a diffeomorphism \(f: M \to M\) such that \(f^*\omega_1 = \omega_0\).
\end{theorem}

Thus if we deform a symplectic form smoothly the symplectic structure remains unchanged.

\begin{remark}\leavevmode
  \begin{enumerate}
  \item Small deformation of symplectic form matters: let \(M = \C\P^1 \times \C\P^1, \Omega_1 = \omega \oplus \omega, \Omega_2 = \omega \oplus t \omega, t > 1\). They are not in the same class.

    Gromov: there is a deformation retract \(\Symp(M, \Omega_1) \simeq \SO(3) \times \SO(3)\). On the other hand \(\pi_1 \Symp(M, \Omega_2)\) is infinite.
  \item A blowup of \(S^2 \times T^2 \times S^2 \times S^2\) has two symplectic forms which are cohomologous but there does not exist a diffeomorphism pulling back one to the other.
  \item A note on closedness: there exist exotic symplectic structures on \(\R^{2n}\) for \(n \geq 2\), i.e.\ no symplectic embedding into \((\R^{2n}, \omega_{\mathrm{std}})\).
  \end{enumerate}
\end{remark}

\begin{proof}
  We are going to show there is an isotopy \(\varphi_t: M \to M, \varphi_0 = \id, \varphi_1 = f\) and \(\varphi_t^* \omega_t = \omega_0\). It's equivalent to looking for a vector field \(X_t \in \Vect(M)\) whose flow is \(\varphi_t\). Recall that the vector field is defined by.
  \[
    \frac{d}{dt} \varphi_t = X_t \compose \varphi_t.
  \]

  Differentiate the relation \(\varphi_t^* \omega_t = \omega_0\),
  \[
    0 = \frac{d}{dt} (\varphi_t^* \omega_t)
    = \varphi_t^*(\frac{d}{dt} \omega_t + \mathcal L_{X_t} \omega_t)
    = \varphi_t^* (\frac{d}{dt} \omega_t + \d \iota_{X_t} \omega_t + \iota_{X_t} \d \omega_t).
    \tag{\ast}
  \]
  Since \([\omega_t]\) is fixed, \(\frac{d}{dt}\omega_t = \d \sigma_t\) where \(\sigma_t \in \Omega^1(M)\) is smooth. Thus (\(\ast\)) is equivalent to \(\d \sigma_t + \d \iota_{X_t} \omega_t = 0\). Thus it is enough to show \(\sigma_t + \iota_{X_t} \omega_t = 0\), which unqiuely determines \(X_t\) as \(\omega_t\) is nondegenerate.
\end{proof}

Slogan for local form theorems: check condition infinitesimally, integrate on small neighbourhood using vector field/Moser type argument.

\begin{theorem}[Darboux]\index{Darboux theorem}
  If \(p \in (M^{2n}, \omega)\), there is a local chart \(f: D_\varepsilon(0) \to M\) with \(f(0) = p\) such that \(f^*\omega = \omega_0\). In other words, symplectic manifolds are locally standard.
\end{theorem}

\begin{proof}
  Fix a chart \(h: D_r(0) = D \to M\) around \(p\). Both \(h^*\omega\) and \(\omega_0\) are symplectic forms on \(D\). Postcomposing an element of \(\Sp_{2n}(\R)\), wlog \(h^*\omega\) and \(\omega_0\) agree at \(p\). Let \(\omega_t = (1 - t)\omega_0 + t h^*\omega\) be the linear interpolation. Being nondegenerate is an open condition, so there is an open neighbourhood \(U \subseteq D\) containing \(0\) on which \(\omega_t\) is symplectic for \(t \in [0, 1]\). \(H^2(D) = 0\) so
  \[
    \frac{d}{dt} \omega_t = -\omega_0 + h^*\omega = \d \sigma
  \]
  for some \(\sigma \in \Omega^1(D)\). wlog \(\sigma\) vanishes at \(0\). Let \(X_t\) be the vector field defined by \(\iota_{X_t} \omega_t = -\sigma\). Note \(X_t\) vanishes at \(0\), so for some \(\varepsilon > 0\) the trajectory of every \(x \in D_\varepsilon(0) \subseteq U\) under the flow of \(X_t\) stays inside \(U\) (and is defined) up to at least time \(1\). Thus on \(D_\varepsilon(0)\) we can integrate \(X_t\) to \(\{\psi_t\}_{t \in [0, 1]}\) such that \(\psi_1^*h^* \omega = \omega_0\).
\end{proof}

\begin{remark}
  This shows that closed nondegenerate \(2\)-form is the same as atlas of charts with transition functions whose derivatives are in \(\Sp_{2n}(\R)\).
\end{remark}

\begin{theorem}[Poincaré lemma]\index{Poincaré lemma}
  Suppose \(M\) is a \(C^\infty\) manifold, \(Q \subseteq M\) closed smooth submanifold, \(\alpha_1, \alpha_2 \in \Omega^k(M)\) agreeing on \(TM|_Q\). Then exists \(\beta \in \Omega^{k - 1}(U_Q)\), where \(U_Q\) is an open neighbourhood of \(Q\), such that \(\d \beta = \alpha_1 - \alpha_2\) and \(\beta = 0\) on \(TM|_Q\).
\end{theorem}

\begin{proof}
  Example sheet 2. Use partition of unity.
\end{proof}

\begin{note}
  If \(Q \subseteq (M, \omega)\) is a symplectic submanifold. Then
  \[
    \nu_{Q/M} \cong (TQ)^\omega \subseteq TM|_Q
  \]
  as a subbundle. On each fibre \(V \cong W \oplus W^\omega\) for \(W \leq V\). Moreover \(\nu_{Q/M} = (TQ)^\omega\) is a symplectic vector bundle.
\end{note}

\begin{theorem}
  If \(Q \subseteq (M, \omega)\) is a compact symplectic submanifold, the symplectic structure of \(M\) near \(Q\) is determined by \(\omega|_Q\) and \(\nu_{Q/M}\) as symplectic vector bundle, i.e.\ if two submanifolds have the same data then they have symplectomorphic tubular neighbourhoods.
\end{theorem}

\begin{proof}
  Choice of a metric gives \(\exp: \nu_{Q/M} \to U_Q\) defined for \(|t| < \varepsilon\). If \(Q_i \subseteq M_i\) symplectic, \(\varphi: Q_1 \to Q_2\) a symplectomorphism then it lifts to \(\Phi: \nu_{Q_1/M_1} \to \nu_{Q_2/M_2}\), an isomorphism of symplectic bundles. Using metric to get \(\tilde \varphi: \exp_{M_2} \compose \Phi \compose \exp_{M_1}^{-1}: U_{Q_1} \to U_{Q_2}\). \(\omega_1\) and \(\tilde \varphi^* \omega_2\) are symplectic forms on \(U_{Q_1}\) which agree on \(TM|_{Q_1}\) so by Poincaré lemma exists \(\sigma \in \Omega^1(U_{Q_1}')\) which vanish on \(TM_{Q_1}\) and such that \(\d \sigma = \omega_1 - \tilde \varphi^* \omega_2\) on \(U_{Q_1}'\). Now apply Moser's method to the path \(\omega_t = t\omega_1 + (1 - t)\tilde \varphi^* \omega_2\). Similar to the proof of Darboux, we can integrate the vector field to time \(1\) on a sufficiently small neighbourhood of \(Q_1\).
\end{proof}

\begin{corollary}
  A neighbourhood of a closed symplectic surface (i.e.\ real 2 dimensional) \(C \subseteq (M^4, \omega)\) is determined by
  \begin{itemize}
  \item topological type of \(C\) (i.e.\ genus), \(\int_C \omega\) (these determine \(C\) as symplectic manifold).
  \item \([C]^2\) (determines \(\nu_{C/M}\)).
  \end{itemize}
\end{corollary}

\begin{definition}\index{Lagrangian submanifold}
  If \((M^{2n}, \omega)\) is symplectic, \(L \subseteq M\) is \emph{Lagrangian} if \(\dim_\R L = n\) and \(\omega|_L = 0\), i.e.\ \(i^* \omega = 0\) where \(i: L \embed M\).
\end{definition}

\begin{eg}\leavevmode
  \begin{enumerate}
  \item \(S^1 \subseteq \Sigma\) a surface is Lagrangian as \(\Omega^2(S^1) = 0\).
  \item \((S^1)^n \subseteq (\C^n, \omega_0)\), the Clifford torus\index{Clifford torus}. Note that the radius of \(S^1\)'s can be arbitrarily small so by Darboux, these exist in any symplectic manifold.
  \end{enumerate}
\end{eg}

\paragraph{Cotangent bundle}

\(\R^{2n} = T^*\R^n\). Let \(q_j\) be coordinates on \(\R^n\), \(p_j = \d q_j\). Together they give coordinates on \(\R^{2n}\). Together they give a 1-form
\[
  \lambda = \sum p_j \d q_j \in \Omega^1(T^*\R^n)
\]
and \(\d \lambda = \omega_0\). A diffeomorphism \(q \mapsto q'\) of \(\R^n\) induces a diffeomorphism of \(T^*\R^n\) via pullback which preserves \(\lambda\): \(\sum p_j \d q_j \mapsto \sum p_j' \d q_j'\), where \(p_j'\) are dual to \(q_j'\). (It doesn't work in general). For a manifold, patch \(\lambda\)'s together to get \(\lambda_{\mathrm{can}} \in \Omega^1(T^*X)\) with \(\d \lambda_{\mathrm{can}}\) symplectic.

Coordinate-free description: for \(v \in T_{(q, p)}T^*X\),
\[
  \lambda_{\mathrm{can} (q, p)}(v) = \langle p, D \pi(v) \rangle
\]
where \(\pi: T^*X \to X\).

\begin{ex}
  If \(\sigma: M \to T^*M\) is a 1-form then
  \[
    \sigma^* \lambda_{\mathrm{can}} = \sigma
  \]
  where on LHS we interpret \(\sigma\) as a morphism and on RHS we treat it as an element of \(\Omega^1(M)\).
\end{ex}

Note \(M \subseteq (T^*M, \d \lambda_{\mathrm{can}})\), the zero section, is Lagrangian. It turns out to be the prototype for Lagrangian submanifolds.

\begin{theorem}[Weinstein tubular neighbourhood theorem]\index{Weinstein tubular neighbourhood theorem}
  If \(L \subseteq (M, \omega)\) is compact Lagrangian, then there is a tubular neighbourhood \(U(L)\) of \(L\) in \(M\) which is symplectomorphic to a neighbourhood \(U'(L) \subseteq T^*L\) of the zero section.
\end{theorem}

\begin{proof}
  Choose an \(\omega\)-compatible almost complex structure \(J\) and let \(g = \omega(\cdot, J \cdot)\). Note the \(g\)-orthogonal complement to \(T_qL \subseteq T_qM\) is \(JT_qL\). Let \(\Phi: T^*M \to TM\) be induced by \(g\): \(g(\Phi(f), v) = f(v)\). Define
  \begin{align*}
    \varphi: T^*L &\to M \\
    (q, f) &\mapsto \exp_q(J \Phi_q(f))
  \end{align*}
  Check
  \[
    D \varphi_{(q, 0)}(v, f) = v + J \Phi_q(f)
  \]
  for \((v, f) \in T_qL \oplus (T_qL)^* \cong T_{(q, 0)}(T^*L)\).
  \begin{align*}
    (\varphi^* \omega_M)(_{(q, 0)}((v, f), (v', f'))
    &= \omega_M|_q(v + J\Phi f, v' + J\Phi f') \\
    &= \omega_M|_q(v, J \Phi f') - \omega_M(v', J\Phi f) \\
    &= g|_q(v, \Phi f') - g|_q(v', \Phi f) \\
    &= f'(v) - f(v') \\
    &= (\d \lambda_{\mathrm{can}})_{(q, 0)} ((v, f), (v', f'))
  \end{align*}
  so \(\varphi: T^*L \to M\) is such that \(\varphi^* \omega_M\) and \(\d \lambda_{\mathrm{can}}\) agree on the zero section, i.e.\ \(T(T^*L)|_L\). The Poincaré lemma now gives \(\sigma \in \Omega^1(T^*L)\) for which \(\d \lambda_{\mathrm{can}} - \varphi^* \omega_M = \d \sigma\) on \(U(L)\) an open neighbourhood of the \(0\)-section, \(\sigma = 0\) on \(0\)-section. Apply Moser's trick to the family of forms \(\omega_t = (1 - t) \varphi^* \omega_M + t \d \lambda_{\mathrm{can}}\).
\end{proof}

\begin{corollary}
  A neighbourhood of a Lagrangian only depends on the smooth topology of \(L\) as a symplectic manifold.
\end{corollary}

\begin{ex}
  Let \(L \subseteq M^4\) be Lagrangian.
  \begin{enumerate}
  \item \(\chi(L) = - [L]^2\) (because \(T^*L \cong \nu_{L/M}\)).
  \item If \(L \subseteq M\) is homologically trivial, i.e.\ \([L] = 0\), and \(L\) is connected closed then \(L\) is a torus or a Klein bottle.
  \end{enumerate}
\end{ex}

\begin{proposition}
  If \(f: M \to M\) is a diffeomorphism then \(f^*\omega = \omega\) if and only if \(\Gamma_f \subseteq (M \times M, \omega \oplus -\omega)\) is Lagrangian.
\end{proposition}

\begin{proof}
  \(f^*\omega = \omega\) if and only if \(f^*\omega - \omega = 0\) if and only if \(i^*(\omega \oplus -\omega) = 0\).
\end{proof}

\begin{eg}
  The antidiagonal \(\Gamma_{\id}\) is Lagrangian.
\end{eg}

\begin{proposition}
  If \(f: M \to M\) is antisymplectic, i.e.\ \(f^*\omega = - \omega\) then \(\mathrm{Fix}(f)\), the fixed points of \(f\), is Lagrangian where smooth.
\end{proposition}

\begin{eg}
  Complex conjugation on \(\C^n\) or \(\C\P^n\) is antisymplectic. The fixed points are \(\R^n\) and \(\R\P^n\) respectively.

  More generally if a quasi-affine or projective smooth variety \(X(\C)\) (smooth submanifold of \(\C^n\) or \(\C\P^n\) cut out by polynomial equations) is defined over \(\R\), then \(X(\R) \subseteq X(\C)\) is a Lagrangian submanifold where smooth.
\end{eg}

\begin{corollary}
  A neighbourhood of \(\id \in \Symp(M)\) where \(M\) compact is homeomorphic to a neighbourhood of \(0\) in the space of closed 1-forms on \(M\). In particular \(\Symp(M)\) is locally path connected.
\end{corollary}

\begin{proof}
  If \(f\) is close to \(\id\) then \(\Gamma_f \subseteq M \times M\) is close to the Lagrangian antidiagonal. By Weinstein \(\Gamma_f \subseteq U\) for \(U \subseteq T^*M\) a neighbourhood of the zero section of \(M\). Moreover the proof of Weinstein shows \(\Gamma_f\) gives a section of \(T^*M\) (near the \(0\)-section). This means that \(\Gamma_f \subseteq T^*M\) can be thought of as the graph of a 1-form, say \(\sigma\).
  \[
    \sigma^* \d \lambda_{\mathrm{can}} = \d \sigma^* \lambda_{\mathrm{can}} = \d \sigma
  \]
  so \(\Gamma_f\) is Lagrangian if and only if \(\sigma\) is closed.
\end{proof}

\begin{corollary}
  Suppose \((M, \omega)\) is compact and \(H^1_{\mathrm{dR}}(M) = 0\). Then any \(f \in \Symp(M)\) which is \(C^1\) close to \(\id\) has at least \(2\) fixed points.
\end{corollary}

\begin{proof}
  If \(f\) is \(C^1\) close to \(\id\) then \(\Gamma_f \subseteq T^*M\) can be written as the graph of a closed \(1\)-form \(\sigma\) (\(C^1\) is enough. See proof of Weinstein). But \(\sigma = \d h\) as \(H^1_{\mathrm{dR}}(M) = 0\). \(p \in \mathrm{Fix}(f)\) if and only if \(\d h_p = 0\), if and only if \(p \in \mathrm{Crit}(h)\). \(M\) is compact so \(h\) has at least 2 critical points, a minimum and a maximum.
\end{proof}

\begin{proposition}
  Suppose \((M, \omega)\) is connected. Then \(\Symp(M, \omega)\) acts transitively on points.
\end{proposition}

\begin{proof}
  \(M\) is path-connected so enough to work in a single Darboux chart. Let \(x \in \R^{2n}\). Translation from \(0\) to \(x\) is symplectic and is the time \(1\) flow of the constant vector field \(X = 0x\). Now need to cut out a chart, which we do by passing to forms. Let \(\sigma = \iota_X \omega_0\). Then
  \[
    \d \sigma = \d \iota_X \omega_0 = \mathcal L_X \omega_0 = 0
  \]
  so \(\sigma\) is exact as \(H^1(\R^{2n}) = 0\). Say \(\sigma = \d f\) for \(f \in C^\infty(\R^{2n})\). Now pick a suitable cut-off function \(\psi\) and replace \(f\) with \(\psi f\), \(X\) with \(Y\) such that \(\iota_Y \omega_0 = \d (\psi f)\).
\end{proof}

Stengthening of Darboux chart.

\begin{theorem}[Gromov-Lees]
  There is a Lagrangian immersion of \(L\) into \(\C^n \cong \R^{2n}\) if and only if \(TL \otimes_\R \C \cong L \times \C^n\) as complex vector bundles.
\end{theorem}

\begin{proof}
  We prove only if and the other direction is beyond the scope of the course so omitted. Note \(V \leq (\C^n, \omega_0)\) is a Lagrangian subspace if and only if \(V \perp i V\) (with respect to standard metric) and \(\dim_\R V = n\) (recall this is also the observation we used for Weinstein). An immersion \(\iota: L \to \C^n\) is Lagrangian if and only if \(\Im (D \iota_x) \perp i \Im(D \iota_x)\) for all \(x \in L\). This gives a map
  \begin{align*}
    T_xL \otimes \C &\to \C^n \\
    v \otimes (a + ib) &\mapsto a D \iota_x(v) + i b D \iota_x (v)
  \end{align*}
  on each fibre and varying over \(x\) induces the trivialisation.

  The other direction is hard. Uses \(h\)-principle.
\end{proof}

\begin{proposition}
  If \(W \subseteq (M, \omega)\) is an isotropic submanifold, i.e.\ \(\omega|_W = 0\), then a neighbourhood of \(W\) is determined symplectically by the smooth topology of \(W\) and the bundle \(TW^\perp/TW\) (which is trivial in Lagrangian case).
\end{proposition}

\begin{proof}
  Example sheet 2.
\end{proof}

\begin{lemma}
  Suppose \(W \immersion X\) is an isotropic immersion and \(\dim W < \frac{1}{2} \dim M\) then isotropic perturbation of \(W\) will be embedded generically.
\end{lemma}

\begin{proof}
  General position argument: flow one branch of \(W\) by a compactly supported vector field near individual self-intersection points to remove them locally.

  Slogan: stengthen a smooth perturbation to be isotropic.
\end{proof}

\begin{corollary}
  If \(L\) compact has a Lagrangian immersion in \(\C^n\) then \(L \times S^1\) embeds in \(\C^{n + 1}\).
\end{corollary}

Note by Gromov-Less this is in fact if and only if.

\begin{proof}
  Have isotropic immersion \(L \immersion \C^n \times \C = \C^{n + 1}\). This can be perturbed to get an isotropic embedding \(L \embed \C^{n + 1}\). Note \(TL^\omega/TL\) is trivial and hence we get a symplectic embedding of an open neighbourhood of \(L \subseteq T^*L \times \C\). This containes a Lagrangian \(L \times S^1\) by taking the radius of \(S^1\) to be sufficiently small.
\end{proof}

Fact: every compact orientable \(3\)-manifold \(Y\) is parallelisable. (proof uses characteristic class)

\begin{corollary}
  If \(Y^3\) is compact orientable then we have a Lagrangian immersion \(Y \immersion \C^3\) and a Lagrangian embedding \(Y \times S^1 \embed \C^4\).
\end{corollary}

\begin{remark}
  Any compact three manifold can be Lagrangian immersed in \(\C^3\). On the other hand if \(L^4\) compact is Lagrangian immersible into \(\C^4\) then by Gromov-Rees \(\chi(L) = 0\) so \(b_1 > 0\). Thus \(\pi_1(L)\) is infinite.
\end{remark}

How to remove double point?

\begin{proposition}
  Suppose \(M \supseteq L_1, L_2\) contains 2 Lagrangian submanifolds which meet transversally at a point \(p\). Then there is a Darboux chart \(\varphi: B(\varepsilon) \to M\) such that \(\varphi(0) = p, \varphi^{-1}(L_1) = \R^n \cap B(\varepsilon), \varphi^{-1}(L_2) = i \R^n \cap B(\varepsilon)\).
\end{proposition}

\begin{proof}
  Exercise.
\end{proof}

\paragraph{Polterovich surgery}

Polterovich surgery\index{Polterovich surgery} replaces \(L_1 \cup L_2\) with \(L_\gamma\), local on a neighbourhood of \(p\).

Let \(\gamma: \R \to \C\) smooth, coincide with \(\R_+ \{0\} \cup \{0\} \{0\} \times \R_-\) away from the origin.
\[
  L_\gamma = \{z_j = \gamma a_j: (a_j) \in S^{n - 1}\}
\]
where \(z_j\) is the complex coordinate, \(a_j \in S^{n - 1} \cap \R^n \subseteq \C^n\). This is a Lagrangian handle.

\begin{eg}
  
For \(n = 1\), \(L_\gamma = \{z = \gamma a: a\in S^0 = \{\pm 1\}\}\). Away from a neighbourhood of \(0\), \(L_\gamma\) agrees with \(L_1 \cup L_2\).

For \(n = 2\), suppose \(\gamma = (\gamma_1(t), \gamma_2(t)) \in \R^2 \cong \C, S^1 = (\cos \theta, \sin \theta)\). Then
\[
  L_\gamma = \{(\gamma_1 \cos \theta, \gamma_2 \cos \theta, \gamma_1 \sin \theta, \gamma_2 \sin \theta)\} \subseteq \R^4 \cong \C^2.
\]
If \(\gamma_1 = 0\) then we get \(i\R^n\) minus a neighbourhood of \(0\). If \(\gamma_2 = 0\) then we get \(\R^n\) minus a neighbourhood of \(0\).
\end{eg}

This depends on the ordering of \(L^1\) and \(L^2\) (corresponding to the canonical order \(\R^n, i\R^n\)), doesn't depend on the choice of \(\gamma\), as long as \(\gamma\) agrees with half-axes outside the ball and \(\gamma \cap (-\gamma) = \emptyset\).

\begin{corollary}
  If \(L_1, L_2\) are Lagrangian submanifolds which meet transversally at a point, there is an embedding Lagrangian submanifold diffeomorphic to the connected sum \(L_1 \# L_2\).
\end{corollary}

General case: can perform surgery separately at any finite number of transverse intersections of Lagrangians. Topologically, each surgery replaces \(B^n \amalg B^n\) with \(\R \times S^{n - 1}\).

\begin{eg}
  If \(L^n \immersion M^{2n}\) Lagrangian with a single double point, get Lagrangian embedding of \(L \# (S^1 \times S^{n - 1}) \embed M\).
\end{eg}

\begin{corollary}
  If \(Y\) is a compact orientable 3-manifold then for some \(k \geq 0\) there is a Lagrangian embedding of \(Y \# k(S^1 \times S^2)\) into \(\C^3\).
\end{corollary}

Highly restrictive

\begin{definition}[prime manifold]\index{prime manifold}
  A closed manifold \(M^n\) is \emph{prime} if it can't be written as \(M = M_1 \# M_2\) unless \(M_1\) or \(M_2\) is \(S^n\).
\end{definition}

\begin{theorem}[Fukaya]
  A compact prime orientable 3-manifold has a Lagrangian embedding in \(\C^3\) if and only if and only if it is diffeomorphic to \(S^1 \times \Sigma_g\).
\end{theorem}

Fix a primitive \(\theta\) of \(\omega_0\) in \(\C^n\), i.e.\ \(\d \theta = \omega_0\). If \(L\) is Lagrangian then \(\omega_0|_L = 0\) so \([\theta|_L] \in H^1(L)\). (By Stoke's this measures the symplectic area of disc with boundary on \(L\)). Define \(L\) to be exact if \([\theta|_L] = 0\).

\begin{theorem}[Gromov]
  There is no compact exact Lagrangian in \(\C^n\).
\end{theorem}

\subsection{Symplectic submanifolds}

\begin{theorem}[Gromov]
  Fix symplectic manifold \((V, \omega_V)\) compact and \((X, \omega_X)\) with \(\dim V \leq \dim X - 4\). Suppose we are given a smooth embedding \(f: V \embed X\) such
  \begin{itemize}
  \item \(f^*[\omega_X] = [\omega_V] \in H^2(V)\),
  \item \(Df\) is symplectic through bundle maps \(TV \to TX\) to a fibrewise symplectic embedding.
  \end{itemize}
  The \(f\) is smoothly isotopic to an embedding \(\tilde f: V \to X\) such that \(\tilde f^*\omega_X = \omega_V\).
\end{theorem}

\begin{proof}
  Omitted. This is an instance of h-principle.
\end{proof}

This fails in general for codimension 2 but

\begin{theorem}[Donaldson]
  If \([\omega] \in H^2(X; \Z)\) then for \(k \gg 0\) there are symplectic submanifolds representing \(PD (k [\omega])\).
\end{theorem}

Idea: \([w] \in H^2(X; \Z)\) represents a complex line bundle \(\pi: L \to X\) with \(c_1(L) = [\omega]\). If \(s: X \to L\) is a section of \(\pi\), cleanly intersecting \(0\)-section, then
\[
  s^{-1}(0) = PD [c_1(L)] = PD[\omega].
\]
(By Sard's theorem, \(s^{-1}(0)\) is a \((2n - 2)\) dimensional manifold). If we use instead \(L^{\otimes k}\) then \(s^{-1}(0) = PD (k[\omega])\). Donaldson's idea is to construct, for sufficiently large \(k\), sections which are ``approximately holomorphic'' (they satisfy Cauchy-Riemann equations up to an error term). The error term is small enough that \(s^{-1}(0)\) is symplectic.

c.f.\ ample/very ample line bundles in algebraic geometry.
\begin{align*}
  \sigma: X &\to \P(H^0(X, L^{\otimes k})^*) = \P^N \\
  x &\mapsto [\varphi_x: x \mapsto s(x)]
\end{align*}
\(\sigma\) is injective for \(k \gg 0\). If that case, \(\P^{N - 1} \cap \sigma(X)\) represents \(PD(k c_1(L))\).

\subsection{Blow-ups}

\begin{definition}[blow-up]\index{blow-up}
  The blow-up of \(0\) in \(\C^n\) is
  \[
    Z = \{(z, \ell) \in \C^n \times \P^{n - 1}: z \in \ell\}
  \]
  together with two projections \(\pi: Z \to \C^n, p: Z \to \P^{n - 1}\).
\end{definition}

\(\pi\) is one-to-one away from \(0\) and over \(0\) the fibre is \(\P^{n - 1}\). \(p: Z \to \P^{n - 1}\) is the tautological bundle. \(c_1 = - PD[\P^{n - 2}]\).

\begin{note}
  Our choice of Kähler form on \(\P^n\) is normalised so that \(\omega(\P^1) = \pi\).
\end{note}

Define
\[
  \omega_\lambda = \pi^* \omega_{\C^n} + \lambda^2 p^* \omega_{\P^{n - 1}}.
\]

\begin{lemma}
  For \(\lambda > 0\), \(\omega_\lambda\) is Kähler and for \(\delta > 0\), let
  \[
    Z(\delta) = \{(z, \ell) \in Z: |z| \leq \delta\}.
  \]
  Then \((Z(\delta) \setminus Z(0), \omega_\lambda)\) is symplectomorphic to \((B(\sqrt{\lambda^2 + \delta^2}) \setminus B(\lambda), \omega_0)\).
\end{lemma}

\begin{proof}
  Recall our definition of Fubin-Study metric: for the natural projection \(\Phi: \C^n \setminus \{0\} \to \P^{n - 1}\), have
  \[
    \Phi^*\omega_{\P^{n - 1}} = \frac{i}{2} \p \conj \p \log |z|^2.
  \]
  Let
  \[
    \mu_\lambda = \frac{i}{2} \p \conj \p (|z|^2 + \lambda^2 \log |z|^2).
  \]
  On \(Z(\delta) \setminus Z(0)\), \(\pi^* \mu_\lambda = \omega_\lambda\). Define a bijection
  \begin{align*}
    F: \C^n\setminus \{0\} &\to \C^n\setminus B(\lambda) \\
    z &\mapsto \frac{z}{|z|} \sqrt{|z|^2 + \lambda^2}
  \end{align*}
  Note \(F^*\omega_0 = \mu_\lambda\). Using the fact that \(\omega_0\) is Kähler, easy to get plush for \(\mu_\lambda\) so Kähler.
\end{proof}

\begin{definition}[blowup]\index{blowup}
  The \emph{weight \(\lambda\) blowup} of \((M, \omega)\) at \(p\) is a symplectic embedding \(\varphi: B(\sqrt{\lambda^2 + \delta^2}) \embed M, \varphi(0) = p\), and
  \[
    \widetilde M = (M \setminus \im \varphi) \cup Z(\delta)
  \]
  with \(\tilde \omega_M = \omega_{M \setminus \im \varphi} \cup \omega_\lambda\).
\end{definition}

\begin{note}\leavevmode
  \begin{enumerate}
  \item \(\Vol(\tilde M) = \Vol(M, \omega) - \Vol(B(\lambda))\) so the volume decreases under blowup.
  \item \([\tilde \omega_\lambda] = \pi^*(\omega) - \pi \lambda^2 PD(E) \in H^2_{\mathrm{dR}}(M)\) where \(E = Z(0) \cong \P^{n - 1}\) is the exceptional divisor.
  \end{enumerate}
\end{note}

\subsection{Fibre sums}

\begin{lemma}
  If \((Q^{2n - 2}, \omega_Q) \embed (M^{2n}, \omega_M)\) is a closed symplectic submanifold of \(M\) then \((Q^{2n - 2} \times B(\varepsilon), \omega_q \oplus \omega_0) \embed (M^{2n}, \omega_M)\) symplectically if and only if \(\nu_{Q/M}\) is symplectically trivial (i.e.\ \(c_1(\nu_{Q/M}) = 0\)).
\end{lemma}

Already proved!

Consider
\begin{align*}
  \psi: B(\varepsilon) \setminus \{0\} &\to B(\varepsilon) \setminus \{0\} \\
  (r, \theta) &\mapsto (\sqrt{\varepsilon^2 - r^2}, - \theta)
\end{align*}
(turn inside out and then flip orientation). Check
\[
  \psi^*\omega_0 = \psi^*(r \d r \w \d \theta) = r \d r \w \d \theta.
\]
Thus \(\psi\) is area preservaing and ``turns the annulus inside out''.

Suppose \(Q^{2n - 2} \embed M_i^{2n}\) symplectically, \(i = 1, 2\), \(Q\) closed symplectic submanifold such that \(\nu_{Q/M_i}\) is symplectically trivial. We can define

\begin{definition}[fibre sum]\index{fibre sum}
  The \emph{fibre sum} of \(M_1\) and \(M_2\) along \(Q\) is
  \[
    M_1 \#_Q M_2 = (M_1 \setminus Q) \cup_{\id \times \psi} (M_2 \setminus Q)
  \]
  where \(\id \times \psi: M_1 \setminus Q \supseteq Q \times B(\varepsilon)^* \to Q \times B^*(\varepsilon) \subseteq M_2 \setminus Q\).
\end{definition}

\begin{note}
  As \(\nu_{Q/M_i}\) is trivial, \(Q \embed M_1 \# M_2\) symplectically.
\end{note}

More generally, suppose \(Q \embed M_i\) symplectically and \(c_1(\nu_{Q/M_1}) = -c_1(\nu_{Q/M_2})\), then we can define \(M_1 \#_Q M_2\) completely analogously.

Note that on each local chart of \(Q\) we use \(\id \times \psi\). These patch together to give a symplectomorphism
\[
  \begin{tikzcd}
    \nu_{Q/M_1} \setminus \{0\} \ar[d] \ar[r] & \mathcal L \setminus \{0\} \ar[d] \\
    Q \ar[r] & Q
  \end{tikzcd}
\]
for some complex line bundle to be determined. This flips the signs of intersections of section and zero section, so \(c_1(\mathcal L) = - c_1(\nu_{Q/M_1})\).

\begin{remark}
  There is a hidden choice: the \emph{framing} of \(Q\) needn't be unique. Fo example if \(\nu_{Q/M}\) is trivial, the choices of symplectic trivialisations of \(\nu_{Q/M}\) are in one-to-one correspondence with \(H^1(Q; \Z)\) (\(\nu_{Q/M}\) is trivial and choices of trivialisation up to homotopy is \([Q, U(1)] = [Q, S^1] = H^1(Q; \Z)\)). More generally, maps \(\nu_{Q/M} \embed U(Q)\) can differ (up to homotopy) be an element of \(H^1(Q; \Z)\).
\end{remark}

\begin{eg}\leavevmode
  \begin{enumerate}
  \item Suppose \(X^4 \supseteq E\) where \(E\) a symplectic sphere of self-intersection number \(-1\). Pick \(Y^4 \supseteq S^2\) symplectic sphere of self-intersection number \(+1\), for example \(H \subseteq \P^2\) a hyperplane. Then we can form \(X^4 \#_{E} \C\P^2\). This is called the \emph{blowdown} of \(X\) along \(E\).

    Note if \(\widetilde W\) is the blowup of \(W^4\) at \(p\) then \(\widetilde W \#_E \C\P^2 \cong W\).
  \item Recall if \(C \subseteq \P^2\) smooth degree \(d\) curve then \([C] = d [\P^1]\), \([C] \cdot [C] = d^2, g(C) = \frac{(d - 1)(d - 2)}{2}\). Suppose \(X^4 \supseteq Q\) symplectic sphere of self-intersection \(-4\), \(\C\P^2 \supseteq C\) curve of degree \(2\). Then can form \(X^4 \#_{Q/C} \C\P^2\).

    Claim \(\C\P^2 \setminus C \cong D_\varepsilon^*(\R\P^2)\), (truncated) cotangent bundle of Lagrangian \(\R\P^2\) (becuas given \(C \subseteq \C\P^2\) of genus \(0\), exists Lagrangian \(\R\P^2\) which doesn't intersect it. Check ``no other topology'')

    Replcaed a symplectic sphere with a Lagrangian \(\R\P^2\). This is a purely symplectic operation --- we can't usually do this in the algebraic geometry world.
  \end{enumerate}
\end{eg}

\subsection{Lefschetz principles}

\begin{definition}[Lefschetz pencil]\index{Lefschetz pencil}
  A \emph{Lefschetz pencil} on a closed oriented 4-manifold \(X\) is a smooth map \(f: X \setminus \{b_1, \dots, b_k\} \to \P^1\) such that
  \begin{itemize}
  \item \(Df\) is onto except at a finite collection of points \(\{p_1, \dots, p_m\} \subseteq X \setminus \{b_1, \dots, b_k\}\),
  \item Near the \(b_i\) (\(p_j\) respectively) there are central local complex coordinates \(z, w \in B_\varepsilon(0) \subseteq \C\) such that \(f(z, w) = \frac{z}{w}\) (\(f(z, w) = z^2 + w^2\) respectively)
  \end{itemize}
  The \(b_i\)'s are called \emph{base points}, \(p_i\)'s \emph{critical points}.
\end{definition}

To get such a form, we use the complex Morse lemma

\begin{lemma}
  If \(U \subseteq \C^n\) is an open subset, \(f: U \to \C\) homomorphic maps \(0\) to \(0\) with a non-degenerate critical point at \(0\). Then exists local coordiantes \(z_1, \dots, z_n\) such that \(f(z_1, \dots, z_n) = \sum_{i = 1}^n z_i^2\).
\end{lemma}

Compare with the real Morse lemma which says that we can find coordinates such that \(f(x_1, \dots, x_n) = \sum_{i = 1}^p x_i^2 - \sum_{i = p + 1}^n x_i^2\). In the complex case we can run the same proof and absorb \(-1\) into the coordinates.

Suppose \(t \in \P^1\) is not a critical value of \(f\). Then \(\overline{f^{-1}(t)} \subseteq X\) (gluing back \(b_i\)), the fibre of \(t\) is smooth. Around each critical point \(p_j\) of the pencil, the equation for a fibre looks like \(\{z^2 + w^2 = 0\}\), i.e.\ two complex lines intersecting transversally. \(p_j\) is also called \emph{ordinary double point} or \emph{node}.

Note all smooth fibres are closed, orientable and have fixed genus (pick a path avoiding the critical points, then the fibre varies smoothly).

Suppose \(X^4\) compact Kähler, \(L \to X\) a very ample holomorphic line bundle. For the purpose of this course, this means a holomorphic embedding \(i: X \embed \P^N\) uch that \(i^*\mathcal O(1) = L\), where \(O(1)\) is the line bundle such that \(c_1(\mathcal O(1)) = PD[\P^{N - 1}]\). Key fact: the restriction of a generic hyperplane \(\P^{N - 1}\) gives \(s^{-1}(0) \cap X\) for some holomorphic sections \(s\) of \(L\).

Let \(s_1, s_2\) be generic holomorphic sections. Then we can define a rational map \(\pi: X \to \P^1, p \mapsto [s_1(p): s_2(p)]\). \(\{s_1 = s_2 = 0\}\) is a set of finitely many points. These are the base points of \(p\). Among critical points of a holomorphic functions, non-degenerate ones are generic, so by complex Morse lemma we have correct local forms near critical points.

What's the local model near \(b \in \{s_1 = s_2 = 0\}\)? \((s_1, s_2)\) give local holomorphic coordintes

\begin{eg}[pencil of cubics in \(\P^2\)]
  Define
  \begin{align*}
    \P^2 &\to \P^1 \\
    [x: y: z] &\mapsto [x^3 + y^3 + z^3: x^3 + y^3 + z^3 + xyz] = [f: g]
  \end{align*}
  \(f\) and \(g\) are sections of \(\mathcal O(3)\). The fibre
  \[
    \pi^{-1}([\lambda: \mu]) = \{\mu f - \lambda g = 0\} \subseteq \P^2
  \]
  is given by a cubic equation (so torus if smooth). The base points are given by \(\{f = g = 0\}\), i.e.\ \(x^3 + y^3 + z^3 = 0, xyz = 0\), which is a set of 9 points (this can also be seen by \([C] \cdot [C'] = 9\) for \(C, C'\) cubic). Explicitly, they are given by \([0: 1: \xi^i]\) and there cyclic permutations, where \(\xi^3 = -1\).

  When is the set \(\{\mu f - \lambda g = 0\} \subseteq \P^2\) singular? either \(\mu = \lambda\), so \(xyz = 0\) three coordinate lines, 3 critical points. If \(\mu \ne \lambda\), this can be rephrased as \(\{x^3 + y^3 + z^3 + axyz = 0\}\). The critical points are given by the zeroes of all three partial derivatives. We get \(xyz = 0\) (already seen) or \(a^3 + 27 = 0\). Thus we get 3 more cricial fibres \(a = 3 \xi^i\). We can factorise \(x^3 + y^3 + z^3 - 3xyz\), which is again three lines.

  In either case, there are three base points on each line.

  (pic)

  Near a base point, maps \((z, w) \mapsto \frac{z}{w}\) (this is the direction of the line). Then we get an honest map to \(\P^1\) by blowing up each of the base points:
  \[
    \pi: E(1) = \C\P^2 \# q \conj{\C\P^2} \to \P^1
  \]
  \(E(1)\) is the \emph{rational elliptic surface}.
  \begin{itemize}
  \item Smooth fibre \(C\) with \(C \cdot C = 0\) inside \(E(1)\) (fibration is locally trivial so can push \(C\) of itself).
  \item \(\pi_1(E(1)) = 0\).
  \item Example sheet 3: \(\pi_1(E(1) \setminus C) = 0\).
  \end{itemize}
\end{eg}

\begin{theorem}
  Suppose \(X^4\) connected closed oriented is the total space of a Lefschetz pencil with at least one basepoint. Then \(X\) admits a symplectic structure.
\end{theorem}

\begin{proof}
  We first outline the general strategy. It is enough to construct a symplectic form on \(\tilde X\) obtained by blowing up basepoints. More precisely, near basepoint \(b\), we have local holomorphic coordinates \((z, w)\) such that \(\pi: X \to \P^1\) is given by \((z, w) \mapsto \frac{z}{w}\). Replace \(D_\varepsilon(0)\) with \(Z(\varepsilon)\) (local blowup model) at every basepoint to get \(\tilde X\). Now \(\pi\) induces a well-defined map \(\pi: \tilde X \to \P^1\). Each basepoint is replaced with exceptional divisor \(E\) (\(\cong \P^1\)), which gives a section of \(\pi\).

  We will construct a symplectic form on \(\tilde X\) such that each smooth fibre is symplectic (ditto symplectic fibre away from critical points). We'll also see that for any \(N \geq 0\), \(\omega + N \pi^* \omega_{\P^1}\) is also symplectic. Taking \(N\) sufficiently large ensures that the sections coming from the basepoints are all symplectic. Now blow each of these down using fibre connected sum to get a symplectic form on \(X\).

  Working on \(\tilde X\). Claim we can find \(\eta \in \Omega^2(\tilde X)\) closed such that
  \begin{itemize}
  \item \(\eta|_F\) is symplectic on neighbourhood of any smooth point points of a fibre.
  \item near a critical point \(p_i\), \(\eta|_{U_i(p_i)} = \frac{i}{2}(\d z \d \conj z + \d w \d \conj w)\), where \(U_i(p_i)\) open in \(\tilde X\) (not the fibre), in local coordinates.
  \end{itemize}

  We postpone the proof. Let \(\omega = \eta + k \pi^* \pi_{\P^1}\). Claim for \(k \gg 0\), \(\omega\) is symplectic. This automatically gives us the sought after form.

  \begin{proof}[Proof of step 3]
    Near a smooth point \(T_x \tilde X = \ker D\pi \oplus (\ker D\pi)^\eta\). \(\ker D\pi = T_xF\) is called the vertical subspace, and the symplectic complement is called a choice of horizontal space. The matrix form for \(\eta + k \pi^* \omega_{\P^1}\) is
    \[
      \begin{pmatrix}
        \eta|_F & 0 \\
        0 & \eta|_{\mathrm{hor}} + k \pi^* \omega_{\P^1}
      \end{pmatrix}
    \]
    Since \(\tilde X\) is compact, we can choose \(k\) sufficiently large such that the this is symplectic.

    Near singular points, the local model is \(\pi: \C^2 \to \C, (z, w) \mapsto z^2 + w^2\). Check
    \[
      (\omega_{\mathrm{std}} + k\pi^* \omega_{\P^1})^2 = (1 + k |(z, w)|^2) \Vol > 0
    \]
    so symplectic.
  \end{proof}

  \begin{proof}[Proof of step 2]
    Model fibre \(F\). Pick a symplectic \(2\)-form \(\sigma \in \Omega^2(F)\) with area 1. Claim exists \(\xi \in \Omega^2(\tilde X)\) closed such that \(\int_{F'} \xi = 1\) for all fibre \(F'\). This is Poincaré duality: the expression says \([\xi] \cap [F'] = 1\). This requires \([F] \ne 0\), which is OK if \(\#\{b_j\} > 0\) as \(E \pitchfork F = \{pt\}\), \(E\) section.

    Pick open sets \(\tilde X\), \(\{U_\alpha\}_{\alpha \in A}, \{V_i\}_{i \in I}\) such that
    \begin{itemize}
    \item \(U_\alpha \cong F \times B_\alpha \to B_\alpha\) for some balls \(B_\alpha \subseteq \P^1\) away from the critical points.
    \item \(V_i \cong B_i^4 \to \C (z^2 + w^2)\) near cricical points.
    \end{itemize}

    \begin{itemize}
    \item On \(U_\alpha\), \(f: F \times B_\alpha \to F\), \(f^*\sigma - \xi = \d \lambda_\alpha\) for some \(\lambda_\alpha \in \Omega^1(U_\alpha)\).
    \item On \(V_i\), \(\pi: B_i^4 \to \C\) smooth fibre, locally \(z^2 + w^2 = a\), \(a \ne 0\). Standard symplectic form restricts to a symplectic form. \(\d \theta_i = \omega_{\mathrm{std}}\).
    \end{itemize}

    Pick partitions of unity subordinate to \(\pi(\{U_\alpha\} \cup \{V_i\})\), say \(\{\varphi_\alpha, \psi_i\}\). Set (?)
    \[
      \eta = \xi - \d (\varphi_\alpha \compose \lambda_\alpha) + d ((\psi_i \compose \pi) \theta_i) \cdot N
    \]
  \end{proof}
\end{proof}

As a corollary

\begin{theorem}
  If \((X^4, \omega)\) is symplectic and \([\omega] \in H^2(X, \Z)\) then for all \(k \gg 0\) there exists a Lefschetz pencil \(X\) with fibre Poincaré dual to  \(k[\omega]\).
\end{theorem}

\begin{proof}
  Omitted.
\end{proof}

Thus every integral symplectic 4 manifold is a Lefschetz pencil.

\subsection{Grompf's theorem}

\begin{theorem}[Grompf]\index{Gompf's theorem}
  If \(G = \langle g_1, \dots, g_n | r_1, \dots, r_\ell \rangle\) is a finitely presented group then there exists \((Y_G, \omega)\) a symplectic 4 manifold with \(\pi_1 Y_G = G\).
\end{theorem}

This is really saying the symplectic category is big enough: the topological version is a basic result in algebraic topology. Symplectic 2 manifolds are surfaces so symplectic 4 manifolds are the simplest objects that the theorem could possibly hold.

\begin{proof}
  The proof strategy is as follow. From example sheet 3 \(\pi_1(E(1) \setminus T) = 0\) and \(c_1(\nu_{T(E(1))}) = 0\). We'll use this to kill off generators in a bigger space. Consider \((T^2 \times \Sigma_g, \omega_{T_2} \oplus \omega_{\Sigma_g})\). Find in here disjoint symplectic tori \(T_i\)'s and construct the fibre sum
  \[
    M = (T^2 \times \Sigma_g) \#_{T_1} E(1) \#_{T_2} E(1) \#_{T_3} \dots \#_{T_k} E(1)
  \]
  and then \(\pi_1M = \pi_1(T^2 \times \Sigma_g)/\langle \pi_1(T_i)\rangle\).

  We first find disjoint tori. Let \(M = T^2 \times \Sigma_g\), regarded as a fibration with base \(\Sigma_g\) (pic). Then
  \[
    \pi_1(M) = \langle u \rangle \times \langle v \rangle \times \langle a_1, b_1, \dots, a_g, b_g | \prod [a_i, b_i] \rangle.
  \]
  Candidate tori: given loops \(\alpha_1, \dots, \alpha_k \in \Sigma_g\) and loops \(u_1, \dots, u_k\). Then \(\alpha_i \times u_i \subseteq M\) are (Lagrangian) tori.

  Step 1: can we find enough tori so that \(\pi_1(M)/\langle \pi_1 T_i\rangle\) is the free group?

  First let \(T_0 = T\) a torus fibre. It is an honest symplectic torus and will kill off \(\langle u \rangle \times \langle v \rangle\). Next consider tori given by \(u_i \times b_i\). This will kill off generators \(b_i\) in \(\pi_1M\). Claim the resulting space has free fundamental group: killing off \(b_i\) is the same as adding handlebodies to the ``holes'' on \(\Sigma_g\) and the resulting space is homotopic to \(g\) loops attached to \(S^2\), whose fundamental group is indeed \(F_g\).

  For a finitely generated group \(G = F_g/ \langle r_1, \dots, r_\ell \rangle\), for each \(r_i\) we find a loop \(\alpha_i \in \Sigma_g\) of \(a_i\) which spell out the relation. We run into a problem: \(\alpha_i\)'s may not be embedded (as a simple example, \(a_1^2\) in \(\Sigma_1\)). We can resolve the problem by adding more generators: for example \(F_1/\langle a_1^2 \rangle = F_2/\langle a_1a_2, a_1a_2^{-1} \rangle\) (pic). Note that it does not matter if these two loops intersect, as we can move them off to different fibres.

  So now we should be convinced that there are many Lagrangian tori \(T_i \subseteq T^2 \times \Sigma_g\) such that \(\pi_1(T^2 \times \Sigma_g)/\pi_1(T_i) = G\).

  Remains to show exists \(\omega \ne \omega_{T^1} \oplus \omega_{\Sigma_g}\) making \(T_i\) symplectic. Oftentimes being Lagrangian is a ``closed'' condition on the space of symplectic forms. For example \(S^2 \subseteq T^*S^2\) is certainly Lagrangian. Consider \(T\P^1\) which is homeomorphic to \(T^*S^2\). \(T\P^1\) has symplectic form compatible with standard complex structure, and \(\P^1\) is complex so symplectic.

  Claim exists \(\omega\) making the \(T_i\)'s symplectic.

  Exists a closed form \(\beta \in \Omega^2(M, \R)\) such that the pullback \(i^*[\beta] \in \Omega^2(T_i^2, \R)\) is positive. Note that \(\beta\) is in the same class as the usual symplectic form but may not itself by symplectic.

  Idea of proof: take forms \(s_1, s_2\) Poincaré dual to loops \(U_1, \alpha_1\) with \(T^2 = u_1 \times a_i\). Then \(\beta = s_1 \w s_2\).

  Suppose \(\beta\) is such a form on \(M\), WTS we can modify this form so its pullback is symplectic. Since \(T_i\)'s are Lagrangian, we know that their neighbourhood is trivial. Let \(j: B_\varepsilon(0) \times T^2 \embed M\) be a neighbourhood of \(T_i\). The pullback \(j^*(\beta)\) on \(B_\varepsilon(0) \times T^2\) is in the same cohomology class of \(p^*_{T^2}\omega_{T^2}\) (?). By abuse of notation, \(\beta, \omega_{T^2} \in \Omega^2(B_\varepsilon(0) \times T^2)\) with \([\beta] = [\omega_{T^2}]\).

  Let \(\eta = \Omega^1(B_\varepsilon \times T^2)\) with \(\d \eta = \omega_{T^2} - \beta\). Pick \(\rho: B_\varepsilon(0) \to \R\) which is 1 in neighbourhood of \(0\) and \(0\) in a neighbourhood of the boundary. Look at the forms
  \[
    \tilde \beta = \rho \omega_{T^2} + (1 - \rho) \beta - \d \rho \w \eta
  \]
  where the last term is presente to make sure \(\tilde \beta\) is closed:
  \[
    \d \tilde \beta = \d \rho \omega_{T^2} - \d \rho \w \beta - \d \rho \w \d \eta = 0.
  \]
  Substitute \(\beta\) on \(M\) with \(\tilde \beta\). Take \(\omega\) for \(T^2 \times \Sigma_g\) to be \(\omega_{T^2} \oplus \omega_{\Sigma_g} + \varepsilon \beta\) for \(\varepsilon\) small. This is nondegenerate since it is an open condition.
\end{proof}

Rcall that a Lefschetz pencil is a map \(\pi: (\tilde X, \omega, J) \setminus \{b_i\} \to \P^n\) that is
\begin{enumerate}
\item \(J\)-holomorphic,
\item at the crtical points of the map exists honestly holomorphic coordinates such that \(\pi(z_1, z_2) = z_1^2 + z_2^2\).
\end{enumerate}

Goal: understand Lagrangian submanifolds in Lefschetz pencilss. For today let \(F_p = \pi^{-1}(p)\).

Idea: suppose \(\ell \subseteq F_p\) is a Lagrangian and \(\gamma \subseteq \P^1\) is a Lagrangian, can we create \(L\) in \(\tilde X\) based on this data? For example if \(\tilde X = \Sigma_g \times \P^1\) then any product \(\ell \times \gamma\) is Lagrangian. We expect this to work locally in the general case.

First claim \((F_p, \omega|_{F_p})\) is symplectic. Let \(i: F_p \embed \tilde X\). To check closedness
\[
  \d i^*\omega = i^* \d \omega = 0.
\]
To check nondegeneracy, note that \(F_p\) is a complex submanifold. Take \(v \in TF_p\). Then
\[
  \omega(v, Jv) = g_J(v, v) > 0
\]
so \(\omega\) is nondegerate. Thus the Lagrangian in the fibre component always makes sense. What about the other?

Observe that at \(x \in \tilde X\), the tangent spaces splits as
\[
  T_x \tilde X = \ker D\pi \oplus (\ker D\pi)^\omega.
\]
As a result \(\tilde X \to \P^1\) carries a connection. Notation: given \(\gamma: [0, 1] \to \P^1 \setminus \{\text{critical values of } \pi\}\), let \(f_\gamma: F_{\gamma_0} \to F_{\gamma_1}\) be the induced parallel transport map.

\begin{theorem}
  \(f_\gamma\) is a symplectomorphism.
\end{theorem}

\begin{proof}
  Need to show that if \(v \in T_x \tilde X\) is a horizontal vector then \(\mathcal L_V \omega\) vanishes on the fibre.
  \[
    \mathcal L_V \omega = \iota_V \d \omega + \d \iota_V \omega = \d \eta
  \]
  with \(\eta = \iota_V \omega\). \(\eta\) vanishes on the vertical tangent space \(\ker D\pi\) as
  \[
    \eta(w) = \omega(v, \omega) = 0.
  \]
  Write \(\eta = \pi^*\alpha\) for \(\alpha\) a \(1\)-form on \(\P^1\). Then \(\d \eta(v_1, v_2) = 0\) if \(v_1, v_2 \in \ker D\pi\).
\end{proof}

\begin{corollary}
  If \(\ell \subseteq F_{\gamma(0)}\) is a Lagrangian submanifold for \(\omega|_{F_{\gamma(0)}}\) then \(\ell \times I \subseteq \tilde X\), given by the parallel transport along a curve \(\gamma\), is a Lagrangian submanifold.
\end{corollary}

Observation: if \(\gamma: [0, 1] \in \P^1\) with \(\gamma(1) \in \mathrm{CritVal}(\pi)\), there is still a map from \(F_{\gamma(0)} \to F_{\gamma(1)}\).

\begin{definition}[vanishing path]\index{vanishing path}
  A \emph{vanishing path} is a path \(\gamma: [0, 1] \to \P^1\) with \(\gamma(t) \in \mathrm{CritVal}(\pi)\) if and only if \(t = 1\).

  The \emph{vanishing cycle} is \(V_{\gamma(0)} = f_\gamma^{-1}(p)\) for \(p \in \mathrm{Crti}(\pi)\), where \(f_\gamma\) is the parallel transport map.
\end{definition}

\begin{proposition}
  \(V_p\) is a sphere.
\end{proposition}

\begin{proof}
  Look at local model near the critical point: \(\C^2 \to \C, (z_1, z_2) \mapsto z_1^1 + z_2^2\). The regular fibre is \(\{z_1^2 + z^2 = a\}\), \(a \ne 0\) and the singular fibre is the union of two lines. Observe that \(\pi(\R \times \R) = \R\). \(\R \times \R\) is Langrangian and therefore the image is under parallel transport. Thus \(V_p = \{z_1^2 + z_2^2 = 1\}, z_1, z_2 \in \R\). Note that we can increase the dimension of fibration \(z_1, z_2, z_3, \dots\).
\end{proof}

Since parallel tarnsport along curves gives symplectomorphisms of fibres, a loop \(\gamma\) passing through \(a\) gives \(f_\gamma: F_a \to F_a\). Claim if \(\gamma\) is contractible in \(\P^1 \setminus \mathrm{CritVal}(\pi)\) then \(f_\gamma\) is Hamiltonian isotopy.

Question: what happens to \(F_a\) if \(\gamma\) travels around a critical point? We study an easier picture first: consider a Lefschetz fibration \(\C \to \C, z \mapsto z^2\). Then over the regular value \(1\) sits two fibres, and the parallel transport induces the (only) nontrivial symplectomorphism that transposes them. (pic)

Consider next \(\C^2 \to \C\). The symplectomorphism twists the hyperboloid.

Now consider the projection \(z_2\)

In general

\begin{theorem}
  There exists a symplectomorphism of \(T^*S^n \to T^*S^n\) fixing the boundary, \(\tau_{S^n}\) the Dehn twist.
\end{theorem}

Idea: geodesic flow on \(T^*S^n\) is a symplectomorphism which fixes \(S^n\). Time \(\pi\) flow is antipodal.

For \(n \geq 2\), \(\tau_{S^n}^2\) is isotopic to identity but not symplectic isotopic.



\printindex
\end{document}

% Text
% Cannas da Silva, Lectures on Symplectic Geometry
% McDuff-Salamon, Introduction to Symplectic Topology
% ---, J-holomoprhic curves and symplectic topology (more advanced)