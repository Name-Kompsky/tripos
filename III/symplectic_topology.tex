\documentclass[a4paper]{article}

\def\npart{III}

\def\ntitle{Symplectic Topology}
\def\nlecturer{A.\ Keating}

\def\nterm{Lent}
\def\nyear{2020}

\ifx \nauthor\undefined
  \def\nauthor{Qiangru Kuang}
\else
\fi

\ifx \ntitle\undefined
  \def\ntitle{Template}
\else
\fi

\ifx \nauthoremail\undefined
  \def\nauthoremail{qk206@cam.ac.uk}
\else
\fi

\ifx \ndate\undefined
  \def\ndate{\today}
\else
\fi

\title{\ntitle}
\author{\nauthor}
\date{\ndate}

%\usepackage{microtype}
\usepackage{mathtools}
\usepackage{amsthm}
\usepackage{stmaryrd}%symbols used so far: \mapsfrom
\usepackage{empheq}
\usepackage{amssymb}
\let\mathbbalt\mathbb
\let\pitchforkold\pitchfork
\usepackage{unicode-math}
\let\mathbb\mathbbalt%reset to original \mathbb
\let\pitchfork\pitchforkold

\usepackage{imakeidx}
\makeindex[intoc]

%to address the problem that Latin modern doesn't have unicode support for setminus
%https://tex.stackexchange.com/a/55205/26707
\AtBeginDocument{\renewcommand*{\setminus}{\mathbin{\backslash}}}
\AtBeginDocument{\renewcommand*{\models}{\vDash}}%for \vDash is same size as \vdash but orginal \models is larger
\AtBeginDocument{\let\Re\relax}
\AtBeginDocument{\let\Im\relax}
\AtBeginDocument{\DeclareMathOperator{\Re}{Re}}
\AtBeginDocument{\DeclareMathOperator{\Im}{Im}}
\AtBeginDocument{\let\div\relax}
\AtBeginDocument{\DeclareMathOperator{\div}{div}}

\usepackage{tikz}
\usetikzlibrary{automata,positioning}
\usepackage{pgfplots}
%some preset styles
\pgfplotsset{compat=1.15}
\pgfplotsset{centre/.append style={axis x line=middle, axis y line=middle, xlabel={$x$}, ylabel={$y$}, axis equal}}
\usepackage{tikz-cd}
\usepackage{graphicx}
\usepackage{newunicodechar}

\usepackage{fancyhdr}

\fancypagestyle{mypagestyle}{
    \fancyhf{}
    \lhead{\emph{\nouppercase{\leftmark}}}
    \rhead{}
    \cfoot{\thepage}
}
\pagestyle{mypagestyle}

\usepackage{titlesec}
\newcommand{\sectionbreak}{\clearpage} % clear page after each section
\usepackage[perpage]{footmisc}
\usepackage{blindtext}

%\reallywidehat
%https://tex.stackexchange.com/a/101136/26707
\usepackage{scalerel,stackengine}
\stackMath
\newcommand\reallywidehat[1]{%
\savestack{\tmpbox}{\stretchto{%
  \scaleto{%
    \scalerel*[\widthof{\ensuremath{#1}}]{\kern-.6pt\bigwedge\kern-.6pt}%
    {\rule[-\textheight/2]{1ex}{\textheight}}%WIDTH-LIMITED BIG WEDGE
  }{\textheight}% 
}{0.5ex}}%
\stackon[1pt]{#1}{\tmpbox}%
}

%\usepackage{braket}
\usepackage{thmtools}%restate theorem
\usepackage{hyperref}

% https://en.wikibooks.org/wiki/LaTeX/Hyperlinks
\hypersetup{
    %bookmarks=true,
    unicode=true,
    pdftitle={\ntitle},
    pdfauthor={\nauthor},
    pdfsubject={Mathematics},
    pdfcreator={\nauthor},
    pdfproducer={\nauthor},
    pdfkeywords={math maths \ntitle},
    colorlinks=true,
    linkcolor={red!50!black},
    citecolor={blue!50!black},
    urlcolor={blue!80!black}
}

\usepackage{cleveref}



% TODO: mdframed often gives bad breaks that cause empty lines. Would like to switch to tcolorbox.
% The current workaround is to set innerbottommargin=0pt.

%\usepackage[theorems]{tcolorbox}





\usepackage[framemethod=tikz]{mdframed}
\mdfdefinestyle{leftbar}{
  %nobreak=true, %dirty hack
  linewidth=1.5pt,
  linecolor=gray,
  hidealllines=true,
  leftline=true,
  leftmargin=0pt,
  innerleftmargin=5pt,
  innerrightmargin=10pt,
  innertopmargin=-5pt,
  % innerbottommargin=5pt, % original
  innerbottommargin=0pt, % temporary hack 
}
%\newmdtheoremenv[style=leftbar]{theorem}{Theorem}[section]
%\newmdtheoremenv[style=leftbar]{proposition}[theorem]{proposition}
%\newmdtheoremenv[style=leftbar]{lemma}[theorem]{Lemma}
%\newmdtheoremenv[style=leftbar]{corollary}[theorem]{corollary}

\newtheorem{theorem}{Theorem}[section]
\newtheorem{proposition}[theorem]{Proposition}
\newtheorem{lemma}[theorem]{Lemma}
\newtheorem{corollary}[theorem]{Corollary}
\newtheorem{axiom}[theorem]{Axiom}
\newtheorem*{axiom*}{Axiom}

\surroundwithmdframed[style=leftbar]{theorem}
\surroundwithmdframed[style=leftbar]{proposition}
\surroundwithmdframed[style=leftbar]{lemma}
\surroundwithmdframed[style=leftbar]{corollary}
\surroundwithmdframed[style=leftbar]{axiom}
\surroundwithmdframed[style=leftbar]{axiom*}

\theoremstyle{definition}

\newtheorem*{definition}{Definition}
\surroundwithmdframed[style=leftbar]{definition}

\newtheorem*{slogan}{Slogan}
\newtheorem*{eg}{Example}
\newtheorem*{ex}{Exercise}
\newtheorem*{remark}{Remark}
\newtheorem*{notation}{Notation}
\newtheorem*{convention}{Convention}
\newtheorem*{assumption}{Assumption}
\newtheorem*{question}{Question}
\newtheorem*{answer}{Answer}
\newtheorem*{note}{Note}
\newtheorem*{application}{Application}

%operator macros

%basic
\DeclareMathOperator{\lcm}{lcm}

%matrix
\DeclareMathOperator{\tr}{tr}
\DeclareMathOperator{\Tr}{Tr}
\DeclareMathOperator{\adj}{adj}

%algebra
\DeclareMathOperator{\Hom}{Hom}
\DeclareMathOperator{\End}{End}
\DeclareMathOperator{\id}{id}
\DeclareMathOperator{\im}{im}
\DeclareMathOperator{\coker}{coker}
\DeclarePairedDelimiter{\generation}{\langle}{\rangle}

%groups
\DeclareMathOperator{\sym}{Sym}
\DeclareMathOperator{\sgn}{sgn}
\DeclareMathOperator{\inn}{Inn}
\DeclareMathOperator{\aut}{Aut}
\DeclareMathOperator{\GL}{GL}
\DeclareMathOperator{\SL}{SL}
\DeclareMathOperator{\PGL}{PGL}
\DeclareMathOperator{\PSL}{PSL}
\DeclareMathOperator{\SU}{SU}
\DeclareMathOperator{\UU}{U}
\DeclareMathOperator{\SO}{SO}
\DeclareMathOperator{\OO}{O}
\DeclareMathOperator{\PSU}{PSU}
\DeclareMathOperator{\Sp}{Sp}


%hyperbolic
\DeclareMathOperator{\sech}{sech}

%field, galois heory
\DeclareMathOperator{\ch}{ch}
\DeclareMathOperator{\gal}{Gal}
\DeclareMathOperator{\emb}{Emb}



%ceiling and floor
%https://tex.stackexchange.com/a/118217/26707
\DeclarePairedDelimiter\ceil{\lceil}{\rceil}
\DeclarePairedDelimiter\floor{\lfloor}{\rfloor}


\DeclarePairedDelimiter{\innerproduct}{\langle}{\rangle}

%\DeclarePairedDelimiterX{\norm}[1]{\lVert}{\rVert}{#1}
\DeclarePairedDelimiter{\norm}{\lVert}{\rVert}



%Dirac notation
%TODO: rewrite for variable number of arguments
\DeclarePairedDelimiterX{\braket}[2]{\langle}{\rangle}{#1 \delimsize\vert #2}
\DeclarePairedDelimiterX{\braketthree}[3]{\langle}{\rangle}{#1 \delimsize\vert #2 \delimsize\vert #3}

\DeclarePairedDelimiter{\bra}{\langle}{\rvert}
\DeclarePairedDelimiter{\ket}{\lvert}{\rangle}




%macros

%general

%divide, not divide
\newcommand*{\divides}{\mid}
\newcommand*{\ndivides}{\nmid}
%vector, i.e. mathbf
%https://tex.stackexchange.com/a/45746/26707
\newcommand*{\V}[1]{{\ensuremath{\symbf{#1}}}}
%closure
\newcommand*{\cl}[1]{\overline{#1}}
%conjugate
\newcommand*{\conj}[1]{\overline{#1}}
%set complement
\newcommand*{\stcomp}[1]{\overline{#1}}
\newcommand*{\compose}{\circ}
\newcommand*{\nto}{\nrightarrow}
\newcommand*{\p}{\partial}
%embed
\newcommand*{\embed}{\hookrightarrow}
%surjection
\newcommand*{\surj}{\twoheadrightarrow}
%power set
\newcommand*{\powerset}{\mathcal{P}}

%matrix
\newcommand*{\matrixring}{\mathcal{M}}

%groups
\newcommand*{\normal}{\trianglelefteq}
%rings
\newcommand*{\ideal}{\trianglelefteq}

%fields
\renewcommand*{\C}{{\mathbb{C}}}
\newcommand*{\R}{{\mathbb{R}}}
\newcommand*{\Q}{{\mathbb{Q}}}
\newcommand*{\Z}{{\mathbb{Z}}}
\newcommand*{\N}{{\mathbb{N}}}
\newcommand*{\F}{{\mathbb{F}}}
%not really but I think this belongs here
\newcommand*{\A}{{\mathbb{A}}}

%asymptotic
\newcommand*{\bigO}{O}
\newcommand*{\smallo}{o}

%probability
\newcommand*{\prob}{\mathbb{P}}
\newcommand*{\E}{\mathbb{E}}

%vector calculus
\newcommand*{\gradient}{\V \nabla}
\newcommand*{\divergence}{\gradient \cdot}
\newcommand*{\curl}{\gradient \cdot}

%logic
\newcommand*{\yields}{\vdash}
\newcommand*{\nyields}{\nvdash}

%differential geometry
\renewcommand*{\H}{\mathbb{H}}
\newcommand*{\transversal}{\pitchfork}
\renewcommand{\d}{\mathrm{d}} % exterior derivative

%number theory
\newcommand*{\legendre}[2]{\genfrac{(}{)}{}{}{#1}{#2}}%Legendre symbol

%algebraic geometry
\DeclareMathOperator{\Spec}{Spec}
\DeclareMathOperator{\Proj}{Proj}

\renewcommand*{\P}{\mathbb{P}}
\newcommand{\w}{\wedge} % wedge product
\DeclareMathOperator{\Vect}{Vect} % vector field
\DeclareMathOperator{\Vol}{Vol} % volume form

\begin{document}

\begin{titlepage}
  \begin{center}
    \includegraphics[width=0.6\textwidth]{logo.jpg}\par
    \vspace{1cm}
    {\scshape\huge Mathamatics Tripos \par}
    \vspace{2cm}
    {\huge Part \npart \par}
    \vspace{0.6cm}
    {\Huge \bfseries \ntitle \par}
    \vspace{1.2cm}
    {\Large\nterm, \nyear \par}
    \vspace{2cm}
    
    {\large \emph{Lectures by } \par}
    \vspace{0.2cm}
    {\Large \scshape \nlecturer}
    
    \vspace{0.5cm}
    {\large \emph{Notes by }\par}
    \vspace{0.2cm}
    {\Large \scshape \href{mailto:\nauthoremail}{\nauthor}}
 \end{center}
\end{titlepage}

\tableofcontents

\setcounter{section}{-1}

\section{Introduction and Motivations}

Let \(M\) be a manifold. In Riemannian geometry, we put a nondegenerate symmetric bilinear form on \(T_xM\). In symplectic geometry we put a non-degenerate skew symmetric form instead. By basic linear algebra, by a change of basis, such a form is
\[
  \Omega =
  \begin{pmatrix}
    0 & 1 \\
    -1 & 0 \\
    & & \ddots \\
    & & & 0 & 1 \\
    & & & -1 & 0
  \end{pmatrix}
\]
We define the \emph{symplectic group} to be
\[
  \Sp_{2n}(\R) = \{A \in \GL_{2n}(\R): A^T\Omega A = \Omega\}.
\]

A symplectic manifold is a \(2n\)-manifold \(M^{2n}\) with an atlas of charts such that the derivatives of transition maps are in \(\Sp_{2n}(\R)\). We will prove that this is equivalent to \((M^{2n}, \omega)\), where \(\omega \in \Omega^2(M)\) closed (\(\d \omega = 0\)), everywhere nondegenerate (\(\omega^{\w n} \ne 0\)) at all points.

\begin{ex}
  \((\R^{2n}, \sum \d x_i \w \d y_i)\) gives \(\Omega\) with respect to \(\frac{\partial  }{\partial x_i}, \frac{\partial  }{\partial y_i}\). We call this sympletic form \(\omega_{\text{std}}\).
\end{ex}

In fact, this example is the ``local model'' for all symplectic manifolds. In other words, they have no local invariants.

Motivation 1: mechanices. Given a particle in \(\R^n\) and a potential \(U\), define \(H = U + \frac{\dot q^2}{2}\) to be the energy. Then we can work out the flow of Hamilton's equations
\[
  \frac{\partial H}{\partial p} = \dot q, \frac{\partial H}{\partial q} = - \dot p.
\]
It is a fact that the flow preserves the symplectic form \(\sum \d p_i \w \d q_i\).

Motivation 2: symmetry groups. We want to classify groups acting locally on \(\R^k\) such that
\begin{itemize}
\item act locally transitively (or reduce dim - orbit)
\item noo invariant foliations: not of the form \((x, y) \mapsto (f(x), g(x, y))\) (or reduce dimension).
\end{itemize}

\begin{theorem}[Lie]
  If such a group is finite dimensional, it is one of finitely many families (e.g.\ \(\SO(n), \SU(n), \SO(p, q)\) etc).
\end{theorem}

\begin{theorem}[Cartan]
  If such a group is infinite dimensional, it is one of
  \begin{itemize}
  \item \(\operatorname{Diff}(\R^k)\): all diffeomorphisms (preserving orientation),
  \item \(\operatorname{Vol}(\R^k)\): all diffeomorphisms preserving volume form,
  \item \(\operatorname{Symp}(\R^{2\ell})\): symplectomorphisms, i.e.\ diffeomorphisms preserving symplectic structure,
  \item \(\operatorname{Cont}(\R^{2\ell + 1})\): contactomorphism (odd dimensional analogue of symplectormorphism)
  \item and their conformal analogues.
  \end{itemize}
\end{theorem}

Motivation 3: difference with volume. WeAs \(A \in \Sp_{2n}(\R)\) implies \(\det A = 1\), we have inclusion \(\operatorname{Symp}(\R^{2n}) \subseteq \operatorname{Vol}(\R^{2n})\).

\begin{theorem}[Moser]\leavevmode
  \begin{enumerate}
  \item Two volume forms on a closed manifold are equivalent if and only if they have the same total volume.
  \item Suppose \(U, V\) are connected open in \(\R^k\). There is a volume form-preserving \(U \embed V\) if and only if \(\operatorname{vol}(U) \leq \operatorname{vol}(V)\).
  \end{enumerate}
\end{theorem}

By contrast

\begin{theorem}[Gromov non-squeezing]
  There is no symplectic embedding \(B^{2n}(R) \embed B^2(r) \times \R^{2n - 2}\) if \(R > r\).
\end{theorem}

Motivation 4: complex geometry. Any smooth affine variety has a natural symplectic form

Course outline:
\begin{itemize}
\item background: extra bit of differential geometry, almost complex structure, first Chern class,
\item basic symplecti geometry: distinguished submanifolds, local models, some constants on symplectic manifolds,
\item constructions: e.g.\ new sympletric manifolds from old,
\item holomorphic curves: invariants given by generalisation of Cauchy-Riemann equations, proof of non-squeezing theorem
\end{itemize}

\section{(More) Differential geometry}

\paragraph{Tensor algebra}

Let \(E\) be a vector space over \(\F\). We define the \emph{tensor algebra} of \(E\) to be
\[
  T(E) = \bigoplus_{i \geq 0} E^{\otimes i}
\]
where \(E^{\otimes 0} \cong \F\). Then we define the \emph{exterior algebra} to be
\[
  \Lambda^*E = T(E)/\langle v \otimes v \rangle_{\text{as algebra}}
\]
which has a natural grading \(\Lambda^*E = \bigoplus_{k \geq 0} \Lambda^kE\),
\[
  \Lambda^kE = E^{\otimes k}/\langle w_1 \otimes \cdots \otimes w_k: w_i = w_j \text{ for some } i \ne j \rangle_{\text{as vector space}}
\]
If \(\dim_\F E = n\) then \(\dim \Lambda^*E = 2^n, \dim \Lambda^kE = \binom{n}{k}\). The tensor product on \(T(E)\) induces wedge product on \(\Lambda^*(E)\) which is bilinear, associative and graded commutative.

If \(A: E \to F\) is a linear map then it induces a map \(\Lambda^k A: \Lambda^k E \to \Lambda^k F\). In \(\dim E = \dim F = n\) then we can identify \(\Lambda^nA: \Lambda^nE \to \Lambda^nF\) can be identified with \(\det A: \F \to \F\).

\paragraph{Vector fields and differential forms}

Suppose \(M^n\) is a manifold\footnote{In this course all manifolds are assumed to be smooth unless stated otherwise.}. Then we have the tangent and cotangent bundle \(TM, T^*M\). \emph{Vector fields} and \emph{\(k\)-forms} on \(M\) are defined to be
\begin{align*}
  \Vect(M) &= \Gamma(TM) = \Gamma(M, TM) \\
  \Omega^k(M) &= \Gamma(M, \Lambda^kT^*M)
\end{align*}
The \(0\)-forms are also the smooth functions on \(M\), \(C^\infty(M) = \Omega^0(M)\).

In local coordinates \(x_1, \dots, x_n\) on \(M\), \(X \in \Vect(M)\) can be written locally as
\[
  X_p = \sum_{i = 1}^n X^i \frac{\partial  }{\partial x_i}|_p
\]
where each \(X^i\) is a smooth function.

Given \(X \in \Vect(M), f \in C^\infty(M)\), we can differentiate \(f\) along \(X\) by \((Xf)_p = X_p(f)\). In local coordinates,
\[
  (Xf)_p = \sum X^i(p) \frac{\partial f}{\partial x_i}|_p.
\]
We can check this is well-defined and it is a derivation in the sense that \(X(fg) = fX(g) + gX(f)\).

\paragraph{Pullbacks}

Suppose \(f: M \to N\) is smooth. It induces \(f^*: C^\infty(M) \to C^\infty(N), g \mapsto g \compose f\) and also induces a function on \(1\)-forms by \((f^*\varphi)_x = (Df_*)^* (\varphi_{f(x)})\). It then induces \(f^*: \Omega^*(N) \to \Omega^*(M)\) with the properties that
\begin{enumerate}
\item \(f^*\) is linear and \(f^*(\varphi \w \theta) = f^*\varphi \w f^* \theta\).
\item \((f \compose g)^* \varphi = g^*f^* \varphi\)
\end{enumerate}

\paragraph{Differential on \(\Omega^*(M)\)}

Let \(U \subseteq M\) be a chart with coordinates \(x_i\). Then a local basis for \(\Lambda^kU\) is \(\{\d x_I = \d x_{i_1} \w \cdots \w \d x_{i_k}\}\) where \(I = \{i_1 < \dots < i_k\}\). If \(\varphi: U \to \R\) then we have
\begin{align*}
  \d: C^\infty(M) &\to \Omega^1(M) \\
  \varphi &\mapsto \d \varphi = \sum \frac{\partial \varphi}{\partial x_i} \d x_i
\end{align*}
Note that \((\d \varphi)X = X\varphi \in C^\infty(M)\). In general, if \(\varphi = \sum \varphi_i \d x_I \in \Omega^k(M)\) where \(\varphi_I\) smooth functions, then \(\d \varphi = \sum \d \varphi_I \w \d x_I\). Can check this is well-defined and satisfies
\begin{enumerate}
\item \(\d (\varphi_1 + \varphi_2) = \d \varphi_1 + \d \varphi_2\),
\item \(\d (\varphi_1 \w \varphi_2) = \d \varphi_1 \w \varphi_2 + (-1)^k \varphi_1 \w \d \varphi_2\) if \(\varphi_1 \in \Omega^k\),
\item \(\d^2 = 0\),
\item \(\d (f^*\varphi) = f^*(\d \varphi)\).
\end{enumerate}
Moreover we can show these properties uniquely determines \(\d\).

It follows that we have the \emph{de Rham complex}
\[
  \begin{tikzcd}
    0 \ar[r] \Omega^0(M) \ar[r, "\d"] & \Omega^1(M) \ar[r, "\d"] & \Omega^2(M) \ar[r] & \cdots \ar[r, "\d"] & \Omega^n(M) \ar[r] & 0
  \end{tikzcd}
\]
which gives rise to de Rham cohomology \(H^*_{\text{dR}}(M)\). By de Rham theorem this is isomorphic to \(H^*(M; \R)\), singular cohomology with coefficients in \(\R\). \((\Omega^*M, \w, \d)\) is the de Rham algebra. Morgan 1978 shows that it characterises the rational homotopy type of algebraic varieties.

\paragraph{Isotopies and vector fields}

\begin{definition}[isotopy]\index{isotopy}
  A smooth map \(\rho: M \times \R \to M\) is an \emph{isotopy} if \(\rho_t = \rho(-, t): M \to M\) is a diffeomorphism for each \(t\) and \(\rho_0 = \id_M\).
\end{definition}

We could replace \(\R\) with open intervals containing \(0\).

Given an isotopy \(\rho\), we get a time-dependent vector field, say \(v_t\), as follows:
\[
  v_t|_p = \frac{d}{ds} \rho_s(q)|_{s = t}
\]
where \(q = \rho_t^{-1}(p)\), i.e.
\[
  \frac{d \rho_t}{dt} = v_t \compose \rho_t.
  \tag{\ast}
\]

Conversely, given a time-dependent vector field \(v_t\), if \(M\) is compact or if \(v_t\) is compactly supported, by Picard's theorem on existence of solutions to ODEs, there is an isotopy \(\rho\) such that \(\rho_0 = \id\) and the ODE (\(\ast\)) is satisfied. For compact \(M\) we have a one-to-one correspondence
\[
  \{\text{isotopies of } M\} \longleftrightarrow \{\text{time-dependent vector fields on } M\}.
\]
For non-compact \(M\), the flow still exists locally (i.e.\ at each point \(p\) for sufficiently small interval of time) by Picard(-Lindelöf).

\begin{definition}\index{exponential map}
  If \(v_t = v\) (independent of \(t\)), its flow is called the \emph{exponential map} of \(v\), denoted \(\exp(tv)\).
\end{definition}

Useful formula (III Differential Geometry Example sheet 2 question 3): for \(\theta \in \Omega^1(M), X, Y \in \Vect(M)\), have
\[
  \d \theta(X, Y) = X \theta(Y) - Y \theta(X) - \theta([X, Y]).
\]

\paragraph{Interior product}

Suppose \(\alpha \in \Omega^{p + 1}(M), X \in \Gamma(TM)\), the we define the \emph{interior product} \(X \lrcorner \alpha = \iota_X a \in \Omega^p(X)\) to be
\[
  \iota_X(\alpha)(u) = \alpha(X \w u)
\]
for \(u \in \Gamma(\Lambda^pTM)\).

\paragraph{Lie derivatives}

DG ES2 Q 11*

Let \(M\) be a manifold, \(X \in \Gamma(TM)\) a vector field. Then we have a local flow \(\varphi_t: M \to M\) for \(t \in (-\delta, \delta)\). Given \(\alpha \in \Omega^*(M)\), the \emph{Lie derivative} of \(\alpha\) with respect to \(X\) is
\[
  \mathcal L_X(\alpha) = \frac{d}{dt} (\varphi_t^*\alpha)|_{t = 0} \in \Omega^*(M)
\]
and has the same degree as \(\alpha\) if \(\alpha\) has pure degree. For \(V \in \Gamma(\Lambda^kTM)\), we similarly define
\[
  \mathcal L_XV = \frac{d }{d t}((\varphi_{-t})_*V)|_{t = 0}
\]
where \((\varphi_t)_* = \Lambda^k D\varphi_t\).

Properties:
\begin{enumerate}
\item \(\mathcal L_Xf = Xf\) for \(f \in C^\infty(M)\).
\item \(\mathcal L_X(Y) = [X, Y]\) for \(X, Y \in \Gamma(TM)\).
\item \(\mathcal L_X(\d \alpha) = \d \mathcal L_X \alpha\) for \(\alpha \in \Omega^*(M)\).
\item Cartan's formula: \(\mathcal L_X = \iota_X \compose \d + \d \compose \iota_X\).
\item For a time-dependent \(X_t\) with flow \(\varphi_t\), \(\frac{d}{dt}(\varphi_t^*\alpha) = \varphi_t^* \mathcal L_{X_t}\alpha\) for \(\alpha \in \Omega^*(M)\).
\end{enumerate}

\begin{proof}[Sketch proof]\leavevmode
  \begin{enumerate}
  \item \(\mathcal L_X f = \frac{d}{dt}|_{t = 0} (f \compose \varphi_t) = Xf\).
  \item Let \(\varphi_t\) be the flow of \(X\). Use the slightly unusual notation \(\varphi_t^*(Y)|_p = (D\varphi_t^{-1})(Y_{\varphi_t(p)})\). Check
    \[
      \varphi_t^*(Y)(f \compose \varphi_t) = Y(f) \compose \varphi_t
    \]
    so have
    \[
      \frac{\varphi_t^*(Y)(f \compose \varphi_t) - \varphi_t^*(Y)(f)}{t} + \frac{\varphi_t^*(Y)(f) - Y(f)}{t} = \frac{Y(f) \compose \varphi_t - Y(f)}{t}
    \]
    take limit as \(t \to 0\),
    \[
      YX(f) + \mathcal L_X(Y)(f) = XY(f).
    \]
  \item Omitted.
  \item General strategy: check the formula holds for \(0\)-forms, both sides commute with \(\d\), both sides are derivations for \((\Omega^*(M), \w)\), and use the fact that the equations are local and for a local coordinate patch \(U\), \(\Omega^*(U)\) is generated as an algebra by \(\Omega^0(U)\) and \(\d \Omega^0(U)\).
  \item Same as 4.
  \end{enumerate}
\end{proof}

\begin{lemma}
  For a smooth family \(\alpha_t \in \Omega^k(M)\),
  \[
    \frac{d}{dt}(\varphi_t^* \alpha_t) = \varphi_t^*(\mathcal L_{X_t} \alpha_t + \frac{d \alpha_t}{dt}).
  \]
\end{lemma}

\begin{proof}
  Treat LHS as the derivative of a function of two variables,
  \[
    \frac{d}{dt}(\varphi_t^* \alpha_t)
    = \frac{d}{dx}(\varphi_x^* \alpha_t)|_{x = t} + \frac{d}{dy}(\varphi_t^* \alpha_y)|_{y = t}
    = \varphi_t^*\mathcal L_{X_t}\alpha_t + \varphi_t^* \frac{d\alpha_t}{dt}.
  \]
\end{proof}

\paragraph{Oreintations}

Let \(E\) be an \(n\)-dim \(\R\) vector space. Then an orientation on \(E\) is an equivalence class of ordered basis \((e_1, \dots, e_n)\) under the equivalence relation \((e_1, \dots, e_n) \sim (f_1, \dots, f_n)\) if and only if the endomorphism \(A: e_i \mapsto f_i\) has \(\det A > 0\).

Let \(\pi: E \to B\) be a rank \(k\) real vector bundle. An orientation on \(E\) is a coherent choice of orientations on each fibre \(E_b\), where ``coherent'' means that for local trivialisation \(\pi^{-1}(U) \cong \R^k \times U\), the choice is constant.

Let \(M^n\) be a manifold. An orientation of \(M\) is an orientation of \(TM\) (if exists). We will denote by \(\overline M\) the manifold \(M\) with opposite orientation. If \(M^n\) is a manifold with boundary \(\p M\), an orientation of \(M\) induces an orientation on \(\p M\): a basis \((e_1, \dots, e_{n - 1})\) for \(T_x(\p M)\) is positively oriented if \((n_x, e_1, \dots, e_{n - 1})\) is for ``\(T_xM\)'', where \(n_x\) is the outward pointing normal vector.

Note if \(M\) is a compact oriented \(1\)-manifold with boundary the \(\sum_{p \in \p M} \operatorname{or}(p) = 0\).

\paragraph{Integration}

In vector calculus, we have if \(f: (U, x_i) \to (V, y_i)\) is a diffeomorphism of open subsets of \(\R^k\), then
\[
  \int_V a dy_1 \cdots dy_k = \int_U (a \compose f) |\det(Df)| dx_1 \cdots dx_k.
\]
In differential geometry we formulate integration in this way: for \(\varphi = a \d y_1 \w \cdots \w \d y_k\), if \(\varphi\) preserves orientation then
\[
  \int_V \varphi = \int_U f^* \varphi.
\]

\begin{lemma}
  If \(X\) is an oriented \(k\)-manifold, there is a well-defined integration map
  \[
    \int_X: \Omega^k_c \to \R
  \]
  where \(\Omega^k_c\) are \(k\)-forms with compact support.
\end{lemma}

A \emph{volume form} on \(M^k\) is a nowhere zero section \(\d \Vol \in \Omega^k(M)\), which is equivalent to a choice of trivialisation \(\Lambda^kT^*M \cong \R \times M\). \(M\) is orientable if and only if \(\Lambda^kT^*M\) is trivial. Note that \(\d (\d \Vol) = 0\).

\begin{theorem}[Stokes]
  \[
    \int_M \d \alpha = \int_{\p M} \alpha.
  \]
\end{theorem}

\begin{corollary}
  For \(X\) closed oriented, we have a surjection \(\int_X: H^k_{\text{dR}}(X) \to \R\).
\end{corollary}

\begin{proof}[Sketch proof]
  Let \(U \subseteq \R^k_+\) be an open chart. Use linearity and partition of unity, it suffices to work in \(U\). Then use standard results from multivariate calculus/Fubini's theorem. See example sheet 1.
\end{proof}







\printindex
\end{document}

% Text
% Cannas da Silva, Lectures on Symplectic Geometry
% McDuff-Salamon, Introduction to Symplectic Topology
% ---, J-holomoprhic curves and symplectic topology (more advanced)