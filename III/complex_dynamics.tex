\documentclass[a4paper]{article}

\def\npart{III}

\def\ntitle{Complex Dynamics}
\def\nlecturer{H.\ Krieger}

\def\nterm{Lent}
\def\nyear{2020}

\ifx \nauthor\undefined
  \def\nauthor{Qiangru Kuang}
\else
\fi

\ifx \ntitle\undefined
  \def\ntitle{Template}
\else
\fi

\ifx \nauthoremail\undefined
  \def\nauthoremail{qk206@cam.ac.uk}
\else
\fi

\ifx \ndate\undefined
  \def\ndate{\today}
\else
\fi

\title{\ntitle}
\author{\nauthor}
\date{\ndate}

%\usepackage{microtype}
\usepackage{mathtools}
\usepackage{amsthm}
\usepackage{stmaryrd}%symbols used so far: \mapsfrom
\usepackage{empheq}
\usepackage{amssymb}
\let\mathbbalt\mathbb
\let\pitchforkold\pitchfork
\usepackage{unicode-math}
\let\mathbb\mathbbalt%reset to original \mathbb
\let\pitchfork\pitchforkold

\usepackage{imakeidx}
\makeindex[intoc]

%to address the problem that Latin modern doesn't have unicode support for setminus
%https://tex.stackexchange.com/a/55205/26707
\AtBeginDocument{\renewcommand*{\setminus}{\mathbin{\backslash}}}
\AtBeginDocument{\renewcommand*{\models}{\vDash}}%for \vDash is same size as \vdash but orginal \models is larger
\AtBeginDocument{\let\Re\relax}
\AtBeginDocument{\let\Im\relax}
\AtBeginDocument{\DeclareMathOperator{\Re}{Re}}
\AtBeginDocument{\DeclareMathOperator{\Im}{Im}}
\AtBeginDocument{\let\div\relax}
\AtBeginDocument{\DeclareMathOperator{\div}{div}}

\usepackage{tikz}
\usetikzlibrary{automata,positioning}
\usepackage{pgfplots}
%some preset styles
\pgfplotsset{compat=1.15}
\pgfplotsset{centre/.append style={axis x line=middle, axis y line=middle, xlabel={$x$}, ylabel={$y$}, axis equal}}
\usepackage{tikz-cd}
\usepackage{graphicx}
\usepackage{newunicodechar}

\usepackage{fancyhdr}

\fancypagestyle{mypagestyle}{
    \fancyhf{}
    \lhead{\emph{\nouppercase{\leftmark}}}
    \rhead{}
    \cfoot{\thepage}
}
\pagestyle{mypagestyle}

\usepackage{titlesec}
\newcommand{\sectionbreak}{\clearpage} % clear page after each section
\usepackage[perpage]{footmisc}
\usepackage{blindtext}

%\reallywidehat
%https://tex.stackexchange.com/a/101136/26707
\usepackage{scalerel,stackengine}
\stackMath
\newcommand\reallywidehat[1]{%
\savestack{\tmpbox}{\stretchto{%
  \scaleto{%
    \scalerel*[\widthof{\ensuremath{#1}}]{\kern-.6pt\bigwedge\kern-.6pt}%
    {\rule[-\textheight/2]{1ex}{\textheight}}%WIDTH-LIMITED BIG WEDGE
  }{\textheight}% 
}{0.5ex}}%
\stackon[1pt]{#1}{\tmpbox}%
}

%\usepackage{braket}
\usepackage{thmtools}%restate theorem
\usepackage{hyperref}

% https://en.wikibooks.org/wiki/LaTeX/Hyperlinks
\hypersetup{
    %bookmarks=true,
    unicode=true,
    pdftitle={\ntitle},
    pdfauthor={\nauthor},
    pdfsubject={Mathematics},
    pdfcreator={\nauthor},
    pdfproducer={\nauthor},
    pdfkeywords={math maths \ntitle},
    colorlinks=true,
    linkcolor={red!50!black},
    citecolor={blue!50!black},
    urlcolor={blue!80!black}
}

\usepackage{cleveref}



% TODO: mdframed often gives bad breaks that cause empty lines. Would like to switch to tcolorbox.
% The current workaround is to set innerbottommargin=0pt.

%\usepackage[theorems]{tcolorbox}





\usepackage[framemethod=tikz]{mdframed}
\mdfdefinestyle{leftbar}{
  %nobreak=true, %dirty hack
  linewidth=1.5pt,
  linecolor=gray,
  hidealllines=true,
  leftline=true,
  leftmargin=0pt,
  innerleftmargin=5pt,
  innerrightmargin=10pt,
  innertopmargin=-5pt,
  % innerbottommargin=5pt, % original
  innerbottommargin=0pt, % temporary hack 
}
%\newmdtheoremenv[style=leftbar]{theorem}{Theorem}[section]
%\newmdtheoremenv[style=leftbar]{proposition}[theorem]{proposition}
%\newmdtheoremenv[style=leftbar]{lemma}[theorem]{Lemma}
%\newmdtheoremenv[style=leftbar]{corollary}[theorem]{corollary}

\newtheorem{theorem}{Theorem}[section]
\newtheorem{proposition}[theorem]{Proposition}
\newtheorem{lemma}[theorem]{Lemma}
\newtheorem{corollary}[theorem]{Corollary}
\newtheorem{axiom}[theorem]{Axiom}
\newtheorem*{axiom*}{Axiom}

\surroundwithmdframed[style=leftbar]{theorem}
\surroundwithmdframed[style=leftbar]{proposition}
\surroundwithmdframed[style=leftbar]{lemma}
\surroundwithmdframed[style=leftbar]{corollary}
\surroundwithmdframed[style=leftbar]{axiom}
\surroundwithmdframed[style=leftbar]{axiom*}

\theoremstyle{definition}

\newtheorem*{definition}{Definition}
\surroundwithmdframed[style=leftbar]{definition}

\newtheorem*{slogan}{Slogan}
\newtheorem*{eg}{Example}
\newtheorem*{ex}{Exercise}
\newtheorem*{remark}{Remark}
\newtheorem*{notation}{Notation}
\newtheorem*{convention}{Convention}
\newtheorem*{assumption}{Assumption}
\newtheorem*{question}{Question}
\newtheorem*{answer}{Answer}
\newtheorem*{note}{Note}
\newtheorem*{application}{Application}

%operator macros

%basic
\DeclareMathOperator{\lcm}{lcm}

%matrix
\DeclareMathOperator{\tr}{tr}
\DeclareMathOperator{\Tr}{Tr}
\DeclareMathOperator{\adj}{adj}

%algebra
\DeclareMathOperator{\Hom}{Hom}
\DeclareMathOperator{\End}{End}
\DeclareMathOperator{\id}{id}
\DeclareMathOperator{\im}{im}
\DeclareMathOperator{\coker}{coker}
\DeclarePairedDelimiter{\generation}{\langle}{\rangle}

%groups
\DeclareMathOperator{\sym}{Sym}
\DeclareMathOperator{\sgn}{sgn}
\DeclareMathOperator{\inn}{Inn}
\DeclareMathOperator{\aut}{Aut}
\DeclareMathOperator{\GL}{GL}
\DeclareMathOperator{\SL}{SL}
\DeclareMathOperator{\PGL}{PGL}
\DeclareMathOperator{\PSL}{PSL}
\DeclareMathOperator{\SU}{SU}
\DeclareMathOperator{\UU}{U}
\DeclareMathOperator{\SO}{SO}
\DeclareMathOperator{\OO}{O}
\DeclareMathOperator{\PSU}{PSU}
\DeclareMathOperator{\Sp}{Sp}


%hyperbolic
\DeclareMathOperator{\sech}{sech}

%field, galois heory
\DeclareMathOperator{\ch}{ch}
\DeclareMathOperator{\gal}{Gal}
\DeclareMathOperator{\emb}{Emb}



%ceiling and floor
%https://tex.stackexchange.com/a/118217/26707
\DeclarePairedDelimiter\ceil{\lceil}{\rceil}
\DeclarePairedDelimiter\floor{\lfloor}{\rfloor}


\DeclarePairedDelimiter{\innerproduct}{\langle}{\rangle}

%\DeclarePairedDelimiterX{\norm}[1]{\lVert}{\rVert}{#1}
\DeclarePairedDelimiter{\norm}{\lVert}{\rVert}



%Dirac notation
%TODO: rewrite for variable number of arguments
\DeclarePairedDelimiterX{\braket}[2]{\langle}{\rangle}{#1 \delimsize\vert #2}
\DeclarePairedDelimiterX{\braketthree}[3]{\langle}{\rangle}{#1 \delimsize\vert #2 \delimsize\vert #3}

\DeclarePairedDelimiter{\bra}{\langle}{\rvert}
\DeclarePairedDelimiter{\ket}{\lvert}{\rangle}




%macros

%general

%divide, not divide
\newcommand*{\divides}{\mid}
\newcommand*{\ndivides}{\nmid}
%vector, i.e. mathbf
%https://tex.stackexchange.com/a/45746/26707
\newcommand*{\V}[1]{{\ensuremath{\symbf{#1}}}}
%closure
\newcommand*{\cl}[1]{\overline{#1}}
%conjugate
\newcommand*{\conj}[1]{\overline{#1}}
%set complement
\newcommand*{\stcomp}[1]{\overline{#1}}
\newcommand*{\compose}{\circ}
\newcommand*{\nto}{\nrightarrow}
\newcommand*{\p}{\partial}
%embed
\newcommand*{\embed}{\hookrightarrow}
%surjection
\newcommand*{\surj}{\twoheadrightarrow}
%power set
\newcommand*{\powerset}{\mathcal{P}}

%matrix
\newcommand*{\matrixring}{\mathcal{M}}

%groups
\newcommand*{\normal}{\trianglelefteq}
%rings
\newcommand*{\ideal}{\trianglelefteq}

%fields
\renewcommand*{\C}{{\mathbb{C}}}
\newcommand*{\R}{{\mathbb{R}}}
\newcommand*{\Q}{{\mathbb{Q}}}
\newcommand*{\Z}{{\mathbb{Z}}}
\newcommand*{\N}{{\mathbb{N}}}
\newcommand*{\F}{{\mathbb{F}}}
%not really but I think this belongs here
\newcommand*{\A}{{\mathbb{A}}}

%asymptotic
\newcommand*{\bigO}{O}
\newcommand*{\smallo}{o}

%probability
\newcommand*{\prob}{\mathbb{P}}
\newcommand*{\E}{\mathbb{E}}

%vector calculus
\newcommand*{\gradient}{\V \nabla}
\newcommand*{\divergence}{\gradient \cdot}
\newcommand*{\curl}{\gradient \cdot}

%logic
\newcommand*{\yields}{\vdash}
\newcommand*{\nyields}{\nvdash}

%differential geometry
\renewcommand*{\H}{\mathbb{H}}
\newcommand*{\transversal}{\pitchfork}
\renewcommand{\d}{\mathrm{d}} % exterior derivative

%number theory
\newcommand*{\legendre}[2]{\genfrac{(}{)}{}{}{#1}{#2}}%Legendre symbol

%algebraic geometry
\DeclareMathOperator{\Spec}{Spec}
\DeclareMathOperator{\Proj}{Proj}

\newcommand{\D}{\mathbb{D}}
  
\begin{document}

\begin{titlepage}
  \begin{center}
    \includegraphics[width=0.6\textwidth]{logo.jpg}\par
    \vspace{1cm}
    {\scshape\huge Mathamatics Tripos \par}
    \vspace{2cm}
    {\huge Part \npart \par}
    \vspace{0.6cm}
    {\Huge \bfseries \ntitle \par}
    \vspace{1.2cm}
    {\Large\nterm, \nyear \par}
    \vspace{2cm}
    
    {\large \emph{Lectures by } \par}
    \vspace{0.2cm}
    {\Large \scshape \nlecturer}
    
    \vspace{0.5cm}
    {\large \emph{Notes by }\par}
    \vspace{0.2cm}
    {\Large \scshape \href{mailto:\nauthoremail}{\nauthor}}
 \end{center}
\end{titlepage}

\tableofcontents

\setcounter{section}{-1}

\section{Introduction}

What is complex dynamics? Iteration of holomorphic self-maps of Riemann surfaces

long term behaviour under iteration

origin: iterative root finding algorithm, e.g. Newton's method. When and why does this work? Algebraically, the problem is to ask the convergence of the iteration of the map
\[
  f(z) = z - \frac{p(z)}{p'(z)}: \hat \C \to \hat \C
\]
where \(\hat \C\) is the Riemann sphere.

Goals:
\begin{itemize}
\item equidistribution theorem

  \begin{theorem}[Friere-Lopez-Mañe, 1983]
    Let \(f: \hat \C \to \hat \C\) with degree \(d \geq 2\) holomorphic.  Then there exists a unique \(f\)-invariant probability measure \(\mu_f\) supported on the unstable locus of \(f\), such that for almost all \(\alpha \in \hat \C\), \(\frac{1}{d^n} \sum_{f^n(z) = \alpha} \delta_z \to \mu_f\) in weak-* topology.
  \end{theorem}
\item universality of the Mandelbrot set: fix \(d \geq 2\), define for \(c \in \C\), \(f_c(z): z^d + c\). The \(d\)-Mandelbrot set is
  \[
    M_d = \{c \in \C: |f_c^n(0)| \nto \infty\}.
  \]

  \begin{theorem}[McMullen 1997]
    The Mandelbrot set is universal for bifurcations, i.e. in any bifurcation locus we see (slightly distorted) copics of some \(M_d\).
  \end{theorem}
\end{itemize}

\section{Riemann surfaces}

Recall at the end of IID Riemann Surfaces

\begin{theorem}[uniformisation]
  Every Riemann surface \(R\) is conformally isomorphic to \(\tilde R/G\) where \(\tilde R\) is one of the three simply connected Riemann surfaces (\(\hat \C, \C, \D\)) and \(G \subseteq \aut(\tilde R)\) acts freely and properly discontinuously.
\end{theorem}

Three cases:
\begin{enumerate}
\item \(\hat \C\): as \(\aut(\hat \C)\) are precisely the Möbius transformations and all Möbius transformations have fixed points, \(G = 1\) so \(R = \hat \C\).
\item \(\C\): \(\aut(\C) = \{p(z) = az + b, a \ne 0\}\) so \(G\) only contains translations, so can be identified by a subgroup of \(\C\). Can show that \(G\) is one of \(\{0\}, \Z \omega_1\) or \(\Lambda = \Z\omega_1 \oplus \Z\omega_2\), a lattice. Then \(R\) is one of \(\C, \C^*\) or \(\C/\Lambda\), a complex torus.
\item \(\D\): \(\aut(\D) = \{\lambda \cdot \frac{z - a}{1 - \conj a z}: \lambda \in S^1, a \in \D\}\), which has a lot of elements with no fixed point. This is called \emph{hyperbolic}.
\end{enumerate}

Recall \(\D\) is equipped with metric \(ds = \frac{2 |dz|}{1 - |z|^2}\), i.e.\ the distance between \(x, y \in \D\) is
\[
  \rho(x, y) = \inf_\gamma \int_a^b \frac{2 |\gamma'(t)|}{1 - |\gamma(t)|^2} dt
\]
where \(\gamma: [a, b] \to \D\) is a smooth curve from \(x\) to \(y\). One can use the origin to show that
\[
  \rho(x, y) = \frac{\log(1 + R)}{\log(1 - R)}
\]
where \(R = |\frac{y - x}{1 - \conj x y}|\). Elements of \(\aut(\D)\) are isometries of this metric so it descends to a hyperbolic metric on any hyperbolic Riemann surface, such that the covering map \(\tilde R \to R\) is a local isometry.

\begin{ex}
  Let \(R = \D/G\), \(f: R \to R\) holomorphic. Then \(f\) lifts to a holomorphic \(\tilde f: \D \to \D\), unique up to a composition with an element of \(G\), and induces a group homomorphism \(\gamma \mapsto \gamma'\) such that \(\tilde f \compose \gamma = \gamma' \compose \tilde f\).
\end{ex}

\begin{theorem}[Pick]
  Let \(f: S \to T\) be hyperbolic Riemann surfaces with Poincaré metric \(\rho_S, \rho_T\) respectively. Then for all \(x, y \in S\),
  \[
    \rho_T(f(x), f(y)) \leq \rho_S(x, y).
  \]
\end{theorem}

\begin{proof}
  By lifting (of \(f \compose \pi_S: \D \to T\)) it suffices to show this for \(S = T = \D\). By computation this is the same as showing
  \[
    \frac{\log (1 + R')}{\log (1 - R')} \leq \frac{\log (1 + R)}{\log (1 - R)}
  \]
  where \(R = |\frac{y - x}{1 - \overline y x}|, R' = |\frac{f(y) - f(x)}{1 - \overline{f(y)} f(x)}|\). Note \(\frac{\log (1 + x)}{\log (1 - x)}\) is strictly increasing, so suffices to show \(R' \leq R\). Let \(\mu_1(z) = \frac{y - z}{1 - \conj y z}, \mu_2(z) = \frac{f(y) - z}{1 - \conj{f(y)} z}\). Let \(g = \mu_2 \compose f \compose \mu_1^{-1}: \D \to \D\). By Schwarz's lemma \(|g(z)| \leq |z|\), i.e.\ \(|\mu_2(f(w))| \leq |\mu_1(w)|\), so \(R' \leq R\).
\end{proof}

\begin{remark}
  It follows from the ``strict inequality'' bit of Schwarz's lemma that exists \(x, y\) such that \(\rho_T(f(x), f(y)) = \rho_S(x, y)\) if and only if \(f\) lifts to a disk automorphism.
\end{remark}

\begin{eg}
  Contracting holomorphic maps is a very strong requirement. Compre to, for example, \(f(z) = z + 1\) on \(\hat \C\) (drawing of different behaviour on two hemispheres).
\end{eg}

Case of \(\hat \C\):

\begin{proposition}
  Let \(f: \hat \C \to \hat \C\) be a holomorphic nonconstant map. Then \(f\) is a rational map, i.e.\ exists \(a_1, \dots, a_m, b_1, \dots, b_n, c \in \C\) such that
  \[
    f(z) = c \cdot \frac{(z - a_1) \cdots (z- a_m)}{(z - b_1) \cdots (z- b_n)}.
  \]
\end{proposition}

\begin{proof}
  wlog \(f(\infty) \in \C\) (if not, replace with \(\frac{1}{f}\)). Then exist a finite collection \(f^{-1}(\infty) = \{b_1, \dots, b_n\} \subseteq \C\). About any \(b_i\) have locally
  \[
    f(z) = \sum_{j = -k}^\infty a_{ij} (z - b_i)^j.
  \]
  Set \(Q_i = \sum_{j = -k}^{-1} a_{ij} (z - b_i)^j\). Then \(g - f - Q_1 - \dots - Q_n\) has no pole so must be constant.
\end{proof}

Universal cover \(\C\): just a remark. Yes there are interesting dynamics. For example \(z \mapsto e^z\) belongs to the realm of transcendental dynamics. \(\C/\Lambda\) also admits nonconstant holomorphic maps. See example sheet 1.

\paragraph{stable and unstable locus}

Motivating example: \(z \mapsto z^2\) on \(\hat \C\). We can restrict \(f\) to the unit disk and we see \(f|_\D^n \to 0\) uniformly on compact subsets of \(\D\). Similarly on \(\hat \C \setminus \overline \D\), \(f|_{\hat \C \setminus \overline \D}^n \to \infty\). On the other hand for \(|z_0| = 1\), there is no neighbourhoood \(z_0 \in U\) such that \(f|_U^n(z)\) converging to a holomorphic function, as such a limit would have a discontinuity.

\begin{definition}[locally uniform convergence/divergence]\index{locally uniform convergence}\index{locally uniform divergence}
  Let \(S, T\) be metric spaces, \(f_n: S \to T\) a sequence of continuous maps. We say \((f_n)\) \emph{converges locally uniformly} if for all compact \(K \subseteq S\), for all \(\varepsilon > 0\) exists \(N \in \N\) such that for all \(m, n > N\), \(\sup_{x \in K} d_T(f_m(x), f_n(x)) < \varepsilon\).

  We say \((f_n)\) \emph{diverges locally uniformly} if for all compact \(K \subseteq S\) and all compact \(K' \subseteq T\), exists \(N \in \N\) such that for all \(n > N\), \(f_n(K) \cap K' = \emptyset\).
\end{definition}

Recall that if holomorphic \(f_n\)'s converge locally uniformly on \(S\) then it has a holomorphic limit function.

\begin{remark}
  If \(T\) is compact then we never have locally uniform divergence.
\end{remark}

\begin{definition}[normal family]\index{normal family}
  We say a family \(\mathcal F = \{f: S \to T\}\) of continuous functions is \emph{normal} if every sequence \((f_n) \subseteq \mathcal F\) has a subsequence which either converges locally uniformly or diverges locally uniformly.
\end{definition}

\begin{ex}\leavevmode
  \begin{enumerate}
  \item Show normality depends only on the topology, not the metric, of the spaces.
  \item Normality is local: if \(\mathcal F\) is a family of continuous maps, \(S = \bigcup_\alpha U_\alpha\), then if \(\mathcal F|_{U_\alpha}\) is normal for all \(\alpha\) then \(\mathcal F\) is normal for \(S\).
  \end{enumerate}
\end{ex}

A word of caution: in some texts (such as Ahlfors) the definition of normality excludes divergence.

\begin{definition}[equicontinuity]\index{equicontinuity}
  A family \(\mathcal F\) of continuous maps on a domain \(U \subseteq \C\) with values in a metric space \(T\) is \emph{equicontinuous} if for all \(\varepsilon > 0\), exists \(\delta > 0\) such that for all \(z, w \in U\) and for all \(f \in \mathcal F\), if \(|z - w| < \delta\) then \(d_T(f(z), f(w)) < \varepsilon\).
\end{definition}

\begin{theorem}[Arzela-Ascoli]
  Let \(\mathcal F\) be a family of continuous maps \(U \to T\) with \(U \subseteq \C\) a domain and \(T\) a metric space. Then \(\mathcal F\) has the property that any sequence has a locally uniformly convergent subsequence if and only if
  \begin{enumerate}
  \item \(\mathcal F\) is equicontinuous on every compact \(K \subseteq U\),
  \item for every \(z \in U\), \(\{f(z): f \in \mathcal F\}\) lies in a compact subset of \(T\).
  \end{enumerate}
\end{theorem}

\begin{corollary}
  If \(T\) is compact then \(\mathcal F\) is normal if and only if it is equicontinuous on compact subsets of \(U\).
\end{corollary}

\begin{proof}
  Only if is easy. For if, we use separability of \(\C\). Let \(\{z_k\}\) be a countable dense subset of \(U\) and fix \(\{f_n\} \subseteq \mathcal F\). Can find a set of indices \(n_{11} < n_{12} < \cdots\) such that \(f_{n_{1i}}(z_1)\) converges, and a subsequence of these \(n_{21} < n_{22} < \cdots\) such that \(f_{n_{2i}}(z_2)\) converges. Let \(g_k = f_{n_{kk}}\). Then for all \(z_i\) the limit \(\lim_{k \to \infty} g_k(z_i)\) exists in \(T\). Now given \(K \subseteq U\) compact, by equicontinuity for any \(\varepsilon > 0\) exists \(\delta > 0\) such that for all \(z, w \in U\) with \(|z - w| < \delta\), we have for all \(f \in \mathcal F\), \(d_T(f(z), f(w)) < \varepsilon\). Cover \(K\) by \(\delta\)-balls, extract a finite subcover, and choose some \(z_i\) in each, say \(z_1, \dots, z_\ell\). For each \(z_i\), \(1 \leq i \leq \ell\), exists \(N_i\) such that for all \(n, m \geq N_i\), \(d_T(g_n(z_i), g_m(z_i)) < \varepsilon\). Let \(N = \max_i N_i\). Then for all \(z \in K\),
  \begin{align*}
    d_T(g_n(z), g_m(z))
    &\leq d_T(g_n(z), g_n(z_i)) + d_T(g_n(z_i), g_m(z_i)) + d_T(g_n(z_i), g_m(z)) \\
    &< 3\varepsilon
  \end{align*}
\end{proof}

\begin{theorem}[Montel]\index{Montel's theorem}
  Suppose \(S, T\) are Riemann surfaces and \(T\) is hyperbolic. Then all families of holomorphic maps \(S \to T\) are normal.
\end{theorem}

\begin{proof}
  If \(S\) is not hyperbolic, lifting plus Liouville's theorem imply that all maps are constant. Given a family \(\mathcal F\) of constant maps, let \(\{f_n\} \subseteq \mathcal F\) be a sequence. Then if \(\{f_n(S)\}\) lies in a compact set in \(T\), then exists a convergent subsequence; if not, exists a subsequence which leaves any compact set, so diverges locally uniformly.

  Suppose \(S\) is hyperbolic, \(\{f_n\} \subseteq \mathcal F\). If exists \(x \in S\) such that \(\{f_n(x)\}\) lies in a compact subset of \(T\), the same is true for all \(y \in S\). Pick's therom implies equicontinuity so by Arzela-Ascoli \(\{f_n\}\) has a convergent subsequence. Otherwise fix \(x \in S\) and \(y \in T\) and exists a subsequence \(\{f_{n_k}\}\) such that \(d_T(f_{n_k}(x), y) \to \infty\). Given \(K \subseteq S, K' \subseteq T\) compact, by Pick's theorem \(d_T(f_{n_k}(z), y) \to \infty\) for all \(z \in K\) and so \(f_{n_k}(K) \cap K' = \emptyset\) for \(k \gg 1\). Thus the sequence diverges locally uniformly.
\end{proof}

\begin{eg}
  If \(f: \hat \C \to \hat \C\) is a rational map and \(D \subseteq \hat \C\), then the family of iterates \(\mathcal F = \{f^n\}_{n \in \N}\) is normal if \(\bigcup_{n \in \N} f^n(D)\) omits 3 or more points of \(\hat \C\), as any domain of \(\hat \C\) with complement of cardinality \(\geq 3\) is hyperbolic.
\end{eg}

\begin{definition}[proper map]\index{proper map}
  Suppose \(U, V\) are open subsets of Riemann surfaces and \(f: U \to V\). \(f\) is proper if for every \(K \subseteq V\) compact, \(f^{-1}(K)\) is compact in \(U\).
\end{definition}

Proper maps have well-defined degrees and satisfies the Riemann-Hurwitz formula. The proofs are similar to the compact case.

\begin{proposition}
  Suppose \(U, V\) are open in \(\hat \C\). If \(f: U \to V\) is proper holomorphic nonconstant then \(f\) has a well-defined degree, i.e.\ for all \(x\in V\), \(|f^{-1}(x)|\) is independent of \(x\), counting multiplicity.
\end{proposition}

\begin{theorem}[Riemann-Hurwitz]
  With same assumptions as above, the Euler characteristic \(\chi(U)\), \(\chi(V)\) satisfy
  \[
    \chi(U) = (\deg f) \chi(V) - \sum_{p \in U}(e_p - 1)
  \]
  where \(e_p\) is the local degree/ramification index of \(f\) at \(p\).
\end{theorem}

Note that one strategy is to observe that for \(U \subseteq \C\), we can cover by \(\delta\)-grid, let \(\delta \to 0\), we obtain a covering of \(U\) by open balls with compact closures and smooth boundaries.

\begin{corollary}
  If \(f: \hat \C \to \hat \C\) is holomorphic of degree \(d\), then \(f\) has \(2d - 2\) critical points, counting multiplicity.
\end{corollary}

\begin{corollary}
  If \(\chi(U) = \chi(V) = 0\) then \(f: U \to V\) proper holomorphic nonconstant is unramified.
\end{corollary}

\begin{definition}[Fatou set, Julia set]\index{Fatou set}\index{Julia set}
  Let \(f: \hat \C \to \hat \C\) be a holomorphic nonconstant map. The \emph{Fatou set} of \(f\) is
  \[
    F(f) = \{z \in \hat \C: \text{ on some nbhd } z \in U, f^n \text{ forms a normal family}\}.
  \]
  The \emph{Julia set} of \(f\) is \(J(f) = \hat \C \setminus F(f)\).
\end{definition}

\begin{eg}
  For \(z \mapsto z^2\), we have normality on \((\hat \C \setminus \overline \D) \cup \D\), and normality fails on any open neighbourhood which intersects \(S^1\). Thus \(J(z^2) = S^1\). This example might be slightly misleading, as we'll see that the Julia set of a rational map is almost never smooth.
\end{eg}

\begin{lemma}
  \(F(f)\) and \(J(f)\) are totally \(f\)-invariant, i.e.\ \(f^{-1}(F(f)) = F(f), f^{-1}(J(f)) = J(f)\).
\end{lemma}

\begin{proof}
  Suffices to show that if \(U \subseteq \hat \C\) then \(\{f^n|_U\}\) is normal on \(U\) if and only if \(\{f^{n + 1}|_{f^{-1}(U)}\}\) is normal on \(f^{-1}(U)\). If \(K \subseteq f^{-1}(U)\) compact then
  \[
    \sup_{z \in K} d(f^n(z), f^m(z)) = \sup_{w \in f(K)} d(f^{n - 1}(z), f^{m - 1}(z)).
  \]
  Since \(f\) is proper (continuous map from compact space to Hausdorff space is proper) and continuous, compactness is preserved by both \(f\) and \(f^{-1}\).
\end{proof}

\begin{lemma}
  \(J(f) = J(f^n)\) and \(F(f) = F(f^n)\) for all \(n\).
\end{lemma}

\begin{proof}
  Exercise.
\end{proof}

\begin{remark}
  The Julia set of \(f\) is the smallest (?) closed subset of \(\hat \C\) which is totally \(f\)-invariant and contains at least \(3\) points (for the moment assume \(|J(f)| \geq 3\)), since the complement of any such set has \(\{f^n\}\) normal by Montel.
\end{remark}

For the rest of the course, we consider only rational maps with \(\deg f \geq 2\). See example sheet 1 for a description of \(J(\mu)\) for \(\mu\) a Möbius transformation.

\begin{theorem}
  Let \(z \in U \subseteq J(f)\) be open. Then the union \(V = \bigcup_{n \in \N} f^n(U)\) contains all but at most 2 points of \(\hat \C\). Any point \(w \notin V\) is a critical point of the Fatou set.
\end{theorem}

\begin{proof}
  The first statement follows from Montel. If \(w \notin V\), since \(f(V) \subseteq V\), then for all \(n \in \N\), \(f^{-n}(w) \cap V = \emptyset\). Suppose there are two points \(\{z_0, z_1\}\). Then examining possible ramification for these two points which are fixed under \(f^{-1}\), the only possiblities are \(z_i \mapsto z_i\) with degree \(d\), or \(z_1 \mapsto z_2, z_2 \mapsto z_1\) with degree \(d\). The case for a single point is similiar.

  Replacing \(f\) by \(f^2\) if needed, it suffcies to show that if \(f(z) = z\) and \(f'(z) = 0\) then \(z \in F(f)\). Note that if \(\mu \in \aut(\hat \C)\), \(g = \mu^{-1} \compose f \compose \mu\), then \(g^n = \mu^{-1} \compose f^n \compose \mu\), and so \(\mu(J(g)) = J(f)\). Thus wlog \(f(0) = 0, f'(0) = 0\). Locally we have \(f(z) = a_2 z^2 + a_3z^3 + \dots = z^2(a_2 + O(z))\) about \(0\), so \(|f(z)| < |z|\) for \(z\) sufficiently close to \(0\), so on a neighbourhood of \(0\), \(f^n \to 0\) so form a normal family.
\end{proof}

The three cases do happen: \(z \mapsto z^d, z \mapsto z^{-d}, z \mapsto p(z)\) for \(p\) a polynomial with nonzero constatnt term.

\begin{remark}
  If \(f^n(z_0) = z_0\) for some \(n \in \N\) and \((f^n)'(z_0) = \prod_{i = 0}^{n - 1} f'(f^i(z_0)) = 0\) then \(z_0 \in F(f)\).
\end{remark}

\begin{corollary}
  If \(J(f)\) contains an interior point then \(J(f) = \hat \C\).
\end{corollary}

\begin{proof}
  If \(U\) is open in \(J(f)\), \(V = \bigcup f^n(U)\) contains all but at most 2 points on \(\hat \C\). Since \(J(f)\) is closed by definition, \(J(f) = \hat \C\).
\end{proof}

This does happen: let \(E_t: y^2 = x(x - 1)(x - t)\) for \(t \in \C \setminus \{0, 1\}\) be an elliptic curve.
\[
  \begin{tikzcd}
    E_t \ar[r, "{[2]}"] \ar[d] & E_t \ar[d] \\
    \hat \C \ar[r, "f_t"] & \hat \C
  \end{tikzcd}
\]
where the vertical maps are quotient by ?, i.e.\ \((x, y) \mapsto x\). Then
\[
  f_t(z) = \frac{(z^2 - t)^2}{4z(z-1)(z - t)}.
\]
We can show \(J(f_t)\) is dense in \(\hat \C\) by showing the Julia set of \([2]\) is dense, and thus \(J(f_t) = \hat \C\).

\begin{definition}[period, multiplier]\index{period}\index{multiplier}
  Let \(z_0 \in \hat \C\). We say \(z_0\) is \emph{periodic} for a rational \(f\) if exists \(m \in \N\) such that \(f^m(z_0) = z_0\). The minimal such \(m\) is the \emph{period} of the cycle containing \(z_0\). If \(m = 1\) we also call it a \emph{fixed point}.

  If \(z_0\) has period \(m\), the \emph{multiplier} of the cycle is
  \[
    (f^m)'(z_0) = \prod_{i = 0}^{m - 1} f'(f^i(z_0)).
  \]
  Let \(\lambda\) be the multiplier of \(z_0\). We say \(z_0\) is
  \begin{enumerate}
  \item \emph{superattracting} if \(\lambda = 0\),
  \item \emph{attracting} if \(0 \leq |\lambda| < 1\),
  \item \emph{indifferent} if \(|\lambda| = 1\),
  \item \emph{repelling} if \(|\lambda| > 1\).
  \end{enumerate}
\end{definition}
Recall that we might have to use the chart at infinity to compute the derivative. For example if \(z_0 = \infty, f(\infty) = \infty\) then
\[
  \lambda = \lim_{z \to \infty} \frac{1}{f'(z)}.
\]

\begin{definition}[basin of attraction]\index{basin of attraction}
  Suppose \(C = \{z_0, f(z_0), \dots, f^{m - 1}(z_0)\}\) is an attracting cycle. The \emph{basin of attraction} for \(C\) is
  \[
    A = \{z \in \hat \C: \lim_{n \to \infty} f^{nm}(z) = f^i(z_0) \text{ for some } 0 \leq i \leq m - 1\}.
  \]
\end{definition}

\begin{theorem}
  If \(f: \hat \C \to \hat \C\) has an attracting cycle then the basin of attraction is in \(F(f)\). On the other hand all repelling cycles are contained in \(J(f)\).
\end{theorem}

\begin{eg}
  The theorem completely describes the Fatou and Julia set of \(z \mapsto z^2\). \(z_0\) is periodic if and only if exists \(n\) such that \(z_0^{2^n} = z_0\), so \(z_0\) is \(0, \infty\) or some root of unity (which forms a dense subset of \(S^1\)). \(A_0 = \D, A_\infty = \hat \C \setminus \overline \D\). All other cycles are repelling and so \(J(f) = S^1\).
\end{eg}

\begin{proof}
  Since \(J(f^m) = J(f)\), wlog assume \(z_0\) is a fixed point. Suppose that \(\lambda = f'(z_0)\) is such that \(|\lambda| < 1\). By Taylor expansion \(|f(z) - z_0| \leq c |z - z_0|\) for some constant \(c < 1\) for \(z\) sufficiently close to \(z_0\). So on a neighbourhood of \(_0\), \(f^n(z)\) converges uniformly on compact subsets to the constant function \(z_0\). So \(z_0 \in F(f)\).

  On the other hand if \(z_0\) is repelling so \(|\lambda| > 1\), suppose for contradiction that \(z_0 \in F(f)\), so exists open neighbourhood \(U\) of \(z_0\) on which \(f^n\) has a subsequence converging to a holomorphic limit. Since \((f^n)'(z_0) = \lambda^n\), absurd.
\end{proof}

\begin{remark}
  We will classify later when indifferent points are Julia.
\end{remark}

\subsection{Holomorphic Lefschetz fixed point formula}

\begin{definition}[residue index]\index{residue index}
  Let \(z_0\) be a fixed point of a rational map \(f\). The \emph{residue index} of \(f\) at \(z_0\) is
  \[
    i_f(z_0) = \frac{1}{2\pi i} \int_\gamma \frac{dz}{z - f(z)}
  \]
  where \(\gamma\) is a small, positively oriented circle about \(z_0\).
\end{definition}

\begin{lemma}
  Let \(z_0\) have multiplier \(\ne 1\). Then \(i_f(z_0) = \frac{1}{1 - \lambda}\).
\end{lemma}

\begin{proof}
  It is an exercise to check the multiplier is coordinate-independent. By definition the resude index is translation/conjugation independent, so wlog \(z_0 = 0\). Then on a neighbourhood of \(0\), \(f(z) = \lambda z + a_2z^2 + \dots\) so
  \[
    \frac{1}{z - f(z)} = \frac{1}{(1 - \lambda) z (1 + O(z))} = \frac{1}{(1 - \lambda) z} + g(z)
  \]
  with \(g\) holomorphic on a neighbourhood of \(0\). Integrate.
\end{proof}

\begin{theorem}[holomorphic Lefschetz on \(\hat \C\)]
  Say \(f: \hat \C \to \hat \C\) of degree \(\geq 2\). Then the fixed points of \(f\) satisfy
  \[
    \sum_{z = f(z)} i_f(z) = 1.
  \]
\end{theorem}

\begin{proof}
  Conjugation if necessary (exercise: use the above lemma to show the residue index is coordinate-independent), wlog \(f(\infty) \ne \infty\). Choose \(R \gg 0\) so that all fixed points of \(f\) are in \(D(0, R)\). Call the positively oriented boundary \(C_R\). By residue theorem
  \begin{align*}
    \sum_{z = f(z)} i_f(z)
    &= \frac{1}{2\pi i} \int_{C_R} \frac{dz}{z - f(z)} \\
    &= \frac{1}{2\pi i} \int_{-C_{1/R}} \frac{-dw}{w^2(\frac{1}{w} - f(\frac{1}{w}))} \\
    &= \frac{1}{2\pi i} \int_{C_{1/R}} \frac{dw}{w(1 - wf(\frac{1}{w}))} \\
    &= \operatorname{Res}_{w = 0} \frac{1}{w(1 - wf(\frac{1}{w}))} \\
    &= 1
  \end{align*}
\end{proof}

\begin{corollary}
  Suppose \(\deg f \geq 2\). Then \(J(f) \ne \emptyset\).
\end{corollary}

\begin{proof}
  Consider the fixed points of \(f\). Assume first no fixed point multiplier is \(1\). Then \(\lambda \mapsto \frac{1}{1 - \lambda}\) sends the unit circle to the line \(\Re = \frac{1}{2}\), and \(\D\) to \(\Re > \frac{1}{2}\). Thus if \(|\lambda| \leq 1\) for all fixed point multipliers, and not equal to \(1\), (there is no multiplicity), there are \(d + 1 \geq 3\) distinct fixed points (?), so \(\Re( \sum_{z = f(z)} i_f(z)) \geq \frac{3}{2}\), absurd. If exists a repelling point then done. So suppose \(z_0\) is fixed with \(\lambda = 1\). Then in local coordinates \(f(z) = z + a_kz^k + \dots\) where \(a_k \ne 0\). Inductively \(f(z) = z + na_kz^k + \dots\) so the \(k\)th derivative of \(f^n(z_0)\) is \(k! na_k \to \infty\) as \(n \to \infty\), so the iterates cannot form a normal family on a neighbourhood of \(z_0\).
\end{proof}

\begin{remark}\leavevmode
  \begin{enumerate}
  \item Suppose \(z_0\) is a indifferent fixed point, \(\lambda\) a root of unity. If \(\lambda^k = 1\) then \((f^k)'(z_0) = \prod_{i = 0}^{k - 1} f'(f(z_0)) = \lambda^k = 1\). Thus the preceding argument shows that \(z_0 \in J(f^k) = J(f)\).
  \item It is possible for a rational map to have non repelling fixed point. For example \(z \mapsto z^2 + \frac{1}{4}\). The fixed points are \(\infty\) and \(\frac{1}{2}\) which is a double fixed point.
  \item Any finite \emph{grand orbit} (the set \(\{z \in \hat \C: f^m(z) = f^n(z_0) \text{ for some } m, n\}\)) is necessarily Fatou (exercise), so \(|J(f)| = \infty\).
  \end{enumerate}
\end{remark}

Recall: suppose \(f: \hat \C \to \hat \C\) is a rational map of degree \(d \geq 2\).

\begin{enumerate}
\item If \(U\) is open, \(U \cap J(f) \ne \emptyset\) then \(\bigcup_{n \geq 1} f^n(U)\) contains all but at most 2 points and contains \(J(f)\).
\item \(J(f)\) contains all repelling cycles and all indifferent cycles with roots of unity multipliers.
\item \(J(f) \ne \emptyset\) and \(|J(f)| = \infty\).
\end{enumerate}

\begin{proposition}
  Suppose \(f\) has a periodic cycle which is attracting, with attracting basin \(A\). Then \(J(f) = \p A\).
\end{proposition}

\begin{proof}
  Given \(U\) an open neighbourhood such that \(U \cap J(f) \ne \emptyset\), exists \(n\) such that \(f^n(U) \cap A = \emptyset\). As \(A\) is closed under preimages, \(U \cap A \ne \emptyset\). Thus \(J(f) \subseteq \overline A\). Since \(A \subseteq F(f)\), \(J(f) \subseteq \p A\).

  Conversely suppose \(z_0 \in \p A\) and \(U\) is a neighbourhood of \(z_0\). Suppose \(\{f^n\}\) forms a normal family on \(U\). On \(U \cap A\), any holomorphic limit \(g\) of iterates of \(f\) must take finitely many constant values, but \(g\) cannot be locally constant as \(U\) contains points not in the basin, absurd. Thus \(z_0 \in J(f)\).
\end{proof}

\begin{eg}
  Any \emph{polynomial} \(f\) has \(J(f)\) the boundary of basin at \(\infty\). Note that it might also be the boundary of another basin, for example \(z \mapsto z^2, z \mapsto z^2 - 1\).
\end{eg}

\begin{corollary}
  Fix \(z_0 \in J(f)\). Then the full preimage \(\{z: f^n(z) = z_0 \text{ for some } n \geq 0\}\) forms a dense subset of \(J(f)\).
\end{corollary}

\begin{proof}
  Fix \(z_1 \in J(f)\) and a neighbourhood \(U \ni z_1\). If it contains no preimage of \(z_0\) then \(\bigcup f^n(U) \notin z_0\), absurd.
\end{proof}

Topological preimage equidistribution

\subsection{Attracting (and repelling) cycles}

\begin{definition}[topologically attracting]\index{topologically attracting}
  A fixed point \(p\) of \(f\) is \emph{topologically attracting} if there exists a neighbourhood \(U \ni p\) such that \(\{f^n\}\) converges locally uniformly to \(p\) on \(U\).
\end{definition}

\begin{lemma}
  A fixed point \(p\) of \(f\) is attracting if and only if it is topologically attracting.
\end{lemma}

\begin{proof}
  Exercise. Taylor's theorem in one direction, and Schwarz lemma in the other.
\end{proof}

\begin{theorem}
  Suppose \(f\) has a fixed point \(p\) with multiplier \(\lambda\), \(|\lambda| \ne 0, 1\). Then exists local holomorphic change of coordinates \(\phi\) such that \(\phi(p) = 0\) and \(\phi \compose f \compose f^{-1}(w) = \lambda w\). Thus coordinate is unique up to multiplication by a constant. \(\phi\) is known as the \emph{Kaenig linearising map}\index{Kaenig linearising map}.
\end{theorem}

\begin{proof}
  wlog \(p = 0\) and first suppose \(0 < |\lambda| < 1\). Choose a constant \(c\) such that \(c^2 < |\lambda| < c\). Find \(r > 0\) such that for all \(z \in D(0, r)\), \(|f(z)| \leq c |z|\), so \(|f^n(z)| \leq c^n r\). We can find \(B > 0\) such that for all \(z \in D(0, r)\), \(|f(z) - \lambda z| \leq B |z|^2\). Thus for all \(z \in D(0, r)\),
  \[
    |f^{n + 1}(z) - \lambda f^n(z)| \leq B |f^n(z)|^2 \leq B r^2 c^{2n}.
  \]
  Let \(w_n = \frac{f^n(z)}{\lambda^n}\). Then
  \[
    |w_{n + 1}(z) - w_n(z)|
    = \left| \frac{f^{n + 1}(z)}{\lambda^{n + 1}} - \frac{f^n(z)}{\lambda^n} \right|
    \leq \frac{1}{|\lambda|^{n + 1}} Br^2 c^{2n}
    = \frac{Br^2}{\lambda} \left|\frac{c^2}{\lambda}\right|^n
  \]
  so \(w_n\) converges locally uniformly on \(D(0, r)\). Set \(\phi(z) = \lim w_n(z)\). As \(z \mapsto w_n(z)\) has derivative \(1\) at \(0\), so does \(\phi\) so it has a holomorphic inverse.

  For uniqueness suppose \(\psi\) is another such coordinate, then for \(w \in \psi(U)\) have \(\lambda \phi(\psi^{-1}(w)) = \phi(\psi^{-1}(\lambda w))\). Done by comparing local power series.

  For \(|\lambda| > 1\) apply the same argument to a branch of \(f^{-1}\).
\end{proof}

\begin{corollary}
  Suppose \(p\) is an attracting fixed point of \(f\) with multiplier \(\lambda \ne 0\) and basin \(A\). Then exists a holomorphic \(\phi: A \to \C\) such that the following diagram commutes
  \[
    \begin{tikzcd}
      A \ar[r, "f"] \ar[d, "\phi"] & A \ar[d, "\phi"] \\
      \C \ar[r, "\lambda"] & \C
    \end{tikzcd}
  \]
\end{corollary}

\begin{proof}
  Define \(\phi(z) = \lim_{n \to \infty} \frac{\phi_0(f^n(z))}{\lambda^n}\) where \(\phi_0\) is the linearlising coordinates on a neighbourhood of \(p\). Check the details.
\end{proof}

\begin{definition}[immediate basin]\index{immediate basin}
  The \emph{immediate basin} of an attracting cycle is the union of the Fatou components containing the cycle elements.
\end{definition}

...

\(\frac{z^2 - 1}{z^2 - c}\) attracting \(5\)-cycle \(\infty \mapsto 1 \mapsto 0 \mapsto \frac{1}{c}\)

\begin{proposition}
  Let \(f\) be a rational map with \(f(p) = p\) an attracting fixed point. Then the immediate basin of \(p\) contains a critical point of \(f\).
\end{proposition}

\begin{proof}
  wlog \(p = 0\). The component \(U\) of \(F(f)\) is hyperbolic as \(|J(f)| = \infty\). Thus we have
  \[
    \begin{tikzcd}
      \D \ar[r, "F"] \ar[d, "\pi"] & \D \ar[d, "\pi"] \\
      U \ar[r, "f"] & U
    \end{tikzcd}
  \]
  If \(f\) has no critical points in \(U\) then \(f \compose \pi\) is a covering map \(\D \to U\) so exists \(G: \D \to \D\) covering it. If \(\pi, F, G\) fix \(0\), \(G\) is inverse to \(F\). Thus \(F \in \aut(\D)\) so \(F, f\) are hyperbolic local isometries, contradicting \(0\) an attracting fixed point.
\end{proof}

\begin{corollary}
  \(f\) has at most \(2d - 2\) attracting cycles.
\end{corollary}

\begin{corollary}
  \(f\) has at most \(4d - 4\) non-repelling cycles.
\end{corollary}

\begin{proof}
  Holomorphic perturbation. Let \(f_t(z) = (1 - t) f(z) + tz^d\). Note \(f_0 = f(z), f_1 = z^d\). Suppose \(f^n(\alpha) = \alpha\) with multiplier \(\lambda \in S^1\). If \(\alpha\) is not a repeated root of \(f^n(z) - z\) there is a neighbourhood of \(0\) and holomorphic \(t \mapsto \alpha(t)\) such that \(\alpha(0) = \alpha\) and \(f^n(\alpha(t)) = \alpha(t)\) for all \(t\), i.e.\ if \(\lambda \ne 1\) (?). But if \(\lambda = 1\) we can base change \(t \mapsto t^k\). We then have \(t \mapsto \lambda(t)\) homomorphic in \(t\), \(\lambda(0) = \lambda\) and \((f^n)'(\alpha(t)) = \lambda(t)\). Either \(\lambda(t)\) is the constant \(1\) (more argument needed), or another constant \(\lambda\), or nonconstant. The first two cases contradict \(z^d\) having no indifferent cycles at \(t = 1\).

  By conformality of holomorphic maps, the measure
  \[
    \mu(\{\theta \in S^1: |\lambda(\varepsilon e^{i\theta})|\}) \to \frac{1}{2}
  \]
  as \(\varepsilon \to 0\). Repeating this process for all indifferent cycles, exists a direction \(\theta\) such that perturbation in the \(\theta\)-direction makes half of these cycles attracting. For sufficiently small choice of \(\varepsilon e^{i\theta}\), attracting cycles remain attracting. Let \(N\) be the number of indifferent cycles of \(f\), \(M\) the number of attracting cycles of \(f\), then the number of non-repelling cycles of \(f\) is \(N + M = 2(M/2 + N/2) \leq 2 (2d - 2)\).
\end{proof}

\begin{remark}
  \(f^n(z) = z\) has \(z^n + 1\) roots counting multiplicity, so must have a repelling cycle.
\end{remark}

Note we can be more precise, see example sheet.

\begin{theorem}
  If \(f(0) = 0\) is attracting with multiplier \(\lambda \ne 0\). Let \(\phi\) be a linearising coordinate with local inverse \(\psi: \D(0, \varepsilon) \to A_0\), where \(A_0\) is the immediate basin of \(0\). \(\psi\) extends to a holomorphic map on a disk \(\D(0, r)\) of some maximal radius \(r\), extending homeomorphically to \(\p \D(0, r)\) and \(\psi(\p \D(0, r))\) contains a critical point of \(f\).
\end{theorem}

\begin{remark}
  Actually detecting whether \(f\) has an attractor is harder. Open problem: does \(z \mapsto z^2 - \frac{3}{2}\) has an attractor?
\end{remark}

Caution: linearising map need not continuously extend to \(J(f)\).

\begin{theorem}
  If \(f\) rational has \(J(f)\) disconnected then \(J(f)\) has uncountably many connected components.
\end{theorem}

\begin{proof}
  If \(J(f) = J_0 \cup J_1\) where \(J_0, J_1\) are disjoint compact nonempty. Given \(z \in J\), define a seuqnce \(\beta(z) = (\beta_n(z))\) where \(\beta_n(z) = i\) if \(f^n(z) \in J_i\). If \(z, w\) are in the same connected component of \(J(f)\) then \(\beta(z) = \beta(w)\). It suffices to show that for any initial \(\beta_1(z), \dots, \beta_k(z)\), exists \(n > k\) such that exists \(z' \in J(f)\) such that \(\beta_i(z') = \beta_i(z)\) for all \(1 \leq i \leq k\) but \(\beta_n(z') \ne \beta_n(z)\). Define
  \[
    U_{z, k} = \{w \in \hat \C: f^i(w) \notin J_{1 - \beta_i(z)} \text{ for all } 1 \leq i \leq k\}.
  \]
  This is open and contains \(F(f)\). Some subsequence \((\beta_{n_j}(z))\) is constant, say the constant \(0\). If \(\beta_i(z') = \beta_i(z)\) for all \(1 \leq i \leq k\) then \(\beta_i(z') = \beta_i(z)\) for all \(i\), then
  \[
    f^{n_j}(U_{z, k}) \subseteq \C \setminus J_1.
  \]
  The maps \(f^{nj}: U_{z, k} \to \C \setminus J_1\) form a normal family, contradiction. Thus \(\{\beta(z): z \in J(f)\}\) is uncountable.
\end{proof}

\paragraph{Superattractor}

\begin{theorem}
  Suppose \(f(0) = 0\) is with local expansion \(f(z) = a_mz^m + a_{m + 1}z^{m + 1} + \dots, m \geq 2\). Then there exists a holomorphic change of coordinates \(\phi\) on a neighbourhood of \(0\) such that \(\phi(0) = 0\), \(\phi(f(z)) = \phi(z)^m\). \(\phi\) is unique up to multiplication by an \((m - 1)\)th root of unity. \(\phi\) is called the \emph{Böttcher coordinate}\index{Böttcher coordinate}.
\end{theorem}

\begin{proof}
  We sketch the proof only. The details are the same as Kaenig's. Write locally \(f(z) = z^m(1 + h(z))\), where \(h(z) \to 0\) as \(z \to 0\) is holomorphic, where \(a_m = 1\) (otherwise conjugate by \(\alpha f(z/\alpha)\)). Write \(1 + h(z) = \exp(k(z))\) for some holomorphic \(h(z)\) on a neighbourhood of \(0\). Then there exists holomorphic \(k_n(z)\) on this neighbourhood so that \(f^n(z) = z^{m^n} \exp(k_n(z))\). Choose the branch \(\phi_n(z)\) of the \(m^n\)th root of \(f^n(z)\) such that \(\phi_n(z) = z(1 + O(z))\). Then \(\phi_n\) converges uniformly to some holomorphic \(\phi\) on this neighbourhood which satisfies the statement. Uniqueness follows from identification of Taylor expansion, a la Kaenig.
\end{proof}

\begin{corollary}
  Let \(f(0) = 0\) be superattracting, with basin \(A\) and Böttcher coordinate \(\phi\) on a neighbourhood of \(0\). Then \(z \mapsto |\phi(z)|\) extends to a continuous map \(|\phi|: A \to [0, 1)\) satisfying \(|\phi(f(z))| = |\phi(z)|^m\) for \(z \in A\).
\end{corollary}

\begin{proof}
  Given \(z \in A\), set \(|\phi|(z) = |\phi(f^n(z))|^{1/m^n}\), where \(n \gg 1\) such that \(f^n(z)\) is in the neighbourhood domain of \(\phi\). The desired equality is immediate.
\end{proof}

\section{Polynomial dynamics}

\begin{definition}[filled Julia set]\index{filled Julia set}
  Let \(p(z) = a_dz^d + a_{d - 1}z^{d - 1} + \dots + a_0, d \geq 2, a_d \ne 0\). The \emph{filled Julia set} of \(p\) is
  \[
    K(p) = \{z \in \C: |f^n(z)| \nto \infty \text{ as } n \to \infty\}.
  \]
\end{definition}

Note this is the complement of the basin of infinity.

From our results on boundaries of basins, \(\p K(p) = J(p)\). We know we have a Böttcher coordinate on a neighbourhood of \(\infty\): choose this (i.e.\ \(\phi(1/z)^{-1}\)) such that \(\phi(\infty) = \infty\).

\begin{definition}[Green's function]\index{Green's function}
  Suppose \(p(z)\) is a degree \(d\) polynomial. The \emph{Green's function} associated to \(p\) is
  \[
    G_p(z) = \lim_{n \to \infty} \frac{\log^+ |p^n(z)|}{d^n}
  \]
  where \(\log^+x = \max\{\log x, 0\}\) for \(x \geq 0\).
\end{definition}

\begin{lemma}
  \(G_p(z)\) satisfying the following:
  \begin{enumerate}
  \item \(G_p\) is continuous everywhere and harmonic on \(\C \setminus K(p)\).
  \item \(G_p(z) = \log |z| + O(1)\) as \(|z| \to \infty\).
  \item \(G_p(z) \to 0\) as \(z \to K(p)\).
  \item \(G_p(p(z)) = d G_p(z)\).
  \end{enumerate}
  1, 2, 4 uniquely characterises \(G_p\), and \(G_p(z) = \log |\phi_p(z)|\), where \(\phi_p\) is a Böttcher coordinate at \(\infty\) on \(\hat \C \setminus K(p)\).
\end{lemma}

\begin{remark}\leavevmode
  \begin{enumerate}
  \item This is how pictures of filled Julia sets are drawn.
  \item The Green's function depends only on \(K(p)\).
  \end{enumerate}
\end{remark}

\begin{proof}\leavevmode
  \begin{enumerate}
  \item Consider the function \(\log^+|p(z)| - d \log^+|z|\) on \(\hat \C\). It is continuous and takes real values, so is bounded by some \(C \in \R\). Then for all \(n\),
    \[
      \left| \frac{\log^+|p^n(z)|}{d^n} - \frac{\log^+|p^{n - 1}(z)|}{d^{n - 1}}\right| \leq \frac{C}{d^n}
    \]
    so for \(m \leq n\),
    \[
      \left| \frac{\log^+|p^n(z)|}{d^n} - \frac{\log^+|p^m(z)|}{d^m}\right| \leq \sum_{k = m + 1}^n \frac{C}{d^k} \leq \frac{C}{d^m (d - 1)}
    \]
    so \(G_p\) is a uniform limit of continuous function so continuous.

    Locally, a function is harmonic if and only if it is the real part of a holomophic function, if and only if it equals to \(\log |f|\) for some holomorphic \(f\) that does not vanish anywhere (since on a simply connected domain we can take logarithm). Given \(z \notin K(p)\), find a small disk \(D \ni p\) such that \(\overline D \cap K(p) = \emptyset\). There exists \(N \gg 1\) such that \(p^n(\overline D) \cap \D = \emptyset\) for \(n \geq N\). Then \(\frac{\log^+ |p^n(z)|}{d^n}\)  is harmonic on \(\overline D\). Since a uniform limit of harmonics is harmonic, we have \(G_p(z)\) is harmonic on \(D\) as well. Note if \(K(p)^{\mathrm{int}}(p) \ne \emptyset\) then \(G_p(z) = 0\) there so is harmonic as well. In other words, \(G_p(z)\) fails to be harmonic precisely on the Julia set (for more rigorous argument see later).
  \item Set \(m = 0\), then the bound in 1 gives
    \[
      \left| \frac{\log^+ |p^n(z)|}{d^n} - \log^+|z| \right| \leq \frac{C}{d - 1}
    \]
    so as \(n \to \infty\), \(|G_p(z) - \log |z|| \leq \frac{C}{d - 1}\) for \(|z| \gg 0\).
  \item \(G_p(z) = 0\) on \(K(p)\).
  \item Definition.
  \end{enumerate}

  Suppose \(H(z)\) is a function satisfying 1, 2 and 4 and consider \(G(z) = G_p(z) - H(z)\). By 1 and 2 it is continuous and bounded on \(\hat \C\). By 4, as \(n \to \infty\), \(G(p^n(z)) = d^n G(z) \to \infty\) unless \(G(z) = 0\). We thus have \(G_p(z) = H(z)\). For \(G_p(z) = \log |\phi_p(z)|\), check continuity, growth at \(\infty\) and transformation.
\end{proof}

\begin{eg}
  For \(z \mapsto z^d\),
  \[
    G_p(z) = \lim \frac{\log^+|z^{d^n}|}{d^n} = \log^+|z|.
  \]
  \(K(p) = \overline D\), where \(\log^+ |z| = 0\). The basin of infinity is \(\hat \C \setminus \D\).
\end{eg}

\begin{remark}
  \(G_p(z)\) is also known as the \emph{potential function}\index{potential function} associated to \(K(p)\).
\end{remark}

Now back to superattractors.

\begin{theorem}
  Suppose \(f(0) = 0\) is superattracting, with Böttcher coordinate \(\phi\) for \(f\) at \(0\). There there exists a unique open disk \(\D(0, r)\) of maximal radius \(0 < r \leq 1\) such that the inverse \(\psi\) of \(\phi\) extends holomorphically to \(\psi: \D(0, r) \to A_0\), the immediate basin of attraction of \(0\). If \(r = 1\) then \(\psi: D(0, t) \cong A_0\) and \(0\) is the only critical points of \(f\) in \(A_0\). On the other hand if \(r < 1\) there exists a nonzero critical point in \(A_0\), which lies on \(\b \psi(\D(0, r))\).
\end{theorem}

\begin{proof}
  Guided on example sheet 2. Non-examinable.
\end{proof}

\begin{eg}
  \(f(z) = z^2 + \frac{1}{2}\). \(\phi\) sends a neighbourhood of \(\infty\) to the complement of a large closed disk in \(\hat \C\) isomorphically. \(\psi\) can be extended until it hits the image of a critical point.
\end{eg}

In the case \(f_c(z) = z^2 + c\), there are two critical points \(\infty, 0\). \(\infty\) is mapped to itself with multiplicity \(2\). We have

\begin{corollary}
  Suppose \(0 \notin K(f_c)\), i.e.\ \(f_c^n(0) \to \infty\) as \(n \to \infty\). Then the Böttcher coordinate \(\phi_c\) of \(f_c\) at \(\infty\) extends to a conformal isomorphism on a neighbourhood of \(\infty\) which contains \(c\).
\end{corollary}

\begin{proof}
  \(0\) is the only critical point that can move around and \(f(0) = c\). Now use extension of Böttcher coordinate.
\end{proof}

\begin{proposition}
  A closed subset of the sphere is connected if and only if the connected components of its complement are simply connected.
\end{proposition}

\begin{proof}
  Beardon, Iteration of Rational Functions and Ahlfors.
\end{proof}








\printindex
\end{document}

% prerequisite:
% ES 0: basics of Riemann surfaces
% measure theory: definitions and basics up to Fubini's
%
% References
% McMullen: Riemann surfaces, complex dynamics and hperbolic geometry
% potential theory: Ramsford book