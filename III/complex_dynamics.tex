\documentclass[a4paper]{article}

\def\npart{III}

\def\ntitle{Complex Dynamics}
\def\nlecturer{H.\ Krieger}

\def\nterm{Lent}
\def\nyear{2020}

\ifx \nauthor\undefined
  \def\nauthor{Qiangru Kuang}
\else
\fi

\ifx \ntitle\undefined
  \def\ntitle{Template}
\else
\fi

\ifx \nauthoremail\undefined
  \def\nauthoremail{qk206@cam.ac.uk}
\else
\fi

\ifx \ndate\undefined
  \def\ndate{\today}
\else
\fi

\title{\ntitle}
\author{\nauthor}
\date{\ndate}

%\usepackage{microtype}
\usepackage{mathtools}
\usepackage{amsthm}
\usepackage{stmaryrd}%symbols used so far: \mapsfrom
\usepackage{empheq}
\usepackage{amssymb}
\let\mathbbalt\mathbb
\let\pitchforkold\pitchfork
\usepackage{unicode-math}
\let\mathbb\mathbbalt%reset to original \mathbb
\let\pitchfork\pitchforkold

\usepackage{imakeidx}
\makeindex[intoc]

%to address the problem that Latin modern doesn't have unicode support for setminus
%https://tex.stackexchange.com/a/55205/26707
\AtBeginDocument{\renewcommand*{\setminus}{\mathbin{\backslash}}}
\AtBeginDocument{\renewcommand*{\models}{\vDash}}%for \vDash is same size as \vdash but orginal \models is larger
\AtBeginDocument{\let\Re\relax}
\AtBeginDocument{\let\Im\relax}
\AtBeginDocument{\DeclareMathOperator{\Re}{Re}}
\AtBeginDocument{\DeclareMathOperator{\Im}{Im}}
\AtBeginDocument{\let\div\relax}
\AtBeginDocument{\DeclareMathOperator{\div}{div}}

\usepackage{tikz}
\usetikzlibrary{automata,positioning}
\usepackage{pgfplots}
%some preset styles
\pgfplotsset{compat=1.15}
\pgfplotsset{centre/.append style={axis x line=middle, axis y line=middle, xlabel={$x$}, ylabel={$y$}, axis equal}}
\usepackage{tikz-cd}
\usepackage{graphicx}
\usepackage{newunicodechar}

\usepackage{fancyhdr}

\fancypagestyle{mypagestyle}{
    \fancyhf{}
    \lhead{\emph{\nouppercase{\leftmark}}}
    \rhead{}
    \cfoot{\thepage}
}
\pagestyle{mypagestyle}

\usepackage{titlesec}
\newcommand{\sectionbreak}{\clearpage} % clear page after each section
\usepackage[perpage]{footmisc}
\usepackage{blindtext}

%\reallywidehat
%https://tex.stackexchange.com/a/101136/26707
\usepackage{scalerel,stackengine}
\stackMath
\newcommand\reallywidehat[1]{%
\savestack{\tmpbox}{\stretchto{%
  \scaleto{%
    \scalerel*[\widthof{\ensuremath{#1}}]{\kern-.6pt\bigwedge\kern-.6pt}%
    {\rule[-\textheight/2]{1ex}{\textheight}}%WIDTH-LIMITED BIG WEDGE
  }{\textheight}% 
}{0.5ex}}%
\stackon[1pt]{#1}{\tmpbox}%
}

%\usepackage{braket}
\usepackage{thmtools}%restate theorem
\usepackage{hyperref}

% https://en.wikibooks.org/wiki/LaTeX/Hyperlinks
\hypersetup{
    %bookmarks=true,
    unicode=true,
    pdftitle={\ntitle},
    pdfauthor={\nauthor},
    pdfsubject={Mathematics},
    pdfcreator={\nauthor},
    pdfproducer={\nauthor},
    pdfkeywords={math maths \ntitle},
    colorlinks=true,
    linkcolor={red!50!black},
    citecolor={blue!50!black},
    urlcolor={blue!80!black}
}

\usepackage{cleveref}



% TODO: mdframed often gives bad breaks that cause empty lines. Would like to switch to tcolorbox.
% The current workaround is to set innerbottommargin=0pt.

%\usepackage[theorems]{tcolorbox}





\usepackage[framemethod=tikz]{mdframed}
\mdfdefinestyle{leftbar}{
  %nobreak=true, %dirty hack
  linewidth=1.5pt,
  linecolor=gray,
  hidealllines=true,
  leftline=true,
  leftmargin=0pt,
  innerleftmargin=5pt,
  innerrightmargin=10pt,
  innertopmargin=-5pt,
  % innerbottommargin=5pt, % original
  innerbottommargin=0pt, % temporary hack 
}
%\newmdtheoremenv[style=leftbar]{theorem}{Theorem}[section]
%\newmdtheoremenv[style=leftbar]{proposition}[theorem]{proposition}
%\newmdtheoremenv[style=leftbar]{lemma}[theorem]{Lemma}
%\newmdtheoremenv[style=leftbar]{corollary}[theorem]{corollary}

\newtheorem{theorem}{Theorem}[section]
\newtheorem{proposition}[theorem]{Proposition}
\newtheorem{lemma}[theorem]{Lemma}
\newtheorem{corollary}[theorem]{Corollary}
\newtheorem{axiom}[theorem]{Axiom}
\newtheorem*{axiom*}{Axiom}

\surroundwithmdframed[style=leftbar]{theorem}
\surroundwithmdframed[style=leftbar]{proposition}
\surroundwithmdframed[style=leftbar]{lemma}
\surroundwithmdframed[style=leftbar]{corollary}
\surroundwithmdframed[style=leftbar]{axiom}
\surroundwithmdframed[style=leftbar]{axiom*}

\theoremstyle{definition}

\newtheorem*{definition}{Definition}
\surroundwithmdframed[style=leftbar]{definition}

\newtheorem*{slogan}{Slogan}
\newtheorem*{eg}{Example}
\newtheorem*{ex}{Exercise}
\newtheorem*{remark}{Remark}
\newtheorem*{notation}{Notation}
\newtheorem*{convention}{Convention}
\newtheorem*{assumption}{Assumption}
\newtheorem*{question}{Question}
\newtheorem*{answer}{Answer}
\newtheorem*{note}{Note}
\newtheorem*{application}{Application}

%operator macros

%basic
\DeclareMathOperator{\lcm}{lcm}

%matrix
\DeclareMathOperator{\tr}{tr}
\DeclareMathOperator{\Tr}{Tr}
\DeclareMathOperator{\adj}{adj}

%algebra
\DeclareMathOperator{\Hom}{Hom}
\DeclareMathOperator{\End}{End}
\DeclareMathOperator{\id}{id}
\DeclareMathOperator{\im}{im}
\DeclareMathOperator{\coker}{coker}
\DeclarePairedDelimiter{\generation}{\langle}{\rangle}

%groups
\DeclareMathOperator{\sym}{Sym}
\DeclareMathOperator{\sgn}{sgn}
\DeclareMathOperator{\inn}{Inn}
\DeclareMathOperator{\aut}{Aut}
\DeclareMathOperator{\GL}{GL}
\DeclareMathOperator{\SL}{SL}
\DeclareMathOperator{\PGL}{PGL}
\DeclareMathOperator{\PSL}{PSL}
\DeclareMathOperator{\SU}{SU}
\DeclareMathOperator{\UU}{U}
\DeclareMathOperator{\SO}{SO}
\DeclareMathOperator{\OO}{O}
\DeclareMathOperator{\PSU}{PSU}
\DeclareMathOperator{\Sp}{Sp}


%hyperbolic
\DeclareMathOperator{\sech}{sech}

%field, galois heory
\DeclareMathOperator{\ch}{ch}
\DeclareMathOperator{\gal}{Gal}
\DeclareMathOperator{\emb}{Emb}



%ceiling and floor
%https://tex.stackexchange.com/a/118217/26707
\DeclarePairedDelimiter\ceil{\lceil}{\rceil}
\DeclarePairedDelimiter\floor{\lfloor}{\rfloor}


\DeclarePairedDelimiter{\innerproduct}{\langle}{\rangle}

%\DeclarePairedDelimiterX{\norm}[1]{\lVert}{\rVert}{#1}
\DeclarePairedDelimiter{\norm}{\lVert}{\rVert}



%Dirac notation
%TODO: rewrite for variable number of arguments
\DeclarePairedDelimiterX{\braket}[2]{\langle}{\rangle}{#1 \delimsize\vert #2}
\DeclarePairedDelimiterX{\braketthree}[3]{\langle}{\rangle}{#1 \delimsize\vert #2 \delimsize\vert #3}

\DeclarePairedDelimiter{\bra}{\langle}{\rvert}
\DeclarePairedDelimiter{\ket}{\lvert}{\rangle}




%macros

%general

%divide, not divide
\newcommand*{\divides}{\mid}
\newcommand*{\ndivides}{\nmid}
%vector, i.e. mathbf
%https://tex.stackexchange.com/a/45746/26707
\newcommand*{\V}[1]{{\ensuremath{\symbf{#1}}}}
%closure
\newcommand*{\cl}[1]{\overline{#1}}
%conjugate
\newcommand*{\conj}[1]{\overline{#1}}
%set complement
\newcommand*{\stcomp}[1]{\overline{#1}}
\newcommand*{\compose}{\circ}
\newcommand*{\nto}{\nrightarrow}
\newcommand*{\p}{\partial}
%embed
\newcommand*{\embed}{\hookrightarrow}
%surjection
\newcommand*{\surj}{\twoheadrightarrow}
%power set
\newcommand*{\powerset}{\mathcal{P}}

%matrix
\newcommand*{\matrixring}{\mathcal{M}}

%groups
\newcommand*{\normal}{\trianglelefteq}
%rings
\newcommand*{\ideal}{\trianglelefteq}

%fields
\renewcommand*{\C}{{\mathbb{C}}}
\newcommand*{\R}{{\mathbb{R}}}
\newcommand*{\Q}{{\mathbb{Q}}}
\newcommand*{\Z}{{\mathbb{Z}}}
\newcommand*{\N}{{\mathbb{N}}}
\newcommand*{\F}{{\mathbb{F}}}
%not really but I think this belongs here
\newcommand*{\A}{{\mathbb{A}}}

%asymptotic
\newcommand*{\bigO}{O}
\newcommand*{\smallo}{o}

%probability
\newcommand*{\prob}{\mathbb{P}}
\newcommand*{\E}{\mathbb{E}}

%vector calculus
\newcommand*{\gradient}{\V \nabla}
\newcommand*{\divergence}{\gradient \cdot}
\newcommand*{\curl}{\gradient \cdot}

%logic
\newcommand*{\yields}{\vdash}
\newcommand*{\nyields}{\nvdash}

%differential geometry
\renewcommand*{\H}{\mathbb{H}}
\newcommand*{\transversal}{\pitchfork}
\renewcommand{\d}{\mathrm{d}} % exterior derivative

%number theory
\newcommand*{\legendre}[2]{\genfrac{(}{)}{}{}{#1}{#2}}%Legendre symbol

%algebraic geometry
\DeclareMathOperator{\Spec}{Spec}
\DeclareMathOperator{\Proj}{Proj}

\newcommand{\D}{\mathbb{D}}
  
\begin{document}

\begin{titlepage}
  \begin{center}
    \includegraphics[width=0.6\textwidth]{logo.jpg}\par
    \vspace{1cm}
    {\scshape\huge Mathamatics Tripos \par}
    \vspace{2cm}
    {\huge Part \npart \par}
    \vspace{0.6cm}
    {\Huge \bfseries \ntitle \par}
    \vspace{1.2cm}
    {\Large\nterm, \nyear \par}
    \vspace{2cm}
    
    {\large \emph{Lectures by } \par}
    \vspace{0.2cm}
    {\Large \scshape \nlecturer}
    
    \vspace{0.5cm}
    {\large \emph{Notes by }\par}
    \vspace{0.2cm}
    {\Large \scshape \href{mailto:\nauthoremail}{\nauthor}}
 \end{center}
\end{titlepage}

\tableofcontents

\setcounter{section}{-1}

\section{Introduction}

What is complex dynamics? Iteration of holomorphic self-maps of Riemann surfaces

long term behaviour under iteration

origin: iterative root finding algorithm, e.g. Newton's method. When and why does this work? Algebraically, the problem is to ask the convergence of the iteration of the map
\[
  f(z) = z - \frac{p(z)}{p'(z)}: \hat \C \to \hat \C
\]
where \(\hat \C\) is the Riemann sphere.

Goals:
\begin{itemize}
\item equidistribution theorem

  \begin{theorem}[Friere-Lopez-Mañe, 1983]
    Let \(f: \hat \C \to \hat \C\) with degree \(d \geq 2\) holomorphic.  Then there exists a unique \(f\)-invariant probability measure \(\mu_f\) supported on the unstable locus of \(f\), such that for almost all \(\alpha \in \hat \C\), \(\frac{1}{d^n} \sum_{f^n(z) = \alpha} \delta_z \to \mu_f\) in weak-* topology.
  \end{theorem}
\item universality of the Mandelbrot set: fix \(d \geq 2\), define for \(c \in \C\), \(f_c(z): z^d + c\). The \(d\)-Mandelbrot set is
  \[
    M_d = \{c \in \C: |f_c^n(0)| \nto \infty\}.
  \]

  \begin{theorem}[McMullen 1997]
    The Mandelbrot set is universal for bifurcations, i.e. in any bifurcation locus we see (slightly distorted) copics of some \(M_d\).
  \end{theorem}
\end{itemize}

\section{Riemann surfaces}

Recall at the end of IID Riemann Surfaces

\begin{theorem}[uniformisation]
  Every Riemann surface \(R\) is conformally isomorphic to \(\tilde R/G\) where \(\tilde R\) is one of the three simply connected Riemann surfaces (\(\hat \C, \C, \D\)) and \(G \subseteq \aut(\tilde R)\) acts freely and properly discontinuously.
\end{theorem}

Three cases:
\begin{enumerate}
\item \(\hat \C\): as \(\aut(\hat \C)\) are precisely the Möbius transformations and all Möbius transformations have fixed points, \(G = 1\) so \(R = \hat \C\).
\item \(\C\): \(\aut(\C) = \{p(z) = az + b, a \ne 0\}\) so \(G\) only contains translations, so can be identified by a subgroup of \(\C\). Can show that \(G\) is one of \(\{0\}, \Z \omega_1\) or \(\Sigma = \Z\omega_1 \oplus \Z\omega_2\), a lattice. Then \(R\) is one of \(\C, \C^*\) or \(\C/\Lambda\), a complex torus.
\item \(\D\): \(\aut(\D) = \{\lambda \cdot \frac{z - a}{1 - \conj a z}: \lambda \in S^1, a \in \D\}\), which has a lot of elements with no fixed point. This is called \emph{hyperbolic}.
\end{enumerate}

Recall \(\D\) is equipped with metric \(ds = \frac{2 |dz|}{1 - |z|^2}\), i.e.\ the distance between \(x, y \in \D\) is
\[
  \rho(x, y) = \inf_\gamma \int_a^b \frac{2 |\gamma'(t)|}{1 - |\gamma(t)|^2} dt
\]
where \(\gamma: [a, b] \to \D\) is a smooth curve from \(x\) to \(y\). One can use the origin to show that
\[
  \rho(x, y) = \frac{\log(1 + R)}{\log(1 - R)}
\]
where \(R = |\frac{y - x}{1 - \conj x y}|\). Elements of \(\aut(\D)\) are isometries of this metric so it descends to a hyperbolic metric on any hyperbolic Riemann surface, such that the covering map \(\tilde R \to R\) is a local isometry.

\begin{ex}
  Let \(R = \D/G\), \(f: R \to R\) holomorphic. Then \(f\) lifts to a holomorphic \(\tilde f: \D \to \D\), unique up to a composition with an element of \(G\), and induces a group homomorphism \(\gamma \mapsto \gamma'\) such that \(\tilde f \compose \gamma = \gamma' \compose \tilde f\).
\end{ex}

\begin{theorem}[Pick]
  Let \(f: S \to T\) be a hyperbolic Riemann surfaces with Poincaré metric \(\rho_S, \rho_T\) respectively. The for all \(x, y \in S\),
  \[
    \rho_T(f(x), f(y)) \leq \rho_S(x, y).
  \]
\end{theorem}

\begin{proof}
  By lifting (of \(f \compose \pi_S: \D \to T\)) it suffices to show this for \(S = T = \D\). By computation this is the same as showing
  \[
    \frac{\log (1 + R')}{\log (1 - R')} \leq \frac{\log (1 + R)}{\log (1 - R)}
  \]
  where \(R = \cdots\). Note \(\frac{\log (1 + x)}{\log (1 - x)}\) is strictly increasing, so suffices to show \(R' \leq R\). Let \(\mu_1(z) = \frac{y - z}{1 - \conj y z}, \mu_2 = \frac{f(y) - z}{1 - \conj{f(y)} z}\). Let \(g = \mu_2 \compose f \compose \mu_1^{-1}: \D \to \D\). By Schwarz's lemma \(|g(z)| \leq |z|\), i.e.\ \(|\mu_2(f(w))| \leq |\mu_1(w)|\), so \(R' \leq R\).
\end{proof}

\begin{remark}
  It follows from the ``strict inequality'' bit of Schwarz's lemma that exists \(x, y\) such that \(\rho_T(f(x), f(y)) = \rho_S(x, y)\) if and only if \(f\) lifts to a disk automorphism.
\end{remark}

\begin{eg}
  Contracting holomorphic maps is a very strong requirement. Compre to, for example, \(f(z) = z + 1\) on \(\hat \C\) (drawing of different behaviour on two hemispheres).
\end{eg}

Case of \(\hat \C\):

\begin{proposition}
  Let \(f: \hat \C \to \hat \C\) be a holomorphic nonconstant map. Then \(f\) is a rational map, i.e.\ exists \(a_1, \dots, a_m, b_1, \dots, b_n, c \in \C\) such that
  \[
    f(z) = c \cdot \frac{(z - a_1) \cdots (z- a_m)}{(z - b_1) \cdots (z- b_n)}.
  \]
\end{proposition}

\begin{proof}
  wlog \(f(\infty) \in \C\) (if not, replace with \(\frac{1}{f}\)). Then exist a finite collection \(f^{-1}(\infty) = \{b_1, \dots, b_n\} \subseteq \C\). About any \(b_i\) have locally
  \[
    f(z) = \sum_{j = -k}^\infty a_{ij} (z - b_i)^j.
  \]
  Set \(Q_i = \sum_{j = -k}^{-1} a_{ij} (z - b_i)^j\). Then \(g - f - Q_1 - \dots - Q_n\) has no pole so must be constant.
\end{proof}

Universal cover \(\C\): just a remark. Yes there are interesting dynamics. For example \(z \mapsto e^z\) belongs to the realm of transcendental dynamics. \(\C/\Lambda\) also admits nonconstant holomorphic maps. See example sheet 1.

\paragraph{stable and unstable locus}

Motivating example: \(z \mapsto z^2\) on \(\hat \C\). We can restrict \(f\) to the unit disk and we see \(f|_\D^n \to 0\) uniformly on compact subsets of \(\D\). Similarly on \(\hat \C \setminus \overline \D\), \(f|_{\hat \C \setminus \overline \D}^n \to \infty\). On the other hand for \(|z_0| = 1\), there is no neighbourhoood \(z_0 \in U\) such that \(f|_U^n(z)\) converging to a holomorphic function, as such a limit would have a discontinuity.

\begin{definition}[locally uniform convergence/divergence]\index{locally uniform convergence}\index{locally uniform divergence}
  Let \(S, T\) be metric spaces, \(f_n: S \to T\) a sequence of continuous maps. We say \((f_n)\) \emph{converges locally uniformly} if for all compact \(K \subseteq S\), for all \(\varepsilon > 0\) exists \(N \in \N\) such that for all \(m, n > N\), \(\sup_{x \in K} d_T(f_m(x), f_n(x)) < \varepsilon\).

  We say \((f_n)\) \emph{diverges locally uniformly} if for all compact \(K \subseteq S\) and all compact \(K' \subseteq T\), exists \(N \in \N\) such that for all \(n > N\), \(f_n(K) \cap K' = \emptyset\).
\end{definition}

Recall that if holomorphic \(f_n\)'s converge locally uniformly on \(S\) then it has a holomorphic limit function.

\begin{remark}
  If \(T\) is compact then we never have locally uniform divergence.
\end{remark}

\begin{definition}[normal family]\index{normal family}
  We say a family \(\mathcal F = \{f: S \to T\}\) of continuous functions is \emph{normal} if every sequence \((f_n) \subseteq \mathcal F\) has a subsequence which either converges locally uniformly or diverges locally uniformly.
\end{definition}

\begin{ex}\leavevmode
  \begin{enumerate}
  \item Show normality depends only on the topology, not the metric, of the spaces.
  \item Normality is local: if \(\mathcal F\) is a family of continuous maps, \(S = \bigcup_\alpha U_\alpha\), then if \(\mathcal F|_{U_\alpha}\) is normal for all \(\alpha\) then \(\mathcal F\) is normal for \(S\).
  \end{enumerate}
\end{ex}

A word of caution: in some texts (such as Ahlfors) the definition of normality excludes divergence.

\begin{definition}[equicontinuity]\index{equicontinuity}
  A family \(\mathcal F\) of continuous maps on a domain \(U \subseteq \C\) with values in a metric space \(T\) is \emph{equicontinuous} if for all \(\varepsilon > 0\), exists \(\delta > 0\) such that for all \(z, w \in U\) and for all \(f \in \mathcal F\), if \(|z - w| < \delta\) then \(d_T(f(z), f(w)) < \varepsilon\).
\end{definition}

\begin{theorem}[Arzela-Ascoli]
  Let \(\mathcal F\) be a family of continuous maps \(U \to T\) with \(U \subseteq \C\) a domain and \(T\) a metric space. Then \(\mathcal F\) has the property that any sequence has a locally uniformly convergent subsequence if and only if
  \begin{enumerate}
  \item \(\mathcal F\) is equicontinuous on every compact \(K \subseteq U\),
  \item for every \(z \in U\), \(\{f(z): f \in \mathcal F\}\) lies in a compact subset of \(T\).
  \end{enumerate}
\end{theorem}

\begin{corollary}
  If \(T\) is compact then \(\mathcal F\) is normal if and only if it is equicontinuous on compact subsets of \(U\).
\end{corollary}

\begin{proof}
  Only if is easy. For if, we use separability of \(\C\). Let \(\{z_k\}\) be a countable dense subset of \(U\) and fix \(\{f_n\} \subseteq \mathcal F\). Can find a set of indices \(n_{11} < n_{12} < \cdots\) such that \(f_{n_{1i}}(z_1)\) converges, and a subsequence of these \(n_{21} < n_{22} < \cdots\) such that \(f_{n_{2i}}(z_2)\) converges. Let \(g_k = f_{n_{kk}}\). Then for all \(z_i\) we have \(\lim_{k \to \infty} g_n(z_i)\) exists in \(T\). Now given \(K \subseteq U\) compact, by equicontinuity for any \(\varepsilon > 0\) exists \(\delta > 0\) such that for all \(z, w \in U\) with \(|z - w| < \delta\), we have for all \(f \in \mathcal F\), \(d_T(f(z), f(w)) < \varepsilon\). Cover \(K\) by \(\delta\)-balls, extract a finite subcover, and choose some \(z_i\) in each, say \(z_1, \dots, z_\ell\). For each \(z_i\), \(1 \leq i \leq \ell\), exists \(N_i\) such that for all \(n, m \geq N_i\), \(d_T(g_n(z_i), g_m(z_i)) < \varepsilon\). Let \(N = \max_i N_i\). Then for all \(z \in K\),
  \[
    d_T(g_n(z), g_m(z)) \leq d_T(g_n(z), g_n(z_I)) + d_T(g_n(z_i), g_m(z_i)) + d_T(g_n(z_i), g_m(z)) < 3\varepsilon.
  \]
\end{proof}

\begin{theorem}[Montel]\index{Montel's theorem}
  Suppose \(S, T\) are Riemann surfaces and \(T\) is hyperbolic. Then all families of holomorphic maps \(S \to T\) is normal.
\end{theorem}

\begin{remark}
  If \(S\) is not hyperbolic, lifting plus Liouville's theorem imply that all maps are constant. Given a family \(\mathcal F\) of constant maps, let \(\{f_n\} \subseteq \mathcal F\) be a sequence. Then if \(\{f_n(S)\}\) lies in a compact set in \(T\), then exists a convergent subsequence; if not, exists a subsequence which leaves any compact set, so diverges locally uniformlly.

  If \(S\) is hyperbolic, then if exists \(x \in S\) such that \(\{f_n(x)\}\) lies in a compact subset of \(T\), the same is true for all \(y \in S\) by Picks. This implies equicontinuity (?), so by Arzela-Ascoli, normal. Otherwise fix \(x \in S\) and \(y \in T\). Given \(\{f_n\} \subseteq \mathcal F\), exists a subsequence \(\{f_{n_k}\}\) such that \(d_T(f_{n_k}(x), y) \to \infty\). Given \(K \subseteq S, K' \subseteq T\) compact, by Picks, \(d_T(f_{n_k}(z), y) \to \infty\) (?) and so \(f_{n_k}(K) \cap K' = \emptyset\) for \(k \gg 1\). Thus the sequence diverges locally uniformlly.
\end{remark}

\begin{eg}
  If \(f: \hat \C \to \hat \C\) is a rational map and \(D \subseteq \hat \C\), then the family of iterates \(\mathcal F = \{f^n\}_{n \in \N}\) is normal if \(\bigcup_{n \in \N} f^n(D)\) omits 3 or more points of \(\hat \C\). Idea: any subset of \(\hat \C\) with complement of cardinality \(\geq 3\) is hyperbolic.
\end{eg}

\begin{definition}[proper map]\index{proper map}
  Suppose \(U, V\) are open subsets of Riemann surfaces and \(f: U \to V\). \(f\) is proper if for every \(K \subseteq V\) compact, \(f^{-1}(K)\) is compact in \(U\).
\end{definition}

The proofs are either the same as the compact case, or can use compact case directly.

\begin{proposition}
  Suppose \(U, V\) are open in \(\hat \C\). If \(f: U \to V\) is proper holomorphic nonconstant, then \(f\) has a well-defined degree, i.e.\ for all \(v \in V\), \(|f^{-1}(x)|\) is independent of \(x\), counting multiplicity.
\end{proposition}

\begin{theorem}[Riemann-Hurwitz]
  Same assumption as above. Then the Euler characteristic \(\chi(U)\), \(\chi(V)\) satisfies
  \[
    \chi(U) = (\deg f) \chi(V) - \sum_{p \in U}(e_p - 1)
  \]
  where \(e_p\) is the local degree/ramification index of \(f\) at \(p\).
\end{theorem}

Note that one strategy is to observe that for \(U \subseteq \C\), we can cover by \(\delta\)-grid, let \(\delta \to 0\), we obtain a covering of \(U\) by open balls with compact closures and smooth boundaries.

\begin{corollary}
  If \(f: \hat \C \to \hat \C\) is holomorphic of degree \(d\), then \(f\) has \(2d - 2\) critical points, counting multiplicity.
\end{corollary}

\begin{corollary}
  If \(\chi(U) = \chi(V) = 0\) then \(f: U \to V\) proper holomorphic nonconstant is unramified.
\end{corollary}

\begin{definition}[Fatou set, Julia set]\index{Fatou set}\index{Julia set}
  Let \(f: \hat \C \to \hat \C\) be a holomorphic nonconstant map. The \emph{Fatou set} of \(f\) is
  \[
    F(f) = \{z \in \hat \C: \text{ on some nbhd } z \in U, f^n \text{ forms a normal family}\}.
  \]
  The \emph{Julia set} of \(f\) is \(J(f) = \hat \C \setminus F(f)\).
\end{definition}


\printindex
\end{document}

% prerequisite:
% ES 0: basics of Riemann surfaces
% measure theory: definitions and basics up to Fubini's
%
% References
% McMullen: Riemann surfaces, complex dynamics and hperbolic geometry
% potential theory: Ramsford book