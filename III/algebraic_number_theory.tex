\documentclass[a4paper]{article}

\def\npart{III}

\def\ntitle{Algebraic Number Theory}
\def\nlecturer{J.\ A.\ Thorne}

\def\nterm{Michaelmas}
\def\nyear{2019}

\ifx \nauthor\undefined
  \def\nauthor{Qiangru Kuang}
\else
\fi

\ifx \ntitle\undefined
  \def\ntitle{Template}
\else
\fi

\ifx \nauthoremail\undefined
  \def\nauthoremail{qk206@cam.ac.uk}
\else
\fi

\ifx \ndate\undefined
  \def\ndate{\today}
\else
\fi

\title{\ntitle}
\author{\nauthor}
\date{\ndate}

%\usepackage{microtype}
\usepackage{mathtools}
\usepackage{amsthm}
\usepackage{stmaryrd}%symbols used so far: \mapsfrom
\usepackage{empheq}
\usepackage{amssymb}
\let\mathbbalt\mathbb
\let\pitchforkold\pitchfork
\usepackage{unicode-math}
\let\mathbb\mathbbalt%reset to original \mathbb
\let\pitchfork\pitchforkold

\usepackage{imakeidx}
\makeindex[intoc]

%to address the problem that Latin modern doesn't have unicode support for setminus
%https://tex.stackexchange.com/a/55205/26707
\AtBeginDocument{\renewcommand*{\setminus}{\mathbin{\backslash}}}
\AtBeginDocument{\renewcommand*{\models}{\vDash}}%for \vDash is same size as \vdash but orginal \models is larger
\AtBeginDocument{\let\Re\relax}
\AtBeginDocument{\let\Im\relax}
\AtBeginDocument{\DeclareMathOperator{\Re}{Re}}
\AtBeginDocument{\DeclareMathOperator{\Im}{Im}}
\AtBeginDocument{\let\div\relax}
\AtBeginDocument{\DeclareMathOperator{\div}{div}}

\usepackage{tikz}
\usetikzlibrary{automata,positioning}
\usepackage{pgfplots}
%some preset styles
\pgfplotsset{compat=1.15}
\pgfplotsset{centre/.append style={axis x line=middle, axis y line=middle, xlabel={$x$}, ylabel={$y$}, axis equal}}
\usepackage{tikz-cd}
\usepackage{graphicx}
\usepackage{newunicodechar}

\usepackage{fancyhdr}

\fancypagestyle{mypagestyle}{
    \fancyhf{}
    \lhead{\emph{\nouppercase{\leftmark}}}
    \rhead{}
    \cfoot{\thepage}
}
\pagestyle{mypagestyle}

\usepackage{titlesec}
\newcommand{\sectionbreak}{\clearpage} % clear page after each section
\usepackage[perpage]{footmisc}
\usepackage{blindtext}

%\reallywidehat
%https://tex.stackexchange.com/a/101136/26707
\usepackage{scalerel,stackengine}
\stackMath
\newcommand\reallywidehat[1]{%
\savestack{\tmpbox}{\stretchto{%
  \scaleto{%
    \scalerel*[\widthof{\ensuremath{#1}}]{\kern-.6pt\bigwedge\kern-.6pt}%
    {\rule[-\textheight/2]{1ex}{\textheight}}%WIDTH-LIMITED BIG WEDGE
  }{\textheight}% 
}{0.5ex}}%
\stackon[1pt]{#1}{\tmpbox}%
}

%\usepackage{braket}
\usepackage{thmtools}%restate theorem
\usepackage{hyperref}

% https://en.wikibooks.org/wiki/LaTeX/Hyperlinks
\hypersetup{
    %bookmarks=true,
    unicode=true,
    pdftitle={\ntitle},
    pdfauthor={\nauthor},
    pdfsubject={Mathematics},
    pdfcreator={\nauthor},
    pdfproducer={\nauthor},
    pdfkeywords={math maths \ntitle},
    colorlinks=true,
    linkcolor={red!50!black},
    citecolor={blue!50!black},
    urlcolor={blue!80!black}
}

\usepackage{cleveref}



% TODO: mdframed often gives bad breaks that cause empty lines. Would like to switch to tcolorbox.
% The current workaround is to set innerbottommargin=0pt.

%\usepackage[theorems]{tcolorbox}





\usepackage[framemethod=tikz]{mdframed}
\mdfdefinestyle{leftbar}{
  %nobreak=true, %dirty hack
  linewidth=1.5pt,
  linecolor=gray,
  hidealllines=true,
  leftline=true,
  leftmargin=0pt,
  innerleftmargin=5pt,
  innerrightmargin=10pt,
  innertopmargin=-5pt,
  % innerbottommargin=5pt, % original
  innerbottommargin=0pt, % temporary hack 
}
%\newmdtheoremenv[style=leftbar]{theorem}{Theorem}[section]
%\newmdtheoremenv[style=leftbar]{proposition}[theorem]{proposition}
%\newmdtheoremenv[style=leftbar]{lemma}[theorem]{Lemma}
%\newmdtheoremenv[style=leftbar]{corollary}[theorem]{corollary}

\newtheorem{theorem}{Theorem}[section]
\newtheorem{proposition}[theorem]{Proposition}
\newtheorem{lemma}[theorem]{Lemma}
\newtheorem{corollary}[theorem]{Corollary}
\newtheorem{axiom}[theorem]{Axiom}
\newtheorem*{axiom*}{Axiom}

\surroundwithmdframed[style=leftbar]{theorem}
\surroundwithmdframed[style=leftbar]{proposition}
\surroundwithmdframed[style=leftbar]{lemma}
\surroundwithmdframed[style=leftbar]{corollary}
\surroundwithmdframed[style=leftbar]{axiom}
\surroundwithmdframed[style=leftbar]{axiom*}

\theoremstyle{definition}

\newtheorem*{definition}{Definition}
\surroundwithmdframed[style=leftbar]{definition}

\newtheorem*{slogan}{Slogan}
\newtheorem*{eg}{Example}
\newtheorem*{ex}{Exercise}
\newtheorem*{remark}{Remark}
\newtheorem*{notation}{Notation}
\newtheorem*{convention}{Convention}
\newtheorem*{assumption}{Assumption}
\newtheorem*{question}{Question}
\newtheorem*{answer}{Answer}
\newtheorem*{note}{Note}
\newtheorem*{application}{Application}

%operator macros

%basic
\DeclareMathOperator{\lcm}{lcm}

%matrix
\DeclareMathOperator{\tr}{tr}
\DeclareMathOperator{\Tr}{Tr}
\DeclareMathOperator{\adj}{adj}

%algebra
\DeclareMathOperator{\Hom}{Hom}
\DeclareMathOperator{\End}{End}
\DeclareMathOperator{\id}{id}
\DeclareMathOperator{\im}{im}
\DeclareMathOperator{\coker}{coker}
\DeclarePairedDelimiter{\generation}{\langle}{\rangle}

%groups
\DeclareMathOperator{\sym}{Sym}
\DeclareMathOperator{\sgn}{sgn}
\DeclareMathOperator{\inn}{Inn}
\DeclareMathOperator{\aut}{Aut}
\DeclareMathOperator{\GL}{GL}
\DeclareMathOperator{\SL}{SL}
\DeclareMathOperator{\PGL}{PGL}
\DeclareMathOperator{\PSL}{PSL}
\DeclareMathOperator{\SU}{SU}
\DeclareMathOperator{\UU}{U}
\DeclareMathOperator{\SO}{SO}
\DeclareMathOperator{\OO}{O}
\DeclareMathOperator{\PSU}{PSU}
\DeclareMathOperator{\Sp}{Sp}


%hyperbolic
\DeclareMathOperator{\sech}{sech}

%field, galois heory
\DeclareMathOperator{\ch}{ch}
\DeclareMathOperator{\gal}{Gal}
\DeclareMathOperator{\emb}{Emb}



%ceiling and floor
%https://tex.stackexchange.com/a/118217/26707
\DeclarePairedDelimiter\ceil{\lceil}{\rceil}
\DeclarePairedDelimiter\floor{\lfloor}{\rfloor}


\DeclarePairedDelimiter{\innerproduct}{\langle}{\rangle}

%\DeclarePairedDelimiterX{\norm}[1]{\lVert}{\rVert}{#1}
\DeclarePairedDelimiter{\norm}{\lVert}{\rVert}



%Dirac notation
%TODO: rewrite for variable number of arguments
\DeclarePairedDelimiterX{\braket}[2]{\langle}{\rangle}{#1 \delimsize\vert #2}
\DeclarePairedDelimiterX{\braketthree}[3]{\langle}{\rangle}{#1 \delimsize\vert #2 \delimsize\vert #3}

\DeclarePairedDelimiter{\bra}{\langle}{\rvert}
\DeclarePairedDelimiter{\ket}{\lvert}{\rangle}




%macros

%general

%divide, not divide
\newcommand*{\divides}{\mid}
\newcommand*{\ndivides}{\nmid}
%vector, i.e. mathbf
%https://tex.stackexchange.com/a/45746/26707
\newcommand*{\V}[1]{{\ensuremath{\symbf{#1}}}}
%closure
\newcommand*{\cl}[1]{\overline{#1}}
%conjugate
\newcommand*{\conj}[1]{\overline{#1}}
%set complement
\newcommand*{\stcomp}[1]{\overline{#1}}
\newcommand*{\compose}{\circ}
\newcommand*{\nto}{\nrightarrow}
\newcommand*{\p}{\partial}
%embed
\newcommand*{\embed}{\hookrightarrow}
%surjection
\newcommand*{\surj}{\twoheadrightarrow}
%power set
\newcommand*{\powerset}{\mathcal{P}}

%matrix
\newcommand*{\matrixring}{\mathcal{M}}

%groups
\newcommand*{\normal}{\trianglelefteq}
%rings
\newcommand*{\ideal}{\trianglelefteq}

%fields
\renewcommand*{\C}{{\mathbb{C}}}
\newcommand*{\R}{{\mathbb{R}}}
\newcommand*{\Q}{{\mathbb{Q}}}
\newcommand*{\Z}{{\mathbb{Z}}}
\newcommand*{\N}{{\mathbb{N}}}
\newcommand*{\F}{{\mathbb{F}}}
%not really but I think this belongs here
\newcommand*{\A}{{\mathbb{A}}}

%asymptotic
\newcommand*{\bigO}{O}
\newcommand*{\smallo}{o}

%probability
\newcommand*{\prob}{\mathbb{P}}
\newcommand*{\E}{\mathbb{E}}

%vector calculus
\newcommand*{\gradient}{\V \nabla}
\newcommand*{\divergence}{\gradient \cdot}
\newcommand*{\curl}{\gradient \cdot}

%logic
\newcommand*{\yields}{\vdash}
\newcommand*{\nyields}{\nvdash}

%differential geometry
\renewcommand*{\H}{\mathbb{H}}
\newcommand*{\transversal}{\pitchfork}
\renewcommand{\d}{\mathrm{d}} % exterior derivative

%number theory
\newcommand*{\legendre}[2]{\genfrac{(}{)}{}{}{#1}{#2}}%Legendre symbol

%algebraic geometry
\DeclareMathOperator{\Spec}{Spec}
\DeclareMathOperator{\Proj}{Proj}

\DeclareMathOperator{\Frac}{Frac}
\DeclareMathOperator{\Div}{Div}

\renewcommand*{\O}{\mathcal{O}}
\DeclareMathOperator{\Cl}{Cl} % ideal class group

\begin{document}

\begin{titlepage}
  \begin{center}
    \includegraphics[width=0.6\textwidth]{logo.jpg}\par
    \vspace{1cm}
    {\scshape\huge Mathamatics Tripos \par}
    \vspace{2cm}
    {\huge Part \npart \par}
    \vspace{0.6cm}
    {\Huge \bfseries \ntitle \par}
    \vspace{1.2cm}
    {\Large\nterm, \nyear \par}
    \vspace{2cm}
    
    {\large \emph{Lectures by } \par}
    \vspace{0.2cm}
    {\Large \scshape \nlecturer}
    
    \vspace{0.5cm}
    {\large \emph{Notes by }\par}
    \vspace{0.2cm}
    {\Large \scshape \href{mailto:\nauthoremail}{\nauthor}}
 \end{center}
\end{titlepage}

\tableofcontents

\setcounter{section}{-1}

\section{Introduction}

In IID Number Fields, we studied finite extensions of \(\Q\) and their rings of integers. We proved two fundamental theorem for \(\O_K\):
\begin{itemize}
\item finiteness of ideal class group,
\item finite generation of \(\O_K^\times\).
\end{itemize}

In this course, we'll study
\begin{itemize}
\item completion at a prime,
\item Galois theory of local and global fields.
\end{itemize}
and finally we'll describe class field theory (description only).

\section{Dedekind domains}

\begin{definition}[discrete valuation ring]\index{discrete valuation ring}
  Let \(A\) be a ring. We say \(A\) is a \emph{discrete valuation ring} (DVR) if \(A\) is a principal ideal domain (PID) and \(A\) has a unique non-zero prime ideal.
\end{definition}

Let \(A\) be a DVR. Then the unique non-zero prime ideal \(\mathfrak m_A\) of \(A\) is also maximal, so \(A\) is also a local ring, i.e.\ \(A\) has a unique maximal ideal. Hence \(k_A = A/\mathfrak m_A\) is a field, the residue field of \(A\).

As \(A\) is a PID, \(\mathfrak m_A = (\pi)\) is principal. Any generator \(\pi\) is called a \emph{uniformiser}\index{uniformiser}. If \(\pi, \pi'\) are uniformisers then \((\pi) = (\pi')\) so \(\pi' = \pi u\) for some \(u \in A^\times\).

Since \(A\) is a local, \(A\) can be written as the disjoint union
\begin{align*}
  A &= A^\times \cup \mathfrak m_A \\
    &= A^\times \cup \pi A \\
    &= A^\times \cup \pi A^\times \cup \pi^2 A \\
    &= \bigcup_{i \geq 0} \pi^i A^\times \cup \bigcap_{i \geq 0} \pi^i A
\end{align*}

In fact, the ideal \(I = \bigcap_{i \geq 0} \pi^i A\) is zero. This follows from
\begin{lemma}[Nakayama's lemma]\index{Nakayama's lemma}
  Let \(R\) be a local ring, \(P \subseteq R\) the unique maximal ideal, \(M\) a finitely generated (fg) \(R\)-module. Then
  \begin{enumerate}
  \item if \(M = P M\) then \(M = 0\). This is equivalent to \(M/PM = 0\).
  \item if \(N \leq M\) is an \(R\)-submodule such that \(N + PM = M\) then \(N = M\). This is saying there is a surjection \(N \surj M/PM\).
  \end{enumerate}
\end{lemma}

\begin{proof}\leavevmode
  \begin{enumerate}
  \item Let \(a_1, \dots a_g\) be a generating set for \(M\) with \(g\) as small as possible, \(g \geq 1\). Then \(a_1 \in M = PM\) so we can write
    \[
      a = \sum_{i = 1}^g x_i a_i
    \]
    where \(x_i \in P\). Hence
    \[
      (1 - x_1) a_1 = \sum_{i = 2}^g x_i a_i.
    \]
    Since \(R\) is local, \(1 - x_1 \in R^\times\) so \(a_1 \in \generation{a_2, \dots, a_g}\), contradicting the minimality of \(g\).
  \item Apply first part to \(M/N\).
  \end{enumerate}
\end{proof}

Now back to the statement. Note \(\pi I = I\) so Nakayama's lemma implies that \(I = 0\). Hence each element of \(x \in A, x \neq 0\) admits a aunique description \(x = \pi^n u\), \(n \geq 0, u \in A^\times\). Each non-zero ideal of \(A\) has the form \((\pi^i)\) for some \(i \geq 0\).

Therefore we can define a function \(v: K^\times \to \Z\) where \(K = \Frac A\) with the following properties:
\begin{enumerate}
\item \(v\) is a surjective homomorphism,
\item for all \(x, y \in K^\times\) such that \(x + y \neq 0\), \(v(x + y) \geq \min(v(x), v(y))\), with equality if \(v(x) \neq v(y)\).
\end{enumerate}
We define \(v(x) = n\) when \(x = \pi^n u\) for some \(n \in \Z, u \in A^\times\).

\begin{proof}\leavevmode
  \begin{enumerate}
  \item \(\pi^n u \cdot \pi^m v = \pi^{n + m} uv\).
  \item wlog \(x = \pi^a u, y = \pi^{a + b} v\) where \(a \in \Z, b \geq 0\). Then
    \[
      x + y = \pi^a(u + v \pi^ b).
    \]
    If \(b > 0\) then \(u + v \pi^b \in A^\times\).
  \end{enumerate}
\end{proof}

\begin{definition}[valuation]\index{valuation}
  If \(L\) is a field, we call a function \(w: L^\times \to \Z\) a \emph{valuation} if satisfies 1, 2 above.
\end{definition}

Thus if we have a DVR then we have a valuation. The converse also holds: if \(w: L^\times \to \Z\) is a valuation, we define
\begin{align*}
  A_L &= \{x \in L^\times: w(x) \geq 0\} \cup \{0\} \\
  \mathfrak m_L &= \{x \in L^\times: w(x) > 0\} \cup \{0\}
\end{align*}

\begin{lemma}
  If \(k\) is a field, then there is a bijection between
  \begin{enumerate}
  \item subrings \(A \leq K\) such that \(A\) is a DVR and \(\Frac A = K\),
  \item valuations \(v: K^\times \to \Z\).
  \end{enumerate}
\end{lemma}

\begin{proof}
  Exercise.
\end{proof}

\begin{eg}\leavevmode
  \begin{enumerate}
  \item Let \(p\) be a prime, \(v: \Q^\times \to \Z\) defined by
    \[
      v(p^n \frac{r}{s}) = n
    \]
    if \(r, s \in \Z\), \((p, rs) = 1\).
  \item Let \(K\) be the field of meromorphic functions on \(\C\), \(v: K^\times \to \Z\) defined by
    \[
      v(f) = \operatorname{ord}_{z = 0} f(z).
    \]
  \end{enumerate}
\end{eg}

We will see via localisation we can reduce problems to DVR. Hence we need a way to recognise DVR. This is the content of the next proposition

\begin{proposition}
  Let \(A\) be a Noetherian domain. Then TFAE:
  \begin{enumerate}
  \item \(A\) is a DVR.
  \item \(A\) is integrally closed in \(\Frac A\) and \(A\) has a unique non-zero prime ideal.
  \end{enumerate}
\end{proposition}

Recall that \(A\) is integrally closed if for all \(\gamma \in K, a_1, \dots, a_n \in A\), if
there is a relation
\[
  \gamma^n + a_1 \gamma^{n - 1} + \dots + a_n = 0
\]
then \(\gamma \in A\). Equivalently, for all \(\gamma \in K\), \(A[\gamma]\) is fg as an \(A\)-module then \(\gamma \in A\).

\begin{proof}\leavevmode
  \begin{itemize}
  \item \(2 \implies 1\): suppose \(\gamma \in K - A\) and there exist \(a_1, \dots, a_n \in A\) such that
    \[
      \gamma^n + a_1 \gamma^{n - 1} + \dots + a_n = 0.
    \]
    We can write \(\gamma = \pi^{-k} u\) for some \(k > 0, u \in A^\times\). Hence
    \[
      -\pi^{-nk} u^n = a_1 \pi^{-(n - 1)k} u^{n - 1} + \dots + a_n.
    \]
    The valuation of LHS is \(-nk\) and the valuation of RHS is at least
    \[
      \min_{i = 1}^n v(a_i \pi^{- (n - i) k}) \geq \min v(\pi^{-(n - i)k}) = \min -(n - i) k \geq -(n - 1) k.
    \]
    These two expressions must be equal, absurd. Thus \(A\) is integrally closed in \(K\). \(A\) has a unique non-zero prime ideal as \(A\) is a DVR.
  \item \(2 \implies 1\): Let \(\mathfrak m \subseteq A\) be the unique non-zero prime ideal. Claim that for any proper non-zero ideal \(I \subseteq A\), there exists \(n \geq 1\) such that \(\mathfrak m^n \subseteq I \subseteq \mathfrak m\).

    \begin{proof}
      \(I \subseteq \mathfrak m\) as \(\mathfrak m\) is the unique maximal ideal. Suppose for contradiction exists \(I\) such that \(\mathfrak m^n \nsubseteq I\) for all \(n \geq 1\). Since \(A\) is Noetherian, we can assume that \(I\) is maximal with this property. Note \(I\) is not prime as otherwise \(I = \mathfrak m\). This means that there exist \(a, b \in A\) such that \(a, b \notin I\) but \(ab \in I\). Then the inclusions \(I \subseteq I + (a), I \subseteq I + (b)\) are proper. By maximality of \(I\), there exists \(n_1, n_2 \geq 1\) such that
      \begin{align*}
        \mathfrak m^{n_1} &\subseteq I + (a) \\
        \mathfrak m^{n_2} &\subseteq I + (b)
      \end{align*}
      Then
      \begin{align*}
        \mathfrak m^{n_1 + n_2} &\subseteq (I + (a)) (I + (b)) \\
                                &\subseteq I + (ab) \\
                                &\subseteq I
      \end{align*}
      as \(ab \in I\). Absurd.
    \end{proof}
    Now we can show \(\mathfrak m\) is principal. Choose \(\alpha \in \mathfrak m - \{0\}\). If \(\mathfrak m = (\alpha)\) then done. Otherwise, choose \(n \geq 2\) minimal such that \(\mathfrak m^n \subseteq (\alpha) \subseteq \mathfrak m\). Then \(\mathfrak m^{n - 1} \nsubseteq (\alpha)\) so exists \(\beta \in \mathfrak m^{n - 1} - (\alpha)\) such that
    \[
      \gamma = \frac{\beta}{\alpha} \in \frac{1}{\alpha} \mathfrak m^{n - 1} - A.
    \]
    Then
    \[
      \gamma \mathfrak m = \frac{\beta}{\alpha} \mathfrak m \subseteq \frac{1}{\alpha} \mathfrak m^{n - 1} \mathfrak m \subseteq \frac{1}{\alpha} \mathfrak m^n \subseteq A.
    \]
    If \(\gamma \mathfrak m \subseteq \mathfrak m\) then \(A[\gamma] \embed \End_A(\mathfrak m)\) as \(A\)-modules. \(\End_A(\mathfrak m)\) is a fg \(A\)-module as \(A\) is Noetherian. So \(A\) integrally closed in \(K\) implies that \(\gamma \in A\). So we must have \(\gamma \mathfrak m = A\). Hence \(\mathfrak m = \gamma^{-1} A\). So \(\pi = \gamma^{-1} \in A\) and \(\pi\) generates \(\mathfrak m\).

    Since \(A\) is a local ring, we have
    \begin{align*}
      A &= A^\times \cup \mathfrak m \\
        &= A^\times \cup \pi A \\
        &= \bigcup_{i \geq 0} \pi^i A^\times \cup I
    \end{align*}
    where \(I = \bigcap_{i \geq 0} \pi^i A\). \(I = 0\) as \(I\) is fg (as \(A\) is Noetherian) and \(\pi I = I\), so we can apply Nakayama's lemma. Hence
    \[
      A = \{0\} \cup \bigcup_{i \geq 0} \pi^iA^\times
    \]
    and \(A\) is a DVR.
  \end{itemize}
\end{proof}

\begin{definition}[multiplicative subset]\index{multiplicative subset}
  Let \(A\) be a ring. A \emph{multiplicative subset} of \(A\) is a subset \(S \subseteq A\) satisfying
  \begin{enumerate}
  \item \(1 \in S\),
  \item for all \(x, y \in S\), \(xy \in S\).
  \end{enumerate}
\end{definition}

\begin{definition}[localisation of ring]\index{localisation}
  Let \(S \subseteq A\) be a multiplicative subset. We define \(S^{-1} A\) to be the set of equivalence classes of pairs \((a, s) \in A \times S\) under the relation \((a, x) \sim (a', s')\) if there exists \(t \in S\) such that \(t(s'a - sa') = 0\).

  We write \(\frac{a}{s} \in S^{-1} A\) for the equivalence class of \((a, s)\).
\end{definition}

\begin{lemma}\leavevmode
  \begin{enumerate}
  \item \(S^{-1}A\) is well-defined and admits a ring structure.
  \item There is a ring homomorphism
    \begin{align*}
      A &\to S^{-1} A \\
      a &\mapsto \frac{a}{1}
    \end{align*}
    with kernel \(\{a \in A: \text{ exists } s \in S, sa = 0\}\).
  \item If \(A\) is a domain and \(0 \notin S\) then \(S^{-1}A\) may be identified with the subring \(\{\frac{a}{s}: a \in A, s \in S\}\) of \(\Frac A = (A - \{0\})^{-1} A\).
  \end{enumerate}
\end{lemma}

\begin{proof}\leavevmode
  \begin{enumerate}
  \item \(\sim\) is an equivalence relation: it is reflexive and symmetric by definition. For transitivity, suppose \((a, s) \sim (a', s') \sim (a'', s'')\), then exist \(t, t' \in S\) such that \(tas' = ta's, t'a's'' = t'a''s'\). Then
    \[
      tt's'as'' = tt's''a's = tt'a''s's
    \]
    i.e.
    \[
      tt's'(as'' - a''s) = 0.
    \]

    To make \(S^{-1}A\) a ring, the zero element is \(\frac{0}{1}\), the multiplicative identity is \(\frac{1}{1}\), and addition and multiplication are defined as
    \begin{align*}
      \frac{a}{s} + \frac{a'}{s'} &= \frac{as' + a's}{ss'} \\
      \frac{a}{s} \cdot \frac{a'}{s'} &= \frac{aa'}{ss'}
    \end{align*}
    Check the ring axioms are satisfied.
  \item \(f: A \to S^{-1}A\) is a ring homomorphism by definition.
    \[
      \ker f = \{a \in A: \frac{a}{1} = \frac{0}{1}\} = \{a \in A: \text{ exists } s \in S \text{ such that } sa = 0\}.
    \]
  \item Now we suppose \(A\) is a domain. Recall that
    \[
      \Frac A = \{(a, s) \in A \times (A - \{0\})\} / \bullet
    \]
    where \((a, s) \bullet (a', s')\) if \(as' = a's\). We need to check that if \(S \subseteq A\) if a multiplicative subset with \(0 \notin S\) then \((a, s) \sim (a', s')\) implies \((a, s) \bullet (a', s')\).
  \end{enumerate}
\end{proof}

\begin{definition}[localisation of module]\index{localisation}
  Let \(S \subseteq A\) to be a multiplicative subset and let \(M\) be an \(A\)-module. Then we define \(S^{-1}M\) to be the set of equivalence classes in \(M \times S\) for the relation \((m, s) \sim (m', s')\) if there exists \(t \in S\) such that \(t(ms - m's) = 0\).

  We write \(\frac{m}{s}\) for the equivalence class of \((m, s)\).
\end{definition}

\begin{ex}\leavevmode
  \begin{enumerate}
  \item \(S^{-1}M\) is an \(S^{-1}A\)-module via
    \begin{align*}
      \frac{a}{s} \cdot \frac{m}{s'} &= \frac{am}{ss'} \\
      \frac{a}{s} + \frac{a'}{s'} &= \frac{as' + a's}{ss'}
    \end{align*}
  \item If \(f: M \to N\) is an \(A\)-module homomorphism then there is a homomorphism
    \begin{align*}
      S^{-1}f: S^{-1}M &\to S^{-1}N \\
      \frac{m}{s} &\mapsto \frac{f(m)}{s}
    \end{align*}
  \item \(S^{-1}\) is a functor from the category of \(A\)-modules to the category of \(S^{-1}A\)-modules.
  \end{enumerate}
\end{ex}

\begin{lemma}
  Let
  \[
    \begin{tikzcd}
      M' \ar[r, "f"] & M \ar[r, "f'"] & M''
    \end{tikzcd}
  \]
  be an exact sequence of \(A\)-modules. Then
  \[
    \begin{tikzcd}
      S^{-1}M' \ar[r, "S^{-1}f"] & S^{-1}M \ar[r, "S^{-1}f'"] & S^{-1}M''
    \end{tikzcd}
  \]
  is also exact.
\end{lemma}

\begin{proof}
  \(f' \compose f = 0\) so \(S^{-1}f' \compose S^{-1} f = 0\). For the other inclusion, let \(\frac{m}{s} \in \ker S^{-1} f'\), i.e.\ \(\frac{f'(m)}{s} = 0\), i.e.\ there exists \(s' \in S\) such that \(0 = s'f'(m) = f'(s'm)\) so \(s'm \in \ker f' = \im f\). Hence there exists \(m' \in M'\) such that \(f(m') = s'm\). Then
  \[
    S^{-1}f (\frac{m'}{ss'}) = \frac{f(m')}{ss'} = \frac{s'm}{ss'} = \frac{m}{s}.
  \]
\end{proof}

\begin{corollary}
  If \(f\) is surjective (injective, resp) then so is \(S^{-1} f\).
\end{corollary}

Let \(I \subseteq A\) be an ideal. Then \(I \embed A\) is an injective homomorphism of \(A\)-modules. Hence \(S^{-1}I \embed S^{-1}A\) is an injective homomorphism of \(S^{-1}A\)-module. Hence \(S^{-1}I\) may be identified with an ideal of \(S^{-1}A\). It's the ideal
\[
  S^{-1}A \cdot I = \{\frac{x}{s}: x \in I, s \in S\} \subseteq S^{-1}A.
\]

\begin{proposition}
  Let \(S \subseteq A\) be a multiplicative subset. Then there is a bijection between the following two sets:
  \begin{enumerate}
  \item prime ideals \(P \subseteq A\) such that \(P \cap S = \emptyset\),
  \item prime ideal \(Q \subseteq S^{-1}A\),
  \end{enumerate}
  given by \(P \mapsto S^{-1}P, Q \mapsto f^{-1}(Q)\) where \(f: A \to S^{-1}A\) is the localisation map.
\end{proposition}

\begin{proof}
  Check the maps are well-defined: if \(1 \in S^{-1}P\) then \(\frac{1}{s} = \frac{x}{s}\) for some \(x \in P, s \in S\) so exists \(t \in S\) such that \(t(s - x) = 0\). Then \(ts = tx \in P\) so \(t \in P\) or \(s \in P\).

  If \(\frac{a}{s}, \frac{a'}{s'} \in S^{-1}A\) and \(\frac{aa'}{ss'} \in S^{-1}P\) then \(\frac{aa'}{ss'} = \frac{x}{t}\). Hence exists \(t' \in S\) such that
  \[
    tt' aa' = t'ss'x \in P.
  \]
  Since \(P\) is prime, \(aa' = P\) so \(a \in P\) or \(a' \in P\). Hence \(\frac{a}{s}\) or \(\frac{a'}{s'} \in S^{-1}P\). Thus \(S^{-1}P\) is prime.

  If \(Q \subseteq S^{-1}A\) then \(S^{-1}A/Q\) is a non-zero domain. Then \(A/f^{-1}(Q) \embed S^{-1}A/Q\) so \(A/f^{-1}(Q)\) is also a non-zero domain so \(f^{-1}(Q)\) is a prime ideal (this in fact follows from the fact that pullback of any prime ideal is prime).

  It is left as an exercise to check the maps in the statement of the proposition are mutually inverse bijections.
\end{proof}

\begin{corollary}
  Let \(A\) be a ring and \(P \subseteq A\) a prime ideal. Then
  \begin{enumerate}
  \item \(S = A - P\) is a multiplicative subset of \(A\).
  \item \(S^{-1}A\) is a local ring with unique maximal ideal \(S^{-1} P\).
  \end{enumerate}
  We usually write \(A_P\) for \((A - P)^{-1}A\).
\end{corollary}

For example \(\Z_{(p)} = (\Z - p\Z)^{-1}\Z\).

\begin{proposition}
  Let \(A\) be a Noetherian domain. Then TFAE:
  \begin{enumerate}
  \item For every non-zero prime ideal \(P \subseteq A\), \(A_P\) is a DVR.
  \item \(A\) is integrally closed in \(K = \Frac A\) and every non-zero prime ideal is maximal.
  \end{enumerate}
\end{proposition}

Consequently, for any \(P\) there is the valuation \(v_P: K^\times \to \Z\) associated to \(A_P\).

\begin{definition}[Dedekind domain]\index{Dedekind domain}
  Any ring satisfying the conditions is called a \emph{Dedekind domain}.
\end{definition}

\begin{proof}\leavevmode
  \begin{itemize}
  \item \(1 \implies 2\): wlog we can assume \(A\) does have non-zero prime ideals. Suppose given a relation
    \[
      a^n + a_1 a^{n - 1} + \dots + a_n = 0
    \]
    where \(a \in A, a_i \in A\). Then \(A_P\) is a DVR implies that \(A_P\) is integrally closed in \(K\) so \(a \in A_P\) for all \(P\). Therefore for all \(P\) we can find \(x_P \in A, s_P \in A - P\) such that \(a = \frac{x_P}{s_P}\) in \(K\). In particular \(s_Pa \in A\).

    The ideal
    \[
      I = (s_P: P \subseteq A \text{ non-zero prime ideal})
    \]
    is the unit ideal, since it is not contained in any maximal ideal of \(A\). Therefore there exists element \(t_P \in A\), with only finitely many non-zero, such that
    \[
      1 = \sum_P t_P s_P.
    \]
    Then
    \[
      a = \sum_P t_P s_Pa \in A.
    \]

    Let \(P \subseteq Q\) be non-zero prime ideals of \(A\), with \(Q\) maximal. Then \(PA_Q \subseteq QA_Q\) are non-zero prime ideals of \(A_Q\), a DVR. Hence \(PA_Q = QA_Q\) and \(P = Q\).
  \item \(2 \implies 1\): Again we can assume that \(A\) has a non-zero prime ideal \(P\). We must show \(A_P\) is a DVR, or equivalently that \(A_P\) is integrally closed in \(K\) and has a unique non-zero prime ideal.

    Suppose given a relation
    \[
      \left( \frac{a}{s} \right)^n + \frac{a_1}{s_1} \left( \frac{a}{s} \right)^{n - 1} + \dots + \frac{a_n}{s_n} = 0
    \]
    where \(a, a_1, \dots, a_n \in A, s_1, \dots, s_n \in A - P, s \in A - \{0\}\). Multiply through by \((s_1 \cdots s_n)^n\),
    \[
      \left( \frac{as_1 \cdots s_n}{s} \right)^n + \left( \frac{as_1 \cdots s_n}{s} \right)^{n - 1} a_1 s_2 \cdots s_n + \dots + s_1s_2 \cdots a_n = 0.
    \]
    As \(A\) is integrally closed, \(\frac{as_1 \cdots s_n}{s} \in A\) so
    \[
      \frac{a}{s} = \frac{as_1 \cdots s_n}{s} \cdot \frac{1}{s_1 \cdots s_n} \in A_P
    \]
    so \(A_P\) is integrally closed.

    Let \(Q \subseteq A_P\) be a non-zero prime ideal. Then eixsts \(Q' \subseteq P\) such that \(Q'A_P = Q\). By assumption we must have \(Q' = P\) and hence \(Q = PA_P\).
  \end{itemize}
\end{proof}

\begin{definition}[fractional ideal]\index{fractional ideal}
  Let \(A\) be a domain, \(K = \Frac A\). A \emph{fractional ideal} of \(A\) is a fg \(A\)-submodule of \(K\).
\end{definition}

If \(I, J \subseteq K\) are fractional ideals then
\begin{align*}
  I + J &= \{x + y: x \in I, y \in J\} \\
  IJ &= \{xy: x \in I, y \in J\} \\
\end{align*}
are also fractional ideals. On the other hand,
\[
  (I: J) = \{x \in K: xJ \subseteq I\}
\]
is an \(A\)-submodule of \(K\) but is in general not fg.

\begin{lemma}
  Let \(A\) be a Noetherian domain, \(S \subseteq A\) a multiplicative subset. Then
  \begin{enumerate}
  \item if \(I, J\) are fractional ideals then \(S^{-1}I\) is a fractional ideal of \(S^{-1}A\) and
    \begin{align*}
      S^{-1}(I + J) &= S^{-1}I + S^{-1}J \\
      S^{-1}(IJ) &= S^{-1}I \cdot S^{-1}J
    \end{align*}
  \item if \(I, J\) are fractional ideals of \(A\) and \(J\) is non-zero then \((I : J)\) is a fractional ideal of \(A\) and
    \[
      S^{-1}(I : J) = (S^{-1}I : S^{-1}J).
    \]
  \end{enumerate}
\end{lemma}

\begin{proof}\leavevmode
  \begin{enumerate}
  \item Exercise.
  \item If \(a \in A - \{0\}\) then
    \[
      (I : (a)) = \{x \in K: x(a) \subseteq I\} = \{x \in K: xa \in I\} = a^{-1}I.
    \]
    In partuclar \((I : (a))\) is fg and hence a fractional ideal.

    In general, write \(J = (a_1, \dots, a_n)\) where \(a_i \in K^\times\). Then
    \[
      (I : J) = \{x \in K: \text{for all } i, xa_i \in I\} = \bigcap_{i = 1}^n a_i^{-1}I.
    \]
    In paricular \((I : (a)) \subseteq (I : (a_1))\). Since \(A\) is Noetherian, any submodule of \((I : (a_1))\) is fg and hence \((I : J)\) is a fractional ideal.

    To show \(S^{-1}(I : J) = (S^{-1}I : S^{-1}J)\), we have
    \begin{align*}
      \text{LHS} &= S^{-1} \bigcap_{i = 1}^n a_i^{-1} I \\
      \text{RHS} &= \bigcap_{i = 1}^n a_i^{-1} S^{-1} I = \bigcap_{i = 1}^n S^{-1} (a_i^{-1} I)
    \end{align*}
    so in fact it's enough to show that if \(I, J \subseteq K\) are fractional ideals of \(A\) then
    \[
      S^{-1}(I \cap J) = (S^{-1}I) \cap (S^{-1}J).
    \]
    We certainly have \(S^{-1}(I \cap J) \subseteq (S^{-1}I) \cap (S^{-1}J)\). Suppose \(\frac{x}{s} = \frac{y}{t}\) where \(x \in I, y \in J, s, t \in S\) then \(xt = sy \in I \cap J\) and
    \[
      \frac{x}{s} = \frac{xt}{st} \in S^{-1}(I \cap J).
    \]
  \end{enumerate}
\end{proof}

\begin{proposition}
  Let \(A\) be a Dedekind domain and let \(\Div A\) be the set of non-zero fractional ideals of \(A\). Then \(\Div A\) forms a group under the multiplication of fractional ideals.
\end{proposition}

\begin{proof}
  \(A = (1)\) is a multiplicative identity. Must show that for any non-zero fractional ideal \(I\),
  \[
    I (A : I) = A.
  \]
  Observe that if \(P \subseteq A\) is a non-zero prime ideal then \(IA_P = (\pi_P^i)\) as \(A_P\) is a DVR, so \((A_P: IA_P) = (\pi_P^{-i})\) so
  \[
    \text{LHS}_P = IA_P (A_P: IA_P) = A_P = \text{RHS}_P
  \]
  so it is enough to show that if \(I, J\) are fractional ideals of \(A\) such that \(IA_P = JA_P\) for all \(P\) then \(I = J\). In fact, we are going to show if \(IA_P \subseteq JA_P\) then \(I \subseteq J\). Suppose \(IA_P \subseteq JA_P\) for any non-zero prime ideal \(P \subseteq A\). Let \(x \in I\). Then \(x \in IA_P \subseteq JA_P\) so we can write \(x = \frac{y_P}{s_P}\) where \(y_P \in J, s_P \in A - P\). Now we use
  \[
    (s_P: P \subseteq A \text{ non-zero prime ideal}) = (1)
  \]
  to write \(1 = \sum_P s_P t_P\) where \(t_P \in A\) and only finitely many are non-zero. Then
  \[
    x = \sum_P t_Ps_P x \in J.
  \]
\end{proof}

Observe that for any \(P\), there is a homomorphism
\begin{align*}
  \Div A &\to \Div A_P \\
  I &\mapsto IA_P
\end{align*}
But \(\Div A_P \cong \Z\) canonically as every non-zero fractional ideal of \(A_P\) has the form \((\pi^i)\) for some \(i \in \Z\).

We can define a homomorphism \(v_P: \Div A \to \Z\) by \(v_P(I) = v_P(x)\) where \(IA_P = (x)\). In particular, for any \(x \in K^\times\), \(v_P((x)) = v_P(x)\). Note that \(v_P\) is surjective since \(P A_P = (\pi)\) as \(v_P(P) = 1\) for any \(P\). Taking the product over all non-zero prime ideals, we get a homomorphism
\[
  \prod_P v_P: \Div A \to \prod_P \Z.
\]
This is injective as we showed that for any \(I, J \in \Div A\), \(I = J\) if and only if for all \(P\), \(I A_P = J A_P\). Now we characterise the image.

\begin{lemma}
  For any \(I \in \Div A\), the set \(\{P: v_P(I) \neq 0\}\) is finite. 
\end{lemma}

In other words, \(\prod_P v_P\) takes values in \(\bigoplus_P \Z \subseteq \prod_P \Z\).

\begin{proof}
  Suppose \(I = (\frac{a_1}{b_1}, \dots, \frac{a_n}{b_n})\) where \(a_i \in A, b_i \in A - \{0\}\). Let \(b = b_1 \cdots b_n\). Then \(J = bI\) is an ideal of \(A\) and
  \[
    v_P(I) = v_P(J) - v_P((b)).
  \]
  So it's enough to prove the lemma in the case \(I \subseteq A\) is an ideal.

  Let \(\alpha \in I - \{0\}\). Then \((\alpha) \subseteq I\) and for any non-zero prime ideal \(P \subseteq A\), \(v_P(\alpha) \geq v_P(I) \geq 0\) so in fact we can assume \(I = (\alpha)\) is principal.

  Now we observe that \(v_P(\alpha) > 0\) if and only if \(\alpha \in P A_P\) if and only if \(\alpha \in P\). So it's enough to show that there are only finitely many \(P\)'s such that \(\alpha \in P\).

  Suppose for contradiction there are infinitely many \(P_1, P_2, \dots\) such that \(\alpha \in P_i\). Define \(J_i = P_1 \cap \cdots \cap P_i\). Then
  \[
    J_1 \supseteq J_2 \supseteq \cdots \supseteq (\alpha)
  \]
  so
  \[
    \alpha J_1^{-1} \subseteq \alpha J_2^{-1} \subseteq \cdots \subseteq A.
  \]
  This is an ascending chain of ideals of \(A\) so there exists \(n \geq 1\) such that \(\alpha J_n^{-1} = \alpha J_{n + 1}^{-1}\). Hence \(J_n = J_{n + 1}\), i.e.
  \[
    P_1 \cap \cdots \cap P_n = P_1 \cap \cdots \cap P_n \cap P_{n + 1}.
  \]
  Choose \(x_i \in P_i - P_{n + 1}\) for \(i \leq n\). Then
  \[
    x_1 \cdots x_n \in P_1 \cap \cdots \cap P_n = P_1 \cap \cdots \cap P_n \cap P_{n + 1}.
  \]
  Since \(P_{n + 1}\) is prime, we have \(x_i \in P_{n + 1}\) for some \(i \leq n\). Absurd.
\end{proof}

\begin{proposition}\leavevmode
  \begin{enumerate}
  \item \(\prod_P v_P: \Div A \to \bigoplus_P \Z\) is an isomorphism.
  \item For any \(I \in \Div A\),
    \[
      I = \prod_I P^{v_P(I)}
    \]
    so we have unique factorisation of fractional ideals.
  \end{enumerate}
\end{proposition}

\begin{proof}\leavevmode
  \begin{enumerate}
  \item It's enough to show
    \[
      v_Q(P) =
      \begin{cases}
        1 & P = Q \\
        0 & \text{otherwise}
      \end{cases}
    \]
    as then \((\delta_{PQ})_Q\) are in the ring and they generate \(\bigoplus_P \Z\). \(v_P(P) = 1\) by definition as \(P A_P\) is the maximal ideal of \(A_P\). If \(Q \neq P\) then we must show \(P A_Q = A_Q\): if \(P A_Q \neq A_Q\) then \(P A_Q \subseteq Q A_Q\) so \(P \subseteq Q\). This is impossible as both \(P\) and \(Q\) are maximal ideals.
  \item \(I = \prod_P P^{v_P(I)}\) if and only if for all \(Q\),
    \[
      v_Q(I) = v_Q(\prod_P P^{v_P(I)}).
    \]
    As \(v_Q(P) = \delta_{PQ}\), RHS equals to \(v_Q(I)\).
  \end{enumerate}
\end{proof}

\section{Complete DVRs}

\begin{definition}[inverse system, inverse limit]\index{inverse limit}
  Suppose given groups \(A_i\) and homomorphisms \(f_i: A_{i + 1} \to A_i\) for all \(i \geq 1\)
  \[
    \begin{tikzcd}
      A_1 & A_2 \ar[l, "f_1"'] & A_3 \ar[l, "f_2"'] & \cdots \ar[l, "f_3"']
    \end{tikzcd}
  \]
  we call such a collection an \emph{inverse system}. Its \emph{inverse limit} is
  \[
    \varprojlim_i A_i = \{(a_i) \in \prod_{i = 1}^\infty A_i: f_i(a_{i + 1}) = a_i \text{ for all } i \geq 1\} \subseteq \prod_{i = 1}^\infty A_i.
  \]
\end{definition}

This is a group. If the \(A_i\)'s (\(f_i\)'s respectively) are abelian groups/rings (homomorphisms/ring homomorphisms) then so is \(\varprojlim_i A_i\).

Suppose \(A\) is a DVR with uniformiser \(\pi\). Then we can make an inverse system
\[
  \begin{tikzcd}
    A/(\pi^1) & A/(\pi^2) \ar[l] & A/(\pi^3) \ar[l] & \cdots \ar[l]
  \end{tikzcd}
\]
with maps the natural quotient maps. There's a homomorphism \(A \to A/(\pi^i)\), hence \(A \to \prod_{i = 1}^\infty A/(\pi^i)\) which takes values in \(\varprojlim_i A/(\pi^i)\).

\begin{definition}[complete]\index{complete DVR}
  We say \(A\) is \emph{complete} if the homomorphism
  \[
    A \to \varprojlim_i A/(\pi^i)
  \]
  is an isomorphism.
\end{definition}

The kernel of this homomorphism is \(\bigcap_{i \geq 1} (\pi^i) = 0\) so \(A\) is complete if and only if the map is surjective.

\begin{lemma}
  TFAE:
  \begin{enumerate}
  \item \(A\) is complete.
  \item \(A\) is complete as a metric space with respect to the metric
    \[
      d(x, y) =
      \begin{cases}
        0 & x = y \\
        2^{-v(x - y)} & x \neq y
      \end{cases}
    \]
  \item \(K\) is complete as a metric space with respect to the metric given by the formula.
  \end{enumerate}
\end{lemma}

\begin{proof}
  We first explain why \(d\) is a metric. \(d\) satisfies the ultrametric triangle inequality
  \[
    d(x, y) \leq \max(d(x, y), d(y, z))
  \]
  for all \(x, y, z\). This is because we can assume \(x, y, z\). Then this is equivalent to
  \[
    v(x - z) \geq \min(v(x - y), v(y - z))
  \]
  but LHS is equal to \(v((x - y) + (y - z))\). This is the axiom defining a valuation.
  \begin{itemize}
  \item \(1 \implies 2\): Let \((a_n)_{n \geq 1}\) be a Cauchy sequence in \(A\), meaning that for all \(\varepsilon > 0\), exists \(N\) such that for all \(n, m \geq N\), \(d(a_n, a_m) < \varepsilon\). Equivalently, for all \(M > 0\), exists \(N(M)\) such that for all \(n, m \geq N(M)\),
    \[
      a_n = a_m \pmod{\pi^M}.
    \]
    We can define \(b = (b_i)_{i \geq 1} \in \prod_{i = 1}^\infty A/(\pi^i)\) by
    \[
      b_n = a_{N(n)} \pmod {\pi^n}
    \]
    By definition of Cauchy sequence, \(b \in \varprojlim_i A/(\pi^i)\). Since \(A\) is a complete DVR, exists \(a \in A\) such that \(a = a_{N(n)} \pmod{\pi^n}\) for all \(n \geq 1\). Hence \(v(a - a_{N(n)}) \geq n\), i.e.\ \(d(a, a_{N(n)}) \leq 2^{-n}\) so \(\lim_{n \to \infty} a_n = a\).
  \item \(2 \implies 1\): Suppose given \((a_n)_{n \geq 1} \in \varprojlim_n A/(\pi^n)\). Let \(\tilde a_n \in A\) be any element such that \(\tilde a_n \pmod{\pi^n} = a_n\). Then for all \(m \geq n\), \(\tilde a_m = \tilde a_n \pmod{\pi^n}\) by definition of inverse limit, i.e.\ \(d(\tilde a_m, \tilde a_n) \leq 2^{-n}\), so \((\tilde a_n)_{n \geq 1}\) is a Cauchy sequence in \(A\). So there exists \(a \in A\) such that \(\tilde a_n \to a\) in \(A\), i.e.\ for all \(M \geq 1\) exists \(N(M)\) such that for all \(n \geq N(M)\), \(\tilde a_n = a \pmod{\pi^M}\). Hence \(a\) is a preimage of \((a_n)_{n \geq 1}\) under the map \(A \to \varprojlim_i A/(\pi^i)\).
  \end{itemize}
\end{proof}


\printindex
\end{document}
