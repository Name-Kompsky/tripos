\documentclass[a4paper]{article}

\def\npart{III}

\def\ntitle{Algebraic Number Theory}
\def\nlecturer{J.\ A.\ Thorne}

\def\nterm{Michaelmas}
\def\nyear{2019}

\ifx \nauthor\undefined
  \def\nauthor{Qiangru Kuang}
\else
\fi

\ifx \ntitle\undefined
  \def\ntitle{Template}
\else
\fi

\ifx \nauthoremail\undefined
  \def\nauthoremail{qk206@cam.ac.uk}
\else
\fi

\ifx \ndate\undefined
  \def\ndate{\today}
\else
\fi

\title{\ntitle}
\author{\nauthor}
\date{\ndate}

%\usepackage{microtype}
\usepackage{mathtools}
\usepackage{amsthm}
\usepackage{stmaryrd}%symbols used so far: \mapsfrom
\usepackage{empheq}
\usepackage{amssymb}
\let\mathbbalt\mathbb
\let\pitchforkold\pitchfork
\usepackage{unicode-math}
\let\mathbb\mathbbalt%reset to original \mathbb
\let\pitchfork\pitchforkold

\usepackage{imakeidx}
\makeindex[intoc]

%to address the problem that Latin modern doesn't have unicode support for setminus
%https://tex.stackexchange.com/a/55205/26707
\AtBeginDocument{\renewcommand*{\setminus}{\mathbin{\backslash}}}
\AtBeginDocument{\renewcommand*{\models}{\vDash}}%for \vDash is same size as \vdash but orginal \models is larger
\AtBeginDocument{\let\Re\relax}
\AtBeginDocument{\let\Im\relax}
\AtBeginDocument{\DeclareMathOperator{\Re}{Re}}
\AtBeginDocument{\DeclareMathOperator{\Im}{Im}}
\AtBeginDocument{\let\div\relax}
\AtBeginDocument{\DeclareMathOperator{\div}{div}}

\usepackage{tikz}
\usetikzlibrary{automata,positioning}
\usepackage{pgfplots}
%some preset styles
\pgfplotsset{compat=1.15}
\pgfplotsset{centre/.append style={axis x line=middle, axis y line=middle, xlabel={$x$}, ylabel={$y$}, axis equal}}
\usepackage{tikz-cd}
\usepackage{graphicx}
\usepackage{newunicodechar}

\usepackage{fancyhdr}

\fancypagestyle{mypagestyle}{
    \fancyhf{}
    \lhead{\emph{\nouppercase{\leftmark}}}
    \rhead{}
    \cfoot{\thepage}
}
\pagestyle{mypagestyle}

\usepackage{titlesec}
\newcommand{\sectionbreak}{\clearpage} % clear page after each section
\usepackage[perpage]{footmisc}
\usepackage{blindtext}

%\reallywidehat
%https://tex.stackexchange.com/a/101136/26707
\usepackage{scalerel,stackengine}
\stackMath
\newcommand\reallywidehat[1]{%
\savestack{\tmpbox}{\stretchto{%
  \scaleto{%
    \scalerel*[\widthof{\ensuremath{#1}}]{\kern-.6pt\bigwedge\kern-.6pt}%
    {\rule[-\textheight/2]{1ex}{\textheight}}%WIDTH-LIMITED BIG WEDGE
  }{\textheight}% 
}{0.5ex}}%
\stackon[1pt]{#1}{\tmpbox}%
}

%\usepackage{braket}
\usepackage{thmtools}%restate theorem
\usepackage{hyperref}

% https://en.wikibooks.org/wiki/LaTeX/Hyperlinks
\hypersetup{
    %bookmarks=true,
    unicode=true,
    pdftitle={\ntitle},
    pdfauthor={\nauthor},
    pdfsubject={Mathematics},
    pdfcreator={\nauthor},
    pdfproducer={\nauthor},
    pdfkeywords={math maths \ntitle},
    colorlinks=true,
    linkcolor={red!50!black},
    citecolor={blue!50!black},
    urlcolor={blue!80!black}
}

\usepackage{cleveref}



% TODO: mdframed often gives bad breaks that cause empty lines. Would like to switch to tcolorbox.
% The current workaround is to set innerbottommargin=0pt.

%\usepackage[theorems]{tcolorbox}





\usepackage[framemethod=tikz]{mdframed}
\mdfdefinestyle{leftbar}{
  %nobreak=true, %dirty hack
  linewidth=1.5pt,
  linecolor=gray,
  hidealllines=true,
  leftline=true,
  leftmargin=0pt,
  innerleftmargin=5pt,
  innerrightmargin=10pt,
  innertopmargin=-5pt,
  % innerbottommargin=5pt, % original
  innerbottommargin=0pt, % temporary hack 
}
%\newmdtheoremenv[style=leftbar]{theorem}{Theorem}[section]
%\newmdtheoremenv[style=leftbar]{proposition}[theorem]{proposition}
%\newmdtheoremenv[style=leftbar]{lemma}[theorem]{Lemma}
%\newmdtheoremenv[style=leftbar]{corollary}[theorem]{corollary}

\newtheorem{theorem}{Theorem}[section]
\newtheorem{proposition}[theorem]{Proposition}
\newtheorem{lemma}[theorem]{Lemma}
\newtheorem{corollary}[theorem]{Corollary}
\newtheorem{axiom}[theorem]{Axiom}
\newtheorem*{axiom*}{Axiom}

\surroundwithmdframed[style=leftbar]{theorem}
\surroundwithmdframed[style=leftbar]{proposition}
\surroundwithmdframed[style=leftbar]{lemma}
\surroundwithmdframed[style=leftbar]{corollary}
\surroundwithmdframed[style=leftbar]{axiom}
\surroundwithmdframed[style=leftbar]{axiom*}

\theoremstyle{definition}

\newtheorem*{definition}{Definition}
\surroundwithmdframed[style=leftbar]{definition}

\newtheorem*{slogan}{Slogan}
\newtheorem*{eg}{Example}
\newtheorem*{ex}{Exercise}
\newtheorem*{remark}{Remark}
\newtheorem*{notation}{Notation}
\newtheorem*{convention}{Convention}
\newtheorem*{assumption}{Assumption}
\newtheorem*{question}{Question}
\newtheorem*{answer}{Answer}
\newtheorem*{note}{Note}
\newtheorem*{application}{Application}

%operator macros

%basic
\DeclareMathOperator{\lcm}{lcm}

%matrix
\DeclareMathOperator{\tr}{tr}
\DeclareMathOperator{\Tr}{Tr}
\DeclareMathOperator{\adj}{adj}

%algebra
\DeclareMathOperator{\Hom}{Hom}
\DeclareMathOperator{\End}{End}
\DeclareMathOperator{\id}{id}
\DeclareMathOperator{\im}{im}
\DeclareMathOperator{\coker}{coker}
\DeclarePairedDelimiter{\generation}{\langle}{\rangle}

%groups
\DeclareMathOperator{\sym}{Sym}
\DeclareMathOperator{\sgn}{sgn}
\DeclareMathOperator{\inn}{Inn}
\DeclareMathOperator{\aut}{Aut}
\DeclareMathOperator{\GL}{GL}
\DeclareMathOperator{\SL}{SL}
\DeclareMathOperator{\PGL}{PGL}
\DeclareMathOperator{\PSL}{PSL}
\DeclareMathOperator{\SU}{SU}
\DeclareMathOperator{\UU}{U}
\DeclareMathOperator{\SO}{SO}
\DeclareMathOperator{\OO}{O}
\DeclareMathOperator{\PSU}{PSU}
\DeclareMathOperator{\Sp}{Sp}


%hyperbolic
\DeclareMathOperator{\sech}{sech}

%field, galois heory
\DeclareMathOperator{\ch}{ch}
\DeclareMathOperator{\gal}{Gal}
\DeclareMathOperator{\emb}{Emb}



%ceiling and floor
%https://tex.stackexchange.com/a/118217/26707
\DeclarePairedDelimiter\ceil{\lceil}{\rceil}
\DeclarePairedDelimiter\floor{\lfloor}{\rfloor}


\DeclarePairedDelimiter{\innerproduct}{\langle}{\rangle}

%\DeclarePairedDelimiterX{\norm}[1]{\lVert}{\rVert}{#1}
\DeclarePairedDelimiter{\norm}{\lVert}{\rVert}



%Dirac notation
%TODO: rewrite for variable number of arguments
\DeclarePairedDelimiterX{\braket}[2]{\langle}{\rangle}{#1 \delimsize\vert #2}
\DeclarePairedDelimiterX{\braketthree}[3]{\langle}{\rangle}{#1 \delimsize\vert #2 \delimsize\vert #3}

\DeclarePairedDelimiter{\bra}{\langle}{\rvert}
\DeclarePairedDelimiter{\ket}{\lvert}{\rangle}




%macros

%general

%divide, not divide
\newcommand*{\divides}{\mid}
\newcommand*{\ndivides}{\nmid}
%vector, i.e. mathbf
%https://tex.stackexchange.com/a/45746/26707
\newcommand*{\V}[1]{{\ensuremath{\symbf{#1}}}}
%closure
\newcommand*{\cl}[1]{\overline{#1}}
%conjugate
\newcommand*{\conj}[1]{\overline{#1}}
%set complement
\newcommand*{\stcomp}[1]{\overline{#1}}
\newcommand*{\compose}{\circ}
\newcommand*{\nto}{\nrightarrow}
\newcommand*{\p}{\partial}
%embed
\newcommand*{\embed}{\hookrightarrow}
%surjection
\newcommand*{\surj}{\twoheadrightarrow}
%power set
\newcommand*{\powerset}{\mathcal{P}}

%matrix
\newcommand*{\matrixring}{\mathcal{M}}

%groups
\newcommand*{\normal}{\trianglelefteq}
%rings
\newcommand*{\ideal}{\trianglelefteq}

%fields
\renewcommand*{\C}{{\mathbb{C}}}
\newcommand*{\R}{{\mathbb{R}}}
\newcommand*{\Q}{{\mathbb{Q}}}
\newcommand*{\Z}{{\mathbb{Z}}}
\newcommand*{\N}{{\mathbb{N}}}
\newcommand*{\F}{{\mathbb{F}}}
%not really but I think this belongs here
\newcommand*{\A}{{\mathbb{A}}}

%asymptotic
\newcommand*{\bigO}{O}
\newcommand*{\smallo}{o}

%probability
\newcommand*{\prob}{\mathbb{P}}
\newcommand*{\E}{\mathbb{E}}

%vector calculus
\newcommand*{\gradient}{\V \nabla}
\newcommand*{\divergence}{\gradient \cdot}
\newcommand*{\curl}{\gradient \cdot}

%logic
\newcommand*{\yields}{\vdash}
\newcommand*{\nyields}{\nvdash}

%differential geometry
\renewcommand*{\H}{\mathbb{H}}
\newcommand*{\transversal}{\pitchfork}
\renewcommand{\d}{\mathrm{d}} % exterior derivative

%number theory
\newcommand*{\legendre}[2]{\genfrac{(}{)}{}{}{#1}{#2}}%Legendre symbol

%algebraic geometry
\DeclareMathOperator{\Spec}{Spec}
\DeclareMathOperator{\Proj}{Proj}

\theoremstyle{definition}
\newtheorem*{observation}{Observation}

\DeclareMathOperator{\Frac}{Frac}
\renewcommand*{\O}{\mathcal{O}}
\DeclareMathOperator{\n}{N}
\DeclareMathOperator{\disc}{disc}
\DeclareMathOperator{\Cl}{Cl} % ideal class group
\newcommand*{\red}[1]{\overline{#1}} % reduction mod
\DeclareMathOperator{\vol}{vol}


\begin{document}

\begin{titlepage}
  \begin{center}
    \includegraphics[width=0.6\textwidth]{logo.jpg}\par
    \vspace{1cm}
    {\scshape\huge Mathamatics Tripos \par}
    \vspace{2cm}
    {\huge Part \npart \par}
    \vspace{0.6cm}
    {\Huge \bfseries \ntitle \par}
    \vspace{1.2cm}
    {\Large\nterm, \nyear \par}
    \vspace{2cm}
    
    {\large \emph{Lectures by } \par}
    \vspace{0.2cm}
    {\Large \scshape \nlecturer}
    
    \vspace{0.5cm}
    {\large \emph{Notes by }\par}
    \vspace{0.2cm}
    {\Large \scshape \href{mailto:\nauthoremail}{\nauthor}}
 \end{center}
\end{titlepage}

\tableofcontents

\setcounter{section}{-1}

\section{Introduction}

In IID Number Fields, we studied finite extensions of \(\Q\) and their rings of integers. We proved two fundamental theorem for \(\O_K\):
\begin{itemize}
\item finiteness of ideal class group,
\item finite generation of \(\O_K^\times\).
\end{itemize}

In this course, we'll study
\begin{itemize}
\item completion at a prime,
\item Galois theory of local and global fields.
\end{itemize}
and finally we'll describe class field theory (description only).

\section{Dedekind domains}

\begin{definition}[discrete valuation ring]\index{discrete valuation ring}
  Let \(A\) be a ring. We say \(A\) is a \emph{discrete valuation ring} (DVR) if \(A\) is a principal ideal domain (PID) and \(A\) has a unique non-zero prime ideal.
\end{definition}

Let \(A\) be a DVR. Then the unique non-zero prime ideal \(\mathfrak m_A\) of \(A\) is also maximal, so \(A\) is also a local ring, i.e.\ \(A\) has a unique maximal ideal. Hence \(k_A = A/\mathfrak m_A\) is a field, the residue field of \(A\).

As \(A\) is a PID, \(\mathfrak m_A = (\pi)\) is principal. Any generator \(\pi\) is called a \emph{uniformiser}\index{uniformiser}. If \(\pi, \pi'\) are uniformisers then \((\pi) = (\pi')\) so \(\pi' = \pi u\) for some \(u \in A^\times\).

Since \(A\) is a local, \(A\) can be written as the disjoint union
\begin{align*}
  A &= A^\times \cup \mathfrak m_A \\
    &= A^\times \cup \pi A \\
    &= A^\times \cup \pi A^\times \cup \pi^2 A \\
    &= \bigcup_{i \geq 0} \pi^i A^\times \cup \bigcap_{i \geq 0} \pi^i A
\end{align*}

In fact, the ideal \(I = \bigcap_{i \geq 0} \pi^i A\) is zero. This follows from
\begin{lemma}[Nakayama's lemma]\index{Nakayama's lemma}
  Let \(R\) be a local ring, \(P \subseteq R\) the unique maximal ideal, \(M\) a finitely generated (fg) \(R\)-module. Then
  \begin{enumerate}
  \item if \(M = P M\) then \(M = 0\). This is equivalent to \(M/PM = 0\).
  \item if \(N \leq M\) is an \(R\)-submodule such that \(N + PM = M\) then \(N = M\). This is saying there is a surjection \(N \surj M/PM\).
  \end{enumerate}
\end{lemma}

\begin{proof}\leavevmode
  \begin{enumerate}
  \item Let \(a_1, \dots a_g\) be a generating set for \(M\) with \(g\) as small as possible, \(g \geq 1\). Then \(a_1 \in M = PM\) so we can write
    \[
      a = \sum_{i = 1}^g x_i a_i
    \]
    where \(x_i \in P\). Hence
    \[
      (1 - x_1) a_1 = \sum_{i = 2}^g x_i a_i.
    \]
    Since \(R\) is local, \(1 - x_1 \in R^\times\) so \(a_1 \in \generation{a_2, \dots, a_g}\), contradicting the minimality of \(g\).
  \item Apply first part to \(M/N\).
  \end{enumerate}
\end{proof}

Now back to the statement. Note \(\pi I = I\) so Nakayama's lemma implies that \(I = 0\). Hence each element of \(x \in A, x \neq 0\) admits a aunique description \(x = \pi^n u\), \(n \geq 0, u \in A^\times\). Each non-zero ideal of \(A\) has the form \((\pi^i)\) for some \(i \geq 0\).

Therefore we can define a function \(v: K^\times \to \Z\) where \(K = \Frac A\) with the following properties:
\begin{enumerate}
\item \(v\) is a surjective homomorphism,
\item for all \(x, y \in K^\times\) such that \(x + y \neq 0\), \(v(x + y) \geq \min(v(x), v(y))\), with equality if \(v(x) \neq v(y)\).
\end{enumerate}
We define \(v(x) = n\) when \(x = \pi^n u\) for some \(n \in \Z, u \in A^\times\).

\begin{proof}\leavevmode
  \begin{enumerate}
  \item \(\pi^n u \cdot \pi^m \cdot v = \pi^{n + m} uv\).
  \item wlog \(x = \pi^a u, y = \pi^{a + b} v\) where \(a \in \Z, b \geq 0\). Then
    \[
      x + y = \pi^a(u + v \pi^ b).
    \]
    If \(b > 0\) then \(u + v \pi^b \in A^\times\).
  \end{enumerate}
\end{proof}

\begin{definition}[valuation]\index{valuation}
  If \(L\) is a field, we call a function \(w: L^\times \to \Z\) a \emph{valuation} if satisfies 1, 2 above.
\end{definition}

Thus if we have a DVR then we have a valuation. The converse also holds: if \(w: L^\times \to \Z\) is a valuation, we define
\begin{align*}
  A_L &= \{x \in L^\times: w(x) \geq 0\} \cup \{0\} \\
  \mathfrak m_L &= \{x \in L^\times: w(x) > 0\} \cup \{0\}
\end{align*}

\begin{lemma}
  If \(k\) is a field, then there is a bijection between
  \begin{enumerate}
  \item subrings \(A \leq K\) such that \(A\) is a DVR and \(\Frac A = K\),
  \item valuations \(v: K^\times \to \Z\).
  \end{enumerate}
\end{lemma}

\begin{proof}
  Exercise.
\end{proof}

\begin{eg}\leavevmode
  \begin{enumerate}
  \item let \(p\) be a prime, \(v: \Q^\times \to \Z\) defined by
    \[
      v(p^n \frac{r}{s}) = n
    \]
    if \(r, s \in \Z\), \((p, rs) = 1\).
  \item Let \(K\) be the field of meromorphic functions on \(\C\), \(v: K^\times \to \Z\) defined by
    \[
      v(f) = \operatorname{ord}_{z = 0} f(z).
    \]
  \end{enumerate}
\end{eg}

We will see via localisation we can reduce problems to DVR. Hence we need a way to recognise DVR. This is the content of the next proposition

\begin{proposition}
  Let \(A\) be a Noetherian domain. Then TFAE:
  \begin{enumerate}
  \item \(A\) is a DVR.
  \item \(A\) is integrally closed in \(\Frac A\) and \(A\) has a unique non-zero prime ideal.
  \end{enumerate}
\end{proposition}

Recall that \(A\) is integrally closed if for all \(\gamma \in K, a_1, \dots, a_n \in A\), if
there is a relation
\[
  \gamma^n + a_1 \gamma^{n - 1} + \dots + a_n = 0
\]
then \(\gamma \in A\). Equivalently, for all \(\gamma \in K\), \(A[\gamma]\) is fg as an \(A\)-module then \(\gamma \in A\).

\begin{proof}\leavevmode
  \begin{itemize}
  \item \(2 \implies 1\): suppose \(\gamma \in K - A\) and there exist \(a_1, \dots a_n \in A\) such that
    \[
      \gamma^n + a_1 \gamma^{n - 1} + \dots + a_n = 0.
    \]
    We can write \(\gamma = \pi^{-k} u\) for some \(k > 0, u \in A^\times\). Hence
    \[
      -\pi^{-nk} u^n = a_1 \pi^{-(n - 1)k} u^{n - 1} + \dots + a_n.
    \]
    The valuation of LHS is \(-nk\) and the valuation of RHS is at least
    \[
      \min_{i = 1}^n v(a_i \pi^{- (n - i) k}) \geq \min v(\pi^{-(n - i)k}) = \min -(n - i) k \geq -(n - 1) k.
    \]
    These two expressions must be equal, absurd. Thus \(A\) is integrally closed in \(K\). \(A\) has a unique non-zero prime ideal as \(A\) is a DVR.
  \item \(2 \implies 1\): Let \(\mathfrak m \subseteq A\) be the unique non-zero prime ideal. Claim that for any proper non-zero ideal \(I \subseteq A\), there exists \(n \geq 1\) such that \(\mathfrak m^n \subseteq I \subseteq \mathfrak m\).

    \begin{proof}
      \(I \subseteq \mathfrak m\) as \(\mathfrak m\) is the unique maximal ideal. Suppose for contradiction exists \(I\) such that \(\mathfrak m^n \nsubseteq I\) for all \(n \geq 1\). Since \(A\) is Noetherian, we can assume that \(I\) is maximal with this property. Note \(I\) is not prime as otherwise \(I = \mathfrak m\). This means that there exist \(a, b \in A\) such that \(a, b \notin I\) but \(ab \in I\). Then the inclusions \(I \subseteq I + (a), I \subseteq I + (b)\) are proper. By maximality of \(I\), there exists \(n_1, n_2 \geq 1\) such that
      \begin{align*}
        \mathfrak m^{n_1} &\subseteq I + (a) \\
        \mathfrak m^{n_2} &\subseteq I + (b)
      \end{align*}
      Then
      \begin{align*}
        \mathfrak m^{n_1 + n_2} &\subseteq (I + (a)) (I + (b)) \\
                                &\subseteq I + (ab) \\
                                &\subseteq I
      \end{align*}
      as \(ab \in I\). Absurd.
    \end{proof}
    Now we can show \(\mathfrak m\) is principal. Choose \(\alpha \in \mathfrak m - \{0\}\). If \(\mathfrak m = (\alpha)\) then done. Otherwise, choose \(n \geq 2\) minimal such that \(\mathfrak m^n \subseteq (\alpha) \subseteq \mathfrak m\). Then \(\mathfrak m^{n - 1} \nsubseteq (\alpha)\) so exists \(\beta \in \mathfrak m^{n - 1} - (\alpha)\) such that
    \[
      \gamma = \frac{\beta}{\alpha} \in \frac{1}{\alpha} \mathfrak m^{n - 1} - A.
    \]
    Then
    \[
      \gamma \mathfrak m = \frac{\beta}{\alpha} \mathfrak m \subseteq \frac{1}{\alpha} \mathfrak m^{n - 1} \mathfrak m \subseteq \frac{1}{\alpha} \mathfrak m^n \subseteq A.
    \]
    If \(\gamma \mathfrak m \subseteq \mathfrak m\) then \(A[\gamma] \embed \End_A(\mathfrak m)\) as \(A\)-modules. \(\End_A(\mathfrak m)\) is a fg \(A\)-module as \(A\) is Noetherian. So \(A\) integrally closed in \(K\) implies that \(\gamma \in A\). So we must have \(\gamma \mathfrak m = A\). Hence \(\mathfrak m = \gamma^{-1} A\). So \(\pi = \gamma^{-1} \in A\) and \(\pi\) generates \(\mathfrak m\).
  \end{itemize}
\end{proof}


\printindex
\end{document}
