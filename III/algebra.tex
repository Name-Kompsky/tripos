\documentclass[a4paper]{article}

\def\npart{III}

\def\ntitle{Algebra}
\def\nlecturer{C.\ Brookes}

\def\nterm{Michaelmas}
\def\nyear{2018}

\ifx \nauthor\undefined
  \def\nauthor{Qiangru Kuang}
\else
\fi

\ifx \ntitle\undefined
  \def\ntitle{Template}
\else
\fi

\ifx \nauthoremail\undefined
  \def\nauthoremail{qk206@cam.ac.uk}
\else
\fi

\ifx \ndate\undefined
  \def\ndate{\today}
\else
\fi

\title{\ntitle}
\author{\nauthor}
\date{\ndate}

%\usepackage{microtype}
\usepackage{mathtools}
\usepackage{amsthm}
\usepackage{stmaryrd}%symbols used so far: \mapsfrom
\usepackage{empheq}
\usepackage{amssymb}
\let\mathbbalt\mathbb
\let\pitchforkold\pitchfork
\usepackage{unicode-math}
\let\mathbb\mathbbalt%reset to original \mathbb
\let\pitchfork\pitchforkold

\usepackage{imakeidx}
\makeindex[intoc]

%to address the problem that Latin modern doesn't have unicode support for setminus
%https://tex.stackexchange.com/a/55205/26707
\AtBeginDocument{\renewcommand*{\setminus}{\mathbin{\backslash}}}
\AtBeginDocument{\renewcommand*{\models}{\vDash}}%for \vDash is same size as \vdash but orginal \models is larger
\AtBeginDocument{\let\Re\relax}
\AtBeginDocument{\let\Im\relax}
\AtBeginDocument{\DeclareMathOperator{\Re}{Re}}
\AtBeginDocument{\DeclareMathOperator{\Im}{Im}}
\AtBeginDocument{\let\div\relax}
\AtBeginDocument{\DeclareMathOperator{\div}{div}}

\usepackage{tikz}
\usetikzlibrary{automata,positioning}
\usepackage{pgfplots}
%some preset styles
\pgfplotsset{compat=1.15}
\pgfplotsset{centre/.append style={axis x line=middle, axis y line=middle, xlabel={$x$}, ylabel={$y$}, axis equal}}
\usepackage{tikz-cd}
\usepackage{graphicx}
\usepackage{newunicodechar}

\usepackage{fancyhdr}

\fancypagestyle{mypagestyle}{
    \fancyhf{}
    \lhead{\emph{\nouppercase{\leftmark}}}
    \rhead{}
    \cfoot{\thepage}
}
\pagestyle{mypagestyle}

\usepackage{titlesec}
\newcommand{\sectionbreak}{\clearpage} % clear page after each section
\usepackage[perpage]{footmisc}
\usepackage{blindtext}

%\reallywidehat
%https://tex.stackexchange.com/a/101136/26707
\usepackage{scalerel,stackengine}
\stackMath
\newcommand\reallywidehat[1]{%
\savestack{\tmpbox}{\stretchto{%
  \scaleto{%
    \scalerel*[\widthof{\ensuremath{#1}}]{\kern-.6pt\bigwedge\kern-.6pt}%
    {\rule[-\textheight/2]{1ex}{\textheight}}%WIDTH-LIMITED BIG WEDGE
  }{\textheight}% 
}{0.5ex}}%
\stackon[1pt]{#1}{\tmpbox}%
}

%\usepackage{braket}
\usepackage{thmtools}%restate theorem
\usepackage{hyperref}

% https://en.wikibooks.org/wiki/LaTeX/Hyperlinks
\hypersetup{
    %bookmarks=true,
    unicode=true,
    pdftitle={\ntitle},
    pdfauthor={\nauthor},
    pdfsubject={Mathematics},
    pdfcreator={\nauthor},
    pdfproducer={\nauthor},
    pdfkeywords={math maths \ntitle},
    colorlinks=true,
    linkcolor={red!50!black},
    citecolor={blue!50!black},
    urlcolor={blue!80!black}
}

\usepackage{cleveref}



% TODO: mdframed often gives bad breaks that cause empty lines. Would like to switch to tcolorbox.
% The current workaround is to set innerbottommargin=0pt.

%\usepackage[theorems]{tcolorbox}





\usepackage[framemethod=tikz]{mdframed}
\mdfdefinestyle{leftbar}{
  %nobreak=true, %dirty hack
  linewidth=1.5pt,
  linecolor=gray,
  hidealllines=true,
  leftline=true,
  leftmargin=0pt,
  innerleftmargin=5pt,
  innerrightmargin=10pt,
  innertopmargin=-5pt,
  % innerbottommargin=5pt, % original
  innerbottommargin=0pt, % temporary hack 
}
%\newmdtheoremenv[style=leftbar]{theorem}{Theorem}[section]
%\newmdtheoremenv[style=leftbar]{proposition}[theorem]{proposition}
%\newmdtheoremenv[style=leftbar]{lemma}[theorem]{Lemma}
%\newmdtheoremenv[style=leftbar]{corollary}[theorem]{corollary}

\newtheorem{theorem}{Theorem}[section]
\newtheorem{proposition}[theorem]{Proposition}
\newtheorem{lemma}[theorem]{Lemma}
\newtheorem{corollary}[theorem]{Corollary}
\newtheorem{axiom}[theorem]{Axiom}
\newtheorem*{axiom*}{Axiom}

\surroundwithmdframed[style=leftbar]{theorem}
\surroundwithmdframed[style=leftbar]{proposition}
\surroundwithmdframed[style=leftbar]{lemma}
\surroundwithmdframed[style=leftbar]{corollary}
\surroundwithmdframed[style=leftbar]{axiom}
\surroundwithmdframed[style=leftbar]{axiom*}

\theoremstyle{definition}

\newtheorem*{definition}{Definition}
\surroundwithmdframed[style=leftbar]{definition}

\newtheorem*{slogan}{Slogan}
\newtheorem*{eg}{Example}
\newtheorem*{ex}{Exercise}
\newtheorem*{remark}{Remark}
\newtheorem*{notation}{Notation}
\newtheorem*{convention}{Convention}
\newtheorem*{assumption}{Assumption}
\newtheorem*{question}{Question}
\newtheorem*{answer}{Answer}
\newtheorem*{note}{Note}
\newtheorem*{application}{Application}

%operator macros

%basic
\DeclareMathOperator{\lcm}{lcm}

%matrix
\DeclareMathOperator{\tr}{tr}
\DeclareMathOperator{\Tr}{Tr}
\DeclareMathOperator{\adj}{adj}

%algebra
\DeclareMathOperator{\Hom}{Hom}
\DeclareMathOperator{\End}{End}
\DeclareMathOperator{\id}{id}
\DeclareMathOperator{\im}{im}
\DeclareMathOperator{\coker}{coker}
\DeclarePairedDelimiter{\generation}{\langle}{\rangle}

%groups
\DeclareMathOperator{\sym}{Sym}
\DeclareMathOperator{\sgn}{sgn}
\DeclareMathOperator{\inn}{Inn}
\DeclareMathOperator{\aut}{Aut}
\DeclareMathOperator{\GL}{GL}
\DeclareMathOperator{\SL}{SL}
\DeclareMathOperator{\PGL}{PGL}
\DeclareMathOperator{\PSL}{PSL}
\DeclareMathOperator{\SU}{SU}
\DeclareMathOperator{\UU}{U}
\DeclareMathOperator{\SO}{SO}
\DeclareMathOperator{\OO}{O}
\DeclareMathOperator{\PSU}{PSU}
\DeclareMathOperator{\Sp}{Sp}


%hyperbolic
\DeclareMathOperator{\sech}{sech}

%field, galois heory
\DeclareMathOperator{\ch}{ch}
\DeclareMathOperator{\gal}{Gal}
\DeclareMathOperator{\emb}{Emb}



%ceiling and floor
%https://tex.stackexchange.com/a/118217/26707
\DeclarePairedDelimiter\ceil{\lceil}{\rceil}
\DeclarePairedDelimiter\floor{\lfloor}{\rfloor}


\DeclarePairedDelimiter{\innerproduct}{\langle}{\rangle}

%\DeclarePairedDelimiterX{\norm}[1]{\lVert}{\rVert}{#1}
\DeclarePairedDelimiter{\norm}{\lVert}{\rVert}



%Dirac notation
%TODO: rewrite for variable number of arguments
\DeclarePairedDelimiterX{\braket}[2]{\langle}{\rangle}{#1 \delimsize\vert #2}
\DeclarePairedDelimiterX{\braketthree}[3]{\langle}{\rangle}{#1 \delimsize\vert #2 \delimsize\vert #3}

\DeclarePairedDelimiter{\bra}{\langle}{\rvert}
\DeclarePairedDelimiter{\ket}{\lvert}{\rangle}




%macros

%general

%divide, not divide
\newcommand*{\divides}{\mid}
\newcommand*{\ndivides}{\nmid}
%vector, i.e. mathbf
%https://tex.stackexchange.com/a/45746/26707
\newcommand*{\V}[1]{{\ensuremath{\symbf{#1}}}}
%closure
\newcommand*{\cl}[1]{\overline{#1}}
%conjugate
\newcommand*{\conj}[1]{\overline{#1}}
%set complement
\newcommand*{\stcomp}[1]{\overline{#1}}
\newcommand*{\compose}{\circ}
\newcommand*{\nto}{\nrightarrow}
\newcommand*{\p}{\partial}
%embed
\newcommand*{\embed}{\hookrightarrow}
%surjection
\newcommand*{\surj}{\twoheadrightarrow}
%power set
\newcommand*{\powerset}{\mathcal{P}}

%matrix
\newcommand*{\matrixring}{\mathcal{M}}

%groups
\newcommand*{\normal}{\trianglelefteq}
%rings
\newcommand*{\ideal}{\trianglelefteq}

%fields
\renewcommand*{\C}{{\mathbb{C}}}
\newcommand*{\R}{{\mathbb{R}}}
\newcommand*{\Q}{{\mathbb{Q}}}
\newcommand*{\Z}{{\mathbb{Z}}}
\newcommand*{\N}{{\mathbb{N}}}
\newcommand*{\F}{{\mathbb{F}}}
%not really but I think this belongs here
\newcommand*{\A}{{\mathbb{A}}}

%asymptotic
\newcommand*{\bigO}{O}
\newcommand*{\smallo}{o}

%probability
\newcommand*{\prob}{\mathbb{P}}
\newcommand*{\E}{\mathbb{E}}

%vector calculus
\newcommand*{\gradient}{\V \nabla}
\newcommand*{\divergence}{\gradient \cdot}
\newcommand*{\curl}{\gradient \cdot}

%logic
\newcommand*{\yields}{\vdash}
\newcommand*{\nyields}{\nvdash}

%differential geometry
\renewcommand*{\H}{\mathbb{H}}
\newcommand*{\transversal}{\pitchfork}
\renewcommand{\d}{\mathrm{d}} % exterior derivative

%number theory
\newcommand*{\legendre}[2]{\genfrac{(}{)}{}{}{#1}{#2}}%Legendre symbol

%algebraic geometry
\DeclareMathOperator{\Spec}{Spec}
\DeclareMathOperator{\Proj}{Proj}

\DeclareMathOperator{\Tor}{Tor}
\DeclareMathOperator{\Ext}{Ext}
\DeclareMathOperator{\Der}{Der}
\DeclareMathOperator{\Innder}{Innder}

\begin{document}

\begin{titlepage}
  \begin{center}
    \includegraphics[width=0.6\textwidth]{logo.jpg}\par
    \vspace{1cm}
    {\scshape\huge Mathamatics Tripos \par}
    \vspace{2cm}
    {\huge Part \npart \par}
    \vspace{0.6cm}
    {\Huge \bfseries \ntitle \par}
    \vspace{1.2cm}
    {\Large\nterm, \nyear \par}
    \vspace{2cm}
    
    {\large \emph{Lectures by } \par}
    \vspace{0.2cm}
    {\Large \scshape \nlecturer}
    
    \vspace{0.5cm}
    {\large \emph{Notes by }\par}
    \vspace{0.2cm}
    {\Large \scshape \href{mailto:\nauthoremail}{\nauthor}}
 \end{center}
\end{titlepage}

\tableofcontents

\section{Hochschild homology}

\begin{eg}
  Examples of bimodules:
  \begin{enumerate}
  \item \(R\) itself is an \(R\)-\(R\) bimodule.
  \item \(R \otimes_k R\) is a bimodule, as so is \(R \otimes_k R \otimes_k \dots \otimes_k R\) \(n\)-times.
  \item There is a bimodule free presentation of \(R\)
    \[
      0 \to \ker \mu \to R \otimes R \xrightarrow{\mu} R \to 0
    \]
    where \(\mu\) is the multiplication \(r \otimes s \to rs\). \(R \otimes_k R^{\text{op}}\) is a free module --- as a \(R \otimes R\) module it is free of rank \(1\) gnerated by \(1 \otimes 1\).
  \end{enumerate}
\end{eg}

\begin{definition}[Hochschild (co)homology]\index{Hochschild homology}\index{Hochschild cohomology}
  Given an \(R\)-\(R\) bimodule \(M\), define \emph{Hochschild homology}
  \[
    HH_n(R, M) = \Tor_M^{R, R} (R, M)
  \]
  and \emph{Hochschild cohomology}
  \[
    HH^n(R, M) = \Ext_{R, R}^M(R, M).
  \]
\end{definition}

In particular
\begin{align*}
  HH^0(R, M) &= \Hom_{R, R} (R, M) = \{m \in M: rm = mr \text{ for all } r \in R\} \\
  HH^0(R, R) &= \{s \in R: rs = sr \text{ for all } r \in R\} = Z(R) \\
  HH_0(R, M) &= R \otimes_{\R \otimes \R^{\text{op}}} M \cong M / \langle rm - mr: m \in M, r \in R \rangle \\
  HH_0(R, R) &= R/[R, R]
\end{align*}
where \([r, s] = rs - sr\) is the \emph{Lie bracket} on \(R\) of example sheet 3 Q4 (correction of example sheet: need \(R\) semisimple).

\begin{definition}
  The Hochschild chain complex gives a free resolution for the bimodule \(R\)
  \[
    \begin{tikzcd}
      \cdots \ar[r] & R \otimes R \otimes R \otimes R \ar[r, "d_1"] & R \otimes R \otimes R \ar[r, "d_0"] & R \otimes R \ar[r, "\mu"] & R \ar[r] & 0
    \end{tikzcd}
  \]
  where
  \begin{align*}
    d_{n - 1}: R^{\otimes n + 2} & R^{\otimes n + 1} \\
    r_0 \otimes \dots \otimes r_{n + 1} &\mapsto \sum_{i = 0}^n (-1)^i r_0 \otimes \dots \otimes (r_ir_{i + 1}) \otimes \dots \otimes r_{n + 1}
  \end{align*}
\end{definition}

\begin{definition}[Hochschild cohomological dimension]\index{Hochschild cohomology!dimension}
  The (Hochschild cohomological) \emph{dimension} \(\dim R\) of \(R\) is
  \[
    \dim R = \sup \{n: HH^n(R, M) \neq 0 \text{ for some bimodule } M\}.
  \]
\end{definition}

\begin{definition}[separable]\index{separable}
  The \(k\)-algebras \(R\) of \(\dim R = 0\) are precisely  those where the bimodule \(R\) is projective. This is when \(R\) is a direct summand of \(R \otimes R\) --- there is a map \(\beta: R \to R \otimes R\) so that \(\mu \compose \beta = \id_R\). These algebras are called \emph{\(k\)-separable}.
\end{definition}

This generalises the notion of a separable field extension, but note that \(k\)-separable algebras must be finite-dimensional as \(k\)-vector spaces. See example sheet.

\begin{eg}\leavevmode
  \begin{enumerate}
  \item \(M_n(k)\), the matrix algebra over \(k\), is \(k\)-separable. Given the map \(\nu: R \to R \otimes R\), the image of \(1\) is called a separating idempotent. For \(M_n(k)\) a separating idempotent is obtained as follow: let \(E_{ij}\) be the elementary matrix which has \(1\) at \(ij\)th entry and \(0\) otherwise. Fix \(j\) and consider \(\sum_i E_{ij} \otimes E_{ji}\). This is a separating idempotent. Note that the image of the under \(\mu\) is the identity matrix.
  \item For \(G\) a finite group, \(\C G\) is \(\C\)-separable:
    \[
      \C G \otimes \C G^{\text{op}}
      \cong \C G \otimes \C G
      \cong \C( G \times G)
    \]
    which is semisimple and completely reducible. so all submodules are direct summands. In particular we get \(\C G\) is a direct summand of \(\C(G \times G)\). Thus \(\dim (\C G) = 0\).
  \item Now consider higher dimensions. Note that
    \[
      \Hom_{R \otimes R} (R - R, M) \cong \Hom_k (k, M).
    \]
    On LHS a map is determined by the image of \(1 \otimes 1\), and on RHS it is determined by \(1\). Moreover
    \[
      \Hom_{R \otimes R} (\underbrace{R \otimes \dots \otimes R}_{n + 2}, M) \cong \Hom_K  (\underbrace{R \otimes \dots \otimes R}_n, M).
    \]
  \end{enumerate}
\end{eg}

\begin{definition}
  The \emph{Hochschild cochain complex} is
  \[
    \begin{tikzcd}
      M \cong \Hom_k(k, M) \ar[r, "\delta_0"] & \Hom_k(R, M) \ar[r, "\delta_1"] & \Hom(R \otimes R, M) \ar[r] & \cdots
    \end{tikzcd}
  \]
  where
  \begin{align*}
    (\delta_0 f)(r) &= rf(1) - f(1) r, \quad f \in \Hom_k(k, M) \\
    (\delta_1 f)(r_1 \otimes r_2) &= r_1 f(r_2) - f(r_1r_2) + f(r_1)r_2, \quad f \in \Hom_k(R, M) \\
    (\delta_2 f)(r_1 \otimes r_2 \otimes r_3) &= r_1 f(r_2 \otimes r_3) - f(r_1r_2 \otimes r_3) + f(r_1 \otimes r_2r_3) - f(r_1 \otimes r_3) r_3 \\
  \end{align*}
  and so on.
\end{definition}

\begin{definition}[derivation, inner derivation]\index{derivation}\index{inner derivation}
  \[
    \ker \delta_1 = \{f \in \Hom_k(R, M): f(r_1r_2) = r_1f(r_2) + f(r_1)r_2\}
  \]
  is called the \emph{derivations} from \(R\) to \(M\) and is denoted \(\Der(R, M)\). On the other hand
  \[
    \im \delta_0 = \{f \in \Hom_k(R, M) \text{ of the form } r \mapsto rm - mr \text{ for some } m \in M\}
  \]
  is called the \emph{inner derivations} and is denoted \(\Innder(R, M)\).
\end{definition}

Thus
\[
  HH^1(R, M) = \frac{\Der(R, M)}{\Innder(R, M)}.
\]
If \(M = R\) we get
\[
  HH^1(R, R) = \frac{Der(R)}{\Innder(R)}.
\]
If \(R\) is commutative then \(\Innder R = 0\) and so \(HH^1(R, R) = \Der R\).

In general, \(\Der R\) fomrs a Lie algebra: if \(D_1, D_2\) are derivations \(R \to R\) then \(D_1D_2 - D_2D_1 \in \End_k R\) is also a derivation.

\begin{eg}
  \(R = k[x]\). Then
  \[
    \Der R = \{p(x) \frac{d}{dx}: p(x) \in k[x]\}.
  \]
\end{eg}





\printindex
\end{document}
