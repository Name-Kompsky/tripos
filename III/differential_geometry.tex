\documentclass[a4paper]{article}

\def\npart{III}

\def\ntitle{Differential Geometry}
\def\nlecturer{A.\ Kovalev}

\def\nterm{Michaelmas}
\def\nyear{2018}

\ifx \nauthor\undefined
  \def\nauthor{Qiangru Kuang}
\else
\fi

\ifx \ntitle\undefined
  \def\ntitle{Template}
\else
\fi

\ifx \nauthoremail\undefined
  \def\nauthoremail{qk206@cam.ac.uk}
\else
\fi

\ifx \ndate\undefined
  \def\ndate{\today}
\else
\fi

\title{\ntitle}
\author{\nauthor}
\date{\ndate}

%\usepackage{microtype}
\usepackage{mathtools}
\usepackage{amsthm}
\usepackage{stmaryrd}%symbols used so far: \mapsfrom
\usepackage{empheq}
\usepackage{amssymb}
\let\mathbbalt\mathbb
\let\pitchforkold\pitchfork
\usepackage{unicode-math}
\let\mathbb\mathbbalt%reset to original \mathbb
\let\pitchfork\pitchforkold

\usepackage{imakeidx}
\makeindex[intoc]

%to address the problem that Latin modern doesn't have unicode support for setminus
%https://tex.stackexchange.com/a/55205/26707
\AtBeginDocument{\renewcommand*{\setminus}{\mathbin{\backslash}}}
\AtBeginDocument{\renewcommand*{\models}{\vDash}}%for \vDash is same size as \vdash but orginal \models is larger
\AtBeginDocument{\let\Re\relax}
\AtBeginDocument{\let\Im\relax}
\AtBeginDocument{\DeclareMathOperator{\Re}{Re}}
\AtBeginDocument{\DeclareMathOperator{\Im}{Im}}
\AtBeginDocument{\let\div\relax}
\AtBeginDocument{\DeclareMathOperator{\div}{div}}

\usepackage{tikz}
\usetikzlibrary{automata,positioning}
\usepackage{pgfplots}
%some preset styles
\pgfplotsset{compat=1.15}
\pgfplotsset{centre/.append style={axis x line=middle, axis y line=middle, xlabel={$x$}, ylabel={$y$}, axis equal}}
\usepackage{tikz-cd}
\usepackage{graphicx}
\usepackage{newunicodechar}

\usepackage{fancyhdr}

\fancypagestyle{mypagestyle}{
    \fancyhf{}
    \lhead{\emph{\nouppercase{\leftmark}}}
    \rhead{}
    \cfoot{\thepage}
}
\pagestyle{mypagestyle}

\usepackage{titlesec}
\newcommand{\sectionbreak}{\clearpage} % clear page after each section
\usepackage[perpage]{footmisc}
\usepackage{blindtext}

%\reallywidehat
%https://tex.stackexchange.com/a/101136/26707
\usepackage{scalerel,stackengine}
\stackMath
\newcommand\reallywidehat[1]{%
\savestack{\tmpbox}{\stretchto{%
  \scaleto{%
    \scalerel*[\widthof{\ensuremath{#1}}]{\kern-.6pt\bigwedge\kern-.6pt}%
    {\rule[-\textheight/2]{1ex}{\textheight}}%WIDTH-LIMITED BIG WEDGE
  }{\textheight}% 
}{0.5ex}}%
\stackon[1pt]{#1}{\tmpbox}%
}

%\usepackage{braket}
\usepackage{thmtools}%restate theorem
\usepackage{hyperref}

% https://en.wikibooks.org/wiki/LaTeX/Hyperlinks
\hypersetup{
    %bookmarks=true,
    unicode=true,
    pdftitle={\ntitle},
    pdfauthor={\nauthor},
    pdfsubject={Mathematics},
    pdfcreator={\nauthor},
    pdfproducer={\nauthor},
    pdfkeywords={math maths \ntitle},
    colorlinks=true,
    linkcolor={red!50!black},
    citecolor={blue!50!black},
    urlcolor={blue!80!black}
}

\usepackage{cleveref}



% TODO: mdframed often gives bad breaks that cause empty lines. Would like to switch to tcolorbox.
% The current workaround is to set innerbottommargin=0pt.

%\usepackage[theorems]{tcolorbox}





\usepackage[framemethod=tikz]{mdframed}
\mdfdefinestyle{leftbar}{
  %nobreak=true, %dirty hack
  linewidth=1.5pt,
  linecolor=gray,
  hidealllines=true,
  leftline=true,
  leftmargin=0pt,
  innerleftmargin=5pt,
  innerrightmargin=10pt,
  innertopmargin=-5pt,
  % innerbottommargin=5pt, % original
  innerbottommargin=0pt, % temporary hack 
}
%\newmdtheoremenv[style=leftbar]{theorem}{Theorem}[section]
%\newmdtheoremenv[style=leftbar]{proposition}[theorem]{proposition}
%\newmdtheoremenv[style=leftbar]{lemma}[theorem]{Lemma}
%\newmdtheoremenv[style=leftbar]{corollary}[theorem]{corollary}

\newtheorem{theorem}{Theorem}[section]
\newtheorem{proposition}[theorem]{Proposition}
\newtheorem{lemma}[theorem]{Lemma}
\newtheorem{corollary}[theorem]{Corollary}
\newtheorem{axiom}[theorem]{Axiom}
\newtheorem*{axiom*}{Axiom}

\surroundwithmdframed[style=leftbar]{theorem}
\surroundwithmdframed[style=leftbar]{proposition}
\surroundwithmdframed[style=leftbar]{lemma}
\surroundwithmdframed[style=leftbar]{corollary}
\surroundwithmdframed[style=leftbar]{axiom}
\surroundwithmdframed[style=leftbar]{axiom*}

\theoremstyle{definition}

\newtheorem*{definition}{Definition}
\surroundwithmdframed[style=leftbar]{definition}

\newtheorem*{slogan}{Slogan}
\newtheorem*{eg}{Example}
\newtheorem*{ex}{Exercise}
\newtheorem*{remark}{Remark}
\newtheorem*{notation}{Notation}
\newtheorem*{convention}{Convention}
\newtheorem*{assumption}{Assumption}
\newtheorem*{question}{Question}
\newtheorem*{answer}{Answer}
\newtheorem*{note}{Note}
\newtheorem*{application}{Application}

%operator macros

%basic
\DeclareMathOperator{\lcm}{lcm}

%matrix
\DeclareMathOperator{\tr}{tr}
\DeclareMathOperator{\Tr}{Tr}
\DeclareMathOperator{\adj}{adj}

%algebra
\DeclareMathOperator{\Hom}{Hom}
\DeclareMathOperator{\End}{End}
\DeclareMathOperator{\id}{id}
\DeclareMathOperator{\im}{im}
\DeclareMathOperator{\coker}{coker}
\DeclarePairedDelimiter{\generation}{\langle}{\rangle}

%groups
\DeclareMathOperator{\sym}{Sym}
\DeclareMathOperator{\sgn}{sgn}
\DeclareMathOperator{\inn}{Inn}
\DeclareMathOperator{\aut}{Aut}
\DeclareMathOperator{\GL}{GL}
\DeclareMathOperator{\SL}{SL}
\DeclareMathOperator{\PGL}{PGL}
\DeclareMathOperator{\PSL}{PSL}
\DeclareMathOperator{\SU}{SU}
\DeclareMathOperator{\UU}{U}
\DeclareMathOperator{\SO}{SO}
\DeclareMathOperator{\OO}{O}
\DeclareMathOperator{\PSU}{PSU}
\DeclareMathOperator{\Sp}{Sp}


%hyperbolic
\DeclareMathOperator{\sech}{sech}

%field, galois heory
\DeclareMathOperator{\ch}{ch}
\DeclareMathOperator{\gal}{Gal}
\DeclareMathOperator{\emb}{Emb}



%ceiling and floor
%https://tex.stackexchange.com/a/118217/26707
\DeclarePairedDelimiter\ceil{\lceil}{\rceil}
\DeclarePairedDelimiter\floor{\lfloor}{\rfloor}


\DeclarePairedDelimiter{\innerproduct}{\langle}{\rangle}

%\DeclarePairedDelimiterX{\norm}[1]{\lVert}{\rVert}{#1}
\DeclarePairedDelimiter{\norm}{\lVert}{\rVert}



%Dirac notation
%TODO: rewrite for variable number of arguments
\DeclarePairedDelimiterX{\braket}[2]{\langle}{\rangle}{#1 \delimsize\vert #2}
\DeclarePairedDelimiterX{\braketthree}[3]{\langle}{\rangle}{#1 \delimsize\vert #2 \delimsize\vert #3}

\DeclarePairedDelimiter{\bra}{\langle}{\rvert}
\DeclarePairedDelimiter{\ket}{\lvert}{\rangle}




%macros

%general

%divide, not divide
\newcommand*{\divides}{\mid}
\newcommand*{\ndivides}{\nmid}
%vector, i.e. mathbf
%https://tex.stackexchange.com/a/45746/26707
\newcommand*{\V}[1]{{\ensuremath{\symbf{#1}}}}
%closure
\newcommand*{\cl}[1]{\overline{#1}}
%conjugate
\newcommand*{\conj}[1]{\overline{#1}}
%set complement
\newcommand*{\stcomp}[1]{\overline{#1}}
\newcommand*{\compose}{\circ}
\newcommand*{\nto}{\nrightarrow}
\newcommand*{\p}{\partial}
%embed
\newcommand*{\embed}{\hookrightarrow}
%surjection
\newcommand*{\surj}{\twoheadrightarrow}
%power set
\newcommand*{\powerset}{\mathcal{P}}

%matrix
\newcommand*{\matrixring}{\mathcal{M}}

%groups
\newcommand*{\normal}{\trianglelefteq}
%rings
\newcommand*{\ideal}{\trianglelefteq}

%fields
\renewcommand*{\C}{{\mathbb{C}}}
\newcommand*{\R}{{\mathbb{R}}}
\newcommand*{\Q}{{\mathbb{Q}}}
\newcommand*{\Z}{{\mathbb{Z}}}
\newcommand*{\N}{{\mathbb{N}}}
\newcommand*{\F}{{\mathbb{F}}}
%not really but I think this belongs here
\newcommand*{\A}{{\mathbb{A}}}

%asymptotic
\newcommand*{\bigO}{O}
\newcommand*{\smallo}{o}

%probability
\newcommand*{\prob}{\mathbb{P}}
\newcommand*{\E}{\mathbb{E}}

%vector calculus
\newcommand*{\gradient}{\V \nabla}
\newcommand*{\divergence}{\gradient \cdot}
\newcommand*{\curl}{\gradient \cdot}

%logic
\newcommand*{\yields}{\vdash}
\newcommand*{\nyields}{\nvdash}

%differential geometry
\renewcommand*{\H}{\mathbb{H}}
\newcommand*{\transversal}{\pitchfork}
\renewcommand{\d}{\mathrm{d}} % exterior derivative

%number theory
\newcommand*{\legendre}[2]{\genfrac{(}{)}{}{}{#1}{#2}}%Legendre symbol

%algebraic geometry
\DeclareMathOperator{\Spec}{Spec}
\DeclareMathOperator{\Proj}{Proj}

\DeclareMathOperator{\Gr}{Gr} % Grassmannian
\DeclareMathOperator{\Lie}{Lie} % Lie functor

\begin{document}

\begin{titlepage}
  \begin{center}
    \includegraphics[width=0.6\textwidth]{logo.jpg}\par
    \vspace{1cm}
    {\scshape\huge Mathamatics Tripos \par}
    \vspace{2cm}
    {\huge Part \npart \par}
    \vspace{0.6cm}
    {\Huge \bfseries \ntitle \par}
    \vspace{1.2cm}
    {\Large\nterm, \nyear \par}
    \vspace{2cm}
    
    {\large \emph{Lectures by } \par}
    \vspace{0.2cm}
    {\Large \scshape \nlecturer}
    
    \vspace{0.5cm}
    {\large \emph{Notes by }\par}
    \vspace{0.2cm}
    {\Large \scshape \href{mailto:\nauthoremail}{\nauthor}}
 \end{center}
\end{titlepage}

\tableofcontents

\section{Manifolds}

We want to generalise curves and surfaces in \(\R^2\). A curve is a map \(\gamma: \R \supseteq I \to \R^2\) or \(\R^3\) that satisfies certain properties we'll find out in a minute. Clearly continuity is necessary but not sufficent, as evidenced by the famous Peano space-filling curve. Smoothness is not quite enough either, as \(t \mapsto (t^2, t^3)\) has a cusp at the origin. The correct requirement will be that \(\gamma\) has regular parameterisation, i.e.\ \(|\dot \gamma(t)| \neq 0\) for all \(t\).

Similarly, a surface should be defined as as a map \(r: \R^2 \supseteq D \to \R^3\) with a regular parameterisation, i.e.\ \(r \in C^\infty(D)\) and \(\left| \frac{\p r}{\p u} \times \frac{\p r}{\p v} \right| \neq 0\) for all \((u, v) \in D\).

We may follow this route and generalise to (hyper)surfaces in \(\R^n\), which will be a generalisation of classical differential geometry of curves and surfaces in \(\R^3\). The good thing is that we can readily apply calculus and it is easy to construct these objects. However, it does suffer from the prolblem of different parameterisations give rise to different geometric objects, as well as some surface requiring more than one parameterisation. Although these can be bypassed more or less, the more serious drawback is the extra technical complexity determined by the higher dimension of the ambient space.

A better concept is \emph{smooth manifolds}. We first begin with a review of topological structure, which every manifold possesses.

\begin{definition}[topological space]
  A \emph{topological space} \(M\) is a choice of class of the \emph{open sets} such that
  \begin{enumerate}
  \item \(\emptyset\) and \(M\) are open,
  \item if \(U\) and \(U'\) are open then so is \(U \cap U'\),
  \item for anly collection of open sets, the union is open.
  \end{enumerate}
\end{definition}

In this course, we always require a topological space to be Hausdorff and second countable.

\begin{definition}[local coordinate chart]\index{chart}
  A \emph{local coordinate chart} on a topological space \(M\) is a homeomorphism \(\varphi: U \to V\) where \(U \subseteq M\) and \(V \subseteq \R^d\) are open. \(U\) is a \emph{coordinate neighbourhood}.
\end{definition}

\begin{definition}[\(C^\infty\)-differentiable structure]\index{differentiable structure}\index{atlas}
  A \emph{\(C^\infty\)-differentiable structure} on a topological space \(M\) is a collection of charts \(\{\varphi_\alpha: U_\alpha \to V_\alpha\}\) where \(V_\alpha \subseteq \R^d\) for all \(\alpha\) such that
    \begin{enumerate}
    \item \(\{U_\alpha\}\) covers \(M\), i.e.\ \(M = \bigcup U_\alpha\),
    \item compatibility condition: for all \(\alpha, \beta\), \(\varphi_\beta \compose \beta_\alpha^{-1}\) is \(C^\infty\) wherever defined, i.e.\ on \(\varphi_\alpha(U_\alpha \cap U_\beta)\),
    \item maximality: if \(\varphi\) is compatible with all the \(\varphi_\alpha\)'s then \(\varphi\) is in the collection.
    \end{enumerate}
    
    The collection of charts is called an \emph{atlas}.
\end{definition}

Note that 2 implies that \(\varphi_\beta \compose \varphi_\alpha^{-1}\) is a diffeomorphism.

\begin{definition}[manifold]\index{manifold}
  A \emph{manifold} is a Hausdorff, second countable topological space with a \(C^\infty\)-differentiable structure.
\end{definition}

\begin{remark}\leavevmode
  \begin{enumerate}
  \item In practice, we almost never specify the topological structure. Instead, we can induce a topology from a \(C^\infty\) structure by declaring \(D \subseteq M\) open if and only if for all \((\varphi_\alpha, U_\alpha)\), \(\varphi_\alpha(U_\alpha \cap D)\) is open in \(\R^d\).
  \item We may replace \(C^\infty\) by \(C^k\) for \(k > 0\) finite. If we set \(k = 0\), the objects become topological manifolds. On the other hand, use \(\C^n\) and holomorphic maps we get complex manifolds.
  \item Requirement of being Hausdorff and second countable are for rather technical reasons. In some cases we may drop Hausdorffness requirement, and such examples do arise naturally. However in that case we lose uniqueness of limits. Similarly non-second countable space, such as the disjoint union of uncountably many \(\R^n\), can become manifold-like structure if we relax the definition.
  \end{enumerate}
\end{remark}

\begin{eg}\leavevmode
  \begin{enumerate}
  \item \(\R^d\) covered by the single chart \(\varphi = \id\).
  \item Unit sphere \(S^n = \{\V x = (x_0, \dots, x_n) \in \R^{n + 1}: \sum_{i = 0}^n x_i^2 = 1\}\). The charts are stereographic projections
    \begin{align}
      \varphi(\V x) &= \frac{1}{1 - x_0} (x_1, \dots, x_n) \\
      \psi(\V x) &= \frac{1}{1 + x_0} (x_1, \dots, x_n)
    \end{align}
    where \(\varphi\) is defined for all points on \(S^n\) except \((1, 0, \dots, 0)\) and similar for \(\psi\). Suppose \(\varphi(P) = u, \psi(P) = v\), then by basic geometry \(v = \psi \compose \varphi^{-1}(u) = \frac{u}{\norm{u}^2}\).
  \item Given \(M_1, M_2\) manifolds of dimension of \(d_1\) and \(d_2\), \(M_1 \times M_2\) is a manifold of dimension \(d_1 + d_2\). We define the \emph{\(n\)-torus} to be \(T_n = \underbrace{S^1 \times \dots \times S^1}_{n}\).
  \item An open subset \(U\) of a manifold \(M\) is a manifold.
  \item The \emph{real projective space},
    \[
      \R P^n = \{\text{all straight lines through \(0\) in } \R^{n + 1}\}.
    \]
    The points in the space are \(x_0 : \dots : x_n\) where \(x_i\)'s are not all zero. Note that
    \[
      x_0 : \dots : x_n = \lambda x_0 : \dots \lambda x_n
    \]
    for all \(\lambda \neq 0\). As per a previous remark, we shall induce the topology by a \(C^\infty\) structure. The charts are \((U_i, \varphi_i)\) where \(U_i = \{x_i \neq 0\}\), and
    \[
      \varphi_i(x_0 : \dots : x_n) = (\frac{x_0}{x_i}, \dots, \hat i, \dots, \frac{x_n}{x_i})
    \]
    where \(\hat i\) denotes that the \(i\)th coordinate is omitted. The \(U_i\)'s cover \(\R P^n\) so we are left to check compatibility. For \(i < j\),
    \[
      \varphi_j \compose \varphi_i^{-1} : (y_1, \dots, y_n)
      \mapsto y_1 : \dots, \underbrace{1}_{i\text{th}} : y_n
      \mapsto (\frac{y_1}{y_j}, \dots, \frac{1}{y_j}, \dots, \hat j, \dots, \frac{y_n}{y_j})
    \]
    is smooth. Since \(i\) and \(j\) are arbitrary, \(\R p^n\) is a \(n\)-manifold.

    Similarly we can check \(\C P^n\) is a \(2n\)-manifold, and \(\H P^n\) is a \(4n\)-manifold (but this is a bit tricky due to noncommutativity).
  \item Grassmannians (over \(\R\) or \(\C\)): we define \(\Gr(k, n)\) to be all \(k\)-dimensional subspaces of the \(n\)-dimensional vector space, which generalises the projective space. For real vector spaces for example, \(\R P^n = \Gr(1, n + 1)\). We can check \(\Gr(k, n)\) is a manifold of dimension \(k(n - k)\).

    The construction is a bit technical so we will give example of one chart. Let \(U\) be the \(k\)-subspaces obtainable as the span of rows of \(k \times n\) matrices of the form \((I_k \quad *)\), and the local coordinate maps a basis of such \(k\)-subspace to the \(k \times (n - k)\) block \(*\). Since the frist \(k\) rows are linearly independent, we call \(U = U_{1 < 2 < \dots < k}\). More generally the domain of charts take the form \(U_{1 \leq i_1 < \dots < i_k \leq n}\). It is an exercise to check that this gives a valid \(C^\infty\) structure.
  \item A non-example: define an equivalence relation \((x, y) \sim (\lambda x, \frac{y}{\lambda})\) for all \(\lambda \neq 0\) and define \(X := \R^2/\sim\). \(\{xy = c\}\) is one equivalence class if \(c \neq 0\). If \(c = 0\), \(\{xy = 0\}\) is three classes \(\{(x, 0): x \neq 0\}, \{(0, 0)\}, \{(0, y): y \neq 0\}\). Thus
    \[
      X \cong (-\infty, 0) \cup \{0', 0'', 0'''\} \cup (0, \infty).
    \]
    Define charts
    \[
      \varphi_i: (-\infty, 0) \cup 0^{(i)} \cup (0, \infty) \to \R
    \]
    in the obvious ways. We can check that it gives \(X\) a valid \(C^\infty\) structure, except that the induced topology is non-Hausdorff!
  \item For a non-second countable example, see example sheet 1 Q12.
  \end{enumerate}
\end{eg}

\begin{definition}[smooth map]\index{smooth map}
  Let \(M, N\) be manifolds, a continuous map \(f: M \to N\) is \emph{smooth} or \(C^\infty\) if for any \(p \in M\), there exists charts \((U, \varphi)\) on \(M\), \((V, \psi)\) on \(N\) such that \(p \in U, f(p) \in V\) and
  \[
    \psi \compose f \compose \varphi^{-1}
  \]
  is smooth wherever it is defined, i.e.\ on \(\varphi(U \cap f^{-1}(V)) \subseteq \R^n\) where \(n = \dim M\).
\end{definition}

\begin{remark}
  Note that by continuity, \(\varphi(U \cap f^{-1}(V))\) is necessarily open. Of course we can do it differently by not requiring \(f\) to be continuous a priori but instead ask the above set to be open.
\end{remark}

For any charts \(\tilde \varphi, \tilde \psi\),
\[
  \tilde \psi \compose f \compose \tilde \varphi^{-1} = (\tilde \psi \compose \psi^{-1}) \compose (\psi \compose f \compose \varphi^{-1}) \compose (\varphi \compose \tilde \varphi^{-1})
\]
is a composition of smooth map so is smooth. Thus smoothness of a map is independent of charts.

\begin{definition}[diffeomorphism]\index{diffeomorphism}
  A smooth map \(f: M \to N\) is a \emph{diffeomorphism} if \(f\) is bijective and \(f^{-1}\) is smooth. If such \(f\) exists, \(M\) and \(N\) are \emph{diffeomorphism}.
\end{definition}

\begin{remark}\leavevmode
\begin{enumerate}
\item Our definition of smoothness is a generalisation of that in calculus. More precisely, \(f: \R^n \to \R^m\) is smooth if and only if it is smooth in the calculus sense.
\item Any chart \(\varphi: U \to \R^d\) is a diffeomorphism onto its image.
\item Composition of smooth maps is smooth.
\end{enumerate}
\end{remark}

\section{Matrix Lie groups}

We can view \(\GL(n, \R)\) as an array of numbers and thus embed it in \(\R^{n^2}\). Furthermore, by considering the determinant function it is an open subset so is an \(n^2\)-manifold. Matrix multiplication is obviously smooth. In the same vein we have \(\GL(n, \C)\), \(\SL(n, \R)\) etc as mannifolds with smooth multiplications.

\begin{definition}[Lie group]\index{Lie group}
  A group \(G\) is a \emph{Lie group} if \(G\) is a manifold and has a compatible group structure, i.e.\ the map \((\sigma, \tau) \mapsto \sigma\tau^{-1}\) is smooth.
\end{definition}

Before Lie groups, let's have a short digression in analysis to discuss the exponential map. For a complex \(n \times n\) matrix \(A = (a_{ij})\), define a norm
\[
  |A| = n \cdot \max_{ij} |a_{ij}|
\]
where the \(n\) in front is such that \(|AB| \leq |A| \cdot |B|\). We now define
\[
  \exp(A) := I + A + \frac{1}{2} A^2 + \dots + \frac{1}{n!} A^n + \dots
\]

Of course we have to check the series makes sense: in fact \(\exp A\) is absolutely convergent for all \(A\) as \(\left| \frac{A^n}{n!} \right| \leq \frac{|A|^n}{n!}\). The series is also uniformly convergent on any compact set, by Weierstrass \(M\)-test. Therefore \(\exp\) is a well-defined continuous map.

In fact, the map is smooth although the proof is technical. A sketch of the proof is like this: note that \(f: A \to A^m\) is smooth for all \(m \in \N\). For \(m = 2\), \(df_A: H \mapsto HA + AH\) so \(\norm{df_A} \leq 2 |A|\) so for all \(m\),
\[
  \norm{df_A} \leq m |A|^{m - 1}
\]
We can thus term-by-term differentiate \(\exp\) and get a locally uniformly convergent series and use the above estimate to bound derivatives.

It is easy to check that:
\begin{enumerate}
\item \(\exp (A^t) = (\exp A)^t\) where \(A^t\) is the transpose of \(A\).
\item \(\exp (CAC^{-1}) = C (\exp A) C^{-1}\) for \(C\) nonsingular.
\item \(\exp (A + B) \neq \exp A \exp B\) unless \(AB = BA\).
\item \(\exp A \exp (-A) = I\) for any matrix \(A\).
\end{enumerate}

The second property prompts us to put \(A\) into Jordan normal form before computing \(\exp A\).

Using the series
\[
  \log (I + A) = A - \frac{A^2}{2} + \dots + (-1)^{n + 1} \frac{A^n}{n} + \dots
\]
we can tackle this similarly to \(\exp\) to check \(\log(I + A)\) is smooth on \(\{|A| < 1\}\). We can also check
\[
  \exp (\log A) = A
\]
if \(|A - I| < 1\), with a proof given by manipulation of double indexed series, which is valid due to absolute convergence. The expression \(\log (\exp A)\) is more subtle. Clearly we need \(|\exp A - I| < 1\). But it is not sufficient: if
\[
  A_\theta =
  \begin{pmatrix}
    0 & -\theta \\
    \theta & 0
  \end{pmatrix}
\]
where \(\theta \in \R\), then
\[
  \exp(A_\theta) =
  \begin{pmatrix}
    \cos \theta & - \sin \theta \\
    \sin \theta & \cos \theta
  \end{pmatrix}
\]
Put \(\theta = 2\pi\), \(\exp A - I = 0\) but
\[
  \log (\exp A_{2\pi}) = 0 \neq A_{2\pi}.
\]
The reason is that the series is no longer absolutely convergent. If we add the additional condition that \(|A| < \log 2\) then
\[
  \log (\exp A) = A.
\]
It is left as an exercise. (Hint: \(|\exp |A| - 1| < 1\) implies absolute convergence.)

\begin{eg}[orthogonal group]
  Recall that
  \[
    O(n) = \{A \in \GL(n, \R), AA^t = I\}.
  \]
  Let \(A \in O(n)\) with \(|A - I| < 1\). Let \(B = \log A\) so \(e^B = A\). There exists \(0 < \varepsilon < 1\) such that whever \(|A - I| < \varepsilon\), have \(|B| < \log 2\) using continuity of \(\log\). Then
  \[
    e^B e^{B^t} = AA^t = I
  \]
  so
  \[
    e^B = A = (A^t)^{-1} = (e^{B^t})^{-1} = e^{-B^t}.
  \]
  Now \(|B^t| = |B| = \log 2\). Taking \(\log\), we find that \(B = -B^t\) so \(B\) is a skew-symmetric matrix.

  Conversely, if \(B = - B^t\), \(|B| < \log 2\) then
  \[
    (e^B)^t = e^{B^t} = e^{-B} = (e^B)^{-1}
  \]
  so \(A = e^B \in O(n)\).

\begin{proposition}
  \(O(n)\) has a \(C^\infty\) structure making it a manifold and Lie group of dimension \(\frac{n(n - 1)}{2}\).
\end{proposition}

\begin{proof}
  Put
  \[
    V_0 := \{B: B \text{ skew-symmetric}, |B| < \log 2\}
  \]
  and \(U := \exp (V_0)\), an open neighbourhood of \(I \in O(n)\). Let
  \begin{align*}
    h: U &\to V_0 \\
    A &\mapsto \log A
  \end{align*}
  which is a well-defined homeomorphism onto \(V_0\), an open subset of the skew-symmetric matrices, which can be identified with \(\R^{n(n - 1)/2}\).

  Now we construct the charts \((U_C, h_C)\). For all \(C = O(n)\), put \(U_C := \{CA: A \in U\}\), i.e.\ left translation of \(U\) by \(C\). Define
  \begin{align*}
    h_C: U_C &\to V_0 \\
    A &\mapsto \log (C^{-1}A)
  \end{align*}
  which is a homeomorphism % ?
  . To check they form an atlas, first note that \(C \in U_C\) so \(O(n) = \bigcup_{C \in O(n)} U_C\). Furthermore
  \[
    h_{C_2} \compose h_{C_1}^{-1} (B) = h_{C_2} (C_1 e^B) = \log (C_2^{-1}C_1 e^B)
  \]
  which is smooth since it is the composition of smooth maps. Thus \(O(n)\) is a manifold.

  To check compatibility of group axioms, define
  \begin{align*}
    F: O(n) \times O(n) &\to O(n) \\
    (A_1, A_2) &\mapsto A_1A_2^{-1}
  \end{align*}
  In local coordinates, it is
  \begin{align*}
    &h_{A_1A_2^{-1}} (F(h_{A_1}^{-1}(B_1), h_{A_2}^{-1}(B_2))) \\
    =& \log [(A_1A_2^{-1})^{-1}A_1 e^{B_1} (A_2 e^{B_2})^{-1}] \\
    =& \log (A_2 e^{B_1} e^{-B_2} A_2^{-1})
  \end{align*}
  which is smooth.
\end{proof}
\end{eg}

The same construction works for other classical groups of matrices. See example sheet 1 Q4.

\section{Tangent space to manifolds}

Consider a curve in \(\R^n\), defined by a smooth parameterisation
\[
  x(t) = (x_i(t))_{i = 1}^n
\]
such that \(x(0) = p \in \R^n\). Then a tangent to the curve is velocity
\[
  \dot x(t) \in T_p\R^n \cong \R^n.
\]
Let \(y = y(x)\) be a \(C^\infty\) change of variables, i.e.\ a local coordinates. Then
\[
  \frac{d}{dt} \Big|_{t = 0} y(x(t)) = \underbrace{\frac{D y}{D x}}_{\text{Jacobian}}(p) \dot x(0)
  = \left( \sum_{j = 1}^n \frac{\p y_i}{\p x_j} a_j \right)_{i = 1}^n
\]

\begin{definition}
  A \emph{tangent vector} \(a\) to a manifold \(M\) at a point \(p \in M\) is the assignment to each chart \((U, \varphi)\), \(p \in U\) a \(n\)-tuple \((a_1, \dots, a_n) \in \R^n\) where \(n = \dim M\) so that for another chart \((U', \varphi')\), \(p \in U'\) with local coordinates \((x_1, \dots, x_n)\) and \((x_1', \dots, x_n')\), we have
  \[
    a_i' = \sum_{j = 1}^n \frac{\p x_i'}{\p x_j} (p) a_j.
  \]
\end{definition}

This is sometimes known as the tensorial definition of tangent space. Other equivalent definitions are: definitions using flows, using derivation on germs of smooth functions, or as dual of the cotangent space of the local ring of germs of functions.

\begin{definition}[tangent space]\index{tangent space}
  The \emph{tangent space} \(T_pM\) is the set of all tangent vectors to \(M\) at \(p\).
\end{definition}

It follows that \(T_pM\) is an \(n\)-dimensional real vector space. Thus \(T_pM \cong \R^n\) although this isomorphism is not canonical. But given a local coordinate chart with coordinates \((x_1, \dots, x_n)\), the tuple \((0, \dots, 1, \dots, 0)\) with \(1\) in \(i\)th position in \(\R^n\) has image under isomorphism \(\frac{\p}{\p x_i} (p)\). Then the transformation law correpsonds to
\[
  \frac{\p}{\p x_j'} = \sum_{i = 1}^n \frac{\p x_i}{\p x_j'} \frac{\p}{\p x_i}
\]
where \((x_1', \dots, x_n')\) is another chart. This is precisely the chain rule for derivatives.

Let \(a = \sum_i a_i \frac{\p}{\p x_i} (p) \in T_pM\) where \(x_i\)'s are local coordinates around \(p\). Then a first order \emph{derivation} at \(p\) is
\begin{align*}
  a: C^\infty(M) &\to \R \\
  f &\mapsto \sum_i a_i \frac{\partial f}{\partial x_i} 
\end{align*}
(where we assume the same charts on RHS) is a well-defined map independent of choice of coordinates. We can interpret, with the \(x_i\)'s,
\[
  a(f) = \frac{d}{dt} \Big|_{t = 0} f(x(t))
\]
for all \(x: (-\varepsilon, \varepsilon) \to M\) smooth, \(x(0) = p\) and \(\dot x(0) = a\).

Now for another choice \(\tilde x_i\) of local coordinates,
\[
  \frac{d}{dt} \Big|_{t = 0} f(\tilde x(t))
  = \sum_j \frac{\partial f}{\partial \tilde x_j} (p) \dot{\tilde x}_j(0)
  = \sum_{j, i} \frac{\partial f}{\partial \tilde x_j}(p) \frac{\partial \tilde x_j}{\partial x_i}(p) \dot x_i(0)
\]
by the transformation law for tangent vectors.% Thus the tensor transformation law is 

The derivations satisfy Leibniz rule, i.e.\
\[
  a(fg) = a(f)g(p) + f(p)a(g).
\]
Conversely, every linear map \(a: C^\infty(M) \to \R\) satisfying the Leibniz rule arises from some \(a \in T_pM\). This is left as an exercise.

\begin{eg}
  An example from classical differential geometry. Consider a surface \(r: D \to S\) where \(D \subseteq \R^2\) and \(S = r(D) \subseteq \R^3\). Then \(S\) is a manifold with \(\varphi = r^{-1}\) as a chart. Then \(r_u, r_v\) at \(p \in S\) corresponds to \(\frac{\partial}{\partial u}, \frac{\partial}{\partial v}\) in our theory.
\end{eg}

\subsection{Lie algebra}

For a Lie group, the tangent spaces get an ``infinitesimal'' version of the group multiplication.

\begin{definition}
  A \emph{Lie algebra} is a vector space with a bilinear multiplication \([\cdot, \cdot]\), i.e.\ a Lie bracket such that
  \begin{enumerate}
  \item anticommutativity: \([a, b] = -[b, a]\),
  \item Jacobi identity: \([[a, b], c] + [[b, c], a] + [[c, a], b] = 0\).
  \end{enumerate}
\end{definition}

\begin{theorem}
  Let \(G\) be a Lie group of \(n \times n\) (real or complex) matrices such that \(\log\) defines a coordinate chart near \(I \in G\), i.e.\ the image of \(\log\) near \(I\) is an open set in some real vector subspace of \(\R^{n^2}\). Identify \(\mathfrak g = T_IG\) with the above open subset. Then \(\mathfrak g\) is a Lie algebra with
  \[
    [B_1, B_2] := B_1B_2 - B_2B_1
  \]
  for \(B_1, B_2 \in \mathfrak g\).
\end{theorem}

\begin{proof}
  Check that \(\mathfrak g\) is a vector space and \([\cdot, \cdot]\) is anticommutative. The Jacobi identity holds for matrices (straightforward check).

  What is left is to show \(B_1, B_2 \in \mathfrak g\) then \([B_1, B_2] \in \mathfrak g\). Consider
  \[
    A(t) = \exp (B_1 t) \exp (B_2 t) \exp(-B_1t) \exp (-B_2t),
  \]
  the commutator of two elements in \(G\). Then \(A(0) = I\). Expand \(\exp\), we get
  \[
    A(t) = I + [B_1, B_2] t^2 + o(t^2)
  \]
  as \(t \to 0\) so
  \[
    B(t) = \log A(t) = [B_1, B_2] t^2 + o(t^2).
  \]
  In addition \(\exp B(t) = A(t)\) holds for \(|t|\) sufficiently small so \(B(t) \in \mathfrak g\) as it is in the image of the \(\log\) chart. It follows that \(\frac{B(t)}{t^2} \in \mathfrak g\) for \(t \neq 0\) as \(\mathfrak g\) is a vector space. Thus
  \[
    [B_1, B_2] = \lim_{t \to 0} \frac{B(t)}{t^2} \in \mathfrak g
  \]
  as every vector subspace of matrix \(n, \C\) is a closed subset.
\end{proof}

\begin{eg}
  For \(G = O(n)\), have \(\mathfrak g = \mathfrak o(n) = \{\text{skew-symmetric \(n \times n\) matrices}\}\) by previous work.
\end{eg}

\begin{definition}
  \(\mathfrak g\) is called the \emph{Lie algebra} of \(G\), write \(\mathfrak g = \Lie(G)\).
\end{definition}

In fact we can show \(\Lie\) is a functor but we won't pursue in that direction.

\begin{definition}[tangent bundle]\index{tangent bundle}
  Let \(M\) be a smooth manifold. Then \(TM = \coprod_{p \in M} T_pM\) is the \emph{tangent bundle} of \(M\).
\end{definition}

\begin{theorem}
  \(TM\) has a natural \(C^\infty\) structure, making it into a smooth manifold of \(\dim TM = 2 \dim M\).
\end{theorem}

\begin{proof}
  We shall induce the topology from the \(C^\infty\) structure. Let \((U, \varphi)\) be a chart on \(M\). Consider \(U_T = \coprod_{p \in U} T_pM\) so \(TM = \bigcup U_T\). For \(a \in T_pM, \varphi(p) = (x_1, \dots, x_n)\) so that \(a = \sum_i a_i \frac{\partial  }{\partial x_i}\). Now define
  \begin{align*}
    \varphi_T: U_T &\to \R^n \times \R^n \\
    a &\mapsto (\varphi(p), (a_i))
  \end{align*}

  To show compatibility, suppose \((U', \varphi')\) is another chart on \(M\) with local coordinates \(x_i'\) and efine \(\tilde \varphi'\) as above. Then
  \[
    \varphi_T' \compose \varphi_T^{-1}(x, a)
    = (x', a')
  \]
  where \(x' = \varphi' \compose \varphi^{-1}(x)\) and \(a'\) is given by the translation law
  \[
    a' = \sum_j \frac{\partial x_i'}{\partial x_j} (x) a_j,
  \]
  so is smooth wherever defined.

  Hausdorffness and second countability follows from that \(M\) and \(\R^n\) are manifolds.
\end{proof}

\begin{note}
  Some remarks on the final statement regarding topological properties:
  \begin{enumerate}
  \item \(M\) is \(\sigma\)-compact, i.e.\ every open cover has a countable subcover.
  \item A basis of topology of \(TM\) is given by \(\{B_1 \times B_2\}\) where \(B_1\) is open in some coordinate neighbourhood \(U \subseteq M\) and \(B_2\) is open in \(\R^n\).
  \end{enumerate}
\end{note}

\begin{corollary}
  The projection
  \begin{align*}
    \pi: TM &\to M \\
    (p, a) &\mapsto p
  \end{align*}
  is smooth.
\end{corollary}

\begin{remark}
  \(TM\) has locally a product structure but in general \(TM\) is not diffeomorphic to \(M \times \R^n\).
\end{remark}

\begin{definition}[vector field]\index{vector field}
  A \emph{vector field} on a manifold \(M\) is a smooth map \(X: M \to TM\) such that \(\pi \compose X = \id_M\), i.e.\ \(X(p) \in T_pM\) for all \(p \in M\).
\end{definition}

\begin{eg}
  Every manifold has at least one vector field: sending every point to \(0\). This is not the most interesting example, however.
\end{eg}














\printindex
\end{document}

% https://www.dpmms.cam.ac.uk/~agk22/teaching.html