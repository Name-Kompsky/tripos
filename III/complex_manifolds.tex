\documentclass[a4paper]{article}

\def\npart{III}

\def\ntitle{Complex Manifolds}
\def\nlecturer{R.\ Dervan}

\def\nterm{Lent}
\def\nyear{2019}

\ifx \nauthor\undefined
  \def\nauthor{Qiangru Kuang}
\else
\fi

\ifx \ntitle\undefined
  \def\ntitle{Template}
\else
\fi

\ifx \nauthoremail\undefined
  \def\nauthoremail{qk206@cam.ac.uk}
\else
\fi

\ifx \ndate\undefined
  \def\ndate{\today}
\else
\fi

\title{\ntitle}
\author{\nauthor}
\date{\ndate}

%\usepackage{microtype}
\usepackage{mathtools}
\usepackage{amsthm}
\usepackage{stmaryrd}%symbols used so far: \mapsfrom
\usepackage{empheq}
\usepackage{amssymb}
\let\mathbbalt\mathbb
\let\pitchforkold\pitchfork
\usepackage{unicode-math}
\let\mathbb\mathbbalt%reset to original \mathbb
\let\pitchfork\pitchforkold

\usepackage{imakeidx}
\makeindex[intoc]

%to address the problem that Latin modern doesn't have unicode support for setminus
%https://tex.stackexchange.com/a/55205/26707
\AtBeginDocument{\renewcommand*{\setminus}{\mathbin{\backslash}}}
\AtBeginDocument{\renewcommand*{\models}{\vDash}}%for \vDash is same size as \vdash but orginal \models is larger
\AtBeginDocument{\let\Re\relax}
\AtBeginDocument{\let\Im\relax}
\AtBeginDocument{\DeclareMathOperator{\Re}{Re}}
\AtBeginDocument{\DeclareMathOperator{\Im}{Im}}
\AtBeginDocument{\let\div\relax}
\AtBeginDocument{\DeclareMathOperator{\div}{div}}

\usepackage{tikz}
\usetikzlibrary{automata,positioning}
\usepackage{pgfplots}
%some preset styles
\pgfplotsset{compat=1.15}
\pgfplotsset{centre/.append style={axis x line=middle, axis y line=middle, xlabel={$x$}, ylabel={$y$}, axis equal}}
\usepackage{tikz-cd}
\usepackage{graphicx}
\usepackage{newunicodechar}

\usepackage{fancyhdr}

\fancypagestyle{mypagestyle}{
    \fancyhf{}
    \lhead{\emph{\nouppercase{\leftmark}}}
    \rhead{}
    \cfoot{\thepage}
}
\pagestyle{mypagestyle}

\usepackage{titlesec}
\newcommand{\sectionbreak}{\clearpage} % clear page after each section
\usepackage[perpage]{footmisc}
\usepackage{blindtext}

%\reallywidehat
%https://tex.stackexchange.com/a/101136/26707
\usepackage{scalerel,stackengine}
\stackMath
\newcommand\reallywidehat[1]{%
\savestack{\tmpbox}{\stretchto{%
  \scaleto{%
    \scalerel*[\widthof{\ensuremath{#1}}]{\kern-.6pt\bigwedge\kern-.6pt}%
    {\rule[-\textheight/2]{1ex}{\textheight}}%WIDTH-LIMITED BIG WEDGE
  }{\textheight}% 
}{0.5ex}}%
\stackon[1pt]{#1}{\tmpbox}%
}

%\usepackage{braket}
\usepackage{thmtools}%restate theorem
\usepackage{hyperref}

% https://en.wikibooks.org/wiki/LaTeX/Hyperlinks
\hypersetup{
    %bookmarks=true,
    unicode=true,
    pdftitle={\ntitle},
    pdfauthor={\nauthor},
    pdfsubject={Mathematics},
    pdfcreator={\nauthor},
    pdfproducer={\nauthor},
    pdfkeywords={math maths \ntitle},
    colorlinks=true,
    linkcolor={red!50!black},
    citecolor={blue!50!black},
    urlcolor={blue!80!black}
}

\usepackage{cleveref}



% TODO: mdframed often gives bad breaks that cause empty lines. Would like to switch to tcolorbox.
% The current workaround is to set innerbottommargin=0pt.

%\usepackage[theorems]{tcolorbox}





\usepackage[framemethod=tikz]{mdframed}
\mdfdefinestyle{leftbar}{
  %nobreak=true, %dirty hack
  linewidth=1.5pt,
  linecolor=gray,
  hidealllines=true,
  leftline=true,
  leftmargin=0pt,
  innerleftmargin=5pt,
  innerrightmargin=10pt,
  innertopmargin=-5pt,
  % innerbottommargin=5pt, % original
  innerbottommargin=0pt, % temporary hack 
}
%\newmdtheoremenv[style=leftbar]{theorem}{Theorem}[section]
%\newmdtheoremenv[style=leftbar]{proposition}[theorem]{proposition}
%\newmdtheoremenv[style=leftbar]{lemma}[theorem]{Lemma}
%\newmdtheoremenv[style=leftbar]{corollary}[theorem]{corollary}

\newtheorem{theorem}{Theorem}[section]
\newtheorem{proposition}[theorem]{Proposition}
\newtheorem{lemma}[theorem]{Lemma}
\newtheorem{corollary}[theorem]{Corollary}
\newtheorem{axiom}[theorem]{Axiom}
\newtheorem*{axiom*}{Axiom}

\surroundwithmdframed[style=leftbar]{theorem}
\surroundwithmdframed[style=leftbar]{proposition}
\surroundwithmdframed[style=leftbar]{lemma}
\surroundwithmdframed[style=leftbar]{corollary}
\surroundwithmdframed[style=leftbar]{axiom}
\surroundwithmdframed[style=leftbar]{axiom*}

\theoremstyle{definition}

\newtheorem*{definition}{Definition}
\surroundwithmdframed[style=leftbar]{definition}

\newtheorem*{slogan}{Slogan}
\newtheorem*{eg}{Example}
\newtheorem*{ex}{Exercise}
\newtheorem*{remark}{Remark}
\newtheorem*{notation}{Notation}
\newtheorem*{convention}{Convention}
\newtheorem*{assumption}{Assumption}
\newtheorem*{question}{Question}
\newtheorem*{answer}{Answer}
\newtheorem*{note}{Note}
\newtheorem*{application}{Application}

%operator macros

%basic
\DeclareMathOperator{\lcm}{lcm}

%matrix
\DeclareMathOperator{\tr}{tr}
\DeclareMathOperator{\Tr}{Tr}
\DeclareMathOperator{\adj}{adj}

%algebra
\DeclareMathOperator{\Hom}{Hom}
\DeclareMathOperator{\End}{End}
\DeclareMathOperator{\id}{id}
\DeclareMathOperator{\im}{im}
\DeclareMathOperator{\coker}{coker}
\DeclarePairedDelimiter{\generation}{\langle}{\rangle}

%groups
\DeclareMathOperator{\sym}{Sym}
\DeclareMathOperator{\sgn}{sgn}
\DeclareMathOperator{\inn}{Inn}
\DeclareMathOperator{\aut}{Aut}
\DeclareMathOperator{\GL}{GL}
\DeclareMathOperator{\SL}{SL}
\DeclareMathOperator{\PGL}{PGL}
\DeclareMathOperator{\PSL}{PSL}
\DeclareMathOperator{\SU}{SU}
\DeclareMathOperator{\UU}{U}
\DeclareMathOperator{\SO}{SO}
\DeclareMathOperator{\OO}{O}
\DeclareMathOperator{\PSU}{PSU}
\DeclareMathOperator{\Sp}{Sp}


%hyperbolic
\DeclareMathOperator{\sech}{sech}

%field, galois heory
\DeclareMathOperator{\ch}{ch}
\DeclareMathOperator{\gal}{Gal}
\DeclareMathOperator{\emb}{Emb}



%ceiling and floor
%https://tex.stackexchange.com/a/118217/26707
\DeclarePairedDelimiter\ceil{\lceil}{\rceil}
\DeclarePairedDelimiter\floor{\lfloor}{\rfloor}


\DeclarePairedDelimiter{\innerproduct}{\langle}{\rangle}

%\DeclarePairedDelimiterX{\norm}[1]{\lVert}{\rVert}{#1}
\DeclarePairedDelimiter{\norm}{\lVert}{\rVert}



%Dirac notation
%TODO: rewrite for variable number of arguments
\DeclarePairedDelimiterX{\braket}[2]{\langle}{\rangle}{#1 \delimsize\vert #2}
\DeclarePairedDelimiterX{\braketthree}[3]{\langle}{\rangle}{#1 \delimsize\vert #2 \delimsize\vert #3}

\DeclarePairedDelimiter{\bra}{\langle}{\rvert}
\DeclarePairedDelimiter{\ket}{\lvert}{\rangle}




%macros

%general

%divide, not divide
\newcommand*{\divides}{\mid}
\newcommand*{\ndivides}{\nmid}
%vector, i.e. mathbf
%https://tex.stackexchange.com/a/45746/26707
\newcommand*{\V}[1]{{\ensuremath{\symbf{#1}}}}
%closure
\newcommand*{\cl}[1]{\overline{#1}}
%conjugate
\newcommand*{\conj}[1]{\overline{#1}}
%set complement
\newcommand*{\stcomp}[1]{\overline{#1}}
\newcommand*{\compose}{\circ}
\newcommand*{\nto}{\nrightarrow}
\newcommand*{\p}{\partial}
%embed
\newcommand*{\embed}{\hookrightarrow}
%surjection
\newcommand*{\surj}{\twoheadrightarrow}
%power set
\newcommand*{\powerset}{\mathcal{P}}

%matrix
\newcommand*{\matrixring}{\mathcal{M}}

%groups
\newcommand*{\normal}{\trianglelefteq}
%rings
\newcommand*{\ideal}{\trianglelefteq}

%fields
\renewcommand*{\C}{{\mathbb{C}}}
\newcommand*{\R}{{\mathbb{R}}}
\newcommand*{\Q}{{\mathbb{Q}}}
\newcommand*{\Z}{{\mathbb{Z}}}
\newcommand*{\N}{{\mathbb{N}}}
\newcommand*{\F}{{\mathbb{F}}}
%not really but I think this belongs here
\newcommand*{\A}{{\mathbb{A}}}

%asymptotic
\newcommand*{\bigO}{O}
\newcommand*{\smallo}{o}

%probability
\newcommand*{\prob}{\mathbb{P}}
\newcommand*{\E}{\mathbb{E}}

%vector calculus
\newcommand*{\gradient}{\V \nabla}
\newcommand*{\divergence}{\gradient \cdot}
\newcommand*{\curl}{\gradient \cdot}

%logic
\newcommand*{\yields}{\vdash}
\newcommand*{\nyields}{\nvdash}

%differential geometry
\renewcommand*{\H}{\mathbb{H}}
\newcommand*{\transversal}{\pitchfork}
\renewcommand{\d}{\mathrm{d}} % exterior derivative

%number theory
\newcommand*{\legendre}[2]{\genfrac{(}{)}{}{}{#1}{#2}}%Legendre symbol

%algebraic geometry
\DeclareMathOperator{\Spec}{Spec}
\DeclareMathOperator{\Proj}{Proj}

\renewcommand{\P}{\mathbb P} % projective space
\newcommand{\w}{\wedge} % wedge product

\begin{document}

\begin{titlepage}
  \begin{center}
    \includegraphics[width=0.6\textwidth]{logo.jpg}\par
    \vspace{1cm}
    {\scshape\huge Mathamatics Tripos \par}
    \vspace{2cm}
    {\huge Part \npart \par}
    \vspace{0.6cm}
    {\Huge \bfseries \ntitle \par}
    \vspace{1.2cm}
    {\Large\nterm, \nyear \par}
    \vspace{2cm}
    
    {\large \emph{Lectures by } \par}
    \vspace{0.2cm}
    {\Large \scshape \nlecturer}
    
    \vspace{0.5cm}
    {\large \emph{Notes by }\par}
    \vspace{0.2cm}
    {\Large \scshape \href{mailto:\nauthoremail}{\nauthor}}
 \end{center}
\end{titlepage}

\tableofcontents

\setcounter{section}{-1}

\section{Introduction}

Motivation: complex geometry is the study of complex manifolds. These locally look like open subsets of \(\C^n\) with holomorphic transition functions. In particular one dimensional complex manifolds are Riemann surfaces. Every (smooth) projective variety is a complex manifolds. A main result of the course is to give a partial converse.

Complex tools are often used to study projective varieties. For example Hodge conjecture and moduli theory. On the other hand there are lots of questions that are also interesting in their own right. Projective surfaces were classified in 1916. Classification of compact complex surfaces is still open (most recent progress in 2005).

\section{Several complex variables}

\begin{definition}[holomorphic]\index{holomorphic}
  Let \(U \subseteq \C^n\) be open. A smooth function \(f: U \to \C\) is \emph{holomorphic} if it is holomorphic in each variable. A function \(F: U \to \C^m\) is \emph{holomorphic} if each coordinate is holomorphic.
\end{definition}

\begin{remark}
  There is an equivalent definition in terms of power series.
\end{remark}

Consider the homeomorphism
\begin{align*}
  \C^n &\to \R^{2n} \\
  (x_1 + iy_1, \dots, x_n + iy_n) &\mapsto (x_1, y_1, \dots, x_n, y_n)
\end{align*}
If \(f = u + iv\) then complex analysis implies that \(f\) is holomorphic if and only if
\begin{align*}
  \frac{\partial u}{\partial x_j} &= \frac{\partial v}{\partial y_j} \\
  \frac{\partial u}{\partial y_i} &= - \frac{\partial v}{\partial x_j}
\end{align*}
this is the Cauchy-Riemann equations. If one formally defines
\begin{align*}
  \frac{\partial  }{\partial z_j} &= \frac{1}{2}(\frac{\partial  }{\partial x_j} - i \frac{\partial }{\partial y_j}) \\
  \frac{\partial  }{\partial \conj{z_j}} &= \frac{1}{2}(\frac{\partial  }{\partial x_j} + i \frac{\partial }{\partial y_j}) \\
\end{align*}
then \(f\) is holomorphic if and only if \(\frac{\partial f}{\partial \conj{z_j}} = 0\) for all \(j\).

\begin{proposition}[maximum principle]\index{maximum principle}
  Let \(U \subseteq \C^n\) open and connected. If \(f\) is holomorphic on some bounded open disk \(U\) with \(\cl D \subseteq U\) then
  \[
    \max_{\cl D} |f(z)| = \max_{\partial \cl D} |f(z)|.
  \]
\end{proposition}

\begin{proof}
  Repeated application of single variable maximum principle.
\end{proof}

Thus if \(|f|\) achieves its maximum at an interior point, \(f\) is constant.

\begin{proposition}[identity principle]\index{identity principle}
  If \(U \subseteq \C^n\) is open connected and \(f: U \to \C\) is holomorphic and \(f\) vanishes on an open subset of \(U\) then \(f = 0\).
\end{proposition}

\begin{proof}
  Repeated application of single variable version of identity principle.
\end{proof}

\section{Complex manifolds}

Let \(X\) be a second countable Hausdorff topological space. We always assume \(X\) is connected.

\begin{definition}[holomorphic atlas]\index{holomorphic atlas}\index{atlas}
  A \emph{holomorphic atlas} for \(X\) is a collection of \((U_\alpha, \varphi_\alpha)\) where \(\varphi_\alpha: U_\alpha \to \varphi_\alpha(U_\alpha) \subseteq \C^n\) is a homeomorphism, with
  \begin{enumerate}
  \item \(X = \bigcup_\alpha U_\alpha\),
  \item \(\varphi_\alpha \compose \varphi_\beta^{-1}\) are holomorphic.
  \end{enumerate}
\end{definition}

\begin{definition}[equivalent atlas]\index{atlas!equivalent}
  Two holomorphic atlases \((U_\alpha, \varphi_\alpha), (\tilde U_\beta, \tilde U_\beta)\) are \emph{equivalent} if \(\varphi_\alpha \compose \tilde \varphi_\beta^{-1}\) is holomorphic fo all \(\alpha, \beta\).

  Equivalently, their union is an atlas.
\end{definition}

\begin{definition}[complex manifold, complex structure]\index{complex manifold, complex structure}
  A \emph{complex manifold} is a topological space as above with an euivalence class of holomorphic atlases. Such an equivalence class is called a \emph{complex structure}.
\end{definition}

\begin{eg}\leavevmode
  \begin{enumerate}
  \item \(\C^n\) is trivially a complex manifold.
  \item \(\Delta = \{z: |z| < 1| \subseteq \C\).
  \item \(\P^n\), the (complex) projective space. As a set this is the one-dimensional linear subspaces of \(\C^{n + 1}\). A point is \([z_0: \dots: z_n]\). A holomorphic atlas is give by
    \begin{align*}
      U_i &= \{z_i \neq 0\} \\
      \varphi_i([z_0: \dots:z_n]) &= (\frac{z_1}{z_i}, \dots, \hat{\frac{z_i}{z_i}}, \dots, \frac{z_n}{z_i})
    \end{align*}
    where the hat denotes that the coordinate is missing. One can check that transition functions are holomorphic. Moreover \(\P^n\) is compact.
  \end{enumerate}
\end{eg}

\begin{definition}[holomorphic, biholomorphic]\index{holomorphic}\index{biholomorphic}
  A smooth function \(f: X \to \C\) is \emph{holomorphic} if \(f \compose \varphi^{-1}: \varphi(U) \to \C\) is holomorphic for all \((U, \varphi)\).

  A smooth map \(F: X \to Y\) is \emph{holomorphic} if for all charts \((U, \varphi)\) for \(X\), \((V, \psi)\) for \(Y\), the map \(\psi \compose F \compose \varphi^{-1}\) is holomorphic. \(F\) is \emph{biholomorphic} if it has a holomorpic inverse.
\end{definition}

\begin{ex}
  If \(X\) is compact then any holomorphic function on \(X\) is constant. As a corollary, compact complex manifold cannot embed in \(\C^m\) for any \(m\).
\end{ex}

\begin{ex}
  If \(X \to \C\) is holomorphic and vanishes on an open set on \(X\) then \(f = 0\). Thus there is no holomorphic analogue of bump functions.
\end{ex}

\begin{definition}[closed complex submanifold]\index{closed submanifold}
  Let \(Y \subseteq X\) be a smooth submanifold of dimension \(2k < 2n = \dim X\). We say \(Y\) is a \emph{closed complex submanifold} if there exists a holomorphic atlas \((U_\alpha, \varphi_\alpha)\) for \(X\) such that it restricts to
  \[
    \varphi_\alpha: U_\alpha \cap Y \to \varphi(U_\alpha) \cap \C^k
  \]
  with \(\C^k \subseteq \C^n\) as \((z_1, \dots, z_k, 0, \dots, 0)\).
\end{definition}

\begin{ex}
  Show that a closed complex submanifold is naturally a complex manifold.
\end{ex}

\begin{definition}[projective manifold]\index{projective manifold}
  We say \(X\) is \emph{projective} is it is biholomorphic to a compact closed complex submanifold of \(\P^m\) for some \(m\).
\end{definition}

We state without a theorem:

\begin{theorem}[Chow]
  A projective complex manifold is projective variety, i.e.\ the vanishing set in \(\P^m\) of some homogeneous polynomial equations.
\end{theorem}

In the example sheet we'll see an example of a compact complex manifold which is not projective.

\section{Almost complex structures}

How much complex structure can be recovered from linear data?

Let \(V\) be a real vector space.

\begin{definition}[complex structure]
  A linear map \(J: V \to V\) with \(J^2 = - \id\) is called a \emph{complex structure}.
\end{definition}

This is motivated by the endomorphism on \(\R^{2n}\)
\[
  (x_1, y_1, \dots, x_n, y_n) \mapsto (y_1, -x_1, \dots, y_n, -x_n).
\]
This is called the \emph{standard complex structure}\index{complex structure!standard}.

As \(J^2 = -\id\), the eigenvalues are \(\pm i\). Since \(V\) is real, there are no eigenspaces. Consider \(V_\C = V \otimes_\R \C\). Then \(J\) extends to \(J: V_\C \to V_\C\) with \(J^2 = -\id\). let \(V^{1, 0}\) and \(V^{0, 1}\) denote the eigenspaces of \(\pm i\) respectively.

\begin{lemma}\leavevmode
  \begin{enumerate}
  \item \(V_\C = V^{1, 0} \oplus V^{0, 1}\).
  \item \(\cl{V^{1, 0}} = V^{0, 1}\).
  \end{enumerate}
\end{lemma}

\begin{proof}\leavevmode
  \begin{enumerate}
  \item For \(v \in V_\C\), write
    \[
      v = \frac{1}{2}\underbrace{(v - iJv)}_{\in V^{1, 0}} + \frac{1}{2}\underbrace{(v + i Jv)}_{\in V^{0, 1}}.
    \]
  \item Follows from 1.
  \end{enumerate}
\end{proof}

\begin{definition}[almost complex structure]\index{almost complex structure}
  Let \(X\) be a smooth manifold. An \emph{almost complex structure} is a bundle isomorphism \(J: TX \to TX\) with \(J_x: T_xX \to T_xX\) and \(J^2 = -\id\).
\end{definition}

One can complexify \(TX\) to obtain \((TX)_\C = TX \otimes \C\) so each fibre of \((TX)_\C \to X\) is a complex vector space. \((TX)_\C\) is called the \emph{complexified tangent bundle}\index{complexified tangent bundle}.

Just as the case for complex structure, \((TX)_\C\) splits as a direct sum
\[
  (TX)_\C \cong TX^{1, 0} \oplus TX^{0, 1}.
\]
To obtain this, one uses, for example,
\begin{align*}
  TX^{1, 0} &= \ker (J - i \id) \\
  TX^{0, 1} &= \ker (J + i \id)
\end{align*}

\begin{ex}
  Let \(U, V \subseteq \C^n\) open, \(f: U \to V\) smooth. Then \(f\) is holomorphic if and only if \(df\) is \(\C\)-linear.
\end{ex}

On \(T\R^{2n}\) there is a natural almost complex structure coming from the one on \(\R^{2n}\), denoted \(J_{\text{st}}\). Let \(X\) be a complex manifold. If  \(U \subseteq X\) is a chart with \(\varphi: U \to \varphi(U) \subseteq \C^n \cong \R^{2n}\), the differential of \(\varphi\) gives a bundle map \(J: TU \to TU, J = d\varphi^{-1} \compose J_{\text{st}} \compose d\varphi\).

\begin{proposition}
  \(J\) defined above is independent of (holomorphic) chart, so gives an almost complex structure on \(X\).
\end{proposition}

\begin{proof}
  Suppose \(\varphi, \psi\) are charts around the same point. What we need to show is
  \[
    d\varphi^{-1} \compose J_{\text{st}} \compose d\varphi = d\psi^{-1} \compose J_{\text{st}} \compose d \psi,
  \]
  i.e.
  \[
    d((\varphi \compose \psi^{-1})^{-1}) \compose J_{\text{st}} \compose d(\varphi \compose \psi^{-1}) = J_{\text{st}}.
  \]
  \(\varphi \compose \psi^{-1}\) is a holomorphic map between open open subsets of \(\C^n\) and so \(d((\varphi \compose \psi^{-1}))\) commutes with \(J_{\text{st}}\), which is similar to the exercise.
\end{proof}

\begin{remark}
  There are lots of almost complex structure not arising in this way. Those that do are called \emph{integrable}\index{integrable}. In general it is difficult to tell whether a smooth manifold with an almost complex structure admits a complex structure. For example \(S^6\) admits an almost complex structure which is \emph{not} integrable. It's an open problem whether or not \(S^6\) admits a complex structure. As an aside, an almost complex structure is integrable if and only if the Nijenhuis tensor vanishes.
\end{remark}

\begin{definition}[holomorphic tangent bundle]\index{holomorphic tangent bundle}
  \(TX^{1, 0}\) is called the \emph{holomorphic tangent bundle} of \(X\).
\end{definition}

If \(V\) is a real vector space and \(J\) is a complex structure then one obtains a complex structure on \(V^*\) in the natural way. Thus analoguously one obtains
\[
  (T^*X)_\C \cong T^*X^{(1, 0)} \oplus T^*X^{(0, 1)}.
\]
Locally if \(\varphi: U \to \C^n\) is a chart, we say that \(z_j = x_j + iy_j\) are local coordinates. Then
\begin{align*}
  J(\frac{\partial  }{\partial x_j}) &= \frac{\partial  }{\partial y_j} \\
  J(\frac{\partial  }{\partial y_j}) &= - \frac{\partial  }{\partial x_j}
\end{align*}
(see the connection with Cauchy-Riemann) and
\begin{align*}
  J(dx_j) &= -dy_j \\
  J(dy_j) &= dx_j
\end{align*}
where we also use \(J\) to denote the dual of \(J\).

\begin{definition}
  We define
  \begin{align*}
    dz_j &= dx_j + idy_j \\
    d\conj z_j &= dx_j - idy_j \\
    \frac{\partial  }{\partial z_j} &= \frac{1}{2} \left( \frac{\partial  }{\partial x_j} - i \frac{\partial  }{\partial y_j} \right) \\
    \frac{\partial  }{\partial \conj z_j} &= \frac{1}{2} \left( \frac{\partial  }{\partial x_j} + i \frac{\partial  }{\partial y_j} \right)
  \end{align*}
  Then \(dz_j, d\conj z_j\) are sections of \((T^*X)_\C\) and \(\frac{\partial  }{\partial z_j}, \frac{\partial  }{\partial \conj z_j}\) are sections of \((TX)_\C\).
\end{definition}

Note that
\begin{align*}
  J(dz_j) &= i dz_j, J(d\conj z_j) = -i d\conj z_j \\
  J(\frac{\partial  }{\partial z_j}) &= i \frac{\partial  }{\partial z_j}, J(\frac{\partial  }{\partial \conj z_j}) = -i \frac{\partial  }{\partial \conj z_j}
\end{align*}
We see the \(dz_j\) form a local frame for \(TX^{1, 0}\), similarly \(d\conj z_j\) form a local frame for \(T^*X^{0, 1}\). Same for tangent bundle.

If \(f: X \to \C\), say \(f = u + iv\) then \(df = du + idv\) is a smooth section of
\[
  (T^*X)_\C \cong T^*X^{1, 0} \oplus T^*X^{0, 1}.
\]
We denote by \(p_1, p_2\) the two projections.

\begin{definition}
  \[
    \p f = p_1 (df)
  \]
\end{definition}

In a local frame,
\[
  df
  = \sum \frac{\partial f}{\partial z_j} dz_j + \sum \frac{\partial f}{\partial \conj z_j} d\conj z_j
  = \p f + \conj \p f
\]
so \(f\) is holomorphic if and only if \(\conj \p f = 0\).

We now do the same for higher degree forms.

\begin{definition}[form]\index{form}
  A section of
  \[
    \Lambda^{(p, q)}T^*X = \Lambda^pT^*X^{1, 0} \otimes \Lambda^qT^*X^{0, 1}.
  \]
  is called a \emph{\((p, q)\)-form}.
\end{definition}

Locally a \((p, q)\)-form looks like
\[
  \sum_f dz_{j_1} \w \dots \w dz_{j_p} \w d\conj z_{\ell_1} \w \dots \w d\conj z_{\ell_q}.
\]
Note that \(f\) is only required to be smooth and not required to be neither holomorphic nor antiholomorphic. So for example \(\conj z dz\) is a section of \(T^*X^{(1, 0)}\).

\begin{definition}
  We denote by \(\mathcal A_\C^k(U)\) the sections of \(\Lambda^k(T^*X)_\C\) over \(U \subseteq X\). We also denote by \(\mathcal A_\C^{p, q}(U)\) the smooth sections of \(\Lambda^{p, q}(U)\).
\end{definition}
In particular \(\mathcal A_\C^{0, 0}(U)\) consists of smooth \(\C\)-valued functions.

\begin{lemma}\leavevmode
  \begin{enumerate}
  \item There is a natural identification
    \[
      \Lambda^k(T^*X)_\C \cong \bigoplus_{p + q = k} \Lambda^{p, q} (T^*X)
    \]
    so
    \[
      \mathcal A_\C^k(U) \cong \bigoplus_{p + q = k} \mathcal A_\C^{p, q} (U).
    \]
  \item If \(\alpha \in \mathcal A_\C^{p, q}(U), \beta \in \mathcal A_\C^{p', q'}(U)\) then \(\alpha \w \beta \in \mathcal A_\C^{p + p', q + q'}(U)\).
  \end{enumerate}
\end{lemma}

\begin{proof}
  Fibrewise this follows from linear algebra. One can use a frame to obtain the bundle results.
\end{proof}

\subsection{Dolbeault cohomology}

Denote by \(d: \mathcal A_\C^k(U) \to \mathcal A_\C^{k + 1}(U)\) the usual exterior derivative.

\begin{definition}
  \(\p: \mathcal A_\C^{p, q}(U) \to \mathcal A_\C^{p + 1, q}(U)\) id defined as \(d\) composed with projection to \(\mathcal A_\C^{p + 1, q}(U)\). Similarly define \(\conj \p: \mathcal A_\C^{p, q}(U) \to \mathcal A_\C^{p, q + 1}(U)\).
\end{definition}

Locally if
\[
  \alpha = \sum fdz_{j_1} \w \dots \w dz_{j_p} \w d \conj z_{\ell_1} \w \dots \w d \conj z_{\ell_q}
\]
then
\[
  d \alpha = \underbrace{\sum (\sum_r \frac{\partial f}{\partial z_r} dz_r \w dz_{j_1} \w \dots dz_{j_p})}_{\p \alpha} + \underbrace{\sum (\sum_r \frac{\partial f}{\partial \conj z_r} d\conj z_r \w d\conj z_{\ell_1} \w \dots d \conj z_{\ell_q})}_{\conj \p \alpha}
\]

\begin{lemma}\leavevmode
  \begin{enumerate}
  \item \(d = \p + \conj \p\).
  \item \(\p^2 = 0, \conj \p^2 = 0, \p \conj \p = - \conj \p \p\).
  \item If \(\alpha \in \mathcal A_\C^{p, q}(U), \beta \in \mathcal A_\C^{p', q'}(U)\) then
    \begin{align*}
      \p(\alpha \w \beta) &= \p \alpha \w \beta + (-1)^{p + q} \alpha \w \p \beta \\
      \conj \p(\alpha \w \beta) &= \conj \p \alpha \w \beta + (-1)^{p + q} \alpha \w \conj \p \beta \\
    \end{align*}
  \end{enumerate}
\end{lemma}

\begin{proof}\leavevmode
  \begin{enumerate}
  \item Follows from local expression.
  \item Follows from \(d^2 = 0\).
  \item Follows from
    \[
      d(\alpha \w \beta) = d\alpha \w \beta + (-1)^{p + q} \alpha \w d \beta.
    \]
  \end{enumerate}
\end{proof}

\begin{definition}[Dolbeault cohomology]\index{Dolbeault cohomology}
  The \emph{\((p, q)\)-Dolbeault cohomology} of \(X\) is given by
  \[
    H_{\conj \p}^{p, q}(X) = \frac{\ker \conj \p: \mathcal A_\C^{p, q}(X) \to \mathcal A_\C^{p, q + 1}(X)}{\im \conj \p: \mathcal A_\C^{p, q - 1}(X) \to \mathcal A_\C^{p, q}(X)}
  \]
  which makes sense as \(\conj \p^2 = 0\). These are vector spaces.
\end{definition}

\begin{remark}
  One could make an analogous definition using \(\p\) and the information would be equivalent. Historically, people are interested in holomorphic functions, i.e.\ \(f\) with \(\conj \p f = 0\).
\end{remark}

Recall the de Rham cohomology group
\[
  H_{\text{dR}}^i(X; \R) = \frac{\ker (d: \mathcal A_\R^i(X) \to \mathcal A_\R^{i + 1}(X))}{\im (d: \mathcal A_\R^{i - 1}(X) \to \mathcal A_\R^i(X))}.
\]
One similarly defines
\[
  H_{\text{dR}}^i(X; \C) = \frac{\ker (d: \mathcal A_\C^i(X) \to \mathcal A_\C^{i + 1}(X))}{\im (d: \mathcal A_\C^{i - 1}(X) \to \mathcal A_\C^i(X))}
  \cong H_{\text{dR}}^i(X; \R) \otimes \C
\]
so we do not gain or lose anything.

Much of the course will be devoted to prove Hodge decomposition, which asserts that for a certain class of compact manifolds (which include projective varieties),
\[
  H_{\text{dR}}^k(X; \C) \cong \bigoplus_{p + q = k} H_{\conj \p}^{p, q}(X).
\]
Note that the statement alone is not true in general.

\begin{ex}
  If \(F: X \to Y\) is holomorphic then \(F\) induces a map
  \[
    F^*: H_{\conj \p}^{p, q}(Y) \to H_{\conj \p}^{p, q}(X)
  \]
  via pullback.
\end{ex}

The Mittag-Leffler problem:  let \(S\) be a Riemann surface, i.e.\ one dimensional complex manifold. A \emph{principal part} at \(x \in S\) is a Laurent series of the form
\[
  \sum_{k = 1}^n a_k z^{-k}
\]
with \(z\) a local coordinate. Then the Mittag-Leffler problem\index{Mittag-Leffler problem} asks given \(x_1, \dots, x_r \in S\) and principal parts \(P_1, \dots, P_r\), is there a meromorphic function on \(S\) with these principal parts at \(x_i\)'s?

Take local solutions \(f_i\) at \(x_i\), defined on some \(U_i\) which form a cover of \(S\)  and a partition of unity \(\rho_i\) subordinate to the \(U_i\). Then \(\sum_{j = 1}^r \rho_j f_j\) is smooth on \(S \setminus \{x_1, \dots, x_r\}\) with described local expression (which is not necessarily holomorphic).

A calculation shows that \(g = \conj \p (\sum_j \rho_j f_j)\) extends to a smooth \((0, 1)\)-form on \(S\). Clearly \(\conj \p g = 0\) as \(\conj \p^2 = 0\) so \([g] \in H_{\conj \p}^{0, 1}(S)\). Suppose \(H_{\conj \p}^{0, 1}(S) = 0\). Then there is a smooth function \(h\) with \(\conj \p h = g\) and \(f = \sum_j \rho_j f_j - h\) solves the Mittag-Leffler problem (this can be shown to be an if and only if using sheaf cohomology).

\subsection{Del bar-Poincaré lemma}
% TODO: resolve the title name

Recall that if \(X\) is a contractible smooth manifold then
\[
  H_{\text{dR}}^i (X; \R) = 0
\]
for \(i > 0\).

We'll show the analogous result for Dolbeault cohomology: if a polydisk \(P = \{|z_i| < r_i\} \subseteq \C^n\) (with \(r = \infty\)) then
\[
  H_{\conj \p}^{p, q}(P) = 0
\]
if \(p + q > 0\).

\begin{proposition}
  Let \(D = D(a, r) \subsetneq \C\) be a disk, \(f \in C^\infty(\cl D), z \in D\). Then
  \[
    f(z) = \frac{1}{2\pi i} \int_{\p D}\frac{f(w)}{w - z} dw + \frac{1}{2\pi i} \int_D \frac{\p f(w)}{\p \conj w} \frac{dw \w d \conj w}{w - z}.
  \]
\end{proposition}
This is a generalisation of Cauchy integral formula, with a correction term for non-holomorphic term.

\begin{proof}
  Let \(D_\varepsilon = D(z, \varepsilon)\) and
  \[
    \eta = \frac{1}{2\pi i} \frac{f(w)}{w - z} dw \in \mathcal A_\C^1 (D \setminus D_\varepsilon).
  \]
  Then
  \[
    d\eta
    = \conj \p \eta
    = -\frac{1}{2\pi i} \frac{\p f(w)}{\p \conj w} \frac{d w \w d \conj w}{w - z}
  \]
  so by Stokes',
  \[
    \frac{1}{2\pi i} \int_{\p D_\varepsilon} \frac{f(w)}{w - z} dw
    = \frac{1}{2\pi i} \int_{\p D} \frac{f(w)}{w - z} dw + \frac{1}{2\pi i} \int_{D \setminus D_\varepsilon} \frac{\p f(w)}{\p \conj w} \frac{dw \w d \conj w}{w - z}
  \]
  The first term converges to \(f(z)\) as \(\varepsilon \to 0\): set \(w - z = re^{i\theta}\) so
  \[
    \frac{1}{2\pi i} \int_{\p D_\varepsilon} \frac{f(w)}{w - z} dw
    = \frac{1}{2\pi} \int_0^{2\pi} f(z + re^{i\theta}) d\theta
  \]
  which goes to \(f(z)\) as \(\varepsilon \to 0\) since \(f\) is smooth.

  As \(dw \w d \conj w = 2ir dr \w d \theta\),
  \[
    \left| \frac{\partial f(w)}{\partial \conj w}  \frac{dw \w d \conj w}{w - z} \right|
    = 2 \left| \frac{\partial f}{\partial \conj w} dr \w d\theta \right|
    \leq C |dr \w d\theta|
  \]
  so
  \[
    \int_{D_\varepsilon} \frac{\partial f(w)}{\partial \conj w} \frac{dw \w d \conj w}{w - z}
    \to 0
  \]
  as \(\varepsilon \to 0\).
\end{proof}

\begin{theorem}[\(\conj \p\)-Poincaré lemma in one variable]
  Let \(D = D(a, r)\) be a disk (\(r < \infty\)) and let \(g \in C^\infty(\cl D)\). Then
  \[
    f(z) = \frac{1}{2\pi i} \int_D \frac{g(w)}{w - z} dw \w d \conj w \in C^\infty(D)
  \]
  and
  \[
    \frac{\p f(z)}{\p \conj z} = g(z).
  \]
\end{theorem}

\begin{proof}
  First reduce to the case \(g\) has compact support. Take \(z_0 \in D\) and \(\varepsilon > 0\) such that
  \[
    D_{2\varepsilon} = D(z_0, 2\varepsilon) \subsetneq D.
  \]
  Using a partition of unity for the cover of \(D\) given by \(\{D \setminus D_\varepsilon, D_{2\varepsilon}\}\), write
  \[
    g(z) = g_1(z) + g_2(z)
  \]
  where \(g_1\) vanishes outside \(D_{2\varepsilon}\) and \(g_2\) vanishes on \(D_\varepsilon\).

  Define
  \[
    f_2(z) = \int_D \frac{g_2(w)}{w - z} dw \w d \conj w.
  \]
  Then \(f_2(z)\) is smooth on \(D\) as \(g_2\) vanishes on \(D_\varepsilon\). Differentiate under the integral sign, get
  \[
    \frac{\partial f_2(z)}{\partial \conj z} = \frac{1}{2\pi i} \int_D \frac{\partial  }{\partial \conj z} \frac{g_2(w)}{w - z} dw \w d \conj w.
  \]
  As \(g_1(z)\) has compact support we can write
  \begin{align*}
    \frac{1}{2\pi i} \int_D \frac{g_1(w)}{w - z} dw \w d\conj w
    &= \frac{1}{2\pi i} \int_\C \frac{g_1(w)}{w - z} dw \w d \conj w \\
    &= \frac{1}{2\pi i} \int_\C \frac{g_1(u + z)}{u} du \w d \conj u \\
    &= -\frac{1}{\pi} \int_\C g_1(z + re^{i\theta}) e^{-i\theta} dr \w d\theta \in C^\infty(D)
  \end{align*}
  Define this to be \(f_1\). The trick here is that we defined \(f\) in this way so that it is automatically smooth. Then
  \[
    \frac{\partial f_1(z)}{\partial \conj z}
    = \int_\C \frac{\partial g_1(w)}{\partial \conj w} \frac{dw \w d \conj w}{w - z}
  \]
  so by Cauchy integral,
  \[
    g_1(z)
    = \frac{1}{2\pi i} \underbrace{\int_{\p D} \frac{g_1(w)}{w - z}dw}_{= 0 \text{ as } g_1 = 0 \text{ on } \p D} + \frac{1}{2\pi i} \int_D \frac{\partial g_1(w)}{\partial \conj w} \frac{dw \w d\conj w}{w - z}
    = \frac{\partial f_1(z)}{\partial \conj z}
  \]
  Setting \(f = f_1 + f_2\) gives
  \[
    \frac{\partial f}{\partial \conj z}(z) = g(z)
  \]
  for \(z \in D_\varepsilon\). But \(z_0\) was arbitrary so this works for all \(z_0\).
\end{proof}

In other words, if \(\alpha = g d \conj z \in \mathcal A_\C^{0, 1}(D)\) and \(f\) is as above, then
\[
  \conj \p f = \alpha.
\]

For the general \(\conj \p\)-Poincaré lemma, we shall use \emph{multiindex notation}: if \(I = (I_1, \dots, I_k)\) then
\begin{align*}
  dz_I &= dz_{I_1} \w \dots \w dz_{I_k} \\
  f_I &= f_{I_1 \dots I_k} \\
  \frac{\partial  }{\partial z_I} &= \frac{\partial }{\partial z_{I_1} \dots z_{I_k}}
\end{align*}
and \(|I| = k\).

At some point we are going to extend the result to \(\C^n\) by taking a sequence of holomorphic functions. The following result justifies the process:

\begin{lemma}
  Let \(U \subseteq \C^n\) be open, \(B \subsetneq B' \subseteq U\) where \(B, B'\) are bounded polydisks. Then for any multiindices \(I, J\) there is a constant \(c_{I, J}\) such that for all \(u\) holomorphic on \(U\), we have
  \[
    \norm*{\frac{\partial  }{\partial z_I} \frac{\partial  }{\partial z_J} u}_{C^0(B)} \leq c_{I, J} \norm u_{C^0(B)}.
  \]
\end{lemma}

\begin{proof}
  Follows from multivariable Cauchy integral formula, which follows from the single variable version.
\end{proof}

\begin{corollary}
  Let \(u_k\) be a sequence of holomorphic functions on \(U\) with \(u_k \to u\) uniformly on compact subsets of \(U\). Then \(u\) is holomorphic.
\end{corollary}

\begin{proof}
  By the previous lemma, \(u\) is smooth. Moreover \(\frac{\partial u_k}{\partial \conj z_j} \to \frac{\partial u}{\partial \conj z_j}\) so since \(\frac{\partial u_k}{\partial \conj z_j} = 0\), \(\conj \p u = 0\) so \(u\) is holomorphic.
\end{proof}

Then we have the following result due to Grothendieck:

\begin{theorem}[\(\conj \p\)-Poincaré lemma]\index{\(\conj \p\)-Poincaré lemma}
  Let
  \[
    P = P(a, r) = \{|z_i - a_i| < r_i\} \subseteq \C^n
  \]
  with \(r_i \in (0, \infty]\). Then for all \(q > 0\) we have
  \[
    H_{\conj \p}^{p, q}(P) = 0.
  \]
  That is, if \(\conj \p \omega = 0\) then exists \(\psi\) with \(\conj \p \psi = \omega\).
\end{theorem}

\begin{proof}
  We first reduce to \(p = 0\). Indeed if \(\omega \in \mathcal A_\C^{p, q}(P)\) is closed then \(\conj \p \omega = 0\) so we may write
  \[
    \omega = \sum_{|I| = p} \varphi_I dz_I
  \]
  where \(\conj \p \varphi_I = 0\). Hence by induction we can find \(\psi_I\) with \(\conj \p \psi_I = \varphi_I\) and then
  \[
    \conj \p \left(\sum_{|I| = p} \psi_I \w dz_I\right) = \omega.
  \]
  Thus we may assume \(p = 0\). The proof is in two steps.

  \paragraph{Step 1}

  Given \(\omega \in \mathcal A_\C^{0, q}(P)\), we show that if \(P' = P(a, s)\) with \(s_i' < r_i\) finite then we can find \(\psi \in \mathcal A_\C^{0, q - 1}(P')\) with \(\conj \p \psi = \omega|_{P'}\).

  Given a form
  \[
    \omega = \sum_{|I| = q} \omega_I d \conj z_I,
  \]
  we say
  \[
    \omega = 0 \mod \{d\conj z_1, \dots, d\conj z_k\}
  \]
  if \(\omega_I = 0\) unless \(I \subseteq \{1, \dots, k\}\). We shall prove that if \(\omega = 0 \mod \{d\conj z_1, \dots, d\conj z_k\}\) then there is \(\psi \in \mathcal A_\C^{0, q - 1}(P')\) such that \(\omega - \conj \p \psi = 0 \mod \{d\conj z_1, \dots, d\conj z_{k - 1}\}\). By induction and \(k = n\) being vacuous, this will prove step 1.

  So suppose \(\omega = 0 \mod \{d\conj z_1, \dots, d\conj z_k\}\) and write
  \[
    \omega = \omega_1 \w d\conj z_k + \omega_2
  \]
  with \(\omega_2 = 0 \mod \{d\conj z_1, \dots, d\conj z_{k - 1}\}\). Have
  \[
    \omega_1 = \sum_{|I| = q, k \in I} \omega_I d \conj z_{I \setminus \{k\}}.
  \]
  Since \(\conj \p \omega = 0\), we have
  \[
    \frac{\partial \omega_I}{\partial \conj z_\ell} = 0
  \]
  for \(\ell > k\). Set
  \[
    \psi = \sum_{|I| = q, k \in I} (-1)^{k - 1} \psi_I d\conj z_{I \setminus \{k\}}
  \]
  where
  \[
    \psi_I = \frac{1}{2\pi i} \int_{|\xi| \leq s_k} \omega_i(z_1, \dots, z_{k - 1}, \xi, z_{k + 1}, \dots, z_n) \frac{d \xi \w d \conj \xi}{\xi - z_k}
  \]
  is given by Cauchy integral formula. Then
  \[
    \frac{\partial \psi_I}{\partial \conj z_k} = \omega_I
  \]
  by \(\conj \p\)-Poincaré in one variable and
 \[
   \frac{\partial \psi_I}{\partial \conj z_k} = \frac{1}{2\pi i} \int_{|\xi| \leq s_k} \frac{\partial \omega_I}{\partial \conj z_j} (z_1, \dots, z_{k - 1}, \xi, z_{k + 1}, \dots, z_n) \frac{d \xi \w d \conj \xi}{\xi - z_k}
   = 0
 \]
 by assumption. Hence \(\omega - \conj \p \psi = 0 \mod d\conj z_1, \dots, d\conj z_{k - 1}\).
  
  \paragraph{Step 2}
  Let \(r_{j, k}\) be a strictly increasing sequence, \(r_{j, k} \to r_k\) as \(j \to \infty\) for all \(k = 1, \dots, n\) and let \(P_j = P(a, r_j)\). By step 1 we can find \(\psi_j \in \mathcal A_\C^{0, q - 1}(P_j)\) with \(\conj \p \psi_j = \omega\) on \(P_j\).

  We induct on \(q\), leaving \(q = 1\) for last. Since \(\conj \p(\psi_j - \psi_{j + 1}) = 0\) on \(P_j\), we can choose \(\beta_{j + 1}\) with
  \[
    \psi_j - \psi_{j + 1} = \conj \p \beta_{j + 1}
  \]
  on \(P_{j - 1}\). Extend \(\psi_{j + 1}, \beta_{j + 1}\) smoothly to \(P\) and set
  \[
    \phi_{j + 1} = \psi_{j + 1} + \conj \p \beta_{j + 1}.
  \]
  This produces a sequence \((\phi_j)\) such that
  \begin{align*}
    \conj \p \phi_{j + 1} &= \omega \text{ on } P_{j + 1} \\
    \phi_{j + 1} &= \phi_j \text{ on } P_{j - 1}
  \end{align*}
  Thus the \((\phi_j)\) converges to \(\phi\) on \(P\) for \(\phi\) such that \(\conj \p \phi = \omega\).

  Now consider the case \(\omega\) is a \((0, 1)\)-form, so \(\psi_j\)'s are functions. We construct a sequence \(\phi_j\) on \(P_j\) such that
  \begin{align*}
    \conj \p \phi_j &= \omega \text{ on } P_j \\
    \phi_{j + 1} &- \phi_j \text{ holomorphic on } P_j \\
    \norm{\phi_{j + 1} - \phi_j}_{C^0(P_{j + 1})} < 2^{-j}
  \end{align*}
  Assuming this, the \((\phi_j)\) converges uniformly to some \(\phi\) on \(P\). Moreover \(\phi - \phi_j\) is holomorphic on \(P_j\) as as a uniform limit of \((\phi_{j + 1} - \phi_j)\), all holomorphic following the corollary. So \(\conj \p \phi = \conj \p \phi_j = \omega\) on \(P_j\), Hence \(\conj \p \phi = \omega\) on \(P\).

  We now construct \((\phi_j)\). SOlve \(\conj \p \psi_j = \omega\) on \(P_j\) as before and set \(\phi_1 = \psi_1\). We construct \(\phi_{j + 1}\), inducting on \(j\). Since \(\conj \p(\phi_j - \psi_{j + 1}) = 0\) on \(P_j\), \(\theta_j - \psi_{j + 1}\) is holomorphic on \(P_j\). Hence it has a Taylor series expansion valid on \(P_j\). Truncating gives a polynomial \(\gamma_{j + 1}\) such that
  \[
    \norm{\phi_j - \psi_{j + 1} - \gamma_{j + 1}}_{C^0(P_j - 1)} < 2^{-j}
  \]
  idea: approximate holomorphic by polynomial to arbitrary small error and extend the polynomial to the entire disk.

  Extend \(\gamma_j\) holomorphically to \(P_j\) and set
  \[
    \phi_{j + 1} = \psi_{j + 1} + \gamma_{j + 1}.
  \]
  Then \(\conj \p \phi_{j + 1} = \omega\) on \(P_{j + 1}\), \(\phi_{j + 1} - \phi_j\) holomorphic on \(P_j\) and \(\norm{\phi_{j + 1} - \phi_j}_{C^0(P_{j - 1})} < 2^{-j}\). This ends the proof.
\end{proof}

\section{Sheaves and cohomology}

We now compare Dolbeault cohomology with sheaf cohomology. Let's begin with general theory of sheaves. Let \(X\) be a topological space.

\begin{definition}
  A \emph{presheaf} \(\mathcal F\) on \(X\) of abelian groups consists of abelian groups \(\mathcal F(U)\) for all \(U \subseteq X\) open and restriction homomorphisms
  \[
    r_{VU}: \mathcal F(U) \to \mathcal F(V)
  \]
  for all \(V \subseteq U\) open.
\end{definition}










\printindex
\end{document}

% https://www.dpmms.cam.ac.uk/~rd430/teaching.html