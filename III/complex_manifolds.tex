\documentclass[a4paper]{article}

\def\npart{III}

\def\ntitle{Complex Manifolds}
\def\nlecturer{R.\ Dervan}

\def\nterm{Lent}
\def\nyear{2019}

\ifx \nauthor\undefined
  \def\nauthor{Qiangru Kuang}
\else
\fi

\ifx \ntitle\undefined
  \def\ntitle{Template}
\else
\fi

\ifx \nauthoremail\undefined
  \def\nauthoremail{qk206@cam.ac.uk}
\else
\fi

\ifx \ndate\undefined
  \def\ndate{\today}
\else
\fi

\title{\ntitle}
\author{\nauthor}
\date{\ndate}

%\usepackage{microtype}
\usepackage{mathtools}
\usepackage{amsthm}
\usepackage{stmaryrd}%symbols used so far: \mapsfrom
\usepackage{empheq}
\usepackage{amssymb}
\let\mathbbalt\mathbb
\let\pitchforkold\pitchfork
\usepackage{unicode-math}
\let\mathbb\mathbbalt%reset to original \mathbb
\let\pitchfork\pitchforkold

\usepackage{imakeidx}
\makeindex[intoc]

%to address the problem that Latin modern doesn't have unicode support for setminus
%https://tex.stackexchange.com/a/55205/26707
\AtBeginDocument{\renewcommand*{\setminus}{\mathbin{\backslash}}}
\AtBeginDocument{\renewcommand*{\models}{\vDash}}%for \vDash is same size as \vdash but orginal \models is larger
\AtBeginDocument{\let\Re\relax}
\AtBeginDocument{\let\Im\relax}
\AtBeginDocument{\DeclareMathOperator{\Re}{Re}}
\AtBeginDocument{\DeclareMathOperator{\Im}{Im}}
\AtBeginDocument{\let\div\relax}
\AtBeginDocument{\DeclareMathOperator{\div}{div}}

\usepackage{tikz}
\usetikzlibrary{automata,positioning}
\usepackage{pgfplots}
%some preset styles
\pgfplotsset{compat=1.15}
\pgfplotsset{centre/.append style={axis x line=middle, axis y line=middle, xlabel={$x$}, ylabel={$y$}, axis equal}}
\usepackage{tikz-cd}
\usepackage{graphicx}
\usepackage{newunicodechar}

\usepackage{fancyhdr}

\fancypagestyle{mypagestyle}{
    \fancyhf{}
    \lhead{\emph{\nouppercase{\leftmark}}}
    \rhead{}
    \cfoot{\thepage}
}
\pagestyle{mypagestyle}

\usepackage{titlesec}
\newcommand{\sectionbreak}{\clearpage} % clear page after each section
\usepackage[perpage]{footmisc}
\usepackage{blindtext}

%\reallywidehat
%https://tex.stackexchange.com/a/101136/26707
\usepackage{scalerel,stackengine}
\stackMath
\newcommand\reallywidehat[1]{%
\savestack{\tmpbox}{\stretchto{%
  \scaleto{%
    \scalerel*[\widthof{\ensuremath{#1}}]{\kern-.6pt\bigwedge\kern-.6pt}%
    {\rule[-\textheight/2]{1ex}{\textheight}}%WIDTH-LIMITED BIG WEDGE
  }{\textheight}% 
}{0.5ex}}%
\stackon[1pt]{#1}{\tmpbox}%
}

%\usepackage{braket}
\usepackage{thmtools}%restate theorem
\usepackage{hyperref}

% https://en.wikibooks.org/wiki/LaTeX/Hyperlinks
\hypersetup{
    %bookmarks=true,
    unicode=true,
    pdftitle={\ntitle},
    pdfauthor={\nauthor},
    pdfsubject={Mathematics},
    pdfcreator={\nauthor},
    pdfproducer={\nauthor},
    pdfkeywords={math maths \ntitle},
    colorlinks=true,
    linkcolor={red!50!black},
    citecolor={blue!50!black},
    urlcolor={blue!80!black}
}

\usepackage{cleveref}



% TODO: mdframed often gives bad breaks that cause empty lines. Would like to switch to tcolorbox.
% The current workaround is to set innerbottommargin=0pt.

%\usepackage[theorems]{tcolorbox}





\usepackage[framemethod=tikz]{mdframed}
\mdfdefinestyle{leftbar}{
  %nobreak=true, %dirty hack
  linewidth=1.5pt,
  linecolor=gray,
  hidealllines=true,
  leftline=true,
  leftmargin=0pt,
  innerleftmargin=5pt,
  innerrightmargin=10pt,
  innertopmargin=-5pt,
  % innerbottommargin=5pt, % original
  innerbottommargin=0pt, % temporary hack 
}
%\newmdtheoremenv[style=leftbar]{theorem}{Theorem}[section]
%\newmdtheoremenv[style=leftbar]{proposition}[theorem]{proposition}
%\newmdtheoremenv[style=leftbar]{lemma}[theorem]{Lemma}
%\newmdtheoremenv[style=leftbar]{corollary}[theorem]{corollary}

\newtheorem{theorem}{Theorem}[section]
\newtheorem{proposition}[theorem]{Proposition}
\newtheorem{lemma}[theorem]{Lemma}
\newtheorem{corollary}[theorem]{Corollary}
\newtheorem{axiom}[theorem]{Axiom}
\newtheorem*{axiom*}{Axiom}

\surroundwithmdframed[style=leftbar]{theorem}
\surroundwithmdframed[style=leftbar]{proposition}
\surroundwithmdframed[style=leftbar]{lemma}
\surroundwithmdframed[style=leftbar]{corollary}
\surroundwithmdframed[style=leftbar]{axiom}
\surroundwithmdframed[style=leftbar]{axiom*}

\theoremstyle{definition}

\newtheorem*{definition}{Definition}
\surroundwithmdframed[style=leftbar]{definition}

\newtheorem*{slogan}{Slogan}
\newtheorem*{eg}{Example}
\newtheorem*{ex}{Exercise}
\newtheorem*{remark}{Remark}
\newtheorem*{notation}{Notation}
\newtheorem*{convention}{Convention}
\newtheorem*{assumption}{Assumption}
\newtheorem*{question}{Question}
\newtheorem*{answer}{Answer}
\newtheorem*{note}{Note}
\newtheorem*{application}{Application}

%operator macros

%basic
\DeclareMathOperator{\lcm}{lcm}

%matrix
\DeclareMathOperator{\tr}{tr}
\DeclareMathOperator{\Tr}{Tr}
\DeclareMathOperator{\adj}{adj}

%algebra
\DeclareMathOperator{\Hom}{Hom}
\DeclareMathOperator{\End}{End}
\DeclareMathOperator{\id}{id}
\DeclareMathOperator{\im}{im}
\DeclareMathOperator{\coker}{coker}
\DeclarePairedDelimiter{\generation}{\langle}{\rangle}

%groups
\DeclareMathOperator{\sym}{Sym}
\DeclareMathOperator{\sgn}{sgn}
\DeclareMathOperator{\inn}{Inn}
\DeclareMathOperator{\aut}{Aut}
\DeclareMathOperator{\GL}{GL}
\DeclareMathOperator{\SL}{SL}
\DeclareMathOperator{\PGL}{PGL}
\DeclareMathOperator{\PSL}{PSL}
\DeclareMathOperator{\SU}{SU}
\DeclareMathOperator{\UU}{U}
\DeclareMathOperator{\SO}{SO}
\DeclareMathOperator{\OO}{O}
\DeclareMathOperator{\PSU}{PSU}
\DeclareMathOperator{\Sp}{Sp}


%hyperbolic
\DeclareMathOperator{\sech}{sech}

%field, galois heory
\DeclareMathOperator{\ch}{ch}
\DeclareMathOperator{\gal}{Gal}
\DeclareMathOperator{\emb}{Emb}



%ceiling and floor
%https://tex.stackexchange.com/a/118217/26707
\DeclarePairedDelimiter\ceil{\lceil}{\rceil}
\DeclarePairedDelimiter\floor{\lfloor}{\rfloor}


\DeclarePairedDelimiter{\innerproduct}{\langle}{\rangle}

%\DeclarePairedDelimiterX{\norm}[1]{\lVert}{\rVert}{#1}
\DeclarePairedDelimiter{\norm}{\lVert}{\rVert}



%Dirac notation
%TODO: rewrite for variable number of arguments
\DeclarePairedDelimiterX{\braket}[2]{\langle}{\rangle}{#1 \delimsize\vert #2}
\DeclarePairedDelimiterX{\braketthree}[3]{\langle}{\rangle}{#1 \delimsize\vert #2 \delimsize\vert #3}

\DeclarePairedDelimiter{\bra}{\langle}{\rvert}
\DeclarePairedDelimiter{\ket}{\lvert}{\rangle}




%macros

%general

%divide, not divide
\newcommand*{\divides}{\mid}
\newcommand*{\ndivides}{\nmid}
%vector, i.e. mathbf
%https://tex.stackexchange.com/a/45746/26707
\newcommand*{\V}[1]{{\ensuremath{\symbf{#1}}}}
%closure
\newcommand*{\cl}[1]{\overline{#1}}
%conjugate
\newcommand*{\conj}[1]{\overline{#1}}
%set complement
\newcommand*{\stcomp}[1]{\overline{#1}}
\newcommand*{\compose}{\circ}
\newcommand*{\nto}{\nrightarrow}
\newcommand*{\p}{\partial}
%embed
\newcommand*{\embed}{\hookrightarrow}
%surjection
\newcommand*{\surj}{\twoheadrightarrow}
%power set
\newcommand*{\powerset}{\mathcal{P}}

%matrix
\newcommand*{\matrixring}{\mathcal{M}}

%groups
\newcommand*{\normal}{\trianglelefteq}
%rings
\newcommand*{\ideal}{\trianglelefteq}

%fields
\renewcommand*{\C}{{\mathbb{C}}}
\newcommand*{\R}{{\mathbb{R}}}
\newcommand*{\Q}{{\mathbb{Q}}}
\newcommand*{\Z}{{\mathbb{Z}}}
\newcommand*{\N}{{\mathbb{N}}}
\newcommand*{\F}{{\mathbb{F}}}
%not really but I think this belongs here
\newcommand*{\A}{{\mathbb{A}}}

%asymptotic
\newcommand*{\bigO}{O}
\newcommand*{\smallo}{o}

%probability
\newcommand*{\prob}{\mathbb{P}}
\newcommand*{\E}{\mathbb{E}}

%vector calculus
\newcommand*{\gradient}{\V \nabla}
\newcommand*{\divergence}{\gradient \cdot}
\newcommand*{\curl}{\gradient \cdot}

%logic
\newcommand*{\yields}{\vdash}
\newcommand*{\nyields}{\nvdash}

%differential geometry
\renewcommand*{\H}{\mathbb{H}}
\newcommand*{\transversal}{\pitchfork}
\renewcommand{\d}{\mathrm{d}} % exterior derivative

%number theory
\newcommand*{\legendre}[2]{\genfrac{(}{)}{}{}{#1}{#2}}%Legendre symbol

%algebraic geometry
\DeclareMathOperator{\Spec}{Spec}
\DeclareMathOperator{\Proj}{Proj}

\usepackage{mathdots} % \iddots
\renewcommand{\P}{\mathbb P} % projective space
\newcommand{\w}{\wedge} % wedge product
\DeclareMathOperator{\Pic}{Pic} % Picard group
\DeclareMathOperator{\ord}{ord} % order
\DeclareMathOperator{\Div}{Div} % divisor group
\newcommand*{\ip}{\innerproduct}

\begin{document}

\begin{titlepage}
  \begin{center}
    \includegraphics[width=0.6\textwidth]{logo.jpg}\par
    \vspace{1cm}
    {\scshape\huge Mathamatics Tripos \par}
    \vspace{2cm}
    {\huge Part \npart \par}
    \vspace{0.6cm}
    {\Huge \bfseries \ntitle \par}
    \vspace{1.2cm}
    {\Large\nterm, \nyear \par}
    \vspace{2cm}
    
    {\large \emph{Lectures by } \par}
    \vspace{0.2cm}
    {\Large \scshape \nlecturer}
    
    \vspace{0.5cm}
    {\large \emph{Notes by }\par}
    \vspace{0.2cm}
    {\Large \scshape \href{mailto:\nauthoremail}{\nauthor}}
 \end{center}
\end{titlepage}

\tableofcontents

\setcounter{section}{-1}

\section{Introduction}

Motivation: complex geometry is the study of complex manifolds. These locally look like open subsets of \(\C^n\) with holomorphic transition functions. In particular one dimensional complex manifolds are Riemann surfaces. Every (smooth) projective variety is a complex manifolds. A main result of the course is to give a partial converse.

Complex tools are often used to study projective varieties. For example Hodge conjecture and moduli theory. On the other hand there are lots of questions that are also interesting in their own right. Projective surfaces were classified in 1916. Classification of compact complex surfaces is still open (most recent progress in 2005).

\section{Several complex variables}

\begin{definition}[holomorphic]\index{holomorphic}
  Let \(U \subseteq \C^n\) be open. A smooth function \(f: U \to \C\) is \emph{holomorphic} if it is holomorphic in each variable. A function \(F: U \to \C^m\) is \emph{holomorphic} if each coordinate is holomorphic.
\end{definition}

\begin{remark}
  There is an equivalent definition in terms of power series.
\end{remark}

Consider the homeomorphism
\begin{align*}
  \C^n &\to \R^{2n} \\
  (x_1 + iy_1, \dots, x_n + iy_n) &\mapsto (x_1, y_1, \dots, x_n, y_n)
\end{align*}
If \(f = u + iv\) then complex analysis implies that \(f\) is holomorphic if and only if
\begin{align*}
  \frac{\partial u}{\partial x_j} &= \frac{\partial v}{\partial y_j} \\
  \frac{\partial u}{\partial y_i} &= - \frac{\partial v}{\partial x_j}
\end{align*}
this is the Cauchy-Riemann equations. If one formally defines
\begin{align*}
  \frac{\partial  }{\partial z_j} &= \frac{1}{2}(\frac{\partial  }{\partial x_j} - i \frac{\partial }{\partial y_j}) \\
  \frac{\partial  }{\partial \conj{z_j}} &= \frac{1}{2}(\frac{\partial  }{\partial x_j} + i \frac{\partial }{\partial y_j}) \\
\end{align*}
then \(f\) is holomorphic if and only if \(\frac{\partial f}{\partial \conj{z_j}} = 0\) for all \(j\).

\begin{proposition}[maximum principle]\index{maximum principle}
  Let \(U \subseteq \C^n\) open and connected. If \(f\) is holomorphic on some bounded open disk \(U\) with \(\cl D \subseteq U\) then
  \[
    \max_{\cl D} |f(z)| = \max_{\partial \cl D} |f(z)|.
  \]
\end{proposition}

\begin{proof}
  Repeated application of single variable maximum principle.
\end{proof}

Thus if \(|f|\) achieves its maximum at an interior point, \(f\) is constant.

\begin{proposition}[identity principle]\index{identity principle}
  If \(U \subseteq \C^n\) is open connected and \(f: U \to \C\) is holomorphic and \(f\) vanishes on an open subset of \(U\) then \(f = 0\).
\end{proposition}

\begin{proof}
  Repeated application of single variable version of identity principle.
\end{proof}

\section{Complex manifolds}

Let \(X\) be a second countable Hausdorff topological space. We always assume \(X\) is connected.

\begin{definition}[holomorphic atlas]\index{atlas}
  A \emph{holomorphic atlas} for \(X\) is a collection of \((U_\alpha, \varphi_\alpha)\) where \(\varphi_\alpha: U_\alpha \to \varphi_\alpha(U_\alpha) \subseteq \C^n\) is a homeomorphism, with
  \begin{enumerate}
  \item \(X = \bigcup_\alpha U_\alpha\),
  \item \(\varphi_\alpha \compose \varphi_\beta^{-1}\) are holomorphic.
  \end{enumerate}
\end{definition}

\begin{definition}[equivalent atlas]\index{atlas!equivalent}
  Two holomorphic atlases \((U_\alpha, \varphi_\alpha), (\tilde U_\beta, \tilde U_\beta)\) are \emph{equivalent} if \(\varphi_\alpha \compose \tilde \varphi_\beta^{-1}\) is holomorphic fo all \(\alpha, \beta\).

  Equivalently, their union is an atlas.
\end{definition}

\begin{definition}[complex manifold, complex structure]\index{complex manifold}\index{complex structure}
  A \emph{complex manifold} is a topological space as above with an equivalence class of holomorphic atlases. Such an equivalence class is called a \emph{complex structure}.
\end{definition}

\begin{eg}\leavevmode
  \begin{enumerate}
  \item \(\C^n\) is trivially a complex manifold.
  \item \(\Delta = \{z: |z| < 1| \subseteq \C\).
  \item \(\P^n\), the (complex) projective space. As a set this is the one-dimensional linear subspaces of \(\C^{n + 1}\). A point is \([z_0: \dots: z_n]\). A holomorphic atlas is given by
    \begin{align*}
      U_i &= \{z_i \neq 0\} \\
      \varphi_i([z_0: \dots:z_n]) &= (\frac{z_1}{z_i}, \dots, \hat{\frac{z_i}{z_i}}, \dots, \frac{z_n}{z_i})
    \end{align*}
    where the hat denotes that the omitted coordinate. One can check that transition functions are holomorphic. Moreover \(\P^n\) is compact.
  \end{enumerate}
\end{eg}

\begin{definition}[holomorphic, biholomorphic]\index{holomorphic}\index{biholomorphic}
  A smooth function \(f: X \to \C\) is \emph{holomorphic} if \(f \compose \varphi^{-1}: \varphi(U) \to \C\) is holomorphic for all \((U, \varphi)\).

  A smooth map \(F: X \to Y\) is \emph{holomorphic} if for all charts \((U, \varphi)\) for \(X\), \((V, \psi)\) for \(Y\), the map \(\psi \compose F \compose \varphi^{-1}\) is holomorphic. \(F\) is \emph{biholomorphic} if it has a holomorpic inverse.
\end{definition}

\begin{ex}
  If \(X\) is compact then any holomorphic function on \(X\) is constant. As a corollary, compact complex manifold cannot embed in \(\C^m\) for any \(m\).
\end{ex}

\begin{ex}
  If \(X \to \C\) is holomorphic and vanishes on an open set on \(X\) then \(f = 0\). Thus there is no holomorphic analogue of bump functions.
\end{ex}

\begin{definition}[closed complex submanifold]\index{closed submanifold}
  Let \(Y \subseteq X\) be a smooth submanifold of dimension \(2k < 2n = \dim X\). We say \(Y\) is a \emph{closed complex submanifold} if there exists a holomorphic atlas \((U_\alpha, \varphi_\alpha)\) for \(X\) such that it restricts to
  \[
    \varphi_\alpha: U_\alpha \cap Y \to \varphi(U_\alpha) \cap \C^k
  \]
  with \(\C^k \subseteq \C^n\) as \((z_1, \dots, z_k, 0, \dots, 0)\).
\end{definition}

\begin{ex}
  Show that a closed complex submanifold is naturally a complex manifold.
\end{ex}

\begin{definition}[projective manifold]\index{projective manifold}
  We say \(X\) is \emph{projective} is it is biholomorphic to a compact closed complex submanifold of \(\P^m\) for some \(m\).
\end{definition}

We state without a theorem:

\begin{theorem}[Chow]
  A projective complex manifold is projective variety, i.e.\ the vanishing set in \(\P^m\) of some homogeneous polynomial equations.
\end{theorem}

In the example sheet we'll see an example of a compact complex manifold which is not projective.

\section{Almost complex structures}

How much complex structure can be recovered from linear data?

Let \(V\) be a real vector space.

\begin{definition}[complex structure]
  A linear map \(J: V \to V\) with \(J^2 = - \id\) is called a \emph{complex structure}.
\end{definition}

This is motivated by the endomorphism on \(\R^{2n}\)
\[
  (x_1, y_1, \dots, x_n, y_n) \mapsto (y_1, -x_1, \dots, y_n, -x_n).
\]
This is called the \emph{standard complex structure}\index{complex structure!standard}.

As \(J^2 = -\id\), the eigenvalues are \(\pm i\). Since \(V\) is real, there are no eigenspaces. Consider \(V_\C = V \otimes_\R \C\). Then \(J\) extends to \(J: V_\C \to V_\C\) with \(J^2 = -\id\). Let \(V^{1, 0}\) and \(V^{0, 1}\) denote the eigenspaces of \(\pm i\) respectively.

\begin{lemma}\leavevmode
  \begin{enumerate}
  \item \(V_\C = V^{1, 0} \oplus V^{0, 1}\).
  \item \(\cl{V^{1, 0}} = V^{0, 1}\).
  \end{enumerate}
\end{lemma}

\begin{proof}\leavevmode
  \begin{enumerate}
  \item For \(v \in V_\C\), write
    \[
      v = \frac{1}{2}\underbrace{(v - iJv)}_{\in V^{1, 0}} + \frac{1}{2}\underbrace{(v + i Jv)}_{\in V^{0, 1}}.
    \]
  \item Follows from 1.
  \end{enumerate}
\end{proof}

\begin{definition}[almost complex structure]\index{almost complex structure}
  Let \(X\) be a smooth manifold. An \emph{almost complex structure} is a bundle isomorphism \(J: TX \to TX\) with \(J^2 = -\id\).
\end{definition}

One can complexify \(TX\) to obtain \((TX)_\C = TX \otimes \C\) so each fibre of \((TX)_\C \to X\) is a complex vector space. \((TX)_\C\) is called the \emph{complexified tangent bundle}\index{complexified tangent bundle}.

Just as the case for complex structure, \((TX)_\C\) splits as a direct sum
\[
  (TX)_\C \cong TX^{1, 0} \oplus TX^{0, 1}.
\]
To obtain this, one uses, for example,
\begin{align*}
  TX^{1, 0} &= \ker (J - i \id) \\
  TX^{0, 1} &= \ker (J + i \id)
\end{align*}

\begin{ex}
  Let \(U, V \subseteq \C^n\) open, \(f: U \to V\) smooth. Then \(f\) is holomorphic if and only if \(df\) is \(\C\)-linear.
\end{ex}

On \(T\R^{2n}\) there is a natural almost complex structure coming from the one on \(\R^{2n}\), denoted \(J_{\text{st}}\). Let \(X\) be a complex manifold. If  \(U \subseteq X\) is a chart with \(\varphi: U \to \varphi(U) \subseteq \C^n \cong \R^{2n}\), the differential of \(\varphi\) gives a bundle map \(J = d\varphi^{-1} \compose J_{\text{st}} \compose d\varphi: TU \to TU\).

\begin{proposition}
  \(J\) defined above is independent of (holomorphic) chart, so gives an almost complex structure on \(X\).
\end{proposition}

\begin{proof}
  Suppose \(\varphi, \psi\) are charts around the same point. What we need to show is
  \[
    d\varphi^{-1} \compose J_{\text{st}} \compose d\varphi = d\psi^{-1} \compose J_{\text{st}} \compose d \psi,
  \]
  i.e.
  \[
    d((\varphi \compose \psi^{-1})^{-1}) \compose J_{\text{st}} \compose d(\varphi \compose \psi^{-1}) = J_{\text{st}}.
  \]
  \(\varphi \compose \psi^{-1}\) is a holomorphic map between open open subsets of \(\C^n\) and so \(d((\varphi \compose \psi^{-1}))\) commutes with \(J_{\text{st}}\), which is similar to the exercise.
\end{proof}

\begin{remark}
  There are lots of almost complex structure not arising in this way. Those that do are called \emph{integrable}\index{integrable}. In general it is difficult to tell whether a smooth manifold with an almost complex structure admits a complex structure. For example \(S^6\) admits an almost complex structure which is \emph{not} integrable. It's an open problem whether or not \(S^6\) admits a complex structure. As an aside, an almost complex structure is integrable if and only if the Nijenhuis tensor vanishes.
\end{remark}

\begin{definition}[holomorphic tangent bundle]\index{holomorphic tangent bundle}
  \(TX^{1, 0}\) is called the \emph{holomorphic tangent bundle} of \(X\).
\end{definition}

If \(V\) is a real vector space and \(J\) is a complex structure then one obtains a complex structure on \(V^*\) in the natural way. Thus analoguously one obtains
\[
  (T^*X)_\C \cong T^*X^{1, 0} \oplus T^*X^{0, 1}.
\]
Locally if \(\varphi: U \to \C^n\) is a chart, we say that \(z_j = x_j + iy_j\) are local coordinates. Then
\begin{align*}
  J(\frac{\partial  }{\partial x_j}) &= \frac{\partial  }{\partial y_j} \\
  J(\frac{\partial  }{\partial y_j}) &= - \frac{\partial  }{\partial x_j}
\end{align*}
(see the connection with Cauchy-Riemann) and
\begin{align*}
  J(\d x_j) &= -\d y_j \\
  J(\d y_j) &= \d x_j
\end{align*}
where we also use \(J\) to denote the dual of \(J\).

\begin{definition}
  We define
  \begin{align*}
    \d z_j &= \d x_j + i\d y_j \\
    \d\conj z_j &= \d x_j - i\d y_j \\
    \frac{\partial  }{\partial z_j} &= \frac{1}{2} \left( \frac{\partial  }{\partial x_j} - i \frac{\partial  }{\partial y_j} \right) \\
    \frac{\partial  }{\partial \conj z_j} &= \frac{1}{2} \left( \frac{\partial  }{\partial x_j} + i \frac{\partial  }{\partial y_j} \right)
  \end{align*}
  Then \(\d z_j, \d\conj z_j\) are sections of \((T^*X)_\C\) and \(\frac{\partial  }{\partial z_j}, \frac{\partial  }{\partial \conj z_j}\) are sections of \((TX)_\C\).
\end{definition}

Note that
\begin{align*}
  J(\d z_j) &= i \d z_j, J(\d\conj z_j) = -i \d\conj z_j \\
  J(\frac{\partial  }{\partial z_j}) &= i \frac{\partial  }{\partial z_j}, J(\frac{\partial  }{\partial \conj z_j}) = -i \frac{\partial  }{\partial \conj z_j}
\end{align*}
We see the \(\d z_j\) form a local frame for \(T^*X^{1, 0}\), similarly \(\d\conj z_j\) form a local frame for \(T^*X^{0, 1}\). Same for tangent bundle.

If \(f: X \to \C\), say \(f = u + iv\) then \(\d f = \d u + i\d v\) is a smooth section of
\[
  (T^*X)_\C \cong T^*X^{1, 0} \oplus T^*X^{0, 1}.
\]
We denote by \(p_1, p_2\) the two projections.

\begin{definition}
  Define
  \begin{align*}
    \p f &= p_1 (\d f) \\
    \conj \p f&= p_2 (\d f)
  \end{align*}
\end{definition}

In a local frame,
\[
  \d f
  = \sum \frac{\partial f}{\partial z_j} \d z_j + \sum \frac{\partial f}{\partial \conj z_j} \d\conj z_j
  = \p f + \conj \p f
\]
so \(f\) is holomorphic if and only if \(\conj \p f = 0\).

We now do the same for higher degree forms.

\begin{definition}[form]\index{form}
  A section of
  \[
    \Lambda^{p, q}T^*X = \Lambda^pT^*X^{1, 0} \otimes \Lambda^qT^*X^{0, 1}.
  \]
  is called a \emph{\((p, q)\)-form}.
\end{definition}

Locally a \((p, q)\)-form looks like
\[
  \sum f \d z_{j_1} \w \cdots \w \d z_{j_p} \w \d\conj z_{\ell_1} \w \cdots \w \d\conj z_{\ell_q}.
\]
Note that \(f\) is only required to be smooth and not required to be either holomorphic or antiholomorphic. For example \(\conj z \d z\) is a section of \(T^*X^{1, 0}\).

\begin{definition}
  We denote by \(\mathcal A_\C^k(U)\) the sections of \(\Lambda^k(T^*X)_\C\) over \(U \subseteq X\). We also denote by \(\mathcal A_\C^{p, q}(U)\) the smooth sections of \(\Lambda^{p, q}(U)\).
\end{definition}
In particular \(\mathcal A_\C^{0, 0}(U)\) consists of smooth \(\C\)-valued functions.

\begin{lemma}\leavevmode
  \begin{enumerate}
  \item There is a natural identification
    \[
      \Lambda^k(T^*X)_\C \cong \bigoplus_{p + q = k} \Lambda^{p, q} (T^*X)
    \]
    so
    \[
      \mathcal A_\C^k(U) \cong \bigoplus_{p + q = k} \mathcal A_\C^{p, q} (U).
    \]
  \item If \(\alpha \in \mathcal A_\C^{p, q}(U), \beta \in \mathcal A_\C^{p', q'}(U)\) then \(\alpha \w \beta \in \mathcal A_\C^{p + p', q + q'}(U)\).
  \end{enumerate}
\end{lemma}

\begin{proof}
  Fibrewise this follows from linear algebra. One can use a frame to obtain the bundle results.
\end{proof}

\subsection{Dolbeault cohomology}

Denote by \(\d: \mathcal A_\C^k(U) \to \mathcal A_\C^{k + 1}(U)\) the usual exterior derivative.

\begin{definition}
  \(\p: \mathcal A_\C^{p, q}(U) \to \mathcal A_\C^{p + 1, q}(U)\) is defined as \(\d\) composed with projection to \(\mathcal A_\C^{p + 1, q}(U)\). Similarly define \(\conj \p: \mathcal A_\C^{p, q}(U) \to \mathcal A_\C^{p, q + 1}(U)\).
\end{definition}

Locally if
\[
  \alpha = \sum f \d z_I \w \d \conj z_J
\]
then
\[
  \d \alpha = \underbrace{\sum \sum_r \frac{\partial f}{\partial z_r} \d z_r \w \d z_I \w \d \conj z_J}_{\p \alpha} + \underbrace{\sum \sum_r \frac{\partial f}{\partial \conj z_r} \d \conj z_r \w \d z_I \w \d \conj z_J}_{\conj \p \alpha}.
\]

\begin{lemma}\leavevmode
  \begin{enumerate}
  \item \(\d = \p + \conj \p\).
  \item \(\p^2 = 0, \conj \p^2 = 0, \p \conj \p = - \conj \p \p\).
  \item If \(\alpha \in \mathcal A_\C^{p, q}(U), \beta \in \mathcal A_\C^{p', q'}(U)\) then
    \begin{align*}
      \p(\alpha \w \beta) &= \p \alpha \w \beta + (-1)^{p + q} \alpha \w \p \beta \\
      \conj \p(\alpha \w \beta) &= \conj \p \alpha \w \beta + (-1)^{p + q} \alpha \w \conj \p \beta \\
    \end{align*}
  \end{enumerate}
\end{lemma}

\begin{proof}\leavevmode
  \begin{enumerate}
  \item Follows from local expression.
  \item Follows from \(\d^2 = 0\).
  \item Follows from
    \[
      \d(\alpha \w \beta) = \d\alpha \w \beta + (-1)^{p + q} \alpha \w \d \beta.
    \]
  \end{enumerate}
\end{proof}

\begin{definition}[Dolbeault cohomology]\index{Dolbeault cohomology}
  The \emph{\((p, q)\)-Dolbeault cohomology} of \(X\) is given by
  \[
    H_{\conj \p}^{p, q}(X) = \frac{\ker \conj \p: \mathcal A_\C^{p, q}(X) \to \mathcal A_\C^{p, q + 1}(X)}{\im \conj \p: \mathcal A_\C^{p, q - 1}(X) \to \mathcal A_\C^{p, q}(X)}
  \]
  which makes sense as \(\conj \p^2 = 0\). These are vector spaces.
\end{definition}

\begin{remark}
  One could make an analogous definition using \(\p\) and the information would be equivalent. Historically, people are interested in holomorphic functions, i.e.\ \(f\) with \(\conj \p f = 0\).
\end{remark}

Recall the de Rham cohomology group\index{de Rham cohomology}
\[
  H_{\text{dR}}^i(X; \R) = \frac{\ker (\d: \mathcal A_\R^i(X) \to \mathcal A_\R^{i + 1}(X))}{\im (\d: \mathcal A_\R^{i - 1}(X) \to \mathcal A_\R^i(X))}.
\]
One similarly defines
\[
  H_{\text{dR}}^i(X; \C) = \frac{\ker (\d: \mathcal A_\C^i(X) \to \mathcal A_\C^{i + 1}(X))}{\im (\d: \mathcal A_\C^{i - 1}(X) \to \mathcal A_\C^i(X))}
  \cong H_{\text{dR}}^i(X; \R) \otimes \C
\]
so we do not gain or lose anything.

Much of the course will be devoted to prove Hodge decomposition, which asserts that for a certain class of compact manifolds (which include projective varieties),
\[
  H_{\text{dR}}^k(X; \C) \cong \bigoplus_{p + q = k} H_{\conj \p}^{p, q}(X).
\]
Note that the statement alone is not true in general.

\begin{ex}
  If \(F: X \to Y\) is holomorphic then \(F\) induces a map
  \[
    F^*: H_{\conj \p}^{p, q}(Y) \to H_{\conj \p}^{p, q}(X)
  \]
  via pullback.
\end{ex}

The Mittag-Leffler problem:  let \(S\) be a Riemann surface, i.e.\ one dimensional complex manifold. A \emph{principal part} at \(x \in S\) is a Laurent series of the form
\[
  \sum_{k = 1}^n a_k z^{-k}
\]
with \(z\) a local coordinate. Then the Mittag-Leffler problem\index{Mittag-Leffler problem} asks given \(x_1, \dots, x_r \in S\) and principal parts \(P_1, \dots, P_r\), is there a meromorphic function on \(S\) with these principal parts at \(x_i\)'s?

Take local solutions \(f_i\) at \(x_i\), defined on some \(U_i\) which form a cover of \(S\)  and a partition of unity \(\rho_i\) subordinate to the \(U_i\). Then \(\sum_{j = 1}^r \rho_j f_j\) is smooth on \(S \setminus \{x_1, \dots, x_r\}\) with described local expression (which is not necessarily holomorphic).

A calculation shows that \(g = \conj \p (\sum_j \rho_j f_j)\) extends to a smooth \((0, 1)\)-form on \(S\). Clearly \(\conj \p g = 0\) as \(\conj \p^2 = 0\) so \([g] \in H_{\conj \p}^{0, 1}(S)\). Suppose \(H_{\conj \p}^{0, 1}(S) = 0\). Then there is a smooth function \(h\) with \(\conj \p h = g\) and \(f = \sum_j \rho_j f_j - h\) solves the Mittag-Leffler problem. This can be shown to be an if and only if using sheaf cohomology.

\subsection{Del bar-Poincaré lemma}
% TODO: resolve the title name

Recall that if \(X\) is a contractible smooth manifold then
\[
  H_{\text{dR}}^i (X; \R) = 0
\]
for \(i > 0\). We'll prove the analogous result for Dolbeault cohomology: a polydisk is
\[
  P = \{|z_i| < r_i\} \subseteq \C^n
\]
(with \(r = \infty\) allowed). Have
\[
  H_{\conj \p}^{p, q}(P) = 0
\]
if \(p + q > 0\).

\begin{proposition}
  Let \(D = D(a, r) \subsetneq \C\) be a disk, \(f \in C^\infty(\cl D), z \in D\). Then
  \[
    f(z) = \frac{1}{2\pi i} \int_{\p D}\frac{f(w)}{w - z} \d w + \frac{1}{2\pi i} \int_D \frac{\p f(w)}{\p \conj w} \frac{\d w \w \d \conj w}{w - z}.
  \]
\end{proposition}
This is a generalisation of Cauchy integral formula, with a correction term for non-holomorphic component.

\begin{proof}
  Let \(D_\varepsilon = D(z, \varepsilon)\) and
  \[
    \eta = \frac{1}{2\pi i} \frac{f(w)}{w - z} \d w \in \mathcal A_\C^1 (D \setminus D_\varepsilon).
  \]
  Then
  \[
    \d\eta
    = \conj \p \eta
    = -\frac{1}{2\pi i} \frac{\p f(w)}{\p \conj w} \frac{\d w \w \d \conj w}{w - z}
  \]
  so by Stokes',
  \[
    \frac{1}{2\pi i} \int_{\p D_\varepsilon} \frac{f(w)}{w - z} \d w
    = \frac{1}{2\pi i} \int_{\p D} \frac{f(w)}{w - z} \d w + \frac{1}{2\pi i} \int_{D \setminus D_\varepsilon} \frac{\p f(w)}{\p \conj w} \frac{\d w \w \d \conj w}{w - z}
  \]
  The first term converges to \(f(z)\) as \(\varepsilon \to 0\): set \(w - z = re^{i\theta}\) so
  \[
    \frac{1}{2\pi i} \int_{\p D_\varepsilon} \frac{f(w)}{w - z} \d w
    = \frac{1}{2\pi} \int_0^{2\pi} f(z + re^{i\theta}) \d\theta
  \]
  which goes to \(f(z)\) as \(\varepsilon \to 0\) since \(f\) is smooth.

  As \(\d w \w \d \conj w = 2ir \d r \w \d \theta\),
  \[
    \left| \frac{\partial f(w)}{\partial \conj w}  \frac{\d w \w \d \conj w}{w - z} \right|
    = 2 \left| \frac{\partial f}{\partial \conj w} \d r \w \d\theta \right|
    \leq C |\d r \w \d\theta|
  \]
  so
  \[
    \int_{D_\varepsilon} \frac{\partial f(w)}{\partial \conj w} \frac{\d w \w \d \conj w}{w - z}
    \to 0
  \]
  as \(\varepsilon \to 0\).
\end{proof}

\begin{theorem}[\(\conj \p\)-Poincaré lemma in one variable]
  Let \(D = D(a, r)\) be a disk (\(r < \infty\)) and let \(g \in C^\infty(\cl D)\). Then
  \[
    f(z) = \frac{1}{2\pi i} \int_D \frac{g(w)}{w - z} \d w \w \d \conj w \in C^\infty(D)
  \]
  and
  \[
    \frac{\p f(z)}{\p \conj z} = g(z).
  \]
\end{theorem}

\begin{proof}
  First reduce to the case \(g\) has compact support. Take \(z_0 \in D\) and \(\varepsilon > 0\) such that
  \[
    D_{2\varepsilon} = D(z_0, 2\varepsilon) \subsetneq D.
  \]
  Using a partition of unity for the cover of \(D\) given by \(\{D \setminus D_\varepsilon, D_{2\varepsilon}\}\), write
  \[
    g(z) = g_1(z) + g_2(z)
  \]
  where \(g_1\) vanishes outside \(D_{2\varepsilon}\) and \(g_2\) vanishes on \(D_\varepsilon\).

  Define
  \[
    f_2(z) = \int_D \frac{g_2(w)}{w - z} \d w \w \d \conj w.
  \]
  Then \(f_2(z)\) is smooth on \(D\) as \(g_2\) vanishes on \(D_\varepsilon\). Differentiate under the integral sign, get
  \[
    \frac{\partial f_2(z)}{\partial \conj z} = \frac{1}{2\pi i} \int_D \frac{\partial  }{\partial \conj z} \frac{g_2(w)}{w - z} \d w \w \d \conj w.
  \]
  As \(g_1(z)\) has compact support we can write
  \begin{align*}
    \frac{1}{2\pi i} \int_D \frac{g_1(w)}{w - z} \d w \w \d\conj w
    &= \frac{1}{2\pi i} \int_\C \frac{g_1(w)}{w - z} \d w \w \d \conj w \\
    &= \frac{1}{2\pi i} \int_\C \frac{g_1(u + z)}{u} \d u \w \d \conj u \\
    &= -\frac{1}{\pi} \int_\C g_1(z + re^{i\theta}) e^{-i\theta} \d r \w \d\theta \in C^\infty(D)
  \end{align*}
  Define this to be \(f_1\). The trick here is that we defined \(f\) in this way so that it is automatically smooth. Then
  \[
    \frac{\partial f_1(z)}{\partial \conj z}
    = \int_\C \frac{\partial g_1(w)}{\partial \conj w} \frac{\d w \w \d \conj w}{w - z}
  \]
  so by Cauchy integral,
  \[
    g_1(z)
    = \frac{1}{2\pi i} \underbrace{\int_{\p D} \frac{g_1(w)}{w - z}\d w}_{= 0 \text{ as } g_1 = 0 \text{ on } \p D} + \frac{1}{2\pi i} \int_D \frac{\partial g_1(w)}{\partial \conj w} \frac{\d w \w \d\conj w}{w - z}
    = \frac{\partial f_1(z)}{\partial \conj z}
  \]
  Setting \(f = f_1 + f_2\) gives
  \[
    \frac{\partial f}{\partial \conj z}(z) = g(z)
  \]
  for \(z \in D_\varepsilon\). But \(z_0\) was arbitrary so this works for all \(z_0\).
\end{proof}

In other words, if \(\alpha = g \d \conj z \in \mathcal A_\C^{0, 1}(D)\) and \(f\) is as above, then
\[
  \conj \p f = \alpha.
\]

For the general \(\conj \p\)-Poincaré lemma, we shall use \emph{multiindex notation}: if \(I = (I_1, \dots, I_k)\) then
\begin{align*}
  \d z_I &= \d z_{I_1} \w \dots \w \d z_{I_k} \\
  f_I &= f_{I_1 \dots I_k} \\
  \frac{\partial  }{\partial z_I} &= \frac{\partial }{\partial z_{I_1} \dots \partial z_{I_k}}
\end{align*}
and \(|I| = k\).

At some point we are going to extend the result to \(\C^n\) by taking a sequence of holomorphic functions. The following result justifies the process:

\begin{lemma}
  Let \(U \subseteq \C^n\) be open, \(B \subsetneq B' \subseteq U\) where \(B, B'\) are bounded polydisks. Then for any multiindices \(I, J\) there is a constant \(c_{I, J}\) such that for all \(u\) holomorphic on \(U\), we have
  \[
    \norm*{\frac{\partial  }{\partial z_I} \frac{\partial  }{\partial z_J} u}_{C^0(B)} \leq c_{I, J} \norm u_{C^0(B)}.
  \]
\end{lemma}

\begin{proof}
  Follows from multivariable Cauchy integral formula, which follows from the single variable version.
\end{proof}

\begin{corollary}
  Let \(u_k\) be a sequence of holomorphic functions on \(U\) with \(u_k \to u\) uniformly on compact subsets of \(U\). Then \(u\) is holomorphic.
\end{corollary}

\begin{proof}
  By the previous lemma, \(u\) is smooth. Moreover \(\frac{\partial u_k}{\partial \conj z_j} \to \frac{\partial u}{\partial \conj z_j}\) so since \(\frac{\partial u_k}{\partial \conj z_j} = 0\), \(\conj \p u = 0\) so \(u\) is holomorphic.
\end{proof}

Then we have the following result due to Grothendieck:

\begin{theorem}[\(\conj \p\)-Poincaré lemma]\index{\(\conj \p\)-Poincaré lemma}
  Let
  \[
    P = P(a, r) = \{|z_i - a_i| < r_i\} \subseteq \C^n
  \]
  with \(r_i \in (0, \infty]\). Then for all \(q > 0\) we have
  \[
    H_{\conj \p}^{p, q}(P) = 0.
  \]
  That is, if \(\conj \p \omega = 0\) then exists \(\psi\) with \(\conj \p \psi = \omega\).
\end{theorem}

\begin{proof}
  We first reduce to \(p = 0\). Indeed if \(\omega \in \mathcal A_\C^{p, q}(P)\) is closed then \(\conj \p \omega = 0\) so we may write
  \[
    \omega = \sum_{|I| = p} \varphi_I \d z_I
  \]
  where \(\conj \p \varphi_I = 0\). Hence by induction we can find \(\psi_I\) with \(\conj \p \psi_I = \varphi_I\) and then
  \[
    \conj \p \left(\sum_{|I| = p} \psi_I \w \d z_I\right) = \omega.
  \]
  Thus we may assume \(p = 0\). The proof is in two steps.

  \paragraph{Step 1}

  Given \(\omega \in \mathcal A_\C^{0, q}(P)\), we show that if \(P' = P(a, s)\) with \(s_i' < r_i\) finite then we can find \(\psi \in \mathcal A_\C^{0, q - 1}(P')\) with \(\conj \p \psi = \omega|_{P'}\).

  Given a form
  \[
    \omega = \sum_{|I| = q} \omega_I \d  \conj z_I,
  \]
  we say
  \[
    \omega = 0 \mod \{\d \conj z_1, \dots, \d \conj z_k\}
  \]
  if \(\omega_I = 0\) unless \(I \subseteq \{1, \dots, k\}\). We shall prove that if \(\omega = 0 \mod \{\d \conj z_1, \dots, \d \conj z_k\}\) then there is \(\psi \in \mathcal A_\C^{0, q - 1}(P')\) such that \(\omega - \conj \p \psi = 0 \mod \{\d \conj z_1, \dots, \d \conj z_{k - 1}\}\). By induction and \(k = n\) being vacuous, this will prove step 1.

  So suppose \(\omega = 0 \mod \{\d \conj z_1, \dots, \d \conj z_k\}\) and write
  \[
    \omega = \omega_1 \w \d \conj z_k + \omega_2
  \]
  with \(\omega_2 = 0 \mod \{\d \conj z_1, \dots, \d \conj z_{k - 1}\}\). Have
  \[
    \omega_1 = \sum_{|I| = q, k \in I} \omega_I \d  \conj z_{I \setminus \{k\}}.
  \]
  Since \(\conj \p \omega = 0\), we have
  \[
    \frac{\partial \omega_I}{\partial \conj z_\ell} = 0
  \]
  for \(\ell > k\). Set
  \[
    \psi = \sum_{|I| = q, k \in I} (-1)^{k - 1} \psi_I \d \conj z_{I \setminus \{k\}}
  \]
  where
  \[
    \psi_I = \frac{1}{2\pi i} \int_{|\xi| \leq s_k} \omega_i(z_1, \dots, z_{k - 1}, \xi, z_{k + 1}, \dots, z_n) \frac{\d  \xi \w \d  \conj \xi}{\xi - z_k}
  \]
  is given by Cauchy integral formula. Then
  \[
    \frac{\partial \psi_I}{\partial \conj z_k} = \omega_I
  \]
  by \(\conj \p\)-Poincaré in one variable and
 \[
   \frac{\partial \psi_I}{\partial \conj z_k} = \frac{1}{2\pi i} \int_{|\xi| \leq s_k} \frac{\partial \omega_I}{\partial \conj z_j} (z_1, \dots, z_{k - 1}, \xi, z_{k + 1}, \dots, z_n) \frac{\d  \xi \w \d  \conj \xi}{\xi - z_k}
   = 0
 \]
 by assumption. Hence \(\omega - \conj \p \psi = 0 \mod \d \conj z_1, \dots, \d \conj z_{k - 1}\).
  
  \paragraph{Step 2}
  Let \(r_{j, k}\) be a strictly increasing sequence, \(r_{j, k} \to r_k\) as \(j \to \infty\) for all \(k = 1, \dots, n\) and let \(P_j = P(a, r_j)\). By step 1 we can find \(\psi_j \in \mathcal A_\C^{0, q - 1}(P_j)\) with \(\conj \p \psi_j = \omega\) on \(P_j\).

  We induct on \(q\), leaving \(q = 1\) for last. Since \(\conj \p(\psi_j - \psi_{j + 1}) = 0\) on \(P_j\), we can choose \(\beta_{j + 1}\) with
  \[
    \psi_j - \psi_{j + 1} = \conj \p \beta_{j + 1}
  \]
  on \(P_{j - 1}\). Extend \(\psi_{j + 1}, \beta_{j + 1}\) smoothly to \(P\) and set
  \[
    \phi_{j + 1} = \psi_{j + 1} + \conj \p \beta_{j + 1}.
  \]
  This produces a sequence \((\phi_j)\) such that
  \begin{align*}
    \conj \p \phi_{j + 1} &= \omega \text{ on } P_{j + 1} \\
    \phi_{j + 1} &= \phi_j \text{ on } P_{j - 1}
  \end{align*}
  Thus the \((\phi_j)\) converges to \(\phi\) on \(P\) for \(\phi\) such that \(\conj \p \phi = \omega\).

  Now consider the case \(\omega\) is a \((0, 1)\)-form, so \(\psi_j\)'s are functions. We construct a sequence \(\phi_j\) on \(P_j\) such that
  \begin{align*}
    \conj \p \phi_j &= \omega \text{ on } P_j \\
    \phi_{j + 1} &- \phi_j \text{ holomorphic on } P_j \\
    \norm{\phi_{j + 1} - \phi_j}_{C^0(P_{j + 1})} &< 2^{-j}
  \end{align*}
  Assuming this, the \((\phi_j)\) converges uniformly to some \(\phi\) on \(P\). Moreover \(\phi - \phi_j\) is holomorphic on \(P_j\) as as a uniform limit of \((\phi_{j + 1} - \phi_j)\), all holomorphic following the corollary. So \(\conj \p \phi = \conj \p \phi_j = \omega\) on \(P_j\), Hence \(\conj \p \phi = \omega\) on \(P\).

  We now construct \((\phi_j)\). Solve \(\conj \p \psi_j = \omega\) on \(P_j\) as before and set \(\phi_1 = \psi_1\). We construct \(\phi_{j + 1}\), inducting on \(j\). Since \(\conj \p(\phi_j - \psi_{j + 1}) = 0\) on \(P_j\), \(\theta_j - \psi_{j + 1}\) is holomorphic on \(P_j\). Hence it has a Taylor series expansion valid on \(P_j\). Truncating gives a polynomial \(\gamma_{j + 1}\) such that
  \[
    \norm{\phi_j - \psi_{j + 1} - \gamma_{j + 1}}_{C^0(P_j - 1)} < 2^{-j}
  \]
  idea: approximate holomorphic by polynomial to arbitrary small error and extend the polynomial to the entire disk.

  Extend \(\gamma_j\) holomorphically to \(P_j\) and set
  \[
    \phi_{j + 1} = \psi_{j + 1} + \gamma_{j + 1}.
  \]
  Then \(\conj \p \phi_{j + 1} = \omega\) on \(P_{j + 1}\), \(\phi_{j + 1} - \phi_j\) holomorphic on \(P_j\) and \(\norm{\phi_{j + 1} - \phi_j}_{C^0(P_{j - 1})} < 2^{-j}\). This ends the proof.
\end{proof}

\section{Sheaves and cohomology}

\subsection{Definitions}

We now compare Dolbeault cohomology with sheaf cohomology. Let's begin with general theory of sheaves. Let \(X\) be a topological space.

\begin{definition}[presheaf]\index{presheaf}
  A \emph{presheaf} \(\mathcal F\) on \(X\) of abelian groups consists of abelian groups \(\mathcal F(U)\) for all \(U \subseteq X\) open and \emph{restriction homomorphisms}
  \[
    r_{VU}: \mathcal F(U) \to \mathcal F(V)
  \]
  for all \(V \subseteq U\) open with
  \begin{align*}
    r_{WV} \compose r_{VU} &= r_{WU} \\
    r_{UU} &= \id
  \end{align*}
  One similarly defines presheaves of vector spaces.
\end{definition}

Most often \(\mathcal F(U)\) is some class of functions on \(U\) with restrictions given by restricting the functions, which we simply write \(r_{VU}(s) = s|_V\). Another frequent example is given by \(\mathcal F(U)\) consisting of sections of vector bundles. We call elements of \(\mathcal F(U)\) \emph{sections}\index{section}.

\begin{definition}[sheaf]\index{sheaf}
  A presheaf \(\mathcal F\) on \(X\) is a \emph{sheaf} if in addition
  \begin{enumerate}
  \item for all \(s \in \mathcal F(U)\), if \(U = \bigcup U_i\) is an open cover and \(s|_{U_i} = 0\) for all \(i\) then \(s = 0\).
  \item If \(U = \bigcup U_i\), \(s_i \in \mathcal F(U_i)\) with
    \[
      s_i|_{U_i \cap U_j} = s_j|_{U_i \cap U_j}
    \]
    then there exists \(s \in \mathcal F(U)\) with \(s|_{U_i} = s_i\).
  \end{enumerate}
\end{definition}

\begin{eg}
  The following are sheaves on complex manifolds:
  \begin{enumerate}
  \item \(C^0(U)\): continuous functions on \(U\).
  \item \(C^\infty(U)\): smooth functions on \(U\).
  \item \(\mathcal A_\C^{p, q}(U)\): \((p, q)\)-forms on \(U\).
  \item \(\mathcal O(U)\): holomorphic functions on \(U\).
  \item \(\mathcal O^*(U)\): nowhere vanishing holomorphic functions on \(U\).
  \item \(\Omega^p(U)\): holomorphic \(p\)-forms on \(U\), which are defined to be sections \(s \in \mathcal A_\C^{p, 0}(U)\) with \(\conj \p s = 0\).
  \end{enumerate}
\end{eg}

\begin{definition}[morphism of (pre)sheaves]\index{sheaf!morphism}\index{sheaf!isomorphism}
  A \emph{morphism} \(\alpha: \mathcal F \to \mathcal G\) of (pre)sheaves on \(X\) consists of homomorphisms \(\alpha_U: \mathcal F(U) \to \mathcal G(U)\) for all \(U \subseteq X\) open such that if \(V \subseteq U\) open then the diagram
  \[
    \begin{tikzcd}
      \mathcal F(U) \ar[r, "\alpha_U"] \ar[d, "r_{VU}"] & \mathcal G(U) \ar[d, "r_{VU}"] \\
      \mathcal F(V) \ar[r, "\alpha_V"] & \mathcal G(V)
    \end{tikzcd}
  \]
  commutes.

  \(\alpha\) is an \emph{isomorphism} if \(\alpha|_U: \mathcal F(U) \to \mathcal G(U)\) is an isomorphism for all \(U \subseteq X\) open.
\end{definition}

\begin{definition}[short exact sequence of sheaves]\index{sheaf!short exact sequence}
  We say that
  \[
    \begin{tikzcd}
      0 \ar[r] & \mathcal F \ar[r, "\alpha"] & \mathcal G \ar[r, "\beta"] & \mathcal H \ar[r] & 0
    \end{tikzcd}
  \]
  is a \emph{short exact sequence} if for all \(U\) the sequence
  \[
    \begin{tikzcd}
      0 \ar[r] & \mathcal F(U) \ar[r, "\alpha_U"] & \mathcal G(U) \ar[r, "\beta_U"] & \mathcal H(U)
    \end{tikzcd}
  \]
  is exact and if \(s \in \mathcal H(U)\) and \(x \in U\) then there exists a neighbourhood \(V\) of \(x\) and \(t \in \mathcal G(V)\) with \(\beta_V(t) = s|_V\).
\end{definition}

\begin{eg}
  The sequence
  \[
    \begin{tikzcd}
      0 \ar[r] & \Z \ar[r, "\times 2\pi i"] & \mathcal O \ar[r, "\exp"] & \mathcal O^* \ar[r] & 0
    \end{tikzcd}
  \]
  is exact. It is called the \emph{exponential short exact squence}\index{exponential short exact sequence}\index{sheaf!short exact sequence!exponential}. Here \(\Z\) is the \emph{constant sheaf}\index{sheaf!constant}: \(\Z(U)\) is the space of continuous functions \(U \to \Z\), i.e.\ \(\Z\)-valued locally constant functions (similarly we define the sheaf \(\C(U)\) to be continuous functions \(U \to \C\) with \(\C\) given the discrete topology).

  The exactness of
  \[
    \begin{tikzcd}
      0 \ar[r] & \Z(U) \ar[r, "\times 2\pi i"] & \mathcal O(U) \ar[r, "\exp"] & \mathcal O^*(U)
    \end{tikzcd}
  \] 
  is clear. If \(f \in \mathcal O^*(U)\) then one can take a local branch of \(\log\) on some \(V \subseteq U\) to obtain the last condition.

  The moral is, we can have local but not global inverse in complex geometry. For example it is not true that
  \[
    \begin{tikzcd}
      0 \ar[r] & \Z(\Delta^*) \ar[r, "\times 2\pi i"] & \mathcal O(\Delta^*) \ar[r, "\exp"] & \mathcal O^*(\Delta^*) \ar[r] & 0
    \end{tikzcd}
  \] 
  is exact where \(\Delta^*\) is the punctured disk.
\end{eg}

\begin{definition}[stalk]\index{stalk}
  Let \(\mathcal F\) be a sheaf on \(X\) and \(x \in X\). The \emph{stalk} of \(\mathcal F\) at \(x\) is
  \[
    \mathcal F_x = \{(U, s): x \in U \subseteq X, s \in \mathcal F(U)\}/\sim
  \]
  where \((U, s) \sim (V, t)\) if there is \(W \subseteq U \cap V\) with \(s|_W = t|_W\).
\end{definition}

A morphism \(\mathcal F \to \mathcal G\) induces a map \(\mathcal F_x \to \mathcal G_x\).

\begin{ex}
  Show
  \[
    \begin{tikzcd}
      0 \ar[r] & \mathcal F \ar[r, "\alpha"] & \mathcal G \ar[r, "\beta"] & \mathcal H \ar[r] & 0
    \end{tikzcd}
  \]
  is exact if and only if
  \[
    \begin{tikzcd}
      0 \ar[r] & \mathcal F_x \ar[r, "\alpha"] & \mathcal G_x \ar[r, "\beta"] & \mathcal H_x \ar[r] & 0
    \end{tikzcd}
  \]
  is exact for all \(x \in X\).
\end{ex}

\begin{definition}[kernel of sheaf morphism]\index{kernel}
  The \emph{kernel} of \(\alpha: \mathcal F \to \mathcal G\) is the sheaf defined by
  \[
    \ker \alpha(U) = \ker (\alpha_U: \mathcal F(U) \to \mathcal G(U)).
  \]
\end{definition}

The definitions of cokernel and image are more complicated. See example sheet.

\subsection{Čech cohomology}

Our aim is to define \(\check H(X, \mathcal F)\) where \(\mathcal F\) is a sheaf on \(X\), and show
\[
  H_{\conj \p}^{p, q}(X) \cong \check H^q(X, \Omega^p).
\]
We begin with an example. Let \(X\) be a topological space with \(X = U \cup V\) where \(U, V\) open. If \(s_U \in \mathcal F(U), s_V \in \mathcal F(V)\), when is there \(s \in \mathcal F(X)\) with \(s|_U = s_U, s|_V = s_V\)?

As \(\mathcal F\) is a sheaf, this happens if and only if
\[
  s_U|_{U \cap V} = s_V|_{U \cap V}.
\]
Define
\begin{align*}
  \delta: \mathcal F(U) \oplus \mathcal F(V) &\to \mathcal F(U \cap V) \\
  (s_U, s_V) &\mapsto s_U|_{U \cap V} - s_V|_{U \cap V}
\end{align*}
then \(\mathcal F(X) \cong \ker \delta\).

\begin{notation}
  Notation: if \(\mathcal U = \{U_\alpha\}_\alpha\) is a locally finite open cover indexed by a subset of \(\N\) (or any ordered set), we write
  \[
    U_{\alpha_0} \cap \dots \cap U_{\alpha_p} = U_{\alpha_0 \dots \alpha_p}.
  \]
\end{notation}

Define
\begin{align*}
  C^0(\mathcal U, \mathcal F) &= \prod_\alpha \mathcal F(U_\alpha) \\
  C^1(\mathcal U, \mathcal F) &= \prod_{\alpha < \beta} \mathcal F(U_{\alpha\beta}) \\
  C^p(\mathcal U, \mathcal F) &= \prod_{\alpha_0 < \dots < \alpha_p} \mathcal F(U_{\alpha_0 \dots \alpha_p}) \\
\end{align*}
If \(\sigma \in C^p(\mathcal U, \mathcal F)\), we also set
\[
  \sigma_{\alpha_0 \dots \alpha_i \alpha_{i + 1} \dots \alpha_p} = - \sigma_{\alpha_0 \dots \alpha_{i + 1} \alpha_i \dots \alpha_p}.
\]
We define the boundary map
\begin{align*}
  \delta: C^p(\mathcal U, \mathcal F) &\to C^{p + 1}(\mathcal U, \mathcal F)
\end{align*}
by
\[
  (\delta\sigma)_{\alpha_0 \dots \alpha_{p + 1}} = \sum_{j = 0}^{p + 1} (-1)^j \sigma_{\alpha_0 \dots \hat \alpha_j \dots \alpha_{p + 1}} |_{U_{\alpha_0 \dots \alpha_{p + 1}}}.
\]

\begin{eg}
  Let \(\mathcal U = \{U_0, U_1, U_2\}\), \(\sigma = \{\sigma_0, \sigma_1, \sigma_2\} \in C^0(\mathcal U, \mathcal F)\). Then \(\delta \sigma\) is given by
  \begin{align*}
    (\delta \sigma)_{01} &= (\sigma_0 - \sigma_1)|_{U_{01}} \\
    (\delta \sigma)_{02} &= (\sigma_0 - \sigma_2)|_{U_{02}} \\
    (\delta \sigma)_{12} &= (\sigma_1 - \sigma_2)|_{U_{12}}
  \end{align*}
  and thus
  \begin{align*}
    \delta \delta \sigma
    &= (\delta \sigma)|_{12} - (\delta \sigma)|_{02} + (\delta \sigma)|_{01} \\
    &= (\sigma_1 - \sigma_2) + (\sigma_0 - \sigma_2) + (\sigma_0 - \sigma_1) \\
    &= 0
  \end{align*}
  which is defined on \(U_{012}\).
\end{eg}

\begin{ex}
  Show \(\delta \compose \delta = 0\) in general.
\end{ex}

\begin{definition}
  Let \(X\) be a topological space and \(\mathcal U\) be a locally finite open cover of \(X\). Let \(\mathcal F\) be a sheaf on \(X\). Define cohomology groups
  \[
    \check H^q(\mathcal U, \mathcal F)
    = \frac{\ker (\delta: C^q(\mathcal U, \mathcal F) \to C^{q + 1}(\mathcal U, \mathcal F))}{\im (\delta: C^{q - 1}(\mathcal U, \mathcal F) \to C^q(\mathcal U, \mathcal F))}
  \]
\end{definition}

\begin{eg}
  Let \(X = \P^1\) with homogeneous coordinates \([z : w]\). Let
  \begin{align*}
    U &= \{[z, 1]: z \in \C\} = \{w \neq 0\} \\
    V &= \{[1: w]: w \in \C\} = \{z \neq 0\}
  \end{align*}
  Then \(U \cong \C, V \cong \C, U \cap V \cong \C^*\). Let \(\mathcal U = \{U, V\}\), with ordering \(U \leq V\). Then
  \begin{align*}
    C^0(\mathcal U, \mathcal O) &= \mathcal O(U) \oplus \mathcal O(V) \\
    C^1(\mathcal U, \mathcal O) &= \mathcal O(U \cap V)
  \end{align*}
  and
  \begin{align*}
    \delta: C^0(\mathcal U, \mathcal O) &\to C^1(\mathcal U, \mathcal O) \\
    (f, g) &\mapsto (z \mapsto f(z) - g(1/z))
  \end{align*}
  so \(\ker \delta\) consists of \((f, g)\) such that \(f = g\) constant: by writing
  \begin{align*}
    f(z) &= \sum_{n = 0}^\infty a_n z^n \\
    g(1/z) &= \sum_{n = 0}^\infty b_n (1/z)^n = \sum_{n = 0}^\infty b_n z^{-n}
  \end{align*}
  it follows that \(a_0 = b_0\) and \(a_i = b_i = 0\) for \(i > 0\).
  \(\im \delta\) consists of all holomorphic functions on \(\C^*\), again by a Laurent series argument. Thus
  \begin{align*}
    \check H^0(\mathcal U, \mathcal O) &= \C \\
    \check H^1(\mathcal U, \mathcal O) &= 0 \text{ for all } i > 0
  \end{align*}
  We'll see that this computes Čech cohomology \(H^i(\P^1, \mathcal O)\), which we will define later.
\end{eg}

However, this definition is dependent on the choice of cover. We now take the direct limit of these cohomology groups with respect to cover refinement.

\begin{definition}[refinement of cover]\index{refinement}
  Given open covers \(\mathcal U, \mathcal V\), we say \(\mathcal V\) \emph{refines} \(\mathcal U\) if there exists \(\varphi: \N \to \N\) increasing such that for all \(\beta\),
  \[
    \mathcal V \ni V_\beta \subseteq U_{\varphi(\beta)} \in \mathcal U.
  \]
  We write \(\mathcal V \leq \mathcal U\).
\end{definition}

If \(\mathcal V \leq \mathcal U\), we have natural maps
\[
  \rho_{\mathcal V \mathcal U}: C^p(\mathcal U, \mathcal F) \to C^p(\mathcal V, \mathcal F)
\]
given by
\[
  (\rho_{\mathcal V \mathcal U} \sigma)_{\beta_0 \cdots \beta_p} = (\sigma_{\varphi(\beta_0) \cdots \varphi(\beta_p)})|_{V_{\beta_0 \cdots \beta_p}}.
\]
One sees \(\rho_{\mathcal V \mathcal U} \compose \delta = \delta \compose \rho_{\mathcal V \mathcal U}\) so \(\rho_{\mathcal V \mathcal U}\) induces a homomorphism
\[
  \rho: \check H^q(\mathcal U, \mathcal F) \to \check H^q(\mathcal V, \mathcal F)
\]
for all \(q\). One can check that this is independent of \(\varphi\).

\begin{definition}[Čech cohomology]\index{Cech cohomolog@Čech cohomology}
  Define \emph{Čech cohomology} to be the direct limit
  \[
    H^q(X, \mathcal F) = \varinjlim_{\mathcal U} \check H^q(\mathcal U, \mathcal F).
  \]
  Note that we omit the check symbol.
\end{definition}

A quick recap of direct limit: if \(I\) is a partially ordered set, \(G_i\) is an abelian group for all \(i \in I\) with maps \(\varphi_{ij}: G_i \to G_j\) for \(i \leq j\) with
\[
  \varphi_{ij} \compose \varphi_{jk} = \varphi_{ik},
\]
then the direct limit is defined to be
\[
  \varinjlim_I G_i = (\bigoplus_{i \in I} G_i) /\sim
\]
where if \(g_i \in G_i, g_j \in G_j\) then \(g_i \sim g_j\) if and only if there is \(k\) with \(i, j \leq k\) such that
\[
  \varphi_{ik} (g_i) = \varphi_{jk} (g_j).
\]
The direct limit is an abelian group.

Thus elements of \(H^q(X, \mathcal F)\) are represented by \(\{\sigma_{\alpha_0 \cdots \alpha_q}\} \in \check H^q(\mathcal U, \mathcal F)\) and equality is checked on a common refinement.

We'll see that
\[
  H^q(X, \mathcal O) \cong \check H^q(\mathcal U, \mathcal O)
\]
when each intersection of the \(U_i\) is isomorphic to a polydisk.

\begin{eg}\leavevmode
  \begin{enumerate}
  \item \(\check H^0(\mathcal U, \mathcal F) = \mathcal F(X)\) for all \(\mathcal U\) so
    \[
      H^0(X, \mathcal F) \cong \mathcal F(X),
    \]
    the global sections.
  \item We show \(H^q(X, \mathcal A_\C^{r, s}) = 0\) for all \(q > 0\). Let \([\sigma] \in H^q(X, \mathcal A_\C^{r, s})\) be represented by \(\sigma \in C^q(\mathcal U, \mathcal A_\C^{r, s})\) for some \(\mathcal U\) with \(\delta \sigma = 0\).

    Let \(\rho_\alpha\) be a partition of unity subordinate to \(\mathcal U = \{U_\alpha\}\). Define
    \[
      \tau_{\alpha_0 \cdots \alpha_{q - 1}} = \sum_\beta \rho_\beta \sigma_{\beta \alpha_0 \cdots \alpha_{q - 1}}
    \]
    and extend by \(0\) to \(U_{\alpha_0 \cdots \alpha_{q - 1}}\) so \(\tau \in C^{q - 1}(\mathcal U, \mathcal A_\C^{r, s})\). We prove the special case where \(\mathcal U = \{U, V, W\}\), \([\sigma] \in H^1(\mathcal U, \mathcal A_\C^{r, s})\). Have
    \begin{align*}
      \delta \sigma &= \sigma_{UV} - \sigma_{UW} + \sigma_{VW} = 0 \\
      \tau_U &= \rho_V \sigma_{VU} + \rho_W \sigma_{WU} \\
      \tau_V &= \rho_V \sigma_{VU} + \rho_W \sigma_{WV} \\
      \tau_W &= \rho_U \sigma_{UW} + \rho_V \sigma_{VW}
    \end{align*}
    Then
    \begin{align*}
      (\delta \tau)_{UV}
      &= \tau_V - \tau_U \\
      &= \rho_V \sigma_{VU} + \rho_W \sigma_{WV} - \rho_V \sigma_{VU} - \rho_W \sigma_{WU} \\
      &= \rho_V \sigma_{VU} + \rho_V \sigma_{UV} + \rho_W \sigma_{WV} - \rho_W \sigma_{WU} \\
      &= (\rho_U + \rho_V + \rho_W) \sigma_{UV} \text{ use cocycle condition} \\
      &= \sigma_{UV}
    \end{align*}
    The general case is an exercise on example sheet 2.

    Similarly \(H^q(X, \mathcal A_\R^k) = 0\) for all \(q > 0\).
  \end{enumerate}
\end{eg}

\subsection{Short exact sequence of sheaves}

Let \(\beta: \mathcal F \to \mathcal G\) be a morphism of sheaves. Then \(\beta\) induces \(C^p(\mathcal U, \mathcal F) \to C^p(\mathcal U, \mathcal G)\) for any \(\mathcal U\). These maps commute with \(\delta\) so induce maps
\[
  \beta^*: H^p(X, \mathcal F) \to H^p(X, \mathcal G).
\]
Suppose we have a short exact sequence of sheaves
\[
  \begin{tikzcd}
    0 \ar[r] & \mathcal E \ar[r, "\alpha"] & \mathcal F \ar[r, "\beta"] & \mathcal G \ar[r] & 0
  \end{tikzcd}
\]
we get maps
\begin{align*}
  \alpha^*: H^p(X, \mathcal E) &\to H^p(X, \mathcal F) \\
  \beta^*: H^p(X, \mathcal F) &\to H^p(X, \mathcal G)
\end{align*}
This induces a long exact sequence of homology groups. Explicitly, we define coboundary maps
\[
  \delta^*: H^p(X, \mathcal G) \to H^{p + 1}(X, \mathcal E).
\]

Given \(\sigma \in C^p(\mathcal U, \mathcal G)\), we can pass to a refinement \(\mathcal V\) of \(\mathcal U\) and find \(\tau \in C^p(\mathcal V, \mathcal F)\) with \(\beta(\tau) = \rho_{\mathcal V \mathcal U} \sigma\). Now we assume \(\delta \sigma = 0\) so
\[
  \beta(\delta \tau)
  = \delta \beta \tau
  = \delta \rho_{\mathcal V \mathcal U} \sigma
  = \rho_{\mathcal V \mathcal U} \delta \sigma
  = 0.
\]
Thus we can find \(\mu \in C^{p + 1}(\mathcal V, \mathcal E)\) such that \(\alpha \mu = \delta \tau\). Then
\[
  \alpha(\delta \mu) = \delta \alpha \mu = \delta^2 \tau = 0.
\]
Since \(\alpha\) is injective, \(\delta \mu = 0\). This defines \(\delta^*[\sigma] = [\mu] \in H^{p + 1}(X, \mathcal E)\).

\begin{theorem}
  The following
  \[
    \begin{tikzcd}
      & & 0 \ar[dll, out=0, in=180] \\
      H^0(X, \mathcal E) \ar[r, "\alpha^*"] & H^0(X, \mathcal F) \ar[r, "\beta^*"] & H^0(X, \mathcal G) \ar[dll, "\delta^*"', out=0, in=180] \\
      H^1(X, \mathcal E) \ar[r, "\alpha^*"] & H^1(X, \mathcal F) \ar[r, "\beta^*"] & H^1(X, \mathcal G) \ar[dll, "\delta^*"', out=0, in=180] \\
      H^2(X, \mathcal E) \ar[r] & \dots
    \end{tikzcd}
  \]
  is a long exact sequence of cohomology groups.
\end{theorem}

We won't prove this in general, but for all sheaves in this course, there exists arbitrarily fine open covers \(\mathcal U\) with
\[
  \begin{tikzcd}
    0 \ar[r] & \mathcal E(U) \ar[r] & \mathcal F(U) \ar[r] & \mathcal G(U) \ar[r] & 0
  \end{tikzcd}
\]
exact for all \(U \in \mathcal U\). In this case the theorem is an exercise.

We say that
\[
  \begin{tikzcd}
    \mathcal F_1 \ar[r, "\alpha_1"] & \mathcal F_2 \ar[r, "\alpha_2"] & \dots
  \end{tikzcd}
\]
is a \emph{complex} of sheaves if \(\alpha_{i + 1} \compose \alpha_i = 0\) for all \(i\). We say that a complex is \emph{exact} if
\[
  \begin{tikzcd}
    0 \ar[r] & \ker \alpha_i \ar[r] & \mathcal F_i \ar[r] & \ker \alpha_{i + 1} \ar[r] & 0 
  \end{tikzcd}
\]
is a short exact sequence for all \(i\). Equivalently the induced sequence on stalk is exact everywhere.

\subsection{Dolbeault's theorem}

\begin{theorem}[de Rham]
  If \(X\) is a smooth manifold then
  \[
    H^i_{\text{dR}}(X; \R) \cong H^i(X, \R).
  \]
\end{theorem}

\begin{remark}
  It follows that
  \[
    H^i(X, \R) \cong H^i_{\text{sing}}(X; \R)
  \]
  where \(H^i_{\text{sing}}(X; \R)\) is the singular cohomology.
\end{remark}

\begin{proof}
  By Poincaré lemma, the complex
  \[
    \begin{tikzcd}
      0 \ar[r] & \R \ar[r] & \mathcal A^0 \ar[r, "d"] & \mathcal A^1 \ar[r, "d"] & \mathcal A^2 \ar[r] & \dots
    \end{tikzcd}
  \]
  is exact. Note that \(\mathcal A^0\) is the sheaf of smooth functions and \(\mathcal A^p\) is the sheaf of \(p\)-forms and \(\d\) is the usual exterior derivative. That is, for all \(p\), if \(\mathcal Z^p = \ker (\d: \mathcal A^p \to \mathcal A^{p + 1})\) we have exact sequences
  \[
    \begin{tikzcd}
      0 \ar[r] & \R \ar[r] & \mathcal A^0 \ar[r] & \mathcal Z^1 \ar[r] & 0 \\
      & & \vdots \\
      0 \ar[r] & \mathcal Z^{p - 1} \ar[r] & \mathcal A^{p - 1} \ar[r] & \mathcal Z^p \ar[r] & 0 \\
    \end{tikzcd}
  \]
  We saw that \(H^q(X, \mathcal A^p) = 0\) for all \(p \geq 0, q > 0\).

  The long exact sequence associated to the first short exact sequence gives
  \begin{align*}
    H^p(X, \R)
    &\cong H^{p - 1}(X, \mathcal Z^1) \quad \text{ as } H^p(X, \mathcal A^0) = H^{p - 1}(X, \mathcal A^0) = 0 \\
    &\cong H^{p - 2}(X, \mathcal Z^2) \\
    &\cong \dots \\
    &\cong H^1(X, \mathcal Z^{p - 1})
  \end{align*}
  Since
  \[
    \begin{tikzcd}
      & & 0 \ar[dll, out=0, in=180] \\
      H^0(X, \mathcal Z^{p - 1}) \ar[r] & H^0(X, \mathcal A^{p - 1}) \ar[r, "d"] & H^0(X, \mathcal Z^p) \ar[dll, out=0, in=180] \\
      H^1(X, \mathcal Z^{p - 1}) \ar[r] & 0
    \end{tikzcd}
  \]
  is exact, we have
  \[
    H^1(X, \mathcal Z^{p - 1})
    \cong \frac{H^0(X, \mathcal Z^p)}{d(H^0(X, \mathcal A^{p - 1}))}
    \cong \frac{\mathcal Z^p(X)}{d(\mathcal A^{p - 1}(X))}
    \cong H^p_{\text{dR}}(X, \R).
  \]
\end{proof}

\begin{theorem}[Dolbeault]\index{Dolbeault's theorem}
  If \(X\) is a complex manifold then
  \[
    H^q(X, \Omega^p) \cong H^{p, q}_{\conj \p}(X)
  \]
  where \(\Omega^p(U) = \{\sigma \in \mathcal A_\C^{p, 0}(U): \conj \p \sigma = 0\}\).
\end{theorem}

\begin{proof}
  Similar to de Rham's theorem but with \(\conj \p\)-Poincaré lemma instead. We have an exact complex
  \[
    \begin{tikzcd}
      0 \ar[r] & \Omega^p \ar[r] & \mathcal A_\C^{p, 0} \ar[r, "\conj \p"] & \mathcal A_\C^{p, 1} \ar[r, "\conj \p"] & \dots
    \end{tikzcd}
  \]
  by the \(\conj \p\)-Poincaré lemma. We write \(\mathcal Z^{p, q} = \ker (\conj \p: \mathcal A_\C^{p, q} \to \mathcal A_\C^{p, q + 1})\). Thus we have exact sequences
  \[
    \begin{tikzcd}
      0 \ar[r] & \Omega^p \ar[r] & \mathcal A^{p, 0} \ar[r] & \mathcal Z^{p, 1} \ar[r] & 0 \\
      & & \vdots \\
      0 \ar[r] & \mathcal Z^{p, q - 1} \ar[r] & \mathcal A^{p, q - 1} \ar[r] & \mathcal Z^{p, q} \ar[r] & 0 \\
    \end{tikzcd}
  \]
  as any open set in \(X\) has an open subset biholomorphic to a polydisk. 

  It follows that \(H^i(X, \mathcal A_\C^{r, s}) = 0\) for all \(i > 0\), for all \(r, s\). Argue as in de Rham's theorem,
  \begin{align*}
    H^q(X, \Omega^p)
    &\cong H^{q - 1}(X, \mathcal Z^{p, 1}) \\
    &\cong \dots \\
    &\cong H^1(X, \mathcal Z^{p, q - 1}) \\
    &\cong \frac{H^0(X, \mathcal Z^{p, q})}{\conj \p (H^0(X, \mathcal A_\C^{p, q - 1}))} \\
    &\cong \frac{\mathcal Z^{p, q}(X)}{\conj \p \mathcal A_\C^{p, q - 1}(X)} \\
    &\cong H^{p, q}_{\conj \p}(X)
  \end{align*}
\end{proof}

\subsection{Computation of Čech cohomology}

The direct limit in the definition of Čech cohomology means that it is very difficult to work out the cohomology directly. However, in a previous example we claimed that \(H^i(\P^1, \mathcal O)\) equals to \(\check H^i(\{\P^1 \setminus \{0\}, \P^1 \setminus \{\infty\}\}, \mathcal O)\), whose computation is manageable. This is due the following theorem:

\begin{theorem}
  Let \(X\) be a complex manifold. Suppose \(\mathcal U\) is an open cover with \(H^p(U_{\alpha_0 \cdots \alpha_s}, \mathcal O) = 0\) for all \(p \geq 1\) and all \(\alpha_0, \dots, \alpha_s\). Then
  \[
    H^p(X, \mathcal O) \cong \check H^p(\mathcal U, \mathcal O).
  \]
\end{theorem}

\begin{proof}
  We have
  \[
    H^1(U_{\alpha_0 \cdots \alpha_s}, \mathcal Z^{0, q - 1})
    = H^{0, q}_{\conj \p}(U_{\alpha_0 \cdots \alpha_s})
    = H^q(U_{\alpha_0 \cdots \alpha_s}, \mathcal O)
    = 0.
  \]
  Thus
  \[
    \begin{tikzcd}
      0 \ar[r] & \mathcal Z^{0, q - 1}(U_{\alpha_0 \cdots \alpha_s}) \ar[r] & \mathcal A_\C^{0, q - 1}(U_{\alpha_0 \cdots \alpha_s}) \ar[r] & \mathcal Z^{0, q}(U_{\alpha_0 \cdots \alpha_s}) \ar[r] & 0
    \end{tikzcd}
  \]
  is exact. It is true for all intersetions so we have a short exact sequence
  \[
    \begin{tikzcd}
      0 \ar[r] & C^p(\mathcal U, \mathcal Z^{0, q - 1}) \ar[r] & C^p(\mathcal U, \mathcal A_\C^{0, q - 1}) \ar[r] & C^p(\mathcal U, \mathcal Z^{0, q}) \ar[r] & 0
    \end{tikzcd}
  \]
  The long exact sequence and \(\check H^p(\mathcal U, \mathcal A^{0, q}) = 0\) gives for all \(p \geq 1, q \geq 1\),
  \[
    \check H^p(\mathcal U, \mathcal Z^{0, q}) \cong \check H^{p + 1}(\mathcal U, \mathcal Z^{0, q - 1}).
  \]
  Argue as before,
  \begin{align*}
    \check H^p(\mathcal U, \mathcal O)
    &= \check H^p(\mathcal U, \mathcal Z^{0, 0}) \\
    &\cong \check H^{p - 1}(\mathcal U, \mathcal Z^{0, 1}) \\
    &\cong \dots \\
    &\cong \check H^1(\mathcal U, \mathcal Z^{0, p - 1})
  \end{align*}
  and
  \[
    \check H^1(\mathcal U, \mathcal Z^{0, p - 1})
    \cong \frac{\mathcal Z^{0, p}(X)}{\conj \p(\mathcal A^{0, p - q}(X))}
    \cong H^{0, p}_{\conj \p}(X)
    \cong H^p(X, \mathcal O).
  \]
\end{proof}

\begin{remark}
  It also shows that under the same hypothesis,
  \[
    H^q(X, \Omega^p) \cong \check H^q(\mathcal U, \Omega^p).
  \]
\end{remark}

\begin{eg}
  \[
    H^q(\C^n, \mathcal O) \cong H^{0, q}(\C^n) = 0
  \]
  for all \(q \geq 1\).
\end{eg}

\begin{remark}\leavevmode
  \begin{enumerate}
  \item One can show if \(H^p(U_\alpha, \mathcal O) = 0\) for all \(U_\alpha \in \mathcal U\) (no higher intersections) then
    \[
      H^p(X, \mathcal O) \cong \check H^p(\mathcal U, \mathcal O).
    \]
    See Voisin Section 4. So if \(X\) is projective then one can take \(\mathcal U\) to be a cover by affine subvarieties. When \(X\) is not projective, one can take a cover by \emph{Stein manifolds}\index{Stein manifold}, which are the complex manifold version of affine subvariety.
  \item \(H^p(X, \Z) \cong H^p_{\text{sing}}(X; \Z)\).
  \item One usually cares about \(H^0(X, \mathcal F)\), the global sections, and the \(H^i\)'s are viewed as obstructions. For example, in short exact sequence, Mittag-Leffler problem. Another reason to care about \(H^i\) is the \emph{Euler characteristic}\index{Euler characteristic}
    \[
      \chi(X, \mathcal F) = \sum_i (-1)^i \dim H^i(X, \mathcal F)
    \]
    which is additive in short exact sequences, and usually constant in families, while \(H^0\) is not. Lastly, \(H^1\) is also ``geometric''.
  \end{enumerate}
\end{remark}

\section{Holomorphic vector bundles}

\begin{definition}[holomorphic vector bundle]\index{holomorphic vector bundle}\index{vector bundle}
  Let \(X\) be a complex manifold. A \emph{holomorphic vector bundle} on \(X\) is a complex manifold \(E\) with a (holomoprhic surjective) map \(\pi: E \to X\) and the structure of an \(r\) dimensional complex vector space on every fibre \(\pi^{-1}(x) = E_x\) satisfying: there is an open cover \(\{U_\alpha\}\) of \(X\) and holomorphic isomorphisms \(\varphi_\alpha: \pi^{-1}(U_\alpha) \to U_\alpha \times \C^r\) commuting with projections to \(U_\alpha\), such that the induced map \(E|_x \cong \C^r\) is \(\C\)-linear.
\end{definition}

\begin{definition}[line bundle]\index{line bundle}
  A \emph{(holomorphic) line bundle} is a holomorphic vector bundle of rank \(1\).
\end{definition}

Any holomorphic vector bundle induces a complex vector bundle but not vice versa.

\begin{definition}[morphism of vector bundles]\index{holomorphic vector bundle!morphism}
  Let \(\pi_E: E \to X, \pi_F: F \to X\) be holomorphic vector bundles. A \emph{morphism} \(f: E \to F\) is a holomorphic map such that
  \begin{enumerate}
  \item \(\pi_F \compose f = f \compose \pi_E\).
  \item the induced map \(f_x: E_x \to F_x\) is linear.
  \item \(\text{rank}(f_x)\) is constant.
  \end{enumerate}

  A morphism is an \emph{isomorphism} if \(f_x\) is an isomorphism for all \(x \in X\).
\end{definition}

\begin{remark}
  In differential geometry one usually does not required 3. We include it to take kernel and cokernel bundles.
\end{remark}

Next up is a review of differential geometry. For a holomorphic vector bundle \(E\), its \emph{transition functions}
\[
  \varphi_{\alpha\beta} = \varphi_\alpha \compose \varphi_\beta^{-1}: (U_\alpha \cap U_\beta) \times \C^r \to (U_\alpha \cap U_\beta) \times \C^r 
\]
can be seen as holomorphic maps
\[
  \varphi_{\alpha\beta}: U_\alpha \cap U_\beta \to \GL_r(\C).
\]
They satisfy the \emph{cocycle conditions}\index{cocycle condition}
\begin{align*}
  \varphi_{\alpha\alpha} &= \id \\
  \varphi_{\alpha\beta} &= \varphi_{\beta\alpha}^{-1} \\
  \varphi_{\alpha\beta} \varphi_{\beta\gamma} \varphi_{\gamma\alpha} &= \id
\end{align*}
which should remind us of cocyle conditions in Čech cohomology.

\begin{proposition}
  Given any open cover \(X = \bigcup U_\alpha\) and holomorphic maps \(\varphi_{\alpha\beta}: U_\alpha \cap U_\beta \to \GL_r(\C)\) satisfying the cocyle conditions, there is a holomorphic vector bundle with these transition function.
\end{proposition}

\begin{proof}
  Same as in differential geometry.
\end{proof}

Given \(E\) and a cover \(\mathcal U = \{U_\alpha\}\) with trivialisations \(\varphi_\alpha: E|_{U_\alpha} \to U_\alpha \times \C^r\), the transition functions
\[
  \{\varphi_{\alpha\beta} = \varphi_\alpha \compose \varphi_\beta^{-1}\} \in C^1(\mathcal U, \GL_r(\C))
\]
satisfy the cocycle conditions, i.e.
\[
  \delta(\{\varphi_{\alpha\beta}\}) = 0
\]
so we obtain an element \([\varphi^E] \in H^1(X, \GL_r(\C))\) (viewing \(\GL_r(\C)\) as a group under multiplication)\footnote{Note that \(\GL_r(\C)\) is not abelian for \(r > 1\), and it is not immediately what the corresponding Čech cohomology should be. However, we'll restrict our attention to line bundles in this course.}. We now specialise to line bundles so \(\GL_r(\C) = \C^*\) so they are abelian. In particular we have
\[
  H^1(X, \GL_r(\C)) = H^1(X, \mathcal O^*).
\]

\begin{proposition}
  There is a canonical bijection
  \[
    \{\text{holomorphic line bundles up to isomorphism}\} \leftrightarrow H^1(X, \mathcal O^*).
  \]
\end{proposition}

\begin{proof}
  We have already constructed maps in each direction. Suppose \(L \cong F\) are isomorphic line bundles. Choose a cover \(\mathcal U = \{U_\alpha\}\) trivialising both by taking their common refinement. We have isomorphisms
  \begin{align*}
    \varphi_\alpha: L|_{U_\alpha} &\to U_\alpha \times \C \\
    \sigma_\alpha: F|_{U_\alpha} &\to U_\alpha \times \C \\
  \end{align*}
  giving \(\varphi_{\alpha\beta}, \sigma_{\alpha\beta}\) as before. We have an isomorphism \(f: L \to F\), giving \(f_\alpha: L|_{U_\alpha} \to F|_{U_\alpha}\). Define
  \[
    h_\alpha = \sigma_\alpha f_\alpha \varphi_\alpha^{-1}: U_\alpha \times \C \to U_\alpha \times \C,
  \]
  which can alternatively be seen as a section of \(\mathcal O^*\). Moreover
  \begin{align*}
    (\delta h)_{\alpha\beta}
    &= h_\alpha h_\beta^{-1} \\
    &= \sigma_\alpha f_\alpha \varphi_\alpha^{-1} \varphi_\beta f_\beta^{-1} \sigma_\beta^{-1} \\
    &= \sigma_\alpha f_\alpha \varphi_{\beta\alpha} f_\beta^{-1} \sigma_\beta^{-1} \\
    &= \sigma_\alpha \varphi_{\beta\alpha} f_\alpha f_\beta^{-1} \sigma_\beta^{-1} \\
    &= \sigma_{\alpha\beta} \varphi_{\alpha\beta}^{-1} \quad \text{as } f_\alpha f_\beta^{-1} = \id
  \end{align*}
  so \([\sigma] = [\tau] \in H^1(X, \mathcal O^*)\).

  Conversely, let \(L\) and \(F\) be line bundles with \([\varphi] = [\sigma] \in H^1(X, \mathcal O^*)\). This means that there is \(h = \{h_\alpha\} \in C^0(\mathcal U, \mathcal O^*)\) with
  \[
    (\delta h)_{\alpha\beta} = \varphi_{\alpha\beta}^{-1} \sigma_{\alpha\beta}.
  \]
  Let
  \[
    f_\alpha = \sigma_\alpha^{-1} h_\alpha \varphi_\alpha: L|_{U_\alpha} \to F|_{U_\alpha}.
  \]
  We claim the \(f_\alpha\)'s induce an isomorphism \(f: L \to M\), i.e.\ \(f_\alpha f_\beta^{-1} = \id\) on \(U_\alpha \cap U_\beta\). Indeed
  \[
    f_\alpha f_\beta^{-1} = \sigma_\alpha^{-1} h_\alpha \varphi_\alpha \varphi_\beta^{-1} h_\beta^{-1} \sigma_\beta = \id
  \]
  as before.
\end{proof}

Note that we did not use any properties of holomorphicity so analogous results hold in smooth/analytic categories.

\begin{remark}
  A similar result is true for vector bundles of all ranks, with the right definition of Čech coholomogy for sheaves of (non-abelian) groups. See course website.
\end{remark}

\begin{definition}[Picard group]\index{Picard group}
  We define the \emph{Picard group} \(\Pic(X)\) to be the set of line bundles on \(X\) up to isomorphism.
\end{definition}

\begin{proposition}
  \(\Pic(X)\) is a group under tensor product of line bundles and
  \[
    \Pic(X) \cong H^1(X, \mathcal O^*).
  \]
\end{proposition}

\begin{proof}
  Easiest proof is using transition functions. The transition functions for \(L \otimes F\) are
  \[
    \varphi_{\alpha\beta} \otimes \sigma_{\alpha\beta} \in \mathcal O^*(U_\alpha \cap U_\beta)
  \]
  so \(L \otimes L^* \cong \mathcal O\) and \(L \otimes \mathcal O \cong L\).
\end{proof}

\begin{eg}
  Any linear algebra operation gives an operation on vector bundles:
  \begin{enumerate}
  \item \(E \oplus F\): transition functions are \(\varphi_{\alpha\beta} \oplus \sigma_{\alpha\beta} \in \GL_{r + r'}(\C)\).
  \item \(E \otimes F\): transition functions \(\varphi_{\alpha\beta} \otimes \sigma_{\alpha\beta} \in \GL(\C^r \otimes \C^{r'})\).
  \item \(\Lambda^kE\): transition functions \(\Lambda^k \varphi_{\alpha\beta}\). If \(k = r\) we write \(\Lambda^r E = \det E\).
  \end{enumerate}
\end{eg}

\begin{definition}[section]\index{section}
  A \emph{(holomorphic) section} \(s\) of a vector bundle \(\pi: E \to X\) over \(U \subseteq X\) is a holomorphic map \(s: U \to E\) with \(\pi \compose s = \id\). We write \(\mathcal O(E)\) for the sheaf of holomorphic sections of \(E\).
\end{definition}

Note that the sheaf of holomorphic functions \(\mathcal O\) can be seen as the sheaf of sections of \(X \times \C\), which we have implicitly used in the proof above.

\begin{definition}[subsheaf]\index{sheaf!subseaf}
  If \(\mathcal F\) is a sheaf on \(X\) and \(U \subseteq X\) open then the \emph{subsheaf} of \(\mathcal F\) on \(U\) is \(\mathcal F|_U(V) = \mathcal F(V)\) for all \(V \subseteq U\) open.
\end{definition}

\begin{definition}[locally free sheaf]\index{sheaf!locally free}
  A sheaf \(\mathcal F\) is \emph{locally free of rank \(r\)} if for all \(x \in X\) there is an open set \(x \in U \subseteq X\) open with
  \[
    \mathcal F|_U \cong \mathcal O^{\oplus r}|_U.
  \]
\end{definition}

\begin{proposition}
  Associating to a holomorphic vector bundle its sheaf of sections gives a canonical bijection between
\[
  \{\text{vector bundles up to isomorphism}\} \leftrightarrow \{\text{locally free sheaves up to isomorphism}\}.
\]
\end{proposition}

\begin{proof}
  Clearly the sheaf of sections of \(E\) is locally free as \(E\) is locally isomorphic to \(U_\alpha \times \C^r\) by definition. Conversely, if we have trivialisations
  \[
    \varphi_\alpha: \mathcal F|_{U_\alpha} \to \mathcal O^{\oplus r}|_{U_\alpha}
  \]
  then the transition maps
  \[
    \varphi_{\alpha\beta} = \varphi_\alpha \compose \varphi_\beta^{-1}: \mathcal O^{\oplus r} (U_\alpha \cap U_\beta) \to \mathcal O^{\oplus r} (U_\alpha \cap U_\beta),
  \]
  which are isomorphisms by definition, are given by a matrix of holomorphic functions on \(U_\alpha \cap U_\beta\), giving a cocycle and hence a holomorphic vector bundle. Checking these maps are inverses to each other is straightforward.
\end{proof}

Thus for a holomorphic vector bundle \(E\), we define its cohomology to be the cohomology of its sheaf of sections
\[
  H^i(X, E) = H^i(X, \mathcal O(E)).
\]

\begin{eg}[holomorphic tangent bundle]\index{holomorphic tangent bundle}
  Recall \(TX^{1, 0}\), the holomorphic tangent bundle. We show this indeed is a holomorphic vector bundle.

  Let \(X = \bigcup U_\alpha\) be an open covering by chart neighbourhoods \((U_\alpha, \varphi_\alpha)\). The Jacobian of a transition map is
  \[
    J(\varphi_{\alpha\beta})(z) = \left( \frac{\partial \varphi_{\alpha\beta}^\gamma}{\partial z^\delta} (\varphi_\beta(z)) \right)_{\gamma, \delta}.
  \]
  Then by example sheet 1 Q1, \(TX^{1, 0}\) has transition functions
  \[
    \psi_{\alpha\beta} = J(\varphi_{\alpha\beta}) \in \GL_n(\C) (U_\alpha \cap U_\beta).
  \]

  \begin{definition}[canonical line bundle]\index{canonical line bundle}\index{line bundle!canonical}
    The \emph{canonical line bundle} of \(X\) is defined to be
    \[
      K_X = \det T^*X^{1, 0}
    \]
    where \(T^*X^{1, 0} = (TX^{1, 0})^*\) and \(\det E = \Lambda^n E\).
\end{definition}
\end{eg}

\begin{eg}[tautological line bundle]\index{tautological line bundle}\index{line bundle!tautological}
  We construct line bundles on \(\P^n\). Each point \(\ell \in \P^n\) corresponds to a line through \(0\) in \(\C^{n + 1}\). Consider the set
  \[
    \mathcal O(-1) = \{(\ell, z) \in \P^n \times \C^{n + 1}: z \in \ell\}.
  \]
  We claim that this is a holomorphic line bundle \(\mathcal O(-1) \to \P^n\). Let \(\P^n = \bigcup_{\alpha = 0}^n U_\alpha\) be the standard cover. A trivialisation of \(\mathcal O(-1)\) over \(U_\alpha\) is given by
  \begin{align*}
    \psi_\alpha: \pi^{-1}(U_\alpha) &\to U_\alpha \times \C \\
    (\ell, z) &\mapsto (\ell, z_\alpha)
  \end{align*}
  The transition functions are
  \begin{align*}
    \psi_{\alpha\beta}(\ell): \C &\to \C \\
    z &\mapsto \frac{\ell_\beta}{\ell_\alpha} z
  \end{align*}
  if \(\ell = [\ell_0 : \dots : \ell_n]\).

  Need to check \(\mathcal O(-1)\) is a complex manifold. If \((U_\alpha, \varphi_\alpha)\) is a chart on \(\P^n\), define chart
  \[
    \hat \varphi_\alpha = (\varphi_\alpha \times \id) \compose \psi_\alpha: \pi^{-1}(U_\alpha) \to \C \times \C^n.
  \]
  \(\mathcal O(-1)\) is called the \emph{tautological line bundle}. \(\mathcal O(1) = \mathcal O(-1)^*\) is the \emph{hyperplane line bundle}\index{hyperplane line bundle}. Finally define
  \begin{align*}
    \mathcal O(k) &= \mathcal O(1)^{\otimes k} \\
    \mathcal O(-k) &= \mathcal O(-1)^{\otimes k} \\
    \mathcal O(0) &= \mathcal O
  \end{align*}
  We will show \(\Pic(\P^n) \cong \Z\) with generator \(\mathcal O(1)\).
\end{eg}

\begin{eg}
  If \(p: Y \to X\) is a morphism, \(E \to X\) is a holomorphic vector bundle then one obtains the \emph{pullback bundle}\index{pullback bundle} \(p^*E \to Y\) by simply pulling back transition functions. We write \(E|_Y\) if \(Y \subseteq X\) to the pullback under the inclusion map.

  If \(X\) is projective, \(X \subseteq \P^n\), then \(X\) has a natural line bundle \(\mathcal O(1)|_X \to X\).
\end{eg}

We now relate sections of line bundles, codimension \(1\) submanifolds and meromorphic functions.

By the implicit function theorem, a subset \(Y \subseteq X\) is a closed complex manifold if and only if for all \(p \in X\) there exists a chart neighbourhood \((U, \varphi)\) of \(p\) and holomorphic functions \(f_1, \dots, f_k: U \to \C\) such that \(0\) is a regular value of \((f_1 \compose \varphi^{-1}, \dots, f_k \compose \varphi^{-1})\) and
\[
  Y \cap U = \bigcap_{i = 1}^k f_i^{-1}(0).
\]

Recall that if \(U \subseteq \C^n\) is open, \(f: U \to \C^k\) holomorphic then
\[
  J(f)(z) = \left( \frac{\partial f_\alpha}{\partial z^\beta}(z) \right)_{\substack{1 \leq \alpha \leq k \\ 1 \leq \beta \leq n}}.
\]
\(z \in U\) is regular if \(J(f)(z)\) is surjective. If every point \(z \in f^{-1}(w)\) is regular, \(w\) is a called a regular value.

\begin{definition}[analytic subvariety]\index{analytic subvariety}
  Let \(X\) be a complex manifold. An \emph{analytic subvariety} of \(X\) is a closed subset \(Y \subseteq X\) such that for all \(p \in Y\), there is a neighbourhood \(U\) of \(p\) in \(X\) and holomorphic functions \(f_1, \dots, f_k\) with
  \[
    Y \cap U = \bigcap_{i = 1}^k f_i^{-1}(0).
  \]

  Say \(y \in Y\) is \emph{regular} or \emph{smooth} if one can choose the \(f_i\)'s such that \(0\) is regular.
\end{definition}

By implicit function theorem, if \(Y^{\text s}\) denotes the points which are not regular, then connected components of \(Y^* = Y \setminus Y^{\text s}\) are naturally complex manifolds.

\begin{definition}[irreducible]\index{analytic subvariety!irreducible}
  An analytic subvariety \(Y\) is \emph{irreducible} if it cannot be written as \(Y = Y_1 \cup Y_2\) where \(Y_1, Y_2\) are analytic subvarieties with \(Y \neq Y_1, Y \neq Y_2\).
\end{definition}

\begin{definition}
  For \(Y\) an irreducible analytic subvariety, we define
  \[
    \dim Y = \dim Y^*.
  \]
  Similarly if each irreducible compoenent has the same dimension.

\end{definition}

If \(\operatorname{codom} Y = 1\) then \(Y\) is an analytic hypersurface.

\section{Commutative algebra on complex manifolds}

Recall that if \(\mathcal F\) is a sheaf on \(X\) and \(x \in X\) we denote by \(\mathcal F_x\) the \emph{stalk}\index{stalk} of \(\mathcal F\) at \(x\).

On \(\C^n\) denote by \(\mathcal O_{\C^n}\) the sheaf of holomorphic functions and set \(\mathcal O_n = \mathcal O_{\C^n, 0}\). Elements of \(\mathcal O_n\) are of the form \((U, f)\) where \(0 \in U\) and \(f \in \mathcal O_{\C^n(U)}\), and \((U, f) = (V, g)\) if there is an open \(W \subseteq U \cap V\) such that \(f|_W = g|_W\).

If \(X\) is an \(n\)-dimensional complex manifold, \(\mathcal O_X\) is the sheaf of holomorphic functions. Have
\[
  \mathcal O_{X, x} \cong \mathcal O_n
\]
for any \(x \in X\). We call elements of \(\mathcal O_{X, x}\) \emph{germs}\index{germ} of holomorphic functions.

\(\mathcal O_n\) is a local ring, in the sense that it has a unique maximal ideal \(\{f: f(0) = 0\}\). Functions not vanishing at \(0\) are invertible. These are the units of the ring.

We now state several results about \(\mathcal O_n\), proved using commutative algebra and complex analysis. We shall not prove them but proofs can be found in Huybrechts Chapter 1.

\begin{theorem}
  \(\mathcal O_n\) is a UFD.
\end{theorem}

\begin{theorem}[weak Nullstellensatz]
  Let \(f, g \in \mathcal O_n\) with \(f\) irreducible, \(U\) a neighbourhood on which \(f, g\) are defined. Suppose \(\{f = 0\} \cap U \subseteq \{g = 0\} \cap U\) then \(f\) divides \(g\) in \(\mathcal O_n\), i.e.\ \(\frac{g}{f}\) is holomorphic near \(0\).
\end{theorem}

\begin{definition}[thin]\index{thin}
  Let \(U \subseteq \C^n\) open. Call a set \(V \subseteq U\) \emph{thin} if \(V\) is locally contained in the vanishing set of a set of holomorphic functions.
\end{definition}

\begin{theorem}\leavevmode
  \label{thm:thin set}
  \begin{enumerate}
  \item Suppose \(f \in \mathcal O_n\) is irreducible. Then there is a thin set \(V\) of codimension \(2\) and an open set \(U\) such that \(f \in \mathcal O_p\) is irreducible for all \(p \in U \setminus V\).
  \item If \(f, g \in \mathcal O_n\) coprime then there are \(U, V\) as above such that \(f, g\) are coprime in \(\mathcal O_p\) for all \(p \in U \setminus V\).
  \end{enumerate}
\end{theorem}

\begin{remark}
  Huybrechts Proposition 1.1.35 claims that one can take \(V = \emptyset\), but this is false by counterexample: \(y^2 - xz^3\) is irreducible at \(0 \in \C^3\) but not at \((x_0, 0, 0)\) for \(x_0\) near \(0\). Instead the proof shows the statement above.
\end{remark}

\begin{definition}[local defining equation]\index{local defining equation}
  Let \(X\) be a complex manifold and \(Y \subseteq X\) an analytic hypersurface. If \(p \in Y\) then there is an open neighbourhood \(p \in U \subseteq X\) and \(f \in \mathcal O_X(U)\) with \(Y \cap U = f^{-1}(0) \cap U\). Such an \(f\) is called a \emph{local defining equation} for \(Y\).
\end{definition}

If \(f\) and \(g\) are both defining equations for \(Y\) and \(f = f_1 \cdots f_n, g = g_1 \cdots g_m\) where \(f_i, g_j\)'s are irreducible then by UFD and weak Nullstellensatz \(f_i = g_i\) and \(n = m\).

\begin{theorem}
  Let \(Y\) be an analytic hypersurface. Then \(Y^*\) is an open dense subset of \(Y\). \(Y^*\) is connected if and only if \(Y\) is irreducible. \(Y^{\text s}\) is contained in an analytic subvariety (of \(X\)) of codimension at least 2.
\end{theorem}

\section{Meromorphic functions and divisors}

\begin{definition}[meromorphic function]\index{meromorphic function}
  Let \(X\) be a complex manifold and \(U \subseteq X\) open. A \emph{meromorphic function} on \(U\) is a map \(f: U \to \coprod_{p \in U} K_p\) where \(K_p\) is the field of fractions of \(\mathcal O_p\), such that for all \(p \in U\), \(f(p) \in K_p\) and there is a neighbourhood \(V \subseteq U\) of \(p\) and \(g, h \in \mathcal O_X(V)\) with \(f(q) = \frac{g}{h}\) for all \(q \in V\).

  We denote by \(\mathcal K\) the corresponding sheaf, and \(\mathcal K^*\) the sheaf of meromorphic functions not identically \(0\).
\end{definition}

\begin{ex}
  Equivalently, one can specify \(f|_{U_\alpha} = \frac{g_\alpha}{h_\alpha}\) where \(g_\alpha, h_\alpha \in \mathcal O(U_\alpha)\).
\end{ex}

A meromorphic ``function'' is undefined (even as \(\infty\)) at point \(p\) where \(g(p) = h(p) = 0\).

\begin{definition}
  Let \(Y \subseteq X\) be an analytic hypersurface, \(p \in Y\) regular, \(f\) a local defining function at \(p\). For \(g \in \mathcal O_{X, p}\), we define the \emph{order} of \(g\) along \(Y\) at \(p\) to be
  \[
    \ord_{Y, p}(g) = \max_{a \in \N} \{a: f^a \text{ divides } g \text{ in } \mathcal O_{X, p}\}.
  \]
  It is well-defined as \(\mathcal O_{X, p}\) is a UFD and is finite.
\end{definition}

\begin{lemma}
  There is a neighboudhood \(U\) of \(p\), a thin set \(V\) of codimension \(2\) such that if \(q \in (U \setminus V) \cap Y\) then
  \[
    \ord_{Y, p}(g) = \ord_{Y, q}(g).
  \]
\end{lemma}

\begin{proof}
  Use \Cref{thm:thin set}.
\end{proof}

\begin{definition}[order]\index{order}
  We define the \emph{order} of \(g\) along \(Y\), \(Y\) irreducible to be
  \[
    \ord_Y(g) = \ord_{Y, p}(g)
  \]
  for any \(p \in Y^*\) away from the thin set in the lemma.
\end{definition}
Here we used \(Y^*\) is thin and \(V\) has codimension 2 in \(X\).

If \(g, h\) are holomorphic around \(p\) then
\[
  \ord_Y (gh) = \ord_Y (g) + \ord_Y (h).
\]

\begin{definition}[order]\index{order}
  Let \(X\) be a complex manifold, \(f\) meromorphic not identically zero. Let \(Y\) be an irreducible analytic hypersurface. We define
  \[
    \ord_Y(f) = \ord_Y(g) - \ord_Y(h)
  \]
  where \(f = \frac{g}{h}\) at some regular point of \(Y\).
\end{definition}
This is well-defined by additivity of \(\ord\).

If \(d = \ord_Y(f) > 0\), we say that \(f\) has \emph{zero} of order \(d\) along \(Y\) and if \(d < 0\), we say that \(f\) has a \emph{pole} of order \(d\) along \(Y\).

\begin{definition}[divisor]\index{divisor}\index{divisor!effective}
  A \emph{divisor} on \(X\) is a formal sum
  \[
    D = \sum a_\alpha Y_\alpha
  \]
  with \(a_\alpha \in \Z\), \(Y_\alpha\) irreducible analytic hypersurface, such that \(D\) is locally finite (if \(x \in X\) then there is a neighbourhood \(V\) of \(x \in X\) with \(Y_\alpha \cap V = \emptyset\) for all but finitely many \(\alpha\)).

  We say \(D\) is \emph{effective} if \(a_\alpha \geq 0\) for all \(\alpha\).
\end{definition}

\begin{eg}
  If \(\dim X = 1\) then this is a collection of points with some multiplicities.
\end{eg}

\begin{definition}
  If \(f \in H^0(X, \mathcal K^*)\), we set
  \[
    (f) = \sum_Y \ord_Y(f) Y
  \]
  summing over all \(Y \subseteq X\) irreducible analytic hypersurfaces.
\end{definition}
This is locally finite as given \(x \in X\) with \(f = \frac{g}{h}\), there are only finitely many \(Y\) with \(\ord_Y(g) \neq 0\) (writing \(g\) as a product of irreducibles).

Note \((f)\) is effective if and only if \(f\) is holomorphic.

\begin{definition}[principal divisor]\index{divisor!principal}\index{divisor!linearly equivalent}
  We call a divisor \(D\) \emph{principal} if \(D = (f)\) for some \(f \in H^0(X, \mathcal K^*)\).

  We say \(D, D'\) are \emph{linearly equivalent} if \(D - D'\) is principal. We write \(D \sim D'\). This is transitive because \((f) + (g) = (fg)\).
\end{definition}

There is an inclusion of sheaves \(\mathcal O^* \embed \mathcal K^*\) as every holomorphic function is meromorphic. Thus we obtain \(\mathcal K^*/\mathcal O^*\), the \emph{quotient sheaf}\index{sheaf!quotient}, by sheafifying the presheaf \(U \mapsto \mathcal K^*(U)/\mathcal O^*(U)\).

A global section \(f \in H^0(X, \mathcal K^*/\mathcal O^*)\) thus consists of an open cover \(\{U_\alpha\}\) of \(X\) and meromorphic functions \(f_\alpha \in \mathcal K^*(U_\alpha)\) with
\[
  \frac{f_\alpha}{f_\beta} \Big|_{U_\alpha \cap U_\beta} \in \mathcal O^*(U_\alpha \cap U_\beta)
\]
when \(U_\alpha \cap U_\beta \neq \emptyset\).

\begin{proposition}
  There is an isomorphism
  \[
    H^0(X, \mathcal K^*/\mathcal O^*) \cong \Div(X).
  \]
\end{proposition}

\begin{proof}
  Let \(f \in H^0(X, \mathcal K^*/\mathcal O^*)\) be given as above. If \(Y\) is an irreducible analytic hypersurface with \(Y \cap U_\alpha \cap U_\beta \neq \emptyset\), we have
  \[
    \ord_Y(f_\alpha) = \ord_Y(f_\beta)
  \]
  as \(\ord_Y(\frac{f_\alpha}{f_\beta}) = 0\) since \(\frac{f_\alpha}{f_\beta} \in \mathcal O^*(U_\alpha \cap U_\beta)\). Thus we may define
  \[
    \ord_Y(f) = \ord_Y(f_\alpha)
  \]
  for any \(U_\alpha\) with \(Y \cap U_\alpha \neq 0\). This gives a map
  \begin{align*}
    H^0(X, \mathcal K^*/\mathcal O^*) &\to \Div(X) \\
    f &\mapsto \sum \ord_Y(f) Y
  \end{align*}
  Clearly this is a group homomorphism by additivity of \(\ord\).

  We next construct an inverse. Suppose \(D = \sum a_\alpha Y_\alpha\). Consider \(Y_\alpha\). Then there is an open cover \(\{U_\beta\}\) of \(X\) and \(g_{\alpha\beta} \in \mathcal O(U_\beta)\) such that
  \[
    Y_\alpha \cap U_\beta = g_{\alpha\beta}^{-1}(0)
  \]
  (with, say, \(g_{\alpha\beta} = 1\) if \(Y_\alpha \cap U_\beta = \emptyset\)) Set
  \[
    f_\beta = \prod_\alpha g_{\alpha\beta}^{a_\alpha},
  \]
  a finite product as divisors are locally finite. Since \(g_{\alpha\beta}\) and \(g_{\alpha\gamma}\) define the same hypersurface on \(U_\beta \cap U_\gamma\), we have
  \[
    \frac{g_{\alpha\beta}}{g_{\alpha\gamma}} \in \mathcal O^*(U_\beta \cap U_\gamma).
  \]
  Thus the \(f_\beta\)'s glue to a section of \(H^0(X, \mathcal K^*/\mathcal O^*)\).

  The maps are clearly mutual inverses.
\end{proof}

We shall say \(D \in \Div(X)\) is given by local data \((U_\alpha, f_\alpha)\) using this construction.

\begin{theorem}
  There exists a natural group homomorphism
  \begin{align*}
    \Div(X) &\to \Pic(X) \\
    D &\mapsto \mathcal O(D)
  \end{align*}
  defined as below, whose kernel is percisely the principal divisors.
\end{theorem}

\begin{proof}
  Let \(D \in \Div(X)\) given by local data \((U_\alpha, f_\alpha)\). Let
  \[
    \varphi_{\alpha\beta} = \frac{f_\alpha}{f_\beta} \in \mathcal O^*(U_\alpha \cap U_\beta).
  \]
  These then satisfy the cocycle condition (\(\varphi_{\alpha\beta} \varphi_{\beta\gamma} \varphi_{\gamma\alpha} = 1\)), so gives an element of \(\Pic (X) \cong H^1(X, \mathcal O^*)\). We check this is well-defined: if \((U_\alpha, f_\alpha')\) is alternative local data then \(f_\alpha = s_\alpha f_\alpha'\) with \(s_\alpha \in \mathcal O^*(U_\alpha)\). The new transition functions are
  \[
    \varphi_{\alpha\beta}' = \varphi_{\alpha\beta} \frac{s_\beta}{s_\alpha}.
  \]
  Then \((U_\alpha, \frac{s_\beta}{s_\alpha})\) satisfy the cocycle conditions, giving a line bundle \(L\) with a nowhere vanishing section \(s\) induced by \(s_\alpha\)'s. The line bundles defined by \((U_\alpha, \varphi_{\alpha\beta})\) and \((U_\alpha, \varphi_{\alpha\beta}')\) are \(H\) and \(H'\) and
  \[
    H \cong H' \otimes L
  \]
  as \(\varphi_{\alpha\beta}' = \varphi_{\alpha\beta} \frac{s_\beta}{s_\alpha}\) and transition functions for tensor products are products of transition functions.

  That this is a group homomorphism is clear: if \(D, D'\) given by local data \((U_\alpha, f_\alpha), (U_\alpha, f_\alpha')\) then \(D + D'\) is given by \((U_\alpha, f_\alpha f_\alpha')\) so
  \[
    \mathcal O(D + D') \cong \mathcal O(D) \otimes \mathcal O(D').
  \]

  To prove the statement about kernel, suppose \(D = (f)\) where \(f \in H^0(X, \mathcal K^*)\) then we can take \((U_\alpha, f_\alpha)\) to be the local data. Then
  \[
    \varphi_{\alpha\beta} = \frac{f_\alpha}{f_\beta} = \id
  \]
  on \(U_\alpha \cap U_\beta\), so \(\mathcal O(D)\) has trivial transition functions and hence
  \[
    \mathcal O(D) \cong \mathcal O.
  \]

  Conversely suppose \(\mathcal O(D) \cong \mathcal O\). let \(s\) be a global nowhere holomorphic section. Suppose \(\mathcal O(D)\) has transition functions \(\{(U_\alpha, \varphi_{\alpha\beta})\}\), so \(D\) is given by \(\{(U_\alpha, f_\alpha)\}\) such that \(\varphi_{\alpha\beta} = \frac{f_\alpha}{f_\beta}\). Set \(s|_{U_\alpha} = s_\alpha\), so
  \[
    s_\alpha = \varphi_{\alpha\beta} s_\beta
  \]
  (this is elaborated upon in example sheet 3) then
  \[
    \frac{s_\alpha}{a_\beta} = \varphi_{\alpha\beta} = \frac{f_\alpha}{f_\beta}.
  \]
  Thus \(g\) defined by \(g|_{U_\alpha} = \frac{f_\alpha}{s_\alpha}\) is a well-defined global meromorphic function on \(X\), as \(\frac{f_\alpha}{s_\alpha} = \frac{f_\beta}{s_\beta}\) on \(U_\alpha \cap U_\beta\). Then \(D = (g)\) since the \(s_\alpha\)'s are nowhere vanishing.
\end{proof}

\begin{ex}
  Show that there is an exact sequence sequence
  \[
    \begin{tikzcd}
      0 \ar[r] & \mathcal O^* \ar[r] & \mathcal K^* \ar[r] & \mathcal K^*/\mathcal O^* \ar[r] & 0
    \end{tikzcd}
  \]
  and use the long exact sequence in cohomology to give another proof of the above. See example sheet 3.
\end{ex}

\begin{proposition}
  To any \(0 \neq s \in H^0(X, L)\) there is an associated \(Z(s) \in \Div(X)\).
\end{proposition}

\begin{proof}
  Fix a trivialisation \(\{(U_\alpha, \varphi_\alpha)\}\) for \(\pi: L \to X\), so \(\varphi_\alpha: \pi^{-1}(U_\alpha) \to U_\alpha \times \C\) is an isomorphism with cocycles \(\{(U_\alpha, \varphi_{\alpha\beta})\}\). Set
  \[
    f_\alpha = \varphi_\alpha(s|_{U_\alpha}) \in \mathcal O(U_\alpha)
  \]
  not identically zero. We thus have
  \[
    f_\alpha f_\beta^{-1}
    = \varphi_\alpha(s|_{U_\alpha}) \varphi(s|_{U_\beta})^{-1}
    = \varphi_{\alpha\beta} \in \mathcal O^*(U_\alpha \cap U_\beta).
  \]
  Thus one obtains \(Z(s) \in \Div(X)\) as \(\{(U_\alpha, f_\alpha)\}\).
\end{proof}
In addition \(Z(s_1 + s_2) = Z(s_1) + Z(s_2)\).

\begin{proposition}\leavevmode
  \begin{enumerate}
  \item Let \(0 \neq s \in H^0(X, L)\). Then
    \[
      \mathcal O(Z(s)) \cong L.
    \]
  \item If \(D\) is effective then exists \(0 \neq s \in H^0(X, \mathcal O(D))\) with \(Z(s) = D\).
  \end{enumerate}
\end{proposition}

\begin{proof}\leavevmode
  \begin{enumerate}
  \item Let \(L\) have trivialisations \(\{(U_\alpha, \varphi_\alpha)\}\). Then \(Z(s)\) is given by \(f \in H^0(X, \mathcal K^*/\mathcal O^*)\) where
    \[
      f_\alpha = f|_{U_\alpha} = \varphi_\alpha(s|_{U_\alpha}).
    \]
    Then \(\mathcal O(Z(s))\) is associated to its cocycle \(\{(U_\alpha, f_\alpha f_\beta^{-1})\}\). But
    \[
      f_\alpha f_\beta^{-1}
      = \varphi_\alpha(s|_{U_\alpha}) \varphi_\beta(s|_{U_\beta})^{-1}
      = \varphi_{\alpha\beta}
    \]
    as above.
  \item Let \(D \in \Div(X)\) be given by \(\{(U_\alpha, f_\alpha)\}\) where \(f_\alpha \in \mathcal K^*(U_\alpha)\). As \(D\) is effective, the \(f_\alpha\)'s are holomorphic. The line bundle \(\mathcal O(D)\) is associated to the cocycle \(\{(U_\alpha, \varphi_{\alpha\beta} = \frac{f_\alpha}{f_\beta})\}\). The \(f_\alpha \in \mathcal O(U_\alpha)\) glue to a global section \(s \in H^0(X, \mathcal O(D))\) as \(f_\alpha = \varphi_{\alpha\beta} f_\beta\).

    Moreover
    \[
      Z(s)|_{U_\alpha} = Z(s|_{U_\alpha}) = Z(f_\alpha) = D \cap U_\alpha.
    \]
  \end{enumerate}
\end{proof}

Note that \(s\) is not unique: if \(\lambda \in H^0(X, \mathcal O^*)\) (for example \(\lambda \in \C^*\)) then \(Z(\lambda s) = Z(s)\).

\begin{corollary}
  If \(0 \neq s \in H^0(X, L), 0 \neq s' \neq H^0(X, L')\) then
  \[
    Z(s) \sim Z(s')
  \]
  if and only if \(L \cong L'\).
\end{corollary}

\begin{proof}
  Follows as \(\mathcal O(Z(s)) \cong L\) and \(\mathcal O(D) \cong \mathcal O\) if and only if \(D\) is principal.
\end{proof}

We conclude this chapter by a few remarks that will be useful for the following chapter on Kähler geometry. Recall the exponential short exact sequence\index{sheaf!short exact sequence!exponential}
\[
  \begin{tikzcd}
    0 \ar[r] & \Z \ar[r, "\times 2\pi i"] & \mathcal O \ar[r, "\exp"] & \mathcal O^* \ar[r] & 0
  \end{tikzcd}
\]
which induces a long exact sequence in sheaf cohomology. In particular, as \(\Pic(X) \cong H^1(X, \mathcal O^*)\), we have a map
\[
  c_1: \Pic(X) \to H^2(X, \Z).
\]

\begin{definition}[first Chern class]\index{first Chern class}
  For \(L \in \Pic(X)\), we call \(c_1(L) \in H^2(X, \Z)\) the \emph{first Chern class} of \(L\).
\end{definition}
We'll return to Chern classes later.

Recall that \(X\) is projective\index{projective manifold} if it is biholomorphic to a closed submanifold of \(\P^m\) for some \(m\).

\begin{definition}[ample line bundle]\index{ample line bundle}\index{line bundle!ample}
  We say that a line bundle \(L\) on \(X\) is \emph{ample} if there is an embedding \(\iota: X \embed \P^m\) for some \(m\) and \(k \in \Z_{> 0}\) such that
  \[
    L^{\otimes k} \cong \iota^*(\mathcal O(1))
  \]
  where \(\mathcal O(1)\) is the hyperplane line bundle\index{hyperplane line bundle} on \(\P^m\).
\end{definition}

Kähler geometry (in part) gives a differential geometric interpretation of amplitude.

\section{Kähler manifolds}

Our goal is to put Riemannian metrics on complex manifolds which interact well with the complex structure. Just as complex structure, we begin by exploring the interaction of inner product and complex structure on a vector space.

Let \(V\) be a real vector space. Let \(J: V \to V\) be a complex structure and let \(\ip{\cdot, \cdot}\) be an inner product on \(V\).

\begin{definition}[fundamental form]\index{fundamental form}
  We say \(\ip{\cdot, \cdot}\) is  \emph{compatible} with \(J\) if
  \[
    \ip{Ju, Jv} = \ip{u, v}
  \]
  for all \(u, v \in V\). In this case the \emph{fundamental form} \(\omega\) is
  \[
    \omega(u, v) = \ip{Ju, v}.
  \]
\end{definition}
Note that \(\omega\) is antisymmetric:
\[
  \omega(u, v) = \ip{Ju, v} = \ip{-u, Jv} = - \omega(v, u).
\]

We now extend to the complexification \(V_\C = V \otimes_\R \C\). The inner product extends to a Hermitian inner product
\[
  \ip{\lambda u, \mu v}_\C = \lambda \conj \mu \ip{u, v}
\]
where \(\lambda, \mu \in \C, u, v \in V\) and using that any \(\alpha \in V_\C\) can be written as \(\alpha = \alpha_1 + i\alpha_2\) where \(\alpha_1, \alpha_2 \in V\). \(\omega\) extend to \(\omega\) (by abuse of notation) on \(V_\C\).

\begin{lemma}\leavevmode
  \begin{enumerate}
  \item The decomposition
    \[
      V_\C = V^{1, 0} \oplus V^{0, 1}
    \]
    is orthogonal with respect to \(\ip{\cdot, \cdot}_\C\).
  \item \(\omega \in \Lambda^{1, 1} V_\C^*\).
  \end{enumerate}
\end{lemma}

\begin{proof}\leavevmode
  \begin{enumerate}
  \item Take \(u \in V^{1, 0}, v \in V^{0, 1}\), so \(Ju = iu, Jv = -iv\) so
    \[
      \ip{u, v}_\C
      = \ip{Ju, Jv}_\C
      = \ip{iu, -iv}_\C
      = i^2 \ip{u, v}_\C
      = -\ip{u, v}_\C
    \]
    so must be \(0\).
  \item Take \(u, v \in V^{1, 0}\). Then
    \[
      \omega(u, v)
      = \omega(Ju, Jv)
      = \omega(iu, iv)
      = -\omega(u, v)
    \]
    so is \(0\). Similar for \(V^{0, 1}\).
  \end{enumerate}
\end{proof}

It is easy to see that this generalises in case of manifolds. Recall from III Differential Geometry

\begin{definition}[Riemannian metric]\index{Riemannian metric}
  A \emph{Riemannian metric} \(g\) on \(X\) is a section of \(T^*X \otimes T^*X\) such that for all \(x \in X\),
  \[
    g_x: T_xX \times T_xX \to \R
  \]
  is an inner product.
\end{definition}

\begin{definition}[fundamental form]\index{fundamental form}
  A Riemannian metric \(g\) is called \emph{compatible} with an almost complex structure \(J\) if for all \(x \in X\), the inner product \(g_x\) on \(T_xX\) is compactible with \(J_x: T_xX \to T_xX\). In this case one define the \emph{fundamental form} \(\omega\) by
  \[
    \omega(u, v) = g(Ju, v).
  \]
\end{definition}
\(\omega\) extends \(\C\)-linearly to \(\omega \in \Lambda^{1, 1} (T^*X)_\C\). The extension \(g_\C\) of \(g\) gives a \emph{hermitian metric}\index{hermitian metric} on \((TX)_\C\) and hence on \(TX^{1, 0}\).

Suppose on \(X\) we have holomorphic coordinates \(z_1, \dots, z_n\). Then \(\d z_1, \dots, \d z_n\) form a local holomorphic frame for \(T^*X^{1, 0}\). Let
\[
  h_{jk} = 2 g_\C(\frac{\partial  }{\partial z_j}, \frac{\partial  }{\partial z_k}).
\]

\begin{ex}
  Show that \((h_{jk})\) is a Hermitian matrix and
  \[
    \omega = \frac{i}{2} \sum_{j, k} h_{jk} \d z_j \w \d \conj{z_k}.
  \]
\end{ex}

\begin{definition}[Kähler form, Kähler class]\index{Kähler form}\index{Kähler class}
  We say that \(\omega\) is a \emph{Kähler form} or \emph{Kähler metric} if \(\d \omega = 0\). We say \([\omega] \in H^2(X; \R)\) is a \emph{Kähler class}.
\end{definition}

\begin{eg}\leavevmode
  \begin{enumerate}
  \item On \(\C^n\) with coordinates \(z_1, \dots, z_n\),
    \[
      \omega = \frac{i}{2} \sum_{j = 1}^n \d z_j \w \d \conj z_j
    \]
    is a Kähler metric.
  \item By a standard partition of unity argument, any complex manifold admits a hermitian metric. Alternatively, if \(g\) is any Riemannian metric then define
    \[
      \tilde g(u, v) = g(u, v) + g(Ju, Jv)
    \]
    which is compatible with \(J\), giving a hermitian metric. The only obstacle to being Kähler form is closedness. If \(\dim X = 1\) then every \((1, 1)\)-form is closed, giving lots of Kähler forms.
  \end{enumerate}
\end{eg}

Note that any two of \(g, J, \omega\) determine the third.

\begin{remark}
  For those taking III Symplectic Topology, any Kähler metric induces a symplectic form. Thus Kähler geometry lies in the intersection of complex geometry, Riemannian geometry and symplectic geometry.
\end{remark}

So far the requirement of closedness seems quite arbitrary, but we'll soon prove that all projective manifolds are Kähler.

\begin{eg}[Fubini-Study metric on \(\P^n\)]\index{Fubini-Study metric}
  Let \(U \subseteq \P^n\) be open and \(\pi: \C^{n + 1} \setminus \{0\} \to \P^n\) be the natural projection. Suppose \(s: U \to \C^{n + 1}\) is a holomorphic lift of \(\pi\), i.e.\ \(\pi(s(z)) = z\) for all \(z \in U\). Let \(U_j = \{[z_0 : \dots : z_n]: z_j \neq 0\}\). Then on \(U_j\),
  \[
    s([z_0: \dots : z_n]) = (\frac{z_0}{z_j}, \dots, \frac{z_{j - 1}}{z_j}, 1, \dots, \frac{z_n}{z_j}).
  \]
  Let
  \[
    \omega_{\text{FS}}|_U = \frac{i}{2\pi} \p \conj \p \log \norm s^2,
  \]
  where \(\norm \cdot\) is the Euclidean norm on \(\C^{n + 1}\). We need to check this is well-define, closed and positive definite.

  Choose another \(s'\) defined on \(U'\). Then \(s' = fs\) for some \(f \in \mathcal O^*(U \cap U')\), by the same argument as in the construction of line bundle and
  \begin{align*}
    \frac{i}{2\pi} \p \conj \p \log \norm{s'}^2
    &= \frac{i}{2\pi} \p \conj \p \log (|f|^2 \norm{s}^2) \\
    &= \frac{i}{2\pi} \p \conj \p (\log |f|^2 + \log \norm s^2) \\
    &= \omega_{\text{FS}}|_U
  \end{align*}
  as
  \[
    i \p \conj \p(\log f + \log \conj f) = 0.
  \]

  Next, note
  \[
    2 \omega_{\text{FS}}
    = \frac{i}{2\pi} (\p + \conj \p) (\conj \p - \p) \log \norm s^2
    = \frac{i}{2\pi} \d (\conj \p - \p) \log \norm s^2
  \]
  so
  \[
    \d \omega_{\text{FS}} = \frac{i}{4\pi} \d (\d (\conj \p - \p) \log \norm s^2) = 0.
  \]

  The tricky part is to show positive definiteness, which is a local condition. We locally write
  \[
    \omega_{\text{FS}} = \frac{i}{2} \sum h_{jk} \d z_j \w \d \conj z_k
  \]
  and need to show \((h_{jk})\) is a positive definite Hermitian matrix. We work on \(U_0\) (proof for \(U_j\) is identical). Set \(w_j = \frac{z_j}{z_0}\). Then
  \begin{align*}
    \omega_{\text{FS}}|_{U_0}
    &= \frac{i}{2\pi} \p \conj \p \log (1 + \sum |w_j|^2) \\
    &= \frac{i}{2\pi} \p \left( \frac{\sum w_j \d \conj w_j}{1 + \sum |w_j|^2} \right) \\
    &= \frac{i}{2\pi} \left( \frac{\sum \d w_j \w \d \conj w_j}{1 + \sum |w_j|^2} - \frac{(\sum \conj w_j \d w_j) \w (\sum w_k \d \conj w_k)}{(1 + \sum |w_j|^2)^2} \right) \\
    &= \frac{i}{2\pi} \left( \sum_{j,k} \frac{(1 + \sum |w_\ell|^2) \delta_{jk} - \conj w_j w_k}{(1 + \sum |w_\ell|)^2} \d w_j \w \d \conj w_k \right) \\
    &= \frac{i}{2\pi} h_{jk} \d w_j \w \d \conj w_k
  \end{align*}
  If \(0 \neq u \in \C^n\) then (ignoring the positive denominator)
  \begin{align*}
    u^T (h_{jk}) \conj u
    &= \ip{u, u} + \ip{w, w} \ip{u, u} - u^T \conj w w^T \conj u \\
    &= \ip{u, u} + \ip{w, w} \ip{u, u} - \ip{u, w} \ip{w, u} \\
    &= \ip{u, u} + \ip{w, w} \ip{u, u} - |\ip{w, u}|^2 \\
    &> 0
  \end{align*}
  By Cauchy-Schwarz.
\end{eg}

\begin{proposition}
  Let \((X, \omega)\) be a Kähler manifold. Then any complex submanifold \(\iota: U \embed X\) is Kähler.
\end{proposition}

\begin{proof}
  \[
    \d (\iota^* \omega) = \iota^* \d \omega = 0.
  \]
  Positive definiteness is clear.
\end{proof}

\begin{corollary}
  Any projective manifold is Kähler.
\end{corollary}

This is also precisely the intuition we should have when dealing with Kähler manifold: that is, they are the closest thing to projective manifold (the class of Kähler manifolds is strictly larger, but they share many similarities with projective manifolds).

Using the hermitian metric \(h = g_\C\) on \(TX^{1, 0}\), choose a unitary frame \(\{\varphi_1, \dots, \varphi_n\}\) of \(T^*X^{1, 0}\) on a neighbourhood \(U\) of \(x \in X\), so that
\[
  h = \sum \varphi_j \otimes \conj \varphi_j.
\]
Let \(\eta_j = \Re \varphi_j, \xi_j = \Im \varphi_j\). One checks
\[
  g
  = \Re (\sum (\eta_j + i \xi_j) \otimes (\eta_j - i \xi_j))
  = \sum \eta_j \otimes \eta_j + \xi_j \otimes \xi_j
\]
with volume form
\[
  \d \text{Vol} = \eta_1 \w \xi_1 \w \cdots \w \eta_n \w \xi_n.
\]
On the other hand
\[
  \omega
  = \frac{i}{2\pi} \sum (\eta_j + i \xi_j) \w (\eta_j - i \xi_j)
  = \frac{1}{2\pi} \sum \eta_j \w \xi_j
\]
so
\[
  \frac{\omega^n}{n!} = \d \text{Vol}
\]
(up to \(2\pi\)). Thus
\[
  \int_X \omega^n > 0
\]
when this is defined, for example when \(X\) is compact.

\begin{proposition}
  If \(X\) is compact Kähler then
  \[
    \dim H^{2q}_{\text{dR}} (X; \R) > 0.
  \]
\end{proposition}

\begin{proof}
  Let \(\omega\) be a Kähler metric and \(\tau = \omega^q\). Then \(\d \tau = 0\) as \(\d \omega = 0\) so \([\tau] \in H^{2q}_{\text{dR}} (X; \R)\). Suppose \(\tau = \d \sigma\) where \(\sigma \in \mathcal A^{2q - 1}_\R(X)\). Then
  \[
    \int_X \omega^n
    = \int_X \omega^{n - q} \w \tau
    = \int_X \d (\sigma \w \omega^{n - q})
    = 0
  \]
  by Stokes' theorem, a contradiction.
\end{proof}

Thus there is a topological obstruction for compact complex manifolds to be Kähler. For example Hopf surface on example sheet 3.

\begin{remark}
  We saw that every (smooth) projective manifold is Kähler. Recall that for \(L \in \Pic(X)\) we defined the first Chern class\index{first Chern class}
  \[
    c_1(L) \in H^2(X; \Z) \subseteq H^2(X; \R).
  \]
  The Kodaira embedding theorem states that on a compact complex manifold, a class \(\alpha \in H^2(X; \Z)\) is a Kähler class (i.e.\ there is a Kähler metric \(\omega \in \alpha\)) if and only if \(\alpha = c_1(L)\) for \(L \in \Pic(X)\) ample. This gives a complex differential geometric interpretation of ampleness and characterises which compact Kähler manifolds are projective.
\end{remark}

\begin{proposition}
  Let \(\omega\) be a \((1, 1)\)-form associated to a hermitian metric \(h\) on \(X\). Then \(\d \omega = 0\) if and only if for all \(x \in X\) there exist holomorphic coordinates \(z_1, \dots, z_n\) around \(x\) such that locally
  \[
    \omega = \frac{i}{2} \sum h_{jk} \d z_j \w \d \conj z_k
  \]
  with
  \[
    h_{jk} = \delta_{jk} + O(|z|^2).
  \]
  Thus \(\omega\) is Kähler if and only if \(\omega = \omega_0 + O(|z|^2)\) where \(\omega_0\) is the usual Kähler form on \(\C^n\).
\end{proposition}
This is analogous to the Riemannian geometric statement that we can choose a normal coordinates with respect to a Riemannian metric of this form.

\begin{proof}
  Let
  \[
    \omega = \frac{i}{2} \sum h_{jk} \d z_j \w \d \conj z_k.
  \]
  Then
  \[
    \d \omega
    = \frac{i}{2} \sum \frac{\partial h_{jk}}{\partial z_\ell} \d z_\ell \w \d z_j \w \d \conj z_k
    + \frac{i}{2} \sum \frac{\partial h_{jk}}{\partial \conj z_\ell} \d \conj z_\ell \w \d z_j \w \d \conj z_k
  \]
  Thus if \(h_{jk} = \delta_{jk} + O(|z|^2)\) then
  \[
    \frac{\partial h_{jk}}{\partial z_\ell}(x) = \frac{\partial h_{jk}}{\partial \conj z_\ell}(x) = 0
  \]
  so \(\d \omega = 0\).

  Conversely, suppose \(\d \omega = 0\) and write
  \[
    \omega = \frac{i}{2} \sum h_{jk} \d z_j \w \d \conj z_k.
  \]
  By a linear change of coordinates, we may assume
  \[
    h_{jk}(x) = \delta_{jk}.
  \]
  The Taylor series expansion looks like
  \[
    h_{jk} = \delta_{jk} + \sum_\ell a_{jk\ell} z_\ell + \sum_\ell b_{jk\ell} \conj z_\ell + O(|z|^2).
  \]
  As \(h\) is Hermitian, \(h_{jk} = \conj h_{kj}\). Thus \(b_{jk\ell} = \conj{a_{kj\ell}}\). As \(\d \omega = 0\),
  \[
    0 = \sum_{j, k, \ell} a_{jk\ell} \d z_\ell \w \d z_j \w \d \conj z_k
    + \sum_{j, k, \ell} b_{jk\ell} \d \conj z_\ell \w \d z_j \w \d \conj z_k.
  \]
  Thus
  \begin{align*}
    a_{jk\ell} &= a_{\ell k j} \\
    b_{jk\ell} &= b_{j \ell k}
  \end{align*}
  Now let
  \[
    \xi_k = z_k + \frac{1}{2} \sum a_{jk\ell} z_j z_\ell,
  \]
  a valid change of coordinates in a neighbourhood of \(x\). Then
  \begin{align*}
    \d \xi_k &= \d z_k + \frac{1}{2} \sum a_{jk\ell} (z_j \d z_\ell + z_\ell \d z_j) \\
    \d \conj \xi_k &= \d \conj z_k + \frac{1}{2} \sum \conj a_{jk\ell} (\conj z_j \d \conj z_\ell + \conj z_\ell \d \conj z_j)
  \end{align*}
  Now we compute their wedge product
  \begin{align*}
    \d \xi_k \w d \conj \xi_k
    &= \sum \d z_k \w \d \conj z_k \\
    &\quad + \frac{1}{2} \sum \conj a_{jk\ell} (\conj z_j \d z_k \w \d \conj z_\ell + \conj z_\ell \d z_k \w \d \conj z_j) \\
    &\quad + \frac{1}{2} \sum a_{jk\ell} (z_j \d z_\ell \w \d \conj z_k + z_\ell \d z_j \w \d \conj z_k)
      + O(|z|^2) \\
    &= \sum \d z_k \w \d \conj z_k \\
    &\quad + \sum a_{jk\ell} z_\ell \d z_j \w \d \conj z_k \\
    &\quad + \sum b_{jk\ell} \conj z_\ell \d z_k \w \d \conj z_j + O(|z|^2) \\
    &= \frac{2}{i} \omega + O(|z|^2)
  \end{align*}
\end{proof}

Thus any identity only involving the metric \(h\) and its first derivative, if true on \(\C^n\) with its usual Kähler metric, is true on any Kähler manifold. We'll use this several times.

\subsection{Kähler identities}

Let \((X, g)\) be an oriented Riemannian manifold of dimension \(2n\). The exterior derivative \(\d: \mathcal A^k \to \mathcal A^{k + 1}\) satisfies \(\d^2 = 0\). Let \(\d \text{Vol}\) be the volume form associated to \(g\). The \emph{Hodge star operator}\index{Hodge star operator} is
\[
  \star: \mathcal A^k \to \mathcal A^{2n - k}
\]
defined in such a way that
\[
  \alpha \w \star \beta = \ip{\alpha, \beta}_g \d \text{Vol}
\]
for \(\alpha, \beta \in \mathcal A^k\).

Set
\[
  \d^* = - \star \d \star: \mathcal A^k \to \mathcal A^{k - 1}.
\]
The \emph{Laplacian}\index{Laplacian} is
\[
  \Delta_\d = \d^* \d + \d \d^*: \mathcal A^k \to \mathcal A^k.
\]
Now suppose \(X\) is a complex manifold of dimension \(n\), with Riemannian metric \(g\) compatible with \(J\). Then the Hodge star operator extends naturally to
\[
  \star: \mathcal A_\C^k \to \mathcal A_\C^{2n - k}
\]
in such a way that
\[
  \alpha \w \star \beta = g_\C(\alpha, \beta) \d \text{Vol}.
\]
Write \(\d = \p + \conj \p\) with
\begin{align*}
  \p: \mathcal A_\C^{p, q} &\to \mathcal A_\C^{p + 1, q} \\
  \conj \p: \mathcal A_\C^{p, q} &\to \mathcal A_\C^{p, q + 1}
\end{align*}
We define
\begin{align*}
  \p^* &= - \star \p \star \\
  \conj \p^* &= - \star \conj \p \star
\end{align*}
and subsequently two more Laplacians
\begin{align*}
  \Delta_\p &= \p^* \p + \p \p^* \\
  \Delta_{\conj \p} &= \conj \p^* \conj \p + \conj \p \conj\p^*
\end{align*}

If \(\omega\) is Kähler, set
\begin{align*}
  L: \mathcal A_\C^{p, q} &\to \mathcal A_\C^{p + 1, q + 1} \\
  \alpha &\mapsto \alpha \w \omega
\end{align*}
This is the \emph{Lefschetz operator}\index{Lefschetz operator}. Finally set
\[
  \Lambda = \star^{-1} L \star: \mathcal A_\C^{p, q} \to \mathcal A_\C^{p - 1, q - 1}
\]
the \emph{inverse Lefschetz operator}\index{inverse Lefschetz operator} or sometimes the \emph{contraction operator}\index{contraction operator}.

\begin{remark}
  For \(\alpha \in \mathcal A_\C^k\),
  \[
    \star \star \alpha = (-1)^{k (2n - k)} \alpha
  \]
  so
  \[
    \star^{-1} = (-1)^{k (2n - k)} \star.
  \]
\end{remark}

The operators \(\p^*\) and \(\conj \p^*\) are adjoint to \(\p, \conj \p\) respectively with respect to \(L^2\) inner product, which is defined as
\[
  \ip{\alpha, \beta}_{L^2}
  = \int_X \alpha \w \star \beta
  = \int_X g_\C(\alpha, \beta) \d \text{Vol}.
\]

\begin{lemma}
  Suppose \(\alpha \in \mathcal A_\C^{p, q}, \beta \in \mathcal A_\C^{p + 1, q}\) then
  \[
    \ip{\p \alpha, \beta}_{L^2} = \ip{\alpha, \p^* \beta}_{L^2}.
  \]
  Similarly if \(\alpha \in \mathcal A_\C^{p, q}, \beta \in \mathcal A_\C^{p, q + 1}\) then
  \[
    \ip{\conj \p \alpha, \beta}_{L^2} = \ip{\alpha, \conj \p^* \beta}_{L^2}.
  \]
\end{lemma}

\begin{proof}
  We prove the first identity. By Stokes' theorem
  \[
    0
    = \int_X \d (\alpha \w \star \beta)
    = \int_X \p (\alpha \w \star \beta)
  \]
  because
  \[
    \alpha \w \star \beta \in \mathcal A_\C^{p + (n - (p + 1)), q + (n - q)} = \mathcal A_\C^{n - 1, n}
  \]
  so \(\conj \p (\alpha \w \star \beta) = 0\). Thus
  \[
    0
    = \int_X \p (\alpha \w \star \beta)
    = \int_X \p\alpha \w \star \beta + (-1)^k \alpha \w \p \star \beta
  \]
  where \(k = p + q\), thus
  \begin{align*}
    \ip{\p \alpha, \beta}_{L^2}
    &= \int_X \p \alpha \w \star \beta \\
    &= (-1)^{k + 1} \int_X \alpha \w \p \star \beta \\
    &= (-1)^{k + 1 + k(2n - k)} \int_X \alpha \w \star (\star \p \star) \beta \\
    &= \ip{\alpha, \p^* \beta}_{L^2}
  \end{align*}
  since \(k(2n - k + 1)\) is even.
\end{proof}

We now prove the \emph{Kähler identities}\index{Kähler identities}:
\begin{align*}
  [\conj \p^*, L] &= i \p, [\p^*, L] = -i \conj \p \\
  [\Lambda, \conj \p] &= - i \p^*, [\Lambda, \p] = i\conj \p^*
\end{align*}

We begin with \(\C^n\) equipped with the standard Kähler metric. We have
\begin{align*}
  \omega &= \frac{i}{2} \sum \d z_j \w \d \conj z_j \\
  g &= \frac{1}{2} \sum \d z_j \otimes \d \conj z_j
\end{align*}

We introduce some notations
\begin{definition}
  For \(\alpha \in \mathcal A_\C^k, \xi \in \mathcal A_\C^1\). Define \(\xi \vee \alpha \in \mathcal A_\C^{k - 1}\) by
  \[
    g_\C(\xi \vee \alpha, \beta) = g_\C(\alpha, \conj \xi \w \beta)
  \]
  for all \(\beta \in \mathcal A_\C^{k - 1}\).
\end{definition}
It is an exercise in linear algebra to check this exists and is well-defined, as \(g_\C\) is nondegenerate. For example in holomorphic coordinates,
\[
  \d z_1 \vee \alpha = \alpha( \frac{\partial  }{\partial z}, -).
\]

For today we write \(g_\C(\alpha, \beta) = \ip{\alpha, \beta}\).

\begin{definition}
  If \(\alpha \in \mathcal A_\C^k\), using multiindex notation, write
  \[
    \alpha = \sum_{|I| + |J| = k} \alpha_{IJ} \d z_I \w \d \conj z_J
  \]
  Define
  \begin{align*}
    \p_j \alpha &= \sum_{|I| + |J| = k} \frac{\partial \alpha_{IJ}}{\partial z_j} \d z_I \w \d \conj z_J \\
    \conj \p_j \alpha &= \sum_{|I| + |J| = k} \frac{\partial \alpha_{IJ}}{\partial \conj z_j} \d z_I \w \d \conj z_J
  \end{align*}
\end{definition}

\begin{lemma}
  \begin{align*}
    \d z_j \vee \d z_k &= 0 \\
    \d z_j \vee \d \conj z_k &= \delta_{jk}
  \end{align*}
\end{lemma}

\begin{proof}
  \begin{align*}
    \d z_j \vee \d z_k &= \ip{\d z_j, \d \conj z_k} = 0 \\
    \d z_j \vee \d \conj z_k &= \ip{\d z_j, \d z_k} = \delta_{ik}
  \end{align*}
\end{proof}

\begin{lemma}\leavevmode
  \begin{enumerate}
  \item \(\conj \p \alpha = \sum_j \d \conj z_j \w \conj \p_j \alpha\).
  \item \(\p_j \ip{\alpha, \beta} = \ip{\p_j, \beta} + \ip{\alpha, \conj \p_j \beta}\).
  \item \(\p_j (\d z_k \vee \alpha) = \d z_k \vee \p_j \alpha\).
  \end{enumerate}
\end{lemma}

\begin{proof}\leavevmode
  \begin{enumerate}
  \item Follows from definition of \(\conj \p\).
  \item Follows as the metric is the standard one so has no dependency on coordinate:
    \[
      \p_j \ip{\alpha, \beta}
      = \p_j \sum \alpha_{IJ} \conj \beta_{IJ}
      = \sum ((\p_j \alpha_{IJ}) \conj \beta_{IJ} + \alpha_{IJ} \p_j \conj \beta_{IJ}).
    \]
  \item Follows as \(\p_j\) commutes with \((\d z_k \vee -)\), since it commutes with \((\d \conj z_k \w -)\). Explicitly,
    \begin{align*}
      \ip{\p_j (\d z_k \vee \alpha), \beta}
      &= \p_j \ip{\d z_k \vee \alpha, \beta} - \ip{\d z_k \vee \alpha, \conj \p_j \beta} \\
      &= \p_j \ip{\alpha, \d \conj z_k \w \beta} - \ip{\alpha, \d \conj z_k \w \conj \p_j \beta} \\
      &= \ip{\p_j \alpha, \d \conj z_k \w \beta} \\
      &= \ip{\d z_k \vee \p_j \alpha, \beta}
    \end{align*}
  \end{enumerate}
\end{proof}

\begin{lemma}
  \[
    \conj \p^* \alpha = - \sum_j \d z_j \vee \p_j \alpha.
  \]
\end{lemma}

\begin{proof}
  Let \(\alpha \in \mathcal A_\C^k, \beta \in \mathcal A_\C^{k - 1}\) have compact support. Then
  \[
    \int_{\C^n} \p_j \ip{\d z_j \vee \alpha, \beta} \d \text{Vol} = 0
  \]
  by Stokes' theorem, with \(\d \text{Vol}\) being the stardard volume form, and \(\beta\) having compact support. Thus
  \begin{align*}
    0
    &= \int_{\C^n} \p_j \ip{\d z_j \vee \alpha, \beta} \d \text{Vol} \\
    &= \ip{\p_j (\d z_j \vee \alpha), \beta}_{L^2} + \ip{\d z_j \vee \alpha, \conj \p_j \beta}_{L^2} \\
    &= \ip{\d z_j \vee \p_j \alpha, \beta}_{L^2} + \ip{\d z_j \vee \alpha, \conj \p_j \beta}_{L^2}
  \end{align*}
  so
  \begin{align*}
    \ip{\conj \p^* \alpha, \beta}_{L^2}
    &= \ip{\alpha, \conj \p \beta}_{L^2} \qquad \text{adjoint relation} \\
    &= \sum \ip{\alpha, \d \conj z_j \w \conj \p_j \beta}_{L^2} \\
    &= \sum \ip{\d z_j \vee \alpha, \conj \p_j \beta}_{L^2} \\
    &= - \sum \ip{\d z_j \vee \p_j \alpha, \beta}_{L^2}
  \end{align*}
  This gives the result as it holds for all such \(\beta\).
\end{proof}

\begin{lemma}
  On \(\C^n\) with the standard metric,
  \[
    [\conj \p^*, L] = i\p.
  \]
\end{lemma}

\begin{proof}
  Give a form \(\alpha\),
  \[
    [\conj \p^*, L] \alpha
    = \conj \p^* L \alpha - L \conj \p^* \alpha
    = \conj \p^* (\omega \w \alpha) - \omega \w \conj \p^* \alpha.
  \]
  The first term is
  \begin{align*}
    \conj \p^* (\omega \w \alpha)
    &= - \sum \d z_j \vee \p_j (\omega \w \alpha) \quad \text{by the previous lemma} \\
    &= - \sum \d z_j \vee ((\p_j \omega \w \alpha) + \omega \w \p_j \alpha) \\
    \intertext{As \(\omega\) is the standard Kähler form, \(\p_j \omega = 0\).}
    &= - \frac{i}{2} \sum \d z_j \vee (\sum_k \d z_k \w \d \conj z_k \w \p_j \alpha) \\
    &= - \frac{i}{2} \sum_{j, k} \underbrace{(\d z_j \vee \d z_k)}_{= 0} \w \d \conj z_k \w \p_j \alpha \\
    &\quad + \frac{i}{2} \sum_{j, k} \d z_k \w \underbrace{(\d z_j \vee \d \conj z_k)}_{= \delta_{jk}} \w \p_j \alpha \\
    &\quad \underbrace{- \frac{i}{2} \sum_{j, k} \d z_k \w \d \conj z_k \w (\d z_k \vee \p_j \alpha)}_{= - \omega \w \sum_j \d z_j \vee \p_j \alpha} \\
    &= 0 + i \p \alpha + \omega \w \conj \p^* \alpha
  \end{align*}
  so indeed
  \[
    [\conj \p^*, L] \alpha
    = i \p \alpha + \omega \w \conj \p^* \alpha - \omega \w \conj \p^* \alpha
    = i \p \alpha.
  \]
\end{proof}

\begin{theorem}[Kähler identities]\index{Kähler identities}\leavevmode
  \begin{enumerate}
  \item \([\conj \p^*, L] = i \p\).
  \item \([\p^*, L] = -i \conj \p\).
  \item \([\Lambda, \conj \p] = - i \p^*\).
  \item \([\Lambda, \p] = i\conj \p^*\).
\end{enumerate}
\end{theorem}

\begin{proof}\leavevmode
  \begin{enumerate}
  \item As \(\omega\) is Kähler around any \(x \in X\) there are coordinates \(z_1, \dots, z_k\) in which
    \[
      \omega = \omega_0 + O(|z|^2)
    \]
    where \(\omega_0\) is the standard metric on \(\C^n\). As \([\conj \p^*, L]\) only involves the metric and the first derivative of its coefficients, this follows from the result for \(\C^n\).
  \item Conjugate 1 and notice that \(\omega\) is real.
  \item Adjoint of 1.
  \item Adjoint of 2.
  \end{enumerate}
\end{proof}

\begin{theorem}
  On a Kähler manifold \((X, \omega)\), we have
  \[
    \Delta_\d = 2 \Delta_\p = 2 \Delta_{\conj \p}.
  \]
\end{theorem}

\begin{remark}
  This is not true on arbitrary complex manifolds.
\end{remark}

\begin{proof}
  First we claim
  \begin{align*}
    \conj \p^* \p + \p \conj \p^* &= 0 \\
    \p^* \conj \p + \conj \p \p^* &= 0
  \end{align*}
  Kähler identities give
  \[
    \conj \p^* = -i [\Lambda, \p]
  \]
  then
  \begin{align*}
    \conj \p^* \p + \p \conj \p^*
    &= - i[\Lambda, \p] \p - i\p[\Lambda, \p] \\
    &= -i\Lambda \p \p + i\p \Lambda \p - i\p \Lambda \p + i \p \p \Lambda \\
    &= 0
  \end{align*}
  as \(\p^2 = 0\). Next we show
  \[
    \Delta_\d = \Delta_\p + \Delta_{\conj \p}.
  \]
  This is because
  \begin{align*}
    \Delta_\d
    &= \d^* \d + \d \d^* \\
    &= (\p^* + \conj \p^*) (\p + \conj \p) + (\p + \conj \p) (\p^* + \conj \p^*) \\
    &= \Delta_\p + \Delta_{\conj \p}
  \end{align*}
  as the cross terms cancel. Finally we next show
  \[
    \Delta_\p = \Delta_{\conj \p}.
  \]
  This is because
  \begin{align*}
    \Delta_\p
    &= \p\p^* + \p^*\p \\
    &= i\p[\Lambda, \conj \p] + i[\Lambda, \conj \p]\p \\
    &= i \p \Lambda \conj \p - i \p \conj \p \Lambda + i \Lambda \conj \p \p - i \conj \p \Lambda \p
  \end{align*}
  and similarly
  \begin{align*}
    \Delta_{\conj \p}
    &= \conj \p \conj \p^* + \conj \p^* \p \\
    &= -i \conj \p[\Lambda, \p] - i [\Lambda, \p] \conj \p \\
    &= \Delta_\p
  \end{align*}
\end{proof}

This theorem shows that no matter which (co)differential we choose, there is no ``weird'' Hodge theory on Kähler manifold as all three Laplacians coincide.

\begin{lemma}
  Let \(\alpha \in \mathcal A_\C^{p, q}(X), \beta \in \mathcal A_\C^{p - 1, q - 1}(X)\). Then
  \[
    g_\C(\alpha, L\beta) = g_\C(\Lambda \alpha, \beta).
  \]
\end{lemma}
So \(L\) is the adjoint of \(\Lambda\).

\begin{proof}
  \begin{align*}
    g_\C(L\alpha, \beta) \d \text{Vol}
    &= L\alpha \w \star \beta \\
    &= \omega \w \alpha \w \star \beta \\
    &= \alpha \w \omega \w \star \beta \\
    &= g_\C(\alpha, \star^{-1} L \star \beta) \d \text{Vol} \\
    &= g_\C(\alpha, \Lambda \beta) \d \text{Vol}
  \end{align*}
  as \(\Lambda = \star^{-1} L \star\).
\end{proof}

\begin{theorem}[Kähler identities II]\index{Kähler identities}
  Let \((X, \omega)\) be a Kähler manifold. Let \(\pi_k: \mathcal A_\C^* \to \mathcal A_\C^k\) be the projection and define the \emph{counting operator}\index{counting operator}
  \[
    H = \sum_{k = 0}^{2n} (n - k) \pi_k
  \]
  where \(2n\) is the real dimension of \(X\). Then
  \begin{enumerate}
  \item \(H, \Lambda, L\) commute with \(\Delta_\d\).
  \item
    \begin{align*}
      [\Lambda, L] &= H \\
      [H, L] &= -2L \\
      [H, \Lambda] &= 2\Lambda
    \end{align*}
  \end{enumerate}
\end{theorem}

\begin{proof}
  We first consider commutators with \(H\). By linearity, it suffices to prove these results for some \(\alpha \in \mathcal A_\C^{p, q}\) where \(p + q = k\). Then
  \[
    [H, \Delta_\d] \alpha
    = (n - k) \Delta_\d \alpha - \Delta_\d (n - k) \alpha
    = 0.
  \]
  Also
  \begin{align*}
    [H, L]\alpha
    &= HL\alpha - LH\alpha \\
    &= (n - (k + 2)) L\alpha - L(n - k) \alpha \\
    &= -2L\alpha
  \end{align*}
  Taking adjoints and using \(H = H^*\) as
  \[
    g_\C(H\alpha, \beta) = g_\C(\alpha, H\beta)
  \]
  gives
  \[
    [H, \Lambda] = 2\Lambda.
  \]

  Showing \([L, \Delta_\d] = 0\) is equivalent to asking \(\Delta_\d \omega = 0\) (i.e.\ \(\omega\) is harmonic) and this is on example sheet 3. As \(\Delta_\d = \Delta_\d^*\),
  \[
    [\Lambda, \Delta_\d] = 0.
  \]

  We show lastly that
  \[
    [\Lambda, L] = H.
  \]
  That is, if \(\alpha \in \mathcal A_\C^{p, q}\) where \(p + q = k\) then
  \[
    [\Lambda, L] \alpha = (n - k) \alpha.
  \]
  This identity has no derivatives, so holds for \((X, \omega)\) if it holds for \(\C^n\) with respect to the standard Kähler metric. We check this explicitly. When \(n = 1\) we have
  \[
    \Lambda(\frac{i}{2} g(z) \d z \w \d \conj z) = g(z)
  \]
  so the identity holds. In general, write
  \begin{align*}
    L &= \sum L_j \\
    L_j \alpha &= \frac{i}{2} \d z_j \w \d \conj z_j \w \alpha
  \end{align*}
  and \(\Lambda = \sum \Lambda_j\), where \(\Lambda_j = L_j^*\) removes \(\d z_j \w \d \conj z_j\) if \(\alpha\) has a \(\d z_j \w \d \conj z_j\) term and \(\Lambda_j \alpha = 0\) otherwise (up to an appropriate dimensional constant). Then
  \[
    [L_j, \Lambda_\ell] = 0
  \]
  if \(j \neq \ell\), so this reduces to (a small variant of) the one dimensional case. By linearity one reduces to
  \[
    \alpha = \frac{i}{2} \d z_j \w \d \conj z_j \w \hat \alpha
  \]
  where \(\hat \alpha \in \mathcal A_\C^{p - 1, q - 1}\), then
  \[
    [\Lambda_j, L_j] \alpha = (n - p - q) \alpha
  \]
  as in the one dimensional case.
\end{proof}

\begin{remark}
  See Huybrechts Proposition 1.2.26 for a proof which carefully keeps track of the constants.
\end{remark}

\section{Hodge Theory}

We wish to understand the Dolbeault cohomology groups\index{Dolbeault cohomology} \(H^{p, q}_{\conj \p}(X)\), and how they compare with the sheaf cohomology \(H^k (X, \C)\) where \(p + q = k\). We begin by picking canonical representatives of cohomology.

Recall that
\begin{definition}[harmonic form]\index{harmonic form}
  Given an oriented Riemannian manifold \((X, g)\), we define the space of \emph{harmonic forms} of degree \(k\) to be
  \[
    \mathcal H^k(X, g) = \{\alpha \in \mathcal A^k(X), \Delta_\d \alpha = 0\}.
  \]
\end{definition}

\begin{remark}
  On \(\R^n\) with the Euclidean metric, if \(f \in C^\infty(\R^n)\) then
  \[
    \Delta_\d f = \Delta f,
  \]
  the usual Laplacian. Thus \(\Delta_\d f = 0\) if and only if \(f\) is harmonic in the classical sense.
\end{remark}

\begin{lemma}
  Suppose \((X, \omega)\) is compact. \(\Delta_{\conj \p} \alpha = 0\) if and only if
  \[
    \conj \p \alpha = \conj \p^* \alpha = 0.
  \]
\end{lemma}

\begin{proof}
  Similar to that in Riemannian geometry. If \(\conj \p \alpha = \conj \p^* \alpha = 0\) then \(\Delta_{\conj \p} \alpha = 0\) by definition of \(\Delta_{\conj \p}\).

  Conversely, if \(\Delta_{\conj \p} \alpha = 0\) then
  \begin{align*}
    0
    &= \ip{\Delta_{\conj \p} \alpha, \alpha}_{L^2} \\
    &= \ip{(\conj \p^* \conj \p + \conj \p \conj \p^*) \alpha, \alpha}_{L^2} \\
    &= \norm{\conj \p \alpha}_{L^2}^2 + \norm{\conj \p^* \alpha}_{L^2}^2
  \end{align*}
  so \(\conj \p \alpha = \conj \p^* \alpha = 0\).
\end{proof}

Recall that if \((X, \omega)\) is Kähler then
\[
  \Delta_\d \alpha = 0 \iff \Delta_{\conj \p} \alpha = 0 \iff \Delta_\p \alpha = 0
\]
so we can define \emph{harmonic forms}\index{harmonic form} on \(X\) with respect to any of the Laplacian
\[
  \mathcal H^{p, q}_{\conj \p} (X, g) = \{\alpha \in \mathcal A_\C^{p, q}(X): \Delta_{\conj \p} \alpha = 0\}.
\]
Recall from III Differential Geometry
\begin{theorem}[Hodge decomposition for Riemannian manifolds]\index{Hodge decomposition}
  If \((X, g)\) is a compact Riemannian manifold then there is an \(L^2\)-orthogonal decomposition
  \begin{align*}
    \mathcal A^k(X)
    &\cong \mathcal H^k(X) \oplus \d \mathcal A^{k - 1}(X) \oplus \d^* \mathcal A^{k + 1}(X) \\
    &\cong \mathcal H^k(X) \oplus \Delta_\d(\mathcal A^k(X))
  \end{align*}
  The space \(\mathcal H^k(X)\) of harmonic forms is finite-dimensional.
\end{theorem}
The second isomorphism is because
\[
  \Delta_\d \mathcal A^k(X)
  = \d \d^* \mathcal A^k(X) \oplus \d^* \d \mathcal A^k(X)
  = \d \mathcal A^{k - 1}(X) \oplus \d^* \mathcal A^{k + 1}(X).
\]
For example if \(\alpha = \d \beta \in \d \mathcal A^{k - 1}(X), \beta = \beta_1 + \beta_2 + \beta_3 \) then
\[
  \d \beta = \d \beta_3 = \d \d^* \gamma
\]
for some \(\gamma\).

\begin{theorem}[Hodge deomposition for Kähler manifolds]\index{Hodge decomposition}
  If \((X, \omega)\) is a compact Kähler manifold then there is an \(L^2\)-orthogonal decomposition
  \begin{align*}
    \mathcal A_\C^{p, q}(X)
    &\cong \mathcal H_{\conj \p}^{p, q}(X) \oplus \conj \p \mathcal A_\C^{p, q - 1}(X) \oplus \conj \p^* \mathcal A_\C^{p, q + 1}(X) \\
    &\cong \mathcal H_\p^{p, q}(X) \oplus \p \mathcal A_\C^{p - 1, q}(X) \oplus \p^* \mathcal A_\C^{p + 1, q}(X)
  \end{align*}
\end{theorem}
Note that
\[
  \mathcal H_\p^{p, q}(X) = \mathcal H_{\conj \p}^{p, q}(X) = \mathcal H_\d^{p, q}(X)
\]
as
\[
  \Delta_\d = 2 \Delta_{\conj \p} = 2 \Delta_\p.
\]

\begin{remark}
  Just as in III Differential Geometry, we shall not prove this result. The proof uses techniques from elliptic PDE theory. See Griffiths-Harris chapter 0.6.
\end{remark}

\begin{corollary}
  The map
  \[
    \mathcal H_{\conj \p}^{p, q}(X) \to H_{\conj \p}^{p, q}(X)
  \]
  sending \(\alpha\) to its class is an isomorphism. That is, each class in \(H_{\conj \p}^{p, q}(X)\) is represented by a unique harmonic form.
\end{corollary}

\begin{proof}
  The map is well-defined: if \(\Delta_{\conj \p} \alpha = 0\) then \(\conj \p \alpha = 0\).

  We first show surjectivity. Let \(\alpha \in \mathcal A_\C^{p, q}(X)\) satisfy \(\conj \p \alpha = 0\). By Hodge decomposition we may write
  \[
    \alpha = \beta_1 + \conj \p \beta_2 + \conj \p^* \beta_3
  \]
  with \(\beta_1\) harmonic. Thus
  \[
    0 = \conj \p \alpha = \conj \p \conj \p^* \beta_3.
  \]
  But then
  \[
    0
    = \ip{\conj \p \conj \p^* \beta_3, \beta_3}_{L^2}
    = \ip{\conj \p^* \beta_3, \conj \p^* \beta_3}_{L^2}
    = \norm{\conj \p^* \beta_3}_{L^2}^2
  \]
  so \(\conj \p^* \beta_3 = 0\). So \(\alpha = \beta_1 + \conj \p \beta_2\) and
  \[
    [\alpha] = [\beta_1] \in H_{\conj \p}^{p, q}(X)
  \]
  with \(\beta_1\) harmonic.

  Now we show injectivity. Suppose \(\alpha \in \mathcal H_{\conj \p}^{p, q}(X)\) is harmonic with \(0 = [\alpha] \in H_{\conj \p}^{p, q}(X)\). Then \(\alpha = \conj \p \beta\). As \(\alpha\) is harmonic,
  \[
    0 = \conj \p^* \alpha = \conj \p^* \conj \p \beta
  \]
  so \(\conj \p \beta = 0\) by an \(L^2\) argument. Thus \(\alpha = 0\).
\end{proof}

\begin{corollary}
  The map
  \[
    \mathcal H_{\conj \p} ^k(X) \to H_{\textrm{dR}}^k(X; \C)
  \]
  is an isomorphism. That is each cohomology class is represented by a unique harmonic form.
\end{corollary}

\begin{proof}
  Same as before.
\end{proof}

\begin{remark}
  The vector spaces \(\mathcal H^{p, q}(X)\) (\(\cong H_{\conj \p}^{p, q}(X)\)) admits the following operations:
  \begin{enumerate}
  \item conjugation \(\alpha \mapsto \conj \alpha\) sends harmonic forms to harmonic forms (since \(\conj{\p} \conj \alpha = \conj{\p \alpha}\)), hence inducing an isomorphism
    \[
      \mathcal H^{p, q}(X) \cong \mathcal H^{q, p}(X).
    \]
    We used Kähler identities (\(\Delta_\p \alpha = 0\) if and only if \(\Delta_{\conj \p} \alpha = 0\)) and this is not true for arbitrary compact complex manifolds.
  \item Hodge star operator \(\alpha \mapsto \star \alpha\) sends harmonic forms to harmonic forms (since \(\p^* \star \alpha = - \star \p \alpha = 0\) if \(\alpha\) is harmonic), hence inducing an isomorphism
    \[
      \mathcal H^{p, q}(X) \cong \mathcal H^{n - p, n - q}(X).
    \]
  \item another way to see this is Serre duality\index{Serre duality}: consider the pairing
    \begin{align*}
      \mathcal H^{p, q}(X) \times \mathcal H^{n - p, n - q}(X) &\to \C \\
      (\alpha, \beta) &\mapsto \int_X \alpha \w \beta
    \end{align*}
    if \(\alpha \neq 0\) then
    \[
      (\alpha, \star \conj \alpha) \mapsto \int_X \alpha \w \star \conj \alpha > 0
    \]
    giving an isomorphism
    \[
      \mathcal H^{p, q}(X) \cong \mathcal H^{n - p, n - q}(X).
    \]
  \item Lefschetz operator
    \begin{align*}
      L: \mathcal A_\C^{p, q}(X) &\to \mathcal A_\C^{p + 1, q + 1}(X) \\
      \alpha &\mapsto \omega \w \alpha
    \end{align*}
    It satisfies \([L, \Delta_{\conj \p}] = 0\), giving a map
    \[
      L: \mathcal H^{p, q}(X) \to \mathcal H^{p + 1, q + 1}(X).
    \]
    We will revisit this shortly.
  \end{enumerate}
  These induce symmetries and pairings on Dolbeault cohomology groups using the canonical isomorphism
  \[
    \mathcal H_{\conj \p}^{p, q}(X) \cong H_{\conj \p}^{p, q}(X).
  \]
\end{remark}

Denote \(h^{p, q} = \dim H_{\conj \p}^{p, q}(X)\). This is finite as \(X\) is compact. The \emph{Hodge diamond}\index{Hodge diamond} is the array
\[
  \begin{array}{ccccccccc}
    &&&& h^{0, 0} \\
    &&& h^{0, 1} && h^{1, 0} \\
    && h^{0, 2} && h^{1, 1} && h^{2, 0} \\
    & \iddots &&& \vdots &&& \ddots \\
    h^{0, n} && \cdots && h^{n, n} && \cdots && h^{n, 0} \\
    & \ddots &&& \vdots &&& \iddots \\
    && h^{n, n - 2} && h^{n - 1, n - 1} && h^{n - 2, n} \\
    &&& h^{n, n - 1} && h^{n - 1, n} \\
    &&&& h^{n, n}
  \end{array}
\]
The rows are symmetric by conjugation and the columns are symmetric by the Hodge star operator.

\begin{theorem}
  Let \((X, \omega)\) be compact Kähler. Then there is a decomposition
  \[
    H_{\text{dR}}^k(X; \C) = H^k(X, \C) \cong \bigoplus_{p + q = k} H^{p, q}_{\conj \p}(X)
  \]
  \emph{independent} of the chosen Kähler metric.
\end{theorem}

\begin{proof}
  The decomposition is induced by the Hodge decomposition
  \[
    H_{\text{dR}}^k(X; \C)
    \cong \mathcal H_{\conj \p}^k(X)
    \cong \bigoplus_{p + q = k} \mathcal H_{\conj \p}^{p, q}(X)
    \cong \bigoplus_{p + q = k} H_{\conj \p}^{p, q}(X).
  \]
  We must show that this decomposition is independent of chosen \(\omega\). It suffices to show that if
  \begin{align*}
    \alpha_1 &\in \mathcal H_{\conj \p}^{p, q}(X, \omega_1) \\
    \alpha_2 &\in \mathcal H_{\conj \p}^{p, q}(X, \omega_2)
  \end{align*}
  with \([\alpha_1] = [\alpha_2] \in H_{\conj \p}^{p, q}(X)\)  then \([\alpha_1] = [\alpha_2] \in H_{\text{dR}}^k(X; \C)\). Write \(\alpha_1 = \alpha_2 + \conj \p \gamma\) for some \(\gamma\). As \(\alpha_1, \alpha_2\) are \(\Delta_\d\)-harmonic, they are \(\d\)-closed (which is independent of the Kähler metric) so
  \[
    \d(\conj \p \gamma) = \d(\alpha_1 - \alpha_2) = 0.
  \]
  Then \(\conj \p \gamma\) is \(L^2\)-orthogonal to \(\mathcal H_{\conj \p}^{p, q}(X, \omega)\) by Kähler Hodge decomposition. As \(\mathcal H_{\conj \p}^k(X, \omega) = \mathcal H^k_\d(X, \omega)\), \(\conj \p \gamma\) is orthogonal to \(\mathcal H_\d^k(X, \omega)\).

  Since
  \[
    \ip{\conj \p \gamma, \d^* \varphi} = 0
  \]
  for all \(\varphi\), so \(\conj \p \gamma \in \d \mathcal A^{k + 1}\). Thus by Riemannian Hodge decomposition \(\conj \p \gamma \in \d \mathcal A^{k - 1}(X)\). Thus \([\alpha_1] = [\alpha_2] \in H_{\text{dR}}^k(X; \C)\).
\end{proof}

\section{Hermitian vector bundles}

Let \(E \to X\) be a complex vector bundle over a complex manifold \(X\).

\begin{definition}
  We define
  \[
    \mathcal A_\C^k(E)(U) = \mathcal A_\C^k(U) \otimes C^\infty(E)(U),
  \]
  where \(C^\infty(E)(U)\) denotes the smooth sections of \(E\).
\end{definition}

We have a splitting
\[
  \mathcal A_\C^k(E) = \bigoplus_{p + q = k} \mathcal A_\C^{p, q}(E)
\]
arising from the splitting \(\mathcal A_\C^k(U) = \bigoplus \mathcal A_\C^{p, q}(U)\) as \((p, q)\)-forms.

\begin{definition}[hermitian metric]\index{hermitian metric}
  A \emph{hermitian metric} \(h\) on \(E\) is a smooth varying hermitian metric \(h_x\) on the fibre \(E_x\) over \(x \in X\).
\end{definition}
If \(e_1, \dots, e_r\) is a local frame for \(E\) (of rank \(r\)), then \([h_{jk} = h(e_j, e_k)]\) is a hermitian matrix for each \(x\) whose coefficients vary smoothly in \(x\). As in the smooth case, a partition of unity argument produces hermitian metrics on any complex vector bundle.

\begin{ex}
  If \(E, F\) are given hermitian metrics then \(E \oplus F, E \otimes F, E^*, \Lambda^j E\) all admit natural hermitian metrics.
\end{ex}

We now take \(E\) to be holomorphic.

\begin{proposition}
  There is a natural \(\C\)-linear operator
  \[
    \conj \p_E: \mathcal A_\C^{p, q}(E) \to \mathcal A_\C^{p, q + 1}(E)
  \]
  satisfying
  \[
    \conj \p_E (\alpha \otimes s) = (\conj \p \alpha) \otimes s + \alpha \otimes \conj \p_E s
  \]
  for all \(\alpha \in \mathcal A_\C^{p, q}(U), s \in C^\infty(E)(U)\).
\end{proposition}

\begin{proof}
  In a local holomorphic frame \(e_1, \dots, e_r\) we define
  \[
    \conj \p_E (\alpha \otimes e_j) = \conj \p \alpha \otimes e_j.
  \]
  To see this is well-defined, let \(e_j' = \sum_{\ell = 1}^r \varphi_{j\ell} e_\ell\) be another local holomorphic frame so that the \(\varphi_{j \ell}\) are local holomorphic functions. Then
  \begin{align*}
    \conj \p_E(\alpha \otimes \sum_\ell \varphi_{j\ell} e_\ell)
    &= \sum_\ell \varphi_{j\ell} \conj \p \alpha \otimes e_\ell \\
    &= \conj \p \alpha \otimes \sum \varphi_{j \ell} e_\ell \\
    &= \conj \p \alpha \otimes e_j'
  \end{align*}
\end{proof}

\begin{definition}[connection]\index{connection}
  A \emph{connection} on a complex vector bundle is a sheaf morphism
  \[
    D: \mathcal A_\C^0(E) \to \mathcal A_\C^1(E)
  \]
  such that
  \[
    D(fs) = \d f \otimes s + f Ds
  \]
  where \(f \in C^\infty(E)(U), s \in \mathcal A_\C^0(E)(U)\).
\end{definition}
If \(e_1, \dots, e_r\) is a local frame for \(E\), this gives a connection matrix
\[
  D e_j = \sum \Theta_{j\ell} e_\ell
\]
where \(\Theta = (\Theta_{j\ell})\) is a matrix of \(1\)-forms.

A connection may be compatible with holomorphic structure or with hermitian structure. We will then prove that there is a unique connection compatible with both.

\begin{definition}[connection compatible with holomorphic structure]
  Let \(E\) be a holomorphic vector bundle. We define
  \begin{align*}
    D': \mathcal A_\C^0(E) &\to \mathcal A_\C^{1, 0}(E) \\
    D'': \mathcal A_\C^0(E) &\to \mathcal A_\C^{0, 1}(E)
  \end{align*}
  by \(D = D' + D''\). We say \(D\) is \emph{compatible} with the holomorphic structure if
  \[
    D'' = \conj \p_E: \mathcal A_\C^0(E) \to \mathcal A_\C^{0, 1}(E).
  \]
\end{definition}

\begin{proposition}
  A connection \(D\) on \(E\) is compatible with the holomorphic structure if and only if for all local holomorphic frames, the connection matrix \((\Theta_{j\ell})\) is given by \((1, 0)\)-forms.
\end{proposition}
This gives a local characterisation of compatibility.

\begin{proof}
  Suppose \(D\) is compatible. Then the \((0, 1)\)-part of \((\Theta_{j\ell})\) vanishes as
  \[
    D e_j = \sum \Theta_{j\ell} e_\ell
  \]
  and \(e_\ell\)'s are holomorphic.

  Conversely, if \(e_1, \dots, e_r\) is a local frame and \(\alpha_j \in C^\infty(U)\) then
  \[
    D(\sum \alpha_j e_j) = \sum \d \alpha_j \otimes e_j + \alpha_j D e_j
  \]
  and projecting to the \((0, 1)\)-part,
  \[
    D''(\sum \alpha_j e_j) = \sum \conj \p \alpha_j \otimes e_j.
  \]
  But this is our local expression for \(\conj \p_E\).
\end{proof}

\begin{definition}[connection compactible with hermitian structure]
  Let \((E, h)\) be a hermitian vector bundle. We say \(D\) is \emph{compatible} with \(h\) if
  \[
    \d(\alpha, \beta)_h = (D\alpha, \beta)_h + (\alpha, D\beta)_h
  \]
  where \(\alpha, \beta \in \mathcal A_\C^0(E)\).
\end{definition}

\begin{proposition}
  A connection \(D\) on \((E, h)\) is compatible with \(h\) if and only if for every unitary frame \(e_1, \dots, e_r\), the connection matrix is skew-Hermitian, i.e.
  \[
    \Theta_{j\ell} = - \conj{\Theta_{\ell j}}.
  \]
\end{proposition}

\begin{proof}
  If \(e_1, \dots, e_r\) is an unitary frame, then \((e_j, e_\ell)_h = \delta_{j \ell}\). Then
  \begin{align*}
    0
    &= \d (e_j, e_\ell)_h \\
    &= (De_j, e_\ell)_h + (e_j, De_\ell)_h \\
    &= (\sum \Theta_{jk} e_k, e_\ell)_h + (e_j, \sum \Theta_{\ell k} e_k)_h \\
    &= \Theta_{j\ell} + \conj{\Theta_{\ell j}}
  \end{align*}

  Conversely, suppose \((\Theta_{j\ell})\) is skew-Hermitian in any unitary frame. It suffices to show
  \[
    \d(\alpha, \beta)_h = (D\alpha, \beta)_h + (\alpha, D\beta)_h
  \]
  locally. This holds by above when \(\alpha, \beta \in \{e_1, \dots, e_r\}\). Thus it suffices to show
  \[
    \d (f \alpha, \beta)_h = (D(f\alpha), \beta)_h + (f\alpha, D\beta)_h.
  \]
  LHS is
  \begin{align*}
    \d(f\alpha, \beta)_h
    &= \d f \otimes (\alpha, \beta)_h + f \d (\alpha, \beta)_h \\
    &= \d f \otimes (\alpha, \beta)_h + f((D\alpha, \beta)_h + (\alpha, D\beta)_h)
  \end{align*}
  RHS is
  \begin{align*}
    (D(f\alpha), \beta)_h + (f\alpha, D\beta)_h
    &= (\d f \otimes \alpha, \beta)_h + (f D\alpha, \beta)_h + (f\alpha, D\beta)_h \\
    &= \d f \otimes (\alpha, \beta)_h + f(D\alpha, \beta)_h + f(\alpha, D\beta)_h
  \end{align*}
\end{proof}

\begin{proposition}
  Let \((E, h)\) be a hermitian vector and holomorphic vector bundle. Then there is a unique connection compatible with both structures.
\end{proposition}

\begin{definition}[Chern connection]\index{Chern connection}
  This connection is called the \emph{Chern connection}.
\end{definition}

\begin{remark}
  In practice, one typically has a hermitian holomorphic vector bundle, and the Chern connection can be seen as the ``canonical'' extra information.
\end{remark}

\begin{proof}
  We begin with uniqueness. Let \(e_1, \dots, e_r\) be a local holomorphic frame (not necessarily unitary) and let
  \[
    h_{ij} = h(e_j, e_k).
  \]
  Define the connection matrix by
  \[
    De_j = \sum_k \Theta_{jk} e_k.
  \]
  Then
  \begin{align*}
    \d h_{jk}
    &= \d h(e_j, e_k) \\
    &= (\sum_\ell \Theta_{j\ell} e_\ell, e_k)_h + (e_j, \sum_\ell \Theta_{k\ell} e_\ell)_h \\
    &= \sum \Theta_{j\ell} h_{\ell k} + \sum \conj{\Theta_{k\ell}} h_{j\ell}
  \end{align*}
  As \(D\) is compatible with the holomorphic structure, \((\Theta_{j\ell})\) is a matrix of \((1, 0)\)-forms. So
  \begin{align*}
    \p h_{jk} &= \sum \Theta_{j\ell} h_{\ell k} \\
    \conj \p h_{jk} &= \sum \conj{\Theta_{k\ell}} h_{j\ell}
  \end{align*}
  thus \(\Theta =  \p h \cdot h^{-1}\). This gives uniqueness.

  This also constructs such a connection on each trivialisation. By uniqueness, these local connections glue to a connection on \((E, h)\).
\end{proof}

\begin{lemma}
  If \(D_1, D_2\) are two connections on a complex vector bundle, then \(D_1 - D_2\) is \(\mathcal A_\C^0\)-linear, hence gives an element of \(\mathcal A^1_\C(\End)\). If \(D\) is a connection on \(E\) and \(a \in \mathcal A_\C^1(\End E)\) then \(D + a\) is a connection.
\end{lemma}

\begin{proof}
  Using that \(\d f \otimes s\) cancel in the definition, we have
  \[
    (D_1 - D_2)(fs) = f D_1 s - f D_2 s.
  \]
  \(a \in \mathcal A_\C^1(\End E)\) acts on \(\mathcal A^0_\C(E)\) by multiplication in the form part and evaluation in the \(E\) component (\(E \times \End E \to E\)). Then
  \[
    (D + a)(fs)
    = D(fs) + a(fs)
    = \d f \otimes s + f Ds + fas
    = \d f \otimes s + f(D + a) s
  \]
  so \(D + a\) is a connection.
\end{proof}
Thus given a connection \(D\), any other connection is given by the sum of \(D\) with an element of \(\mathcal A_\C^1(\End)\).

A connection extends to
\[
  D: \mathcal A^p_\C(E) \to \mathcal A^{p + 1}_\C(E)
\]
by
\[
  D(\alpha \otimes s) = \d \alpha \otimes + (-1)^p \alpha \w Ds
\]
for \(\alpha \in \mathcal A_\C^p(U), s \in C^\infty(E)(U)\).

\begin{definition}[curvature]\index{curvature}
  The \emph{curvature} of \(D\) is the map
  \[
    F_D = D \compose D: \mathcal A^0_\C(E) \to \mathcal A^2_\C(E).
  \]
\end{definition}

\begin{lemma}
  \(F_D\) is \(C^\infty\)-linear.
\end{lemma}

\begin{proof}
  For \(f \in C^\infty_\C(u), s \in \mathcal A_\C^0(E)(U)\),
  \begin{align*}
    F_D(fs)
    &= D(\d f \otimes s + f Ds) \\
    &= \d^2 f \otimes s - \d f \otimes Ds + \d f \otimes Ds + fD^2s \\
    &= fD^2s \\
    &= fF_D(s)
  \end{align*}
\end{proof}

\begin{corollary}
  \(F_D\) is induced by an element of \(\mathcal A_\C^2(\End E)\).
\end{corollary}

Let \(e_1, \dots, e_r\) be a local frame. Let \(\Theta\) be the connection matrix defined by \(D e_j = \sum \Theta_{jk} e_k\) where \(\Theta_{jk}\)'s are \(1\)-forms. Given a local section \(s = \sum s_j e_j\), we have
\[
  Ds = \sum \d s_j \otimes e_j + \sum s_j \Theta_{jk} e_k.
\]
We write this as
\[
  D = \d + \Theta.
\]
In this notation we can also write down the expression for curvature. Have
\begin{align*}
  F_D s
  &= D^2s \\
  &= (\d + \Theta)(\d + \Theta)s \\
  &= \d^2 + (\d \Theta)s - \Theta(\d s) + \Theta(\d s) + \Theta \w \Theta s \\
  &= (\d \Theta + \Theta \w \Theta) s
\end{align*}

\begin{lemma}\leavevmode
  \begin{enumerate}
  \item If \((E, h)\) is hermitian and \(D\) is compatible with \(h\) then
    \[
      h(F_D s_j, s_k) + h(s_j, F_D s_k) = 0.
    \]
  \item If \(E\) is holomorphic and \(D\) is compatible with the holomorphic structure then \(F_D\) has no \((0, 2)\)-component, i.e.
    \[
      F_D \in \mathcal A_\C^{2, 0}(\End E) \oplus \mathcal A_\C^{1, 1}(\End E).
    \]
  \item If \(D\) is the Chern connection then \(F_D\) is a skew-Hermitian form in \(\mathcal A_\C^{1, 1}(\End E)\).
  \end{enumerate}
\end{lemma}

\begin{proof}\leavevmode
  \begin{enumerate}
  \item The statement is local so let \(e_1, \dots, e_r\) be a local unitary frame, \(D = \d + \Theta\) with \(\Theta^* = - \Theta\). We have
    \begin{align*}
      (\d \Theta + \Theta \w \Theta)^*
      &= (\d \Theta)^* - \Theta^* \w \Theta^* \\
      &= \d \Theta^* - \Theta^* \w \Theta^* \\
      &= -\d \Theta - \Theta \w \Theta \\
      &= -F_D
    \end{align*}
  \item \(D: \mathcal A_\C^k(E) \to \mathcal A_\C^{k + 1}(E)\) splits as \(D = D' + D''\). Then \(D'' = \conj \p_E\) by hypothesis. Thus
    \[
      D \compose D
      = (D' + \conj \p_E) \compose (D' + \conj \p_E)
      = D' \compose D' + D' \compose \conj \p_E + \conj \p_E D' + \underbrace{\conj \p_E^2}_{= 0}
    \]
    so the \((0, 2)\)-component vanishes.
  \item Follows from 1 and 2.
  \end{enumerate}
\end{proof}

From now on we focus on line bundles. Let \((L, h)\) be a hermitian holomorphic line bundle and \(D\) be the Chern connection. Then \(F_D \in \mathcal A_\C^{1, 1}(\End L)\) is skew-Hermitian, so \(F_D\) is a real \((1, 1)\)-form. In this case
\begin{align*}
  \Theta &= \p \log h = h^{-1} \p h \\
  F_D &= \conj \p \log h
\end{align*}
We can interpret Fubini-Study metric\index{Fubini-Study metric} now. If \(X = \P^n\) and \(L = \mathcal O(1)\), there is a natural hermitian metric on \(\mathcal O(-1)\) arising from the usual hermitian metric on \(\C^{n + 1}\). This induces a hermitian metric on \(L = \mathcal O(1)\). Then on \(U_0 = \{[z_0: \cdots z_n]: z_0 \neq 0\}\)
\[
  \omega_{\text{FS}} = \frac{i}{2\pi} \p \conj \p \log (1 + \sum |z_j|^2)
\]
(\(z_0 = 1\)) which is \(\frac{i}{2\pi} F_D\) where \(F_D\) is the curvature of the natural hermitian metric on \(\mathcal O(1)\).

\begin{definition}[positive]
  We say that \(L\) is \emph{positive} if there is a hermitian metric \(h\) on \(L\) such that \(\frac{i}{2\pi}F_D\), where \(F_D\) is the curvature of the Chern connection, is a Kähler metric on \(X\).
\end{definition}

\begin{ex}
  Show that \([\frac{i}{2\pi} F_D] \in H^2(X, \C)\) is equal \(c_1(L)\), the first Chern class\index{first Chern class} of \(L\).
\end{ex}
Moreover, \(L\) is positive if and only if \(c_1(L)\) is a Kähler class, i.e.\ admits a Kähler metric.







 












\printindex
\end{document}

% https://www.dpmms.cam.ac.uk/~rd430/teaching.html