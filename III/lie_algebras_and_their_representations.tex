\documentclass[a4paper]{article}

\def\npart{III}

\def\ntitle{Lie Algebras and Their Representations}
\def\nlecturer{I.\ Grojnowski}

\def\nterm{Michaelmas}
\def\nyear{2019}

\ifx \nauthor\undefined
  \def\nauthor{Qiangru Kuang}
\else
\fi

\ifx \ntitle\undefined
  \def\ntitle{Template}
\else
\fi

\ifx \nauthoremail\undefined
  \def\nauthoremail{qk206@cam.ac.uk}
\else
\fi

\ifx \ndate\undefined
  \def\ndate{\today}
\else
\fi

\title{\ntitle}
\author{\nauthor}
\date{\ndate}

%\usepackage{microtype}
\usepackage{mathtools}
\usepackage{amsthm}
\usepackage{stmaryrd}%symbols used so far: \mapsfrom
\usepackage{empheq}
\usepackage{amssymb}
\let\mathbbalt\mathbb
\let\pitchforkold\pitchfork
\usepackage{unicode-math}
\let\mathbb\mathbbalt%reset to original \mathbb
\let\pitchfork\pitchforkold

\usepackage{imakeidx}
\makeindex[intoc]

%to address the problem that Latin modern doesn't have unicode support for setminus
%https://tex.stackexchange.com/a/55205/26707
\AtBeginDocument{\renewcommand*{\setminus}{\mathbin{\backslash}}}
\AtBeginDocument{\renewcommand*{\models}{\vDash}}%for \vDash is same size as \vdash but orginal \models is larger
\AtBeginDocument{\let\Re\relax}
\AtBeginDocument{\let\Im\relax}
\AtBeginDocument{\DeclareMathOperator{\Re}{Re}}
\AtBeginDocument{\DeclareMathOperator{\Im}{Im}}
\AtBeginDocument{\let\div\relax}
\AtBeginDocument{\DeclareMathOperator{\div}{div}}

\usepackage{tikz}
\usetikzlibrary{automata,positioning}
\usepackage{pgfplots}
%some preset styles
\pgfplotsset{compat=1.15}
\pgfplotsset{centre/.append style={axis x line=middle, axis y line=middle, xlabel={$x$}, ylabel={$y$}, axis equal}}
\usepackage{tikz-cd}
\usepackage{graphicx}
\usepackage{newunicodechar}

\usepackage{fancyhdr}

\fancypagestyle{mypagestyle}{
    \fancyhf{}
    \lhead{\emph{\nouppercase{\leftmark}}}
    \rhead{}
    \cfoot{\thepage}
}
\pagestyle{mypagestyle}

\usepackage{titlesec}
\newcommand{\sectionbreak}{\clearpage} % clear page after each section
\usepackage[perpage]{footmisc}
\usepackage{blindtext}

%\reallywidehat
%https://tex.stackexchange.com/a/101136/26707
\usepackage{scalerel,stackengine}
\stackMath
\newcommand\reallywidehat[1]{%
\savestack{\tmpbox}{\stretchto{%
  \scaleto{%
    \scalerel*[\widthof{\ensuremath{#1}}]{\kern-.6pt\bigwedge\kern-.6pt}%
    {\rule[-\textheight/2]{1ex}{\textheight}}%WIDTH-LIMITED BIG WEDGE
  }{\textheight}% 
}{0.5ex}}%
\stackon[1pt]{#1}{\tmpbox}%
}

%\usepackage{braket}
\usepackage{thmtools}%restate theorem
\usepackage{hyperref}

% https://en.wikibooks.org/wiki/LaTeX/Hyperlinks
\hypersetup{
    %bookmarks=true,
    unicode=true,
    pdftitle={\ntitle},
    pdfauthor={\nauthor},
    pdfsubject={Mathematics},
    pdfcreator={\nauthor},
    pdfproducer={\nauthor},
    pdfkeywords={math maths \ntitle},
    colorlinks=true,
    linkcolor={red!50!black},
    citecolor={blue!50!black},
    urlcolor={blue!80!black}
}

\usepackage{cleveref}



% TODO: mdframed often gives bad breaks that cause empty lines. Would like to switch to tcolorbox.
% The current workaround is to set innerbottommargin=0pt.

%\usepackage[theorems]{tcolorbox}





\usepackage[framemethod=tikz]{mdframed}
\mdfdefinestyle{leftbar}{
  %nobreak=true, %dirty hack
  linewidth=1.5pt,
  linecolor=gray,
  hidealllines=true,
  leftline=true,
  leftmargin=0pt,
  innerleftmargin=5pt,
  innerrightmargin=10pt,
  innertopmargin=-5pt,
  % innerbottommargin=5pt, % original
  innerbottommargin=0pt, % temporary hack 
}
%\newmdtheoremenv[style=leftbar]{theorem}{Theorem}[section]
%\newmdtheoremenv[style=leftbar]{proposition}[theorem]{proposition}
%\newmdtheoremenv[style=leftbar]{lemma}[theorem]{Lemma}
%\newmdtheoremenv[style=leftbar]{corollary}[theorem]{corollary}

\newtheorem{theorem}{Theorem}[section]
\newtheorem{proposition}[theorem]{Proposition}
\newtheorem{lemma}[theorem]{Lemma}
\newtheorem{corollary}[theorem]{Corollary}
\newtheorem{axiom}[theorem]{Axiom}
\newtheorem*{axiom*}{Axiom}

\surroundwithmdframed[style=leftbar]{theorem}
\surroundwithmdframed[style=leftbar]{proposition}
\surroundwithmdframed[style=leftbar]{lemma}
\surroundwithmdframed[style=leftbar]{corollary}
\surroundwithmdframed[style=leftbar]{axiom}
\surroundwithmdframed[style=leftbar]{axiom*}

\theoremstyle{definition}

\newtheorem*{definition}{Definition}
\surroundwithmdframed[style=leftbar]{definition}

\newtheorem*{slogan}{Slogan}
\newtheorem*{eg}{Example}
\newtheorem*{ex}{Exercise}
\newtheorem*{remark}{Remark}
\newtheorem*{notation}{Notation}
\newtheorem*{convention}{Convention}
\newtheorem*{assumption}{Assumption}
\newtheorem*{question}{Question}
\newtheorem*{answer}{Answer}
\newtheorem*{note}{Note}
\newtheorem*{application}{Application}

%operator macros

%basic
\DeclareMathOperator{\lcm}{lcm}

%matrix
\DeclareMathOperator{\tr}{tr}
\DeclareMathOperator{\Tr}{Tr}
\DeclareMathOperator{\adj}{adj}

%algebra
\DeclareMathOperator{\Hom}{Hom}
\DeclareMathOperator{\End}{End}
\DeclareMathOperator{\id}{id}
\DeclareMathOperator{\im}{im}
\DeclareMathOperator{\coker}{coker}
\DeclarePairedDelimiter{\generation}{\langle}{\rangle}

%groups
\DeclareMathOperator{\sym}{Sym}
\DeclareMathOperator{\sgn}{sgn}
\DeclareMathOperator{\inn}{Inn}
\DeclareMathOperator{\aut}{Aut}
\DeclareMathOperator{\GL}{GL}
\DeclareMathOperator{\SL}{SL}
\DeclareMathOperator{\PGL}{PGL}
\DeclareMathOperator{\PSL}{PSL}
\DeclareMathOperator{\SU}{SU}
\DeclareMathOperator{\UU}{U}
\DeclareMathOperator{\SO}{SO}
\DeclareMathOperator{\OO}{O}
\DeclareMathOperator{\PSU}{PSU}
\DeclareMathOperator{\Sp}{Sp}


%hyperbolic
\DeclareMathOperator{\sech}{sech}

%field, galois heory
\DeclareMathOperator{\ch}{ch}
\DeclareMathOperator{\gal}{Gal}
\DeclareMathOperator{\emb}{Emb}



%ceiling and floor
%https://tex.stackexchange.com/a/118217/26707
\DeclarePairedDelimiter\ceil{\lceil}{\rceil}
\DeclarePairedDelimiter\floor{\lfloor}{\rfloor}


\DeclarePairedDelimiter{\innerproduct}{\langle}{\rangle}

%\DeclarePairedDelimiterX{\norm}[1]{\lVert}{\rVert}{#1}
\DeclarePairedDelimiter{\norm}{\lVert}{\rVert}



%Dirac notation
%TODO: rewrite for variable number of arguments
\DeclarePairedDelimiterX{\braket}[2]{\langle}{\rangle}{#1 \delimsize\vert #2}
\DeclarePairedDelimiterX{\braketthree}[3]{\langle}{\rangle}{#1 \delimsize\vert #2 \delimsize\vert #3}

\DeclarePairedDelimiter{\bra}{\langle}{\rvert}
\DeclarePairedDelimiter{\ket}{\lvert}{\rangle}




%macros

%general

%divide, not divide
\newcommand*{\divides}{\mid}
\newcommand*{\ndivides}{\nmid}
%vector, i.e. mathbf
%https://tex.stackexchange.com/a/45746/26707
\newcommand*{\V}[1]{{\ensuremath{\symbf{#1}}}}
%closure
\newcommand*{\cl}[1]{\overline{#1}}
%conjugate
\newcommand*{\conj}[1]{\overline{#1}}
%set complement
\newcommand*{\stcomp}[1]{\overline{#1}}
\newcommand*{\compose}{\circ}
\newcommand*{\nto}{\nrightarrow}
\newcommand*{\p}{\partial}
%embed
\newcommand*{\embed}{\hookrightarrow}
%surjection
\newcommand*{\surj}{\twoheadrightarrow}
%power set
\newcommand*{\powerset}{\mathcal{P}}

%matrix
\newcommand*{\matrixring}{\mathcal{M}}

%groups
\newcommand*{\normal}{\trianglelefteq}
%rings
\newcommand*{\ideal}{\trianglelefteq}

%fields
\renewcommand*{\C}{{\mathbb{C}}}
\newcommand*{\R}{{\mathbb{R}}}
\newcommand*{\Q}{{\mathbb{Q}}}
\newcommand*{\Z}{{\mathbb{Z}}}
\newcommand*{\N}{{\mathbb{N}}}
\newcommand*{\F}{{\mathbb{F}}}
%not really but I think this belongs here
\newcommand*{\A}{{\mathbb{A}}}

%asymptotic
\newcommand*{\bigO}{O}
\newcommand*{\smallo}{o}

%probability
\newcommand*{\prob}{\mathbb{P}}
\newcommand*{\E}{\mathbb{E}}

%vector calculus
\newcommand*{\gradient}{\V \nabla}
\newcommand*{\divergence}{\gradient \cdot}
\newcommand*{\curl}{\gradient \cdot}

%logic
\newcommand*{\yields}{\vdash}
\newcommand*{\nyields}{\nvdash}

%differential geometry
\renewcommand*{\H}{\mathbb{H}}
\newcommand*{\transversal}{\pitchfork}
\renewcommand{\d}{\mathrm{d}} % exterior derivative

%number theory
\newcommand*{\legendre}[2]{\genfrac{(}{)}{}{}{#1}{#2}}%Legendre symbol

%algebraic geometry
\DeclareMathOperator{\Spec}{Spec}
\DeclareMathOperator{\Proj}{Proj}

\DeclareMathOperator{\Mat}{Mat}
\newcommand*{\Lie}[1]{\mathfrak{#1}} % Lie groups
\renewcommand*{\P}{\mathbb{P}}
\DeclareMathOperator{\ad}{ad} % adjoint
\let\ch\relax
\DeclareMathOperator{\ch}{ch} % character
\DeclareMathOperator{\cha}{ch} % characteristic

\begin{document}

\begin{titlepage}
  \begin{center}
    \includegraphics[width=0.6\textwidth]{logo.jpg}\par
    \vspace{1cm}
    {\scshape\huge Mathamatics Tripos \par}
    \vspace{2cm}
    {\huge Part \npart \par}
    \vspace{0.6cm}
    {\Huge \bfseries \ntitle \par}
    \vspace{1.2cm}
    {\Large\nterm, \nyear \par}
    \vspace{2cm}
    
    {\large \emph{Lectures by } \par}
    \vspace{0.2cm}
    {\Large \scshape \nlecturer}
    
    \vspace{0.5cm}
    {\large \emph{Notes by }\par}
    \vspace{0.2cm}
    {\Large \scshape \href{mailto:\nauthoremail}{\nauthor}}
 \end{center}
\end{titlepage}

\tableofcontents

\section{Introduction \& Motivation}

The objects of interest in this course are
\begin{align*}
  \SL_n &= \{A \in \Mat_n: \det A = 1\} \\
  \SO_n &= \{A \in \SL_n: AA^T = I\} \\
  \Sp_{2n} &= \cdots
\end{align*}
and five more examples. First of all they are algebraic groups.

We have \(\SU_2 \subseteq \SL_2\). Note that \(\SU_2\) is homeomorphic to \(S^3\) and so is compact. In fact it is maximal compact and every maximal compact subgroup of \(\SL_2\) is conjugate to \(\SU_2\).

We will look at the tangent space of the group at the identity, which is just a finite-dimensional vector space.

\begin{definition}
  A \emph{linear algebraic group} is a subgroup of \(\Mat_n\) which is defined by polynomial equations in the matrix coefficients.
\end{definition}

For example \(\SL_n\) and \(\SO_n\) are linear algebraic groups. \(\GL_n\) is also an example as we have embedding
\begin{align*}
  \GL_n &\to \Mat_{n + 1} \\
  A &\to
      \begin{pmatrix}
        A & \\
        & \lambda
      \end{pmatrix}
\end{align*}
where the image is given by \(\det A \cdot \lambda = 1\).

\begin{eg}
  Let \(G = \SL_2\) and let
  \[
    g =
    \begin{pmatrix}
      1 & \\
      & 1
    \end{pmatrix}
    + \varepsilon
    \begin{pmatrix}
      a & b \\
      c & d
    \end{pmatrix}
    + \cdots
  \]
  so
  \[
    \det g = 1 + \varepsilon(a + d) + \text{higher terms}
  \]
  so \(\det g = 1\) if and only if \(a + d = 0\) if we pretend to be physicists for a second. Now introduce the dual numbers
  \[
    E = \C[\varepsilon]/(\varepsilon^2) = \{a + b \varepsilon: a, b \in \C\}.
  \]
  If \(G\) is an algebraic group then we define
  \[
    G(E) = \{A \in \Mat_n(E): A \text{ satisfies the defining equations of } G\}.
  \]
  Then
  \[
    \SL_2(E) = \{
    \begin{pmatrix}
      \alpha & \beta \\
      \gamma & \delta
    \end{pmatrix}
    : \alpha, \beta, \gamma, \delta \in E, \alpha \delta - \beta \gamma = 1\}
  \]
  Now the map \(E \to \C, \varepsilon \mapsto 0\) defines a map \(\pi: G(E) \to G\). We define the \emph{Lie algebra}\index{Lie algebra} of \(G\) to be
  \[
    \Lie g \cong \pi^{-1}(I) \cong \{X \in \Mat_n(\C): I + \varepsilon X \in G(E)\}.
  \]
  In particular,
  \[
    \SL_2 = \{
    \begin{pmatrix}
      a & b \\
      c & d
    \end{pmatrix}
    \in \Mat_2(\C): a + d = 0\}.
  \]
\end{eg}

\begin{ex}
  Show that \(G(E) = TG\) is the tangent bundle of \(G\) and \(\Lie g\) is the tangent space at \(1\), \(I + X \varepsilon\) is the germ of a curve through \(1 \in G\).
\end{ex}

\begin{eg}
  Let \(G = \GL_n\). Then
  \begin{align*}
    G(E) &= \{\tilde A \in \Mat_n(E): \tilde A^{-1} \text{ exists}\} \\
         &= \{A + B \varepsilon: A, B \in \Mat_n(\C), A^{-1} \text{ exists}\}
  \end{align*}
  where the second equality is because
  \[
    (A + B \varepsilon) (A^{-1} - A^{-1}B A^{-1} \varepsilon) = I.
  \]
  So there is no condition on \(B\) so \(\Lie{gl}_n = \Mat_n(\C)\). Another explantion for this result is that \(\det\) does not vanish in a neighbourhood of the identity matrix so we get all matrices in the Lie algebra.
\end{eg}

\begin{ex}
  Let \(G = \SL_n\). Show that
  \[
    \det (I + \varepsilon X) = 1 + \varepsilon \tr X
  \]
  and hence
  \[
    \Lie{sl}_n = \{X \in \Mat_n(\C): \tr X = 0\}.
  \]
\end{ex}

\begin{eg}
  Let
  \[
    G = \OO_n = \{A \in \Mat_n: AA^T = I\}.
  \]
  Then
  \begin{align*}
    \Lie g &= \{X \in \Mat_n(\C): (I + \varepsilon X)(I + \varepsilon X)^T = I\} \\
                &= \{X \in \Mat_n(\C): X + X^T = 0\}
  \end{align*}
  Note \(\tr X^T = \tr X\) so \(\tr X = \tr X^T = 0\). Thus \(\SO_n\) has the same Lie algebra. In other words, by just looking into the Lie algebras we cannot distinguish the groups \(\OO_n\) and \(\SO_n\). This is because \(\OO_n\) has two connected component, and the component of the identity is \(\SO_n\). Of course the tangent space at the identity doesn't tell us anything in the other component. Thus this undesirable situation can be remedied by restricting to connected Lie groups.
\end{eg}

What structure does \(\Lie g\) have that it inherits from \(G\)? It is not a (multiplicative) group as
\[
  (I + A \varepsilon) (I + B \varepsilon) = I + \varepsilon (A + B)
\]
has nothing to do with multiplication. Instead, we can consider the commutator
\begin{align*}
  G \times G &\to G \\
  (P, Q) &\mapsto PQP^{-1}Q^{-1}
\end{align*}
This sends \((I, I) \mapsto I\) so by differentiating at the origin we get a map \(\Lie g \times \Lie g \to \Lie g\). Actually, we want a bilinear map \(\Lie g \times \Lie g \to \Lie g\), so differentiate in each variable separately: fix \(P\) and differentiate \(f_P: Q \mapsto PQP^{-1}Q^{-1}\) to get \(df_P: \Lie g \to \Lie g\). Then we differentiate it as a function of \(P\).

Explicitly, write
\begin{align*}
  P &= I + \varepsilon A \\
  Q &= I + \delta B
\end{align*}
where \(\varepsilon^2 = \delta^2 = 0, \varepsilon \delta = \delta \varepsilon \neq 0\). Then
\[
  PQP^{-1}Q^{-1} = I + (AB - BA) \varepsilon\delta
\]
so the map constructed out of the commutators is
\begin{align*}
  \Lie g \times \Lie g &\to \Lie g \\
  (A, B) &\mapsto AB - BA
\end{align*}
This is called the \emph{Lie bracket} of \(A\) and \(B\).

\begin{ex}\leavevmode
  \begin{enumerate}
  \item Show by differentiation that
    \[
      (PQP^{-1}Q^{-1})^{-1} = QPQ^{-1}P^{-1}
    \]
    implies that
    \[
      [B, A] = -[A, B]
    \]
    so the Lie bracket is anti-symmetric.
  \item Show associativity of multiplication implies that
    \[
      [[X, Y], Z] + [[Y, Z], X] + [[Z, X], Y] = 0.
    \]
    This is the \emph{Jacobi identity}.

    Also show this is true from the definition \([A, B] = AB - BA \in \Mat_n\).
  \end{enumerate}
\end{ex}

\begin{definition}[Lie algebra]\index{Lie algebra}
  Let \(k\) be a field, \(\cha k \neq 2\). A \emph{Lie algebra} \(\Lie g\) is a \(k\)-vector space equipped with a bilinear map \([\cdot, \cdot]: \Lie g \times \Lie g \to \Lie g\) that
  \begin{enumerate}
  \item is anti-symmetric: \([X, Y] = - [Y, X]\),
  \item satisfies the Jacobi identity
    \[
      [[X, Y], Z] + [[Y, Z], X] + [[Z, X], Y] = 0.
    \]
  \end{enumerate}
\end{definition}

\begin{eg}\leavevmode
  \begin{enumerate}
  \item \(\Lie{gl}_n = \Mat_n\) with \([A, B] = AB - BA\). More generally, if \(V\) is a vector space, write \(\Lie{gl}(V) = \End(V)\).
  \item \(\Lie{so}_n = \{A \in \Lie{gl}_n: A + A^T = 0\}\).
  \item \(\Lie{sl}_n = \{A \in \Lie{gl}_n: \tr A = 0\}\).
  \item \(\Lie{sp}_{2n} = \{A \in \Lie{gl}_{2n}: JA^TJ^{-1} + A = 0\}\) where
    \[
      J =
      \begin{psmallmatrix}
        & & & & & 1 \\
        & & & & 1 \\
        & & & \cdots \\
        & -1 \\
        -1
      \end{psmallmatrix}
    \]
  \item \(\Lie{b}_n = \{
    \begin{psmallmatrix}
      * & \cdots & * \\
      & \ddots & * \\
      0 & & *
    \end{psmallmatrix}
    \}
    \) of upper triangular matrices.
  \item \(\Lie u_n\) of strictly upper triangular matrices.
  \item If \(V\) is any vector space, let \([\cdot, \cdot]: V \times V \to V\) be the zero map. This is a Lie algebra, called \emph{abelian Lie algebra}.
  \end{enumerate}
\end{eg}

\begin{ex}\leavevmode
  \begin{enumerate}
  \item Show \(\Lie{gl}_n\) is a Lie algebra.
  \item Show examples 2 - 7 are sub-Lie algebras of \(\Lie{gl}_n\).
  \item Find algebraic groups whose Lie algebras are the examples above.
  \item Show \(\{
    \begin{psmallmatrix}
      * & * \\
      * & 0
    \end{psmallmatrix}
    \} \subseteq \Lie{gl}_2\) is not a Lie algebra.
  \end{enumerate}
\end{ex}

\begin{eg}
  Any \(1\)-dim Lie algebra is abelian by anti-symmetry.
\end{eg}

\begin{ex}
  Classify all Lie algebras of dimension \(3\).
\end{ex}

\begin{definition}[representation]\index{representation}
  A \emph{representation} of a Lie algebra \(\Lie g\) on a vector space \(V\) is a Lie algebra homomorphism \(\Lie g \to \Lie{gl}(V)\). We say \(\Lie g\) acts on \(V\).
\end{definition}

We have the silly example of trivial representation: \(\Lie g\) acts on \(V = k\) by \(x \mapsto 0\).

Less trivially, for any \(x \in \Lie g\), define
\begin{align*}
  \ad x: \Lie g &\to \Lie g \\
  y &\mapsto [x, y]
\end{align*}

\begin{lemma}
  \(\ad: \Lie g \to \End(\Lie g)\) is a representation of \(\Lie g\), i.e.\ \(\Lie g\) acts on it self. This is called the \emph{adjoint representation}\index{adjoint representation}.
\end{lemma}

\begin{proof}
  Must show
  \[
    \ad [x, y] = \ad x \ad y - \ad y \ad x.
  \]
  If \(z \in \Lie g\) then
  \begin{align*}
    (\ad [x, y])(z) &= [[x, y], z] \\
    \text{RHS}(z) &= [x, [y, z]] - [y, [x, z]] = -[[y, z], x] - [[z, x], y]
  \end{align*}
  and they are equal by Jacobi.
\end{proof}

\begin{definition}[center]\index{center}
  The \emph{center} of \(\Lie g\) is
  \[
    \{x \in \Lie g: [x, y] = 0 \text{ for all } y \in \Lie g\} = \ker (\ad: \Lie g \to \Lie{gl}(\Lie g)),
  \]
  which is an abelian Lie algebra.
\end{definition}

In particular, the center of \(\Lie g\) is \(0\) if and only if \(\ad\) is an embedding. Question: does every finite-dimensional Lie algebra \(\Lie g\) have a faithful finite-dimensional representation? In other words, does \(\Lie g \embed \Lie{gl}(V)\) for some \(V\)?

Note: every affine algebraic group has a faithful representation.

\begin{theorem}[Ado]
  Any finite-dimensional Lie algebra \(\Lie g\) over \(k\) has a faithful finite-dimensional rep, i.e.\ \(\Lie g \embed \Lie{gl}_n\) for some \(.\)
\end{theorem}

\begin{eg}
  Let \(\Lie g = \Lie{sl}_2\) with basis
  \[
    e =
    \begin{pmatrix}
      0 & 1 \\
      0 & 0
    \end{pmatrix},
    f =
    \begin{pmatrix}
      0 & 0 \\
      1 & 0
    \end{pmatrix},
    h =
    \begin{pmatrix}
      1 & 0 \\
      0 & -1
    \end{pmatrix}
  \]
  so we have
  \[
    [e, f] = h,
    [h, e] = 2e,
    [h, f] = -2f
  \]
  so a representation of \(\Lie{sl}_2\) is a triple of matrices \(E, F, H \in \Mat_n\) with these relations. How can we find such? The answer, at this moment, is to find reps of the algebraic group \(\SL_2\) and differentiating. Later we will find them just by using linear algebra.
\end{eg}

\begin{definition}[algebraic representation]\index{algebraic representation}
  If \(G\) is an algebraic group. An \emph{algebraic representation} of \(G\) on a vector space \(V\) is a homomorphism \(G \to \GL(V)\) defined by polynomial equations in the matrix coefficients.
\end{definition}

Let \(\rho: G \to \GL(V)\) be an algebraic rep. We have \(\rho(I) = I\). Consider the map \(G(E) \to \GL(V)(E)\). We get
\[
  \rho(I + A\varepsilon) = I + \varepsilon d\rho(A)
\]
for some function \(d\rho(A)\) of \(A\).

\begin{ex}
  \(d \rho\) is the derivative of \(\rho\) at identity.
\end{ex}

\begin{ex}
  \(\rho: G \to \GL(V)\) implies that \(d\rho: \Lie g \to \Lie{gl}(V)\) is a Lie algebra homomorphism, so \(V\) is a representation of \(\Lie g\).
\end{ex}

Let \(G = \SL_2\) and let \(L(n)\) be homogeneous polynomials in \(x, y\) of degree \(n\), with basis \(x^n, x^{n - 1}y, \cdots, y^n\), so has dimension \(n + 1\). \(\GL_2\) acts on \(L(n)\) by change of coordinates: if \(g =
\begin{pmatrix}
  a & b \\
  c & d
\end{pmatrix}
\), \(f \in L(n)\) then
\[
  (\rho_n(g)f)(x, y) = f(ax + cy, bx + dy).
\]
Check that
\begin{enumerate}
\item \(\rho_0\) is the trivial rep.
\item \(\rho_1\) is the usual \(2\)-dim rep.
\item
  \[
    \rho_2
    \begin{pmatrix}
      a & b \\
      c & d
    \end{pmatrix}
    =
    \begin{pmatrix}
      a^2 & ab & b^2 \\
      2ac & ad + bc & 2bd \\
      c^2 & cd & d^2
    \end{pmatrix}
  \]
\end{enumerate}
Differentiate and we get an action of \(\Lie{sl}_2\) on \(L(n)\). Explicitly,
\[
  \rho(I + \varepsilon e) x^iy^j
  = x^i (y + \varepsilon x)^j
  = x^iy^j + \varepsilon jx^{i + 1} y^{j - 1}
\]
and hence
\[
  d\rho(e) x^iy^j = jx^{i + 1} y^{j - 1}.
\]

\begin{ex}\leavevmode
  \begin{enumerate}
  \item The Lie algebra acts by
    \begin{align*}
      e \cdot (x^iy^j) &= jx^{i + 1} y^{j - 1} \\
      f \cdot (x^iy^j) &= ix^{i - 1} y^{j + 1} \\
      h \cdot (x^iy^j) &= (i - j) x^iy^j
    \end{align*}
  \item Check directly this gives a rep of \(\Lie{sl}_2\).
  \item Show \(L(2)\) is isomorphic to the adjoint rep.
  \item Show that
    \[
      e = x \frac{\partial  }{\partial y}, f = y \frac{\partial  }{\partial x}, h = x \frac{\partial  }{\partial x} - y \frac{\partial  }{\partial y}
    \]
    defines an (infinite-dimensional) rep of \(\Lie{sl}_2\) on \(k[x, y]\). Some implication: this can be defined for all characteristics, and the differential operator is suggesting that reps of Lie groups might have something to do with calculus.
  \item Show if \(\cha k = 0\) then \(L(n)\) is irreducible as an \(\Lie{sl}_2\), hence \(\SL_2\)-module.
  \end{enumerate}
\end{ex}

The map \(\rho \mapsto d \rho\) defines a functor from the category of a linear algebraic group \(G\) to the category of Lie algebra reps of \(\Lie g\). However, this is not as nice a map as you might hope.

\begin{eg}
  Let \(G = \C^\times\) so \(\Lie g = \C\) is the abelian Lie algebra. A rep of \(\Lie g\) on a vector space \(V\) is the same as an element \(A \in \End(V)\). A submodule \(W \subseteq V\) is a subspace \(W\) such that \(gW \subseteq W\), i.e.\ \(A \cdot W \subseteq W\), so the same as an \(A\)-subspace of \(V\). Check that \(A\) and \(A'\) in \(\End(V)\) determine isomorphic reps of \(\Lie g\) if and only if \(A, A'\) are conjugate. Hence isomorphism classes of reps of \(\Lie g = \C\) is in bijection with conjugacy classes of matrices, and hence is determined by its Jordan normal form.

  In addition, any \(A \in \End(V)\) has an eigenvector as \(V\) is a vector space over \(\C\). Thus the only irreducible rep of \(\Lie g\) are the 1-dim ones.

  A rep is isomorphic to a direct sum of irred reps if and only if \(A\) is diagonalisable. For example if \(A =
  \begin{psmallmatrix}
    0 & 1 \\
    & 0 & 1 \\
    & & & \ddots \\
    & & & & 1 \\
    & & & & 0
  \end{psmallmatrix}
  \) then the associated rep is \emph{indecomposable}, i.e.\ it does not split into a direct sum, as the only \(A\)-subspaces are \(\langle e_1, \rangle, \langle e_1, e_2 \rangle, \cdots, \langle e_1, \dots, e_n \rangle\).

  Now in constrast consider reps of \(G = \C^\times\). It is a theorem that the irred algebraic reps of \(\C^\times\) are the 1-dim reps where \(z \in \C^\times\) acts on \(\C\) by \(z \cdot v = z^n v\) for \(n \in \Z\). In other words they are given by \(G \to \GL_1, z \mapsto z^n\). Moreover, any finite-dimensional rep of \(G\) is a direct sum of irreducible (this is similar to the proof that the only irred reps of the compact group \(S^1\) are given by \(z \mapsto z^n\), once we set up the theory of algebraic groups).
\end{eg}

\begin{ex}
  Show \(\rho \mapsto d\rho\) sends \(z \mapsto z^n\) to the algebraic rep \(n \in \C\).
\end{ex}

The rep of Lie algebra \(\C\) is continuous while that of the algebraic group \(\C^\times\) is discrete. This has something to do with \(S^1\) and its topology. Later we'll see that the functor \(d\) gives an equivalence of category when restricted to simply connected Lie groups.

\begin{note}
  Notice \(\Lie g\) is also the Lie algebra of the additive group \((\C, +)\), whose algebraic reps resemble the reps of \(\Lie g\).
\end{note}

Less distressingly, if \(Z \subseteq G\) is a finite central subgroup then \(T_1(G/Z) = T_1G\) so the Lie algebras of \(G\) and \(G/Z\) agree.

\begin{ex}
  Let \(G_n = \C^* \ltimes \C\) where \(\C^*\) acts on \(\C\) by \(t \cdot \lambda = t^n \lambda\) so
  \[
    (t, \lambda) (t', \lambda') = (tt', t'^n \lambda + \lambda').
  \]
  Show that \(G_n \cong G_m\) if and only if \(n = \pm m\), but
  \[
    \operatorname{Lie} G_n = \operatorname{Lie} G_m = \C x + \C y
  \]
  where \([x, y] = y\), so the functor is not faithful.
\end{ex}
As a side note, the functor is not surjective either.

\section{Representations of \(\Lie{sl}_2\)}

Recall that \(\Lie{sl}_2\) has basis
\[
  e =
  \begin{pmatrix}
    0 & 1 \\
    0 & 0
  \end{pmatrix}
  , f =
  \begin{pmatrix}
    0 & 0 \\
    1 & 0
  \end{pmatrix}
  , h =
  \begin{pmatrix}
    1 & 0 \\
    0 & -1
  \end{pmatrix}
\]
so we have
\[
  [e, f] = h, [h, e] = 2e, [h, f] = -2f
\]
 
We would like to prove
\begin{theorem}\leavevmode
  \begin{enumerate}
  \item For each \(n \geq 0\) there is a unique irreducible rep of \(\Lie{sl}_2\) of dimension \(n + 1\).
  \item Every finite-dimensional rep of \(\Lie{sl}_2\) is a direct sum of irred reps.
  \end{enumerate}
\end{theorem}

\begin{definition}[weight space]\index{weight space}
  Let \(V\) be a rep of \(\Lie{sl}_2\). If \(\lambda \in \C\), the \emph{\(\lambda\)-weight space} of \(V\) is
  \[
    V_\lambda = \{v \in V: h v = \lambda v\},
  \]
  the eigenspace of \(h\).
\end{definition}

\begin{eg}
  \(L(n)_\lambda = \C x^i y^j\) if \(i - j = \lambda\).
\end{eg}

Let \(v \in V_\lambda\) and we have
\[
  h \cdot ev = (he - eh + eh) v = ([h, e] + eh) v = 2ev + e \lambda v = (\lambda + 2) ev
\]
so if \(v \in V_\lambda\) then \(ev \in V_{\lambda + 2}\), if and only if \(ev \neq 0\). Similarly \(fv \in V_{\lambda - 2}\). Thus \(f\) and \(e\) shifts between a string of spaces \(V_{\lambda + 2}, V_\lambda, V_{\lambda - 2}, \dots\)
\[
  \begin{tikzcd}
    \cdots \ar[r,  shift left] & V_{\lambda - 2} \ar[l, shift left] \ar[r, "e", shift left] & V_\lambda \ar[l, "f", shift left] \ar[r, "e", shift left] & V_{\lambda + 2} \ar[l, "f", shift left] \ar[r, shift left] & \cdots \ar[l, shift left]
  \end{tikzcd}
\]

If \(v \in V_\lambda \cap \ker e\), that is \(ev = 0, hv = \lambda v\) we say \(v\) is a \emph{highest weight vector with highest weight \(\lambda\)}.

\begin{lemma}
  Let \(V\) be a rep of \(\Lie{sl}_2\), \(v \in V_\lambda\) a highest weight vector of weight \(\lambda\) then \(W = \langle v, fv, f^2v, \cdots \rangle\) is an \(\Lie{sl}_2\)-invariant subspace, that is a subrep of \(V.\)
\end{lemma}

\begin{proof}
  We must show the image of \(W\) under \(f, h, e\) are contained in \(W\). \(fW \subseteq W\) by construction. As \(v \in V_\lambda\), we see that \(f^kv \in V_{\lambda - 2k}\) and so \(hW \subseteq W\). Finally \(ev = 0 \in W\) and
  \begin{align*}
    e \cdot fv &= (ef - fe + fe) v = hv = \lambda v \in W \\
    e \cdot f^2 v &= ([e, f] + fe) fv = (\lambda - 2) fv + f\cdot \lambda v = (2\lambda - 2) fv \in W \\
    e \cdot f^3 v &= ([e, f] + fe) f^2 = (\lambda - 4) f^2v + f(2\lambda - 2) fv = (3\lambda - 6)f^2 v \in W
  \end{align*}
  and so on. It is an exercise to show by induction
  \[
    e\cdot f^n v = n (\lambda - n + 1) f^{n - 1} v.
  \]
\end{proof}

We have a surprising result:
\begin{lemma}
  Let \(V\) be a finite-dimensional \(\C\)-space and a rep of \(\Lie{sl}_2\) and \(v \in V\) a highest weight vector with highest weight \(\lambda\) then \(\lambda \in \{0, 1, \dots \} = \Z_{\geq 0}\).
\end{lemma}

\begin{proof}
  Note that all \(f^k v\) lie in different eigenspaces for \(h\) so if non-zero they are linearly independent. But \(V\) is finite dimensional so exists \(k\) such that \(f^k v \neq 0, f^{k + 1} v = 0\). The exercise shows
  \[
    0 = ef^{k + 1}v = (k + 1)( \lambda - k)f^k v
  \]
  so \(k + 1 \neq 0\) so \(\lambda = k\).
\end{proof}

\begin{lemma}
  If \(V\) is a finite-dimensional rep of \(\Lie{sl}_2\) then it has a highest weight vector.
\end{lemma}

\begin{proof}
  As \(V\) is a \(\C\)-space \(h\) has an eigenvector. Apply \(e\) to it get \(v, ev, e^2v, \dots\) which are eigenvectors with different eigenvectors so if nonzero are linearly independent so exists \(k\) such that \(e^kv = 0\), so \(e^kv\) is a highest weight eigenvector.
\end{proof}

\begin{corollary}
  Let \(k = \C\). If \(V\) is an irreducible finite dimensional representation of \(\Lie{sl}_2\) then \(\dim V = n + 1\) and \(V\) has basis \(v_0, v_1, \dots, v_n\) with
  \begin{align*}
    hv_i &= (n - 2i) v_i \\
    fv_i &= v_{i + 1} \\
    ev_i &= i (n - i + 1) v_{i - 1}
  \end{align*}
  In particular there is a unique irreducible representation of dimension \(n + 1\), which is isomorphic to \(L(n)\).
\end{corollary}

(Picture of string)

\begin{ex}\leavevmode
  \begin{enumerate}
  \item Find the explicit relation between this basis and the \(x^ay^b\) basis earlier, where \(a + b = n\).
  \item Recall \(\C[x, y] = \bigoplus_{n \geq 0} L(n)\) as a representation of \(\Lie{sl}_2\) where \(e, h, f\) acts as differential operators. Show that the same operators give a rep of \(\Lie{sl}_2\) on \(x^\lambda y^\mu \C[x/y, y/x]\) for all \(\lambda, \mu \in \C\). Determine the submodules of this rep.
  \end{enumerate}
\end{ex}

Now we show that all reps can be written as direct sum of the irreducible ones. This is one of the more difficult theorem but will lead us towards the general result later. We will show strings of different lengths don't interact, then strings of the same lengths do not interact.

\begin{definition}
  Let \(V\) be a rep of \(\Lie{sl}_2\). Define \(\Omega \in \End(V)\) by
  \[
    \Omega = ef + fe + \frac{1}{2}h^2,
  \]
  the \emph{Casimir} of \(\Lie{sl}_2\).
\end{definition}

\begin{lemma}
  \(\Omega\) is central, that is \(e\Omega = \Omega e, f \Omega = \Omega f, h \Omega = \Omega h\).
\end{lemma}

\begin{proof}
  We will later show a slick proof. For now this is left as an exercise. For example
  \begin{align*}
    e \Omega
    &= e (ef +fe + \frac{1}{2} h^2) \\
    &= e(ef - fe) + 2efe \\
    &+ \frac{1}{2} (eh - he) h + \frac{1}{2} heh \\
    &= 2efe + \frac{1}{2} heh \\
    &= \cdots \\
    &= \Omega e
  \end{align*}
\end{proof}

\begin{corollary}
  If \(V\) is an irreducible rep of \(\Lie{sl}_2\), then \(\Omega\) acts on \(V\) by a scalar.
\end{corollary}

\begin{proof}
  Similar to Schur's lemma.
\end{proof}

\begin{lemma}
  \(\Omega\) acts on \(L(n)\) as multiplication by \(\frac{1}{2} n^2 + n\).
\end{lemma}

\begin{proof}
  We can choose any nonzero element and use the above corollary. Alternatively we can do it by hand. Let \(v\) be the highest weight vector of \(L(n)\) so \(ev = 0, hv = nv\). Then
  \[
    \Omega = (ef - fe) + 2fe + \frac{1}{2} h^2 = (\frac{1}{2} h^2 + h) + 2fe
  \]
  so
  \begin{align*}
    \Omega v &= (\frac{1}{2} n^2 + n) v \\
    \Omega (f^k v) &= f^k \Omega v = (\frac{1}{2} n^2 + n) f^k v
  \end{align*}
\end{proof}

This immediately implies ``strings of different lengths don't interact'', which we shall make sense of now.

Let \(V\) be a finite dimensional rep of \(\Lie{sl}_2\). Let
\[
  V^\lambda = \{v \in V: (\Omega - \lambda)^{\dim V} v = 0\}
\]
be the generalised eigenspace for \(\Omega\) with eigenvalue \(\lambda\). By linear algebra, \(V = \bigoplus_\lambda V^\lambda\). Claim that each \(V^\lambda\) is a subrep, i.e.\ preserved by \(\Lie{sl}_2\), so this is a direct sum decomposition of \(V\) as reps of \(\Lie{sl}_2\).

\begin{proof}
  Let \(x \in \Lie{sl}_2, v \in V^\lambda\). Then
  \[
    (\Omega - \lambda)^{\dim V} xv = x(\Omega - \lambda)^{\dim V} v = 0
  \]
  as \(\Omega\) is central so \(xv \in V^\lambda\).
\end{proof}

Claim that if \(V^\lambda \neq 0\) then \(\lambda = \frac{1}{2} n^2 + n\) for a unique \(n \in \Z_{\geq 0}\), and ``\(V^\lambda\) is glued together from copies of \(L(n)\)''. Formally, ``gluing'' refers to the following:

\begin{definition}[composition series]\index{composition series}
  Let \(W\) be a finite dimensional representation of \(\Lie g\). A \emph{composition series} for \(W\) is a sequence of submodules
  \[
    0 = W_0 \subseteq W_1 \subseteq W_2 \subseteq \cdots \subseteq W_r = W
  \]
  such that each \(W_i/W_{i - 1}\) is a non-zero irreducible module.
\end{definition}

\begin{eg}\leavevmode
  \begin{enumerate}
  \item Let \(\Lie g = \C, W = \C^r\) where \(1 \in \Lie g\) acts as \(
    \begin{psmallmatrix}
    0 & 1 \\
    & 0 & 1 \\
    & & & \ddots \\
    & & & & 1 \\
    & & & & 0
  \end{psmallmatrix}
  \). Then there is a unique composition series for \(W\), namely
  \[
    0 \subseteq \langle e_1 \rangle \subseteq \langle e_1, e_2 \rangle \subseteq \cdots \subseteq \langle e_1, \dots, e_r \rangle.
  \]
\item Let \(\Lie g = \C, W = \C^r\) and \(1 \in \Lie g\) acts as \(0\). Then any chain of subspaces
  \[
    W_0 \subseteq W_1 \subseteq \cdots \subseteq W_r
  \]
  with \(\dim W_i = i\) is a composition series.
  \end{enumerate}
\end{eg}

The intuition is that by choosing a suitable basis, we can put each element of \(\Lie g\) into \emph{block triangular form}, with the diagonal blocks \(A_i\) the action on the subquotient \(W_i/W_{i - 1}\), which we require to be irreducible.
\[
  \begin{pmatrix}
    A_1 & & & * \\
    & A_2 \\
    & & \ddots \\
    0 & & & A_r
  \end{pmatrix}
\]

\begin{lemma}
  Composition series always exist.
\end{lemma}

\begin{proof}
  Induct on \(\dim W\). Take an irreducible subrep of \(W\) (why does it always exist?), call it \(W_1\). Then \(W/W_1\) has smaller dimension than \(W\) so has a composition series. Take the preimage of this in \(W\) and stick \(W_1\) in the front.
\end{proof}

\begin{remark}
  The subquotients \(W_i/W_{i - 1}\) are unique (up to reordering). This requires proof in general, but will follow for Lie algebras from what we show in a bit.
\end{remark}

Now we can rephrase the claim as follow: if \(V^\lambda \neq 0\) then \(\lambda = \frac{1}{2} n^2 + n\) for a unique \(n \in \Z_{\geq 0}\), and \(V^\lambda\) has a composition series where all of the subquotients \(W_i/W_{i - 1}\) are isomorphic to \(L(n)\). This proves the slogan ``strings of different lengths don't interact''.

\begin{proof}
  First observe that if \(n \neq m\) then \(\Omega\) acts on \(L(n)\) and \(L(m)\) by different numbers, as \(n \mapsto \frac{1}{2} n^2 + n\) is an increasing function for \(n \geq -1\). Thus if \(V^\lambda \neq 0\), let \(L(n)\) be an irreducible submodule of \(V^\lambda\). As \(\Omega\) acts on \(L(n)\) by \(\frac{1}{2} n^2 + n\), we have \(\lambda = \frac{1}{2}n^2 + n\), and then \(\Omega\) acts on \(V^\lambda/L(n)\) with generalised eigenvalue \(\lambda = \frac{1}{2} n^2 + n\), and for the same reason all composition factors of \(V^\lambda\) must be \(L(n)\) for this \(n\).
\end{proof}

Now we have \(V = \bigoplus_{n \geq 0} V^{\frac{1}{2}n^2 + n}\) where each \(V^{\frac{1}{2}n^2 + n}\) has all composition factors \(L(n)\). We now show strings of the same lengths don't interact.

\begin{lemma}\leavevmode
  \begin{enumerate}
  \item \(hf^k = f^k (h - 2k)\) for all \(k \geq 0\).
  \item \(ef^{k + 1} = f^{k + 1} e + (k + 1) f^k (h - k)\) for all \(k \geq 0\).
  \end{enumerate}
\end{lemma}

\begin{proof}
  Exercise.
\end{proof}

If \(W' \subseteq W\) and \(h\) preserves \(W'\) then the set of generalised eigenvalues of \(h\) on \(W\) is the union on that of \(h\) on \(W'\) and \(W/W'\). As a result, \(h\) acts on \(V^\lambda\) with generalised eigenvalues in \(\{-n, -n + 2, \dots, n - 2, n\}\). Also the only generalised eigenvalue of \(h\) on \(\ker (e: V^\lambda \to V^\lambda)\) is \(n\), that is \((h - n)^{\dim V^\lambda} \cdot x = 0\) for all \(x \in V^\lambda \cap \ker e\).

\begin{proposition}
  \(h\) acts diagonally on \(\ker (e: V^\lambda \to V^\lambda)\), that is it acts by multiplication by \(n\). Thus
  \[
    \ker (e: V^\lambda \to V^\lambda) = (V^\lambda)_n = \{x \in V^\lambda: hx = nx\}.
  \]
\end{proposition}

\begin{proof}
  If \(hx = nx\) then \(ex \in (V^\lambda)_{n + 2} = 0\) so \(x \in \ker e\). Conversely let \(x \in \ker e\). We know \((h - n)^{\dim V^\lambda} x = 0\). By exercises
  \[
    (h - n + 2k)^{\dim V^\lambda} f^kx = f^k (h - n)^{\dim V^\lambda} x = 0
  \]
  so \(f^n x\) is in the generalised eigenspace of \(h\) with eigenvalue \(n - 2k\). Claim that on the other hand, for any \(0 \neq y \in \ker e\), \(f^n y \neq 0\).
  \begin{proof}
    Let \(0 = W_0 \subseteq W_1 \subseteq \cdots \subseteq W_r = V^\lambda\) be a composition series of \(V^\lambda\) such that \(W_i/W_{i - 1} \cong L(n)\) for all \(i\). Then exists \(i\) such that \(y \in W_i, y \neq W_{i - 1}\). Then \(\overline y = y + W_{i - 1} \in W_i/W_{i - 1} \cong L(n)\). Then \(\overline y\) is a highest weight vector of \(L(n)\), so \(f^n(\overline y) \neq 0 \in W_i/W_{i - 1}\) so \(f^n y \neq 0 \in W_i \subseteq V^\lambda\).
  \end{proof}
  
  Now \(f^{n + 1}x\) belongs to the generalised eigenspace of \(h\) with eigenvalue \(-n - 2\), which must be \(0\) by the observation above. Thus \(0 = ef^{n + 1}x\). By exercise this equals to
  \[
    0 = ef^{n + 1}x = (n + 1)f^n (h - n)x + \underbrace{f^{n + 1} ex}_{= 0}
  \]
  so \((n + 1) f^n (h - n)x = 0\). As \(e(h - n)x = (h - n - 2)ex = 0\), we have \((h - n) x \in \ker e\) so if \((h - n)x \neq 0\) then \(f^n (h - n)x \neq 0\). As we are over \(\C\), \(n + 1 \neq 0\) and we just showed \(y \ne o, y \in \ker e\) but \(f^ny \neq 0\), impossible. Thus \((h - x)x = 0\) so \(hx = nx\).
\end{proof}

To show complete reducibility, do the following exercise:

\begin{ex}
  Take a basis \(w_1, \dots, w_k\) of \(\ker e\) and consider the string generated by each \(w_i\), that is \(w_i, fw_i, \dots, f^n w_i\). Show that these give a basis of \(V^\lambda\), each such string is a subrep isomorphic to \(L(n)\) and this gives a direct sum decomposition. In particular \(h\) acts diagonally on all of \(V\) for \(V\) a finite-dimensional rep.
\end{ex}

\begin{ex}
  Show all of this is false in characteristic \(p\). More precisely, show the irreducible reps of \(\Lie{sl}_2\) over \(\overline F_p\) are \emph{not} parameterised by \(n \in \Z_{\geq 0}\). Find a rep of \(\Lie{sl}_2(\overline F_p)\) which does not decompose as a direct sum.
\end{ex}

\subsection{Consequences}

\begin{definition}[tensor product]\index{tensor product}
  Let \(V\) and \(W\) be \(\Lie g\)-reps. Then the \emph{tensor product} of \(V\) and \(W\) is a rep via the map
  \begin{align*}
    \Lie g &\to \End(V \otimes W) = \End(V) \otimes \End(W) \\
    x &\mapsto x \otimes 1 + 1 \otimes x
  \end{align*}
\end{definition}

\begin{ex}\leavevmode
  \begin{enumerate}
  \item Show the above map is a homomorphism of Lie algebras.
  \item Suppose \(G\) acts on \(V\) and \(W\). Show it acts on \(V \otimes W\) by \(g \mapsto g \otimes g\) and the above action is obtained by differentiating this action.
  \end{enumerate}
\end{ex}

Take \(\Lie g = \Lie{sl}_2\). Then by complete reducibility we know \(L(n) \otimes L(m) \cong \bigoplus_{a \geq 0} m_a L(a)\) for some \(m_a\)'s.

\begin{ex}
  Find the highest weight vectors in \(L(1) \otimes L(n)\) and \(L(2) \otimes L(n)\) and hence decompose these.
\end{ex}

To start, let \(v_a\) be a highest weight vector in \(L(a)\). Claim that \(v_n \otimes v_m\) is a highest weight vector in \(L(n) \otimes L(m)\):
\begin{align*}
  h(v_n \otimes v_m &= (hv_n) \otimes v_m + v_n \otimes (hv_m) = (n + m) (v_n \otimes v_m) \\
  e(v_n \otimes v_m) &= (ev_n) \otimes v_m + v_n \otimes (ev_m) = 0
\end{align*}
so \(L(n) \otimes L(m) = L(n + m) \oplus \text{ other stuff}\).

\begin{definition}[character]\index{character}
  Let \(V\) be a finite-dimensional rep of \(\Lie{sl}_2\). Define the \emph{character} of \(V\) to be
  \[
    \ch V = \sum_{n \in \Z} \dim V_n \cdot z^n \in \N[z, z^{-1}].
  \]
\end{definition}

It has the following properties:
\begin{enumerate}
\item \(\ch V|_{z = 1} = \dim V\). This is a consequence of the fact that \(h\) is diagonalisable with integer eigenvalues.
\item \(\ch L(n) = z^n + z^{n - 2} + \dots + z^{2 - n} + z^{-n} = \frac{z^{n + 1} - z^{-{n + 1}}}{z - z^{-1}}\).
\item \(\ch V = \ch W\) if and only if \(V \cong W\) as \(\Lie{sl}_2\) reps.
  \begin{proof}
    Notice that
    \begin{align*}
      \ch L(0) &= 1 \\
      \ch L(1) &= z + z^{-1} \\
      \ch L(2) &= z^2 + 1 + z^{-1} \\
               &\cdots
    \end{align*}
    form a basis of \(\Z[z, z^{-1}]^{S_2}\), the space of symmetric Laurent polynomials with integer coefficients. Now by complete reducibility if \(V \cong \bigoplus _{a\geq 0} n_a L(a), W \cong \bigoplus_{a \geq 0} m_a L(a)\) then \(V \cong W\) if and only if \(n_a = m_a\) for all \(a \geq 0\). As \(\{\ch L(n): n \geq 0\}\) is a basis of \(\Z[z, z^{-1}]^{S_2}\), \(\ch V = \sum m_a \ch L(n)\) determines \(V\).
  \end{proof}
\item \(\ch (V \otimes W) = \ch V \cdot \ch W\).

  Exercise: show that \(V_n \otimes W_m \subseteq (V \otimes W)_{n + m}\) and hence \((V \otimes W)_p = \bigoplus_{n + m = p} V_n \otimes W_m\)
\end{enumerate}








\printindex
\end{document}
