\documentclass[a4paper]{article}

\def\npart{III}

\def\ntitle{Lie Algebras and Their Representations}
\def\nlecturer{I.\ Grojnowski}

\def\nterm{Michaelmas}
\def\nyear{2019}

\ifx \nauthor\undefined
  \def\nauthor{Qiangru Kuang}
\else
\fi

\ifx \ntitle\undefined
  \def\ntitle{Template}
\else
\fi

\ifx \nauthoremail\undefined
  \def\nauthoremail{qk206@cam.ac.uk}
\else
\fi

\ifx \ndate\undefined
  \def\ndate{\today}
\else
\fi

\title{\ntitle}
\author{\nauthor}
\date{\ndate}

%\usepackage{microtype}
\usepackage{mathtools}
\usepackage{amsthm}
\usepackage{stmaryrd}%symbols used so far: \mapsfrom
\usepackage{empheq}
\usepackage{amssymb}
\let\mathbbalt\mathbb
\let\pitchforkold\pitchfork
\usepackage{unicode-math}
\let\mathbb\mathbbalt%reset to original \mathbb
\let\pitchfork\pitchforkold

\usepackage{imakeidx}
\makeindex[intoc]

%to address the problem that Latin modern doesn't have unicode support for setminus
%https://tex.stackexchange.com/a/55205/26707
\AtBeginDocument{\renewcommand*{\setminus}{\mathbin{\backslash}}}
\AtBeginDocument{\renewcommand*{\models}{\vDash}}%for \vDash is same size as \vdash but orginal \models is larger
\AtBeginDocument{\let\Re\relax}
\AtBeginDocument{\let\Im\relax}
\AtBeginDocument{\DeclareMathOperator{\Re}{Re}}
\AtBeginDocument{\DeclareMathOperator{\Im}{Im}}
\AtBeginDocument{\let\div\relax}
\AtBeginDocument{\DeclareMathOperator{\div}{div}}

\usepackage{tikz}
\usetikzlibrary{automata,positioning}
\usepackage{pgfplots}
%some preset styles
\pgfplotsset{compat=1.15}
\pgfplotsset{centre/.append style={axis x line=middle, axis y line=middle, xlabel={$x$}, ylabel={$y$}, axis equal}}
\usepackage{tikz-cd}
\usepackage{graphicx}
\usepackage{newunicodechar}

\usepackage{fancyhdr}

\fancypagestyle{mypagestyle}{
    \fancyhf{}
    \lhead{\emph{\nouppercase{\leftmark}}}
    \rhead{}
    \cfoot{\thepage}
}
\pagestyle{mypagestyle}

\usepackage{titlesec}
\newcommand{\sectionbreak}{\clearpage} % clear page after each section
\usepackage[perpage]{footmisc}
\usepackage{blindtext}

%\reallywidehat
%https://tex.stackexchange.com/a/101136/26707
\usepackage{scalerel,stackengine}
\stackMath
\newcommand\reallywidehat[1]{%
\savestack{\tmpbox}{\stretchto{%
  \scaleto{%
    \scalerel*[\widthof{\ensuremath{#1}}]{\kern-.6pt\bigwedge\kern-.6pt}%
    {\rule[-\textheight/2]{1ex}{\textheight}}%WIDTH-LIMITED BIG WEDGE
  }{\textheight}% 
}{0.5ex}}%
\stackon[1pt]{#1}{\tmpbox}%
}

%\usepackage{braket}
\usepackage{thmtools}%restate theorem
\usepackage{hyperref}

% https://en.wikibooks.org/wiki/LaTeX/Hyperlinks
\hypersetup{
    %bookmarks=true,
    unicode=true,
    pdftitle={\ntitle},
    pdfauthor={\nauthor},
    pdfsubject={Mathematics},
    pdfcreator={\nauthor},
    pdfproducer={\nauthor},
    pdfkeywords={math maths \ntitle},
    colorlinks=true,
    linkcolor={red!50!black},
    citecolor={blue!50!black},
    urlcolor={blue!80!black}
}

\usepackage{cleveref}



% TODO: mdframed often gives bad breaks that cause empty lines. Would like to switch to tcolorbox.
% The current workaround is to set innerbottommargin=0pt.

%\usepackage[theorems]{tcolorbox}





\usepackage[framemethod=tikz]{mdframed}
\mdfdefinestyle{leftbar}{
  %nobreak=true, %dirty hack
  linewidth=1.5pt,
  linecolor=gray,
  hidealllines=true,
  leftline=true,
  leftmargin=0pt,
  innerleftmargin=5pt,
  innerrightmargin=10pt,
  innertopmargin=-5pt,
  % innerbottommargin=5pt, % original
  innerbottommargin=0pt, % temporary hack 
}
%\newmdtheoremenv[style=leftbar]{theorem}{Theorem}[section]
%\newmdtheoremenv[style=leftbar]{proposition}[theorem]{proposition}
%\newmdtheoremenv[style=leftbar]{lemma}[theorem]{Lemma}
%\newmdtheoremenv[style=leftbar]{corollary}[theorem]{corollary}

\newtheorem{theorem}{Theorem}[section]
\newtheorem{proposition}[theorem]{Proposition}
\newtheorem{lemma}[theorem]{Lemma}
\newtheorem{corollary}[theorem]{Corollary}
\newtheorem{axiom}[theorem]{Axiom}
\newtheorem*{axiom*}{Axiom}

\surroundwithmdframed[style=leftbar]{theorem}
\surroundwithmdframed[style=leftbar]{proposition}
\surroundwithmdframed[style=leftbar]{lemma}
\surroundwithmdframed[style=leftbar]{corollary}
\surroundwithmdframed[style=leftbar]{axiom}
\surroundwithmdframed[style=leftbar]{axiom*}

\theoremstyle{definition}

\newtheorem*{definition}{Definition}
\surroundwithmdframed[style=leftbar]{definition}

\newtheorem*{slogan}{Slogan}
\newtheorem*{eg}{Example}
\newtheorem*{ex}{Exercise}
\newtheorem*{remark}{Remark}
\newtheorem*{notation}{Notation}
\newtheorem*{convention}{Convention}
\newtheorem*{assumption}{Assumption}
\newtheorem*{question}{Question}
\newtheorem*{answer}{Answer}
\newtheorem*{note}{Note}
\newtheorem*{application}{Application}

%operator macros

%basic
\DeclareMathOperator{\lcm}{lcm}

%matrix
\DeclareMathOperator{\tr}{tr}
\DeclareMathOperator{\Tr}{Tr}
\DeclareMathOperator{\adj}{adj}

%algebra
\DeclareMathOperator{\Hom}{Hom}
\DeclareMathOperator{\End}{End}
\DeclareMathOperator{\id}{id}
\DeclareMathOperator{\im}{im}
\DeclareMathOperator{\coker}{coker}
\DeclarePairedDelimiter{\generation}{\langle}{\rangle}

%groups
\DeclareMathOperator{\sym}{Sym}
\DeclareMathOperator{\sgn}{sgn}
\DeclareMathOperator{\inn}{Inn}
\DeclareMathOperator{\aut}{Aut}
\DeclareMathOperator{\GL}{GL}
\DeclareMathOperator{\SL}{SL}
\DeclareMathOperator{\PGL}{PGL}
\DeclareMathOperator{\PSL}{PSL}
\DeclareMathOperator{\SU}{SU}
\DeclareMathOperator{\UU}{U}
\DeclareMathOperator{\SO}{SO}
\DeclareMathOperator{\OO}{O}
\DeclareMathOperator{\PSU}{PSU}
\DeclareMathOperator{\Sp}{Sp}


%hyperbolic
\DeclareMathOperator{\sech}{sech}

%field, galois heory
\DeclareMathOperator{\ch}{ch}
\DeclareMathOperator{\gal}{Gal}
\DeclareMathOperator{\emb}{Emb}



%ceiling and floor
%https://tex.stackexchange.com/a/118217/26707
\DeclarePairedDelimiter\ceil{\lceil}{\rceil}
\DeclarePairedDelimiter\floor{\lfloor}{\rfloor}


\DeclarePairedDelimiter{\innerproduct}{\langle}{\rangle}

%\DeclarePairedDelimiterX{\norm}[1]{\lVert}{\rVert}{#1}
\DeclarePairedDelimiter{\norm}{\lVert}{\rVert}



%Dirac notation
%TODO: rewrite for variable number of arguments
\DeclarePairedDelimiterX{\braket}[2]{\langle}{\rangle}{#1 \delimsize\vert #2}
\DeclarePairedDelimiterX{\braketthree}[3]{\langle}{\rangle}{#1 \delimsize\vert #2 \delimsize\vert #3}

\DeclarePairedDelimiter{\bra}{\langle}{\rvert}
\DeclarePairedDelimiter{\ket}{\lvert}{\rangle}




%macros

%general

%divide, not divide
\newcommand*{\divides}{\mid}
\newcommand*{\ndivides}{\nmid}
%vector, i.e. mathbf
%https://tex.stackexchange.com/a/45746/26707
\newcommand*{\V}[1]{{\ensuremath{\symbf{#1}}}}
%closure
\newcommand*{\cl}[1]{\overline{#1}}
%conjugate
\newcommand*{\conj}[1]{\overline{#1}}
%set complement
\newcommand*{\stcomp}[1]{\overline{#1}}
\newcommand*{\compose}{\circ}
\newcommand*{\nto}{\nrightarrow}
\newcommand*{\p}{\partial}
%embed
\newcommand*{\embed}{\hookrightarrow}
%surjection
\newcommand*{\surj}{\twoheadrightarrow}
%power set
\newcommand*{\powerset}{\mathcal{P}}

%matrix
\newcommand*{\matrixring}{\mathcal{M}}

%groups
\newcommand*{\normal}{\trianglelefteq}
%rings
\newcommand*{\ideal}{\trianglelefteq}

%fields
\renewcommand*{\C}{{\mathbb{C}}}
\newcommand*{\R}{{\mathbb{R}}}
\newcommand*{\Q}{{\mathbb{Q}}}
\newcommand*{\Z}{{\mathbb{Z}}}
\newcommand*{\N}{{\mathbb{N}}}
\newcommand*{\F}{{\mathbb{F}}}
%not really but I think this belongs here
\newcommand*{\A}{{\mathbb{A}}}

%asymptotic
\newcommand*{\bigO}{O}
\newcommand*{\smallo}{o}

%probability
\newcommand*{\prob}{\mathbb{P}}
\newcommand*{\E}{\mathbb{E}}

%vector calculus
\newcommand*{\gradient}{\V \nabla}
\newcommand*{\divergence}{\gradient \cdot}
\newcommand*{\curl}{\gradient \cdot}

%logic
\newcommand*{\yields}{\vdash}
\newcommand*{\nyields}{\nvdash}

%differential geometry
\renewcommand*{\H}{\mathbb{H}}
\newcommand*{\transversal}{\pitchfork}
\renewcommand{\d}{\mathrm{d}} % exterior derivative

%number theory
\newcommand*{\legendre}[2]{\genfrac{(}{)}{}{}{#1}{#2}}%Legendre symbol

%algebraic geometry
\DeclareMathOperator{\Spec}{Spec}
\DeclareMathOperator{\Proj}{Proj}

\DeclareMathOperator{\Mat}{Mat}
\newcommand*{\Lie}[1]{\mathfrak{#1}} % Lie groups
\renewcommand*{\P}{\mathbb{P}}
\DeclareMathOperator{\ad}{ad} % adjoint

\begin{document}

\begin{titlepage}
  \begin{center}
    \includegraphics[width=0.6\textwidth]{logo.jpg}\par
    \vspace{1cm}
    {\scshape\huge Mathamatics Tripos \par}
    \vspace{2cm}
    {\huge Part \npart \par}
    \vspace{0.6cm}
    {\Huge \bfseries \ntitle \par}
    \vspace{1.2cm}
    {\Large\nterm, \nyear \par}
    \vspace{2cm}
    
    {\large \emph{Lectures by } \par}
    \vspace{0.2cm}
    {\Large \scshape \nlecturer}
    
    \vspace{0.5cm}
    {\large \emph{Notes by }\par}
    \vspace{0.2cm}
    {\Large \scshape \href{mailto:\nauthoremail}{\nauthor}}
 \end{center}
\end{titlepage}

\tableofcontents

\section{Introduction \& Motivation}

The objects of interest in this course are
\begin{align*}
  \SL_n &= \{A \in \Mat_n: \det A = 1\} \\
  \SO_n &= \{A \in \SL_n: AA^T = I\} \\
  \Sp_{2n} &= \cdots
\end{align*}
and five more examples. First of all they are algebraic groups.

We have \(\SU_2 \subseteq \SL_2\). Note that \(\SU_2\) is homeomorphic to \(S^3\) and so is compact. In fact it is maximal compact and every maximal compact subgroup of \(\SL_2\) is conjugate to \(\SU_2\).

We will look at the tangent space of the group at the identity, which is just a finite-dimensional vector space.

\begin{definition}
  A \emph{linear algebraic group} is a subgroup of \(\Mat_n\) which is defined by polynomial equations in the matrix coefficients.
\end{definition}

For example \(\SL_n\) and \(\SO_n\) are linear algebraic groups. \(\GL_n\) is also an example as we have embedding
\begin{align*}
  \GL_n &\to \Mat_{n + 1} \\
  A &\to
      \begin{pmatrix}
        A & \\
        & \lambda
      \end{pmatrix}
\end{align*}
where the image is given by \(\det A \cdot \lambda = 1\).

\begin{eg}
  Let \(G = \SL_2\) and let
  \[
    g =
    \begin{pmatrix}
      1 & \\
      & 1
    \end{pmatrix}
    + \varepsilon
    \begin{pmatrix}
      a & b \\
      c & d
    \end{pmatrix}
    + \cdots
  \]
  so
  \[
    \det g = 1 + \varepsilon(a + d) + \text{higher terms}
  \]
  so \(\det g = 1\) if and only if \(a + d = 0\) if we pretend to be physicists for a second. Now introduce the dual numbers
  \[
    E = \C[\varepsilon]/(\varepsilon^2) = \{a + b \varepsilon: a, b \in \C\}.
  \]
  If \(G\) is an algebraic group then we define
  \[
    G(E) = \{A \in \Mat_n(E): A \text{ satisfies the defining equations of } G\}.
  \]
  Then
  \[
    \SL_2(E) = \{
    \begin{pmatrix}
      \alpha & \beta \\
      \gamma & \delta
    \end{pmatrix}
    : \alpha, \beta, \gamma, \delta \in E, \alpha \delta - \beta \gamma = 1\}
  \]
  Now the map \(E \to \C, \varepsilon \mapsto 0\) defines a map \(\pi: G(E) \to G\). We define the \emph{Lie algebra}\index{Lie algebra} of \(G\) to be
  \[
    \Lie g \cong \pi^{-1}(I) \cong \{X \in \Mat_n(\C): I + \varepsilon X \in G(E)\}.
  \]
  In particular,
  \[
    \SL_2 = \{
    \begin{pmatrix}
      a & b \\
      c & d
    \end{pmatrix}
    \in \Mat_2(\C): a + d = 0\}.
  \]
\end{eg}

\begin{ex}
  Show that \(G(E) = TG\) is the tangent bundle of \(G\) and \(\Lie g\) is the tangent space at \(1\), \(I + X \varepsilon\) is the germ of a curve through \(1 \in G\).
\end{ex}

\begin{eg}
  Let \(G = \GL_n\). Then
  \begin{align*}
    G(E) &= \{\tilde A \in \Mat_n(E): \tilde A^{-1} \text{ exists}\} \\
         &= \{A + B \varepsilon: A, B \in \Mat_n(\C), A^{-1} \text{ exists}\}
  \end{align*}
  where the second equality is because
  \[
    (A + B \varepsilon) (A^{-1} - A^{-1}B A^{-1} \varepsilon) = I.
  \]
  So there is no condition on \(B\) so \(\Lie{gl}_n = \Mat_n(\C)\). Another explantion for this result is that \(\det\) does not vanish in a neighbourhood of the identity matrix so we get all matrices in the Lie algebra.
\end{eg}

\begin{ex}
  Let \(G = \SL_n\). Show that
  \[
    \det (I + \varepsilon X) = 1 + \varepsilon \tr X
  \]
  and hence
  \[
    \Lie{sl}_n = \{X \in \Mat_n(\C): \tr X = 0\}.
  \]
\end{ex}

\begin{eg}
  Let
  \[
    G = \OO_n = \{A \in \Mat_n: AA^T = I\}.
  \]
  Then
  \begin{align*}
    \Lie g &= \{X \in \Mat_n(\C): (I + \varepsilon X)(I + \varepsilon X)^T = I\} \\
                &= \{X \in \Mat_n(\C): X + X^T = 0\}
  \end{align*}
  Note \(\tr X^T = \tr X\) so \(\tr X = \tr X^T = 0\). Thus \(\SO_n\) has the same Lie algebra. In other words, by just looking into the Lie algebras we cannot distinguish the groups \(\OO_n\) and \(\SO_n\). This is because \(\OO_n\) has two connected component, and the component of the identity is \(\SO_n\). Of course the tangent space at the identity doesn't tell us anything in the other component. Thus this undesirable situation can be remedied by restricting to connected Lie groups.
\end{eg}

What structure does \(\Lie g\) have that it inherits from \(G\)? It is not a (multiplicative) group as
\[
  (I + A \varepsilon) (I + B \varepsilon) = I + \varepsilon (A + B)
\]
has nothing to do with multiplication. Instead, we can consider the commutator
\begin{align*}
  G \times G &\to G \\
  (P, Q) &\mapsto PQP^{-1}Q^{-1}
\end{align*}
This sends \((I, I) \mapsto I\) so by differentiating at the origin we get a map \(\Lie g \times \Lie g \to \Lie g\). Actually, we want a bilinear map \(\Lie g \times \Lie g \to \Lie g\), so differentiate in each variable separately: fix \(P\) and differentiate \(f_P: Q \mapsto PQP^{-1}Q^{-1}\) to get \(df_P: \Lie g \to \Lie g\). Then we differentiate it as a function of \(P\).

Explicitly, write
\begin{align*}
  P &= I + \varepsilon A \\
  Q &= I + \delta B
\end{align*}
where \(\varepsilon^2 = \delta^2 = 0, \varepsilon \delta = \delta \varepsilon \neq 0\). Then
\[
  PQP^{-1}Q^{-1} = I + (AB - BA) \varepsilon\delta
\]
so the map constructed out of the commutators is
\begin{align*}
  \Lie g \times \Lie g &\to \Lie g \\
  (A, B) &\mapsto AB - BA
\end{align*}
This is called the \emph{Lie bracket} of \(A\) and \(B\).

\begin{ex}\leavevmode
  \begin{enumerate}
  \item Show by differentiation that
    \[
      (PQP^{-1}Q^{-1})^{-1} = QPQ^{-1}P^{-1}
    \]
    implies that
    \[
      [B, A] = -[A, B]
    \]
    so the Lie bracket is anti-symmetric.
  \item Show associativity of multiplication implies that
    \[
      [[X, Y], Z] + [[Y, Z], X] + [[Z, X], Y] = 0.
    \]
    This is the \emph{Jacobi identity}.

    Also show this is true from the definition \([A, B] = AB - BA \in \Mat_n\).
  \end{enumerate}
\end{ex}

\begin{definition}[Lie algebra]\index{Lie algebra}
  Let \(k\) be a field, \(\ch k \neq 2\). A \emph{Lie algebra} \(\Lie g\) is a \(k\)-vector space equipped with a bilinear map \([\cdot, \cdot]: \Lie g \times \Lie g \to \Lie g\) that
  \begin{enumerate}
  \item is anti-symmetric: \([X, Y] = - [Y, X]\),
  \item satisfies the Jacobi identity
    \[
      [[X, Y], Z] + [[Y, Z], X] + [[Z, X], Y] = 0.
    \]
  \end{enumerate}
\end{definition}

\begin{eg}\leavevmode
  \begin{enumerate}
  \item \(\Lie{gl}_n = \Mat_n\) with \([A, B] = AB - BA\). More generally, if \(V\) is a vector space, write \(\Lie{gl}(V) = \End(V)\).
  \item \(\Lie{so}_n = \{A \in \Lie{gl}_n: A + A^T = 0\}\).
  \item \(\Lie{sl}_n = \{A \in \Lie{gl}_n: \tr A = 0\}\).
  \item \(\Lie{sp}_{2n} = \{A \in \Lie{gl}_{2n}: JA^TJ^{-1} + A = 0\}\) where
    \[
      J =
      \begin{psmallmatrix}
        & & & & & 1 \\
        & & & & 1 \\
        & & & \cdots \\
        & -1 \\
        -1
      \end{psmallmatrix}
    \]
  \item \(\Lie{b}_n = \{
    \begin{psmallmatrix}
      * & \cdots & * \\
      & \ddots & * \\
      0 & & *
    \end{psmallmatrix}
    \}
    \) of upper triangular matrices.
  \item \(\Lie u_n\) of strictly upper triangular matrices.
  \item If \(V\) is any vector space, let \([\cdot, \cdot]: V \times V \to V\) be the zero map. This is a Lie algebra, called \emph{abelian Lie algebra}.
  \end{enumerate}
\end{eg}

\begin{ex}\leavevmode
  \begin{enumerate}
  \item Show \(\Lie{gl}_n\) is a Lie algebra.
  \item Show examples 2 - 7 are sub-Lie algebras of \(\Lie{gl}_n\).
  \item Find algebraic groups whose Lie algebras are the examples above.
  \item Show \(\{
    \begin{psmallmatrix}
      * & * \\
      * & 0
    \end{psmallmatrix}
    \} \subseteq \Lie{gl}_2\) is not a Lie algebra.
  \end{enumerate}
\end{ex}

\begin{eg}
  Any \(1\)-dim Lie algebra is abelian by anti-symmetry.
\end{eg}

\begin{ex}
  Classify all Lie algebras of dimension \(3\).
\end{ex}

\begin{definition}[representation]\index{representation}
  A \emph{representation} of a Lie algebra \(\Lie g\) on a vector space \(V\) is a Lie algebra homomorphism \(\Lie g \to \Lie{gl}(V)\). We say \(\Lie g\) acts on \(V\).
\end{definition}

We have the silly example of trivial representation: \(\Lie g\) acts on \(V = k\) by \(x \mapsto 0\).

Less trivially, for any \(x \in \Lie g\), define
\begin{align*}
  \ad x: \Lie g &\to \Lie g \\
  y &\mapsto [x, y]
\end{align*}

\begin{lemma}
  \(\ad: \Lie g \to \End(\Lie g)\) is a representation of \(\Lie g\), i.e.\ \(\Lie g\) acts on it self. This is called the \emph{adjoint representation}\index{adjoint representation}.
\end{lemma}

\begin{proof}
  Must show
  \[
    \ad [x, y] = \ad x \ad y - \ad y \ad x.
  \]
  If \(z \in \Lie g\) then
  \begin{align*}
    (\ad [x, y])(z) &= [[x, y], z] \\
    \text{RHS}(z) &= [x, [y, z]] - [y, [x, z]] = -[[y, z], x] - [[z, x], y]
  \end{align*}
  and they are equal by Jacobi.
\end{proof}

\begin{definition}[center]\index{center}
  The \emph{center} of \(\Lie g\) is
  \[
    \{x \in \Lie g: [x, y] = 0 \text{ for all } y \in \Lie g\} = \ker (\ad: \Lie g \to \Lie{gl}(\Lie g)),
  \]
  which is an abelian Lie algebra.
\end{definition}

In particular, the center of \(\Lie g\) is \(0\) if and only if \(\ad\) is an embedding. Question: does every finite-dimensional Lie algebra \(\Lie g\) have a faithful finite-dimensional representation? In other words, does \(\Lie g \embed \Lie{gl}(V)\) for some \(V\)?

Note: every affine algebraic group has a faithful representation.

\begin{theorem}[Ado]
  Any finite-dimensional Lie algebra \(\Lie g\) over \(k\) has a faithful finite-dimensional rep, i.e.\ \(\Lie g \embed \Lie{gl}_n\) for some \(.\)
\end{theorem}

\begin{eg}
  Let \(\Lie g = \Lie{sl}_2\) with basis
  \[
    e =
    \begin{pmatrix}
      0 & 1 \\
      0 & 0
    \end{pmatrix},
    f =
    \begin{pmatrix}
      0 & 0 \\
      1 & 0
    \end{pmatrix},
    h =
    \begin{pmatrix}
      1 & 0 \\
      0 & -1
    \end{pmatrix}
  \]
  so we have
  \[
    [e, f] = h,
    [h, e] = 2e,
    [h, f] = -2f
  \]
  so a representation of \(\Lie{sl}_2\) is a triple of matrices \(E, F, H \in \Mat_n\) with these relations. How can we find such? The answer, at this moment, is to find reps of the algebraic group \(\SL_2\) and differentiating. Later we will find them just by using linear algebra.
\end{eg}

\begin{definition}[algebraic representation]\index{algebraic representation}
  If \(G\) is an algebraic group. An \emph{algebraic representation} of \(G\) on a vector space \(V\) is a homomorphism \(G \to \GL(V)\) defined by polynomial equations in the matrix coefficients.
\end{definition}

Let \(\rho: G \to \GL(V)\) be an algebraic rep. We have \(\rho(I) = I\). Consider the map \(G(E) \to \GL(V)(E)\). We get
\[
  \rho(I + A\varepsilon) = I + \varepsilon d\rho(A)
\]
for some function \(d\rho(A)\) of \(A\).

\begin{ex}
  \(d \rho\) is the derivative of \(\rho\) at identity.
\end{ex}

\begin{ex}
  \(\rho: G \to \GL(V)\) implies that \(d\rho: \Lie g \to \Lie{gl}(V)\) is a Lie algebra homomorphism, so \(V\) is a representation of \(\Lie g\).
\end{ex}

Let \(G = \SL_2\) and let \(L(n)\) be homogeneous polynomials in \(x, y\) of degree \(n\), with basis \(x^n, x^{n - 1}y, \cdots, y^n\), so has dimension \(n + 1\). \(\GL_2\) acts on \(L(n)\) by change of coordinates: if \(g =
\begin{pmatrix}
  a & b \\
  c & d
\end{pmatrix}
\), \(f \in L(n)\) then
\[
  (\rho_n(g)f)(x, y) = f(ax + cy, bx + dy).
\]
Check that
\begin{enumerate}
\item \(\rho_0\) is the trivial rep.
\item \(\rho_1\) is the usual \(2\)-dim rep.
\item
  \[
    \rho_2
    \begin{pmatrix}
      a & b \\
      c & d
    \end{pmatrix}
    =
    \begin{pmatrix}
      a^2 & ab & b^2 \\
      2ac & ad + bc & 2bd \\
      c^2 & cd & d^2
    \end{pmatrix}
  \]
\end{enumerate}
Differentiate and we get an action of \(\Lie{sl}_2\) on \(L(n)\). Explicitly,
\[
  \rho(I + \varepsilon e) x^iy^j
  = x^i (y + \varepsilon x)^j
  = x^iy^j + \varepsilon jx^{i + 1} y^{j - 1}
\]
and hence
\[
  d\rho(e) x^iy^j = jx^{i + 1} y^{j - 1}.
\]

\begin{ex}\leavevmode
  \begin{enumerate}
  \item The Lie algebra acts by
    \begin{align*}
      e \cdot (x^iy^j) &= jx^{i + 1} y^{j - 1} \\
      f \cdot (x^iy^j) &= ix^{i - 1} y^{j + 1} \\
      h \cdot (x^iy^j) &= (i - j) x^iy^j
    \end{align*}
  \item Check directly this gives a rep of \(\Lie{sl}_2\).
  \item Show \(L(2)\) is isomorphic to the adjoint rep.
  \item Show that
    \[
      e = x \frac{\partial  }{\partial y}, f = y \frac{\partial  }{\partial x}, h = x \frac{\partial  }{\partial x} - y \frac{\partial  }{\partial y}
    \]
    defines an (infinite-dimensional) rep of \(\Lie{sl}_2\) on \(k[x, y]\). Some implication: this can be defined for all characteristics, and the differential operator is suggesting that reps of Lie groups might have something to do with calculus.
  \end{enumerate}
\end{ex}




\printindex
\end{document}
