\documentclass[a4paper]{article}

\def\npart{III}

\def\ntitle{Elliptic Curves}
\def\nlecturer{T.\ A.\ Fisher}

\def\nterm{Michaelmas}
\def\nyear{2019}

\ifx \nauthor\undefined
  \def\nauthor{Qiangru Kuang}
\else
\fi

\ifx \ntitle\undefined
  \def\ntitle{Template}
\else
\fi

\ifx \nauthoremail\undefined
  \def\nauthoremail{qk206@cam.ac.uk}
\else
\fi

\ifx \ndate\undefined
  \def\ndate{\today}
\else
\fi

\title{\ntitle}
\author{\nauthor}
\date{\ndate}

%\usepackage{microtype}
\usepackage{mathtools}
\usepackage{amsthm}
\usepackage{stmaryrd}%symbols used so far: \mapsfrom
\usepackage{empheq}
\usepackage{amssymb}
\let\mathbbalt\mathbb
\let\pitchforkold\pitchfork
\usepackage{unicode-math}
\let\mathbb\mathbbalt%reset to original \mathbb
\let\pitchfork\pitchforkold

\usepackage{imakeidx}
\makeindex[intoc]

%to address the problem that Latin modern doesn't have unicode support for setminus
%https://tex.stackexchange.com/a/55205/26707
\AtBeginDocument{\renewcommand*{\setminus}{\mathbin{\backslash}}}
\AtBeginDocument{\renewcommand*{\models}{\vDash}}%for \vDash is same size as \vdash but orginal \models is larger
\AtBeginDocument{\let\Re\relax}
\AtBeginDocument{\let\Im\relax}
\AtBeginDocument{\DeclareMathOperator{\Re}{Re}}
\AtBeginDocument{\DeclareMathOperator{\Im}{Im}}
\AtBeginDocument{\let\div\relax}
\AtBeginDocument{\DeclareMathOperator{\div}{div}}

\usepackage{tikz}
\usetikzlibrary{automata,positioning}
\usepackage{pgfplots}
%some preset styles
\pgfplotsset{compat=1.15}
\pgfplotsset{centre/.append style={axis x line=middle, axis y line=middle, xlabel={$x$}, ylabel={$y$}, axis equal}}
\usepackage{tikz-cd}
\usepackage{graphicx}
\usepackage{newunicodechar}

\usepackage{fancyhdr}

\fancypagestyle{mypagestyle}{
    \fancyhf{}
    \lhead{\emph{\nouppercase{\leftmark}}}
    \rhead{}
    \cfoot{\thepage}
}
\pagestyle{mypagestyle}

\usepackage{titlesec}
\newcommand{\sectionbreak}{\clearpage} % clear page after each section
\usepackage[perpage]{footmisc}
\usepackage{blindtext}

%\reallywidehat
%https://tex.stackexchange.com/a/101136/26707
\usepackage{scalerel,stackengine}
\stackMath
\newcommand\reallywidehat[1]{%
\savestack{\tmpbox}{\stretchto{%
  \scaleto{%
    \scalerel*[\widthof{\ensuremath{#1}}]{\kern-.6pt\bigwedge\kern-.6pt}%
    {\rule[-\textheight/2]{1ex}{\textheight}}%WIDTH-LIMITED BIG WEDGE
  }{\textheight}% 
}{0.5ex}}%
\stackon[1pt]{#1}{\tmpbox}%
}

%\usepackage{braket}
\usepackage{thmtools}%restate theorem
\usepackage{hyperref}

% https://en.wikibooks.org/wiki/LaTeX/Hyperlinks
\hypersetup{
    %bookmarks=true,
    unicode=true,
    pdftitle={\ntitle},
    pdfauthor={\nauthor},
    pdfsubject={Mathematics},
    pdfcreator={\nauthor},
    pdfproducer={\nauthor},
    pdfkeywords={math maths \ntitle},
    colorlinks=true,
    linkcolor={red!50!black},
    citecolor={blue!50!black},
    urlcolor={blue!80!black}
}

\usepackage{cleveref}



% TODO: mdframed often gives bad breaks that cause empty lines. Would like to switch to tcolorbox.
% The current workaround is to set innerbottommargin=0pt.

%\usepackage[theorems]{tcolorbox}





\usepackage[framemethod=tikz]{mdframed}
\mdfdefinestyle{leftbar}{
  %nobreak=true, %dirty hack
  linewidth=1.5pt,
  linecolor=gray,
  hidealllines=true,
  leftline=true,
  leftmargin=0pt,
  innerleftmargin=5pt,
  innerrightmargin=10pt,
  innertopmargin=-5pt,
  % innerbottommargin=5pt, % original
  innerbottommargin=0pt, % temporary hack 
}
%\newmdtheoremenv[style=leftbar]{theorem}{Theorem}[section]
%\newmdtheoremenv[style=leftbar]{proposition}[theorem]{proposition}
%\newmdtheoremenv[style=leftbar]{lemma}[theorem]{Lemma}
%\newmdtheoremenv[style=leftbar]{corollary}[theorem]{corollary}

\newtheorem{theorem}{Theorem}[section]
\newtheorem{proposition}[theorem]{Proposition}
\newtheorem{lemma}[theorem]{Lemma}
\newtheorem{corollary}[theorem]{Corollary}
\newtheorem{axiom}[theorem]{Axiom}
\newtheorem*{axiom*}{Axiom}

\surroundwithmdframed[style=leftbar]{theorem}
\surroundwithmdframed[style=leftbar]{proposition}
\surroundwithmdframed[style=leftbar]{lemma}
\surroundwithmdframed[style=leftbar]{corollary}
\surroundwithmdframed[style=leftbar]{axiom}
\surroundwithmdframed[style=leftbar]{axiom*}

\theoremstyle{definition}

\newtheorem*{definition}{Definition}
\surroundwithmdframed[style=leftbar]{definition}

\newtheorem*{slogan}{Slogan}
\newtheorem*{eg}{Example}
\newtheorem*{ex}{Exercise}
\newtheorem*{remark}{Remark}
\newtheorem*{notation}{Notation}
\newtheorem*{convention}{Convention}
\newtheorem*{assumption}{Assumption}
\newtheorem*{question}{Question}
\newtheorem*{answer}{Answer}
\newtheorem*{note}{Note}
\newtheorem*{application}{Application}

%operator macros

%basic
\DeclareMathOperator{\lcm}{lcm}

%matrix
\DeclareMathOperator{\tr}{tr}
\DeclareMathOperator{\Tr}{Tr}
\DeclareMathOperator{\adj}{adj}

%algebra
\DeclareMathOperator{\Hom}{Hom}
\DeclareMathOperator{\End}{End}
\DeclareMathOperator{\id}{id}
\DeclareMathOperator{\im}{im}
\DeclareMathOperator{\coker}{coker}
\DeclarePairedDelimiter{\generation}{\langle}{\rangle}

%groups
\DeclareMathOperator{\sym}{Sym}
\DeclareMathOperator{\sgn}{sgn}
\DeclareMathOperator{\inn}{Inn}
\DeclareMathOperator{\aut}{Aut}
\DeclareMathOperator{\GL}{GL}
\DeclareMathOperator{\SL}{SL}
\DeclareMathOperator{\PGL}{PGL}
\DeclareMathOperator{\PSL}{PSL}
\DeclareMathOperator{\SU}{SU}
\DeclareMathOperator{\UU}{U}
\DeclareMathOperator{\SO}{SO}
\DeclareMathOperator{\OO}{O}
\DeclareMathOperator{\PSU}{PSU}
\DeclareMathOperator{\Sp}{Sp}


%hyperbolic
\DeclareMathOperator{\sech}{sech}

%field, galois heory
\DeclareMathOperator{\ch}{ch}
\DeclareMathOperator{\gal}{Gal}
\DeclareMathOperator{\emb}{Emb}



%ceiling and floor
%https://tex.stackexchange.com/a/118217/26707
\DeclarePairedDelimiter\ceil{\lceil}{\rceil}
\DeclarePairedDelimiter\floor{\lfloor}{\rfloor}


\DeclarePairedDelimiter{\innerproduct}{\langle}{\rangle}

%\DeclarePairedDelimiterX{\norm}[1]{\lVert}{\rVert}{#1}
\DeclarePairedDelimiter{\norm}{\lVert}{\rVert}



%Dirac notation
%TODO: rewrite for variable number of arguments
\DeclarePairedDelimiterX{\braket}[2]{\langle}{\rangle}{#1 \delimsize\vert #2}
\DeclarePairedDelimiterX{\braketthree}[3]{\langle}{\rangle}{#1 \delimsize\vert #2 \delimsize\vert #3}

\DeclarePairedDelimiter{\bra}{\langle}{\rvert}
\DeclarePairedDelimiter{\ket}{\lvert}{\rangle}




%macros

%general

%divide, not divide
\newcommand*{\divides}{\mid}
\newcommand*{\ndivides}{\nmid}
%vector, i.e. mathbf
%https://tex.stackexchange.com/a/45746/26707
\newcommand*{\V}[1]{{\ensuremath{\symbf{#1}}}}
%closure
\newcommand*{\cl}[1]{\overline{#1}}
%conjugate
\newcommand*{\conj}[1]{\overline{#1}}
%set complement
\newcommand*{\stcomp}[1]{\overline{#1}}
\newcommand*{\compose}{\circ}
\newcommand*{\nto}{\nrightarrow}
\newcommand*{\p}{\partial}
%embed
\newcommand*{\embed}{\hookrightarrow}
%surjection
\newcommand*{\surj}{\twoheadrightarrow}
%power set
\newcommand*{\powerset}{\mathcal{P}}

%matrix
\newcommand*{\matrixring}{\mathcal{M}}

%groups
\newcommand*{\normal}{\trianglelefteq}
%rings
\newcommand*{\ideal}{\trianglelefteq}

%fields
\renewcommand*{\C}{{\mathbb{C}}}
\newcommand*{\R}{{\mathbb{R}}}
\newcommand*{\Q}{{\mathbb{Q}}}
\newcommand*{\Z}{{\mathbb{Z}}}
\newcommand*{\N}{{\mathbb{N}}}
\newcommand*{\F}{{\mathbb{F}}}
%not really but I think this belongs here
\newcommand*{\A}{{\mathbb{A}}}

%asymptotic
\newcommand*{\bigO}{O}
\newcommand*{\smallo}{o}

%probability
\newcommand*{\prob}{\mathbb{P}}
\newcommand*{\E}{\mathbb{E}}

%vector calculus
\newcommand*{\gradient}{\V \nabla}
\newcommand*{\divergence}{\gradient \cdot}
\newcommand*{\curl}{\gradient \cdot}

%logic
\newcommand*{\yields}{\vdash}
\newcommand*{\nyields}{\nvdash}

%differential geometry
\renewcommand*{\H}{\mathbb{H}}
\newcommand*{\transversal}{\pitchfork}
\renewcommand{\d}{\mathrm{d}} % exterior derivative

%number theory
\newcommand*{\legendre}[2]{\genfrac{(}{)}{}{}{#1}{#2}}%Legendre symbol

%algebraic geometry
\DeclareMathOperator{\Spec}{Spec}
\DeclareMathOperator{\Proj}{Proj}

% define \Sh to be Cyrillic sha
% https://tex.stackexchange.com/a/124746

\DeclareFontFamily{U}{wncy}{} \DeclareFontShape{U}{wncy}{m}{n}{<->wncyr10}{} \DeclareSymbolFont{mcy}{U}{wncy}{m}{n} \DeclareMathSymbol{\Sh}{\mathord}{mcy}{"58}

\theoremstyle{definition}
\newtheorem*{fact}{Fact}

\theoremstyle{theorem}
\newtheorem*{conjecture}{Conjecture}

\renewcommand*{\P}{\mathbb{P}}
\DeclareMathOperator{\ord}{ord}
\DeclareMathOperator{\Div}{Div} % divisor
\DeclareMathOperator{\Pic}{Pic} % Picard group
\newcommand{\rational}{\dashrightarrow} % rational map
\renewcommand*{\O}{\mathcal{O}}
\DeclareMathOperator{\Cl}{Cl} % Class group
\DeclareMathOperator{\Sum}{sum} % sum map

\begin{document}


\begin{titlepage}
  \begin{center}
    \includegraphics[width=0.6\textwidth]{logo.jpg}\par
    \vspace{1cm}
    {\scshape\huge Mathamatics Tripos \par}
    \vspace{2cm}
    {\huge Part \npart \par}
    \vspace{0.6cm}
    {\Huge \bfseries \ntitle \par}
    \vspace{1.2cm}
    {\Large\nterm, \nyear \par}
    \vspace{2cm}
    
    {\large \emph{Lectures by } \par}
    \vspace{0.2cm}
    {\Large \scshape \nlecturer}
    
    \vspace{0.5cm}
    {\large \emph{Notes by }\par}
    \vspace{0.2cm}
    {\Large \scshape \href{mailto:\nauthoremail}{\nauthor}}
 \end{center}
\end{titlepage}

\tableofcontents

\section{Fermat's method of infinite descent}

Let \(\Delta = (a, b, c)\) be a right angle triangle with sides \(a, b, c\) where \(c\) is the hypotenuse.

\begin{definition}
  \(\Delta\) is rational if \(a, b, c \in \Q\). \(\Delta\) is primitive if \(a, b, c \in \Z\) and coprime.
\end{definition}

\begin{lemma}
  Every primitive triangle is of the form \((u^2 - v^2, 2uv, u^2 + v^2)\) for some \(u, v \in \Z, u > v > 0\).
\end{lemma}

\begin{proof}
  \(a\) and \(b\) cannot be both even. They cannot be both odd as then \(c^2 = 2 \mod 4\). Thus wlog \(a\) is odd and \(b\) is even, so \(c\) odd. Then
  \[
    \left(\frac{b}{2}\right)^2 = \frac{c + a}{2} \cdot \frac{c - a}{2}
  \]
  and the two terms on RHS are coprime positive integers. By unique factorisation in \(\Z\), there exist \(u, v \in \Z\) such that
  \begin{align*}
    \frac{c + a}{2} &= u^2 \\
    \frac{c - a}{2} &= v^2
  \end{align*}
  Rearrange.
\end{proof}

\begin{definition}
  \(D \in \Q_{> 0}\) is a \emph{congruent number} if there exists a right angle triangle whose area is \(D\).
\end{definition}

\begin{note}
  Suffices to consider \(D \in \Z_{> 0}\) square-free.
\end{note}

\begin{eg}
  \(D = 5, 6\) are congruent.
\end{eg}

\begin{lemma}
  \(D \in \Q_{> 0}\) is congruent if and only if \(D y^2 = x^3 - x\) for some \(x, y \in \Q, y \neq 0\).
\end{lemma}

\begin{proof}
  Lemma 1 shows that \(D\) is congruent if and only if \(Dw^2 = uv(u^2 - v^2)\) for some \(u, v, w \in \Q, w \neq 0\). Let \(x = \frac{u}{v}, y = \frac{w}{v^2}\).
\end{proof}

Fermat showed that \(1\) is not a congruent number.

\begin{theorem}
  There are no solutions to
  \begin{equation}
    \label{eqn:fermat}
    w^2 = uv (u - v)(u + v)
    \tag{\ast}
  \end{equation}
  for \(u, v, w \in \Z, w \neq 0\).
\end{theorem}

\begin{proof}
  wlog \(u, v \) coprime, \(u > 0, w > 0\). If \(v < 0\) then replace \((u, v, w)\) by \((-v, u, w)\). If \(u = v \mod 2\) then replace \((u, v, w)\) by \((\frac{u + v}{2}, \frac{u - v}{2}, \frac{w}{2})\). Then \(u, v, u - v, u + v\) are positive coprime integers whose product is a square. By unique prime factorisation, \(u = a^2, v = b^2, u + v = c^2, u - v = d^2\) for some \(a, b, c, d \in \Z_{> 0}\). As \(u \neq v \mod 2\), \(c, d\) are both odd. Consider a new triangle with sides \(\frac{c + d}{2}, \frac{c - d}{2}\). Then
  \[
    \left( \frac{c + d}{2} \right)^2 + \left( \frac{c - d}{2} \right)^2 = \frac{c^2 + d^2}{2} = u = a^2
  \]
  so this is another primitive triangle. Its area is
  \[
    \frac{c^2 - d^2}{8} = \frac{v}{4} = \left( \frac{b}{2} \right)^2.
  \]

  Let \(w_1 = \frac{b}{2}\) so by lemma 1
  \[
    w_1^2 = u_1v_1 (u_1 - v_1)(u_1 + v_1),
  \]
  i.e.\ we have a new solution to \eqref{eqn:fermat}. But \(4 w_1^2 = b^2 = v \divides w^2\) so \(w_1 \leq \frac{1}{2} w\). So by Fermat's method of infinite descend, there is no solution to \eqref{eqn:fermat}.
\end{proof}

\subsection{A variant for polynomials}

Let \(K\) be a field with \(\ch K \neq 2\). Let \(\overline K\) be an algebraic closure of \(k\).

\begin{lemma}
  Let \(u, v \in K[t]\) coprime. If \(\alpha u + \beta v\) is a square for four distinct \((\alpha: \beta) \in \P^1\) then \(u, v \in K\).
\end{lemma}

\begin{proof}
  wlog \(K = \overline K\). Changing coordinates on \(\P^1\), we may assume the ratio \((\alpha: \beta)\) are \((1: 0), (0: 1), (1: -1), (1: -\lambda)\) for some \(\lambda \in K \setminus \{0, 1\}\). Thus we have
  \begin{align*}
    u &= a^2 \\
    v &= b^2 \\
    u - v &= (a - b)(a + b) \\
    u - \lambda v &= (a - \mu b)(a + \mu b)
  \end{align*}
  where \(\mu = \sqrt \lambda\). Use unqiue factorisation in \(K[t]\),  as \(a, b\) are coprime, \(a + b, a - b, a - \mu b, a + \mu b\) are squares. But
  \[
    \max (\deg (a), \deg (b)) \leq \frac{1}{2} \max (\deg (u), \deg (v))
  \]
  so by Fermat's method of infinite descend, \(u, v \in K\).
\end{proof}

\begin{definition}[elliptic curve]\index{ellptic curve}\leavevmode
  \begin{enumerate}
  \item An \emph{elliptic curve} \(E/K\) is the projective closure of a plane affine curve \(y^2 = f(x)\) where \(f \in K[x]\) is a monic cubic polynomial with distinct roots in \(\overline K\). The equation \(y^2 = f(x)\) is called a \emph{Weierstrass function}\index{Weierstrass function}.
  \item For \(L/K\) a field extension,
    \[
      E(L) = \{(x, y) \in L^2: y^2 = f(x)\} \cup \{0\}
    \]
    where \(0\) is the point at infinity in the projective closure.
  \end{enumerate}
\end{definition}

Fact: \(E(L)\) is naturally an abelian group.

In this course we study \(E(L)\) for \(L\) finite field, local field (meaning \(L/\Q_p\) finite in this course) or number field (\(L/\Q\) finite).

\begin{theorem}
  If \(E: y^2 = x^3 - x\) then \(E(\Q) = \{0, (0, 0), (\pm 1, 0)\}\).
\end{theorem}

\begin{corollary}
  Let \(E/K\) be an elliptic curve. Then \(E(K(t)) = E(K)\).
\end{corollary}

\begin{proof}
  wlog \(K = \overline K\). By a change of coordinates we may assume
  \[
    E: y^2 = x(x - 1)(x - \lambda)
  \]
  for some \(\lambda \in K \setminus \{0, 1\}\). Suppose \((x, y) \in E(K(t))\). Write \(x = \frac{u}{v}\) where \(u, v \in K[t]\) coprime. Then
  \[
    w^2 = uv(u - v)(u - \lambda v)
  \]
  for some \(w \in K[t]\). Using same unique factorisation argument as before, \(u, v, u - v, u - \lambda v\) are all squares so by lemma \(u, v \in K\) so \(x, y \in K\).
\end{proof}

\section{Some remarks on algebraic curves}

Let \(K = \overline K, \ch K \neq 2\).

\begin{definition}[rational plane curve]\index{rational plane curve}
  A plane algebraic curve (always assumed to be irreducible)
  \[
    C = \{f(x, y) = 0\} \subseteq \A^2
  \]
  is \emph{rational} if it has a rational parameterisation, i.e.\ there exist \(\phi, \psi \in K(t)\) such that
  \begin{enumerate}
  \item \(\A^1 \to \A^2, t \mapsto (\phi(t), \psi(t))\) is injective on \(\A^1 \setminus \{\text{finite set}\}\).
  \item \(f(\phi(t), \psi(t)) = 0\).
  \end{enumerate}
\end{definition}

\begin{eg}\leavevmode
  \begin{enumerate}
  \item Any nonsingular plane conic is rational. For example \(x^2 + y^2 = 1\). Pick a point \((-1, 0)\). Putting a line through the point with slope \(t\), i.e.\ \(y = t(x + 1)\). Solve for the intersection. In general we will get a root, which is not rational. But in the quadratic case we already have one solution so the other solution can be expressed as a rational function. we have
    \[
      x^2 + t^2(x + 1)^2 = 1
    \]
    which is saying
    \[
      (x + 1)(x - 1 + t^2(x + 1)) = 0
    \]
    so \(x = -1\) or \(x = \frac{1 - t^2}{1 + t^2}\). Similarly one can solve \(y\). Then we get rational parameterisation
    \[
      (x, y) = \left( \frac{1 - t^2}{1 + t^2}, \frac{2t}{1 + t^2} \right).
    \]
  \item Any singular plane curve is rational. Two examples: \(y^2 = x^3, y^2 = x^2 (x + 1)\). Same recipe as before except that we have to pick the singular point, which is the origin in both cases. The line \(y = tx\) intersects the curve. We get rational parameterisation \((x, y) = (t^2, t^3)\) for the first one. The second is an exercise.
  \item Corollary 1.6 shows that elliptic curves are \emph{not} rational.
  \end{enumerate}
\end{eg}

\begin{remark}
  The genus \(g(C) \in \Z_{\geq 0}\) is an invariant of a smooth projective curve \(C\). Some facts:
  \begin{enumerate}
  \item if \(k = \C\) then \(g(C)\) is the genus of the Riemann surface.
  \item a smooth plane curve \(C \subseteq \P^2\) of degree \(d\) has genus \(g(C) = \frac{(d - 1)(d - 2)}{2}\).
  \end{enumerate}
\end{remark}

\begin{proposition}
  Let \(C\) be a smooth projective curve.
  \begin{enumerate}
  \item \(C\) is rational if and only if \(g(C) = 0\).
  \item \(C\) is an elliptic curve if and only if \(g(C) = 1\).
  \end{enumerate}
\end{proposition}

\begin{proof}\leavevmode
  \begin{enumerate}
  \item Omitted.
  \item For only if, check the projective closure is smooth and use remark. For if, see later.
  \end{enumerate}
\end{proof}

\subsection{Order of vanishing}

Let \(C\) be an algebraic curve with function field \(K(C)\). Let \(P \in C\) be a smooth point. We write \(\ord_P(f)\) to be the order of vanishing to be the order of vanishing of \(f \in K(C)\) at \(P\). It is negative if \(f\) has a pole at \(P\).

Some facts: \(\ord_P(f): K(C)^* \to \Z\) is a discrete valuation, i.e.
\begin{align*}
  \ord_P(f_1f_2) &= \ord_P(f_1) + \ord_P(f_2) \\
  \ord_P(f_1 + f_2) & \geq \min(\ord_P(f_1), \ord_P(f_2))
\end{align*}

\begin{definition}[uniformiser]\index{uniformiser}
  \(t \in K(C)^*\) is a \emph{uniformiser} at \(P\) if \(\ord_P(t) = 1\).
\end{definition}

\begin{eg}
  Let \(C = \{g = 0\} \subseteq \A^2\) for some \(g \in K[x, y]\) irreducible. Then
  \[
    K(C) = \operatorname{Frac} \frac{K[x, y]}{(g)}.
  \]
  Write
  \[
    g = g_0 + g_1(x, y) + g_2(x, y) + \dots
  \]
  where \(g_i\) is homogeneous of degree \(i\). Suppose \(P = (0, 0) \in C\) is smooth, i.e.\ \(g_0 = 0, g_1(x, y) = \alpha x + \beta y\) where \(\alpha, \beta\) not both zero. (Picture). Let \(\gamma, \delta \in K\). It is a fact that \(\gamma x + \delta y \in K(C)\) is a uniformiser at \(P\) if and only if \(\alpha \delta - \beta \gamma \neq 0\).
\end{eg}

\begin{eg}
  Consider \(\{y^2 = x(x - 1)(x - \lambda)\} \subseteq \A^2\) where \(\lambda \neq 0, 1\). Its projective closure is \(\{Y^2Z = X(X - Z)(X - \lambda Z)\} \subseteq \P^2\), then we get one point \(P = (0: 1: 0)\) at infinity. We can compute \(\ord_P(x)\) and \(\ord_P(y)\). We work on the affine piece \(\{Y \neq 0\}\). Put \(w = \frac{Z}{Y}, t = \frac{X}{Y}\), then the equation becomes
  \[
    w = t(t - w)(t - \lambda w).
  \]
  Now \(P\) is the point \((t, w) = (0, 0)\). This is a smooth point and using the fact in the above example,
  \[
    \ord_P(t) = \ord_P(t - w) = \ord_P(t - \lambda w) = 1,
  \]
  so \(\ord_P(w) = 3\). Finally,
  \begin{align*}
    \ord_P(x) &= \ord_P \frac{X}{Z} = \ord_P \frac{t}{w} = -2 \\
    \ord_P(y) &= \ord_P \frac{Y}{Z} = \ord_P \frac{1}{w} = -3
  \end{align*}
\end{eg}

Let \(C\) be a smooth projective curve.

\begin{definition}[divisor]\index{divisor}
  A \emph{divisor} is a formal sum of points on \(C\), say \(D = \sum_{P \in C} n_P P\) with \(n_P \in \Z\) and \(n_P = 0\) for all but finitely many \(P\). The \emph{degree} of \(D\) is
  \[
    \deg D = \sum n_P.
  \]
\end{definition}

\begin{definition}[effective divisor]\index{divisor!effective}
  A divisor \(D\) is \emph{effective}, written \(D \geq 0\), if \(n_P \geq 0\) for all \(P\).
\end{definition}

If \(f \in K(C)^*\) then we write
\[
  \div(f) = \sum_{P \in C} \ord_P(f) P.
\]

The \emph{Riemann-Roch space} of \(D \in \Div(C)\) is
\[
  \mathcal L(D) = \{f \in K(C)^*: \div(f) + D \geq 0\} \cup \{0\},
\]
i.e.\ the \(K\)-vector space of rational functions on \(C\) with ``pole no worse than specified by \(D\)''.

Riemann-Roch for genus \(1\) curve says that
\[
  \dim \mathcal L(D) =
  \begin{cases}
    \deg D & \deg D > 0 \\
    0 \text{ or } 1 & \deg D = 0 \\
    0 & \deg D < 0 
  \end{cases}
\]

\begin{eg}
  Let us revisit some of the previous example. Consider \(\{y^2 = x(x - 1)(x - \lambda)\} \subseteq \A^2\) and let \(P\) the point at infinity. We calculated \(\ord_P(x) = -2, \ord_P(y) = -3\). Then
  \begin{align*}
    \mathcal L(2P) &= \langle 1, x \rangle \\
    \mathcal L(3P) &= \langle 1, x, y \rangle
  \end{align*}
\end{eg}

\begin{proposition}
  Let \(C \subseteq \P^2\) be a smooth plane cubic and \(P \in C\) a point of inflection. Then we can change coordinates such that \(C: Y^2Z = X(X - Z)(X - \lambda Z)\) and \(P = (0: 1: 0)\).
\end{proposition}

\begin{fact}
  The points of inflection on \(C = \{F = 0\} \subseteq \P^2\) are given by
  \[
    F = \det \frac{\p^2 F}{\p x_i \p x_j} = 0.
  \]
\end{fact}

\begin{proof}
  We change coordinates such that \(P = (0 : 1 : 0)\) and \(T_pC = \{Z = 0\}\), where \(C = \{F(X, Y, Z) = 0\}\). \(P \in C\) is a point of inflection, meaning that the intersection of the tangent at \(P\) with \(C\) has multiplicity \(3\), so \(F(t, 1, 0)\) is a constant multiple of \(t^3\). Thus there is no \(X^2Y, XY^2\) and \(Y^3\) term, so
  \[
    F \in \langle Y^2Z, XYZ, YZ^2, X^3, X^2Z, XZ^2, Z^3 \rangle.
  \]
  The coefficient of \(X^3\) is nonzero as otherwise \(\{Z = 0\} \subseteq C\). The coefficient of \(Y^2Z\) is nonzero as otherwise \(P \in C\) is singular. We are free to rescale \(X, Y, Z\) and \(F\), so wlog \(C\) is defined by
  \[
    Y^2Z + a_1 XYZ + a_3 YZ^2 = X^3 + a_2 X^2Z + a_4 XZ^2 + a_6 Z^3.
  \]
  Making substitutions \(Y \mapsto Y - \frac{1}{2} a_1X - \frac{1}{2} a_3 X\), w may asssume \(a_1 = a_3 = 0\). Now \(C: Y^2Z = Z^3 f(X/Z)\) where \(f\) is a monic cubic polynomial. As \(C\) is smooth, \(f\) has distinct roots so wlog \(0, 1, \lambda\) so \(C\) is
  \[
    Y^2Z = X(X - Z)(X - \lambda Z).
  \]
\end{proof}

The equation
\[
  Y^2Z + a_1 XYZ + a_3 YZ^2 = X^3 + a_2 X^2Z + a_4 XZ^2 + a_6 Z^3
\]
is called \emph{Weierstrass form}\index{Weierstrass equation} and
\[
  Y^2Z = X(X - Z)(X - \lambda Z)
\]
is called \emph{Legendre form}.

\subsection{Degree of a morphism}

Let \(\phi: C_1 \to C_2\) be a nonconstant morphism of smooth projective curves. Let \(\phi^*: K(C_2) \to K(C_1)\) be the pullback by \(\phi\).

\begin{definition}[degree of morphism]
  The \emph{degree} of \(\phi\) is
  \[
    \deg \phi = [K(C_1): \phi^*K(C_2)],
  \]
  the degree of the field extension. \(\phi\) is \emph{separable} if the corresponding field extension is separable (which is automatic if \(\ch K = 0\)).
\end{definition}

\begin{fact}
  \(\deg \phi = 1\) if and only if \(\phi\) is an isomorphism.
\end{fact}

\begin{definition}[ramification index]\index{ramification index}
  Suppose \(P \in C_1, Q \in C_2\) are such that \(\phi(P) = Q\). Let \(t \in K(C_2)\) be an uniformiser at \(Q\). The \emph{ramification index} of \(\phi\) at \(P\) is
  \[
    e_\phi(P) = \ord_P(\phi^*t).
  \]
\end{definition}

It is independent of the choice of uniformiser and is always greater than \(0\).
 
\begin{theorem}
  Let \(\phi: C_1 \to C_2\) be a nonconstant morphism of smooth projective curves. Then
  \[
    \sum_{P \in \phi^{-1}(Q)} e_\phi(P) = \deg \phi
  \]
  for all \(Q \in C_2\).

  Moreover, if \(\phi\) is separable then \(e_\phi(P) = 1\) for all but finitely many \(P \in C_1\).
\end{theorem}

In particular,
\begin{enumerate}
\item \(\phi\) is surjective (note that we are working over algebraically closed fields).
\item \(\# \phi^{-1}(Q) \leq \deg \phi\) with equality for all but finitely many \(Q \in C_2\).
\end{enumerate}

\begin{remark}
  Let \(C\) be an algebraic curve. A rational map is given by
  \begin{align*}
    \phi: C &\rational \P^n \\
    P &\mapsto (f_0(P) : f_1(P) : \cdots : f_n(P))
  \end{align*}
  where \(f_0, \dots, f_n \in K(C)\) not all zero.
\end{remark}

\begin{fact}
  If \(C\) is smooth then \(\phi: C \rational \P^n\) is a morphism.
\end{fact}

\section{Weierstrass equations}

We assume \(K\) is a perfect field with algebraic closure \(\overline K\) in this chapter.

\begin{definition}[elliptic curve]\index{elliptic curve}
  An \emph{elliptic curve} \(E\) over \(K\) is a smooth projective curve of genus \(1\) defined over \(K\) with a specified \(K\)-rational point \(0_E\).
\end{definition}

\begin{eg}
  \(\{X^3 + pY^3 + p^2Z^3 = 0\} \subseteq \P^2\) is smooth but is \emph{not} an elliptic curve over \(\Q\) since it has no \(\Q\)-rational pionts.
\end{eg}

\begin{theorem}
  Every elliptic curve \(E\) is isomorphic over \(K\) to a curve in Weierstrass form via an isomorphism taking \(0_E\) to \((0 : 1 : 0)\).
\end{theorem}

\begin{remark}
  Proposition 2.7 treated the special case \(E\) is a smooth plane cubic and \(0_E\) is a point of inflection.
\end{remark}

\begin{fact}
  If \(D \in \Div(E)\) is defined over \(K\) (i.e. it is fixed by \(\operatorname{Gal}(\overline K/K)\)) then \(\mathcal L(D)\) has a basis in \(K(E)\) (not just \(\overline K(E)\).
\end{fact}

\begin{proof}
  We have \(\mathcal L(2 \cdot 0_E) \subseteq \mathcal L(3 \cdot 0_E)\) with dimension \(2\) and \(3\) respectively. Pick basis \(1, x\) for \(\mathcal L(2 \cdot 0_E)\) and \(1, x, y \in \mathcal L(3 \cdot 0_E)\). Note that this implies \(\ord_{0_E}(x) = 2, \ord_{0_E}(y) = 3\). The seven elements \(1, x, y, x^2, xy, x^3, y^2\) in the \(6\)-dim vector space \(\mathcal L(6 \cdot 0_E)\) must satisfy a dependence relation. Leaving out \(x^3\) or \(y^2\) gives a basis for \(\mathcal L(6 \cdot 0_E)\) since each term has a different order of pole at \(0_E\), so coefficients of \(x^3\) and \(y^2\) are nonzero. Rescaling \(x\) and \(y\), we get
  \[
    y^2 + a_1 xy + a_3 y = x^3 + a_2 x^2 + a_4 x + a_6.
  \]
  By the fact above, we can take \(a_i \in K\).

  Let \(E'\) be the projective closure of the curve defined by Weierstrass form. There is a morphism
  \begin{align*}
    \phi: E &\to E' \\
    p &\mapsto (x(P) : y(P) : 1)
  \end{align*}
  Left to show \(\phi\) is an isomorphism, i.e.\ \(\deg \phi = 1\). We have
  \begin{align*}
    [K(E) : K(x)] &= \deg (x: E \to \P^1) = \ord_{0_E}(\frac{1}{x}) = 2 \\
    [K(E) : K(y)] &= \deg (y: E \to \P^1) = \ord_{0_E}(\frac{1}{y}) = 3
  \end{align*}
  So by tower law
  \[
    [K(E): K(x, y)] = 1.
  \]
  As \(K(x, y) = \phi^* K(E')\) so \(\deg \phi = 1\) so \(\sigma\) is birational. If \(E'\) is singular then (? genus 0) \(E\) and \(E'\) are both rational. So \(E'\) is nonsingular and \(\phi^{-1}\) is a morphism.

  To find the image of \(0_E\), we cannot simply plug \(0_E\) in as \(x, y\) both have poles at infinity. Instead, we multiply through to get
  \begin{align*}
    \phi: E &\to E' \\
    P &\mapsto (\frac{x}{y} (P) : 1 : \frac{1}{y} (P))
  \end{align*}
  so \(\phi(0_E) = (0 : 1 : 0)\).
\end{proof}

\begin{proposition}
  Let \(E\) and \(E'\) be elliptic curves over \(K\) in Weierstrass form. Then \(E \cong E'\) over \(K\) if and only if the equations are related by a change of variables
  \begin{align*}
    x &= u^2 x' + r \\
    y &= u^3 y' + u^2 s x' + t
  \end{align*}
  where \(u, r, s, t \in K, u \neq 0\).
\end{proposition}

\begin{proof}
  We check the process of putting a single elliptic curve in Weierstrass form and see what choices we can make. Suppose
  \begin{align*}
    \langle 1, x \rangle &= \mathcal L(2 \cdot 0_E) = \langle 1, x' \rangle \\
    \langle 1, x, y \rangle &= \mathcal L(3 \cdot 0_E) = \langle 1, x', y' \rangle
  \end{align*}
  so
  \begin{align*}
    x &= \lambda x' + r \\
    y &= \mu y' + \sigma x' + t
  \end{align*}
  where \(\lambda, r, \mu, \sigma, t \in K, \lambda, \mu \neq 0\). Looking at coefficients of \(x^3\) and \(y^2\), must have \(\lambda^3 = \mu^2\) so \((\lambda, \mu) = (u^2, u^3)\) for some \(u \in K^*\). Finally put \(s = \sigma/u^2\).
\end{proof}

A Weierstrass equation defines an elliptic curve if and only if it defines a smooth curve, if and only if \(\Delta(a_1, \dots a_6) \neq 0\) where \(\Delta \in \Z[a_1, \dots, a_6]\) is a certain polynomial. Details can be found out in the lecture handout.

If \(\ch K \neq 2, 3\) then we can reduce the curve to \(E: y^2 = x^3 + ax + b\) with discriminant \(\Delta = -16(4a^3 + 27b^2)\).

\begin{corollary}
  Assume \(\ch k \neq 2, 3\). Elliptic curves
  \begin{align*}
    E: y^2 &= x^3 + ax + b \\
    E': y^2 &= x^3 + a'x + b'
  \end{align*}
  are isomorphic over \(K\) if and only if
  \begin{align*}
    a' &= u^4a \\
    b' &= u^6b
  \end{align*}
  for some \(u \in K^*\).
\end{corollary}

\begin{proof}
  \(E\) and \(E'\) are related as in proposition 3.2 with \(r = s = t = 0\).
\end{proof}

\begin{definition}[\(j\)-invariant]\index{\(j\)-invariant}
  The \emph{\(j\)-invariant} of an elliptic curve \(E\) is
  \[
    j(E) = \frac{1728 (4a^3)}{4a^3 + 27b^2}.
  \]
\end{definition}

This is just the ratio \((a^3 : b^2)\) up to a Möbius transform.

\begin{corollary}
  If \(E \cong E'\) then \(j(E) = j(E')\) and the converse holds if \(K = \overline K\).
\end{corollary}

\begin{proof}
  \(E \cong E'\) if and only if \(a' = u^4 a, b' = u^6 b\) for some \(u \in K^*\), which implies that \((a^3: b^2) = ((a')^3 : (b')^2)\), which holds if and only if \(j(E) = j(E')\). If \(K = \overline K\) then we can extract roots and the converse of the second implication holds.
\end{proof}

\section{The group law}

Let \(E \subseteq \P^2\) be a smooth plane cubic and \(0_E \in E(K)\). \(E\) meets each line in 3 points, counted with multiplicity. Given \(P, Q \in E\), let \(S\) be the third point of intersection of \(PQ\) and \(E\). Let \(R\) be the third point of intersection of \(0_ES\) and \(E\). We define
\[
  P \oplus Q = R.
\]
If \(P = Q\) then take the tangent at \(P\) instead of \(PQ\). This is the ``chord and tangent process''.

\begin{theorem}
  \((E, \oplus)\) is an abelian group.
\end{theorem}

Here we recall a convention: if we don't specify the field extension the we mean the algebraic claosure. In notation: \(E = E(\overline K)\).

\begin{proof}\leavevmode
  \begin{enumerate}
  \item \(P \oplus Q = Q \oplus P\).
  \item \(0_E\) is the identity.
  \item For inverse, let \(S\) be the point of intersection of \(T_{0_E}E\) and \(E\), \(Q\) the third point of intersection of \(PS\) and \(E\). Then \(P \oplus Q = 0_E\).
  \item Associativity is much harder, and we'll prove it using divisors.
  \end{enumerate}
\end{proof}

\begin{definition}[linearly equivalent divisor]\index{divisor!linearly equivalent}
  \(D_1, D_2 \in \Div(E)\) are \emph{linearly equivalent}, written \(D_1 \sim D_2\), if exists \(f \in \overline K(E)^*\) such that \(\div (f) = D_1 - D_2\).
\end{definition}

This is an equivalence relation and we define

\begin{definition}[Picard group]\index{Picard group}
  The \emph{Picard group} is defined to be
  \[
    \Pic(E) = \Div (E)/\sim.
  \]
\end{definition}

\begin{definition}
  We let
  \[
    \Div^0(E) = \ker (\deg: \Div(E) \to \Z)
  \]
  and
  \[
    \Pic^0(E) = \Div^0(E)/\sim.
  \]
\end{definition}

\begin{proposition}
  Let
  \begin{align*}
    \phi: E &\to \Pic^0(E) \\
    P &\mapsto [P - 0_E]
  \end{align*}
  then
  \begin{enumerate}
  \item \(\phi(P \oplus Q) = \phi(P) + \phi(Q)\).
  \item \(\phi\) is a bijection.
  \end{enumerate}
\end{proposition}

\begin{proof}\leavevmode
  \begin{enumerate}
  \item Let \(\ell\) be the line \(PQ\) and \(m\) the curve \(0_ES\). Then
    \[
      \div (\frac{\ell}{m})
      = (P) + (S) + (Q) - (R) - (S) - (0_E)
      = (P) + (Q) - (P \oplus Q) - (0_E)
    \]
    so \((P) + (Q) \sim (P \oplus Q) + (0_E)\) and so
    \[
      (P) - (0_E) + (Q) - (0_E) = (P \oplus Q) - (0_E)
    \]
    so \(\phi(P \oplus Q) = \phi(P) + \phi(Q)\).
  \item For injectivity, suppose \(\phi(P) = \phi(Q)\) for \(P \neq Q\). Then exists \(f \in \overline K(E)^*\) such that \(\div (f) = P - Q\). Then
    \[
      \deg (f: E \to \P^1) = \ord_P(f) = 1
    \]
    so \(E \cong \P^1\), absurd.

    For surjectivity, let \([D] \in \Pic^0(E)\). Then \(D + (0_E)\) has degree \(1\). Riemann-Roch tells us that \(\mathcal L(D + (0_E)) = 1\) so exists \(f \in \overline K(E)^*\) such that
    \[
      \div(f) + D + (0_E) \geq 0
    \]
    and furthermore LHS has degree \(1\). Thus it has to be \((P)\) for some \(P \in E\). It follows that \((P) - (0_E) \sim D\).
  \end{enumerate}
\end{proof}

In a nutshell, \(\phi\) identifies \((E, \oplus)\) with \((\Pic^0(E), +)\) so \(\oplus\) is associative.

\subsection{Explicit formula for the group law}

We consider \(E\) in Weierstrass form and \(0_E\) the point at infinity.
\[
  y^2 + a_1xy + a_3 y = x^3 + a_2x^2 + a_4x + a_6
\]

\begin{remark}
  \(0_E\) is a point of inflection so now we can characterise the group law as \(P_1 \oplus P_2 \oplus P_3 = 0_E\) if and only if \(P_1, P_2, P_3\) are colinear.
\end{remark}

The inverse of \(P = (x_1, y_1)\) is the intersection of \(P0_E\), which is the vertical line, and \(E\) so is given by
\[
  \ominus P = (x_1, -(a_1x_1 + a_3) - y_1).
\]
Given \(P_1 = (x_1, y_1), P_2 = (x_2, y_2)\), want to find an expression for \(P_3 = P_1 \oplus P_2\). Let \(P_1P_2\) intersect \(E\) at \(P' = (x', y')\). Then \(P_3 = P_1 \oplus P_2 = \ominus P'\). Substitute \(y = \lambda x + \nu\) into * and looking at the coefficient of \(x^2\) gives
\[
  \lambda^2 + a_1 \lambda - a_2 = x_1 + x_2 + x'
\]
which gives
\begin{align*}
  x_3 &= \lambda^2 + a_1 \lambda - a_2 - x_1 - x_2 \\
  y_3 &= -(a_1x' + a_3) - (\lambda x' + \nu) = -(\lambda + a_1) x_3 - \nu - a_3
\end{align*}
It remains to find formula for \(\lambda\) and \(\nu\). If \(x_1 = x_2\) and \(P_1 \neq P_2\) then \(P_1 \oplus P_2 = 0_E\). For the general case \(x_1 \neq x_2\), have
\begin{align*}
  \lambda &= \frac{y_2 - y_1}{x_2 - x_1} \\
  \nu &= y_1 - \lambda x_1 = \frac{x_2y_1 - x_1 y_2}{x_2 - x_1}
\end{align*}
Finally the case \(P_1 = P_2\) is left as an exercise.

\begin{corollary}
  \(E(K)\) is an abelian group.
\end{corollary}

\begin{proof}
  It is a subgroup of \(E\):
  \begin{itemize}
  \item identity: \(0_E \in E(K)\) by definition,
  \item closure/inverses: see formula above.
  \item associativity/commutativity: inherited.
  \end{itemize}
\end{proof}

\begin{theorem}
  Elliptic curves are group varieties, i.e.\ \([-1]: E \to E, +: E \times E \to E\) are morphisms of algebraic varieties.
\end{theorem}

\begin{proof}
  The above formulae show \([-1]\) and \(+\) are rational maps. \([-1]: E \to E\) is a map from a smooth curve to a projective variety so is a morphism. Unfortunately there is no such result for surfaces. Instead, the formulae also show \(+\) is regular on
  \[
    U = \{(P, Q) \in E \times E: P, Q, P + Q, P - Q \neq 0_E\}.
  \]
  For \(P \in E\), let \(\tau_P: E \to E, X \mapsto P + X\) be translation by \(P\). \(\tau_P\) is a rational map so a morphism. We factor \(+\) as
  \[
    \begin{tikzcd}
      E \times E \ar[r, "\tau_{-A} \times \tau_{-B}"] & E \times E \ar[r, "+"] & E \ar[r, "\tau_{A + B}"] & E
    \end{tikzcd}
  \]
  so \(+\) is regular on \((\tau_A, \tau_B)(U)\) for all \(A, B \in E\) so \(+\) is regular on \(E \times E\).
\end{proof}

\begin{definition}[torsion subgroup]\index{torsion subgroup}
  For \(n \in \Z\), let \([n]: E \to E\) be the ``\(n\) times'' map. The \emph{\(n\)-torsion subgroup} of \(E\) is \(E[n] = \ker([n]: E \to E)\).
\end{definition}

\begin{lemma}
  Assume \(\ch k \neq 2\) and \(E: y^2 = f(x) = (x - e_1)(x - e_2)(x - e_3)\) where \(e_i \in \overline K\) distinct. Then
  \[
    E[2] = \{0_E, (e_1, 0), (e_2, 0), (e_3, 0)\} \cong (\Z/2\Z)^2.
  \]
\end{lemma}

\begin{proof}
  Let \(P = (x, y) \in E\). Then \([2] P = 0\) if and only if \(P = - P\) so \((x, y) = (x, -y)\) so \(y = 0\).
\end{proof}

\paragraph{Elliptic curves over \(C\)}

Let \(\Lambda = \{a \omega_1 + b \omega_2: a, b \in \Z\}\) be a lattice, where \(\omega_1, \omega_2\) is a basis for \(\C\) as an \(\R\)-vector space. The the set of meromorphic functions on the Riemann surface \(\C/\Lambda\) is the same as \(\Lambda\)-invariant meromorphisc functions on \(\C\). This field is generated by \(\wp(z)\) and \(\wp'(z)\) where
\[
  \wp(z) = \frac{1}{z^2} + \sum_{\lambda \in \Lambda \setminus \{0\}} \left(\frac{1}{(z - \lambda)^2} - \frac{1}{\lambda^2}\right)
\]
They satisfy
\[
  \wp'(z)^2 = 4 \wp(z)^3 - g_2 \wp(z) - g_3
\]
for some \(g_2, g_3 \in \Lambda\) depending on \(\Lambda\). One shows \(\C/\Lambda \cong E(\C)\) where \(E\) is the elliptic curve
\[
  y_2 = 4x^3 - g_2x - g_3.
\]
The isomorphism is understood as isomorphism of Riemann surfaces and isomorphism of groups.

\begin{theorem}
  Every elliptic curve over \(\C\) arises this way.
\end{theorem}

For elliptic curve \(E/\C\) we have
\begin{enumerate}
\item \(E[n] \cong (\Z/n\Z)^2\).
\item \(\deg [n] = n^2\).
\end{enumerate}
We'll show 2 holds for any field \(K\), and 1 holds if \(\ch k \ndivides n\).

Statement of results
\begin{enumerate}
\item If \(K = \C\) then \(E(\C) \cong \C/\Lambda \cong \R/\Z \cong \R/\Z\).
\item If \(K = \R\) then \(E(\R) \cong
  \begin{cases}
    \Z/2\Z \times \R/\Z & \Delta > 0 \\
    \R/\Z & \Delta < 0
  \end{cases}
  \)
\item If \(K = \F_q\) then \(|E(\F_q) - (q + 1)| \leq 2 \sqrt q\). This is Hasse's theorem.
\item If \([K: \Q_p] < \infty\) with rings of integers \(\mathcal O_K\) then \(E(K)\) has a subgroup of finite index isomorphic to \((\mathcal O_K, +)\).
\item If \([K: \Q] < \infty\) then \(E(K)\) is a finitely generated abelian group. This is Mordell-Weil theorem.
\end{enumerate}

\begin{remark}
  The isomorphisms in 1, 2 and 4 resepcted the relevant topologies.
\end{remark}

\section{Isogenies}

Let \(K\) be any perfect field in this chapter.

Let \(E_1, E_2\) be elliptic curves.

\begin{definition}[isogeny]\index{isogeny}
  An \emph{isogeny} \(\phi: E_1 \to E_2\) is a nonconstant morphism with \(\phi(0_{E_1}) = 0_{E_2}\). We say \(E_1\) and \(E_2\) are \emph{isogenous} if there exists an isogeny from \(E_1\) to \(E_2\).

  We define \(\Hom(E_1, E_2)\) to the be set of all isogenies \(E_1 \to E_2\) plus \(0\). This is a group under
  \[
    (\phi + \psi)(P) = \phi(P) + \psi(P).
  \]
\end{definition}
Note that nonconstant implies that surjectivity on \(\overline K\)-points. The composition of isogenies is an isogeny.

\begin{lemma}
  If \(0 \neq n \in \Z\) then \([n]: E \to E\) is an isogeny.
\end{lemma}

\begin{proof}
  We have checked that \([n]\) is a morphism. We must show \([n] \neq 0\). There is a trick that we can use, if we assume \(\ch K \neq 2\). If \(n = 2\) then we computed last time that \(\E[2]\) has 4 points so \([2] \neq 0\). If \(n\) is odd then let \(T \in E[2]\) be nonzero then \(nT = T \neq 0\) so again \([n] \neq 0\). Now use \([mn] = [m] \compose [n]\).

  If \(\ch K = 2\), we can compute \(E[3]\) as in the lemma before.
\end{proof}

\begin{corollary}
  \(\Hom(E_1, E_2)\) is torsion-free as a \(\Z\)-module.
\end{corollary}

\begin{lemma}
  Let \(\phi: E_1 \to E_2\) be an isogeny. Then \(\phi(P + Q) = \phi(P) + \phi(Q)\) for all \(P, Q \in E\).
\end{lemma}

\begin{proof}[Sketch proof]
  \(\phi\) induces a map
  \begin{align*}
    \phi_*: \Div^0(E_1) &\to \Div^0(E_2) \\
    \sum n_P P &\mapsto \sum n_P \phi(P)
  \end{align*}
  Recall we have a field extension \(\phi^*: K(E_2) \to K(E_1)\) so there is a norm map \(N_{K(E_1)/K(E_2)}: K(E_1) \to K(E_2)\). It is a fact that if \(f \in K(E_1)^*\) then
  \[
    \div (N_{K(E_1)/K(E_2)} f) = \phi_*(\div f)
  \]
  so \(\phi_*\) takes principal divisors to principal divisors. Since \(\phi(0_{E_1}) = 0_{E_2}\), we have a commutative diagram
  \[
    \begin{tikzcd}
      E_1 \ar[r, "\phi"] \ar[d, "\cong"] & E_2 \ar[d, "\cong"] \\
      \Pic^0(E_1) \ar[r, "\phi_*"] & \Pic^0(E_2)
    \end{tikzcd}
  \]
  As \(\phi_*\) is a group homomorphism, so is \(\phi\).
\end{proof}

\begin{eg}
  Let \(E/K\) be an elliptic curve. Suppose \(\ch K \neq 2\) and exists \(0 \neq T \in E(K)[2]\). wlog assume \(E: y^2 = x(x^2 + ax + b)\) with \(a, b \in K, b(a^2 - 4b) \neq 0\) so \(T = (0, 0)\). If \(P = (x, y)\) and \(P' = P + T = (x', y')\) then
  \begin{align*}
    x' &= \left( \frac{y}{x} \right)^2 - a - x = \frac{b}{x} \\
    y' &= - \left( \frac{y}{x} \right) x' = \frac{-by}{x^2}
  \end{align*}
  We define two variables that remain unchanged under (?) swapping
  \begin{align*}
    \xi &= x + x' + a = \left( \frac{y}{x} \right)^2 \\
    \eta &= y + y' = \frac{y}{x} (x - \frac{b}{x})
  \end{align*}
  Then
  \begin{align*}
    \eta^2 &= \left( \frac{y}{x} \right)^2 ((x + \frac{b}{x})^2 - 4b) \\
           &= \zeta ((\zeta - a)^2 - 4b) \\
           &= \zeta (\zeta^2 - 2a\zeta + a^2 - 4b)
  \end{align*}
  Let \(E': y^2 = (x^2 + a'x + b')\) where \(a' = -2a, b' = a^2 - 4b\). Then there is an isogeny
  \begin{align*}
    \phi: E &\to E' \subseteq \P^2 \\
    (x, y) &\mapsto (\xi : \eta : 1)
  \end{align*}
  Left to show \(\phi(0_E) = 0_{E'}\). The three coordinates has a pole of order \(-2, -3, 0\) respectively at \(0_E\) so multiply by uniformiser to the power of three we get \((0:1:0)\).
\end{eg}

\begin{lemma}
  Let \(\phi: E_1 \to E_2\) be an isogeny. Then exists morphism \(\xi\) making the following diagram commute
  \[
    \begin{tikzcd}
      E_1 \ar[r, "\phi"] \ar[d, "x_1"] & E_2 \ar[d, "x_2"] \\
      \P^1 \ar[r, "\xi"] & \P^1
    \end{tikzcd}
  \]
  where \(x_i\) is the \(x\) coordinate on a Weierstrass equation for \(E_i\). Moreover if \(\xi(t) = \frac{r(t)}{s(t)}\) where \(r, s \in K[t]\) coprime then
  \[
    \deg \phi = \deg \xi = \max (\deg (r), \deg (s)).
  \]
\end{lemma}

\begin{eg}
  In the example above we just have \(\xi = \frac{x^2 + ax + b}{x}\) so in particular it has degree \(2\).
\end{eg}

\begin{proof}
  For \(i = 1, 2\), \(K(E_i)/K(x_i)\) is a degree \(2\) Galois extension with Galois group generated by \([-1]^*\).

  \[
    \begin{tikzcd}
      & K(E_1) \ar[dd] \\
      K(x_1) \ar[ur] \ar[dd, dashed] \\
      & K(E_2) \\
      K(x_2) \ar[ur]
    \end{tikzcd}
  \]
  If \(f \in K(x_2)\) then \([-1]^* f = f\) so
  \[
    [-1]^*(\phi^* f) = \phi^*([-1]^* f) = \phi^* f
  \]
  so indeed \(\phi^*f \in K(x_1)\). Taking \(f = x_2\) gives \(\phi^*x_2 = \xi(x_1)\) for some rational function \(\xi\). By tower law \(\deg \phi = \deg \xi\). Now \(K(x_2) \embed K(x_1), x_2 \mapsto \xi(x_1) = \frac{r(x_1)}{s(x_1)}\) for some \(r, s \in K[t]\) coprime. Claim the minimal polynomial of \(x_1\) over \(K(x_2)\) is
  \[
    f(t) = r(t) - s(t)x_2 \in K(x_2)[t].
  \]
  Check \(f(x_1) = 0\). \(f\) is irreducible in \(k[x_2, t]\) (since \(r, s\) are corpime) so by Gauss' lemma \(f\) is irreducible in \(K(x_2)[t]\). Therefore
  \[
    \deg \phi = \deg \xi = [K(x_1): K(x_2)] = \deg(f) = \max(\deg(r), \deg(s)).
  \]
\end{proof}

The lemma shows that the example \(\phi\) above has degree 2. We say \(\phi\) is a \emph{\(2\)-isogeny}.

\begin{lemma}
  \(\deg [2] = 4\).
\end{lemma}

\begin{proof}
  Assume \(\ch K \neq 2, 3\) so write \(E: y^2 = f(x) = x^3 + ax + b\). If \(P = (x, y)\) then
  \[
    x(2P)
    = \left( \frac{2x^2 + a}{2y} \right)^2 - 2x
    = \frac{(3x^2 + a)^2 - 8x f(x)}{4 f(x)} 
    = \frac{x^4 + \cdots}{4 f(x)}
  \]
  The numerator and the denominator are coprime. Indeed otherwise exists \(\theta \in \overline K\) with \(f(\theta) = f'(\theta) = 0\), so \(f\) has a multiple root, absurd. Therefore by the lemma \(\deg [2] = max(4, 3) = 4\).
\end{proof}

We will show that \(\deg [n] = n^2\) by showing that \(\deg\) is a quadratic form. This will also be useful when we prove Hasse's theorem later.

\begin{definition}
  Let \(A\) be an abelian group. \(q: A \to \Z\) is a quadratic form if
  \begin{enumerate}
  \item \(q(nx) = n^2 q(x)\) for all \(n \in \Z, x \in A\).
  \item \((x, y) \mapsto q(x + y) - q(x) - q(y)\) is \(\Z\)-bilinear.
  \end{enumerate}
\end{definition}

\begin{lemma}
  \(q: A \to \Z\) is a quadratic form if and only if it satisfies the parallelogram law
  \[
    q(x + y) + q(x - y) = 2q(x) + 2q(y)
  \]
  for all \(x, y \in A\).
\end{lemma}

\begin{proof}
  Only if is an easy exercise. If will be on example sheet 2.
\end{proof}

\begin{theorem}
  \(\deg: \Hom(E_1, E_2) \to \Z\) is a quadratic form.
\end{theorem}
Here by convention the 0 map has degree \(0\).

For the proof we assume \(\ch K \neq 2, 3\) and write \(E_2: y^2 = f(x) = x^3 + ax + b\). Let \(P, Q \in E_2\) with \(P, Q, P + Q, P - Q \neq 0\). Let \(x_1, \dots, x_4\) be the \(x\) coordinates of these four points.

\begin{lemma}
  There exist \(W_0, W_1, W_2 \in \Z[a, b][x_1, x_2]\) of degree \(\leq 2\) in \(x_1\) and of degree \(\leq 2\) in \(x_2\) such that
  \[
    (1: x_3 + x_4: x_3x_4) = (W_0: W_1: W_2).
  \]
\end{lemma}

\begin{proof}
  Method 1 is to calculate directly and get \(W_0 = (x_1 - x_2)^2, \dots\). See formula sheet.

  Method 2: let \(y = \lambda x + \nu\) be the line through \(P\) and \(Q\) so
  \[
    f(x) - (\lambda x + \nu)^2 = (x - x_1)(x - x_2)(x - x_3).
  \]
  By comparing coefficients we get
  \begin{align*}
    \lambda^2 &= s_1 \\
    -2 \lambda \nu &= s_2 - a \\
    \nu^2 &= s_3 + b
  \end{align*}
  where \(s_i\) is the \(i\)th elementary symmetric polynomial in \(x_1, x_2, x_3\). Eliminating \(\lambda\) and \(\mu\) gives
  \[
    \underbrace{(s_2 - a)^2 - 4s_1 (s_3 + b)}_{F(x_1, x_2, x_3)} = 0
  \]
  where \(F\) has degree \(\leq 2\) in each \(x_i\). \(x_3\) is a root of the quadratic \(W(t) = F(x_1, x_2, t)\). Repeating for line through \(P\) and \(-Q\) shows \(x_4\) is also a root of \(W(t)\). Write \(W(t) = W_0t^2 - W_1t + W_2\) and then
  \[
    (1: x_3 + x_4: x_3x_4) = (W_0: W_1: W_2).
  \]
\end{proof}

We show that if \(\phi, \psi \in \Hom(E_1, E_2)\) then
\[
  \deg (\phi + \psi) + \deg (\phi - \psi) \leq 2 \deg(\phi) + 2 \deg(\psi).
\]
We may assume \(\phi, \psi, \phi + \psi, \phi - \psi \neq 0\) as the other cases are trivial or we may use \(\deg [2] = 4\). Let the \(x\) coordinate of \(\phi(x, y), \psi(x, y), (\phi + \psi)(x, y), (\phi - \psi)(x, y)\) be \(\xi_1(x), \dots, \xi_4(x)\) respectively. Put \(\xi_i = \frac{r_i}{s_i}\) where \(r_i, s_i \in K[x]\) coprime and use the above lemma, we get
\[
  (s_3s_4: r_3s_4 + r_4s_3: r_3r_4) = ((r_1s_2 - r_2s_1)^2: \cdots ).
\]
Note that the three coordinates on LHS are coprime. We have
\begin{align*}
  &\deg (\phi + \psi) + \deg (\phi - \psi) \\
  &= \max(\deg (r_3), \deg (s_3)) + \max (\deg (r_4), \deg (s_4)) \\
  &= \max(\deg (s_3s_4), \deg (r_3s_4 + r_4s_3), \deg (r_3r_4)) \quad \text{case checking} \\
  &\leq 2 \max(\deg(r_1), \deg(s_1)) + 2 \max(\deg (r_2), \deg (s_2)) \quad \text{as terms on LHS are coprime} \\
  &= 2 \deg (\phi) + 2 \deg (\psi)
\end{align*}

Now replace \(\phi, \psi\) by \(\phi + \psi\) and \(\phi - \psi\) to get
\[
  \deg (2\phi) + \deg (2\psi) \leq 2 \deg (\phi + \psi) + 2 \deg(\phi - \psi)
\]
Since \(\deg [2] = 4\) we get
\[
  2 \deg (\phi) + 2 \deg (\psi) \leq \deg (\phi + \psi) + \deg (\phi - \psi)
\]
Together they show \(\deg\) satisfies the parallelogram law, so \(\deg\) is a quadratic form.

\begin{corollary}
  \(\deg (n\phi) = n^2 \deg (\phi)\) for all \(n \in \Z, \phi \in \Hom(E_1, E_2)\). In particular \(\deg [n] = n^2\).
\end{corollary}

\section{Invariant differential}

We want to find out when a morphism is separable so we may apply Riemann-Hurwitz. To do so we use differentials.

Let \(C\) be an algebraic curve over \(K = \overline K\). The space of differentials \(\Omega_C\) is the \(K(C)\)-vector spaces generated by \(df\) for \(f \in K(C)\) subject to the relations
\begin{enumerate}
\item \(d(f + g) = df + dg\),
\item \(d(fg) = f dg + g df\),
\item \(da = 0\) for all \(a \in K\).
\end{enumerate}

\begin{fact}
  \(\Omega_C\) is a \(1\)-dimensional \(K(C)\)-vector space.
\end{fact}

Let \(0 \neq \omega \in \Omega_C\). Let \(P \in C\) be a smooth point with uniformiser \(t \in K(C)\). It is a fact that \(dt \neq 0\) so we may write \(\omega = f dt\) for some \(f \in K(C)^*\). We define \(\ord_p(\omega) = \ord_p(f)\). This is independent of choice of \(t\).

\begin{fact}
  Suppose \(f \in K(C)^*\) and \(\ord_P(f) = n \neq 0\). If \(\ch K \ndivides n\) then \(\ord_P(df) = n - 1\).
\end{fact}

We now assume \(C\) is a smooth projective curve.

\begin{fact}
  \(\ord_p(\omega) = 0\) for all but finitely many \(P \in C\).
\end{fact}

\begin{definition}
  We define \(\div(\omega) = \sum_{P \in C} \ord_P(\omega) P \in \Div(C)\).
\end{definition}

\begin{definition}
  We define the genus of \(C\) to be
  \[
    g(C) = \dim_K \{\omega \in \Omega_C: \div(\omega) \geq 0\},
  \]
  the dimension of the space of \emph{regular differentials}.
\end{definition}

As a consequence of Riemann-Roch, we have if \(0 \neq \omega \in \Omega_C\) then \(\deg(\div(\omega)) = 2 g(C) - 2\).

\begin{lemma}
  Assume \(\ch k \neq 2\) and \(E: y^2 = (x - e_1)(x - e_2)(x - e_3)\). Then \(\omega = \frac{dx}{y}\) is a differential on \(E\) with no zeros or poles. In particular \(g(E) = 1\) and the \(K\)-vector space of regular differentials on \(E\) is \(1\)-dimensional, spanned by \(\omega\).
\end{lemma}

\begin{proof}
  Let \(T_i = (e_i, 0)\) and we know \(E[2] = \{0, T_1, T_2, T_3\}\). We have
  \[
    \div (y) = (T_1) + (T_2) + (T_3) - 3(0_E)
  \]
  \(T_i\) appears with multiplicity \(1\) in \(\div y\) since we know \(\deg \div y = 0\). If \(P \in E \setminus \{0\}\) then
  \[
    \div (x - x_P) = (P) + (-P) - 2(0_E).
  \]
  If \(P \in E \setminus E[2]\) then \(\ord_P(x - x_P) = 1\) so \(\ord_P(dx) = 0\). If \(P = T_i\) then \(\ord_P(x - x_P) = 2\) so \(\ord_P(dx) = 1\). Finally if \(P = 0_E\) then \(\ord_P(x) = -2\) so \(\ord_P(dx) = -3\). Therefore
  \[
    \div(dx) = (T_1) + (T_2) + (T_3) - 3(0_E).
  \]
  It follows that \(\div (\frac{dx}{y}) = 0\).
\end{proof}

\begin{definition}
  If \(\phi: C_1 \to C_2\) is a nonconstant morphism then we have \emph{pullback of differentials} defined by
  \begin{align*}
    \phi^*: \Omega_{C_2} &\to \Omega_{C_1} \\
    f dg &\mapsto (\phi^* f) d(\phi^*g)
  \end{align*}
\end{definition}

\begin{lemma}
  Let \(P \in E\) and \(\tau_P: E \to E, X \mapsto P + X\). If \(\omega = \frac{dx}{y}\) then \(\tau_P^* \omega = \omega\). \(\omega\) is called the \emph{invariant differential}\index{invariant differential}.
\end{lemma}

\begin{proof}
  \(\tau_p^*\omega\) is again a regular differential on \(E\) so \(\tau_P^* \omega = \lambda_P \omega\) for some \(\lambda_P \in K^*\). The map \(E \to \P^1, P \mapsto \lambda_P\) (after a calculation we know the map is rational) is a morphism of smooth projective curve but \emph{not} surjective, as it misses \(0, \infty\). Therefore it is constant. Thus exists \(\lambda \in K^*\) such that \(\tau_P^* \omega = \lambda \omega\) for all \(P \in E\). Taking \(P = 0_E\) shows \(\lambda = 1\).
\end{proof}

\begin{remark}
  If \(K = \C\) then remember we have an isomorphism \(\C/\Lambda \cong E(\C), z \mapsto (\wp(z), \wp'(z))\) so
  \[
    \frac{dx}{y} = \frac{\wp'(z) dz}{\wp'(z)} = dz,
  \]
  which is manifestly invariant under \(z \mapsto z + \text{ constant}\).
\end{remark}

\begin{lemma}
  Let \(\phi, \psi \in \Hom(E_1, E_2)\) and \(\omega\) the invariant differential on \(E_2\). Then \((\phi + \psi)^* \omega = \phi^*\omega + \psi^*\omega\).
\end{lemma}

\begin{proof}
  Write \(E = E_2\). We have three maps
  \begin{align*}
    E \times E &\to E \\
    \mu: (P, Q) &\mapsto P + Q \\
    \pi_1: (P, Q) &\mapsto P \\
    \pi_2: (P, Q) &\mapsto Q
  \end{align*}
  As \(E \times E\) is \(2\)-dimensional, it is a fact that \(\Omega_{E \times E}\) is a \(2\)-dimensional \(K(E \times E)\)-vector space with basis \(\pi_1^* \omega, \pi_2^* \omega\). Then \(\mu^* \omega = f \pi_1^* \omega + g \pi_2^* \omega\) for some \(f, g \in K(E \times E)\). For \(Q \in E\) let \(\iota_Q: E \to E \times E, P \mapsto (P, Q)\). Applying \(\iota_Q^*\) gives
  \[
    (\mu \iota_Q)^* \omega = (\iota_Q^* f) (\pi_1 \iota_Q)^* \omega + (\iota_Q^* g) (\pi_2 \iota_Q)^* \omega,
  \]
  i.e.
  \[
    \tau_Q^* \omega = (\iota_Q^* f) \omega + 0
  \]
  so \(\iota_Q^*f = 1\) for all \(Q \in E\), so \(f(P, Q) = 1\) for all \(P, Q \in E\). Similarly \(g(P, Q) = 1\). Thus \(\mu^*\omega = \pi_1^* \omega + \pi_2^* \omega\). Now pullback by \(E \to E \times E, P \mapsto (\phi(P), \psi(P))\) to get
  \[
    (\phi + \psi)^*\omega = \phi^*\omega + \psi^*\omega.
  \]
\end{proof}

\begin{lemma}
  Let \(\phi: C_1 \to C_2\) be a nonconstant morphism. Then \(\phi\) is separable if and only if \(\phi^*: \Omega_{C_2} \to \Omega_{C_1}\) is non-zero.
\end{lemma}

\begin{proof}
  Omitted.
\end{proof}

\begin{eg}
  Consider the group variety \(\mathbb G_m = \A^1 \setminus \{0\} = \P^1 \setminus \{0, \infty\}\) with group law being multiplication. Let \(n \geq 2\) be an intger and consider \(\phi(x) = x^n\). We know from Galois theory that if \(\ch K \ndivides n\) then \(\ker \phi\) has \(n\) elements. This can also be deducted geometrically using differentials: \(\phi^* (dx) = dx^n = nx^{n - 1}dx\) so if \(\ch K \ndivides n\) then \(\phi\) is separable. Then \(\#\phi^{-1}(Q) = \deg \phi\) for all but finitely many \(Q \in \mathbb G_m\). \(\phi\) is a group homomorphism so \(\# \phi^{-1}(Q) = \ker \phi\) for all \(Q \in \mathbb G_m\) so in fact \(\# \ker \phi = \deg \phi= n\). Thus \(K\) (which is algebraically closed) contains exactly \(n\) \(n\)th roots of unity.
\end{eg}

\begin{theorem}
  If \(\ch K \ndivides n\) then \(E[n] \cong (\Z/n\Z)^2\).
\end{theorem}

\begin{proof}
  By induction \([n]^*\omega = n\omega\) so if \(\ch K \ndivides n\) then \([n]: E \to E\) is separable. Thus by the theorem \(\# [n]^{-1}(Q) = \deg [n]\) for all but finitely many \(Q \in E\). But \([n]\) is a group homomorphism so \(\#[n]^{-1}(Q) = \# E[n]\) for all \(Q \in E\). Thus
  \[
    \# E[n] = \deg [n] = n^2.
  \]
  By classification of finitely generated abelian groups,
  \[
    E[n] \cong \Z/d_1\Z \times \Z/d_2\Z \times \cdots \times \Z/d_t\Z
  \]
  with \(d_1 \divides d_2 \divides \cdots \divides d_t \divides n\) and \(\prod d_i = n^2\). If \(p\) is a prime with \(p \divides d_1\) then \(E[p] \cong (\Z/p\Z)^t\). But \(\#E[p] = p^2\) so \(t = 2\) and \(d_1 \divides d_2 \divides n\), \(d_1d_2 = n^2\) so \(d_1 = d_2 = n\).
\end{proof}

\begin{remark}
  If \(\ch K = p\) then \([p]\) is inseparable. It can be shown that either \(E[p^r] \cong \Z/p^r\Z\) for all \(r \geq 1\), or \(E[p^r] = 0\) for all \(r \geq 1\). They are called ordinary and supersingular.
\end{remark}

\section{Elliptic curves over finite fields}

We begin by proving a form of Cauchy-Schwarz.

\begin{lemma}
  Let \(A\) be an abelian group and \(q: A \to \Z\) a positive definite quadratic form. If \(x, y \in A\) then
  \[
    |q(x + y) - q(x) - q(y)| \leq 2 \sqrt{q(x) q(y)}.
  \]
\end{lemma}

\begin{notation}
  \(\langle x, y \rangle = q(x + y) - q(x) - q(y)\) and note that \(\langle x, x \rangle = 2q(x)\).
\end{notation}

\begin{proof}
  We may assume \(x \neq 0\) as otherwise the result is clear. Let \(m, n \in \Z\). Then
  \begin{align*}
    0 &\leq q(mx + ny) \\
      & \frac{1}{2} \langle mx + ny, mx + ny \rangle \\
      &= m^2 qx + mn \langle x, y \rangle + n62 q(y) \\
      &= q(x) (m + \frac{n \langle x, y \rangle}{2q(x)})^2 + n^2 (q(y) - \frac{\langle x, y\rangle^2}{4q(x)}
  \end{align*}
  Take \(m = \langle x, y \rangle, n = -2q(x)\) to deduce
  \[
    \langle x, y \rangle^2 \leq 4q(x)q(y).
  \]
\end{proof}

Let \(\F_q\) be the field with \(q\) elements where \(q = p^m\) for some \(p\) prime. Then \(\gal(\F_{q^r}/\F_q)\) is cyclic of order \(r\) generated by the Frobenius map \(x \mapsto x^q\).

\begin{theorem}[Hasse]\index{Hasse's theorem}
  Let \(E/\F_q\) be an elliptic curve. Then
  \[
    |\#E(\F_q) - (q + 1)| \leq 2 \sqrt q.
  \]
\end{theorem}

\begin{proof}
  Let \(E\) have Weierstrass equation with coefficients \(a_1, \dots, a_6 \in \F_q\) so \(a_i^q = a_i\) for all \(i\). Define the \emph{Frobenius endomorphism}\index{Frobenius endomorphism} \(\phi: E \to E, (x, y) \mapsto (x^q, y^q)\) which is an isogeny of degree \(q\). Then
  \[
    E(\F_q) = \{P \in E: \phi(P) = P\} = \ker (1 - \phi).
  \]
  Note \(\phi\) is not separable as
  \[
    \phi^*\omega = \phi^* (\frac{dx}{y}) = \frac{d x^q}{y^q} = \frac{qx^{q - 1} dx}{y^q} = 0
  \]
  but
  \[
    (1 - \phi)^* \omega = \omega - \phi^* \omega = \omega \ne 0
  \]
  so \(1 - \phi\) is separable. Same as before, we have \(\# \ker (1 - \phi) = \deg (1 - \phi)\).

  Recall that \(\deg: \End(E) \to \Z\) is a positive definite quadratic form so by Cauchy-Schwarz
  \[
    |\deg (1 - \phi) - \deg [1] - \deg [\phi]| \leq 2 \sqrt{\deg [1] \deg [\phi]}
  \]
  so
  \[
    |\#E(\F_q) - 1 - q| \leq 2 \sqrt q
  \]
  as required.
\end{proof}

\subsection{Zeta function}

For \(K\) a number field, define
\[
  \zeta_K(s) = \sum_{\mathfrak a \subseteq \O_K} \frac{1}{N(\mathfrak a)^s} = \prod_{\mathfrak p \subseteq \O_K \text{ prime}} \left( 1- \frac{1}{(N(\mathfrak p))^s} \right)^{-1}
\]
For \(K\) a function field, i.e.\ \(K = \F_q(C)\) where \(C/\F_q\) is a smoth projective curve, we define
\[
  \zeta_K(s) = \prod_{x \in |C|} \left(1 - \frac{1}{(Nx)^s} \right)^{-1}
\]
where \(|C|\) is the set of closed points of \(C\), and is the same as the orbits of \(\gal(\overline \F_q/\F_q)\) on \(C(\overline F_q)\). Have \(Nx = q^{\deg x}\) where \(\deg x\) is the size of the orbit.

We have \(\zeta_K(s) = F(q^{-s})\) for some \(F \in \Q[[T]]\). Explicitly
\[
  F(T) = \prod_{x \in |C|} (1 - T^{\deg x})^{-1}.
\]
Take logarithm of the formal power series, we get
\begin{align*}
  \log F(T) &= \sum_{x \in |C|} \sum_{m = 1}^\infty \frac{1}{m} T^{m \deg x} \\
  T \frac{d}{dT} \log F(T) &= \sum_{x \in |C|} \sum_{m = 1}^\infty (\deg x) T^{m \deg x} \\
            &= \sum_{n = 1}^\infty (\sum_{x \in |C|, \deg x|n} \deg x) T^n \\
            &= \sum_{n = 1}^\infty \# C(\F_{q^n}) T^n
\end{align*}
Now reverse the process,
\[
  F(T) = \exp \sum_{n = 1}^\infty \frac{\# C(\F_{q^n})}{n} T^n.
\]

We define \(\tr: \End(E) \to \Z, \phi \mapsto \langle \phi, 1 \rangle\).

\begin{lemma}
  If \(\phi \in \End(E)\) then
  \[
    \phi^2 - (\tr \phi) \phi + \deg \phi = 0.
  \]
\end{lemma}

\begin{proof}
  Example sheet 2.
\end{proof}

\begin{definition}[zeta function]\index{zeta function}
  The \emph{zeta function} of a variety \(V/\F_q\) is the formal power series (?)
  \[
  Z_V(T) = \exp \sum_{n = 1}^\infty \frac{\# V(\F_{q^n})}{n} T^n.
  \]
\end{definition}

\begin{lemma}
  Suppose \(E/\F_q\) is an elliptic curve, \(\# E(\F_q) = q + 1 - a\). Then
  \[
    Z_E(T) = \frac{1 - aT + qT^2}{(1 - T)(1 - qT)}.
  \]
\end{lemma}

\begin{proof}
  Let \(\phi: E \to E\) be the \(q\)-power Frobenius. By the proof of Hasse's theorem
  \[
    \#E(\F_q) = \deg (1 - \phi) = q + 1 - \tr \phi
  \]
  so \(a = \tr \phi\) and \(\deg \phi = q\). By the above lemma \(\phi^2 - a\phi + q = 0\) so \(\phi^{n + 2} - a \phi^{n + 1} + q \phi^n = 0\). Upon taking trace,
  \[
    \tr \phi^{n + 2} - a \tr \phi^{n + 1} + q \tr \phi^n = 0.
  \]
  This second order difference equation with initial condition \(\tr 1 = 2, \tr \phi = q\) has solution \(\tr \phi^n = \alpha^n + \beta^n\) where \(\alpha, \beta \in \C\) ar roots of \(X^2 - aX + q = 0\). Then
  \[
    \# E(\F_{q^n}) = \deg (1 - \phi^n)
    = \deg \phi^n + 1 - \tr \phi^n
    = q^n + 1 - \alpha^n - \beta^n
  \]
  Thus the zeta function is
  \[
    Z_V(T) = \exp \sum_{n = 1}^\infty \frac{1}{n} (T^n + (qT)^n - (\alpha T)^n - (\beta T)^n)
    = \frac{(1 - \alpha T)(1 - \beta T)}{(1 - T)(1 - qT)}
  \]
  using \(-\log (1 - x) = \sum_{m = 1}^\infty \frac{x^m}{m}\). Expand.
\end{proof}

\begin{remark}
  Hasse's theorem as Riemann hypothesis for finite fields: Hasse's theorem gives a bound \(|a| \leq 2 \sqrt q\) so \(\alpha = \overline \beta\). As \(\alpha\beta = q\), have \(|\alpha| = |\beta| = sqrt q\). Let \(K = \F_q(E)\). Then \(\zeta_K(s) = 0\) if and only if \(Z_E(q^{-s}) = 0\), so \(q^s = \alpha \text{ or } \beta\) so \(q^{\Re s} = \sqrt q\), i.e.\ \(\Re s = \frac{1}{2}\). Thus we have proven the Riemann hypothesis.
\end{remark}

\section{Formal groups}

\begin{definition}[\(I\)-adic topology]\index{\(I\)-adic topology}
  Let \(R\) be a ring and \(I \subseteq R\) an ideal. The \emph{\(I\)-adic topology} is the topology on \(R\) with basis \(\{r + I^n: r \in R, n \geq 1\}\)
\end{definition}

\begin{definition}
  A sequence \((x_n)\) in \(R\) is \emph{Cauchy} if for all \(k\) exists \(N\) such that for all \(m, n \geq N\), have \(x_m - x_n \in I^k\).
\end{definition}

\begin{definition}
  \(R\) is \emph{complete} if
  \begin{enumerate}
  \item \(\bigcap_{n \geq 0} I^n = \{0\}\) (Hausdorff condition),
  \item every Cauchy sequence converges.
  \end{enumerate}
\end{definition}

\begin{remark}
  Suppose \(R\) is complete. If \(x \in I\) then \(\frac{1}{1 - x} = 1 + x + x^2 + \cdots\) so \(1 - x \in R^*\).
\end{remark}

\begin{eg}\leavevmode
  \begin{enumerate}
  \item \(R = \Z_p\) with \(I = p\Z_p\). This is complete by construction.
  \item \(R = \Z[[t]]\) with \(I = (t)\).
  \end{enumerate}
\end{eg}

\begin{lemma}[Hensel's lemma]\index{Hensel's lemma}
  Let \(R\) be an integral domain and is complete with respect to the ideal \(I\). Let \(F \in R[X]\), \(s \geq 1\). Suppose \(a \in R\) satisfies \(F(a) = 0 \pmod{I^s}, F'(a) \in R^\times\). Then there exists a unique \(b \in R\) satisfying \(F(b) = 0, b = a \pmod{I^s}\).
\end{lemma}

\begin{proof}
  Let \(u \in R^\times\) with \(F'(a) = u \pmod I\). Replacing \(F\) by \(\frac{X + A}{u}\), we may assume \(a = 0\) and \(F'(0 = 1 \pmod I\). We define
  \[
    x_0 = 0, \quad x_{n + 1} = x_n - F(x_n).
  \]
  An easy induction shows \(x_n = 0 \pmod{I^s}\) for all \(n\). Also
  \[
    F(X) - F(Y) = (X - Y) (F'(0) + X G(X, Y) + Y H(X, Y))
  \]
  for some \(G, H \in R[X, Y]\). Claim that \(x_{n + 1} = x_n \pmod{I^{n + s}}\) for all \(n \geq 0\).

  \begin{proof}
    Induction on \(n\). \(n = 0\) holds. Suppose \(x_n = x_{n - 1} \pmod{I^{n + s - 1}}\). Then
    \[
      F(x_n) - F(x_{n - 1}) = (x_n - x_{n - 1})(1 + c)
    \]
    for some \(c \in I\). Modulo \(I^{n + s}\), get
    \[
      F(x_n) - F(x_{n - 1}) = x_n - x_{n - 1} \pmod{I^{n + s]}}.
    \]
    Rearrange to get
    \[
      x_{n + 1} = x_n - F(x_n) = x_{n - 1} - F(x_{n - 1}) = x_n \pmod{I^{n + s}}.
    \]
  \end{proof}
  Thus by completeness \(x_n \to b\) as \(n \to \infty\) for some \(b \in R\). Taking limit of the recurrence relation and use the continuity of \(F\) to get \(F(b) = 0\). Taking limit in \(x_n = 0 \pmod{I^s}\) gives \(b = 0 \pmod{I^s}\). Uniqueness follows from the assumption \(R\) is an integral domain.
\end{proof}

Consider \(E: Y^2Z + a_1 XYZ + a_3 yZ^2 = X^3 + a_2X^2Z + a_4XZ^2 + a_6Z^3\). We want to study the behaviour near \(0_E\) so use the affine piece \(Y \neq 0\). Let \(t = -X/Y, w = -Z/Y\). Then
\[
  w = f(t, w) = t^3 + a_1tw + a_2t^2w + a_3w^2 + a_4tw^2 + a_6w^3.
\]
Apply Hensel's lemma to \(R = \Z[a_1, \dots, a_6][[t]], I = (t)\) and \(F(X) = X - f(t, X)\). The approximate root is \(a = 0\) for \(s = 3\). Check \(F(0) = -t^3, F'(0) = 1 - a_1t - a_2t^2 \in R^\times\). Then there exists a unique \(w(t) \in \Z[a_1, \dots, a_6][[t]]\) such that \(w(t) = f(t, w(t))\) and \(w(t) = 0 \pmod{t^3}\).

To see \(w(t)\) explicitly, we follow the proof of Hensel's lemma (with \(u = 1\)) and get \(w(t) = \lim_{n \to \infty} w_n(t)\) where
\[
  w_0(t) = 0, \quad w_{n + 1}(t) = f(t, w_n(t)).
\]
In fact
\[
  \omega(t) = t^3 (1 + A_1t + A_2t^2 + \dots) = \sum_{n = 2}^\infty A_{n - 2}t^{n + 1}
\]
where \(A_1 = a_1, A_2 = a_1^2 + a_2, A_3 = a_1^3 + 2a_1a_2 + a_3, \dots\)

\begin{lemma}
  Let \(R\) be an integral domain, complete with respect to an ideal \(I\). Let \(a_1, \dots, a_6 \in R\) and \(K\) the field of fraction of \(R\). Then
  \[
    \hat E(I) = \{(t, w) \in E(K): t, w \in I\}
  \]
  is a subgroup of \(E(K)\).
\end{lemma}

\begin{remark}
  By unqiueness in Hensel's lemma (with \(s = 1\)), we can also describe \(\hat E(I)\) as
  \[
    \hat E(I) = \{(t, w(t)) \in E(K): t \in I\}.
  \]
\end{remark}

\begin{proof}
  Taking \((t, w) = (0 0)\) shows \(0_E \in \hat E(I)\), so suffices to show if \(P_1, P_2 \in \hat E(I)\) then \(-P_1 - P_2 \in \hat E(I)\). Suppose \(P_i = (t_i, w_i)\). The line \(P_1P_2\) is given by \(\omega = \lambda t + \nu\) where
  \[
    \lambda =
    \begin{cases}
      \frac{w(t_2) - w(t_1)}{t_2 - t_1} & t_1 \neq t_2 \\
      w'(t_1) & t_1 = t_2
    \end{cases}
  \]
  so
  \begin{align*}
    \lambda &= \sum_{n = 2}^\infty A_{n - 2}(t_1^n + t_1^{n - 1}t_2 + \dots + t_2^n) \in I \\
    \nu &= w_1 - \lambda t_1 \in I
  \end{align*}
  Subsituting \(w = \lambda t + \nu\) into \(w = f(t, w)\), we get
  \begin{align*}
    A &= \text{ coefficient of } t^3 = 1 + a_2 \lambda + a_4 \lambda^2 + a_6 \lambda^3 \\
    B &= \text{ coefficient of } t^2 = a_1 \lambda + a_2 \nu + a_3 \lambda^2 + 2a_4 \lambda \nu + 3a_6 \lambda^2 \nu
  \end{align*}
  we have \(A \in R^\times, B \in I\) so \(t_3 = -B/A - t_1 - t_2 \in I\) and \(w_3 = \lambda t_3 + \nu \in I\).
\end{proof}

Taking \(R = \Z[a_1, \dots, a_t][[t]], I = (t)\). The lemma shows that there exists \(\iota(t) \in \Z[a_1, \dots, a_6][[t]]\) with \(\iota(0) = 0\) such that \([-1] (t, w(t)) = (\iota(t), w(\iota(t)))\). Taking \(R = \Z[a_1, \dots, a_6][[t_1, t_2]], I = (t_1, t_2)\), the lemma says there exists \(F \in \Z[a_1, \dots, a_6][[t]]\) with \(F(0, 0) = 0\) such that
\[
  (t_1, w(t_1)) + (t_2, w(t_2)) = (F(t_1, t_2), w(F(t_1, t_2))).
\]
In fact
\begin{align*}
  \iota(X) &= - X - a_1X^2 - a_2X^3 - (a_1^3 + a_3) X^4 + \dots \\
  F(X, Y) &= X + Y - a_1XY - a_2(X^2Y + XY^2) + \dots
\end{align*}
By properties of the group law we deduce
\begin{enumerate}
\item \(F(X, Y) = F(Y, X)\).
\item \(F(X, 0) = X\) and \(F(0, Y) = Y\).
\item \(F(F(X, Y), Z) = F(X, F(Y, Z))\).
\item \(F(X, \iota(X)) = 0\).
\end{enumerate}

\begin{definition}[formal group]\index{formal group}
  Let \(R\) be a ring. A \emph{formal group} over \(R\) is a power series \(F(X, Y) \in R[[X, Y]]\) satisfying 1, 2, 3.
\end{definition}

A question on example sheet 2 shows that for any formal group, there exists a unique \(\iota(t) = -t + \dots \in R[[t]]\) satisfying 4.

\begin{eg}\leavevmode
  \begin{enumerate}
  \item \(F(X, Y) = X + Y\). We call this formal group \(\hat{\mathbb G_a}\).
  \item \(F(X, Y) = X + Y + XY = (1 + X)(1 + Y) - 1\) so is secretly the same as above. We call this formal group \(\hat{\mathbb G_m}\).
  \item \(F\) arising from an elliptic curve. We call it \(\hat E\).
  \end{enumerate}
\end{eg}

\begin{definition}
  Let \(\mathcal F\) and \(\mathcal G\) be formal groups, given by power series \(F\) and \(G\).
  \begin{enumerate}
  \item A \emph{morphism} \(f: \mathcal F \to \mathcal G\) is a power series \(f(T) \in R[[T]]\) with \(f(0) = 0\) satisfying \(f(F(X, Y)) = G(f(X), f(Y))\).
  \item \(\mathcal F \cong \mathcal G\) if there exists morphisms \(f: \mathcal F \to \mathcal G, g: \mathcal G \to \mathcal F\) such that \(f(g(X)) = X, g(f(X)) = X\).
  \end{enumerate}
\end{definition}

\begin{theorem}
  If \(\ch R = 0\) then every formal group \(\mathbb F\) over \(R\) is isomorphic to \(\hat{\mathbb G_a}\) over \(R \otimes \Q\). More precisely,
  \begin{enumerate}
  \item there is a unique power series \(\log(T) = T + \frac{a_2}{2} T^2 + \frac{a_3}{3}T^3 + \cdots\) with \(a_i \in R\) such that
    \[
      \log F(X, Y) = \log(X) + \log(Y).
      \tag{\ast}
    \]
  \item there is a unique power series \(\exp(T) = T + \frac{b_2}{2!} T^2 + \frac{b_3}{3!} T^3 + \cdots\) with \(b_i \in R\) such that
    \[
      \exp \log (T) = \log \exp (T) = T.
    \]
  \end{enumerate}
\end{theorem}

\begin{proof}\leavevmode
  \begin{enumerate}
  \item Write \(F_1(X, Y) = \frac{\partial F}{\partial X}(X, Y)\). For uniqueness, let
    \[
      p(T) = \frac{d}{dT} \log T = 1 + a_2 T + a_3 T^2 + \dots.
    \]
    Differentiating \((\ast)\) with respect to \(X\) gives
    \[
      p(F(X, Y)) F_1(X, Y) = p(X).
    \]
    Putting \(X = 0\) gives \(p(Y) F_1(0, Y) = 1\) so \(p(Y) = F_1(0, Y)^{-1}\) is unqiue. Thus \(\log\) is unique.

    For existence, let \(p(T) = F_1(0, T)^{-1} = 1 + a_2 T + a_3 T^2 + \dots\) for some \(a_i \in R\). Let \(\log T = T + \frac{a_2}{2}T^2 + \dots\). Differentiate the associativity law with respect to \(X\) we get
    \[
      F_1(F(X, Y), Z) F_1(X, Y) = F_1(X, F(Y, Z)).
    \]
    Sub \(X = 0\) and use identity law,
    \[
      F_1(Y, Z) F_1(0, Y) = F_1(0, F(Y, Z))
    \]
    so
    \[
      F_1(Y, Z) p(F(Y, Z)) = p(Y).
    \]
    Integrate with repsect to \(Y\) to get
    \[
      \log (F(Y, Z)) = \log Y + h(Z)
    \]
    for some power series \(h\). By symmetry in \(Y, Z\) have \(h(Z) = \log Z\).
  \item
    We use
    \begin{lemma}
      Let \(f = aT + \cdots \in R[[t]]\) with \(a \in R^\times\). Then exists a unique \(g = a^{-1}T + \cdots \in R[[T]]\) such that \(f(g(T)) = g(f(T)) = T\).
    \end{lemma}

    \begin{proof}
      We construct polynomials \(g_n(T)\) such that \(f(g_n(T)) = T \pmod{T^{n + 1}}\) and \(g_{n + 1}(T) = g_n(T) \pmod{T^{n + 1}}\). Then \(g(T) = \lim_{n \to \infty} g_n(T)\) exists and satisfies \(f(g(T)) = T\).

      To start the induction set \(g_1(T) = a^{-1}T\). Now suppose \(n \geq 2\) and \(g_{n - 1}(T)\) exists so \(f(g_{n - 1}(T)) = T + bT^n \pmod{T^{n + 1}}\) for some \(b \in R\). We put \(g_n(T) = g_{n - 1}(T) + \lambda T^n\) for some \(\lambda \in R\) to be chosen later. Then
      \begin{align*}
        f(g_n(T))
        &= f(g_{n - 1}(T) + \lambda T^n) \\
        &= f(g_{n - 1}(T)) + \lambda aT^n \pmod{T^{n + 1}} \\
        &= T + (b + \lambda a)T^n \pmod{T^{n + 1}}
      \end{align*}
      so we take \(\lambda = -b/a\).

      We get \(g(T) = a^{-1}T + \cdots \in R[[T]]\) such that \(f(g(T)) = T\). Applying the same argument to \(g\) gives \(h(T) = aT + \cdots \in R[[T]]\) such that \(g(h(T)) = T\). Then
      \[
        f(T) = f(g(h(T))) = h(T).
      \]
    \end{proof}
    The theorem then follows except for showing \(b_n \in R\) (not just \(R \otimes \Q\)). This is on example sheet 2.
  \end{enumerate}
\end{proof}

\begin{notation}
  Let \(\mathcal F\) (e.g.\ \(\hat{\mathbb G_a}, \hat{\mathbb G_m}, \hat E\)) be a formal group given by \(F \in R[[X, Y]]\). Suppose \(R\) is complete with respect to \(I\). For \(x, y \in I\) put \(x \oplus_{\mathcal F} y = F(x, y) \in I\). Then \(\mathcal F(I) = (I, \oplus_{\mathcal F})\) is an abelian group. For example \(\hat{\mathbb G(I)} = (I, +), \hat{\mathbb G_m(I)} \cong (1 + I, \times)\) and \(\hat E(I) \subseteq E(K)\) as in lemma 8.2. This also explains the earlier choice of notation.
\end{notation}

\begin{corollary}
  Let \(\mathcal F\) be a formal group over \(R\) and \(n \in \Z\). Suppose \(n \in R^\times\). Then
  \begin{enumerate}
  \item \([n]: \mathcal F \to \mathcal F\) is an isomorphism.
  \item If \(R\) is complete with respect to an ideal \(I\) then \(\times n: \mathcal F(I) \to \mathcal F(I)\) is an isomorphism. In particular \(\mathcal F(I)\) has no \(n\)-torsion.
  \end{enumerate}
\end{corollary}

\begin{proof}
  We first explain the notation \([n]\). We inductively define \([1](T) = T, [n](T) = F([n - 1] T, T)\) for \(n \geq 2\) (for \(n < 0\), use \([-1](T) = \iota(T)\)). An easy induction show \([n](T) = nT + \cdots \in R[[T]]\) so by Lemma 8.4 it is an isomorphism.
\end{proof}

\section{Elliptic curves over local fields}

Let \(K\) be a field, complete with respect to a a discrete valuation \(v: K^* \surj \Z\). The valuation ring, also known as ring of integers, is
\[
  \O_K = \{x \in K^*: v(x) \geq 0\} \cup \{0\}
\]
with unit group
\[
  \O_K^* = \{x \in K^*: v(x) = 0\}
\]
and maximal ideal \(\pi \O_K\) where \(v(\pi) = 1\). It has residue field \(k = \O_k/\pi\O_K\). We assume \(\ch K = 0, \ch k = p > 0\). For example \(K = \Q_p, \O_K = \Z_p, k = \F_p\).

Let \(E/K\) be an elliptic curve.

\begin{definition}[integral/minimal Weierstrass equation]\index{Weierstrass equation!integral}\index{Weierstrass equation!minimal}
  A Weierstrass equation for \(E\) with coefficients \(a_1, \dots, a_6 \in K\) is \emph{integral} if \(a_1, \dots, a_6 \in \O_K\) and is \emph{minimal} if \(v(\Delta)\) is minimal among all integral equations for \(E\).
\end{definition}

\begin{remark}\leavevmode
  \begin{enumerate}
  \item Putting \(x = u^2 x', y = u^3 y'\) gives \(a_i = u^i a_i'\) so integral equation exists.
  \item If \(a_1, \dots, a_6 \in \O_K\) then \(\Delta \in \O_K\) so \(v(\Delta) \geq 0\) so minimal Weierstrass equations exist.
  \item If \(\ch k \neq 2, 3\) then exists a minimal Weierstrass equation of the form \(y^2 = x^3 + ax + b\).
  \end{enumerate}
\end{remark}

\begin{lemma}
  Let \(E/K\) have integral Weierstrass equation
  \[
    y^2 + a_1 xy + a_3 y = x^3 + a_2 x^2 + a_4x + a_6.
  \]
  Let \(0 \neq P \in E(K)\), say \(P = (x, y)\). Then either \(x, y \in \O_K\) or \(v(x) = -2s, v(y) = -3s\) for some \(s \geq 1\).
\end{lemma}

\begin{proof}
  First we deal with the case \(v(x) \geq 0\) (or \(x = 0\)). If \(v(y) < 0\) then \(v(\mathrm{LHS}) = 0\) while \(v(\mathrm{RHS}) > 0\), absurd so \(x, y \in \O_K\).

  Now suppose \(v(x) < 0\). Then
  \[
    v(\mathrm{LHS}) \geq \min(2 v(y), v(x) + v(y), v(y)), \quad v(\mathrm{RHS}) = 3v(x).
  \]
  In each of the three cases, \(v(y) < v(x)\) so \(2v(y) = 3v(x)\).
\end{proof}

\begin{remark}
  See example sheet 1.
\end{remark}

Fix a minimal Weierstrass equation for \(E/K\), we get a formal group \(\hat E\) over \(\O_K\), and
\begin{align*}
  \hat E (\pi^r \O_K)
  &= \{(x, y) \in E(K): -\frac{x}{y}, -\frac{1}{y} \in \pi^r \O_K\} \cup \{0\} \\
  &= \{(x, y) \in E(K): v(\frac{x}{y}) \geq r, v(\frac{1}{y}) \geq r\} \cup \{0\} \\
  &= \{(x, y) \in E(K): v(x) \leq -2r, v(y) \leq -2r\} \cup \{0\}
\end{align*}
by using the lemma. This is a \(\pi\)-neighbourhood of \(0\). By theorem 8.2 this is a subgroup of \(E(K)\), say \(E_r(K)\). Then we have a nested sequence of groups
\[
  E_1(K) \supseteq E_2(K) \supseteq \cdots
\]
More generally for \(\mathcal F\) a formal group over \(\O_K\), we have
\[
  \mathcal F(\pi \O_K) \supseteq \mathcal F(\pi^2 \O_K) \supseteq \cdots
\]
We will show that \(\mathcal F(\pi^r \O_K) \cong (\O_K, +)\) for \(r\) sufficiently large and
\[
  \frac{\mathcal F(\pi^r \O_K)}{\mathcal F (\pi^{r + 1} \O_K)} \cong (k, +)
\]
for all \(r \geq 1\).

A reminder we are working over \(\ch K = 0, \ch k = p\).

\begin{proposition}
  Let \(\mathcal F\) be a formal group over \(\O_K\). Let \(e = v(p)\). If \(r > \frac{e}{p - 1}\) then
  \[
    \log: \mathcal F(\pi^r \O_K) \to \hat{\mathbb G_a}(\pi^r \O_K)
  \]
  is an isomorphism with inverse \(\exp\).
\end{proposition}

\begin{proof}
  For \(x \in \pi^r \O_K\) we must show that the power series \(\exp\) and \(\log\) in theorem 8.3 converge. Recall \(\exp(T) = T + \frac{b_2}{2!} T^2 + \dots\) where \(b_n \in \O_K\). Note that while a ``big'' denominator is good in Archimedean analysis, the situation is the opposite in the non-Archimedean case. Claim \(v_p(n!) = \frac{n - 1}{p - 1}\).

  \begin{proof}
    \[
      v_p(n!) = \sum_{r = 1}^\infty \floor*{\frac{n}{p^r}} < \sum_{r = 1}^\infty \frac{n}{p^r} = \frac{n}{p - 1}
    \]
    so \((p - 1) v_p(n!) < n\). By noting that it is integer valued we get the required inequality.
  \end{proof}

  Now
  \[
    v(\frac{b_n x^n}{n!} \geq nr - e \left(\frac{n - 1}{p - 1}\right) = (n - 1) \underbrace{(r - \frac{e}{p - 1})}_{> 0} + r
  \]
  This is always \(\geq r\) and goes to infinity as \(n \to \infty\) so \(\exp x\) converges and belongs to \(\pi^r \O_K\). \(\log x\) is similar but easier.
\end{proof}

\begin{proposition}
  For \(r \geq 1\),
  \[
    \frac{\mathcal F(\pi^r \O_K)}{\mathcal F (\pi^{r + 1} \O_K)} \cong (k, +).
  \]
\end{proposition}

\begin{proof}
  Recall \(F(X, Y) = X + Y + XY (\cdots)\) so if \(x, y \in \O_K\),
  \[
    F(\pi^rx, \pi^ry) = \pi^r(x + y) \pmod{\pi^{r + 1}}.
  \]
  Thus
  \begin{align*}
    \mathcal F(\pi^r \O_K) &\to (k, +) \\
    \pi^r x &\mapsto x \pmod \pi
  \end{align*}
  is a surjective homomorphism with kernel \(\mathcal F(\pi^{r + 1}\O_K)\).
\end{proof}

\begin{corollary}
  If \(k\) is finite then \(\mathcal F(\pi \O_K)\) contains a subgroup of finite index and is isomorphic to \((\O_K, +)\).
\end{corollary}

\begin{notation}
  We denote reduction mod \(\pi\) by \(x \mapsto \tilde x\).
\end{notation}

\begin{proposition}
  Suppose \(E/K\) is an elliptic curve. The reduction mod \(\pi\) of two minimal Weierstrass equations for \(E\) define isomorphic curves over \(k\).
\end{proposition}

\begin{proof}
  Say Weierstrass equations are related by \([u; r, s, t]\) where \(u \in K^\times, r, s, t \in K\). Then \(\Delta_1 = u^{12} \Delta_2\). Minimality of equations implies that \(u \in \O_K^*\). By transformation formula for \(a_i\) and \(b_i\), we conclude \(r, s, t \in \O_K\). Then the Weierstrass equation for the reductions mod \(\pi\) are related by \([\tilde u; \tilde r, \tilde s, \tilde t]\). Note that all these are to ensure that things work in characteristic \(2\) or \(3\).
\end{proof}

\begin{definition}[reduction]\index{reduction}
  The \emph{reduction} \(\widetilde E/k\) of \(E/K\) is defined to be the reduction of a minimal Weierstrass equation.

  \(E\) has \emph{good reduction} if \(\widetilde E\) is nonsingular (and so is an elliptic curve), otherwise \emph{bad reduction}.
\end{definition}

For an integral Weierstras equation, \(v(\Delta) = 0\) is a sufficient condition for good reduction. On the other hand if \(0 < v(\Delta) < 12\) then by \(\Delta_1 = u^{12} \Delta_2\) we have bad reduction. If \(v(\Delta) \geq 12\) then the equation might not be minimal.

There is a well-defined map
\begin{align*}
  \P^2(K) &\to \P^2(k) \\
  (x:y:z) &\mapsto (\tilde x: \tilde y: \tilde z)
\end{align*}
where we choose representatives with \(\min(v(x), v(y), v(z)) = 0\) to ensure we do not get \((0: 0: 0)\). We restrict to get \(E(K) \to E(k), P \mapsto \widetilde P\). If \(P = (x, y) \in E(K)\) then either \(x, y \in \O_K\) so \(\widetilde P = (\tilde x, \tilde y)\), or \(v(x) = -2s, v(y) = -3s\) and we choose \(P = (\pi^{3s} x: \pi^{3s}y: \pi^{3s})\) which reduces to \(\widetilde P = (0: 1: 0)\). Thus
\[
  E_1(K) = \hat E(\pi \O_K) = \{P \in E(K): \widetilde P = 0\}
\]
is the \emph{kernel of reduction}\index{kernel of reduction}.

Let \(\widetilde E_{\mathrm{ns}}\) be the set of nonsingular points on \(\widetilde E\). If \(E\) has good reduction then this is the same as \(\widetilde E\). Otherwise we delete the singular points. The chord and tangent process still defines a group law on \(\widetilde E_{\mathrm{ns}}\) (since the third intersection point only has multiplicity \(1\)). In case of bad reduction \(\widetilde E_{\mathrm{ns}} \cong \mathbb G_a\) or \(\mathbb G_m\) (over \(\overline k\)), called additive reduction or multiplicative reduction. For simpicity suppose \(\ch k \neq 2\) and we have \(\widetilde E: y^2 = f(x)\). Then \(\widetilde E\) is singular if and only if \(f\) has a repeated root. For double root (\(y^2 = x^2(x + 1)\)) we have a curve with a node and we use multiplicative reduction. For triple root (\(y^2 = x^3\)) we have a curve with a cusp and we use additive reduction
\begin{align*}
  \widetilde E_{\mathrm{ns}} &\to \mathbb G_a \\
  (x, y) &\mapsto \frac{x}{y} \\
  (t^{-2}, t^{-3}) &\mapsfrom t \\
  \infty &\mapsfrom 0
\end{align*}
We check this is a group homomorphism. Let \(P_1, P_2, P_3\) be on the line \(ax + by = 1\). Write \(P_i = (x_i, y_i), t_i = \frac{x_i}{y_u}\). Then \(x_i^3 = y_i^2 = y_i^2(ax_i + by_i)\) so \(t_1, t_2, t_3\) are roots of \(X^3 - aX - b = 0\). Looking at the coefficient of \(X^2\) gives \(t_1 + t_2 + t_3 = 0\).

The node case is on example sheet.

\begin{definition}
  We define
  \[
    E_0(K) = \{P \in E(K): \widetilde P \in \widetilde E_{\mathrm{ns}}(k)\},
  \]
  the points that do not become singular upon reduction.
\end{definition}

\begin{proposition}
  \(E_0(K)\) is a subgroup of \(E(K)\) and reduction mod \(\pi\) is a surjective group homomorphism \(E_0(K) \to \widetilde E_{\mathrm{ns}}(k)\).
\end{proposition}

\begin{proof}
  First check this is a group homomorphism. A line \(\ell\) in \(\P^2\) defined over \(K\) has equation \(aX + bY + cZ = 0\) where \(a, b, c \in K\). We may assume \(\min (v(a), v(b), v(c)) = 0\). Reduction mod \(\pi\) given the line \(\tilde \ell\) \(\tilde a X + \tilde b Y + \tilde c Z = 0\). If \(P_1, P_2, P_3 \in E(K)\) with \(P_1 + P_2 + P_3 = 0\) then they lie on a line \(\ell\). Then \(\widetilde P_1, \widetilde P_2, \widetilde P_3\) lie on \(\tilde \ell\). If \(\widetilde P_1, \widetilde P_2 \in \widetilde E_{\mathrm{ns}}(k)\) then \(\widetilde P_3 \in \widetilde E_{\mathrm{ns}}(k)\) so if \(P_1, P_2 \in E_0(K)\) then \(P_3 \in E_0(K)\) and \(\widetilde P_1 + \widetilde P_2 + \widetilde P_3 = 0\). It is an exercise to check that this still works when \(\widetilde P_1, \widetilde P_2, \widetilde P_3\) are not necessarily distinct.

  Now we show surjectivity. Let \(f(x, y) = y^2 + a_1xy + a_3y - (x^3 + \dots)\) be the Weierstrass equation. Let \(\widetilde P \in \widetilde E_{\mathrm{ns}}(k) \setminus \{0\}\), say \(\widetilde P = (\tilde x_0, \tilde y_0)\) for some \(x_0, y_0 \in \O_K\). \(\widetilde P\) nonsingular implies that either \(\frac{\partial f}{\partial x}(x_0, y_0) \neq 0 \pmod \pi\) or \(\frac{\partial f}{\partial y}(x_0, y_0) \neq 0 \pmod \pi\). In the first case put \(g(t) = f(t, y_0) \in \O_K[t]\). Then
  \[
    g(x_0) = 0 \pmod \pi, \quad g'(x_0) \in \O_K^*
  \]
  so by Hensel's lemma exists \(b \in \O_K\) such that \(g(b) = 0, b = x_0 \pmod \pi\). Then \(P = (b, y_0) \in E(K)\) has reduction \(\widetilde P\). The second case is similar.
\end{proof}

Recall that for \(r \geq 1\) we put
\[
  E_r(K) = \{(x, y) \in E(K): v(x) \leq -2r, v(y) \leq -3r\} \cup \{0\}
\]
and we have a nested sequence of groups
\[
  (\O_K, +) \cong E_r(K) \subseteq \cdots \subseteq E_2(K) \subseteq E_1(K) \subseteq E_0(K) \subseteq E(K)
\]
for \(r > \frac{e}{p - 1}\). The quotient \(\frac{E_0(K)}{E_1(K)} \cong \widetilde E_{\mathrm{ns}}(K)\) and all quotients \(\frac{E_{t + 1}}{E_t} \cong (k, +)\). What about \(E_0(K) \subseteq E(K)\)? There are much to be said about this but we only cover a special case here. More can be found is Silverman's sequel.

\begin{lemma}
  If \(|k| < \infty\) then \(\P^n(K)\) is compact (with respect to \(\pi\)-adic topology).
\end{lemma}

\begin{proof}
  If \(|k| < \infty\) then \(\frac{\O_K}{\pi^r \O_K}\) is finite for \(r \geq 1\) so \(\O_K \cong \varprojlim_r \O_K/\pi^r \O_K\) is compact. \(\P^n(K)\) is the union of compact sets
  \[
    \{(a_0: a_1: \cdots : a_{i - 1}: 1 : a_{i + 1}: \cdots: a_n): a_j \in \O_K\}
  \]
  and hence compact.
\end{proof}

\begin{lemma}
  If \(|k| < \infty\) then \(E_0(K) \subseteq E(K)\) has finite index.
\end{lemma}

\begin{proof}
  \(E(K) \subseteq \P^2(K)\) is a closed subset so \((E(K), +)\) is a compact topological group. If \(\widetilde E\) has singular point \((\tilde x_0, \tilde y_0)\) then
  \[
    E(K) \setminus E_0(K) = \{(x, y) \in E(K): v(x - x_0) \geq 1, v(y - y_0) \geq 1\}
  \]
  (?) is a closed subset of \(E(K)\) and so \(E_0(K)\) is an open subgroup of \(E(K)\). The cosets of \(E_0(K)\) are an open cover of \(E(K)\), and thus \(E_0(K)\) has finite index in \(E(K)\) by compactness. The index is called \emph{Tamagawa number} and is denoted \(c_K(E)\).
\end{proof}

\begin{remark}
  Good reduction implies that \(c_K(E) = 1\) but the converse is false.
\end{remark}

\begin{fact}
  For these facts it is essential that \(E\) is defined by a minimal Weierstrass equation, but we don't need \(|k| < \infty\).

  Either \(c_K(E) = v(\Delta)\) or \(c_K(E) \leq 4\)
\end{fact}

\begin{theorem}
  If \([K: \Q_p] < \infty\) then \(E(K)\) contains a subgroup \(E_r(K)\) of finite index with \(E_r(K) \cong (\O_K, +)\).
\end{theorem}

\begin{proof}
  We have \(|k| < \infty\). Combine all results in this chapter.
\end{proof}

\begin{corollary}
  \(E(K)_{\mathrm{tors}}\) injects into \(\frac{E(K)}{E_r(K)}\) and is therefore finite.
\end{corollary}

We now quote some results from algebraic number theory. Let \([K: \Q_p] < \infty\) and \(L/K\) a finite extension. Then \([L: K] = ef\) where \(v_L|_{K^*} = e v_K\) and \(f = [k': k]\) where \(k'\) and \(k\) are the residue fields of \(L\) and \(K\) respectively. If \(L/K\) is Galois then there is a natural group homomorphism \(\gal(L/K) \to \gal(k'/k)\). This map is surjective with kernel of order \(e\).

\begin{definition}[unramified extension]\index{unramified extension}
  \(L/K\) is \emph{unramified} if \(e = 1\).
\end{definition}

\begin{fact}
  For each integer \(m \geq 1\),
  \begin{enumerate}
  \item \(k\) has a unique extension of degree \(m\), say \(k_m\).
  \item \(K\) has a unique unramified extension of degree \(m\), say \(K_m\).
  \end{enumerate}
\end{fact}

\begin{definition}[maximal unramified extension]\index{maximal unramified extension}
  We define the \emph{maximal unramified extension} to be \(K^{\mathrm{nr}} = \bigcup_{m \geq 1} K_m\) (inside \(\overline K\)).
\end{definition}

\begin{theorem}
  Suppose \([K: \Q_p] < \infty\), \(E/K\) an elliptic curve with good reduction and \(p \ndivides n\). If \(P \in E(K)\) then \(K([n]^{-1}P)/K\) is unramified.
\end{theorem}

Recall that when we do not specify a base field then we refer to the algebraic closure so
\[
  [n]^{-1}P = \{Q \in E(\overline K) = nQ = P\}.
\]
Also we denote
\[
  K(\{P_1, \dots, P_r\}) = K(X_1, \dots, x_r, y_1, \dots, y_r)
\]
where \(P_i = (x_i, y_i)\).

\begin{proof}
  For each \(m \geq 1\) there is a short exact sequence
  \[
    \begin{tikzcd}
      0 \ar[r] & E_1(K_m) \ar[r] & E(K_m) \ar[r] & \widetilde E(k_m) \ar[r] & 0
    \end{tikzcd}
  \]
  Taking union over all \(m \geq 1\) gives a commutative diagram with exact rows
  \[
    \begin{tikzcd}
      0 \ar[r] & E_1(K^{\mathrm{nr}}) \ar[d, "n"] \ar[r] & E(K^{\mathrm{nr}}) \ar[r] \ar[d, "n"] & \widetilde E(\overline k) \ar[r] \ar[d, "n"] & 0 \\
      0 \ar[r] & E_1(K^{\mathrm{nr}}) \ar[r] & E(K^{\mathrm{nr}}) \ar[r] & \widetilde E(\overline k) \ar[r] & 0
    \end{tikzcd}
  \]
  The left vertical map is an isomorphism by corollary 8.5, which applies since \(p \ndivides n\) implies \(n \in \O_K^*\). The right vertical map is surjective by Theorem 2.8 and has kernel isomorphic to \((\Z/n\Z)^2\) by theorem 6.5. Then by snake lemma
  \[
    E(K^{\mathrm{nr}})[n] \cong (\Z/n\Z)^2, \frac{E(K^{\mathrm{nr}})}{n E(K^{\mathrm{nr}})} = 0
  \]
  so if \(P \in E(K)\) then \(P = nQ\) for some \(Q \in E(K^{\mathrm{nr}})\) so
  \[
    [n]^{-1} P = \{Q + T: T \in E[n]\} \subseteq E(K^{\mathrm{nr}})
  \]
  so \(K([n]^{-1} P) \subseteq K^{\mathrm{nr}}\) so \(K([n]^{-1}P)/K\) is unramified.
\end{proof}

\section{Elliptic curves over number fields}

Suppose \([K: \Q] < \infty\) and \(E/K\) is an elliptic curve. Throughout we let \(\mathfrak p\) be a prime of \(K\) (i.e.\ of \(\O_K\)), \(K_{\mathfrak p}\) the \(\mathfrak p\)-adic completion of \(K\) and \(k_{\mathfrak p} = \O_k/\mathfrak p\).

\begin{definition}[prime of good reduction]\index{good reduction}
  \(\mathfrak p\) is a prime of \emph{good reduction} for \(E/K\) if \(E/K_{\mathfrak p}\) has good reduction.
\end{definition}

\begin{lemma}
  \(E/K\) has only finitely many primes of bad reduction.
\end{lemma}

\begin{proof}
  Take a Weierstrass equation for \(E\) with coefficients \(a_1, \dots, a_6 \in \O_K\). \(E\) is nonsingular implies that \(0 \neq \Delta \in \O_K\). Write \((\Delta) = \mathfrak p_1^{\alpha_1} \cdots \mathfrak p_r^{\alpha_r}\) for the factorisation into prime ideals. Let \(S = \{\mathfrak p_1, \dots, \mathfrak p_r\}\). If \(\mathfrak p \notin S\) then \(v_{\mathfrak p}(\Delta) = 0\) so \(E/K_{\mathfrak p}\) has good reduction.
\end{proof}

\begin{remark}
  If \(K\) has class number \(1\) (e.g.\ \(K = \Q\)) then we can always find a Weierstrass equation for \(a_1, \dots, a_6 \in \O_K\) which is minimal at all primes \(\mathfrak p\).
\end{remark}

\begin{lemma}
  \(E(K)_{\mathrm{tor}}\) is finite.
\end{lemma}

\begin{proof}
  Take any \(\mathfrak p\). Note \(K \subseteq K_{\mathfrak p}\) and apply theorem 9.8.
\end{proof}

\begin{lemma}
  Let \(\mathfrak p\) be a prime of good reduction with \(\mathfrak p \ndivides n\). Then reduction modulo \(\mathfrak p\) gives an injection \(E(K)[n] \embed \widetilde E(k_{\mathfrak p})[n]\).
\end{lemma}

\begin{proof}
  Proposition 9.5 says that \(E(K_{\mathfrak p}) \to \widetilde E(k_{\mathfrak p})\) is a group homomorphism with kernel \(E_1(K_{\mathfrak p})\). Then corollary 8.5 implies that \(E_1(K_{\mathfrak p})\) has no \(n\)-torsion.
\end{proof}

\begin{eg}
  Let \(E/\Q: y^2 + y = x^3 - x^2\). \(\Delta = -11\). \(E\) has good reduction at all primes \(p \neq 11\).
  \begin{table}[h]
    \centering
    \begin{tabular}{c|cccccc}
      p & 2 & 3 & 5 & 7 & 11 & 13 \\ \hline
      \(\# \widetilde E(\F_p)\) & 5 & 5 & 5 & 10 & - & 10
    \end{tabular}
  \end{table}
  so by looking at \(2\) and \(3\), \(\#E(\Q)_{\mathrm{tor}} \divides 5 \cdot 2^a\) for some \(a \geq 0\). \(\#E(\Q)_{\mathrm{tor}} \divides 5 \cdot 3^b\) for some \(b \geq 0\), so \(\#E(\Q)_{\mathrm{tor}} \divides 5\). Let \(T = (0, 0) \in E(\Q)\). We can check that \(5T = 0\) so \(E(\Q)_{\mathrm{tor}} \cong \Z/5\Z\).
\end{eg}

\begin{eg}
  Let \(E/\Q: y^2 + y = x^3 + x\). \(\Delta = -43\). \(E\) has good reduction at all \(p \neq 43\).
  \begin{table}[h]
    \centering
    \begin{tabular}{c|cccccc}
      p & 2 & 3 & 5 & 7 & 11 & 13 \\ \hline
      \(\# \widetilde E(\F_p)\) & 5 & 6 & 10 & 8 & 9 & 19
    \end{tabular}
  \end{table}
  By considering \(p = 2, 11\) we show \(E(\Q)_{\mathrm{tor}} = \{0\}\). Thus \(P = (0, 0) \in E(\Q)\) is a point of infinite order. Thus rank of \(E(\Q) \geq 1\).
\end{eg}

\begin{eg}
  Let \(E_D: y^2 = y^2 = x^3 - D^2 x\) where \(D \in \Z\) square free and \(\Delta = 2^6 D^6\). We know the torsion group contains \(\{0, (0, 0), (\pm d, 0)\} \cong (\Z/2\Z)^2\). Let \(f(x) = x^3 - D^2x\). We can count the number of points using Legendre symbol. If \(p \ndivides 2D\) then
  \[
    \# \widetilde E_D(\F_p) = 1 + \sum_{x \in \F_p} (\legendre{f(x)}{p} + 1).
  \]
  If \(p = 3 \pmod 4\) then since \(f(x)\) is an odd function,
  \[
    \legendre{f(-x)}{p} = \legendre{-f(x)}{p} = \legendre{-1}{p} \legendre{f(x)}{p} = - \legendre{f(x)}{p}
  \]
  so \(\# \widetilde E_D(\F_p) = p + 1\).

  Let \(m = \# E_D(\Q)_{\mathrm{tor}}\). We have \(4 \divides m \divides (p + 1)\) for all sufficiently large primes \(p\) with \(p = 3 \pmod 4\). Then by \(m = 4\) as otherwise we will get a contradiction to Dirichlet's theorem on primes in arithmetic progression. Thus \(E_D(\Q)_{\mathrm{tor}} \cong (\Z/2\Z)^2\). Thus rank \(E_D(\Q) \geq 1\) if and only if there exists \(x, y \in \Q\) with \(y \neq 0\) and \(y^2 = x^3 - D^2x\), if and only if \(D\) is a congruent number.
\end{eg}

\begin{lemma}
  Let \(E/\Q\) be given by a Weierstrass equation with \(a_1, \dots, a_6 \in \Z\). Suppose \(0 \neq T = (x, y) \in E(\Q)_{\mathrm{tor}}\). Then
  \begin{enumerate}
  \item \(4x, 8y \in \Z\),
  \item if \(2 \divides a_1\) or \(2T \neq 0\) then \(x, y \in \Z\).
  \end{enumerate}
\end{lemma}

\begin{proof} \leavevmode
  \begin{enumerate}
  \item The Weierstrass equation defines a formal group \(\hat E\) over \(\Z\). For \(r \geq 1\), recall
    \[
      \hat E(p^r \Z_p) = \{(x, y) \in E(\Q_p): v_p(x) \leq -2r, v_p(y) \leq -3r\} \cup \{0\}.
    \]
    Proposition 9.2 says \(\hat E(p^r\Z_p) \cong (\Z_p, +)\) if \(r > \frac{1}{p - 1}\). Thus \(\hat E(4 \Z_2)\) and \(\hat E(p\Z_p)\) for \(p\) odd are torsion free. Thus if \(0 \neq T = (x, y) \in E(\Q)_{\mathrm{tors}}\) then \(T \notin \hat E(4\Z_2)\), so \(v_2(x) \geq -2, v_2(y) \geq -3\). \(T \notin \hat E(p\Z_p)\) so \(v_p(X) \geq 0, v_p(y) \geq 0\).
  \item Suppose \(T \in \hat E(2 \Z_2)\), i.e.\ \(v_2(x) = -2, v_3(y) = -3\). Since \(\frac{\hat E(2\Z_2)}{\hat E(4\Z_2)} \cong (\F_2, +)\) and \(\hat E(4\Z_2)\) is torsion free, we get \(2T = 0\). Also
    \[
      (x, y) = T = -T = (x, -y - a_1x - a_3)
    \]
    so \(2y + a_1x + a_3 = 0\). Thus \(8y + a_1 (4x) + 4a_3 = 0\), and \(8y, 4x\) are both odd and \(4a_3 = 0\) so \(a_1\) is odd. Thus if \(2T \neq 0\) or \(a_1\) is even then \(T \in \hat E(2\Z_2)\) and so \(x, y \in \Z\).
  \end{enumerate}
\end{proof}

\begin{eg}
  \(y^2 + xy + x^3 + 4x + 1\) has \((-\frac{1}{4}, \frac{1}{8}) \in E(\Q)[2]\).
\end{eg}

\begin{theorem}[Lutz Nagell]
  Let \(E/\Q: y^2 = x^3 + ax + b\) where \(a, b \in \Z\). Suppose \(0 \neq T = (x, y) \in E(\Q)_{\mathrm{tors}}\). Then \(x, y \in \Z\) and either \(y = 0\) or \(y^2 \divides (4a^2 + 27b^2)\).
\end{theorem}

\begin{proof}
  Lemma 10.4 implies \(x, y \in \Z\). If \(2T = 0\) then \(y = 0\). Otherwise \(0 \neq 2T = (x_2, y_2)\) is torsion so \(x_2, y_2 \in \Z\). Then \(x_2 = \left(\frac{f'(x)}{2y}\right)^2 - 2x\). Everything is integer so \(y \divides f'(x)\). \(E\) is nonsingular so \(f(X)\) and \(f'(X)\) are coprime. \(f(X)\) and \(f'(X)^2\) are coprime so exists \(g, h \in \Q[X]\) such that \(g(X) f(X) + h(X) f'(X)^2 = 1\). A calculation gives
  \[
    (3X^3 + 4a) f'(X)^2 - 27(X^3 + aX - b)f(X) = 4a^3 + 27b^2.
  \]
  Since \(y \divides f'(x)\) and \(y^2 = f(x)\) we get \(y^2 \divides (4a^3 + 27b^2)\).
\end{proof}

\begin{remark}
  Mazur has shown that if \(E/\Q\) is an elliptic curve then \(E(\Q)_{\mathrm{tors}}\) is isomorphic to one of the below:
  \[
    \Z/n\Z \text{ for } 1 \leq n \leq 12, n \neq 11 \text{ or } \Z/2\Z \times \Z/2n\Z \text{ for } 1 \leq n \leq 4.
  \]
  Moreover all 15 possibilities occur.
\end{remark}

\section{Kummer theory}

Let \(K\) be a field with \(\ch K \ndivides n\). Assume \(\mu_n \subseteq K\).

\begin{lemma}
  Let \(\Delta \subseteq K^*/(K^*)^n\) be a finite subgroup. Let \(L = K(\sqrt[n]{\Delta})\). Then \(L/K\) is Galois and
  \[
    \gal(L/K) \cong \Hom(\Delta, \mu_n).
  \]
\end{lemma}

\begin{proof}
  \(L/K\) is Galois since \(\mu_n \subseteq K\) and \(\ch K \ndivides n\). Define the \emph{Kummer pairing}\index{Kummer pairing}
  \begin{align*}
    \langle \cdot , \cdot \rangle: \gal(L/K) \times \Delta &\to \mu_n \\
    (\sigma, x) &\mapsto \frac{\sigma(\sqrt[n]{x})}{\sqrt[n]{x}}
  \end{align*}
  Check this is well-defined: if \(\alpha, \beta \in L\) with \(\alpha^n = \beta^n = x\) then \((\frac{\alpha}{\beta})^n = 1\) so \(\frac{\alpha}{\beta} \in \mu_n \subseteq K\) so \(\sigma(\frac{\alpha}{\beta}) = \frac{\alpha}{\beta}\) so \(\frac{\sigma(\alpha)}{\alpha} = \frac{\sigma(\beta)}{\beta}\). It is bilinear:
  \begin{align*}
    \langle \sigma\tau, x \rangle
    &= \frac{\sigma(\tau \sqrt[n]{x})}{\tau \sqrt[n]x} \frac{\tau \sqrt[n]x}{\sqrt[n]x}
      = \langle \sigma, x \rangle \langle \tau, x \rangle \\
    \langle \sigma, xy \rangle
    &= \frac{\sigma \sqrt[n]{xy}}{\sqrt[n]{xy}}
      = \frac{\sigma \sqrt[n]x}{\sqrt[n]x} \frac{\sigma \sqrt[n]y}{\sqrt[n]y}
      = \langle \sigma, x\rangle \langle \sigma, y\rangle
  \end{align*}
  The pairing is nondegenerate in both arguments: let \(\sigma \in \gal(L/K)\). If \(\langle \sigma, x\rangle = 1\) for all \(x \in \Delta\) then \(\sigma \sqrt[n]x = \sqrt[n]x\) for all \(x \in \Delta\) so \(\sigma\) fixes \(L\) pointwise so \(\sigma = 1\). Conversely let \(x \in \Delta\). If \(\langle \sigma, x \rangle = 1\) for all \(\sigma \in \gal(L/K)\) then \(\sigma \sqrt[n]x = \sqrt[n]x\) for all \(\sigma\) so \(\sqrt[n]x \in K^*\) so \(x \in (K^*)^n\).

  To put it in another way \(\gal(L/K)\) and \(\Delta\) are dual groups to each other and we have two injective group homomorphisms
  \begin{enumerate}
  \item \(\gal(L/K) \embed \Hom(\Delta, \mu_n)\),
  \item \(\Delta \embed \Hom(\gal(L/K), \mu_n)\).
  \end{enumerate}
  Statement 1 implies \(\gal(L/K)\) is an abelian group of exponent dividing \(n\). Now similar to the fact that the dual group of a finite abelian group has the same size, we have \(|\Hom(\Delta, \mu_n)| = |\Delta|\) and same for the other so
  \[
    |\gal(L/K)| \leq |\Delta| \leq |\gal(L/K)|
  \]
  so 1 and 2 are isomorphisms.
\end{proof}

\begin{eg}
  \(\gal(\Q(\sqrt 2, \sqrt 3, \sqrt 5)/\Q) \cong (\Z/2\Z)^3\).
\end{eg}

\begin{theorem}
  There is a bijection
  \begin{align*}
    \left\{
    \begin{tabular}{c}
      finite subgroups \\
      \(\Delta \subseteq K^*/(K^*)^n\)
    \end{tabular}
    \right\}
    &\longleftrightarrow
      \left\{
      \begin{tabular}{c}
        finite abelian extensions \\
        \(L/K\) of exponent \\
        dividing \(n\)
      \end{tabular}
    \right\} \\
    \Delta &\mapsto K(\sqrt[n] \Delta) \\
    \frac{(L^*)^n \cap K^*}{(K^*)^n} &\mapsfrom L
  \end{align*}
\end{theorem}

\begin{proof}
  Let \(\Delta \subseteq K^*/(K^*)^n\) be a finite subgroup. Let \(L = K(\sqrt[n]\Delta)\) and \(\Delta' = \frac{(L^*)^n \cap K^*}{(K^*)^n}\). Clearly \(\Delta \subseteq \Delta'\). To show equality,
  \[
    L = K(\sqrt[n]\Delta) \subseteq K(\sqrt[n]{\Delta'}) \subseteq L
  \]
  so \(K(\sqrt[n]\Delta) = K(\sqrt[n]{\Delta'})\) so \(|\Delta| = |\Delta'|\) by the lemma. Thus equality.

  Conversely let \(L/K\) be a finite abelian extension of exponent dividing \(n\). Let \(\Delta\) be as defined in the statement. Then \(K(\sqrt[n]{\Delta}) \subseteq L\). We aim to show equality by showing \([K(\sqrt[n]{\Delta}) : K] = [L : K]\). Let \(G = \gal(L/K)\). The Kummer pairing defines an injective group homomorphism \(\Delta \embed \Hom(G, \mu_n)\). Claim this is surjective.

  \begin{proof}
    Let \(\chi: G \to \mu_n\) be a group homomorphism. From basic Galois theory distinct automorphisms are linearly independent so exists \(a \in L\) such that \(y = \sum_{\tau \in G} \chi(\tau)^{-1} \tau(a) \neq 0\). Let \(\sigma \in G\). Then
    \[
      \sigma(y)
      = \sum_{\tau \in G} \chi(\tau)^{-1} \sigma \tau(a)
      = \sum_{\tau \in G} \chi(\sigma^{-1} \tau)^{-1} \tau(a)
      = \chi(\sigma) y
    \]
    Thus \(\sigma(y^n) = y^n\) for all \(\sigma \in G\) so \(x = y^n \in K^* \cap (L^*)^n\). Then \(x \in \Delta\) and \(\chi: \sigma \mapsto \frac{\sigma(y)}{y} = \frac{\sigma \sqrt[n]{x}}{\sqrt[n]{x}}\).
  \end{proof}

  Now
  \[
    [K(\sqrt[n]{\Delta}) : K] = |\Delta| = |\Hom(G, \mu_n)| = |G| = [L : K].
  \]
\end{proof}

\begin{proposition}
  Let \(K\) be a number field and \(\mu_n \subseteq K\). Let \(S\) be a finite set of primes of \(K\). There are only finitely many extensions \(L/K\) such that
  \begin{enumerate}
  \item \(L/K\) is abelian of exponent dividing \(n\).
  \item \(L/K\) is unramified at all primes \(\mathfrak p \notin S\).
  \end{enumerate}
\end{proposition}

\begin{proof}
  By 11.2 \(L = K(\sqrt[n]{\Delta})\) for some finite subgroup \(\Delta \subseteq K^*/(K^*)^n\). Let \(\mathfrak p\) be a prime of \(K\) with
  \[
    \mathfrak p \O_L = \mathfrak P_1^{e_1} \cdots \mathfrak P_r^{e_r}
  \]
  for distinct primes \(\mathfrak P_i\) of \(L\). If \(x \in K^*\) represents an element of \(\Delta\) then
  \[
    n v_{\mathfrak P_i}(\sqrt[n]{x}) = v_{\mathfrak P_i}(x) = e_i v_{\mathfrak p}(x).
  \]
  If \(\mathfrak p \notin S\) then \(e_i = 1\) for all \(i\) so \(v_{\mathfrak p}(x) = 0 \pmod n\). Thus \(\Delta \subseteq K(S, n)\) where
  \[
    K(S, n) = \{x \in K^*/(K^*)^n: v_{\mathfrak p}(x) = 0 \pmod n \text{ for all } \mathfrak p \notin S\}.
  \]

  \begin{lemma}
    \(K(S, n)\) is finite.
  \end{lemma}

  \begin{proof}
    The map
    \begin{align*}
      K(S, n) &\to (\Z/n\Z)^{|S|} \\
      x &\mapsto (v_{\mathfrak p}(x) \pmod n)_{\mathfrak p \in S}
    \end{align*}
    is a group homomorphism with kernel \(K(\emptyset, n)\) so suffice to prove the lemma with \(S = \emptyset\). If \(x \in K^*\) represents an element of \(K(\emptyset, n)\) then \((x) = \mathfrak a^n\) for some ideal \(\mathfrak a\). There is an exact sequence
    \[
      \begin{tikzcd}
        0 \ar[r] & \O_K^*/(\O_K^*)^n \ar[r] & K(\emptyset, n) \ar[r] & \Cl_K[n] \ar[r] & 0
      \end{tikzcd}
    \]
    From algebraic number theory \(|\Cl_K| < \infty\) and \(\O_K^*\) is finitely generated (Dirichlet's unit theorem) so \(K(\emptyset, n)\) is finite.
  \end{proof}
\end{proof}

\section{Elliptic curves over number fields II}

Mordell-Weil Theorem

\begin{lemma}
  Let \(E/K\) be an elliptic curve and \(L/K\) be a finite Galois extension. Then the map \(\frac{E(K)}{n E(K)} \to \frac{E(L)}{n E(L)}\) has finite kernel.
\end{lemma}

\begin{proof}
  For each element in the kernel we pick a coset representative \(P \in E(K)\) and then exists \(Q \in E(L)\) such that \(n Q = P\). \(\gal(L/K)\) is finite and \(E[n]\) is finite so there are only finitely many possibilities for the map \(\gal(L/K) \to E[n], \sigma \mapsto \sigma Q - Q\). But if \(P_1, P_2 \in E(K)\) with \(P_i = nQ_i\) and \(\sigma Q_1 - Q_2 = \sigma Q_2 - Q_2\) for all \(\sigma \in \gal(L/K)\) then \(\sigma(Q_1 - Q_2) = Q_2 - Q_2\) so \(Q_1 - Q_2 \in E(K)\), and hence \(P_1 - P_2 \in n E(K)\).
\end{proof}

\begin{theorem}[weak Mordell-Weil theorem]\index{Mordell-Weil theorem}
  Let \(K\) be a number field and \(E/K\) an elliptic curve. Then for \(n \geq 2\), \(|\frac{E(K)}{n E(K)}| < \infty\).
\end{theorem}

\begin{proof}
  By lemma wlog we can assume \(\mu_n \subseteq K\) and \(E[n] \subseteq E(K)\). Let \(S = \{\mathfrak p \divides n\} \cup \{\text{primes of bad reduction for } E\}\). For each \(P \in E(K)\) the extension \(K([n]^{-1}P)/K\) is unramified outside \(S\) by theorem 9.9.

  Let \(Q \in [n]^{-1}P\). Since \(E[n] \subseteq E(K)\), \(K(Q) = K([n]^{-1}P)\) is a Galois extension of \(K\). Define
  \begin{align*}
    \gal(K(Q)/K) &\to E[n] \cong (\Z/n\Z)^2 \\
    \sigma &\mapsto \sigma Q - Q
  \end{align*}
  Check this is a homomorphism:
  \[
    \sigma\tau Q - Q = \sigma(\tau Q - Q) + \sigma Q - Q = (\tau Q - Q) + (\sigma Q - Q).
  \]
  It is injective as \(\sigma Q = Q\) implies \(\sigma\) fixes \(K(Q)\) so \(\sigma = 1\). Thus \(K(Q)/K\) is an abelian extension of exponent dividing \(n\), unramified outside \(S\). By 11.3 only there are only finitely many possibilities for \(K(Q)\). Let \(L\) be the composite of all such extensions (i.e.\ for all \(P \in E(K)\)). Then \(L/K\) is finite (and Galois) and \(\frac{E(K)}{nE(K)} \to \frac{E(L)}{nE(L)}\) is the zero map. Apply lemma 12.1.
\end{proof}

\begin{remark}
  If \(K = \R\) or \(\C\) or \([K : \Q_p] < \infty\) then \(|\frac{E(K)}{nE(K)}| < \infty\), yet \(E(K)\) is not finitely generated (even uncountable).
\end{remark}

\begin{fact}
  Let \(E/K\) be a elliptic curve over a number field. Then there exists a quadratic form, called \emph{canonical height}\index{canonical height} \(\hat h: E(K) \to \R_{\geq 0}\) with the property that for any \(B \geq 0\), \(\{P \in E(K): \hat h(P) \leq B\}\) is finite.
\end{fact}

\begin{theorem}[Mordell-Weil]\index{Mordell-Weil theorem}
  Let \(K\) be a number field and \(E/K\) an elliptic curve. Then \(E(K)\) is a finitely generated abelian group.
\end{theorem}

\begin{proof}
  Fix an integer \(n \geq 2\). Weak Mordell-Weil implies that \(|\frac{E(K)}{nE(K)}| < \infty\). Pick coset representatives \(P_1, \dots, P_m\). Let \(\Sigma = \{P \in E(K): \hat h(P) \leq \max_{1 \leq i \leq n} \hat h(P_i)\}\). Claim \(\Sigma\) generates \(E(K)\).

  \begin{proof}
    Suppose not. Then exists \(P \in E(K) \setminus \{\text{subgroup generated by }\Sigma\}\) of minimal height. Then \(P = P_i + nQ\) for some \(1 \leq i \leq m\) where \(Q \in E(K) \setminus \{\text{subgroup generated by } \Sigma\}\). Then \(\hat h(P) \leq \hat h(Q)\). Then
    \begin{align*}
      4 \hat h(P)
      &\leq 4 \hat h(Q) \\
      &\leq n^2 \hat(Q) \\
      &= \hat h(nQ) \\
      &= \hat h(P - P_2) \\
      &\leq \hat h(P - P_i) + \hat h(P + P_i) \\
      &= 2 \hat h(P) + 2 \hat h(P_1) \text{ parallalogram law}
    \end{align*}
    so \(\hat h(P) \in \hat h(P_i)\) so \(P \in \Sigma\), contradiction.
  \end{proof}
  \(\Sigma\) is finite so done.
\end{proof}

\section{Heights}

For simplicity take \(K = \Q\). Write \(P \in \P^n(\Q)\) as \(P = (a_1: \cdots: a_n)\) where \(a_0, \dots, a_n \in \Z, \gcd(a_0, \dots, a_n) = 1\).

\begin{definition}[height]\index{height}
  We define the \emph{height} of \(P\) to be
  \[
    H(P) = \max_{0 \leq i \leq n} |a_i|.
  \]
\end{definition}

\begin{lemma}
  Let \(f_1, f_2 \in \Q[X_1, X_2]\) be coprime homogeneous polynomials of degree \(d\). Let
  \begin{align*}
    F: \P^1 &\to \P^1 \\
    (x_1: x_2) &\mapsto (f_1(x_1, x_2): f_2(x_1, x_2))
  \end{align*}
  Then exists \(c_1, c_2 > 0\) such that
  \[
    c_1 H(P)^d \leq H(F(P)) \leq c_2 H(P)^d
  \]
  for all \(P \in \P^1(\Q)\).
\end{lemma}

\begin{proof}
  wlog \(f_1, f_2 \in \Z[X_1, X_2]\). We prove the upper bound first. Write \(P = (a: b)\) where \(a, b \in \Z\) coprime. Then
  \[
    H(F(P))
    \leq \max(|f_1(a, b)|, |f_2(a, b)|)
    \leq c_2 \max(|a|^d, |b|^d)
    = c_2 H(P)^d
  \]
  where \(c_2\) is the maximum of the sum of absolute values of coefficients of \(f_1\) and \(f_2\).

  For the lower bound, we claim exists \(g_{ij} \in \Z[X_1, X_2]\) homogeneous of degree \(d - 1\) and \(\kappa \in \Z_{> 0}\) such that
  \[
    \sum_{j = 1}^2 g_{ij}f_j = \kappa  X_i^{2d - 1}.
    \tag{\(\dagger\)}
  \]
  \begin{proof}
    Indeed running Euclid's algorihm on \(f_1(X, 1)\) and \(f_2(X, 1)\) gives \(r, s \in \Q[X]\) such that
    \[
      r(X) f_1(X, 1) + s(X) f_2(X, 1) = 1.
    \]
    Homgogenising and clearing denominators gives (\(\dagger\)) for \(i = 2\) Likewise for \(i = 1\).
  \end{proof}

  Write \(P = (a_1: a_2)\) where \(a_1, a_2 \in \Z\) coprime. Then (\(\dagger\)) gives
  \[
    \sum_{j = 1}^w g_{ij}(a_i, a_2) f_j(a_1, a_2) = \kappa a_i^{2d - 1}.
  \]
  Thus \(\gcd(f_1(a_1, a_2), f_2(a_1, a_2))\) divides \(\gcd(\kappa a_1^{2d - 1}, \kappa a_2^{2d - 1}) = \kappa\). But also
  \[
    |\kappa a_i^{2d - 1}| \leq \underbrace{\max_{j = 1, 2} |f_j (a_i, a_2)|}_{\leq \kappa H(F(P))} \underbrace{\sum_{j = 1}^2 |g_{ij}(a_1, a_2)|}_{\leq \gamma_i H(P)^{d - 1}}.
  \]
  where \(\gamma_i\) is the sum over \(j\) of absolute values of coefficients of \(g_{ij}\). Thus
  \[
    |a_i|^{2d - 1} \leq \gamma_i H(F(P)) H(P)^{d - 1}
  \]
  for \(i = 1, 2\). Thus
  \[
    H(P)^{2d - 1} \leq \max(\gamma_1, \gamma_2) H(F(P)) H(P)^{d - 1}.
  \]
  Take \(c_1 = \max(\gamma_1, \gamma_2)^{-1}\).
\end{proof}

\begin{notation}
  For \(x \in \Q\) we define \(H(x) = H((x: 1)) = \max(|u|, |v|)\) where \(x = \frac{u}{v}\) for \(u, v \in \Z\) coprime.
\end{notation}

Let \(E/\Q\) be an elliptic curve of the form \(y^2 = x^3 + ax + b\).

\begin{definition}[height]\index{height}
  The \emph{height} is defined as the map
  \begin{align*}
    H: E(\Q) &\to \R_{\geq 1} \\
    P &\mapsto
        \begin{cases}
          H(x) & P = (x, y) \\
          1 & P = 0_E
        \end{cases}
  \end{align*}

  We define the \emph{logarithmic height} to be \(h = \log H\).
\end{definition}

\begin{lemma}
  Let \(E, E'\) be elliptic curves over \(\Q\), \(\phi: E \to E'\) an isogeny defined over \(\Q\). Then exists \(c > 0\) such that
  \[
    |h(\phi(P)) - \deg(\phi) h(P)| \leq c
  \]
  for all \(P \in E(\Q)\). Note that \(c\) depends on \(E, E'\) and \(\phi\).
\end{lemma}

\begin{proof}
  Recall (Lemma 5.4) we have commutative diagram
  \[
    \begin{tikzcd}
      E \ar[r, "\phi"] \ar[d, "x"] & E' \ar[d, "x"] \\
      \P^1 \ar[r, "\xi"] & \P^1
    \end{tikzcd}
  \]
  and \(\deg \phi = \deg \xi = d\), say. Lemma 13.1 says that there exist \(c_1, c_2 > 0\) such that
  \[
    c_1 H(P)^d \leq H(\phi(P)) \leq c_2 H(P)^d
  \]
  for all \(P \in E(\Q)\). Taking logs gives
  \[
    |h(\phi(P)) - d h(P)| \leq \max(\log c_2, -\log c_1).
  \]
\end{proof}

\begin{eg}
  Let \(\phi = [2]: E \to E\). Then exists \(c > 0\) such that
  \[
    |h(2P) - 4h(P)| < c
  \]
  for all \(P \in E(\Q)\).
\end{eg}

\begin{definition}[canonical height]\index{canonical height}
  The \emph{canonical height} is
  \[
    \hat h(P) = \lim_{n \to \infty} \frac{1}{4^n} h(2^nP).
  \]
\end{definition}

Check convergence: for \(m \geq n\),
\begin{align*}
  |\frac{1}{4^m} h(2^m P) - \frac{1}{4^n} h(2^n P)|
  &\leq \sum_{r = n}^{m - 1} |\frac{1}{4^{r + 1}} h(2^{r + 1}P) - \frac{1}{4^r} h(2^r P)| \\
  &\leq \sum_{r = n}^{m - 1} \frac{1}{4^{r + 1}} |h(2^{r + 1}P) - 4h(2^r P)| \\
  &\leq c \sum_{r = n}^\infty \frac{1}{4^{r + 1}} \\
  &\to 0
\end{align*}
as \(n \to \infty\) so the sequence is Cauchy so \(\hat h(P)\) exists.

\begin{lemma}
  \(|h(P) - \hat h(P)|\) is bounded for \(P \in E(\Q)\).
\end{lemma}

\begin{proof}
  Put \(n = 0\) in the above calcultion to give
  \[
    |\frac{1}{4^m} h(2^m P) - h(P)| \leq \frac{c}{3}.
  \]
  Take limit as \(m \to \infty\).
\end{proof}

\begin{corollary}
  For any \(B > 0\), \(\# \{P \in E(\Q): \hat h(P) < B\} < \infty\).
\end{corollary}

\begin{proof}
  By the lemma \(\hat h(P)\) is bounded implies \(h(P)\) is bounded, so only finitely many possibilities for \(x\). Each \(x\) leaves at most 2 choices for \(y\).
\end{proof}

\begin{lemma}
  Suppose \(\phi: E \to E'\) is an isogeny defined over \(\Q\). Then
  \[
    \hat h(\phi P) = (\deg \phi) \hat h(P)
  \]
  for all \(P \in E(\Q)\).
\end{lemma}

\begin{proof}
  By lemma 13.2 exists \(c > 0\) such that
  \[
    |h(\phi P) - (\deg \phi) h(P)| < c
  \]
  for all \(P \in E(\Q)\). Replace \(P\) by \(2^nP\), divide by \(4^n\) and take limit as \(n \to \infty\).
\end{proof}

\begin{remark}\leavevmode
  \begin{enumerate}
  \item The case \(\deg \phi = 1\) shows that \(\hat h\), unlike \(h\), is independent of the choice of Weierstrass equation.
  \item Taking \(\phi = [n]: E \to E\) gives \(\hat h(nP) = n^2 \hat h(P)\) for all \(P \in E(\Q)\).
  \end{enumerate}
\end{remark}

(Going to prove \(\hat h\) is a quadratic form by showing that it satisfies the parallelogram law).

\begin{lemma}
  Let \(E/\Q\) be an ellitpic curve. There exists \(c > 0\) such that
  \[
    H(P + Q) H(P - Q) \leq c H(P)^2 H(Q)^2
  \]
  for all \(P, Q, P + Q, P - Q \ne 0_E\).
\end{lemma}

\begin{proof}
  Let \(E\) have Weierstrass equation \(y^2 = x^3 + ax + b\), \(a, b \in \Z\). Let \(P, Q, P + Q, P - Q\) has \(x\) coordinates \(x_1, \dots, x_4\). By lemma 5.8 there exist \(W_0, W_1, W_2 \in \Z[x_1, x_2]\) of degree \(\leq 2\) in \(x_1\) and degree \(\leq 2\) in \(x_2\) such that
  \[
    (1: x_3 + x_4: x_3x_4) = (W_0: W_1: W_2)
  \]
  and
  \(W_0 = (x_1 - x_2)^2\). Write \(x_i = \frac{r_i}{s_i}\) where \(r_i, s_i \in \Z\) coprime. Then we get
  \[
    (s_3s_4: r_3s_4 + r_4s_3: r_3r_4) = ((r_1s_2 - r_2s_1)^2: \cdots ).
  \]
  So
  \begin{align*}
    H(P + Q) H(P - Q)
    &= \max(|r_3|, |s_3|) \max(|r_4|, |s_4|) \\
    &\leq 2 \max(|s_3s_4|, |r_3s_4 + r_4s_3|, |r_3r_4|) \\
    &\leq 2 \max(|r_1s_2 - r_2s_1|, \cdots) \\
    &\leq c H(P)^2 H(Q)^2
  \end{align*}
  where \(c\) depends on \(E\) but not on \(P\) and \(Q\).
\end{proof}

\begin{theorem}
  \(\hat h: E(\Q) \to \R_{\geq 0}\) is a quadratic form.
\end{theorem}

\begin{proof}
  Lemma 13.6 and \(|h(2P) - 4h(P)|\) bounded implies that
  \[
    h(P + Q) + h(P - Q) \leq 2 h(P) + 2h(Q) + c
  \]
  for \(P, Q \in E(\Q)\) (there are several special cases to check). Replacing \(P, Q\) by \(2^n P, 2^n Q\), dividing by \(4^n\) and taking limit \(n \to \infty\) gives
  \[
    \hat h(P + Q) + \hat h(P - Q) \leq 2 \hat h(P) + 2 \hat h(Q).
  \]
  Replacing \(P, Q\) by \(P + Q, P - Q\) and writing \(\hat h(2P) = 4 \hat h(P)\) gives the reverse inequality. Thus \(\hat h\) satisfies the parallelogram law and \(\hat h\) is a quadratic form.
\end{proof}

\begin{remark}
  For \(K\) a number field, \(P = (a_0: \cdots :a_n) \in \P^n(K)\), define
  \[
    H(P) = \prod_v \max_{0 \leq i \leq n} |a_i|_v
  \]
  where the product is over all places \(v\) and the absolute values \(|\cdot|_v\) are normalised such that \(\prod_v |\lambda|_v = 1\) for all \(\lambda \in K^*\). Then all results in this section generalises to \(K\).
\end{remark}

\section{Dual isogenies \& Weil pairing}

Let \(K\) be a perfect field and \(E/K\) an elliptic field.

\begin{proposition}
  Let \(\Phi \subseteq E(\overline K)\) be a finite \(\gal(\overline K/K)\)-stable subgroup. Then exists an elliptic curve \(E'/K\) and a separable isogeny \(\phi: E \to E'\) defined over \(K\) with kernel \(\Phi\) such that for every \(\psi: E \to E''\) with \(\psi \subseteq \ker \psi\) factors uniquely via \(\phi\).
  \[
    \begin{tikzcd}
      E \ar[d, "\phi"] \ar[r, "\psi"] & E'' \\
      E' \ar[ur, dotted, "\exists !"']
    \end{tikzcd}
  \]
\end{proposition}

\begin{proof}
  Omitted. See Silverman Chapter 3.
\end{proof}

\begin{proposition}
  Let \(\phi: E \to E'\) be an isogeny of degree \(n\). Then exists a unique isogeny \(\hat \phi: E' \to E\) such that \(\hat \phi \phi = [n]\). \(\hat \phi\) is called the \emph{dual isogeny}\index{dual isogeny}.
\end{proposition}

\begin{proof}
  Case \(\phi\) separable: \(|\ker \phi| = n\) so \(\ker \phi \subseteq \E[n]\). Apply proposition 14.1 with \(\psi = [n]\). The \(\phi\) inseparble case is omitted (see Silverman. Suffice to check for Frobenius map). For uniqueness if \(\psi_1 \phi = \psi_2 \phi = [n]\) then \((\psi_1 - \psi_2) \phi = 0\) so \(\psi_1 = \psi_2\) since \(\phi\) nonconstant is surjective.
\end{proof}

\begin{remark}\leavevmode
  \begin{enumerate}
  \item The relation of elliptic curves being isogenous is an equivalence relation.
  \item If \(\deg \phi = n\) then \(\deg [n] = n^2\) implies that \(\deg \hat \phi = \deg \phi\) and \(\widehat{[n]} = [n]\).
  \item \(\phi \hat \phi \phi = \phi [n]_E = [n]_{E'} \phi\) implies that \(\phi \hat \phi = [n]_{E'}\). In particular \(\hat{\hat \phi} = \phi\).
  \item If \(E \xrightarrow{\psi} E' \xrightarrow{\phi} E''\) then \(\widehat{\phi\psi} = \hat \psi \hat \phi\).
  \item If \(\phi \in \End(E)\) then by example sheet 2
    \[
      \phi^2 - (\tr \phi) \phi + \deg \phi = 0
    \]
    so
    \[
      \underbrace{([\tr \phi] - \phi)}_{\hat \phi} \phi = [\deg \phi]
    \]
    and hence \(\tr \phi = \phi + \hat \phi\).
  \end{enumerate}
\end{remark}

\begin{lemma}
  If \(\phi, \psi \in \Hom(E, E')\) then \(\widehat{\phi + \psi} = \hat \phi + \hat \psi\).
\end{lemma}

\begin{proof}
  If \(E = E'\) then this follows from \(\tr(\phi + \psi) = \tr \phi + \tr \psi\). In general let \(\alpha: E' \to E\) be any isogeny (e.g.\ \(\hat \phi\)). Thus
  \[
    \widehat{(\alpha \phi + \alpha \psi)} = \widehat{\alpha \phi} + \hat{\alpha \psi}
  \]
  so
  \[
    \widehat{\phi + \psi} \hat \alpha = (\hat \phi + \hat \psi) \hat \alpha.
  \]
\end{proof}

\begin{remark}
  In Silverman's book, he proves Lemma 14.3 first and uses this to show \(\deg: \Hom(E, E') \to \Z\) is a quadratic form.
\end{remark}

\begin{definition}[sum]\index{sum}
  The \emph{sum map} is defined as
  \begin{align*}
    \Sum: \Div(E) &\to E \\
    \sum n_P(P) &\mapsto \sum n_P P
  \end{align*}
  where LHS is a formal sum and RHS is sum using group law.
\end{definition}

Recall that we have a group isomorphism \(E \to \Pic^0(E), P \mapsto [P - 0]\). Thus \(\Sum D \mapsto [D]\) for all \(D \in \Div^0(E)\).

\begin{lemma}
  Let \(D \in \Div(E)\). Then \(D \sim 0\) if and only if \(\deg D = 0\) and \(\Sum D = 0\).
\end{lemma}

Let \(\phi: E \to E'\) be an isogeny of degree \(n\) with dual isogeny \(\hat \phi: E' \to E\). Assume \(\ch K \ndivides n\). We define the \emph{Weil pairing}\index{Weil pairing} \(e_\phi: E[\phi] \times E'[\hat \phi] \to \mu_n\). Let \(T \in E'[\hat \phi]\). Then \(nT = 0\) so exists \(f \in \overline K(E')\) such that \(\div(f) = n(T) - n(0)\). Pick \(T_0 \in E(\overline K)\) with \(\phi(T_0) = T\). Then
\[
  \phi^*(T) - \phi^*(0) = \sum_{P \in E[\phi]} (P + T_0) - \sum_{P \in E[\phi]}(P)
\]
has sum \(nT_0 = \hat \phi \phi T_0 = \hat \phi T = 0\) so exists \(g \in \overline K(E)\) such that \(\div(g) = \phi^*(T) - \phi^*(0)\). Now \(\div(\phi^*f) = \phi^*(\div f) = n(\phi^*(T) - \phi^*(0)) = \div (g^n)\) so \(\phi^* f = c g^n\) for some \(c \in \overline K^*\). Recaling \(f\), wlog \(c = 1\), i.e.\ \(\phi^*f = g^n\).

If \(S \in E[\phi]\) then \(\tau_S^*(\div g) = \div g\) so \(\div(\tau_S^* g) = \div g\) so \(\tau_S^* g = \zeta g\) for some \(\zeta \in \overline K^*\), i.e.\ \(\zeta = \frac{g(X + S)}{g(X)}\) independent of choice of \(X \in E(\overline K)\). Now
\[
  \zeta^n = \frac{g(X + S)^n}{g(X)^n} = \frac{f(\phi(X + S))}{f(\phi(X))} = 1
\]
since \(S \in E[\phi]\). Thus \(\zeta \in \mu_n\). Finally we define
\[
  e_\phi(S, T) = \frac{g(X + S)}{g(X)}
\]
for any \(X \in E\).

\begin{proposition}
  \(e_\phi\) is bliniear and nondegenerate.
\end{proposition}

\begin{proof}
  Linearity in first argument:
  \[
    e_\phi(S_1 + S_2, T) = \frac{g(X + S_1 + S_2)}{g(X + S_2)} \frac{g(X + S_2)}{g(X)} = e_\phi(S_1, T) e_\phi(S_2, T).
  \]

  Linearity in second argument: let \(T_1, T_2 \in E'[\hat \phi]\). We can find \(f_i, g_i\) such that \(\div (f_i) = n(T_i) - n(0), \phi^* f_i = g_n^n\). There exists \(h \in \overline K(E')\) such that
  \[
    \div (h) = (T_1) + (T_2) - (T_1 + T_2) - (0).
  \]
  Then put \(f = \frac{f_1f_2}{h^n}, g = \frac{g_1g_2}{\phi^*(h)}\). Check
  \begin{align*}
    \div (f) &= n(T_1 + T_2) - n(0) \\
    \phi^*f &= \frac{\phi^*f_1 \phi^* f_2}{(\phi^* h)^n} = \left(\frac{g_1g_2}{\phi^*(h)}\right)^n = g^n
  \end{align*}
  so
  \begin{align*}
    e_\phi(S, T_1 + T_2) &= \frac{g(X + S)}{g(X)} \\
                         &= \frac{g_1(X + S)}{g_1(X)} \frac{g_2(X + S)}{g_2(X)} \underbrace{\frac{h(\phi(X))}{h(\phi(X + S))}}_{= 1} \\
                         &= e_\phi(S, T_1) e_\phi(S, T_2)
  \end{align*}

  \(e_\phi\) is nondegenerate: fix \(T \in E'[\hat \phi]\). Suppose \(e_\phi(S, T) = 1\) for all \(S \in E[\phi]\), so \(\tau_S^*g = g\) for all \(S \in E[\phi]\). Thus
  \[
    \begin{tikzcd}
      \overline K(E) \ar[d, dash] \\
      \phi^*\overline K(E')
    \end{tikzcd}
  \]
  is a Galois extension with group \(E[\phi]\), with \(S \in E[\phi]\) acting as \(\tau_S^*\). Thus \(g = \phi^*h\) for some \(h \in \overline K(E')^*\). Thus \(\phi^*f = g^n = \phi^* h^n\) so \(f = h^n\). Thus \(\div h = (T) - (0)\) so \(T = 0_E\).

  For the other direction, we've show \(E'[\hat \phi] \embed \Hom(E[\phi], \mu_n)\). It is an isomorphism by counting.
\end{proof}

\begin{remark}\leavevmode
  \begin{enumerate}
  \item If \(E, E'\) and \(\phi\) are defined over \(K\) then \(e_\phi\) is Galois equivariant, i.e.\ \(e_\phi(\sigma S, \sigma T) = \sigma(e_\phi(S, T))\).
  \item Taking \(\phi = [n]: E \to E\) (so \(\hat \phi = [n]\)) gives \(e_n: E[n] \times E[n] \to \mu_{n^2} = \mu_n\) since \(e_n\) is bilinear.
  \end{enumerate}
\end{remark}

\begin{corollary}
  If \(E[n] \subseteq E(K)\) then \(\mu_n \subseteq K\).
\end{corollary}

\begin{proof}
  We claim exists \(S, T \in E[n]\) such that \(e_n(S, T)\) is a primitive \(n\)th root of unit, say \(\zeta_n\). We pick \(T \in E[n]\) of order \(n\). The group homomorphism \(E[n] \to \mu_n, S \mapsto e_n(S, T)\) has image \(\mu_d\) for some \(d \divides n\). Then \(e_n(S, dT) = 1\) for all \(S \in E[n]\). By nondegeneracy \(dT = 0\) so \(d = n\), proving the claim. To show \(\zeta_n \in K\) we use Galois equivariance: for all \(\sigma \in \gal(\overline K/K)\),
  \[
    \sigma(\zeta_n) \sigma(e_n(S, T)) = e_n(\sigma S, \sigma T) = e_n(S, T) = \zeta_n
  \]
  so \(\zeta_n \in K\).
\end{proof}

\begin{eg}
  There does not exist \(E/\Q\) with \(E(\Q)_{\mathrm{tor}} \cong (\Z/3\Z)^2\).
\end{eg}

\begin{remark}
  In fact \(e_n\) is alternating, i.e.\ \(e_n(T, T) = 1\) for all \(T \in E[n]\). By expanding \(e_n(S + T, S + T)\), we have \(e_n\) alternating: \(e_n(S, T) = e_n(T, S)^{-1}\).
\end{remark}

\section{Galois cohomology}

Let \(G\) be a group and \(A\) a \(G\)-module, i.e.\ an abelian group with an action of \(G\) via group homomorphism (in other words a \(\Z[G]\)-module). We begin with a very practical definition of group cohomology (or more precisely, \(H^0\) and \(H^1\)).

\begin{definition}[group cohomology]\index{group cohomology}
  We define
  \[
    H^0(G, A) = A^G = \{a \in A: \sigma(a) = a \text{ for all } \sigma \in G\}.
  \]
  We define the first cochains, cocyles and coboundaries 
  \begin{align*}
    C^1(G, A) &= \{G \to A\} \\
    Z^1(G, A) &= \{(a_\sigma)_{\sigma \in G}: a_{\sigma\tau} = \sigma(a_\tau) + a_\sigma\} \\
    B^1(G, A) &= \{(\sigma b - b)_{\sigma \in G}: b \in A\}
  \end{align*}
  Then we define
  \[
    H^1(G, A) = \frac{Z^1(G, A)}{B^1(G, A)}.
  \]
\end{definition}

\begin{remark}
  If \(G\) acts trivially on \(A\) then \(H^1(G, A) = \Hom(G, A)\).
\end{remark}

We quote some elementary results from homological algebra:

\begin{theorem}
  A short exact sequence of \(G\)-modules
  \[
    \begin{tikzcd}
      0 \ar[r] & A \ar[r, "\phi"] & B \ar[r, "\psi"] & C \ar[r] & 0
    \end{tikzcd}
  \]
  gives rise to a long exact sequence of abelian groups
  \[
    \begin{tikzcd}[column sep=small]
      0 \ar[r] & A^G \ar[r] & B^G \ar[r] & C^G \ar[r] & H^1(G, A) \ar[r] & H^1(G, B) \ar[r] & H^1(G, C)
    \end{tikzcd}
  \]
\end{theorem}

\begin{proof}
  Omitted. We note the definition of \(\delta: C^G \to H^1(G, A)\): given \(c \in C^G\), exists \(b \in B\) such that \(\psi(b) = c\). Then
  \[
    \tau(\sigma b - b) = \sigma c - c = 0
  \]
  for all \(\sigma \in G\) so \(\sigma b - b = \phi(a_\sigma)\) for some \(a_\sigma \in A\). Can show \((a_\sigma)_{\sigma \in G} \in Z^1(G, A)\). We define \(\delta(c)\) to be the class of \((a_\sigma)_{\sigma \in G}\) in \(H^1(G, A)\).
\end{proof}

\begin{theorem}
  Let \(A\) be a \(G\)-module and \(H \normal G\) be a normal subgroup. Then there is an \emph{inflation-restriction exact sequence}\index{inflation-restriciton exact sequence}
  \[
    \begin{tikzcd}
      0 \ar[r] & H^1(G/H, A^H) \ar[r, "\mathrm{inf}"] & H^1(G, A) \ar[r, "\mathrm{res}"] & H^1(H, A)
    \end{tikzcd}
  \]
\end{theorem}

\begin{proof}
  Omitted.
\end{proof}

Let \(K\) be a perfect field. Then \(\gal(\overline K/K)\) is a topological group with basis of open subgroups \(\gal(\overline K/L)\) for \([L: K] < \infty\). If \(G = \gal(\overline K/K)\) we modify the definition of \(H^1(G, A)\) by insisting
\begin{enumerate}
\item the stabiliser of each \(a \in A\) is an open subgroup of \(G\),
\item all cochains \(G \to A\) are continuous, where \(A\) is given the discrete topology.
\end{enumerate}
Then
\[
  H^1(\gal(\overline K/K), A) = \varinjlim_{L/K \text{ finite Galois}} H^1(\gal(L/K), A^{\gal(\overline K/L)}).
\]
Here the direct limit is with respect to inflation maps.

\begin{theorem}[Hilbert theorem 90]\index{Hilbert theorem 90}
  Suppose \(L/K\) is a finite Galois extension. Then
  \[
    H^1(\gal(L/K), L^*) = 0.
  \]
\end{theorem}

\begin{proof}
  Let \(G = \gal(L/K)\) and \((a_\sigma)_{\sigma \in G} \in Z^1(G, L^*)\). Distinct automorphisms are linearly independent so exists \(y\) such that
  \[
    x = \sum_{\tau \in G} a_\tau^{-1} \tau(y) \neq 0.
  \]
  For \(\sigma \in G\),
  \[
    \sigma(x) = \sum_{\tau \in G} \sigma(a_\tau)^{-1} \sigma\tau(y)
    = a_\sigma \sum_{\tau \in G} a_{\sigma\tau}^{-1} \sigma\tau(y)
    = a_\sigma x.
  \]
  Thus \(a_\sigma = \frac{\sigma(x)}{x}\) so \((a_\sigma)_{\sigma \in G} \in B^1(G, L^*)\). Thus \(H^1(G, L^*) = 0\).
\end{proof}

\begin{corollary}
  \(H^1(\gal(\overline K/K), \overline K^*) = 0\).
\end{corollary}

As an application, assume \(\ch K \ndivides n\). There is a short exact sequence of \(\gal(\overline K/K)\)-modules
\[
  \begin{tikzcd}
    0 \ar[r] & \mu_n \ar[r] & \overline K^* \ar[r, "x \mapsto x^n"] & \overline K^* \ar[r] & 0
  \end{tikzcd}
\]
so we have a long exact sequence
\[
  \begin{tikzcd}
    K^* \ar[r, "x \mapsto x^n"] & K^* \ar[r] & H^1(\gal(\overline K/K), \mu_n) \ar[r] & H^1(\gal(\overline K/K), \overline K^*) = 0
  \end{tikzcd}
\]
so
\[
  H^1(\gal(\overline K/K), \mu_n) \cong K^*/(K^*)^n.
\]
Now let's revisit Kummer theory. If \(\mu_n \subseteq K\) then
\[
  \Hom(\gal(\overline K/K), \mu_n) \cong K^*/(K^*)^n.
\]
Finite subgroups of LHS are of the form \(\Hom(\gal(L/K), \mu_n)\) for \(L/K\) a finite abelian extension of exponent dividing \(n\). Thus we get another proof of Theorem 11.2.

\begin{remark}
  Every continuous group homomorphism \(\chi: \gal(\overline K/K) \to \mu_n\) factorises uniquely as
  \[
    \gal(\overline K/K) \surj \gal(L/K) \embed \mu_n
  \]
  for \(L\) the fixed field of \(\ker \chi\).
\end{remark}

\begin{notation}
  Since we are dealing with Galois cohomology, write \(H^1(K, -)\) for \(H^1(\gal(\overline K/K), -)\).
\end{notation}

Let \(\phi: E \to E'\) be an isogeny of elliptic curves over \(K\). There is a short exact sequence of \(\gal(\overline K/K)\)-modules
\[
  \begin{tikzcd}
    0 \ar[r] & E[\phi] \ar[r] & E \ar[r, "\phi"] & E' \ar[r] & 0
  \end{tikzcd}
\]
which induces a long exact seqeucne
\[
  \begin{tikzcd}
    E(K) \ar[r, "\phi"] & E'(K) \ar[r, "\delta"] & H^1(K, E[\phi]) \ar[r] & H^1(K, E) \ar[r, "\phi_*"] & H^1(K, E')
  \end{tikzcd}
\]
from which we get a short exact sequence
\[
  \begin{tikzcd}
    0 \ar[r] & \frac{E'(K)}{\phi E(K)} \ar[r] & H^1(K, E[\phi]) \ar[r] & H^1(K, E)[\phi_*] \ar[r] & 0
  \end{tikzcd}
\]
Now take \(K\) a number field. For each place \(v\) of \(K\) we fix an embedding \(\overline K \subseteq \overline K_v\). Then \(\gal(\overline K_V/K_V) \subseteq \gal(\overline K/K)\). We get a commutative diagram
\[
  \begin{tikzcd}
    0 \ar[r] & \frac{E'(K)}{\phi E(K)} \ar[r] \ar[d] & H^1(K, E[\phi]) \ar[r] \ar[d, "\mathrm{res}_V"] & H^1(K, E)[\phi_*] \ar[r] \ar[d, "\mathrm{res}_V"] & 0 \\
    0 \ar[r] & \frac{E'(K_v)}{\phi E(K_v)} \ar[r] & H^1(K_v, E[\phi]) \ar[r] & H^1(K_v, E)[\phi_*] \ar[r] & 0
  \end{tikzcd}
\]

\begin{definition}[Selmer group]\index{Selmer group}
  The \emph{\(\phi\)-Selmer group} \(S^{(\phi)}(E/K)\) is the kernel of the dotted arrow in
  \[
    \begin{tikzcd}[column sep=scriptsize]
      0 \ar[r] & \frac{E'(K)}{\phi E(K)} \ar[r] \ar[d] & H^1(K, E[\phi]) \ar[r] \ar[d, "\mathrm{res}_V"] \ar[dr, dotted] & H^1(K, E)[\phi_*] \ar[r] \ar[d, "\mathrm{res}_V"] & 0 \\
      0 \ar[r] & \prod_v \frac{E'(K_v)}{\phi E(K_v)} \ar[r, "\delta_v"] & \prod_v H^1(K_v, E[\phi]) \ar[r] & \prod_v H^1(K_v, E)[\phi_*] \ar[r] & 0
    \end{tikzcd}
  \]
  so
  \begin{align*}
    S^{(\phi)}(E/K)
    &= \ker(H^1(K, E[\phi]) \to \prod_v H^1(K_v, E)) \\
    &= \{\alpha\in H^1(K, E[\phi]): \mathrm{res}_V(\alpha) \in \im (\delta_v) \text{ for all } v\}
  \end{align*}
\end{definition}

\begin{definition}[Tate-Shafarevich group]\index{Tate-Shafarevich group}
  The \emph{Tate-Shafarevich group} is
  \[
    \Sh (E/K) = \ker(H^1(K, E) \to \prod_v H^1(K_v, E)).
  \]
\end{definition}

We get a short exact sequence
\[
  \begin{tikzcd}
    0 \ar[r] & \frac{E'(K)}{\phi E(K)} \ar[r] & S^{(\theta)}(E/K) \ar[r] & Ш(E/K)[\phi_*] \ar[r] & 0
  \end{tikzcd}
\]
In particular we can specialise to \(\phi = [n]\). Rearranging our proof of weak Mordell-Weil gives

\begin{theorem}
  \(S^{(n)}(E/K)\) is finite.
\end{theorem}

\begin{proof}
  For \(L/K\) a finite Galois extension there is an exact sequence
  \[
    \begin{tikzcd}
      0 \ar[r] & H^1(\gal(L/K), E(L)[n]) \ar[r, "\mathrm{inf}"] & H^1(K, E[n]) \ar[d, "\supseteq"] \ar[r, "\mathrm{res}"] & H^1(L, E[n]) \ar[d, "\supseteq"] \\
      & & S^{(n)}(E/K) \ar[r] & S^{(n)}(E/K)
    \end{tikzcd}
  \]
  As \(H^1(\gal(L/K), E(L)[n])\) is finite, we we extend our field \(K\) and assume \(E[n] \subseteq E(K)\) and hence \(\mu_n \subseteq K\). Thus \(E[n] \cong \mu_n \times \mu_n\) as Galois modules. Thus
  \[
    H^1(K, E[n]) \cong H^1(K, \mu_n) \times H^1(K, \mu_n) \cong K^*/(K^*)^n \times K^*/(K^*)^n.
  \]

  Let \(S\) be the union of primes of bad reduction for \(E\), \(v\) such that \(v \divides n\) and the infinite places. Note \(S\) is a finite set of places.

  \begin{definition}
    The subgroup of \(H^1(K, A)\) unramified outside \(S\) is
    \[
      H^1(K, A; S) = \ker(H^1(K, A) \to \prod_{v \notin S} H^1(K_v^{\mathrm{nr}}, A)).
    \]
  \end{definition}
  There is a commutative diagram with exact rows
  \[
    \begin{tikzcd}
      E(K_v) \ar[r, "\times n"] \ar[d, hook] & E(K_v) \ar[r, "\delta_v"] \ar[d, hook] & H^1(K_v, E[n]) \ar[d, "\mathrm{res}"] \\
      E(K_v^{\mathrm{nr}}) \ar[r, "\times n"] & E(K_v^{\mathrm{nr}}) \ar[r, "0"] & H^1(K_v^{\mathrm{nr}}, E[n])
    \end{tikzcd}
  \]
  Multiplication by \(n\) on the second row is surjective for all \(v \notin S\) (Thm 9.9). Thus
  \begin{align*}
    S^{(n)}(E/K) &= \{\alpha \in H^1(K, E[n]): \mathrm{res}_v(\alpha) \in \im(\delta_v) \text{ for all } v\} \\
                 &\subseteq H^1(K, E[n]; S) \\
                 &\cong H^1(K, \mu_n; S) \times H^1(K, \mu_n; S)
  \end{align*}
  (?using the fact that \(\mathrm{res} \compose \delta_v = 0\)) But
  \[
    H^1(K, \mu_n; S)
    = \ker(K^*/(K^*)^n \to \prod_{v \notin S} (K_v^{\mathrm{nr}})^*/(K_v^{\mathrm{nr}})^{*n})
    = K(S, n)
  \]
  which is finite.
\end{proof}

\begin{remark}
  \(S^{(n)}(E/K)\) is finite and effectively computable. It is conjectured that \(|Ш(E/K)| < \infty\). This would imply that \(\mathrm{rank} E(K)\) is effctively computable.
\end{remark}

\section{Descent by cyclic isogeny}

Let \(E, E'\) be elliptic curves over a number field \(K\). Let \(\phi: E \to E'\) be an isogeny of degree \(n\). Suppose \(E'[\hat \phi] \cong \Z/n\Z\) is generated by \(T \in E'(K)\). Then \(E[\phi] \cong \mu_n, S \mapsto e_\phi(S, T)\) as a \(\gal(\overline K/K)\)-module. We have a short exact sequence of \(\gal(\overline K/K)\)-modules
\[
  \begin{tikzcd}
    0 \ar[r] & \mu_n \ar[r] & E \ar[r, "\phi"] & E' \ar[r] & 0
  \end{tikzcd}
\]
giving rise to long exact sequence
\[
  \begin{tikzcd}
    E(K) \ar[r] & E'(K) \ar[r, "\delta"] \ar[dr, "\alpha"'] & H^1(K, \mu_n) \ar[r] \ar[d, "\cong"] & H^1(K, E) \\
    & & K^*/(K^*)^n
  \end{tikzcd}
\]

\begin{theorem}
  Let \(f \in K(E')\) and \(g \in K(E)\) with \(\div (f) = n(T) - n(0)\) and \(\phi^* f = g^n\). Then \(\alpha(P) = f(P) \pmod{(K^*)^n}\) for all \(P \in E'(K) \setminus \{0, T\}\).
\end{theorem}

\begin{proof}
  Let \(Q \in \phi^{-1}P\). Then \(\delta(P) \in H^1(K, \mu_n)\) is represented by the cocyle \(\sigma \mapsto \sigma Q - Q \in E[\phi] \cong \mu_n\). For any \(X \in E\) not a zero or pole of \(g\),
  \[
    e_\phi(\sigma Q - Q, T)
    = \frac{g(\sigma Q - Q + X)}{g(X)}
    = \frac{g(\sigma Q)}{g(Q)}
    = \frac{\sigma(g(Q))}{g(Q)}
    = \frac{\sigma(\sqrt[n]{f(P)})}{\sqrt[n]{f(P)}}
  \]
  But
  \begin{align*}
    H^1(K, \mu_n) &\cong K^*/(K^*)^n \\
    \sigma \mapsto \frac{\sigma \sqrt[n]{x}}{\sqrt[n]{x}} &\mapsfrom x
  \end{align*}
  so \(\alpha(P) = f(P) \pmod{(K^*)^n}\).
\end{proof}

\paragraph{Descent by \(2\)-isogeny}

Let \(E: y^2 = x(x^2 + ax + b), E': y^2 = x(x^2 + a'x + b')\) where \(b(a^2 - 4b) \ne 0, a' = -2a, b' = a^2 - 4b\). Define
\begin{align*}
  \phi: E &\to E' \\
  (x, y) &\mapsto ((\frac{y}{x})^2, \frac{y(x^2 - b)}{x^2}) \\
  \hat \phi: E' &\to E \\
  (x, y) &\mapsto (\frac{1}{4} (\frac{y}{x})^2, \frac{y(x^2 - b')}{8x^2})
\end{align*}
Check they are dual to each other. Have \(E[\phi] = \{0, T\}, E'[\hat \phi] = \{0, T'\}\) where \(T = (0, 0) \in E(K), E' = (0, 0) \in E'(K)\).

\begin{proposition}
  There is a group homomorphism
  \begin{align*}
    E'(K) &\to K^*/(K^*)^2 \\
    (x, y) &\mapsto
             \begin{cases}
               x \pmod{(K^*)^2} & x \ne 0 \\
               b' \pmod{(K^*)^2} & x = 0
             \end{cases}
  \end{align*}
  with kernel \(\phi(E(K))\).
\end{proposition}

\begin{proof}
  Either apply theorem 16.1 with \(f = x \in K(E'), g = \frac{y}{x} \in K(E)\), or direct calculation, see example sheet 4.
\end{proof}

Let
\[
  \alpha_E: \frac{E(K)}{\hat \phi(E'(K))} \embed K^*/(K^*)^2, \alpha_{E'}: \frac{E'(K)}{\phi(E(K))} \embed K^*/(K^*)^2.
\]

\begin{lemma}
  \(2^{\mathrm{rank} E(K)} = \frac{1}{4} |\im \alpha_E| \cdot |\im \alpha_{E'}|\).
\end{lemma}

\begin{proof}
  Since \(\hat \phi \phi = [2]_E\) there is an exact sequence
  \[
    \begin{tikzcd}
      0 \ar[r] & E(K)[\phi] \ar[r] & E(K)[2] \ar[r, "\phi"] & E'(K)[\hat \phi] \ar[dll, overlay, out=0, in=180] \\
      & \frac{E'(K)}{\phi E(K)} \ar[r, "\hat \phi"] & \frac{E(K)}{2E(K)} \ar[r] & \frac{E(K)}{\hat E'(K)} \ar[r] & 0
    \end{tikzcd}
  \]
  so the alternative product of group orders is \(1\). Thus
  \[
    \frac{|E(K)/2E(K)|}{E(K)[2]} = \frac{|\im \alpha_E| \cdot |\im \alpha_{E'}|}{4}.
  \]

  By Mordell-Weil \(E(K) \cong \Delta \times \Z^r\) where \(\Delta\) is finite and \(r\) is the rank of \(E(K)\). Thus
  \[
    \frac{E(K)}{2E(K)} \cong \frac{\Delta}{2\Delta} \times (\Z/2\Z)^r, E(K)[2] \cong \Delta[2].
  \]
  Since \(\Delta\) is finite, \(\frac{\Delta}{2\Delta}\) and \(\Delta[2]\) have the same order. The result thus follows.
\end{proof}

\begin{lemma}
  If \(K\) is a number field and \(a, b \in \O_K\) then \(\im \alpha_E \subseteq K(S, 2)\) where \(S = \{\text{primes dividing } b\}\).
\end{lemma}

\begin{proof}
  Must show if \(x, y \in K\), \(y^2 = x(x^2 + ax + b)\) and \(v_{\mathfrak p}(b) = 0\) then \(v_{\mathfrak p}(x)\) is even. If \(v_{\mathfrak p}(x) < 0\) then by lemma 9.1 \(v_{\mathfrak p}(x) = -2r, v_{\mathfrak p}(y) = -3r\) for some \(r \geq 1\). If \(v_{\mathfrak p}(x) > 0\) then \(v_{\mathfrak p}(x^2 + ax + b) = 0\) so \(v_{\mathfrak p}(x) = v_{\mathfrak p}(y^2) = 2 v_{\mathfrak p}(y)\).
\end{proof}

\begin{lemma}
  If \(b_1b_2 = b\) then \(b_1(K^*)^2 \in \im \alpha_E\) if and only if
  \[
    w^2 = b_1 u^4 + au^2v^2 + b_2v^4
  \]
  is soluble for \(u, v, w \in K\) not all zero.
\end{lemma}

\begin{proof}
  If \(b_1 \in (K^*)^2\) or \(b_2 \in (K^*)^2\) then both conditions are satisfied so may assume \(b_1, b_2 \notin (K^*)^2\). \(b_1(K^*)^2 \in \im \alpha_E\) if and only if exists \((x, y) \in E(K)\) such that \(x = b_1t^2\) for some \(t \in K^*\), so
  \[
    y^2 = b_1t^2 ((b_1t^2)^2 + ab_1t^2 + b)
  \]
  so
  \[
    (\frac{y}{b_1t})^2 = b_1t^4 + at^2 + b_2
  \]
  so have solution \((u, v, w) = (t, 1, \frac{w}{b_1t})\).

  Conversely if \((u, v, w)\) is a solution then \(uv \ne 0\). Check \((b_1 (\frac{u}{v})^2, b_1 \frac{uw}{v^3}) \in E(K)\).
\end{proof}

Now take \(K = \Q\).

\begin{eg}
  \(E: y^2 = x^3 - x\). By lemma 16.4, \(\im \alpha_E \subseteq \langle -1 \rangle \subseteq \Q^*/(\Q^*)^2\). But we know \((0, 0) \in \im \alpha_E\), equality. \(E': y^2 = x^3 + 4x\), \(\im \alpha_{E'} \subseteq \langle -1, 2 \rangle \subseteq \Q^*/(\Q^*)^2\). Need to check
  \begin{align*}
    b_1 = 1, & w^2 = - u^4 - 4u^4 \\
    b_1 = 2, & w^2 = 2u^4 + 2v^4 \\
    b_1 = -2, & w^2 = -2u^4 - 2v^4
  \end{align*}
  The first and third are not soluble over \(\R\). The second has solution \((u, v, w) = (1, 1, 2)\) so \(\im \alpha_{E'} = \langle 2\rangle \subseteq \Q^*/(\Q^*)^2\). Thus \(\mathrm{rank} E(\Q) = 0\) so \(1\) is not a congurent number.
\end{eg}

\begin{eg}
  \(E: y^2 = x^3 + px\) where \(p\) is a prime, \(p = 5 \pmod 8\). \(b_1 = -1, w^2 = -u^4 - pv^4\) is insoluble over \(\R\) so \(\im \alpha_E = \langle p\rangle \subseteq \Q^*/(\Q^*)^2\). \(E': y^2 = x^3 - 4px\) so \(\im \alpha_{E'} \subseteq \langle -1, 2, p \rangle \subseteq \Q^*/(\Q^*)^2\). Note \(\alpha_{E'}(T') = (-4p) (\Q^*)^2 = (-p) (\Q^*)^2\) so only need to consider
  \begin{align*}
    b_1 = 2, & w^2 = 2u^4 - 2pv^4 \\
    b_1 = -2, & w^2 = -2u^4 + 2pv^4 \\
    b_1 = p, & w^2 = pu^4 - 4v^4
  \end{align*}
  Suppose equation 1 is soluble. wlog \(u, v, w \in \Z, \gcd(u, v) = 1\). If \(p \divides u\) then \(p \divides w\) and then \(p \divides v\), absurd. Thus \(w^2 = 2u^4 \ne 0 \pmod p\) so \(\legendre{2}{p} = 1\), contradicting \(p = 5 \pmod 8\).

  Likewise 2 has no solution since \(\legendre{-2}{p} = -1\).
\end{eg}

To recall, for \(E: y^2 = x(x^2 + ax + b)\), \(\phi: E \to E'\) a \(2\)-isogeny. \(w^2 = b_1 u^4 + au^2v^2 + b_2v^4 (*)\). Have a short exact sequence
\[
  \begin{tikzcd}
    0 \ar[r] & \frac{E'(\Q)}{\phi E(\Q)} \ar[r] \ar[dr, "\alpha_{E'}"] & S^{(\phi)} (E/\Q) \ar[r] & \Sh (E/\Q)[\phi_*] \ar[r] & 0 \\
    & & \Q^*/(\Q^*)^2
  \end{tikzcd}
\]
\begin{align*}
  \im \alpha_{E'} &= \{b_1 (\Q^*)^2: \text{\(\ast\) is soluble over \(\Q\)}\} \\
  \subseteq S^{(\phi)}(E/\Q) &= \{b_1 (\Q^*)^2: \text{\(\ast\) is soluble over \(\R\) and over \(\Q_p\) for all \(p\)}\}
\end{align*}

\begin{fact}
  (Uses example sheet 3 question 9 and Hensel's lemma) If \(a, b_1, b_2 \in \Z\) and \(p \ndivides 2b(a^2 - 4b)\) then \(\ast\) is solubleover \(\Q_p\).
\end{fact}

\begin{eg}[example 2 continued]
  \(E: y^2 = x^3 + px\), \(p = 5 \pmod 8\), \(w^2 = pu^4 - 4v^4 \dagger\). \(E(\Q)\) has rank \(0\) if (\(\dagger\)) is insoluble over \(\Q\) and rank \(1\) if soluble. By the fact we only have to look at \(p\)- and \(2\)-adics.
  \begin{itemize}
  \item \(\dagger\) is soluble over \(\Q_p\) since \(\legendre{-1}{p} = 1\) so \(-1 \in (\Z_p^*)^2\) (by Hensel's lemma).
  \item soluble over \(\Q_2\) since \(p - 4 = 1 \pmod 8\) so \(p - 4 \in (\Z_2^*)^2\).
  \item soluble over \(\R\) since \(\sqrt p \in \R\).
  \end{itemize}
  We can try to spot solutions:
  \[
    \begin{array}{c|ccc}
      p & u & v & w \\ \hline
      5 & 1 & 1 & 1 \\
      13 & 1 & 1 & 3 \\
      29 & 1 & 1 & 5 \\
      37 & 5 & 3 & 151 \\
      53 & 1 & 1 & 7
    \end{array}
  \]
  Conjecture: \(\mathrm{rank}(E(\Q)) = 1\) for all primes \(p = 5 \pmod 8\).
\end{eg}

\begin{eg}[Lind]
  \(E: y^2 = x^3 + 17x\). \(\im \alpha_E = \langle 17 \rangle \subseteq \Q^*/(\Q^*)^2\). \(E': y^2 = x^3 - 68x\). \(\im \alpha_{E'} \subseteq \langle -1, 2, 17 \rangle \subseteq \Q^*/(\Q^*)^2\). Consider \(b_1 = 2\). \(w^2 = 2u^4 - 34v^4\). Replace \(w\) by \(2w\) and divide through by \(2\) to get \(C: 2w^2 = u^4 - 17v^4\). Denote by
  \[
    C(K) = \{(u, v, w) \in K^3 \setminus \{0\} \text{ satisfying } C\}/\sim
  \]
  where \((u, v, w) \sim (\lambda u, \lambda v, \lambda^2 w)\) for all \(\lambda \in K^*\).

  \(C(\Q_2) \ne \emptyset\) as \(17 \in (\Z_2^*)^4\). \(C(\Q_{17}) \neq \emptyset\) since \(2 \in (\Z_{17}^*)^2\). \(C(\R) \ne \emptyset\) since \(\sqrt 2 \in \R\). Thus \(C(\Q_v) \ne \emptyset\) for all places of \(\Q\). However it has no solution over \(\Q\): suppose \((u, v, w) \in C(\Q)\). wlog \(u, v\in \Z, \gcd(u, v) = 1\), then \(w \in \Z\) and can assume \(w > 0\). If \(17 \divides w\) then \(17 \divides u\) and then \(17 \divides v\), absurd. So if \(p \divides w\) then \(p \ne 17\) and \(\legendre{17}{p} = 1\) so by quadratic reciprocity \(\legendre{p}{17} = \legendre{17}{2} = 1\) (for \(p\) odd. For \(p = 2\) have \(\legendre{2}{17} = 1\). Thus \(\legendre{w}{17} = 1\). But \(2w^2 = u^4 \pmod{17}\) so \(2 \in (\F_{17}^*)^4 = \{\pm 1, \pm 4\}\), absurd. Thus \(C(\Q) = \emptyset\). \(C\) is a counterexample to the Hasse principle. It representes a non-trivial element in \(\Sh(E/\Q)\).
\end{eg}

\paragraph{Birch Swinnerton-Dyer conjecture}
\index{Birch Sinnerton-Dyer conjecture}

Let \(E/\Q\) be an elliptic curve.

\begin{definition}[\(l\)-function]\index{\(L\)-function}
  The \emph{\(L\)-function} of \(E\) is \(L(E, s) = \prod_p L_p(E, s)\) where
  \[
    L_p(E, s) =
    \begin{cases}
      (1 - a_p p^{-s} + p^{1 - 2s})^{-1} & \text{good reduction} \\
      (1 - p^{-s})^{-1} & \text{split multiplicative reduction} \\
      (1 + p^{-s})^{-1} & \text{nonsplit multiplicative reduction} \\
      1 & \text{additive reduction}
    \end{cases}
  \]
  where \(\#(\F_p) = p + 1 - a_p\).
\end{definition}

Hasse's theorem says that \(|a_p| < s \sqrt p\) so \(L(E, s)\) converges for \(\Re s > \frac{3}{2}\).

\begin{theorem}[Wiles, Breuil, Conrad, Diamond, Taylor]
  \(L(E, s)\) is the \(L\)-function of a weight \(2\) modular form and hence has an analytic continuation to all of \(\C\) (and a functional equation relating \(L(E, s)\) and \(L(E, 2 - s)\)).
\end{theorem}

\begin{conjecture}[weak Birch Swinnerton-Dyer conjecutre]
  \(\ord_{s = 1} L(E, s) = \mathrm{rank} E(\Q)\).
\end{conjecture}

Assuming weak BSD and let \(r = \ord_{s = 1} L(E, s)\) be the analytic rank, we have

\begin{conjecture}[strong Birch Swinnerton-Dyer conjecutre]
  \[
    \lim_{s \to 1} \frac{1}{(s - 1)^r} L(E, s) = \frac{\Omega_E |\Sh(E/\Q)| \mathrm{Reg} E(\Q) \prod_P c_p}{|E(\Q)_{\mathrm{tors}}|^2}
  \]
  where
  \begin{itemize}
  \item \(c_p = [E(\Q_p): E_0(\Q_p)] = \text{ tamagawa number of } E/\Q_p\), if \(\frac{E(\Q)}{E(\Q)_{\mathrm{tors}}} = \langle P_1, \dots, P_r \rangle\) then
    \[
      \mathrm{Reg} E(\Q) = \det ([P_i, P_j])_{ij}
    \]
    where \([P, Q] = \hat h (P + Q) - \hat h(P) - \hat h(Q)\).
  \item \(\Omega_E = \int_{E(\R)} \frac{dx}{|2y + a_1x + a_3|}\) where \(a_i\) is the coefficient of a globally minimal Weierstrass equation for \(E\).
  \end{itemize}
\end{conjecture}

Best result so far:

\begin{theorem}[Kolvragin]
  If \(\ord_{s = 1} L(E, s) = 0\) or \(1\) then weak BSD is trus and \(|\Sh (E/\Q)| < \infty\).
\end{theorem}

\printindex
\end{document}

% Silverman, The arithmetic of elliptic curves, Springer 1986
% Cassels, Lectures on ellptic curves, CUP 1991

% Introductory reading
% Silverman & Tate, Rational points on elliptic curves, Springer 1992
% Milne, Elliptic curves, Booksurge 2006