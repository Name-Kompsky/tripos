\documentclass[a4paper]{article}

\def\npart{III}

\def\ntitle{Elliptic Curves}
\def\nlecturer{T.\ A.\ Fisher}

\def\nterm{Michaelmas}
\def\nyear{2019}

\ifx \nauthor\undefined
  \def\nauthor{Qiangru Kuang}
\else
\fi

\ifx \ntitle\undefined
  \def\ntitle{Template}
\else
\fi

\ifx \nauthoremail\undefined
  \def\nauthoremail{qk206@cam.ac.uk}
\else
\fi

\ifx \ndate\undefined
  \def\ndate{\today}
\else
\fi

\title{\ntitle}
\author{\nauthor}
\date{\ndate}

%\usepackage{microtype}
\usepackage{mathtools}
\usepackage{amsthm}
\usepackage{stmaryrd}%symbols used so far: \mapsfrom
\usepackage{empheq}
\usepackage{amssymb}
\let\mathbbalt\mathbb
\let\pitchforkold\pitchfork
\usepackage{unicode-math}
\let\mathbb\mathbbalt%reset to original \mathbb
\let\pitchfork\pitchforkold

\usepackage{imakeidx}
\makeindex[intoc]

%to address the problem that Latin modern doesn't have unicode support for setminus
%https://tex.stackexchange.com/a/55205/26707
\AtBeginDocument{\renewcommand*{\setminus}{\mathbin{\backslash}}}
\AtBeginDocument{\renewcommand*{\models}{\vDash}}%for \vDash is same size as \vdash but orginal \models is larger
\AtBeginDocument{\let\Re\relax}
\AtBeginDocument{\let\Im\relax}
\AtBeginDocument{\DeclareMathOperator{\Re}{Re}}
\AtBeginDocument{\DeclareMathOperator{\Im}{Im}}
\AtBeginDocument{\let\div\relax}
\AtBeginDocument{\DeclareMathOperator{\div}{div}}

\usepackage{tikz}
\usetikzlibrary{automata,positioning}
\usepackage{pgfplots}
%some preset styles
\pgfplotsset{compat=1.15}
\pgfplotsset{centre/.append style={axis x line=middle, axis y line=middle, xlabel={$x$}, ylabel={$y$}, axis equal}}
\usepackage{tikz-cd}
\usepackage{graphicx}
\usepackage{newunicodechar}

\usepackage{fancyhdr}

\fancypagestyle{mypagestyle}{
    \fancyhf{}
    \lhead{\emph{\nouppercase{\leftmark}}}
    \rhead{}
    \cfoot{\thepage}
}
\pagestyle{mypagestyle}

\usepackage{titlesec}
\newcommand{\sectionbreak}{\clearpage} % clear page after each section
\usepackage[perpage]{footmisc}
\usepackage{blindtext}

%\reallywidehat
%https://tex.stackexchange.com/a/101136/26707
\usepackage{scalerel,stackengine}
\stackMath
\newcommand\reallywidehat[1]{%
\savestack{\tmpbox}{\stretchto{%
  \scaleto{%
    \scalerel*[\widthof{\ensuremath{#1}}]{\kern-.6pt\bigwedge\kern-.6pt}%
    {\rule[-\textheight/2]{1ex}{\textheight}}%WIDTH-LIMITED BIG WEDGE
  }{\textheight}% 
}{0.5ex}}%
\stackon[1pt]{#1}{\tmpbox}%
}

%\usepackage{braket}
\usepackage{thmtools}%restate theorem
\usepackage{hyperref}

% https://en.wikibooks.org/wiki/LaTeX/Hyperlinks
\hypersetup{
    %bookmarks=true,
    unicode=true,
    pdftitle={\ntitle},
    pdfauthor={\nauthor},
    pdfsubject={Mathematics},
    pdfcreator={\nauthor},
    pdfproducer={\nauthor},
    pdfkeywords={math maths \ntitle},
    colorlinks=true,
    linkcolor={red!50!black},
    citecolor={blue!50!black},
    urlcolor={blue!80!black}
}

\usepackage{cleveref}



% TODO: mdframed often gives bad breaks that cause empty lines. Would like to switch to tcolorbox.
% The current workaround is to set innerbottommargin=0pt.

%\usepackage[theorems]{tcolorbox}





\usepackage[framemethod=tikz]{mdframed}
\mdfdefinestyle{leftbar}{
  %nobreak=true, %dirty hack
  linewidth=1.5pt,
  linecolor=gray,
  hidealllines=true,
  leftline=true,
  leftmargin=0pt,
  innerleftmargin=5pt,
  innerrightmargin=10pt,
  innertopmargin=-5pt,
  % innerbottommargin=5pt, % original
  innerbottommargin=0pt, % temporary hack 
}
%\newmdtheoremenv[style=leftbar]{theorem}{Theorem}[section]
%\newmdtheoremenv[style=leftbar]{proposition}[theorem]{proposition}
%\newmdtheoremenv[style=leftbar]{lemma}[theorem]{Lemma}
%\newmdtheoremenv[style=leftbar]{corollary}[theorem]{corollary}

\newtheorem{theorem}{Theorem}[section]
\newtheorem{proposition}[theorem]{Proposition}
\newtheorem{lemma}[theorem]{Lemma}
\newtheorem{corollary}[theorem]{Corollary}
\newtheorem{axiom}[theorem]{Axiom}
\newtheorem*{axiom*}{Axiom}

\surroundwithmdframed[style=leftbar]{theorem}
\surroundwithmdframed[style=leftbar]{proposition}
\surroundwithmdframed[style=leftbar]{lemma}
\surroundwithmdframed[style=leftbar]{corollary}
\surroundwithmdframed[style=leftbar]{axiom}
\surroundwithmdframed[style=leftbar]{axiom*}

\theoremstyle{definition}

\newtheorem*{definition}{Definition}
\surroundwithmdframed[style=leftbar]{definition}

\newtheorem*{slogan}{Slogan}
\newtheorem*{eg}{Example}
\newtheorem*{ex}{Exercise}
\newtheorem*{remark}{Remark}
\newtheorem*{notation}{Notation}
\newtheorem*{convention}{Convention}
\newtheorem*{assumption}{Assumption}
\newtheorem*{question}{Question}
\newtheorem*{answer}{Answer}
\newtheorem*{note}{Note}
\newtheorem*{application}{Application}

%operator macros

%basic
\DeclareMathOperator{\lcm}{lcm}

%matrix
\DeclareMathOperator{\tr}{tr}
\DeclareMathOperator{\Tr}{Tr}
\DeclareMathOperator{\adj}{adj}

%algebra
\DeclareMathOperator{\Hom}{Hom}
\DeclareMathOperator{\End}{End}
\DeclareMathOperator{\id}{id}
\DeclareMathOperator{\im}{im}
\DeclareMathOperator{\coker}{coker}
\DeclarePairedDelimiter{\generation}{\langle}{\rangle}

%groups
\DeclareMathOperator{\sym}{Sym}
\DeclareMathOperator{\sgn}{sgn}
\DeclareMathOperator{\inn}{Inn}
\DeclareMathOperator{\aut}{Aut}
\DeclareMathOperator{\GL}{GL}
\DeclareMathOperator{\SL}{SL}
\DeclareMathOperator{\PGL}{PGL}
\DeclareMathOperator{\PSL}{PSL}
\DeclareMathOperator{\SU}{SU}
\DeclareMathOperator{\UU}{U}
\DeclareMathOperator{\SO}{SO}
\DeclareMathOperator{\OO}{O}
\DeclareMathOperator{\PSU}{PSU}
\DeclareMathOperator{\Sp}{Sp}


%hyperbolic
\DeclareMathOperator{\sech}{sech}

%field, galois heory
\DeclareMathOperator{\ch}{ch}
\DeclareMathOperator{\gal}{Gal}
\DeclareMathOperator{\emb}{Emb}



%ceiling and floor
%https://tex.stackexchange.com/a/118217/26707
\DeclarePairedDelimiter\ceil{\lceil}{\rceil}
\DeclarePairedDelimiter\floor{\lfloor}{\rfloor}


\DeclarePairedDelimiter{\innerproduct}{\langle}{\rangle}

%\DeclarePairedDelimiterX{\norm}[1]{\lVert}{\rVert}{#1}
\DeclarePairedDelimiter{\norm}{\lVert}{\rVert}



%Dirac notation
%TODO: rewrite for variable number of arguments
\DeclarePairedDelimiterX{\braket}[2]{\langle}{\rangle}{#1 \delimsize\vert #2}
\DeclarePairedDelimiterX{\braketthree}[3]{\langle}{\rangle}{#1 \delimsize\vert #2 \delimsize\vert #3}

\DeclarePairedDelimiter{\bra}{\langle}{\rvert}
\DeclarePairedDelimiter{\ket}{\lvert}{\rangle}




%macros

%general

%divide, not divide
\newcommand*{\divides}{\mid}
\newcommand*{\ndivides}{\nmid}
%vector, i.e. mathbf
%https://tex.stackexchange.com/a/45746/26707
\newcommand*{\V}[1]{{\ensuremath{\symbf{#1}}}}
%closure
\newcommand*{\cl}[1]{\overline{#1}}
%conjugate
\newcommand*{\conj}[1]{\overline{#1}}
%set complement
\newcommand*{\stcomp}[1]{\overline{#1}}
\newcommand*{\compose}{\circ}
\newcommand*{\nto}{\nrightarrow}
\newcommand*{\p}{\partial}
%embed
\newcommand*{\embed}{\hookrightarrow}
%surjection
\newcommand*{\surj}{\twoheadrightarrow}
%power set
\newcommand*{\powerset}{\mathcal{P}}

%matrix
\newcommand*{\matrixring}{\mathcal{M}}

%groups
\newcommand*{\normal}{\trianglelefteq}
%rings
\newcommand*{\ideal}{\trianglelefteq}

%fields
\renewcommand*{\C}{{\mathbb{C}}}
\newcommand*{\R}{{\mathbb{R}}}
\newcommand*{\Q}{{\mathbb{Q}}}
\newcommand*{\Z}{{\mathbb{Z}}}
\newcommand*{\N}{{\mathbb{N}}}
\newcommand*{\F}{{\mathbb{F}}}
%not really but I think this belongs here
\newcommand*{\A}{{\mathbb{A}}}

%asymptotic
\newcommand*{\bigO}{O}
\newcommand*{\smallo}{o}

%probability
\newcommand*{\prob}{\mathbb{P}}
\newcommand*{\E}{\mathbb{E}}

%vector calculus
\newcommand*{\gradient}{\V \nabla}
\newcommand*{\divergence}{\gradient \cdot}
\newcommand*{\curl}{\gradient \cdot}

%logic
\newcommand*{\yields}{\vdash}
\newcommand*{\nyields}{\nvdash}

%differential geometry
\renewcommand*{\H}{\mathbb{H}}
\newcommand*{\transversal}{\pitchfork}
\renewcommand{\d}{\mathrm{d}} % exterior derivative

%number theory
\newcommand*{\legendre}[2]{\genfrac{(}{)}{}{}{#1}{#2}}%Legendre symbol

%algebraic geometry
\DeclareMathOperator{\Spec}{Spec}
\DeclareMathOperator{\Proj}{Proj}

\begin{document}

\renewcommand*{\P}{\mathbb{P}}

\begin{titlepage}
  \begin{center}
    \includegraphics[width=0.6\textwidth]{logo.jpg}\par
    \vspace{1cm}
    {\scshape\huge Mathamatics Tripos \par}
    \vspace{2cm}
    {\huge Part \npart \par}
    \vspace{0.6cm}
    {\Huge \bfseries \ntitle \par}
    \vspace{1.2cm}
    {\Large\nterm, \nyear \par}
    \vspace{2cm}
    
    {\large \emph{Lectures by } \par}
    \vspace{0.2cm}
    {\Large \scshape \nlecturer}
    
    \vspace{0.5cm}
    {\large \emph{Notes by }\par}
    \vspace{0.2cm}
    {\Large \scshape \href{mailto:\nauthoremail}{\nauthor}}
 \end{center}
\end{titlepage}

\tableofcontents

\section{Fermat's method of infinite descent}

Let \(\Delta = (a, b, c)\) be a right angle triangle with sides \(a, b, c\) where \(c\) is the hypotenuse.

\begin{definition}
  \(\Delta\) is rational if \(a, b, c \in \Q\). \(\Delta\) is primitive if \(a, b, c \in \Z\) and coprime.
\end{definition}

\begin{lemma}
  Every primitive triangle is of the form \((u^2 - v^2, 2uv, u^2 + v^2)\) for some \(u, v \in \Z, u > v > 0\).
\end{lemma}

\begin{proof}
  \(a\) and \(b\) cannot be both even. They cannot be both odd as then \(c^2 = 2 \mod 4\). Thus wlog \(a\) is odd and \(b\) is even, so \(c\) odd. Then
  \[
    \left(\frac{b}{2}\right)^2 = \frac{c + a}{2} \cdot \frac{c - a}{2}
  \]
  and the two terms on RHS are coprime positive integers. By unique factorisation in \(\Z\), there exist \(u, v \in \Z\) such that
  \begin{align*}
    \frac{c + a}{2} &= u^2 \\
    \frac{c - a}{2} &= v^2
  \end{align*}
  Rearrange.
\end{proof}

\begin{definition}
  \(D \in \Q_{> 0}\) is a \emph{congruent number} if there exists a right angle triangle whose area is \(D\).
\end{definition}

\begin{note}
  Suffices to consider \(D \in \Z_{> 0}\) square-free.
\end{note}

\begin{eg}
  \(D = 5, 6\) are congruent.
\end{eg}

\begin{lemma}
  \(D \in \Q_{> 0}\) is congruent if and only if \(D y^2 = x^3 - x\) for some \(x, y \in \Q, y \neq 0\).
\end{lemma}

\begin{proof}
  Lemma 1 shows that \(D\) is congruent if and only if \(Dw^2 = uv(u^2 - v^2)\) for some \(u, v, w \in \Q, w \neq 0\). Let \(x = \frac{u}{v}, y = \frac{w}{v^2}\).
\end{proof}

Fermat showed that \(1\) is not a congruent number.

\begin{theorem}
  There are no solutions to
  \begin{equation}
    \label{eqn:fermat}
    w^2 = uv (u - v)(u + v)
    \tag{\ast}
  \end{equation}
  for \(u, v, w \in \Z, w \neq 0\).
\end{theorem}

\begin{proof}
  wlog \(u, v \) coprime, \(u > 0, w > 0\). If \(v < 0\) then replace \((u, v, w)\) by \((-v, u, w)\). If \(u = v \mod 2\) then replace \((u, v, w)\) by \((\frac{u + v}{2}, \frac{u - v}{2}, \frac{w}{2})\). Then \(u, v, u - v, u + v\) are positive coprime integers whose product is a square. By unique prime factorisation, \(u = a^2, v = b^2, u + v = c^2, u - v = d^2\) for some \(a, b, c, d \in \Z_{> 0}\). As \(u \neq v \mod 2\), \(c, d\) are both odd. Consider a new triangle with sides \(\frac{c + d}{2}, \frac{c - d}{2}\). Then
  \[
    \left( \frac{c + d}{2} \right)^2 + \left( \frac{c - d}{2} \right)^2 = \frac{c^2 + d^2}{2} = u = a^2
  \]
  so this is another primitive triangle. Its area is
  \[
    \frac{c^2 - d^2}{8} = \frac{v}{4} = \left( \frac{b}{2} \right)^2.
  \]

  Let \(w_1 = \frac{b}{2}\) so by lemma 1
  \[
    w_1^2 = u_1v_1 (u_1 - v_1)(u_1 + v_1),
  \]
  i.e.\ we have a new solution to \eqref{eqn:fermat}. But \(4 w_1^2 = b^2 = v \divides w^2\) so \(w_1 \leq \frac{1}{2} w\). So by Fermat's method of infinite descend, there is no solution to \eqref{eqn:fermat}.
\end{proof}

\subsection{A variant for polynomials}

Let \(K\) be a field with \(\ch K \neq 2\). Let \(\overline K\) be an algebraic closure of \(k\).

\begin{lemma}
  Let \(u, v \in K[t]\) coprime. If \(\alpha u + \beta v\) is a square for four distinct \((\alpha: \beta) \in \P^1\) then \(u, v \in K\).
\end{lemma}

\begin{proof}
  wlog \(K = \overline K\). Changing coordinates on \(\P^1\), we may assume the ratio \((\alpha: \beta)\) are \((1: 0), (0: 1), (1: -1), (1: -\lambda)\) for some \(\lambda \in K \setminus \{0, 1\}\). Thus we have
  \begin{align*}
    u &= a^2 \\
    v &= b^2 \\
    u - v &= (a - b)(a + b) \\
    u - \lambda v &= (a - \mu b)(a + \mu b)
  \end{align*}
  where \(\mu = \sqrt \lambda\). Use unqiue factorisation in \(K[t]\),  as \(a, b\) are coprime, \(a + b, a - b, a - \mu b, a + \mu b\) are squares. But
  \[
    \max (\deg (a), \deg (b)) \leq \frac{1}{2} \max (\deg (u), \deg (v))
  \]
  so by Fermat's method of infinite descend, \(u, v \in K\).
\end{proof}

\begin{definition}[elliptic curve]\index{ellptic curve}\leavevmode
  \begin{enumerate}
  \item An \emph{elliptic curve} \(E/K\) is the projective closure of a plane affine curve \(y^2 = f(x)\) where \(f \in K[x]\) is a monic cubic polynomial with distinct roots in \(\overline K\). The equation \(y^2 = f(x)\) is called a \emph{Weierstrass function}\index{Weierstrass function}.
  \item For \(L/K\) a field extension,
    \[
      E(L) = \{(x, y) \in L^2: y^2 = f(x)\} \cup \{0\}
    \]
    where \(0\) is the point at infinity in the projective closure.
  \end{enumerate}
\end{definition}

Fact: \(E(L)\) is naturally an abelian group.

In this course we study \(E(L)\) for \(L\) finite field, local field (meaning \(L/\Q_p\) finite in this course) or number field (\(L/\Q\) finite).

\begin{theorem}
  If \(E: y^2 = x^3 - x\) then \(E(\Q) = \{0, (0, 0), (\pm 1, 0)\}\).
\end{theorem}

\printindex
\end{document}

% Silverman, The arithmetic of elliptic curves, Springer 1986
% Cassels, Lectures on ellptic curves, CUP 1991

% Introductory reading
% Silverman & Tate, Rational points on elliptic curves, Springer 1992
% Milne, Elliptic curves, Booksurge 2006